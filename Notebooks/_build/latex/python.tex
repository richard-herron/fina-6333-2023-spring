%% Generated by Sphinx.
\def\sphinxdocclass{jupyterBook}
\documentclass[letterpaper,10pt,english]{jupyterBook}
\ifdefined\pdfpxdimen
   \let\sphinxpxdimen\pdfpxdimen\else\newdimen\sphinxpxdimen
\fi \sphinxpxdimen=.75bp\relax
\ifdefined\pdfimageresolution
    \pdfimageresolution= \numexpr \dimexpr1in\relax/\sphinxpxdimen\relax
\fi
%% let collapsible pdf bookmarks panel have high depth per default
\PassOptionsToPackage{bookmarksdepth=5}{hyperref}
%% turn off hyperref patch of \index as sphinx.xdy xindy module takes care of
%% suitable \hyperpage mark-up, working around hyperref-xindy incompatibility
\PassOptionsToPackage{hyperindex=false}{hyperref}
%% memoir class requires extra handling
\makeatletter\@ifclassloaded{memoir}
{\ifdefined\memhyperindexfalse\memhyperindexfalse\fi}{}\makeatother

\PassOptionsToPackage{warn}{textcomp}

\catcode`^^^^00a0\active\protected\def^^^^00a0{\leavevmode\nobreak\ }
\usepackage{cmap}
\usepackage{fontspec}
\defaultfontfeatures[\rmfamily,\sffamily,\ttfamily]{}
\usepackage{amsmath,amssymb,amstext}
\usepackage{polyglossia}
\setmainlanguage{english}



\setmainfont{FreeSerif}[
  Extension      = .otf,
  UprightFont    = *,
  ItalicFont     = *Italic,
  BoldFont       = *Bold,
  BoldItalicFont = *BoldItalic
]
\setsansfont{FreeSans}[
  Extension      = .otf,
  UprightFont    = *,
  ItalicFont     = *Oblique,
  BoldFont       = *Bold,
  BoldItalicFont = *BoldOblique,
]
\setmonofont{FreeMono}[
  Extension      = .otf,
  UprightFont    = *,
  ItalicFont     = *Oblique,
  BoldFont       = *Bold,
  BoldItalicFont = *BoldOblique,
]



\usepackage[Bjarne]{fncychap}
\usepackage[,numfigreset=1,mathnumfig]{sphinx}

\fvset{fontsize=\small}
\usepackage{geometry}


% Include hyperref last.
\usepackage{hyperref}
% Fix anchor placement for figures with captions.
\usepackage{hypcap}% it must be loaded after hyperref.
% Set up styles of URL: it should be placed after hyperref.
\urlstyle{same}

\addto\captionsenglish{\renewcommand{\contentsname}{Lectures and Practice}}

\usepackage{sphinxmessages}



        % Start of preamble defined in sphinx-jupyterbook-latex %
         \usepackage[Latin,Greek]{ucharclasses}
        \usepackage{unicode-math}
        % fixing title of the toc
        \addto\captionsenglish{\renewcommand{\contentsname}{Contents}}
        \hypersetup{
            pdfencoding=auto,
            psdextra
        }
        % End of preamble defined in sphinx-jupyterbook-latex %
        

\title{FINA 6333}
\date{Sep 08, 2023}
\release{}
\author{Richard Herron}
\newcommand{\sphinxlogo}{\vbox{}}
\renewcommand{\releasename}{}
\makeindex
\begin{document}

\pagestyle{empty}
\sphinxmaketitle
\pagestyle{plain}
\sphinxtableofcontents
\pagestyle{normal}
\phantomsection\label{\detokenize{_introduction::doc}}


\sphinxAtStartPar
Welcome to FINA 6333 for Spring 2023!

\begin{DUlineblock}{0em}
\item[] \sphinxstylestrong{\Large Website}
\end{DUlineblock}

\sphinxAtStartPar
I will maintain our notebooks on this website because Canvas does not render notebooks.
This website renders our notebooks and offers several ways to interact:
\begin{enumerate}
\sphinxsetlistlabels{\arabic}{enumi}{enumii}{}{.}%
\item {} 
\sphinxAtStartPar
Open and run them on an \sphinxstyleemphasis{ad hoc} JupyterLab server by clicking the rocket in the upper\sphinxhyphen{}right corner of each page (this page is not a notebook, so you cannot open and run it)

\item {} 
\sphinxAtStartPar
Download them by clicking the down arrow in the upper\sphinxhyphen{}right corner of each page

\item {} 
\sphinxAtStartPar
Visit their GitHub repository by clicking the cat in the upper\sphinxhyphen{}right corner of each page

\end{enumerate}

\sphinxAtStartPar
This GitHub repository (\sphinxurl{https://github.com/richard-herron/fina-6333-2023-spring}) also generates this website.
If GitHub intimidates you, you can ignore our GitHub repository and download our notebooks one at a time, as I describe above in option 1.
If GitHub does not intimidate you, you can use our GitHub repository in a few ways:
\begin{enumerate}
\sphinxsetlistlabels{\arabic}{enumi}{enumii}{}{.}%
\item {} 
\sphinxAtStartPar
Fork it and maintain a personal repository that you update throughout the semester

\item {} 
\sphinxAtStartPar
Create issues for each notebook when you have questions and comments

\end{enumerate}

\begin{DUlineblock}{0em}
\item[] \sphinxstylestrong{\large Notebooks}
\end{DUlineblock}

\sphinxAtStartPar
Each course topic has one notebook for its lecture and one for its practice exercises.
So we will typically have one notebook per chapter in McKinney {[}\hyperlink{cite._introduction:id4}{McK22}{]} and Welch {[}\hyperlink{cite._introduction:id5}{Wel22}{]}.
I plan to “flip the classroom” this semester, which means:
\begin{enumerate}
\sphinxsetlistlabels{\arabic}{enumi}{enumii}{}{.}%
\item {} 
\sphinxAtStartPar
Each week, I will record a short lecture based on the lecture notebook

\item {} 
\sphinxAtStartPar
Before class, you will watch this short lecture, review the lecture notebook, and read the textbook

\item {} 
\sphinxAtStartPar
During class, we will work through the practice exercise notebook together

\item {} 
\sphinxAtStartPar
After class, I will update and post these notebooks, and you will complete DataCamp courses, quizzes, and projects to reinforce what we learn

\end{enumerate}

\sphinxAtStartPar
I will maintain these notebooks on this website but everything else on Canvas (and Gradescope).

\begin{DUlineblock}{0em}
\item[] \sphinxstylestrong{\large JupyterLab}
\end{DUlineblock}

\sphinxAtStartPar
I suggest you run JupyterLab on Research Computing’s Open OnDemand (OOD) service: \sphinxurl{https://ood.discovery.neu.edu}.
This service is fast, reliable, available everywhere, and can access a shared folder of course notebooks.
You can log into OOD with your Northeastern username and password and start a JupyterLab session under “My Interactive Sessions”.

\sphinxAtStartPar
You can also install Anaconda on your laptop to run JupyterLab or use DataCamp Workspaces.
There are more ways to run JupyterLab and even more ways to interact with Python, and I am happy to help you during office hours.
However, we will only use JupyterLab in class, and OOD is the easiest way to use JupyterLab.

\sphinxAtStartPar
\sphinxstyleemphasis{\sphinxstylestrong{Update:}} Halfway through the semester, OOD has proven to be slow and unreliable at the scale of this course.
Instead, I suggest you install Anaconda on your laptop or use DataCamp Workspaces.

\begin{DUlineblock}{0em}
\item[] \sphinxstylestrong{\large References}
\end{DUlineblock}

\sphinxstepscope


\part{Lectures and Practice}

\sphinxstepscope


\chapter{McKinney Chapter 2 \sphinxhyphen{} Python Language Basics, IPython, and Jupyter Notebooks}
\label{\detokenize{mckinney_02_lecture:mckinney-chapter-2-python-language-basics-ipython-and-jupyter-notebooks}}\label{\detokenize{mckinney_02_lecture::doc}}

\section{Introduction}
\label{\detokenize{mckinney_02_lecture:introduction}}
\sphinxAtStartPar
We must understand the basics of Python before we can use it to analyze financial data.
Chapter 2 of Wes McKinney’s \sphinxhref{https://wesmckinney.com/book/}{\sphinxstyleemphasis{Python for Data Analysis}} provides a crash course in Python’s syntax, and chapter 3 provides a crash course in Python’s built\sphinxhyphen{}in data structures.
This notebook focuses on the “Python Language Basics” in section 2.3, which covers language semantics, scalar types, and control flow.

\sphinxAtStartPar
\sphinxstyleemphasis{\sphinxstylestrong{Note:}}
Indented block quotes are from McKinney unless otherwise indicated.
The section numbers here differ from McKinney because we will only discuss some topics.


\section{Language Semantics}
\label{\detokenize{mckinney_02_lecture:language-semantics}}

\subsection{Indentation, not braces}
\label{\detokenize{mckinney_02_lecture:indentation-not-braces}}\begin{quote}

\sphinxAtStartPar
Python uses whitespace (tabs or spaces) to structure code instead of using braces as in many other languages like R, C++, Java, and Perl.
\end{quote}

\sphinxAtStartPar
\sphinxstyleemphasis{\sphinxstylestrong{So, spaces are more than cosmetic in Python.}}
For non\sphinxhyphen{}Python programmers, white space is often Python’s defining feature.
Here is a for loop with an if block that shows how Python uses white space.

\begin{sphinxuseclass}{cell}\begin{sphinxVerbatimInput}

\begin{sphinxuseclass}{cell_input}
\begin{sphinxVerbatim}[commandchars=\\\{\}]
\PYG{n}{array} \PYG{o}{=} \PYG{p}{[}\PYG{l+m+mi}{1}\PYG{p}{,} \PYG{l+m+mi}{2}\PYG{p}{,} \PYG{l+m+mi}{3}\PYG{p}{]}
\PYG{n}{pivot} \PYG{o}{=} \PYG{l+m+mi}{2}
\PYG{n}{less} \PYG{o}{=} \PYG{p}{[}\PYG{p}{]}
\PYG{n}{greater} \PYG{o}{=} \PYG{p}{[}\PYG{p}{]}

\PYG{k}{for} \PYG{n}{x} \PYG{o+ow}{in} \PYG{n}{array}\PYG{p}{:}
    \PYG{k}{if} \PYG{n}{x} \PYG{o}{\PYGZlt{}} \PYG{n}{pivot}\PYG{p}{:}
        \PYG{n+nb}{print}\PYG{p}{(}\PYG{l+s+sa}{f}\PYG{l+s+s1}{\PYGZsq{}}\PYG{l+s+si}{\PYGZob{}}\PYG{n}{x}\PYG{l+s+si}{\PYGZcb{}}\PYG{l+s+s1}{ is less than }\PYG{l+s+si}{\PYGZob{}}\PYG{n}{pivot}\PYG{l+s+si}{\PYGZcb{}}\PYG{l+s+s1}{\PYGZsq{}}\PYG{p}{)}
        \PYG{n}{less}\PYG{o}{.}\PYG{n}{append}\PYG{p}{(}\PYG{n}{x}\PYG{p}{)}        
    \PYG{k}{else}\PYG{p}{:}
        \PYG{n+nb}{print}\PYG{p}{(}\PYG{l+s+sa}{f}\PYG{l+s+s1}{\PYGZsq{}}\PYG{l+s+si}{\PYGZob{}}\PYG{n}{x}\PYG{l+s+si}{\PYGZcb{}}\PYG{l+s+s1}{ is NOT less than }\PYG{l+s+si}{\PYGZob{}}\PYG{n}{pivot}\PYG{l+s+si}{\PYGZcb{}}\PYG{l+s+s1}{\PYGZsq{}}\PYG{p}{)}
        \PYG{n}{greater}\PYG{o}{.}\PYG{n}{append}\PYG{p}{(}\PYG{n}{x}\PYG{p}{)}
\end{sphinxVerbatim}

\end{sphinxuseclass}\end{sphinxVerbatimInput}
\begin{sphinxVerbatimOutput}

\begin{sphinxuseclass}{cell_output}
\begin{sphinxVerbatim}[commandchars=\\\{\}]
1 is less than 2
2 is NOT less than 2
3 is NOT less than 2
\end{sphinxVerbatim}

\end{sphinxuseclass}\end{sphinxVerbatimOutput}

\end{sphinxuseclass}
\begin{sphinxuseclass}{cell}\begin{sphinxVerbatimInput}

\begin{sphinxuseclass}{cell_input}
\begin{sphinxVerbatim}[commandchars=\\\{\}]
\PYG{n}{less}
\end{sphinxVerbatim}

\end{sphinxuseclass}\end{sphinxVerbatimInput}
\begin{sphinxVerbatimOutput}

\begin{sphinxuseclass}{cell_output}
\begin{sphinxVerbatim}[commandchars=\\\{\}]
[1]
\end{sphinxVerbatim}

\end{sphinxuseclass}\end{sphinxVerbatimOutput}

\end{sphinxuseclass}
\begin{sphinxuseclass}{cell}\begin{sphinxVerbatimInput}

\begin{sphinxuseclass}{cell_input}
\begin{sphinxVerbatim}[commandchars=\\\{\}]
\PYG{n}{greater}
\end{sphinxVerbatim}

\end{sphinxuseclass}\end{sphinxVerbatimInput}
\begin{sphinxVerbatimOutput}

\begin{sphinxuseclass}{cell_output}
\begin{sphinxVerbatim}[commandchars=\\\{\}]
[2, 3]
\end{sphinxVerbatim}

\end{sphinxuseclass}\end{sphinxVerbatimOutput}

\end{sphinxuseclass}
\sphinxAtStartPar
\sphinxstyleemphasis{\sphinxstylestrong{Note:}}
We will use f\sphinxhyphen{}string print statements wherever we can.
These f\sphinxhyphen{}string print statements are easy to use, and I do not want to teach old approaches when the new ones are better.


\subsection{Comments}
\label{\detokenize{mckinney_02_lecture:comments}}\begin{quote}

\sphinxAtStartPar
Any text preceded by the hash mark (pound sign) \# is ignored by the Python interpreter. This is often used to add comments to code. At times you may also want to exclude certain blocks of code without deleting them.
\end{quote}

\sphinxAtStartPar
The Python interpreter ignores any code after a hash mark \sphinxcode{\sphinxupquote{\#}} on a given line.
We can quickly comment/un\sphinxhyphen{}comment lines of code with the \sphinxcode{\sphinxupquote{<Ctrl>\sphinxhyphen{}/}} shortcut.

\begin{sphinxuseclass}{cell}\begin{sphinxVerbatimInput}

\begin{sphinxuseclass}{cell_input}
\begin{sphinxVerbatim}[commandchars=\\\{\}]
\PYG{c+c1}{\PYGZsh{} 5 + 5}
\end{sphinxVerbatim}

\end{sphinxuseclass}\end{sphinxVerbatimInput}

\end{sphinxuseclass}

\subsection{Function and object method calls}
\label{\detokenize{mckinney_02_lecture:function-and-object-method-calls}}\begin{quote}

\sphinxAtStartPar
You call functions using parentheses and passing zero or more arguments, optionally assigning the returned value to a variable:

\begin{sphinxVerbatim}[commandchars=\\\{\}]
    \PYG{n}{result} \PYG{o}{=} \PYG{n}{f}\PYG{p}{(}\PYG{n}{x}\PYG{p}{,} \PYG{n}{y}\PYG{p}{,} \PYG{n}{z}\PYG{p}{)}
    \PYG{n}{g}\PYG{p}{(}\PYG{p}{)}
\end{sphinxVerbatim}

\sphinxAtStartPar
Almost every object in Python has attached functions, known as methods, that have access to the object’s internal contents. You can call them using the following > syntax:

\begin{sphinxVerbatim}[commandchars=\\\{\}]
    \PYG{n}{obj}\PYG{o}{.}\PYG{n}{some\PYGZus{}method}\PYG{p}{(}\PYG{n}{x}\PYG{p}{,} \PYG{n}{y}\PYG{p}{,} \PYG{n}{z}\PYG{p}{)}
\end{sphinxVerbatim}

\sphinxAtStartPar
Functions can take both positional and keyword arguments:

\begin{sphinxVerbatim}[commandchars=\\\{\}]
    \PYG{n}{result} \PYG{o}{=} \PYG{n}{f}\PYG{p}{(}\PYG{n}{a}\PYG{p}{,} \PYG{n}{b}\PYG{p}{,} \PYG{n}{c}\PYG{p}{,} \PYG{n}{d}\PYG{o}{=}\PYG{l+m+mi}{5}\PYG{p}{,} \PYG{n}{e}\PYG{o}{=}\PYG{l+s+s1}{\PYGZsq{}}\PYG{l+s+s1}{foo}\PYG{l+s+s1}{\PYGZsq{}}\PYG{p}{)}
\end{sphinxVerbatim}

\sphinxAtStartPar
More on this later.
\end{quote}

\sphinxAtStartPar
We can write a function that adds two numbers.

\begin{sphinxuseclass}{cell}\begin{sphinxVerbatimInput}

\begin{sphinxuseclass}{cell_input}
\begin{sphinxVerbatim}[commandchars=\\\{\}]
\PYG{k}{def} \PYG{n+nf}{add\PYGZus{}numbers}\PYG{p}{(}\PYG{n}{a}\PYG{p}{,} \PYG{n}{b}\PYG{p}{)}\PYG{p}{:}
    \PYG{k}{return} \PYG{n}{a} \PYG{o}{+} \PYG{n}{b}
\end{sphinxVerbatim}

\end{sphinxuseclass}\end{sphinxVerbatimInput}

\end{sphinxuseclass}
\begin{sphinxuseclass}{cell}\begin{sphinxVerbatimInput}

\begin{sphinxuseclass}{cell_input}
\begin{sphinxVerbatim}[commandchars=\\\{\}]
\PYG{n}{add\PYGZus{}numbers}\PYG{p}{(}\PYG{l+m+mi}{5}\PYG{p}{,} \PYG{l+m+mi}{5}\PYG{p}{)}
\end{sphinxVerbatim}

\end{sphinxuseclass}\end{sphinxVerbatimInput}
\begin{sphinxVerbatimOutput}

\begin{sphinxuseclass}{cell_output}
\begin{sphinxVerbatim}[commandchars=\\\{\}]
10
\end{sphinxVerbatim}

\end{sphinxuseclass}\end{sphinxVerbatimOutput}

\end{sphinxuseclass}
\sphinxAtStartPar
We can write a function that adds two strings separated by a space.

\begin{sphinxuseclass}{cell}\begin{sphinxVerbatimInput}

\begin{sphinxuseclass}{cell_input}
\begin{sphinxVerbatim}[commandchars=\\\{\}]
\PYG{k}{def} \PYG{n+nf}{add\PYGZus{}strings}\PYG{p}{(}\PYG{n}{a}\PYG{p}{,} \PYG{n}{b}\PYG{p}{)}\PYG{p}{:}
    \PYG{k}{return} \PYG{n}{a} \PYG{o}{+} \PYG{l+s+s1}{\PYGZsq{}}\PYG{l+s+s1}{ }\PYG{l+s+s1}{\PYGZsq{}} \PYG{o}{+} \PYG{n}{b}
\end{sphinxVerbatim}

\end{sphinxuseclass}\end{sphinxVerbatimInput}

\end{sphinxuseclass}
\begin{sphinxuseclass}{cell}\begin{sphinxVerbatimInput}

\begin{sphinxuseclass}{cell_input}
\begin{sphinxVerbatim}[commandchars=\\\{\}]
\PYG{n}{add\PYGZus{}strings}\PYG{p}{(}\PYG{l+s+s1}{\PYGZsq{}}\PYG{l+s+s1}{5}\PYG{l+s+s1}{\PYGZsq{}}\PYG{p}{,} \PYG{l+s+s1}{\PYGZsq{}}\PYG{l+s+s1}{5}\PYG{l+s+s1}{\PYGZsq{}}\PYG{p}{)}
\end{sphinxVerbatim}

\end{sphinxuseclass}\end{sphinxVerbatimInput}
\begin{sphinxVerbatimOutput}

\begin{sphinxuseclass}{cell_output}
\begin{sphinxVerbatim}[commandchars=\\\{\}]
\PYGZsq{}5 5\PYGZsq{}
\end{sphinxVerbatim}

\end{sphinxuseclass}\end{sphinxVerbatimOutput}

\end{sphinxuseclass}
\sphinxAtStartPar
\sphinxstyleemphasis{\sphinxstylestrong{What is the difference between \sphinxcode{\sphinxupquote{print()}} and \sphinxcode{\sphinxupquote{return}}?}}
\sphinxcode{\sphinxupquote{print()}} returns its arguments to the console or “standard output”, whereas \sphinxcode{\sphinxupquote{return}} returns its argument as an output we can assign to variables.
In the example below, we use the \sphinxcode{\sphinxupquote{return}} line to assign the output of \sphinxcode{\sphinxupquote{add\_string\_2()}} to the variable \sphinxcode{\sphinxupquote{return\_from\_add\_strings\_2}}.
The \sphinxcode{\sphinxupquote{print()}} line prints to the console or “standard output”, but its output is not assigned or captured.

\begin{sphinxuseclass}{cell}\begin{sphinxVerbatimInput}

\begin{sphinxuseclass}{cell_input}
\begin{sphinxVerbatim}[commandchars=\\\{\}]
\PYG{k}{def} \PYG{n+nf}{add\PYGZus{}strings\PYGZus{}2}\PYG{p}{(}\PYG{n}{a}\PYG{p}{,} \PYG{n}{b}\PYG{p}{)}\PYG{p}{:}
    \PYG{n}{string\PYGZus{}to\PYGZus{}print} \PYG{o}{=} \PYG{n}{a} \PYG{o}{+} \PYG{l+s+s1}{\PYGZsq{}}\PYG{l+s+s1}{ }\PYG{l+s+s1}{\PYGZsq{}} \PYG{o}{+} \PYG{n}{b} \PYG{o}{+} \PYG{l+s+s1}{\PYGZsq{}}\PYG{l+s+s1}{ (this is from the print statement)}\PYG{l+s+s1}{\PYGZsq{}}
    \PYG{n}{string\PYGZus{}to\PYGZus{}return} \PYG{o}{=} \PYG{n}{a} \PYG{o}{+} \PYG{l+s+s1}{\PYGZsq{}}\PYG{l+s+s1}{ }\PYG{l+s+s1}{\PYGZsq{}} \PYG{o}{+} \PYG{n}{b} \PYG{o}{+} \PYG{l+s+s1}{\PYGZsq{}}\PYG{l+s+s1}{ (this is from the return statement)}\PYG{l+s+s1}{\PYGZsq{}}
    \PYG{n+nb}{print}\PYG{p}{(}\PYG{n}{string\PYGZus{}to\PYGZus{}print}\PYG{p}{)}
    \PYG{k}{return} \PYG{n}{string\PYGZus{}to\PYGZus{}return}
\end{sphinxVerbatim}

\end{sphinxuseclass}\end{sphinxVerbatimInput}

\end{sphinxuseclass}
\begin{sphinxuseclass}{cell}\begin{sphinxVerbatimInput}

\begin{sphinxuseclass}{cell_input}
\begin{sphinxVerbatim}[commandchars=\\\{\}]
\PYG{n}{returned} \PYG{o}{=} \PYG{n}{add\PYGZus{}strings\PYGZus{}2}\PYG{p}{(}\PYG{l+s+s1}{\PYGZsq{}}\PYG{l+s+s1}{5}\PYG{l+s+s1}{\PYGZsq{}}\PYG{p}{,} \PYG{l+s+s1}{\PYGZsq{}}\PYG{l+s+s1}{5}\PYG{l+s+s1}{\PYGZsq{}}\PYG{p}{)}
\end{sphinxVerbatim}

\end{sphinxuseclass}\end{sphinxVerbatimInput}
\begin{sphinxVerbatimOutput}

\begin{sphinxuseclass}{cell_output}
\begin{sphinxVerbatim}[commandchars=\\\{\}]
5 5 (this is from the print statement)
\end{sphinxVerbatim}

\end{sphinxuseclass}\end{sphinxVerbatimOutput}

\end{sphinxuseclass}
\begin{sphinxuseclass}{cell}\begin{sphinxVerbatimInput}

\begin{sphinxuseclass}{cell_input}
\begin{sphinxVerbatim}[commandchars=\\\{\}]
\PYG{n}{returned}
\end{sphinxVerbatim}

\end{sphinxuseclass}\end{sphinxVerbatimInput}
\begin{sphinxVerbatimOutput}

\begin{sphinxuseclass}{cell_output}
\begin{sphinxVerbatim}[commandchars=\\\{\}]
\PYGZsq{}5 5 (this is from the return statement)\PYGZsq{}
\end{sphinxVerbatim}

\end{sphinxuseclass}\end{sphinxVerbatimOutput}

\end{sphinxuseclass}

\subsection{Variables and argument passing}
\label{\detokenize{mckinney_02_lecture:variables-and-argument-passing}}\begin{quote}

\sphinxAtStartPar
When assigning a variable (or name) in Python, you are creating a reference to the object on the righthand side of the equals sign.
\end{quote}

\begin{sphinxuseclass}{cell}\begin{sphinxVerbatimInput}

\begin{sphinxuseclass}{cell_input}
\begin{sphinxVerbatim}[commandchars=\\\{\}]
\PYG{n}{a} \PYG{o}{=} \PYG{p}{[}\PYG{l+m+mi}{1}\PYG{p}{,} \PYG{l+m+mi}{2}\PYG{p}{,} \PYG{l+m+mi}{3}\PYG{p}{]}
\end{sphinxVerbatim}

\end{sphinxuseclass}\end{sphinxVerbatimInput}

\end{sphinxuseclass}
\begin{sphinxuseclass}{cell}\begin{sphinxVerbatimInput}

\begin{sphinxuseclass}{cell_input}
\begin{sphinxVerbatim}[commandchars=\\\{\}]
\PYG{n}{a}
\end{sphinxVerbatim}

\end{sphinxuseclass}\end{sphinxVerbatimInput}
\begin{sphinxVerbatimOutput}

\begin{sphinxuseclass}{cell_output}
\begin{sphinxVerbatim}[commandchars=\\\{\}]
[1, 2, 3]
\end{sphinxVerbatim}

\end{sphinxuseclass}\end{sphinxVerbatimOutput}

\end{sphinxuseclass}
\sphinxAtStartPar
If we assign \sphinxcode{\sphinxupquote{a}} to a new variable \sphinxcode{\sphinxupquote{b}}, both \sphinxcode{\sphinxupquote{a}} and \sphinxcode{\sphinxupquote{b}} refer to the \sphinxstyleemphasis{same} object.
This same object is the list \sphinxcode{\sphinxupquote{{[}1, 2, 3{]}}}.
If we change \sphinxcode{\sphinxupquote{a}}, we also change \sphinxcode{\sphinxupquote{b}}, because these variables or names refer to the \sphinxstyleemphasis{same} object.

\begin{sphinxuseclass}{cell}\begin{sphinxVerbatimInput}

\begin{sphinxuseclass}{cell_input}
\begin{sphinxVerbatim}[commandchars=\\\{\}]
\PYG{n}{b} \PYG{o}{=} \PYG{n}{a}
\end{sphinxVerbatim}

\end{sphinxuseclass}\end{sphinxVerbatimInput}

\end{sphinxuseclass}
\begin{sphinxuseclass}{cell}\begin{sphinxVerbatimInput}

\begin{sphinxuseclass}{cell_input}
\begin{sphinxVerbatim}[commandchars=\\\{\}]
\PYG{n}{b}
\end{sphinxVerbatim}

\end{sphinxuseclass}\end{sphinxVerbatimInput}
\begin{sphinxVerbatimOutput}

\begin{sphinxuseclass}{cell_output}
\begin{sphinxVerbatim}[commandchars=\\\{\}]
[1, 2, 3]
\end{sphinxVerbatim}

\end{sphinxuseclass}\end{sphinxVerbatimOutput}

\end{sphinxuseclass}
\sphinxAtStartPar
\sphinxstyleemphasis{\sphinxstylestrong{Variables \sphinxcode{\sphinxupquote{a}} and \sphinxcode{\sphinxupquote{b}} refer to the same object, a list \sphinxcode{\sphinxupquote{{[}1, 2, 3{]}}}.}}
We will learn more about lists (and tuples and dictionaries) in chapter 3 of McKinney.

\begin{sphinxuseclass}{cell}\begin{sphinxVerbatimInput}

\begin{sphinxuseclass}{cell_input}
\begin{sphinxVerbatim}[commandchars=\\\{\}]
\PYG{n}{a} \PYG{o+ow}{is} \PYG{n}{b}
\end{sphinxVerbatim}

\end{sphinxuseclass}\end{sphinxVerbatimInput}
\begin{sphinxVerbatimOutput}

\begin{sphinxuseclass}{cell_output}
\begin{sphinxVerbatim}[commandchars=\\\{\}]
True
\end{sphinxVerbatim}

\end{sphinxuseclass}\end{sphinxVerbatimOutput}

\end{sphinxuseclass}
\sphinxAtStartPar
\sphinxstyleemphasis{\sphinxstylestrong{If we modify \sphinxcode{\sphinxupquote{a}} by appending a 4, we change \sphinxcode{\sphinxupquote{b}} because \sphinxcode{\sphinxupquote{a}} and \sphinxcode{\sphinxupquote{b}} refer to the same list.}}

\begin{sphinxuseclass}{cell}\begin{sphinxVerbatimInput}

\begin{sphinxuseclass}{cell_input}
\begin{sphinxVerbatim}[commandchars=\\\{\}]
\PYG{n}{a}\PYG{o}{.}\PYG{n}{append}\PYG{p}{(}\PYG{l+m+mi}{4}\PYG{p}{)}
\end{sphinxVerbatim}

\end{sphinxuseclass}\end{sphinxVerbatimInput}

\end{sphinxuseclass}
\begin{sphinxuseclass}{cell}\begin{sphinxVerbatimInput}

\begin{sphinxuseclass}{cell_input}
\begin{sphinxVerbatim}[commandchars=\\\{\}]
\PYG{n}{a}
\end{sphinxVerbatim}

\end{sphinxuseclass}\end{sphinxVerbatimInput}
\begin{sphinxVerbatimOutput}

\begin{sphinxuseclass}{cell_output}
\begin{sphinxVerbatim}[commandchars=\\\{\}]
[1, 2, 3, 4]
\end{sphinxVerbatim}

\end{sphinxuseclass}\end{sphinxVerbatimOutput}

\end{sphinxuseclass}
\begin{sphinxuseclass}{cell}\begin{sphinxVerbatimInput}

\begin{sphinxuseclass}{cell_input}
\begin{sphinxVerbatim}[commandchars=\\\{\}]
\PYG{n}{b}
\end{sphinxVerbatim}

\end{sphinxuseclass}\end{sphinxVerbatimInput}
\begin{sphinxVerbatimOutput}

\begin{sphinxuseclass}{cell_output}
\begin{sphinxVerbatim}[commandchars=\\\{\}]
[1, 2, 3, 4]
\end{sphinxVerbatim}

\end{sphinxuseclass}\end{sphinxVerbatimOutput}

\end{sphinxuseclass}
\sphinxAtStartPar
\sphinxstyleemphasis{\sphinxstylestrong{Likewise, if we modify \sphinxcode{\sphinxupquote{b}} by appending a 5, we change \sphinxcode{\sphinxupquote{a}}, too!}}

\begin{sphinxuseclass}{cell}\begin{sphinxVerbatimInput}

\begin{sphinxuseclass}{cell_input}
\begin{sphinxVerbatim}[commandchars=\\\{\}]
\PYG{n}{b}\PYG{o}{.}\PYG{n}{append}\PYG{p}{(}\PYG{l+m+mi}{5}\PYG{p}{)}
\end{sphinxVerbatim}

\end{sphinxuseclass}\end{sphinxVerbatimInput}

\end{sphinxuseclass}
\begin{sphinxuseclass}{cell}\begin{sphinxVerbatimInput}

\begin{sphinxuseclass}{cell_input}
\begin{sphinxVerbatim}[commandchars=\\\{\}]
\PYG{n}{a}
\end{sphinxVerbatim}

\end{sphinxuseclass}\end{sphinxVerbatimInput}
\begin{sphinxVerbatimOutput}

\begin{sphinxuseclass}{cell_output}
\begin{sphinxVerbatim}[commandchars=\\\{\}]
[1, 2, 3, 4, 5]
\end{sphinxVerbatim}

\end{sphinxuseclass}\end{sphinxVerbatimOutput}

\end{sphinxuseclass}
\begin{sphinxuseclass}{cell}\begin{sphinxVerbatimInput}

\begin{sphinxuseclass}{cell_input}
\begin{sphinxVerbatim}[commandchars=\\\{\}]
\PYG{n}{b}
\end{sphinxVerbatim}

\end{sphinxuseclass}\end{sphinxVerbatimInput}
\begin{sphinxVerbatimOutput}

\begin{sphinxuseclass}{cell_output}
\begin{sphinxVerbatim}[commandchars=\\\{\}]
[1, 2, 3, 4, 5]
\end{sphinxVerbatim}

\end{sphinxuseclass}\end{sphinxVerbatimOutput}

\end{sphinxuseclass}
\sphinxAtStartPar
The behavior is useful but a double\sphinxhyphen{}edged sword!
\sphinxhref{https://nedbatchelder.com/text/names.html}{Here} is a deeper discussion of this behavior.


\subsection{Dynamic references, strong types}
\label{\detokenize{mckinney_02_lecture:dynamic-references-strong-types}}\begin{quote}

\sphinxAtStartPar
In contrast with many compiled languages, such as Java and C++, object references in Python have no type associated with them.
\end{quote}

\sphinxAtStartPar
In Python,
\begin{enumerate}
\sphinxsetlistlabels{\arabic}{enumi}{enumii}{}{.}%
\item {} 
\sphinxAtStartPar
We do not declare variables and their types

\item {} 
\sphinxAtStartPar
We can change variables’ types because variables are only names that refer to objects

\end{enumerate}

\sphinxAtStartPar
\sphinxstyleemphasis{Dynamic references} mean we can reassign a variable to a new object in Python.
For example, we can reassign \sphinxcode{\sphinxupquote{a}} from a list to an integer to a string.

\begin{sphinxuseclass}{cell}\begin{sphinxVerbatimInput}

\begin{sphinxuseclass}{cell_input}
\begin{sphinxVerbatim}[commandchars=\\\{\}]
\PYG{n}{a}
\end{sphinxVerbatim}

\end{sphinxuseclass}\end{sphinxVerbatimInput}
\begin{sphinxVerbatimOutput}

\begin{sphinxuseclass}{cell_output}
\begin{sphinxVerbatim}[commandchars=\\\{\}]
[1, 2, 3, 4, 5]
\end{sphinxVerbatim}

\end{sphinxuseclass}\end{sphinxVerbatimOutput}

\end{sphinxuseclass}
\begin{sphinxuseclass}{cell}\begin{sphinxVerbatimInput}

\begin{sphinxuseclass}{cell_input}
\begin{sphinxVerbatim}[commandchars=\\\{\}]
\PYG{n+nb}{type}\PYG{p}{(}\PYG{n}{a}\PYG{p}{)}
\end{sphinxVerbatim}

\end{sphinxuseclass}\end{sphinxVerbatimInput}
\begin{sphinxVerbatimOutput}

\begin{sphinxuseclass}{cell_output}
\begin{sphinxVerbatim}[commandchars=\\\{\}]
list
\end{sphinxVerbatim}

\end{sphinxuseclass}\end{sphinxVerbatimOutput}

\end{sphinxuseclass}
\begin{sphinxuseclass}{cell}\begin{sphinxVerbatimInput}

\begin{sphinxuseclass}{cell_input}
\begin{sphinxVerbatim}[commandchars=\\\{\}]
\PYG{n}{a} \PYG{o}{=} \PYG{l+m+mi}{5}
\PYG{n+nb}{type}\PYG{p}{(}\PYG{n}{a}\PYG{p}{)}
\end{sphinxVerbatim}

\end{sphinxuseclass}\end{sphinxVerbatimInput}
\begin{sphinxVerbatimOutput}

\begin{sphinxuseclass}{cell_output}
\begin{sphinxVerbatim}[commandchars=\\\{\}]
int
\end{sphinxVerbatim}

\end{sphinxuseclass}\end{sphinxVerbatimOutput}

\end{sphinxuseclass}
\begin{sphinxuseclass}{cell}\begin{sphinxVerbatimInput}

\begin{sphinxuseclass}{cell_input}
\begin{sphinxVerbatim}[commandchars=\\\{\}]
\PYG{n}{a} \PYG{o}{=} \PYG{l+s+s1}{\PYGZsq{}}\PYG{l+s+s1}{foo}\PYG{l+s+s1}{\PYGZsq{}}
\PYG{n+nb}{type}\PYG{p}{(}\PYG{n}{a}\PYG{p}{)}
\end{sphinxVerbatim}

\end{sphinxuseclass}\end{sphinxVerbatimInput}
\begin{sphinxVerbatimOutput}

\begin{sphinxuseclass}{cell_output}
\begin{sphinxVerbatim}[commandchars=\\\{\}]
str
\end{sphinxVerbatim}

\end{sphinxuseclass}\end{sphinxVerbatimOutput}

\end{sphinxuseclass}
\sphinxAtStartPar
\sphinxstyleemphasis{Strong types} mean Python typically will not convert object types.
For example, the code  returns either \sphinxcode{\sphinxupquote{'55'}} as a string or \sphinxcode{\sphinxupquote{10}} as an integer in many programming languages.
However, \sphinxcode{\sphinxupquote{'5' + 5}} returns an error in Python.

\begin{sphinxuseclass}{cell}\begin{sphinxVerbatimInput}

\begin{sphinxuseclass}{cell_input}
\begin{sphinxVerbatim}[commandchars=\\\{\}]
\PYG{c+c1}{\PYGZsh{} \PYGZsq{}5\PYGZsq{} + 5}
\end{sphinxVerbatim}

\end{sphinxuseclass}\end{sphinxVerbatimInput}

\end{sphinxuseclass}
\sphinxAtStartPar
However, Python implicitly converts integers to floats.

\begin{sphinxuseclass}{cell}\begin{sphinxVerbatimInput}

\begin{sphinxuseclass}{cell_input}
\begin{sphinxVerbatim}[commandchars=\\\{\}]
\PYG{n}{a} \PYG{o}{=} \PYG{l+m+mf}{4.5}
\PYG{n}{b} \PYG{o}{=} \PYG{l+m+mi}{2}
\PYG{n+nb}{print}\PYG{p}{(}\PYG{l+s+sa}{f}\PYG{l+s+s1}{\PYGZsq{}}\PYG{l+s+s1}{a is }\PYG{l+s+si}{\PYGZob{}}\PYG{n+nb}{type}\PYG{p}{(}\PYG{n}{a}\PYG{p}{)}\PYG{l+s+si}{\PYGZcb{}}\PYG{l+s+s1}{, b is }\PYG{l+s+si}{\PYGZob{}}\PYG{n+nb}{type}\PYG{p}{(}\PYG{n}{b}\PYG{p}{)}\PYG{l+s+si}{\PYGZcb{}}\PYG{l+s+s1}{\PYGZsq{}}\PYG{p}{)}
\PYG{n}{a} \PYG{o}{/} \PYG{n}{b}
\end{sphinxVerbatim}

\end{sphinxuseclass}\end{sphinxVerbatimInput}
\begin{sphinxVerbatimOutput}

\begin{sphinxuseclass}{cell_output}
\begin{sphinxVerbatim}[commandchars=\\\{\}]
a is \PYGZlt{}class \PYGZsq{}float\PYGZsq{}\PYGZgt{}, b is \PYGZlt{}class \PYGZsq{}int\PYGZsq{}\PYGZgt{}
\end{sphinxVerbatim}

\begin{sphinxVerbatim}[commandchars=\\\{\}]
2.25
\end{sphinxVerbatim}

\end{sphinxuseclass}\end{sphinxVerbatimOutput}

\end{sphinxuseclass}
\sphinxAtStartPar
In the previous code cell:
\begin{enumerate}
\sphinxsetlistlabels{\arabic}{enumi}{enumii}{}{.}%
\item {} 
\sphinxAtStartPar
The ‘a is …’ output prints because of the explicit \sphinxcode{\sphinxupquote{print()}} function call

\item {} 
\sphinxAtStartPar
The output of \sphinxcode{\sphinxupquote{a / b}} prints (or displays) because it is the last line in the code cell

\end{enumerate}

\sphinxAtStartPar
If we want integer division (or floor division), we have to use \sphinxcode{\sphinxupquote{//}}.

\begin{sphinxuseclass}{cell}\begin{sphinxVerbatimInput}

\begin{sphinxuseclass}{cell_input}
\begin{sphinxVerbatim}[commandchars=\\\{\}]
\PYG{l+m+mi}{5} \PYG{o}{/}\PYG{o}{/} \PYG{l+m+mi}{2}
\end{sphinxVerbatim}

\end{sphinxuseclass}\end{sphinxVerbatimInput}
\begin{sphinxVerbatimOutput}

\begin{sphinxuseclass}{cell_output}
\begin{sphinxVerbatim}[commandchars=\\\{\}]
2
\end{sphinxVerbatim}

\end{sphinxuseclass}\end{sphinxVerbatimOutput}

\end{sphinxuseclass}
\begin{sphinxuseclass}{cell}\begin{sphinxVerbatimInput}

\begin{sphinxuseclass}{cell_input}
\begin{sphinxVerbatim}[commandchars=\\\{\}]
\PYG{l+m+mi}{5} \PYG{o}{/} \PYG{l+m+mi}{2}
\end{sphinxVerbatim}

\end{sphinxuseclass}\end{sphinxVerbatimInput}
\begin{sphinxVerbatimOutput}

\begin{sphinxuseclass}{cell_output}
\begin{sphinxVerbatim}[commandchars=\\\{\}]
2.5
\end{sphinxVerbatim}

\end{sphinxuseclass}\end{sphinxVerbatimOutput}

\end{sphinxuseclass}

\subsection{Attributes and methods}
\label{\detokenize{mckinney_02_lecture:attributes-and-methods}}
\sphinxAtStartPar
We can use tab completion to list attributes (characteristics stored inside objects) and methods (functions associated with objects).

\begin{sphinxuseclass}{cell}\begin{sphinxVerbatimInput}

\begin{sphinxuseclass}{cell_input}
\begin{sphinxVerbatim}[commandchars=\\\{\}]
\PYG{n}{a} \PYG{o}{=} \PYG{l+s+s1}{\PYGZsq{}}\PYG{l+s+s1}{foo}\PYG{l+s+s1}{\PYGZsq{}}
\end{sphinxVerbatim}

\end{sphinxuseclass}\end{sphinxVerbatimInput}

\end{sphinxuseclass}
\begin{sphinxuseclass}{cell}\begin{sphinxVerbatimInput}

\begin{sphinxuseclass}{cell_input}
\begin{sphinxVerbatim}[commandchars=\\\{\}]
\PYG{n}{a}\PYG{o}{.}\PYG{n}{capitalize}\PYG{p}{(}\PYG{p}{)}
\end{sphinxVerbatim}

\end{sphinxuseclass}\end{sphinxVerbatimInput}
\begin{sphinxVerbatimOutput}

\begin{sphinxuseclass}{cell_output}
\begin{sphinxVerbatim}[commandchars=\\\{\}]
\PYGZsq{}Foo\PYGZsq{}
\end{sphinxVerbatim}

\end{sphinxuseclass}\end{sphinxVerbatimOutput}

\end{sphinxuseclass}

\subsection{Imports}
\label{\detokenize{mckinney_02_lecture:imports}}\begin{quote}

\sphinxAtStartPar
In Python a module is simply a file with the .py extension containing Python code.
\end{quote}

\sphinxAtStartPar
We can import with \sphinxcode{\sphinxupquote{import}} statements, which have several syntaxes.
The basic syntax uses the module name as the prefix to separate module items from our current namespace.

\begin{sphinxuseclass}{cell}\begin{sphinxVerbatimInput}

\begin{sphinxuseclass}{cell_input}
\begin{sphinxVerbatim}[commandchars=\\\{\}]
\PYG{k+kn}{import} \PYG{n+nn}{pandas}
\end{sphinxVerbatim}

\end{sphinxuseclass}\end{sphinxVerbatimInput}

\end{sphinxuseclass}
\sphinxAtStartPar
The \sphinxcode{\sphinxupquote{import as}} syntax lets us define an abbreviated prefix.

\begin{sphinxuseclass}{cell}\begin{sphinxVerbatimInput}

\begin{sphinxuseclass}{cell_input}
\begin{sphinxVerbatim}[commandchars=\\\{\}]
\PYG{k+kn}{import} \PYG{n+nn}{pandas} \PYG{k}{as} \PYG{n+nn}{pd}
\end{sphinxVerbatim}

\end{sphinxuseclass}\end{sphinxVerbatimInput}

\end{sphinxuseclass}
\sphinxAtStartPar
We can also import one or more items from a package into our namespace with the following syntaxes.

\begin{sphinxuseclass}{cell}\begin{sphinxVerbatimInput}

\begin{sphinxuseclass}{cell_input}
\begin{sphinxVerbatim}[commandchars=\\\{\}]
\PYG{k+kn}{from} \PYG{n+nn}{pandas} \PYG{k+kn}{import} \PYG{n}{DataFrame}
\end{sphinxVerbatim}

\end{sphinxuseclass}\end{sphinxVerbatimInput}

\end{sphinxuseclass}
\begin{sphinxuseclass}{cell}\begin{sphinxVerbatimInput}

\begin{sphinxuseclass}{cell_input}
\begin{sphinxVerbatim}[commandchars=\\\{\}]
\PYG{k+kn}{from} \PYG{n+nn}{pandas} \PYG{k+kn}{import} \PYG{n}{DataFrame} \PYG{k}{as} \PYG{n}{df}
\end{sphinxVerbatim}

\end{sphinxuseclass}\end{sphinxVerbatimInput}

\end{sphinxuseclass}

\subsection{Binary operators and comparisons}
\label{\detokenize{mckinney_02_lecture:binary-operators-and-comparisons}}
\sphinxAtStartPar
Binary operators work like Excel.

\begin{sphinxuseclass}{cell}\begin{sphinxVerbatimInput}

\begin{sphinxuseclass}{cell_input}
\begin{sphinxVerbatim}[commandchars=\\\{\}]
\PYG{l+m+mi}{5} \PYG{o}{\PYGZhy{}} \PYG{l+m+mi}{7}
\end{sphinxVerbatim}

\end{sphinxuseclass}\end{sphinxVerbatimInput}
\begin{sphinxVerbatimOutput}

\begin{sphinxuseclass}{cell_output}
\begin{sphinxVerbatim}[commandchars=\\\{\}]
\PYGZhy{}2
\end{sphinxVerbatim}

\end{sphinxuseclass}\end{sphinxVerbatimOutput}

\end{sphinxuseclass}
\begin{sphinxuseclass}{cell}\begin{sphinxVerbatimInput}

\begin{sphinxuseclass}{cell_input}
\begin{sphinxVerbatim}[commandchars=\\\{\}]
\PYG{l+m+mi}{12} \PYG{o}{+} \PYG{l+m+mf}{21.5}
\end{sphinxVerbatim}

\end{sphinxuseclass}\end{sphinxVerbatimInput}
\begin{sphinxVerbatimOutput}

\begin{sphinxuseclass}{cell_output}
\begin{sphinxVerbatim}[commandchars=\\\{\}]
33.5
\end{sphinxVerbatim}

\end{sphinxuseclass}\end{sphinxVerbatimOutput}

\end{sphinxuseclass}
\begin{sphinxuseclass}{cell}\begin{sphinxVerbatimInput}

\begin{sphinxuseclass}{cell_input}
\begin{sphinxVerbatim}[commandchars=\\\{\}]
\PYG{l+m+mi}{5} \PYG{o}{\PYGZlt{}}\PYG{o}{=} \PYG{l+m+mi}{2}
\end{sphinxVerbatim}

\end{sphinxuseclass}\end{sphinxVerbatimInput}
\begin{sphinxVerbatimOutput}

\begin{sphinxuseclass}{cell_output}
\begin{sphinxVerbatim}[commandchars=\\\{\}]
False
\end{sphinxVerbatim}

\end{sphinxuseclass}\end{sphinxVerbatimOutput}

\end{sphinxuseclass}
\sphinxAtStartPar
We can operate during an assignment to avoid two names referring to the same object.

\begin{sphinxuseclass}{cell}\begin{sphinxVerbatimInput}

\begin{sphinxuseclass}{cell_input}
\begin{sphinxVerbatim}[commandchars=\\\{\}]
\PYG{n}{a} \PYG{o}{=} \PYG{p}{[}\PYG{l+m+mi}{1}\PYG{p}{,} \PYG{l+m+mi}{2}\PYG{p}{,} \PYG{l+m+mi}{3}\PYG{p}{]}
\PYG{n}{b} \PYG{o}{=} \PYG{n}{a}
\PYG{n}{c} \PYG{o}{=} \PYG{n+nb}{list}\PYG{p}{(}\PYG{n}{a}\PYG{p}{)}
\end{sphinxVerbatim}

\end{sphinxuseclass}\end{sphinxVerbatimInput}

\end{sphinxuseclass}
\begin{sphinxuseclass}{cell}\begin{sphinxVerbatimInput}

\begin{sphinxuseclass}{cell_input}
\begin{sphinxVerbatim}[commandchars=\\\{\}]
\PYG{n}{a} \PYG{o+ow}{is} \PYG{n}{b}
\end{sphinxVerbatim}

\end{sphinxuseclass}\end{sphinxVerbatimInput}
\begin{sphinxVerbatimOutput}

\begin{sphinxuseclass}{cell_output}
\begin{sphinxVerbatim}[commandchars=\\\{\}]
True
\end{sphinxVerbatim}

\end{sphinxuseclass}\end{sphinxVerbatimOutput}

\end{sphinxuseclass}
\begin{sphinxuseclass}{cell}\begin{sphinxVerbatimInput}

\begin{sphinxuseclass}{cell_input}
\begin{sphinxVerbatim}[commandchars=\\\{\}]
\PYG{n}{a} \PYG{o+ow}{is} \PYG{n}{c}
\end{sphinxVerbatim}

\end{sphinxuseclass}\end{sphinxVerbatimInput}
\begin{sphinxVerbatimOutput}

\begin{sphinxuseclass}{cell_output}
\begin{sphinxVerbatim}[commandchars=\\\{\}]
False
\end{sphinxVerbatim}

\end{sphinxuseclass}\end{sphinxVerbatimOutput}

\end{sphinxuseclass}
\sphinxAtStartPar
Here \sphinxcode{\sphinxupquote{a}} and \sphinxcode{\sphinxupquote{c}} have the same \sphinxstyleemphasis{values} but are not the same object!

\begin{sphinxuseclass}{cell}\begin{sphinxVerbatimInput}

\begin{sphinxuseclass}{cell_input}
\begin{sphinxVerbatim}[commandchars=\\\{\}]
\PYG{n}{a} \PYG{o}{==} \PYG{n}{c}
\end{sphinxVerbatim}

\end{sphinxuseclass}\end{sphinxVerbatimInput}
\begin{sphinxVerbatimOutput}

\begin{sphinxuseclass}{cell_output}
\begin{sphinxVerbatim}[commandchars=\\\{\}]
True
\end{sphinxVerbatim}

\end{sphinxuseclass}\end{sphinxVerbatimOutput}

\end{sphinxuseclass}
\begin{sphinxuseclass}{cell}\begin{sphinxVerbatimInput}

\begin{sphinxuseclass}{cell_input}
\begin{sphinxVerbatim}[commandchars=\\\{\}]
\PYG{n}{a} \PYG{o+ow}{is} \PYG{o+ow}{not} \PYG{n}{c}
\end{sphinxVerbatim}

\end{sphinxuseclass}\end{sphinxVerbatimInput}
\begin{sphinxVerbatimOutput}

\begin{sphinxuseclass}{cell_output}
\begin{sphinxVerbatim}[commandchars=\\\{\}]
True
\end{sphinxVerbatim}

\end{sphinxuseclass}\end{sphinxVerbatimOutput}

\end{sphinxuseclass}
\sphinxAtStartPar
In Python, \sphinxcode{\sphinxupquote{=}} is the assignment operator, \sphinxcode{\sphinxupquote{==}} tests equality, and \sphinxcode{\sphinxupquote{!=}} tests inequality.

\begin{sphinxuseclass}{cell}\begin{sphinxVerbatimInput}

\begin{sphinxuseclass}{cell_input}
\begin{sphinxVerbatim}[commandchars=\\\{\}]
\PYG{n}{a} \PYG{o}{==} \PYG{n}{c}
\end{sphinxVerbatim}

\end{sphinxuseclass}\end{sphinxVerbatimInput}
\begin{sphinxVerbatimOutput}

\begin{sphinxuseclass}{cell_output}
\begin{sphinxVerbatim}[commandchars=\\\{\}]
True
\end{sphinxVerbatim}

\end{sphinxuseclass}\end{sphinxVerbatimOutput}

\end{sphinxuseclass}
\begin{sphinxuseclass}{cell}\begin{sphinxVerbatimInput}

\begin{sphinxuseclass}{cell_input}
\begin{sphinxVerbatim}[commandchars=\\\{\}]
\PYG{n}{a} \PYG{o}{!=} \PYG{n}{c}
\end{sphinxVerbatim}

\end{sphinxuseclass}\end{sphinxVerbatimInput}
\begin{sphinxVerbatimOutput}

\begin{sphinxuseclass}{cell_output}
\begin{sphinxVerbatim}[commandchars=\\\{\}]
False
\end{sphinxVerbatim}

\end{sphinxuseclass}\end{sphinxVerbatimOutput}

\end{sphinxuseclass}
\sphinxAtStartPar
\sphinxcode{\sphinxupquote{a}} and \sphinxcode{\sphinxupquote{c}} have the same values but reference different objects in memory.

\sphinxAtStartPar
\sphinxstyleemphasis{\sphinxstylestrong{Table 2\sphinxhyphen{}1}} from McKinney summarizes the binary operators.
\begin{itemize}
\item {} 
\sphinxAtStartPar
\sphinxcode{\sphinxupquote{a + b}} : Add a and b

\item {} 
\sphinxAtStartPar
\sphinxcode{\sphinxupquote{a \sphinxhyphen{} b}} : Subtract b from a

\item {} 
\sphinxAtStartPar
\sphinxcode{\sphinxupquote{a * b}} : Multiply a by b

\item {} 
\sphinxAtStartPar
\sphinxcode{\sphinxupquote{a / b}} : Divide a by b

\item {} 
\sphinxAtStartPar
\sphinxcode{\sphinxupquote{a // b}} : Floor\sphinxhyphen{}divide a by b, dropping any fractional remainder

\item {} 
\sphinxAtStartPar
\sphinxcode{\sphinxupquote{a ** b}} : Raise a to the b power

\item {} 
\sphinxAtStartPar
\sphinxcode{\sphinxupquote{a \& b}} : True if both a and b are True; for integers, take the bitwise AND

\item {} 
\sphinxAtStartPar
\sphinxcode{\sphinxupquote{a | b}} : True if either a or b is True; for integers, take the bitwise OR

\item {} 
\sphinxAtStartPar
\sphinxcode{\sphinxupquote{a \textasciicircum{} b}} : For booleans, True if a or b is True , but not both; for integers, take the bitwise EXCLUSIVE\sphinxhyphen{}OR

\item {} 
\sphinxAtStartPar
\sphinxcode{\sphinxupquote{a == b}} : True if a equals b

\item {} 
\sphinxAtStartPar
\sphinxcode{\sphinxupquote{a != b}}: True if a is not equal to b

\item {} 
\sphinxAtStartPar
\sphinxcode{\sphinxupquote{a <= b, a < b}} : True if a is less than (less than or equal) to b

\item {} 
\sphinxAtStartPar
\sphinxcode{\sphinxupquote{a > b, a >= b}}: True if a is greater than (greater than or equal) to b

\item {} 
\sphinxAtStartPar
\sphinxcode{\sphinxupquote{a is b}} : True if a and b reference the same Python object

\item {} 
\sphinxAtStartPar
\sphinxcode{\sphinxupquote{a is not b}} : True if a and b reference different Python objects

\end{itemize}


\subsection{Mutable and immutable objects}
\label{\detokenize{mckinney_02_lecture:mutable-and-immutable-objects}}\begin{quote}

\sphinxAtStartPar
Most objects in Python, such as lists, dicts, NumPy arrays, and most user\sphinxhyphen{}defined
types (classes), are mutable. This means that the object or values that they contain can
be modified.
\end{quote}

\sphinxAtStartPar
Lists are mutable, so we can modify them.

\begin{sphinxuseclass}{cell}\begin{sphinxVerbatimInput}

\begin{sphinxuseclass}{cell_input}
\begin{sphinxVerbatim}[commandchars=\\\{\}]
\PYG{n}{a\PYGZus{}list} \PYG{o}{=} \PYG{p}{[}\PYG{l+s+s1}{\PYGZsq{}}\PYG{l+s+s1}{foo}\PYG{l+s+s1}{\PYGZsq{}}\PYG{p}{,} \PYG{l+m+mi}{2}\PYG{p}{,} \PYG{p}{[}\PYG{l+m+mi}{4}\PYG{p}{,} \PYG{l+m+mi}{5}\PYG{p}{]}\PYG{p}{]}
\end{sphinxVerbatim}

\end{sphinxuseclass}\end{sphinxVerbatimInput}

\end{sphinxuseclass}
\sphinxAtStartPar
\sphinxstyleemphasis{\sphinxstylestrong{Python is zero\sphinxhyphen{}indexed! The first element has a zero subscript \sphinxcode{\sphinxupquote{{[}0{]}}}!}}

\begin{sphinxuseclass}{cell}\begin{sphinxVerbatimInput}

\begin{sphinxuseclass}{cell_input}
\begin{sphinxVerbatim}[commandchars=\\\{\}]
\PYG{n}{a\PYGZus{}list}\PYG{p}{[}\PYG{l+m+mi}{0}\PYG{p}{]}
\end{sphinxVerbatim}

\end{sphinxuseclass}\end{sphinxVerbatimInput}
\begin{sphinxVerbatimOutput}

\begin{sphinxuseclass}{cell_output}
\begin{sphinxVerbatim}[commandchars=\\\{\}]
\PYGZsq{}foo\PYGZsq{}
\end{sphinxVerbatim}

\end{sphinxuseclass}\end{sphinxVerbatimOutput}

\end{sphinxuseclass}
\begin{sphinxuseclass}{cell}\begin{sphinxVerbatimInput}

\begin{sphinxuseclass}{cell_input}
\begin{sphinxVerbatim}[commandchars=\\\{\}]
\PYG{n}{a\PYGZus{}list}\PYG{p}{[}\PYG{l+m+mi}{2}\PYG{p}{]}
\end{sphinxVerbatim}

\end{sphinxuseclass}\end{sphinxVerbatimInput}
\begin{sphinxVerbatimOutput}

\begin{sphinxuseclass}{cell_output}
\begin{sphinxVerbatim}[commandchars=\\\{\}]
[4, 5]
\end{sphinxVerbatim}

\end{sphinxuseclass}\end{sphinxVerbatimOutput}

\end{sphinxuseclass}
\begin{sphinxuseclass}{cell}\begin{sphinxVerbatimInput}

\begin{sphinxuseclass}{cell_input}
\begin{sphinxVerbatim}[commandchars=\\\{\}]
\PYG{n}{a\PYGZus{}list}\PYG{p}{[}\PYG{l+m+mi}{2}\PYG{p}{]}\PYG{p}{[}\PYG{l+m+mi}{0}\PYG{p}{]}
\end{sphinxVerbatim}

\end{sphinxuseclass}\end{sphinxVerbatimInput}
\begin{sphinxVerbatimOutput}

\begin{sphinxuseclass}{cell_output}
\begin{sphinxVerbatim}[commandchars=\\\{\}]
4
\end{sphinxVerbatim}

\end{sphinxuseclass}\end{sphinxVerbatimOutput}

\end{sphinxuseclass}
\begin{sphinxuseclass}{cell}\begin{sphinxVerbatimInput}

\begin{sphinxuseclass}{cell_input}
\begin{sphinxVerbatim}[commandchars=\\\{\}]
\PYG{n}{a\PYGZus{}list}\PYG{p}{[}\PYG{l+m+mi}{2}\PYG{p}{]} \PYG{o}{=} \PYG{p}{(}\PYG{l+m+mi}{3}\PYG{p}{,} \PYG{l+m+mi}{4}\PYG{p}{)}
\end{sphinxVerbatim}

\end{sphinxuseclass}\end{sphinxVerbatimInput}

\end{sphinxuseclass}
\sphinxAtStartPar
Tuples are \sphinxstyleemphasis{immutable}, so we cannot modify them.

\begin{sphinxuseclass}{cell}\begin{sphinxVerbatimInput}

\begin{sphinxuseclass}{cell_input}
\begin{sphinxVerbatim}[commandchars=\\\{\}]
\PYG{n}{a\PYGZus{}tuple} \PYG{o}{=} \PYG{p}{(}\PYG{l+m+mi}{3}\PYG{p}{,} \PYG{l+m+mi}{5}\PYG{p}{,} \PYG{p}{(}\PYG{l+m+mi}{4}\PYG{p}{,} \PYG{l+m+mi}{5}\PYG{p}{)}\PYG{p}{)}
\end{sphinxVerbatim}

\end{sphinxuseclass}\end{sphinxVerbatimInput}

\end{sphinxuseclass}
\sphinxAtStartPar
The Python interpreter returns an error if we try to modify \sphinxcode{\sphinxupquote{a\_tuple}} because tuples are immutable.

\begin{sphinxuseclass}{cell}\begin{sphinxVerbatimInput}

\begin{sphinxuseclass}{cell_input}
\begin{sphinxVerbatim}[commandchars=\\\{\}]
\PYG{c+c1}{\PYGZsh{} a\PYGZus{}tuple[1] = \PYGZsq{}four\PYGZsq{}}
\end{sphinxVerbatim}

\end{sphinxuseclass}\end{sphinxVerbatimInput}

\end{sphinxuseclass}
\sphinxAtStartPar
\sphinxstyleemphasis{\sphinxstylestrong{Note:}}
Tuples do not require \sphinxcode{\sphinxupquote{()}}, but \sphinxcode{\sphinxupquote{()}} improve readability.

\begin{sphinxuseclass}{cell}\begin{sphinxVerbatimInput}

\begin{sphinxuseclass}{cell_input}
\begin{sphinxVerbatim}[commandchars=\\\{\}]
\PYG{n}{test} \PYG{o}{=} \PYG{l+m+mi}{1}\PYG{p}{,} \PYG{l+m+mi}{2}\PYG{p}{,} \PYG{l+m+mi}{3}
\end{sphinxVerbatim}

\end{sphinxuseclass}\end{sphinxVerbatimInput}

\end{sphinxuseclass}
\begin{sphinxuseclass}{cell}\begin{sphinxVerbatimInput}

\begin{sphinxuseclass}{cell_input}
\begin{sphinxVerbatim}[commandchars=\\\{\}]
\PYG{n+nb}{type}\PYG{p}{(}\PYG{n}{test}\PYG{p}{)}
\end{sphinxVerbatim}

\end{sphinxuseclass}\end{sphinxVerbatimInput}
\begin{sphinxVerbatimOutput}

\begin{sphinxuseclass}{cell_output}
\begin{sphinxVerbatim}[commandchars=\\\{\}]
tuple
\end{sphinxVerbatim}

\end{sphinxuseclass}\end{sphinxVerbatimOutput}

\end{sphinxuseclass}
\sphinxAtStartPar
We will learn more about Python’s built\sphinxhyphen{}in data structures in McKinney chapter 3.


\section{Scalar Types}
\label{\detokenize{mckinney_02_lecture:scalar-types}}\begin{quote}

\sphinxAtStartPar
Python along with its standard library has a small set of built\sphinxhyphen{}in types for handling numerical data, strings, boolean ( True or False ) values, and dates and time. These “single value” types are sometimes called scalar types and we refer to them in this book as scalars. See Table 2\sphinxhyphen{}4 for a list of the main scalar types. Date and time handling will be discussed separately, as these are provided by the datetime module in the standard  library.
\end{quote}

\sphinxAtStartPar
\sphinxstyleemphasis{\sphinxstylestrong{Table 2\sphinxhyphen{}2}} from McKinney lists the standard scalar types.
\begin{itemize}
\item {} 
\sphinxAtStartPar
\sphinxcode{\sphinxupquote{None}}: The Python “null” value (only one instance of the None object exists)

\item {} 
\sphinxAtStartPar
\sphinxcode{\sphinxupquote{str}}: String type; holds Unicode (UTF\sphinxhyphen{}8 encoded) strings

\item {} 
\sphinxAtStartPar
\sphinxcode{\sphinxupquote{bytes}}: Raw ASCII bytes (or Unicode encoded as bytes)

\item {} 
\sphinxAtStartPar
\sphinxcode{\sphinxupquote{float}}: Double\sphinxhyphen{}precision (64\sphinxhyphen{}bit) floating\sphinxhyphen{}point number (note there is no separate double type)

\item {} 
\sphinxAtStartPar
\sphinxcode{\sphinxupquote{bool}}: A True or False value

\item {} 
\sphinxAtStartPar
\sphinxcode{\sphinxupquote{int}}: Arbitrary precision signed integer

\end{itemize}


\subsection{Numeric types}
\label{\detokenize{mckinney_02_lecture:numeric-types}}
\sphinxAtStartPar
In Python, integers are unbounded, and \sphinxcode{\sphinxupquote{**}} raises numbers to a power.
So, \sphinxcode{\sphinxupquote{ival ** 6}} is \$17239781\textasciicircum{}6\$.

\begin{sphinxuseclass}{cell}\begin{sphinxVerbatimInput}

\begin{sphinxuseclass}{cell_input}
\begin{sphinxVerbatim}[commandchars=\\\{\}]
\PYG{n}{ival} \PYG{o}{=} \PYG{l+m+mi}{17239871}
\PYG{n}{ival} \PYG{o}{*}\PYG{o}{*} \PYG{l+m+mi}{6}
\end{sphinxVerbatim}

\end{sphinxuseclass}\end{sphinxVerbatimInput}
\begin{sphinxVerbatimOutput}

\begin{sphinxuseclass}{cell_output}
\begin{sphinxVerbatim}[commandchars=\\\{\}]
26254519291092456596965462913230729701102721
\end{sphinxVerbatim}

\end{sphinxuseclass}\end{sphinxVerbatimOutput}

\end{sphinxuseclass}
\sphinxAtStartPar
Floats (decimal numbers) are 64\sphinxhyphen{}bit in Python.

\begin{sphinxuseclass}{cell}\begin{sphinxVerbatimInput}

\begin{sphinxuseclass}{cell_input}
\begin{sphinxVerbatim}[commandchars=\\\{\}]
\PYG{n}{fval} \PYG{o}{=} \PYG{l+m+mf}{7.243}
\end{sphinxVerbatim}

\end{sphinxuseclass}\end{sphinxVerbatimInput}

\end{sphinxuseclass}
\begin{sphinxuseclass}{cell}\begin{sphinxVerbatimInput}

\begin{sphinxuseclass}{cell_input}
\begin{sphinxVerbatim}[commandchars=\\\{\}]
\PYG{n+nb}{type}\PYG{p}{(}\PYG{n}{fval}\PYG{p}{)}
\end{sphinxVerbatim}

\end{sphinxuseclass}\end{sphinxVerbatimInput}
\begin{sphinxVerbatimOutput}

\begin{sphinxuseclass}{cell_output}
\begin{sphinxVerbatim}[commandchars=\\\{\}]
float
\end{sphinxVerbatim}

\end{sphinxuseclass}\end{sphinxVerbatimOutput}

\end{sphinxuseclass}
\sphinxAtStartPar
Dividing integers yields a float, if necessary.

\begin{sphinxuseclass}{cell}\begin{sphinxVerbatimInput}

\begin{sphinxuseclass}{cell_input}
\begin{sphinxVerbatim}[commandchars=\\\{\}]
\PYG{l+m+mi}{3} \PYG{o}{/} \PYG{l+m+mi}{2}
\end{sphinxVerbatim}

\end{sphinxuseclass}\end{sphinxVerbatimInput}
\begin{sphinxVerbatimOutput}

\begin{sphinxuseclass}{cell_output}
\begin{sphinxVerbatim}[commandchars=\\\{\}]
1.5
\end{sphinxVerbatim}

\end{sphinxuseclass}\end{sphinxVerbatimOutput}

\end{sphinxuseclass}
\sphinxAtStartPar
We have to use \sphinxcode{\sphinxupquote{//}} if we want C\sphinxhyphen{}style integer division (i.e., \$3 / 2 = 1\$).

\begin{sphinxuseclass}{cell}\begin{sphinxVerbatimInput}

\begin{sphinxuseclass}{cell_input}
\begin{sphinxVerbatim}[commandchars=\\\{\}]
\PYG{l+m+mi}{3} \PYG{o}{/}\PYG{o}{/} \PYG{l+m+mi}{2}
\end{sphinxVerbatim}

\end{sphinxuseclass}\end{sphinxVerbatimInput}
\begin{sphinxVerbatimOutput}

\begin{sphinxuseclass}{cell_output}
\begin{sphinxVerbatim}[commandchars=\\\{\}]
1
\end{sphinxVerbatim}

\end{sphinxuseclass}\end{sphinxVerbatimOutput}

\end{sphinxuseclass}

\subsection{Booleans}
\label{\detokenize{mckinney_02_lecture:booleans}}\begin{quote}

\sphinxAtStartPar
The two Boolean values in Python are written as True and False. Comparisons and other conditional expressions evaluate to either True or False. Boolean values are combined with the and and or keywords.
\end{quote}

\sphinxAtStartPar
Python is case sensitive, so we must type Booleans as \sphinxcode{\sphinxupquote{True}} and \sphinxcode{\sphinxupquote{False}}.

\begin{sphinxuseclass}{cell}\begin{sphinxVerbatimInput}

\begin{sphinxuseclass}{cell_input}
\begin{sphinxVerbatim}[commandchars=\\\{\}]
\PYG{k+kc}{True} \PYG{o+ow}{and} \PYG{k+kc}{True}
\end{sphinxVerbatim}

\end{sphinxuseclass}\end{sphinxVerbatimInput}
\begin{sphinxVerbatimOutput}

\begin{sphinxuseclass}{cell_output}
\begin{sphinxVerbatim}[commandchars=\\\{\}]
True
\end{sphinxVerbatim}

\end{sphinxuseclass}\end{sphinxVerbatimOutput}

\end{sphinxuseclass}
\begin{sphinxuseclass}{cell}\begin{sphinxVerbatimInput}

\begin{sphinxuseclass}{cell_input}
\begin{sphinxVerbatim}[commandchars=\\\{\}]
\PYG{p}{(}\PYG{l+m+mi}{5} \PYG{o}{\PYGZgt{}} \PYG{l+m+mi}{1}\PYG{p}{)} \PYG{o+ow}{and} \PYG{p}{(}\PYG{l+m+mi}{10} \PYG{o}{\PYGZgt{}} \PYG{l+m+mi}{5}\PYG{p}{)}
\end{sphinxVerbatim}

\end{sphinxuseclass}\end{sphinxVerbatimInput}
\begin{sphinxVerbatimOutput}

\begin{sphinxuseclass}{cell_output}
\begin{sphinxVerbatim}[commandchars=\\\{\}]
True
\end{sphinxVerbatim}

\end{sphinxuseclass}\end{sphinxVerbatimOutput}

\end{sphinxuseclass}
\begin{sphinxuseclass}{cell}\begin{sphinxVerbatimInput}

\begin{sphinxuseclass}{cell_input}
\begin{sphinxVerbatim}[commandchars=\\\{\}]
\PYG{k+kc}{False} \PYG{o+ow}{or} \PYG{k+kc}{True}
\end{sphinxVerbatim}

\end{sphinxuseclass}\end{sphinxVerbatimInput}
\begin{sphinxVerbatimOutput}

\begin{sphinxuseclass}{cell_output}
\begin{sphinxVerbatim}[commandchars=\\\{\}]
True
\end{sphinxVerbatim}

\end{sphinxuseclass}\end{sphinxVerbatimOutput}

\end{sphinxuseclass}
\begin{sphinxuseclass}{cell}\begin{sphinxVerbatimInput}

\begin{sphinxuseclass}{cell_input}
\begin{sphinxVerbatim}[commandchars=\\\{\}]
\PYG{p}{(}\PYG{l+m+mi}{5} \PYG{o}{\PYGZgt{}} \PYG{l+m+mi}{1}\PYG{p}{)} \PYG{o+ow}{or} \PYG{p}{(}\PYG{l+m+mi}{10} \PYG{o}{\PYGZgt{}} \PYG{l+m+mi}{5}\PYG{p}{)}
\end{sphinxVerbatim}

\end{sphinxuseclass}\end{sphinxVerbatimInput}
\begin{sphinxVerbatimOutput}

\begin{sphinxuseclass}{cell_output}
\begin{sphinxVerbatim}[commandchars=\\\{\}]
True
\end{sphinxVerbatim}

\end{sphinxuseclass}\end{sphinxVerbatimOutput}

\end{sphinxuseclass}
\sphinxAtStartPar
We can substitute \sphinxcode{\sphinxupquote{\&}} for \sphinxcode{\sphinxupquote{and}} and \sphinxcode{\sphinxupquote{|}} for \sphinxcode{\sphinxupquote{or}}.

\begin{sphinxuseclass}{cell}\begin{sphinxVerbatimInput}

\begin{sphinxuseclass}{cell_input}
\begin{sphinxVerbatim}[commandchars=\\\{\}]
\PYG{k+kc}{True} \PYG{o}{\PYGZam{}} \PYG{k+kc}{True}
\end{sphinxVerbatim}

\end{sphinxuseclass}\end{sphinxVerbatimInput}
\begin{sphinxVerbatimOutput}

\begin{sphinxuseclass}{cell_output}
\begin{sphinxVerbatim}[commandchars=\\\{\}]
True
\end{sphinxVerbatim}

\end{sphinxuseclass}\end{sphinxVerbatimOutput}

\end{sphinxuseclass}
\begin{sphinxuseclass}{cell}\begin{sphinxVerbatimInput}

\begin{sphinxuseclass}{cell_input}
\begin{sphinxVerbatim}[commandchars=\\\{\}]
\PYG{k+kc}{False} \PYG{o}{|} \PYG{k+kc}{True}
\end{sphinxVerbatim}

\end{sphinxuseclass}\end{sphinxVerbatimInput}
\begin{sphinxVerbatimOutput}

\begin{sphinxuseclass}{cell_output}
\begin{sphinxVerbatim}[commandchars=\\\{\}]
True
\end{sphinxVerbatim}

\end{sphinxuseclass}\end{sphinxVerbatimOutput}

\end{sphinxuseclass}

\subsection{Type casting}
\label{\detokenize{mckinney_02_lecture:type-casting}}
\sphinxAtStartPar
We can “recast” variables to change their types.

\begin{sphinxuseclass}{cell}\begin{sphinxVerbatimInput}

\begin{sphinxuseclass}{cell_input}
\begin{sphinxVerbatim}[commandchars=\\\{\}]
\PYG{n}{s} \PYG{o}{=} \PYG{l+s+s1}{\PYGZsq{}}\PYG{l+s+s1}{3.14159}\PYG{l+s+s1}{\PYGZsq{}}
\end{sphinxVerbatim}

\end{sphinxuseclass}\end{sphinxVerbatimInput}

\end{sphinxuseclass}
\begin{sphinxuseclass}{cell}\begin{sphinxVerbatimInput}

\begin{sphinxuseclass}{cell_input}
\begin{sphinxVerbatim}[commandchars=\\\{\}]
\PYG{n+nb}{type}\PYG{p}{(}\PYG{n}{s}\PYG{p}{)}
\end{sphinxVerbatim}

\end{sphinxuseclass}\end{sphinxVerbatimInput}
\begin{sphinxVerbatimOutput}

\begin{sphinxuseclass}{cell_output}
\begin{sphinxVerbatim}[commandchars=\\\{\}]
str
\end{sphinxVerbatim}

\end{sphinxuseclass}\end{sphinxVerbatimOutput}

\end{sphinxuseclass}
\begin{sphinxuseclass}{cell}\begin{sphinxVerbatimInput}

\begin{sphinxuseclass}{cell_input}
\begin{sphinxVerbatim}[commandchars=\\\{\}]
\PYG{l+m+mi}{1} \PYG{o}{+} \PYG{n+nb}{float}\PYG{p}{(}\PYG{n}{s}\PYG{p}{)}
\end{sphinxVerbatim}

\end{sphinxuseclass}\end{sphinxVerbatimInput}
\begin{sphinxVerbatimOutput}

\begin{sphinxuseclass}{cell_output}
\begin{sphinxVerbatim}[commandchars=\\\{\}]
4.14159
\end{sphinxVerbatim}

\end{sphinxuseclass}\end{sphinxVerbatimOutput}

\end{sphinxuseclass}
\begin{sphinxuseclass}{cell}\begin{sphinxVerbatimInput}

\begin{sphinxuseclass}{cell_input}
\begin{sphinxVerbatim}[commandchars=\\\{\}]
\PYG{n}{fval} \PYG{o}{=} \PYG{n+nb}{float}\PYG{p}{(}\PYG{n}{s}\PYG{p}{)}
\end{sphinxVerbatim}

\end{sphinxuseclass}\end{sphinxVerbatimInput}

\end{sphinxuseclass}
\begin{sphinxuseclass}{cell}\begin{sphinxVerbatimInput}

\begin{sphinxuseclass}{cell_input}
\begin{sphinxVerbatim}[commandchars=\\\{\}]
\PYG{n+nb}{type}\PYG{p}{(}\PYG{n}{fval}\PYG{p}{)}
\end{sphinxVerbatim}

\end{sphinxuseclass}\end{sphinxVerbatimInput}
\begin{sphinxVerbatimOutput}

\begin{sphinxuseclass}{cell_output}
\begin{sphinxVerbatim}[commandchars=\\\{\}]
float
\end{sphinxVerbatim}

\end{sphinxuseclass}\end{sphinxVerbatimOutput}

\end{sphinxuseclass}
\begin{sphinxuseclass}{cell}\begin{sphinxVerbatimInput}

\begin{sphinxuseclass}{cell_input}
\begin{sphinxVerbatim}[commandchars=\\\{\}]
\PYG{n+nb}{int}\PYG{p}{(}\PYG{n}{fval}\PYG{p}{)}
\end{sphinxVerbatim}

\end{sphinxuseclass}\end{sphinxVerbatimInput}
\begin{sphinxVerbatimOutput}

\begin{sphinxuseclass}{cell_output}
\begin{sphinxVerbatim}[commandchars=\\\{\}]
3
\end{sphinxVerbatim}

\end{sphinxuseclass}\end{sphinxVerbatimOutput}

\end{sphinxuseclass}
\sphinxAtStartPar
Zero is Boolean \sphinxcode{\sphinxupquote{False}}, and all other values are Boolean \sphinxcode{\sphinxupquote{True}}.

\begin{sphinxuseclass}{cell}\begin{sphinxVerbatimInput}

\begin{sphinxuseclass}{cell_input}
\begin{sphinxVerbatim}[commandchars=\\\{\}]
\PYG{n+nb}{bool}\PYG{p}{(}\PYG{l+m+mi}{0}\PYG{p}{)}
\end{sphinxVerbatim}

\end{sphinxuseclass}\end{sphinxVerbatimInput}
\begin{sphinxVerbatimOutput}

\begin{sphinxuseclass}{cell_output}
\begin{sphinxVerbatim}[commandchars=\\\{\}]
False
\end{sphinxVerbatim}

\end{sphinxuseclass}\end{sphinxVerbatimOutput}

\end{sphinxuseclass}
\begin{sphinxuseclass}{cell}\begin{sphinxVerbatimInput}

\begin{sphinxuseclass}{cell_input}
\begin{sphinxVerbatim}[commandchars=\\\{\}]
\PYG{n+nb}{bool}\PYG{p}{(}\PYG{l+m+mi}{1}\PYG{p}{)}
\end{sphinxVerbatim}

\end{sphinxuseclass}\end{sphinxVerbatimInput}
\begin{sphinxVerbatimOutput}

\begin{sphinxuseclass}{cell_output}
\begin{sphinxVerbatim}[commandchars=\\\{\}]
True
\end{sphinxVerbatim}

\end{sphinxuseclass}\end{sphinxVerbatimOutput}

\end{sphinxuseclass}
\begin{sphinxuseclass}{cell}\begin{sphinxVerbatimInput}

\begin{sphinxuseclass}{cell_input}
\begin{sphinxVerbatim}[commandchars=\\\{\}]
\PYG{n+nb}{bool}\PYG{p}{(}\PYG{o}{\PYGZhy{}}\PYG{l+m+mi}{1}\PYG{p}{)}
\end{sphinxVerbatim}

\end{sphinxuseclass}\end{sphinxVerbatimInput}
\begin{sphinxVerbatimOutput}

\begin{sphinxuseclass}{cell_output}
\begin{sphinxVerbatim}[commandchars=\\\{\}]
True
\end{sphinxVerbatim}

\end{sphinxuseclass}\end{sphinxVerbatimOutput}

\end{sphinxuseclass}
\sphinxAtStartPar
We can recast the string \sphinxcode{\sphinxupquote{'5'}} to an integer or the integer \sphinxcode{\sphinxupquote{5}} to a string to prevent the \sphinxcode{\sphinxupquote{5 + '5'}} error above.

\begin{sphinxuseclass}{cell}\begin{sphinxVerbatimInput}

\begin{sphinxuseclass}{cell_input}
\begin{sphinxVerbatim}[commandchars=\\\{\}]
\PYG{l+m+mi}{5} \PYG{o}{+} \PYG{n+nb}{int}\PYG{p}{(}\PYG{l+s+s1}{\PYGZsq{}}\PYG{l+s+s1}{5}\PYG{l+s+s1}{\PYGZsq{}}\PYG{p}{)}
\end{sphinxVerbatim}

\end{sphinxuseclass}\end{sphinxVerbatimInput}
\begin{sphinxVerbatimOutput}

\begin{sphinxuseclass}{cell_output}
\begin{sphinxVerbatim}[commandchars=\\\{\}]
10
\end{sphinxVerbatim}

\end{sphinxuseclass}\end{sphinxVerbatimOutput}

\end{sphinxuseclass}
\begin{sphinxuseclass}{cell}\begin{sphinxVerbatimInput}

\begin{sphinxuseclass}{cell_input}
\begin{sphinxVerbatim}[commandchars=\\\{\}]
\PYG{n+nb}{str}\PYG{p}{(}\PYG{l+m+mi}{5}\PYG{p}{)} \PYG{o}{+} \PYG{l+s+s1}{\PYGZsq{}}\PYG{l+s+s1}{5}\PYG{l+s+s1}{\PYGZsq{}}
\end{sphinxVerbatim}

\end{sphinxuseclass}\end{sphinxVerbatimInput}
\begin{sphinxVerbatimOutput}

\begin{sphinxuseclass}{cell_output}
\begin{sphinxVerbatim}[commandchars=\\\{\}]
\PYGZsq{}55\PYGZsq{}
\end{sphinxVerbatim}

\end{sphinxuseclass}\end{sphinxVerbatimOutput}

\end{sphinxuseclass}

\subsection{None}
\label{\detokenize{mckinney_02_lecture:none}}
\sphinxAtStartPar
In Python, \sphinxcode{\sphinxupquote{None}} is null.
\sphinxcode{\sphinxupquote{None}} is like \sphinxcode{\sphinxupquote{\#N/A}} or \sphinxcode{\sphinxupquote{=na()}} in Excel.

\begin{sphinxuseclass}{cell}\begin{sphinxVerbatimInput}

\begin{sphinxuseclass}{cell_input}
\begin{sphinxVerbatim}[commandchars=\\\{\}]
\PYG{n}{a} \PYG{o}{=} \PYG{k+kc}{None}
\PYG{n}{a} \PYG{o+ow}{is} \PYG{k+kc}{None}
\end{sphinxVerbatim}

\end{sphinxuseclass}\end{sphinxVerbatimInput}
\begin{sphinxVerbatimOutput}

\begin{sphinxuseclass}{cell_output}
\begin{sphinxVerbatim}[commandchars=\\\{\}]
True
\end{sphinxVerbatim}

\end{sphinxuseclass}\end{sphinxVerbatimOutput}

\end{sphinxuseclass}
\begin{sphinxuseclass}{cell}\begin{sphinxVerbatimInput}

\begin{sphinxuseclass}{cell_input}
\begin{sphinxVerbatim}[commandchars=\\\{\}]
\PYG{n}{b} \PYG{o}{=} \PYG{l+m+mi}{5}
\PYG{n}{b} \PYG{o+ow}{is} \PYG{o+ow}{not} \PYG{k+kc}{None}
\end{sphinxVerbatim}

\end{sphinxuseclass}\end{sphinxVerbatimInput}
\begin{sphinxVerbatimOutput}

\begin{sphinxuseclass}{cell_output}
\begin{sphinxVerbatim}[commandchars=\\\{\}]
True
\end{sphinxVerbatim}

\end{sphinxuseclass}\end{sphinxVerbatimOutput}

\end{sphinxuseclass}
\begin{sphinxuseclass}{cell}\begin{sphinxVerbatimInput}

\begin{sphinxuseclass}{cell_input}
\begin{sphinxVerbatim}[commandchars=\\\{\}]
\PYG{n+nb}{type}\PYG{p}{(}\PYG{k+kc}{None}\PYG{p}{)}
\end{sphinxVerbatim}

\end{sphinxuseclass}\end{sphinxVerbatimInput}
\begin{sphinxVerbatimOutput}

\begin{sphinxuseclass}{cell_output}
\begin{sphinxVerbatim}[commandchars=\\\{\}]
NoneType
\end{sphinxVerbatim}

\end{sphinxuseclass}\end{sphinxVerbatimOutput}

\end{sphinxuseclass}

\section{Control Flow}
\label{\detokenize{mckinney_02_lecture:control-flow}}\begin{quote}

\sphinxAtStartPar
Python has several built\sphinxhyphen{}in keywords for conditional logic, loops, and other standard control flow concepts found in other programming languages.
\end{quote}

\sphinxAtStartPar
If you understand Excel’s \sphinxcode{\sphinxupquote{if()}}, then you understand Python’s \sphinxcode{\sphinxupquote{if}}, \sphinxcode{\sphinxupquote{elif}}, and \sphinxcode{\sphinxupquote{else}}.


\subsection{if, elif, and else}
\label{\detokenize{mckinney_02_lecture:if-elif-and-else}}
\begin{sphinxuseclass}{cell}\begin{sphinxVerbatimInput}

\begin{sphinxuseclass}{cell_input}
\begin{sphinxVerbatim}[commandchars=\\\{\}]
\PYG{n}{x} \PYG{o}{=} \PYG{o}{\PYGZhy{}}\PYG{l+m+mi}{1}
\end{sphinxVerbatim}

\end{sphinxuseclass}\end{sphinxVerbatimInput}

\end{sphinxuseclass}
\begin{sphinxuseclass}{cell}\begin{sphinxVerbatimInput}

\begin{sphinxuseclass}{cell_input}
\begin{sphinxVerbatim}[commandchars=\\\{\}]
\PYG{n+nb}{type}\PYG{p}{(}\PYG{n}{x}\PYG{p}{)}
\end{sphinxVerbatim}

\end{sphinxuseclass}\end{sphinxVerbatimInput}
\begin{sphinxVerbatimOutput}

\begin{sphinxuseclass}{cell_output}
\begin{sphinxVerbatim}[commandchars=\\\{\}]
int
\end{sphinxVerbatim}

\end{sphinxuseclass}\end{sphinxVerbatimOutput}

\end{sphinxuseclass}
\begin{sphinxuseclass}{cell}\begin{sphinxVerbatimInput}

\begin{sphinxuseclass}{cell_input}
\begin{sphinxVerbatim}[commandchars=\\\{\}]
\PYG{k}{if} \PYG{n}{x} \PYG{o}{\PYGZlt{}} \PYG{l+m+mi}{0}\PYG{p}{:}
    \PYG{n+nb}{print}\PYG{p}{(}\PYG{l+s+s2}{\PYGZdq{}}\PYG{l+s+s2}{It}\PYG{l+s+s2}{\PYGZsq{}}\PYG{l+s+s2}{s negative}\PYG{l+s+s2}{\PYGZdq{}}\PYG{p}{)}
\end{sphinxVerbatim}

\end{sphinxuseclass}\end{sphinxVerbatimInput}
\begin{sphinxVerbatimOutput}

\begin{sphinxuseclass}{cell_output}
\begin{sphinxVerbatim}[commandchars=\\\{\}]
It\PYGZsq{}s negative
\end{sphinxVerbatim}

\end{sphinxuseclass}\end{sphinxVerbatimOutput}

\end{sphinxuseclass}
\sphinxAtStartPar
Single quotes and double quotes (\sphinxcode{\sphinxupquote{'}} and \sphinxcode{\sphinxupquote{"}}) are equivalent in Python.
However, in the previous code cell, we use double quotes to differentiate between the enclosing quotes and the apostrophe in \sphinxcode{\sphinxupquote{"It's"}}.

\sphinxAtStartPar
Python’s \sphinxcode{\sphinxupquote{elif}} avoids Excel’s nested \sphinxcode{\sphinxupquote{if()}}s.
\sphinxcode{\sphinxupquote{elif}} continues an \sphinxcode{\sphinxupquote{if}} block, and \sphinxcode{\sphinxupquote{else}} runs if the other conditions are not met.

\begin{sphinxuseclass}{cell}\begin{sphinxVerbatimInput}

\begin{sphinxuseclass}{cell_input}
\begin{sphinxVerbatim}[commandchars=\\\{\}]
\PYG{n}{x} \PYG{o}{=} \PYG{l+m+mi}{10}
\PYG{k}{if} \PYG{n}{x} \PYG{o}{\PYGZlt{}} \PYG{l+m+mi}{0}\PYG{p}{:}
    \PYG{n+nb}{print}\PYG{p}{(}\PYG{l+s+s2}{\PYGZdq{}}\PYG{l+s+s2}{It}\PYG{l+s+s2}{\PYGZsq{}}\PYG{l+s+s2}{s negative}\PYG{l+s+s2}{\PYGZdq{}}\PYG{p}{)}
\PYG{k}{elif} \PYG{n}{x} \PYG{o}{==} \PYG{l+m+mi}{0}\PYG{p}{:}
    \PYG{n+nb}{print}\PYG{p}{(}\PYG{l+s+s1}{\PYGZsq{}}\PYG{l+s+s1}{Equal to zero}\PYG{l+s+s1}{\PYGZsq{}}\PYG{p}{)}
\PYG{k}{elif} \PYG{l+m+mi}{0} \PYG{o}{\PYGZlt{}} \PYG{n}{x} \PYG{o}{\PYGZlt{}} \PYG{l+m+mi}{5}\PYG{p}{:}
    \PYG{n+nb}{print}\PYG{p}{(}\PYG{l+s+s1}{\PYGZsq{}}\PYG{l+s+s1}{Positive but smaller than 5}\PYG{l+s+s1}{\PYGZsq{}}\PYG{p}{)}
\PYG{k}{else}\PYG{p}{:}
    \PYG{n+nb}{print}\PYG{p}{(}\PYG{l+s+s1}{\PYGZsq{}}\PYG{l+s+s1}{Positive and larger than or equal to 5}\PYG{l+s+s1}{\PYGZsq{}}\PYG{p}{)}
\end{sphinxVerbatim}

\end{sphinxuseclass}\end{sphinxVerbatimInput}
\begin{sphinxVerbatimOutput}

\begin{sphinxuseclass}{cell_output}
\begin{sphinxVerbatim}[commandchars=\\\{\}]
Positive and larger than or equal to 5
\end{sphinxVerbatim}

\end{sphinxuseclass}\end{sphinxVerbatimOutput}

\end{sphinxuseclass}
\sphinxAtStartPar
We can combine comparisons with \sphinxcode{\sphinxupquote{and}} and \sphinxcode{\sphinxupquote{or}}.

\begin{sphinxuseclass}{cell}\begin{sphinxVerbatimInput}

\begin{sphinxuseclass}{cell_input}
\begin{sphinxVerbatim}[commandchars=\\\{\}]
\PYG{n}{a} \PYG{o}{=} \PYG{l+m+mi}{5}
\PYG{n}{b} \PYG{o}{=} \PYG{l+m+mi}{7}
\PYG{n}{c} \PYG{o}{=} \PYG{l+m+mi}{8}
\PYG{n}{d} \PYG{o}{=} \PYG{l+m+mi}{4}
\PYG{k}{if} \PYG{n}{a} \PYG{o}{\PYGZlt{}} \PYG{n}{b} \PYG{o+ow}{or} \PYG{n}{c} \PYG{o}{\PYGZgt{}} \PYG{n}{d}\PYG{p}{:}
    \PYG{n+nb}{print}\PYG{p}{(}\PYG{l+s+s1}{\PYGZsq{}}\PYG{l+s+s1}{Made it}\PYG{l+s+s1}{\PYGZsq{}}\PYG{p}{)}
\end{sphinxVerbatim}

\end{sphinxuseclass}\end{sphinxVerbatimInput}
\begin{sphinxVerbatimOutput}

\begin{sphinxuseclass}{cell_output}
\begin{sphinxVerbatim}[commandchars=\\\{\}]
Made it
\end{sphinxVerbatim}

\end{sphinxuseclass}\end{sphinxVerbatimOutput}

\end{sphinxuseclass}

\subsection{for loops}
\label{\detokenize{mckinney_02_lecture:for-loops}}
\sphinxAtStartPar
We use \sphinxcode{\sphinxupquote{for}} loops to loop over a collection, like a list or tuple.
The \sphinxcode{\sphinxupquote{continue}} keyword skips the remainder of the \sphinxcode{\sphinxupquote{if}} block for that loop iteration.

\sphinxAtStartPar
The following example assigns values with \sphinxcode{\sphinxupquote{+=}}, where \sphinxcode{\sphinxupquote{a += 5}} is an abbreviation for \sphinxcode{\sphinxupquote{a = a + 5}}.
There are equivalent abbreviations for subtraction, multiplication, and division (\sphinxcode{\sphinxupquote{\sphinxhyphen{}=}}, \sphinxcode{\sphinxupquote{*=}}, and \sphinxcode{\sphinxupquote{/=}}).

\begin{sphinxuseclass}{cell}\begin{sphinxVerbatimInput}

\begin{sphinxuseclass}{cell_input}
\begin{sphinxVerbatim}[commandchars=\\\{\}]
\PYG{n}{sequence} \PYG{o}{=} \PYG{p}{[}\PYG{l+m+mi}{1}\PYG{p}{,} \PYG{l+m+mi}{2}\PYG{p}{,} \PYG{k+kc}{None}\PYG{p}{,} \PYG{l+m+mi}{4}\PYG{p}{,} \PYG{k+kc}{None}\PYG{p}{,} \PYG{l+m+mi}{5}\PYG{p}{,} \PYG{l+s+s1}{\PYGZsq{}}\PYG{l+s+s1}{Alex}\PYG{l+s+s1}{\PYGZsq{}}\PYG{p}{]}
\PYG{n}{total} \PYG{o}{=} \PYG{l+m+mi}{0}
\PYG{k}{for} \PYG{n}{value} \PYG{o+ow}{in} \PYG{n}{sequence}\PYG{p}{:}
    \PYG{k}{if} \PYG{n}{value} \PYG{o+ow}{is} \PYG{k+kc}{None} \PYG{o+ow}{or} \PYG{n+nb}{type}\PYG{p}{(}\PYG{n}{value}\PYG{p}{)} \PYG{o+ow}{is} \PYG{n+nb}{str}\PYG{p}{:}
        \PYG{k}{continue}
    \PYG{n}{total} \PYG{o}{+}\PYG{o}{=} \PYG{n}{value} \PYG{c+c1}{\PYGZsh{} the += operator is equivalent to \PYGZdq{}total = total + value\PYGZdq{}}
\end{sphinxVerbatim}

\end{sphinxuseclass}\end{sphinxVerbatimInput}

\end{sphinxuseclass}
\begin{sphinxuseclass}{cell}\begin{sphinxVerbatimInput}

\begin{sphinxuseclass}{cell_input}
\begin{sphinxVerbatim}[commandchars=\\\{\}]
\PYG{n}{total}
\end{sphinxVerbatim}

\end{sphinxuseclass}\end{sphinxVerbatimInput}
\begin{sphinxVerbatimOutput}

\begin{sphinxuseclass}{cell_output}
\begin{sphinxVerbatim}[commandchars=\\\{\}]
12
\end{sphinxVerbatim}

\end{sphinxuseclass}\end{sphinxVerbatimOutput}

\end{sphinxuseclass}
\sphinxAtStartPar
The \sphinxcode{\sphinxupquote{break}} keyword exits the loop altogether.

\begin{sphinxuseclass}{cell}\begin{sphinxVerbatimInput}

\begin{sphinxuseclass}{cell_input}
\begin{sphinxVerbatim}[commandchars=\\\{\}]
\PYG{n}{sequence} \PYG{o}{=} \PYG{p}{[}\PYG{l+m+mi}{1}\PYG{p}{,} \PYG{l+m+mi}{2}\PYG{p}{,} \PYG{l+m+mi}{0}\PYG{p}{,} \PYG{l+m+mi}{4}\PYG{p}{,} \PYG{l+m+mi}{6}\PYG{p}{,} \PYG{l+m+mi}{5}\PYG{p}{,} \PYG{l+m+mi}{2}\PYG{p}{,} \PYG{l+m+mi}{1}\PYG{p}{]}
\PYG{n}{total\PYGZus{}until\PYGZus{}5} \PYG{o}{=} \PYG{l+m+mi}{0}
\PYG{k}{for} \PYG{n}{value} \PYG{o+ow}{in} \PYG{n}{sequence}\PYG{p}{:}
    \PYG{k}{if} \PYG{n}{value} \PYG{o}{==} \PYG{l+m+mi}{5}\PYG{p}{:}
        \PYG{k}{break}
    \PYG{n}{total\PYGZus{}until\PYGZus{}5} \PYG{o}{+}\PYG{o}{=} \PYG{n}{value}
\end{sphinxVerbatim}

\end{sphinxuseclass}\end{sphinxVerbatimInput}

\end{sphinxuseclass}
\begin{sphinxuseclass}{cell}\begin{sphinxVerbatimInput}

\begin{sphinxuseclass}{cell_input}
\begin{sphinxVerbatim}[commandchars=\\\{\}]
\PYG{n}{total\PYGZus{}until\PYGZus{}5}
\end{sphinxVerbatim}

\end{sphinxuseclass}\end{sphinxVerbatimInput}
\begin{sphinxVerbatimOutput}

\begin{sphinxuseclass}{cell_output}
\begin{sphinxVerbatim}[commandchars=\\\{\}]
13
\end{sphinxVerbatim}

\end{sphinxuseclass}\end{sphinxVerbatimOutput}

\end{sphinxuseclass}

\subsection{range}
\label{\detokenize{mckinney_02_lecture:range}}\begin{quote}

\sphinxAtStartPar
The range function returns an iterator that yields a sequence of evenly spaced
integers.
\end{quote}
\begin{itemize}
\item {} 
\sphinxAtStartPar
With one argument, \sphinxcode{\sphinxupquote{range()}} creates an iterator from 0 to that number \sphinxstyleemphasis{but excludes that number} (so \sphinxcode{\sphinxupquote{range(10)}} is an interator with a length of 10 that starts at 0)

\item {} 
\sphinxAtStartPar
With two arguments, the first argument is the \sphinxstyleemphasis{inclusive} starting value, and the second argument is the \sphinxstyleemphasis{exclusive} ending value

\item {} 
\sphinxAtStartPar
With three arguments, the third argument is the iterator step size

\end{itemize}

\begin{sphinxuseclass}{cell}\begin{sphinxVerbatimInput}

\begin{sphinxuseclass}{cell_input}
\begin{sphinxVerbatim}[commandchars=\\\{\}]
\PYG{n+nb}{range}\PYG{p}{(}\PYG{l+m+mi}{10}\PYG{p}{)}
\end{sphinxVerbatim}

\end{sphinxuseclass}\end{sphinxVerbatimInput}
\begin{sphinxVerbatimOutput}

\begin{sphinxuseclass}{cell_output}
\begin{sphinxVerbatim}[commandchars=\\\{\}]
range(0, 10)
\end{sphinxVerbatim}

\end{sphinxuseclass}\end{sphinxVerbatimOutput}

\end{sphinxuseclass}
\sphinxAtStartPar
We can cast a range to a list.

\begin{sphinxuseclass}{cell}\begin{sphinxVerbatimInput}

\begin{sphinxuseclass}{cell_input}
\begin{sphinxVerbatim}[commandchars=\\\{\}]
\PYG{n+nb}{list}\PYG{p}{(}\PYG{n+nb}{range}\PYG{p}{(}\PYG{l+m+mi}{10}\PYG{p}{)}\PYG{p}{)}
\end{sphinxVerbatim}

\end{sphinxuseclass}\end{sphinxVerbatimInput}
\begin{sphinxVerbatimOutput}

\begin{sphinxuseclass}{cell_output}
\begin{sphinxVerbatim}[commandchars=\\\{\}]
[0, 1, 2, 3, 4, 5, 6, 7, 8, 9]
\end{sphinxVerbatim}

\end{sphinxuseclass}\end{sphinxVerbatimOutput}

\end{sphinxuseclass}
\begin{sphinxuseclass}{cell}\begin{sphinxVerbatimInput}

\begin{sphinxuseclass}{cell_input}
\begin{sphinxVerbatim}[commandchars=\\\{\}]
\PYG{n+nb}{list}\PYG{p}{(}\PYG{n+nb}{range}\PYG{p}{(}\PYG{l+m+mi}{1}\PYG{p}{,} \PYG{l+m+mi}{10}\PYG{p}{)}\PYG{p}{)}
\end{sphinxVerbatim}

\end{sphinxuseclass}\end{sphinxVerbatimInput}
\begin{sphinxVerbatimOutput}

\begin{sphinxuseclass}{cell_output}
\begin{sphinxVerbatim}[commandchars=\\\{\}]
[1, 2, 3, 4, 5, 6, 7, 8, 9]
\end{sphinxVerbatim}

\end{sphinxuseclass}\end{sphinxVerbatimOutput}

\end{sphinxuseclass}
\begin{sphinxuseclass}{cell}\begin{sphinxVerbatimInput}

\begin{sphinxuseclass}{cell_input}
\begin{sphinxVerbatim}[commandchars=\\\{\}]
\PYG{n+nb}{list}\PYG{p}{(}\PYG{n+nb}{range}\PYG{p}{(}\PYG{l+m+mi}{1}\PYG{p}{,} \PYG{l+m+mi}{10}\PYG{p}{,} \PYG{l+m+mi}{1}\PYG{p}{)}\PYG{p}{)}
\end{sphinxVerbatim}

\end{sphinxuseclass}\end{sphinxVerbatimInput}
\begin{sphinxVerbatimOutput}

\begin{sphinxuseclass}{cell_output}
\begin{sphinxVerbatim}[commandchars=\\\{\}]
[1, 2, 3, 4, 5, 6, 7, 8, 9]
\end{sphinxVerbatim}

\end{sphinxuseclass}\end{sphinxVerbatimOutput}

\end{sphinxuseclass}
\begin{sphinxuseclass}{cell}\begin{sphinxVerbatimInput}

\begin{sphinxuseclass}{cell_input}
\begin{sphinxVerbatim}[commandchars=\\\{\}]
\PYG{n+nb}{list}\PYG{p}{(}\PYG{n+nb}{range}\PYG{p}{(}\PYG{l+m+mi}{0}\PYG{p}{,} \PYG{l+m+mi}{20}\PYG{p}{,} \PYG{l+m+mi}{2}\PYG{p}{)}\PYG{p}{)}
\end{sphinxVerbatim}

\end{sphinxuseclass}\end{sphinxVerbatimInput}
\begin{sphinxVerbatimOutput}

\begin{sphinxuseclass}{cell_output}
\begin{sphinxVerbatim}[commandchars=\\\{\}]
[0, 2, 4, 6, 8, 10, 12, 14, 16, 18]
\end{sphinxVerbatim}

\end{sphinxuseclass}\end{sphinxVerbatimOutput}

\end{sphinxuseclass}
\sphinxAtStartPar
Python intervals are “closed” (inclusive) on the left and “open” (exclusive) on the right.
The following is an empty list because we cannot count from 5 to 0 by steps of +1.

\begin{sphinxuseclass}{cell}\begin{sphinxVerbatimInput}

\begin{sphinxuseclass}{cell_input}
\begin{sphinxVerbatim}[commandchars=\\\{\}]
\PYG{n+nb}{list}\PYG{p}{(}\PYG{n+nb}{range}\PYG{p}{(}\PYG{l+m+mi}{5}\PYG{p}{,} \PYG{l+m+mi}{0}\PYG{p}{)}\PYG{p}{)}
\end{sphinxVerbatim}

\end{sphinxuseclass}\end{sphinxVerbatimInput}
\begin{sphinxVerbatimOutput}

\begin{sphinxuseclass}{cell_output}
\begin{sphinxVerbatim}[commandchars=\\\{\}]
[]
\end{sphinxVerbatim}

\end{sphinxuseclass}\end{sphinxVerbatimOutput}

\end{sphinxuseclass}
\sphinxAtStartPar
However, we can count from 5 to 0 in steps of \sphinxhyphen{}1.

\begin{sphinxuseclass}{cell}\begin{sphinxVerbatimInput}

\begin{sphinxuseclass}{cell_input}
\begin{sphinxVerbatim}[commandchars=\\\{\}]
\PYG{n+nb}{list}\PYG{p}{(}\PYG{n+nb}{range}\PYG{p}{(}\PYG{l+m+mi}{5}\PYG{p}{,} \PYG{l+m+mi}{0}\PYG{p}{,} \PYG{o}{\PYGZhy{}}\PYG{l+m+mi}{1}\PYG{p}{)}\PYG{p}{)}
\end{sphinxVerbatim}

\end{sphinxuseclass}\end{sphinxVerbatimInput}
\begin{sphinxVerbatimOutput}

\begin{sphinxuseclass}{cell_output}
\begin{sphinxVerbatim}[commandchars=\\\{\}]
[5, 4, 3, 2, 1]
\end{sphinxVerbatim}

\end{sphinxuseclass}\end{sphinxVerbatimOutput}

\end{sphinxuseclass}
\sphinxAtStartPar
For loops have the following syntax in many other programming languages.

\begin{sphinxuseclass}{cell}\begin{sphinxVerbatimInput}

\begin{sphinxuseclass}{cell_input}
\begin{sphinxVerbatim}[commandchars=\\\{\}]
\PYG{n}{seq} \PYG{o}{=} \PYG{p}{[}\PYG{l+m+mi}{1}\PYG{p}{,} \PYG{l+m+mi}{2}\PYG{p}{,} \PYG{l+m+mi}{3}\PYG{p}{,} \PYG{l+m+mi}{4}\PYG{p}{]}
\PYG{k}{for} \PYG{n}{i} \PYG{o+ow}{in} \PYG{n+nb}{range}\PYG{p}{(}\PYG{n+nb}{len}\PYG{p}{(}\PYG{n}{seq}\PYG{p}{)}\PYG{p}{)}\PYG{p}{:}
    \PYG{n}{val} \PYG{o}{=} \PYG{n}{seq}\PYG{p}{[}\PYG{n}{i}\PYG{p}{]}
\end{sphinxVerbatim}

\end{sphinxuseclass}\end{sphinxVerbatimInput}

\end{sphinxuseclass}
\sphinxAtStartPar
However, in Python, we can directly loop over the list \sphinxcode{\sphinxupquote{seq}}.
The following code cell is equivalent to the previous code cell and more “Pythonic”.

\begin{sphinxuseclass}{cell}\begin{sphinxVerbatimInput}

\begin{sphinxuseclass}{cell_input}
\begin{sphinxVerbatim}[commandchars=\\\{\}]
\PYG{k}{for} \PYG{n}{i} \PYG{o+ow}{in} \PYG{n}{seq}\PYG{p}{:}
    \PYG{n}{val} \PYG{o}{=} \PYG{n}{i}
\end{sphinxVerbatim}

\end{sphinxuseclass}\end{sphinxVerbatimInput}

\end{sphinxuseclass}

\subsection{Ternary expressions}
\label{\detokenize{mckinney_02_lecture:ternary-expressions}}
\sphinxAtStartPar
We said above that Python \sphinxcode{\sphinxupquote{if}} and \sphinxcode{\sphinxupquote{else}} is cumbersome relative to Excel’s \sphinxcode{\sphinxupquote{if()}}.
We can complete simple comparisons on one line in Python.

\begin{sphinxuseclass}{cell}\begin{sphinxVerbatimInput}

\begin{sphinxuseclass}{cell_input}
\begin{sphinxVerbatim}[commandchars=\\\{\}]
\PYG{n}{x} \PYG{o}{=} \PYG{l+m+mi}{5}
\PYG{n}{value} \PYG{o}{=} \PYG{l+s+s1}{\PYGZsq{}}\PYG{l+s+s1}{Non\PYGZhy{}negative}\PYG{l+s+s1}{\PYGZsq{}} \PYG{k}{if} \PYG{n}{x} \PYG{o}{\PYGZgt{}}\PYG{o}{=} \PYG{l+m+mi}{0} \PYG{k}{else} \PYG{l+s+s1}{\PYGZsq{}}\PYG{l+s+s1}{Negative}\PYG{l+s+s1}{\PYGZsq{}}
\end{sphinxVerbatim}

\end{sphinxuseclass}\end{sphinxVerbatimInput}

\end{sphinxuseclass}
\sphinxstepscope


\section{McKinney Chapter 2 \sphinxhyphen{} Practice (Blank)}
\label{\detokenize{mckinney_02_practice:mckinney-chapter-2-practice-blank}}\label{\detokenize{mckinney_02_practice::doc}}

\subsection{Practice}
\label{\detokenize{mckinney_02_practice:practice}}

\subsubsection{Extract the year, month, and day from an integer 8\sphinxhyphen{}digit date (i.e., YYYYMMDD format) using \sphinxstyleliteralintitle{\sphinxupquote{//}} (integer division) and \sphinxstyleliteralintitle{\sphinxupquote{\%}} (modulo division).}
\label{\detokenize{mckinney_02_practice:extract-the-year-month-and-day-from-an-integer-8-digit-date-i-e-yyyymmdd-format-using-integer-division-and-modulo-division}}
\sphinxAtStartPar
Try \sphinxcode{\sphinxupquote{20080915}}.


\subsubsection{Use your answer above to write a function \sphinxstyleliteralintitle{\sphinxupquote{date}} that accepts an integer 8\sphinxhyphen{}digit date argument and returns a tuple of the year, month, and date (e.g., \sphinxstyleliteralintitle{\sphinxupquote{return (year, month, date)}}).}
\label{\detokenize{mckinney_02_practice:use-your-answer-above-to-write-a-function-date-that-accepts-an-integer-8-digit-date-argument-and-returns-a-tuple-of-the-year-month-and-date-e-g-return-year-month-date}}

\subsubsection{Rewrite \sphinxstyleliteralintitle{\sphinxupquote{date}} to accept an 8\sphinxhyphen{}digit date as an integer or string.}
\label{\detokenize{mckinney_02_practice:rewrite-date-to-accept-an-8-digit-date-as-an-integer-or-string}}

\subsubsection{Finally, rewrite \sphinxstyleliteralintitle{\sphinxupquote{date}} to accept a list of 8\sphinxhyphen{}digit dates as integers or strings.}
\label{\detokenize{mckinney_02_practice:finally-rewrite-date-to-accept-a-list-of-8-digit-dates-as-integers-or-strings}}
\sphinxAtStartPar
Return a list of tuples of year, month, and date.


\subsubsection{Write a for loop that prints the squares of integers from 1 to 10.}
\label{\detokenize{mckinney_02_practice:write-a-for-loop-that-prints-the-squares-of-integers-from-1-to-10}}

\subsubsection{Write a for loop that prints the squares of \sphinxstyleemphasis{even} integers from 1 to 10.}
\label{\detokenize{mckinney_02_practice:write-a-for-loop-that-prints-the-squares-of-even-integers-from-1-to-10}}

\subsubsection{Write a for loop that sums the squares of integers from 1 to 10.}
\label{\detokenize{mckinney_02_practice:write-a-for-loop-that-sums-the-squares-of-integers-from-1-to-10}}

\subsubsection{Write a for loop that sums the squares of integers from 1 to 10 but stops before the sum exceeds 50.}
\label{\detokenize{mckinney_02_practice:write-a-for-loop-that-sums-the-squares-of-integers-from-1-to-10-but-stops-before-the-sum-exceeds-50}}

\subsubsection{FizzBuzz}
\label{\detokenize{mckinney_02_practice:fizzbuzz}}
\sphinxAtStartPar
Write a for loop that prints the numbers from 1 to 100.
For multiples of three print “Fizz” instead of the number.
For multiples of five print “Buzz”.
For numbers that are multiples of both three and five print “FizzBuzz”.
More \sphinxhref{https://blog.codinghorror.com/why-cant-programmers-program/}{here}.


\subsubsection{Use ternary expressions to make your FizzBuzz code more compact.}
\label{\detokenize{mckinney_02_practice:use-ternary-expressions-to-make-your-fizzbuzz-code-more-compact}}

\subsubsection{Triangle}
\label{\detokenize{mckinney_02_practice:triangle}}
\sphinxAtStartPar
Write a function \sphinxcode{\sphinxupquote{triangle}} that accepts a positive integer \$N\$ and prints a numerical triangle of height \$N\sphinxhyphen{}1\$.
For example, \sphinxcode{\sphinxupquote{triangle(N=6)}} should print:

\begin{sphinxVerbatim}[commandchars=\\\{\}]
\PYG{l+m+mi}{1}
\PYG{l+m+mi}{22}
\PYG{l+m+mi}{333}
\PYG{l+m+mi}{4444}
\PYG{l+m+mi}{55555}
\end{sphinxVerbatim}


\subsubsection{Two Sum}
\label{\detokenize{mckinney_02_practice:two-sum}}
\sphinxAtStartPar
Write a function \sphinxcode{\sphinxupquote{two\_sum}} that does the following.

\sphinxAtStartPar
Given an array of integers \sphinxcode{\sphinxupquote{nums}} and an integer \sphinxcode{\sphinxupquote{target}}, return the indices of the two numbers that add up to target.

\sphinxAtStartPar
You may assume that each input would have exactly one solution, and you may not use the same element twice.

\sphinxAtStartPar
You can return the answer in any order.

\sphinxAtStartPar
Here are some examples:

\sphinxAtStartPar
Example 1:

\sphinxAtStartPar
Input: \sphinxcode{\sphinxupquote{nums = {[}2,7,11,15{]}}}, \sphinxcode{\sphinxupquote{target = 9}} \\
Output: \sphinxcode{\sphinxupquote{{[}0,1{]}}} \\
Explanation: Because \sphinxcode{\sphinxupquote{nums{[}0{]} + nums{[}1{]} == 9}}, we return \sphinxcode{\sphinxupquote{{[}0, 1{]}}}.

\sphinxAtStartPar
Example 2:

\sphinxAtStartPar
Input: \sphinxcode{\sphinxupquote{nums = {[}3,2,4{]}}}, \sphinxcode{\sphinxupquote{target = 6}} \\
Output: \sphinxcode{\sphinxupquote{{[}1,2{]}}} \textbackslash{}

\sphinxAtStartPar
Example 3:

\sphinxAtStartPar
Input: \sphinxcode{\sphinxupquote{nums = {[}3,3{]}}}, \sphinxcode{\sphinxupquote{target = 6}} \\
Output: \sphinxcode{\sphinxupquote{{[}0,1{]}}} \textbackslash{}

\sphinxAtStartPar
I saw this question on \sphinxhref{https://leetcode.com/problems/two-sum/description/}{LeetCode}.


\subsubsection{Best Time}
\label{\detokenize{mckinney_02_practice:best-time}}
\sphinxAtStartPar
Write a function \sphinxcode{\sphinxupquote{best\_time}} that solves the following.

\sphinxAtStartPar
You are given an array \sphinxcode{\sphinxupquote{prices}} where \sphinxcode{\sphinxupquote{prices{[}i{]}}} is the price of a given stock on the \$i\textasciicircum{}\{th\}\$ day.

\sphinxAtStartPar
You want to maximize your profit by choosing a single day to buy one stock and choosing a different day in the future to sell that stock.

\sphinxAtStartPar
Return the maximum profit you can achieve from this transaction. If you cannot achieve any profit, return 0.

\sphinxAtStartPar
Here are some examples:

\sphinxAtStartPar
Example 1:

\sphinxAtStartPar
Input: \sphinxcode{\sphinxupquote{prices = {[}7,1,5,3,6,4{]}}} \\
Output: \sphinxcode{\sphinxupquote{5}} \\
Explanation: Buy on day 2 (price = 1) and sell on day 5 (price = 6), profit = 6\sphinxhyphen{}1 = 5.
Note that buying on day 2 and selling on day 1 is not allowed because you must buy before you sell.

\sphinxAtStartPar
Example 2:

\sphinxAtStartPar
Input: \sphinxcode{\sphinxupquote{prices = {[}7,6,4,3,1{]}}} \\
Output: \sphinxcode{\sphinxupquote{0}} \\
Explanation: In this case, no transactions are done and the max profit = 0.

\sphinxAtStartPar
I saw this question on \sphinxhref{https://leetcode.com/problems/best-time-to-buy-and-sell-stock/}{LeetCode}.

\sphinxstepscope


\section{McKinney Chapter 2 \sphinxhyphen{} Practice (Section 3, Monday 2:45 PM)}
\label{\detokenize{mckinney_02_practice_03:mckinney-chapter-2-practice-section-3-monday-2-45-pm}}\label{\detokenize{mckinney_02_practice_03::doc}}

\subsection{Practice}
\label{\detokenize{mckinney_02_practice_03:practice}}

\subsubsection{Extract the year, month, and day from an integer 8\sphinxhyphen{}digit date (i.e., YYYYMMDD format) using \sphinxstyleliteralintitle{\sphinxupquote{//}} (integer division) and \sphinxstyleliteralintitle{\sphinxupquote{\%}} (modulo division).}
\label{\detokenize{mckinney_02_practice_03:extract-the-year-month-and-day-from-an-integer-8-digit-date-i-e-yyyymmdd-format-using-integer-division-and-modulo-division}}
\sphinxAtStartPar
Try \sphinxcode{\sphinxupquote{20080915}}.

\begin{sphinxuseclass}{cell}\begin{sphinxVerbatimInput}

\begin{sphinxuseclass}{cell_input}
\begin{sphinxVerbatim}[commandchars=\\\{\}]
\PYG{n}{ymd} \PYG{o}{=} \PYG{l+m+mi}{20080915}
\end{sphinxVerbatim}

\end{sphinxuseclass}\end{sphinxVerbatimInput}

\end{sphinxuseclass}
\begin{sphinxuseclass}{cell}\begin{sphinxVerbatimInput}

\begin{sphinxuseclass}{cell_input}
\begin{sphinxVerbatim}[commandchars=\\\{\}]
\PYG{n}{ymd} \PYG{o}{/}\PYG{o}{/} \PYG{l+m+mi}{10000}
\end{sphinxVerbatim}

\end{sphinxuseclass}\end{sphinxVerbatimInput}
\begin{sphinxVerbatimOutput}

\begin{sphinxuseclass}{cell_output}
\begin{sphinxVerbatim}[commandchars=\\\{\}]
2008
\end{sphinxVerbatim}

\end{sphinxuseclass}\end{sphinxVerbatimOutput}

\end{sphinxuseclass}
\begin{sphinxuseclass}{cell}\begin{sphinxVerbatimInput}

\begin{sphinxuseclass}{cell_input}
\begin{sphinxVerbatim}[commandchars=\\\{\}]
\PYG{n}{ymd} \PYG{o}{\PYGZpc{}} \PYG{l+m+mi}{10000}
\end{sphinxVerbatim}

\end{sphinxuseclass}\end{sphinxVerbatimInput}
\begin{sphinxVerbatimOutput}

\begin{sphinxuseclass}{cell_output}
\begin{sphinxVerbatim}[commandchars=\\\{\}]
915
\end{sphinxVerbatim}

\end{sphinxuseclass}\end{sphinxVerbatimOutput}

\end{sphinxuseclass}
\begin{sphinxuseclass}{cell}\begin{sphinxVerbatimInput}

\begin{sphinxuseclass}{cell_input}
\begin{sphinxVerbatim}[commandchars=\\\{\}]
\PYG{n}{year} \PYG{o}{=} \PYG{n}{ymd} \PYG{o}{/}\PYG{o}{/} \PYG{l+m+mi}{10000}
\end{sphinxVerbatim}

\end{sphinxuseclass}\end{sphinxVerbatimInput}

\end{sphinxuseclass}
\begin{sphinxuseclass}{cell}\begin{sphinxVerbatimInput}

\begin{sphinxuseclass}{cell_input}
\begin{sphinxVerbatim}[commandchars=\\\{\}]
\PYG{n}{month} \PYG{o}{=} \PYG{p}{(}\PYG{n}{ymd} \PYG{o}{/}\PYG{o}{/} \PYG{l+m+mi}{100}\PYG{p}{)} \PYG{o}{\PYGZpc{}} \PYG{l+m+mi}{100}
\end{sphinxVerbatim}

\end{sphinxuseclass}\end{sphinxVerbatimInput}

\end{sphinxuseclass}
\begin{sphinxuseclass}{cell}\begin{sphinxVerbatimInput}

\begin{sphinxuseclass}{cell_input}
\begin{sphinxVerbatim}[commandchars=\\\{\}]
\PYG{n}{day} \PYG{o}{=} \PYG{n}{ymd} \PYG{o}{\PYGZpc{}} \PYG{l+m+mi}{100}
\end{sphinxVerbatim}

\end{sphinxuseclass}\end{sphinxVerbatimInput}

\end{sphinxuseclass}
\begin{sphinxuseclass}{cell}\begin{sphinxVerbatimInput}

\begin{sphinxuseclass}{cell_input}
\begin{sphinxVerbatim}[commandchars=\\\{\}]
\PYG{p}{(}\PYG{n}{year}\PYG{p}{,} \PYG{n}{month}\PYG{p}{,} \PYG{n}{day}\PYG{p}{)}
\end{sphinxVerbatim}

\end{sphinxuseclass}\end{sphinxVerbatimInput}
\begin{sphinxVerbatimOutput}

\begin{sphinxuseclass}{cell_output}
\begin{sphinxVerbatim}[commandchars=\\\{\}]
(2008, 9, 15)
\end{sphinxVerbatim}

\end{sphinxuseclass}\end{sphinxVerbatimOutput}

\end{sphinxuseclass}

\subsubsection{Use your answer above to write a function \sphinxstyleliteralintitle{\sphinxupquote{date}} that accepts an integer 8\sphinxhyphen{}digit date argument and returns a tuple of the year, month, and date (e.g., \sphinxstyleliteralintitle{\sphinxupquote{return (year, month, date)}}).}
\label{\detokenize{mckinney_02_practice_03:use-your-answer-above-to-write-a-function-date-that-accepts-an-integer-8-digit-date-argument-and-returns-a-tuple-of-the-year-month-and-date-e-g-return-year-month-date}}
\begin{sphinxuseclass}{cell}\begin{sphinxVerbatimInput}

\begin{sphinxuseclass}{cell_input}
\begin{sphinxVerbatim}[commandchars=\\\{\}]
\PYG{k}{def} \PYG{n+nf}{date}\PYG{p}{(}\PYG{n}{ymd}\PYG{p}{)}\PYG{p}{:}
    \PYG{n}{year} \PYG{o}{=} \PYG{n}{ymd} \PYG{o}{/}\PYG{o}{/} \PYG{l+m+mi}{10000}
    \PYG{n}{month} \PYG{o}{=} \PYG{p}{(}\PYG{n}{ymd} \PYG{o}{/}\PYG{o}{/} \PYG{l+m+mi}{100}\PYG{p}{)} \PYG{o}{\PYGZpc{}} \PYG{l+m+mi}{100}
    \PYG{n}{day} \PYG{o}{=} \PYG{n}{ymd} \PYG{o}{\PYGZpc{}} \PYG{l+m+mi}{100}
    \PYG{k}{return} \PYG{p}{(}\PYG{n}{year}\PYG{p}{,} \PYG{n}{month}\PYG{p}{,} \PYG{n}{day}\PYG{p}{)}
\end{sphinxVerbatim}

\end{sphinxuseclass}\end{sphinxVerbatimInput}

\end{sphinxuseclass}
\begin{sphinxuseclass}{cell}\begin{sphinxVerbatimInput}

\begin{sphinxuseclass}{cell_input}
\begin{sphinxVerbatim}[commandchars=\\\{\}]
\PYG{n}{date}\PYG{p}{(}\PYG{l+m+mi}{20080915}\PYG{p}{)}
\end{sphinxVerbatim}

\end{sphinxuseclass}\end{sphinxVerbatimInput}
\begin{sphinxVerbatimOutput}

\begin{sphinxuseclass}{cell_output}
\begin{sphinxVerbatim}[commandchars=\\\{\}]
(2008, 9, 15)
\end{sphinxVerbatim}

\end{sphinxuseclass}\end{sphinxVerbatimOutput}

\end{sphinxuseclass}

\subsubsection{Rewrite \sphinxstyleliteralintitle{\sphinxupquote{date}} to accept an 8\sphinxhyphen{}digit date as an integer or string.}
\label{\detokenize{mckinney_02_practice_03:rewrite-date-to-accept-an-8-digit-date-as-an-integer-or-string}}
\begin{sphinxuseclass}{cell}\begin{sphinxVerbatimInput}

\begin{sphinxuseclass}{cell_input}
\begin{sphinxVerbatim}[commandchars=\\\{\}]
\PYG{k}{def} \PYG{n+nf}{date}\PYG{p}{(}\PYG{n}{ymd}\PYG{p}{)}\PYG{p}{:}
    \PYG{k}{if} \PYG{n+nb}{type}\PYG{p}{(}\PYG{n}{ymd}\PYG{p}{)} \PYG{o+ow}{is} \PYG{n+nb}{str}\PYG{p}{:}
        \PYG{n}{ymd} \PYG{o}{=} \PYG{n+nb}{int}\PYG{p}{(}\PYG{n}{ymd}\PYG{p}{)}
    \PYG{n}{year} \PYG{o}{=} \PYG{n}{ymd} \PYG{o}{/}\PYG{o}{/} \PYG{l+m+mi}{10000}
    \PYG{n}{month} \PYG{o}{=} \PYG{p}{(}\PYG{n}{ymd} \PYG{o}{/}\PYG{o}{/} \PYG{l+m+mi}{100}\PYG{p}{)} \PYG{o}{\PYGZpc{}} \PYG{l+m+mi}{100}
    \PYG{n}{day} \PYG{o}{=} \PYG{n}{ymd} \PYG{o}{\PYGZpc{}} \PYG{l+m+mi}{100}
    \PYG{k}{return} \PYG{p}{(}\PYG{n}{year}\PYG{p}{,} \PYG{n}{month}\PYG{p}{,} \PYG{n}{day}\PYG{p}{)}
\end{sphinxVerbatim}

\end{sphinxuseclass}\end{sphinxVerbatimInput}

\end{sphinxuseclass}
\begin{sphinxuseclass}{cell}\begin{sphinxVerbatimInput}

\begin{sphinxuseclass}{cell_input}
\begin{sphinxVerbatim}[commandchars=\\\{\}]
\PYG{n}{date}\PYG{p}{(}\PYG{l+m+mi}{20080915}\PYG{p}{)}
\end{sphinxVerbatim}

\end{sphinxuseclass}\end{sphinxVerbatimInput}
\begin{sphinxVerbatimOutput}

\begin{sphinxuseclass}{cell_output}
\begin{sphinxVerbatim}[commandchars=\\\{\}]
(2008, 9, 15)
\end{sphinxVerbatim}

\end{sphinxuseclass}\end{sphinxVerbatimOutput}

\end{sphinxuseclass}

\subsubsection{Finally, rewrite \sphinxstyleliteralintitle{\sphinxupquote{date}} to accept a list of 8\sphinxhyphen{}digit dates as integers or strings.}
\label{\detokenize{mckinney_02_practice_03:finally-rewrite-date-to-accept-a-list-of-8-digit-dates-as-integers-or-strings}}
\begin{sphinxuseclass}{cell}\begin{sphinxVerbatimInput}

\begin{sphinxuseclass}{cell_input}
\begin{sphinxVerbatim}[commandchars=\\\{\}]
\PYG{k}{def} \PYG{n+nf}{date}\PYG{p}{(}\PYG{n}{ymd}\PYG{p}{)}\PYG{p}{:}
    \PYG{n}{dates} \PYG{o}{=} \PYG{p}{[}\PYG{p}{]}
    \PYG{k}{if} \PYG{n+nb}{type}\PYG{p}{(}\PYG{n}{ymd}\PYG{p}{)} \PYG{o+ow}{is} \PYG{o+ow}{not} \PYG{n+nb}{list}\PYG{p}{:}
        \PYG{n}{ymd} \PYG{o}{=} \PYG{p}{[}\PYG{n}{ymd}\PYG{p}{]}
    \PYG{k}{for} \PYG{n}{i} \PYG{o+ow}{in} \PYG{n}{ymd}\PYG{p}{:}
        \PYG{k}{if} \PYG{n+nb}{type}\PYG{p}{(}\PYG{n}{i}\PYG{p}{)} \PYG{o+ow}{is} \PYG{n+nb}{str}\PYG{p}{:}
            \PYG{n}{i} \PYG{o}{=} \PYG{n+nb}{int}\PYG{p}{(}\PYG{n}{i}\PYG{p}{)}
        \PYG{n}{year} \PYG{o}{=} \PYG{n}{i} \PYG{o}{/}\PYG{o}{/} \PYG{l+m+mi}{10000}
        \PYG{n}{month} \PYG{o}{=} \PYG{p}{(}\PYG{n}{i} \PYG{o}{/}\PYG{o}{/} \PYG{l+m+mi}{100}\PYG{p}{)} \PYG{o}{\PYGZpc{}} \PYG{l+m+mi}{100}
        \PYG{n}{day} \PYG{o}{=} \PYG{n}{i} \PYG{o}{\PYGZpc{}} \PYG{l+m+mi}{100}
        \PYG{n}{dates}\PYG{o}{.}\PYG{n}{append}\PYG{p}{(}\PYG{p}{(}\PYG{n}{year}\PYG{p}{,} \PYG{n}{month}\PYG{p}{,} \PYG{n}{day}\PYG{p}{)}\PYG{p}{)}
    \PYG{k}{return} \PYG{n}{dates}
\end{sphinxVerbatim}

\end{sphinxuseclass}\end{sphinxVerbatimInput}

\end{sphinxuseclass}
\begin{sphinxuseclass}{cell}\begin{sphinxVerbatimInput}

\begin{sphinxuseclass}{cell_input}
\begin{sphinxVerbatim}[commandchars=\\\{\}]
\PYG{n}{date}\PYG{p}{(}\PYG{l+m+mi}{20080915}\PYG{p}{)}
\end{sphinxVerbatim}

\end{sphinxuseclass}\end{sphinxVerbatimInput}
\begin{sphinxVerbatimOutput}

\begin{sphinxuseclass}{cell_output}
\begin{sphinxVerbatim}[commandchars=\\\{\}]
[(2008, 9, 15)]
\end{sphinxVerbatim}

\end{sphinxuseclass}\end{sphinxVerbatimOutput}

\end{sphinxuseclass}
\begin{sphinxuseclass}{cell}\begin{sphinxVerbatimInput}

\begin{sphinxuseclass}{cell_input}
\begin{sphinxVerbatim}[commandchars=\\\{\}]
\PYG{n}{date}\PYG{p}{(}\PYG{p}{[}\PYG{l+m+mi}{20080915}\PYG{p}{,} \PYG{l+s+s1}{\PYGZsq{}}\PYG{l+s+s1}{20080915}\PYG{l+s+s1}{\PYGZsq{}}\PYG{p}{]}\PYG{p}{)}
\end{sphinxVerbatim}

\end{sphinxuseclass}\end{sphinxVerbatimInput}
\begin{sphinxVerbatimOutput}

\begin{sphinxuseclass}{cell_output}
\begin{sphinxVerbatim}[commandchars=\\\{\}]
[(2008, 9, 15), (2008, 9, 15)]
\end{sphinxVerbatim}

\end{sphinxuseclass}\end{sphinxVerbatimOutput}

\end{sphinxuseclass}

\subsubsection{Write a for loop that prints the squares of integers from 1 to 10.}
\label{\detokenize{mckinney_02_practice_03:write-a-for-loop-that-prints-the-squares-of-integers-from-1-to-10}}
\begin{sphinxuseclass}{cell}\begin{sphinxVerbatimInput}

\begin{sphinxuseclass}{cell_input}
\begin{sphinxVerbatim}[commandchars=\\\{\}]
\PYG{k}{for} \PYG{n}{i} \PYG{o+ow}{in} \PYG{n+nb}{range}\PYG{p}{(}\PYG{l+m+mi}{1}\PYG{p}{,} \PYG{l+m+mi}{11}\PYG{p}{)}\PYG{p}{:}
    \PYG{n+nb}{print}\PYG{p}{(}\PYG{n}{i}\PYG{o}{*}\PYG{o}{*}\PYG{l+m+mi}{2}\PYG{p}{,} \PYG{n}{end}\PYG{o}{=}\PYG{l+s+s1}{\PYGZsq{}}\PYG{l+s+s1}{ }\PYG{l+s+s1}{\PYGZsq{}}\PYG{p}{)}
\end{sphinxVerbatim}

\end{sphinxuseclass}\end{sphinxVerbatimInput}
\begin{sphinxVerbatimOutput}

\begin{sphinxuseclass}{cell_output}
\begin{sphinxVerbatim}[commandchars=\\\{\}]
1 4 9 16 25 36 49 64 81 100 
\end{sphinxVerbatim}

\end{sphinxuseclass}\end{sphinxVerbatimOutput}

\end{sphinxuseclass}
\sphinxAtStartPar
Above, I change the \sphinxcode{\sphinxupquote{end}} argument from the default ‘\textbackslash{}n’ to ‘ ‘.
The default ‘\textbackslash{}n’ inserts a new line after value, making the output too long.


\subsubsection{Write a for loop that prints the squares of \sphinxstyleemphasis{even} integers from 1 to 10.}
\label{\detokenize{mckinney_02_practice_03:write-a-for-loop-that-prints-the-squares-of-even-integers-from-1-to-10}}
\begin{sphinxuseclass}{cell}\begin{sphinxVerbatimInput}

\begin{sphinxuseclass}{cell_input}
\begin{sphinxVerbatim}[commandchars=\\\{\}]
\PYG{k}{for} \PYG{n}{i} \PYG{o+ow}{in} \PYG{n+nb}{range}\PYG{p}{(}\PYG{l+m+mi}{1}\PYG{p}{,} \PYG{l+m+mi}{11}\PYG{p}{)}\PYG{p}{:}
    \PYG{k}{if} \PYG{n}{i} \PYG{o}{\PYGZpc{}} \PYG{l+m+mi}{2} \PYG{o}{==} \PYG{l+m+mi}{0}\PYG{p}{:}
        \PYG{n+nb}{print}\PYG{p}{(}\PYG{n}{i}\PYG{o}{*}\PYG{o}{*}\PYG{l+m+mi}{2}\PYG{p}{,} \PYG{n}{end}\PYG{o}{=}\PYG{l+s+s1}{\PYGZsq{}}\PYG{l+s+s1}{ }\PYG{l+s+s1}{\PYGZsq{}}\PYG{p}{)}
\end{sphinxVerbatim}

\end{sphinxuseclass}\end{sphinxVerbatimInput}
\begin{sphinxVerbatimOutput}

\begin{sphinxuseclass}{cell_output}
\begin{sphinxVerbatim}[commandchars=\\\{\}]
4 16 36 64 100 
\end{sphinxVerbatim}

\end{sphinxuseclass}\end{sphinxVerbatimOutput}

\end{sphinxuseclass}

\subsubsection{Write a for loop that sums the squares of integers from 1 to 10.}
\label{\detokenize{mckinney_02_practice_03:write-a-for-loop-that-sums-the-squares-of-integers-from-1-to-10}}
\begin{sphinxuseclass}{cell}\begin{sphinxVerbatimInput}

\begin{sphinxuseclass}{cell_input}
\begin{sphinxVerbatim}[commandchars=\\\{\}]
\PYG{n}{total} \PYG{o}{=} \PYG{l+m+mi}{0}
\PYG{k}{for} \PYG{n}{i} \PYG{o+ow}{in} \PYG{n+nb}{range}\PYG{p}{(}\PYG{l+m+mi}{1}\PYG{p}{,} \PYG{l+m+mi}{11}\PYG{p}{)}\PYG{p}{:}
    \PYG{n}{total} \PYG{o}{+}\PYG{o}{=} \PYG{n}{i}\PYG{o}{*}\PYG{o}{*}\PYG{l+m+mi}{2}
\end{sphinxVerbatim}

\end{sphinxuseclass}\end{sphinxVerbatimInput}

\end{sphinxuseclass}
\begin{sphinxuseclass}{cell}\begin{sphinxVerbatimInput}

\begin{sphinxuseclass}{cell_input}
\begin{sphinxVerbatim}[commandchars=\\\{\}]
\PYG{n}{total}
\end{sphinxVerbatim}

\end{sphinxuseclass}\end{sphinxVerbatimInput}
\begin{sphinxVerbatimOutput}

\begin{sphinxuseclass}{cell_output}
\begin{sphinxVerbatim}[commandchars=\\\{\}]
385
\end{sphinxVerbatim}

\end{sphinxuseclass}\end{sphinxVerbatimOutput}

\end{sphinxuseclass}
\sphinxAtStartPar
Above, I use \sphinxcode{\sphinxupquote{total += i}}, which is equivalent to \sphinxcode{\sphinxupquote{total = total + i}}.


\subsubsection{Write a for loop that sums the squares of integers from 1 to 10 but stops before the sum exceeds 50.}
\label{\detokenize{mckinney_02_practice_03:write-a-for-loop-that-sums-the-squares-of-integers-from-1-to-10-but-stops-before-the-sum-exceeds-50}}
\begin{sphinxuseclass}{cell}\begin{sphinxVerbatimInput}

\begin{sphinxuseclass}{cell_input}
\begin{sphinxVerbatim}[commandchars=\\\{\}]
\PYG{n}{total} \PYG{o}{=} \PYG{l+m+mi}{0}
\PYG{k}{for} \PYG{n}{i} \PYG{o+ow}{in} \PYG{n+nb}{range}\PYG{p}{(}\PYG{l+m+mi}{1}\PYG{p}{,} \PYG{l+m+mi}{11}\PYG{p}{)}\PYG{p}{:}
    \PYG{k}{if} \PYG{p}{(}\PYG{n}{total} \PYG{o}{+} \PYG{n}{i}\PYG{o}{*}\PYG{o}{*}\PYG{l+m+mi}{2}\PYG{p}{)} \PYG{o}{\PYGZgt{}} \PYG{l+m+mi}{50}\PYG{p}{:}
        \PYG{k}{break}
    \PYG{n}{total} \PYG{o}{+}\PYG{o}{=} \PYG{n}{i}\PYG{o}{*}\PYG{o}{*}\PYG{l+m+mi}{2}
\end{sphinxVerbatim}

\end{sphinxuseclass}\end{sphinxVerbatimInput}

\end{sphinxuseclass}
\begin{sphinxuseclass}{cell}\begin{sphinxVerbatimInput}

\begin{sphinxuseclass}{cell_input}
\begin{sphinxVerbatim}[commandchars=\\\{\}]
\PYG{n}{total}
\end{sphinxVerbatim}

\end{sphinxuseclass}\end{sphinxVerbatimInput}
\begin{sphinxVerbatimOutput}

\begin{sphinxuseclass}{cell_output}
\begin{sphinxVerbatim}[commandchars=\\\{\}]
30
\end{sphinxVerbatim}

\end{sphinxuseclass}\end{sphinxVerbatimOutput}

\end{sphinxuseclass}

\subsubsection{FizzBuzz}
\label{\detokenize{mckinney_02_practice_03:fizzbuzz}}
\sphinxAtStartPar
Write a for loop that prints the numbers from 1 to 100.
For multiples of three print “Fizz” instead of the number.
For multiples of five print “Buzz”.
For numbers that are multiples of both three and five print “FizzBuzz”.
More \sphinxhref{https://blog.codinghorror.com/why-cant-programmers-program/}{here}.

\begin{sphinxuseclass}{cell}\begin{sphinxVerbatimInput}

\begin{sphinxuseclass}{cell_input}
\begin{sphinxVerbatim}[commandchars=\\\{\}]
\PYG{k}{for} \PYG{n}{i} \PYG{o+ow}{in} \PYG{n+nb}{range}\PYG{p}{(}\PYG{l+m+mi}{1}\PYG{p}{,} \PYG{l+m+mi}{101}\PYG{p}{)}\PYG{p}{:}
    \PYG{n}{is\PYGZus{}mult\PYGZus{}3} \PYG{o}{=} \PYG{p}{(}\PYG{n}{i} \PYG{o}{\PYGZpc{}} \PYG{l+m+mi}{3} \PYG{o}{==} \PYG{l+m+mi}{0}\PYG{p}{)}
    \PYG{n}{is\PYGZus{}mult\PYGZus{}5} \PYG{o}{=} \PYG{p}{(}\PYG{n}{i} \PYG{o}{\PYGZpc{}} \PYG{l+m+mi}{5} \PYG{o}{==} \PYG{l+m+mi}{0}\PYG{p}{)}
    \PYG{k}{if} \PYG{n}{is\PYGZus{}mult\PYGZus{}3} \PYG{o+ow}{and} \PYG{n}{is\PYGZus{}mult\PYGZus{}5}\PYG{p}{:}
        \PYG{n+nb}{print}\PYG{p}{(}\PYG{l+s+s1}{\PYGZsq{}}\PYG{l+s+s1}{FizzBuzz}\PYG{l+s+s1}{\PYGZsq{}}\PYG{p}{,} \PYG{n}{end}\PYG{o}{=}\PYG{l+s+s1}{\PYGZsq{}}\PYG{l+s+s1}{ }\PYG{l+s+s1}{\PYGZsq{}}\PYG{p}{)}
    \PYG{k}{elif} \PYG{n}{is\PYGZus{}mult\PYGZus{}3}\PYG{p}{:}
        \PYG{n+nb}{print}\PYG{p}{(}\PYG{l+s+s1}{\PYGZsq{}}\PYG{l+s+s1}{Fizz}\PYG{l+s+s1}{\PYGZsq{}}\PYG{p}{,} \PYG{n}{end}\PYG{o}{=}\PYG{l+s+s1}{\PYGZsq{}}\PYG{l+s+s1}{ }\PYG{l+s+s1}{\PYGZsq{}}\PYG{p}{)}
    \PYG{k}{elif} \PYG{n}{is\PYGZus{}mult\PYGZus{}5}\PYG{p}{:}
        \PYG{n+nb}{print}\PYG{p}{(}\PYG{l+s+s1}{\PYGZsq{}}\PYG{l+s+s1}{Buzz}\PYG{l+s+s1}{\PYGZsq{}}\PYG{p}{,} \PYG{n}{end}\PYG{o}{=}\PYG{l+s+s1}{\PYGZsq{}}\PYG{l+s+s1}{ }\PYG{l+s+s1}{\PYGZsq{}}\PYG{p}{)}
    \PYG{k}{else}\PYG{p}{:}
        \PYG{n+nb}{print}\PYG{p}{(}\PYG{n}{i}\PYG{p}{,} \PYG{n}{end}\PYG{o}{=}\PYG{l+s+s1}{\PYGZsq{}}\PYG{l+s+s1}{ }\PYG{l+s+s1}{\PYGZsq{}}\PYG{p}{)}
\end{sphinxVerbatim}

\end{sphinxuseclass}\end{sphinxVerbatimInput}
\begin{sphinxVerbatimOutput}

\begin{sphinxuseclass}{cell_output}
\begin{sphinxVerbatim}[commandchars=\\\{\}]
1 2 Fizz 4 Buzz Fizz 7 8 Fizz Buzz 11 Fizz 13 14 FizzBuzz 16 17 Fizz 19 Buzz Fizz 22 23 Fizz Buzz 26 Fizz 28 29 FizzBuzz 31 32 Fizz 34 Buzz Fizz 37 38 Fizz Buzz 41 Fizz 43 44 FizzBuzz 46 47 Fizz 49 Buzz Fizz 52 53 Fizz Buzz 56 Fizz 58 59 FizzBuzz 61 62 Fizz 64 Buzz Fizz 67 68 Fizz Buzz 71 Fizz 73 74 FizzBuzz 76 77 Fizz 79 Buzz Fizz 82 83 Fizz Buzz 86 Fizz 88 89 FizzBuzz 91 92 Fizz 94 Buzz Fizz 97 98 Fizz Buzz 
\end{sphinxVerbatim}

\end{sphinxuseclass}\end{sphinxVerbatimOutput}

\end{sphinxuseclass}

\subsubsection{Use ternary expressions to make your FizzBuzz code more compact.}
\label{\detokenize{mckinney_02_practice_03:use-ternary-expressions-to-make-your-fizzbuzz-code-more-compact}}
\begin{sphinxuseclass}{cell}\begin{sphinxVerbatimInput}

\begin{sphinxuseclass}{cell_input}
\begin{sphinxVerbatim}[commandchars=\\\{\}]
\PYG{k}{for} \PYG{n}{i} \PYG{o+ow}{in} \PYG{n+nb}{range}\PYG{p}{(}\PYG{l+m+mi}{1}\PYG{p}{,} \PYG{l+m+mi}{101}\PYG{p}{)}\PYG{p}{:}
    \PYG{n}{is\PYGZus{}mult\PYGZus{}3} \PYG{o}{=} \PYG{p}{(}\PYG{n}{i} \PYG{o}{\PYGZpc{}} \PYG{l+m+mi}{3} \PYG{o}{==} \PYG{l+m+mi}{0}\PYG{p}{)}
    \PYG{n}{is\PYGZus{}mult\PYGZus{}5} \PYG{o}{=} \PYG{p}{(}\PYG{n}{i} \PYG{o}{\PYGZpc{}} \PYG{l+m+mi}{5} \PYG{o}{==} \PYG{l+m+mi}{0}\PYG{p}{)}
    \PYG{n+nb}{print}\PYG{p}{(}\PYG{l+s+s1}{\PYGZsq{}}\PYG{l+s+s1}{Fizz}\PYG{l+s+s1}{\PYGZsq{}}\PYG{o}{*}\PYG{n}{is\PYGZus{}mult\PYGZus{}3} \PYG{o}{+} \PYG{l+s+s1}{\PYGZsq{}}\PYG{l+s+s1}{Buzz}\PYG{l+s+s1}{\PYGZsq{}}\PYG{o}{*}\PYG{n}{is\PYGZus{}mult\PYGZus{}5} \PYG{k}{if} \PYG{n}{is\PYGZus{}mult\PYGZus{}3} \PYG{o+ow}{or} \PYG{n}{is\PYGZus{}mult\PYGZus{}5} \PYG{k}{else} \PYG{n}{i}\PYG{p}{,} \PYG{n}{end}\PYG{o}{=}\PYG{l+s+s1}{\PYGZsq{}}\PYG{l+s+s1}{ }\PYG{l+s+s1}{\PYGZsq{}}\PYG{p}{)}
\end{sphinxVerbatim}

\end{sphinxuseclass}\end{sphinxVerbatimInput}
\begin{sphinxVerbatimOutput}

\begin{sphinxuseclass}{cell_output}
\begin{sphinxVerbatim}[commandchars=\\\{\}]
1 2 Fizz 4 Buzz Fizz 7 8 Fizz Buzz 11 Fizz 13 14 FizzBuzz 16 17 Fizz 19 Buzz Fizz 22 23 Fizz Buzz 26 Fizz 28 29 FizzBuzz 31 32 Fizz 34 Buzz Fizz 37 38 Fizz Buzz 41 Fizz 43 44 FizzBuzz 46 47 Fizz 49 Buzz Fizz 52 53 Fizz Buzz 56 Fizz 58 59 FizzBuzz 61 62 Fizz 64 Buzz Fizz 67 68 Fizz Buzz 71 Fizz 73 74 FizzBuzz 76 77 Fizz 79 Buzz Fizz 82 83 Fizz Buzz 86 Fizz 88 89 FizzBuzz 91 92 Fizz 94 Buzz Fizz 97 98 Fizz Buzz 
\end{sphinxVerbatim}

\end{sphinxuseclass}\end{sphinxVerbatimOutput}

\end{sphinxuseclass}
\sphinxAtStartPar
The solution above is shorter and uses from neat tricks, but I consider the previous solution easier to read and troubleshoot.


\subsubsection{Triangle}
\label{\detokenize{mckinney_02_practice_03:triangle}}
\sphinxAtStartPar
Write a function \sphinxcode{\sphinxupquote{triangle}} that accepts a positive integer \$N\$ and prints a numerical triangle of height \$N\sphinxhyphen{}1\$.
For example, \sphinxcode{\sphinxupquote{triangle(N=6)}} should print:

\begin{sphinxVerbatim}[commandchars=\\\{\}]
\PYG{l+m+mi}{1}
\PYG{l+m+mi}{22}
\PYG{l+m+mi}{333}
\PYG{l+m+mi}{4444}
\PYG{l+m+mi}{55555}
\end{sphinxVerbatim}

\begin{sphinxuseclass}{cell}\begin{sphinxVerbatimInput}

\begin{sphinxuseclass}{cell_input}
\begin{sphinxVerbatim}[commandchars=\\\{\}]
\PYG{k}{def} \PYG{n+nf}{triangle}\PYG{p}{(}\PYG{n}{N}\PYG{p}{)}\PYG{p}{:}
    \PYG{k}{for} \PYG{n}{i} \PYG{o+ow}{in} \PYG{n+nb}{range}\PYG{p}{(}\PYG{l+m+mi}{1}\PYG{p}{,} \PYG{n}{N}\PYG{p}{)}\PYG{p}{:}
        \PYG{n+nb}{print}\PYG{p}{(}\PYG{n+nb}{str}\PYG{p}{(}\PYG{n}{i}\PYG{p}{)} \PYG{o}{*} \PYG{n}{i}\PYG{p}{)}
\end{sphinxVerbatim}

\end{sphinxuseclass}\end{sphinxVerbatimInput}

\end{sphinxuseclass}
\begin{sphinxuseclass}{cell}\begin{sphinxVerbatimInput}

\begin{sphinxuseclass}{cell_input}
\begin{sphinxVerbatim}[commandchars=\\\{\}]
\PYG{n}{triangle}\PYG{p}{(}\PYG{l+m+mi}{6}\PYG{p}{)}
\end{sphinxVerbatim}

\end{sphinxuseclass}\end{sphinxVerbatimInput}
\begin{sphinxVerbatimOutput}

\begin{sphinxuseclass}{cell_output}
\begin{sphinxVerbatim}[commandchars=\\\{\}]
1
22
333
4444
55555
\end{sphinxVerbatim}

\end{sphinxuseclass}\end{sphinxVerbatimOutput}

\end{sphinxuseclass}
\sphinxAtStartPar
The solution above works because a multiplying a string by \sphinxcode{\sphinxupquote{i}} concatenates \sphinxcode{\sphinxupquote{i}} copies of the string.

\begin{sphinxuseclass}{cell}\begin{sphinxVerbatimInput}

\begin{sphinxuseclass}{cell_input}
\begin{sphinxVerbatim}[commandchars=\\\{\}]
\PYG{l+s+s1}{\PYGZsq{}}\PYG{l+s+s1}{Test}\PYG{l+s+s1}{\PYGZsq{}} \PYG{o}{+} \PYG{l+s+s1}{\PYGZsq{}}\PYG{l+s+s1}{Test}\PYG{l+s+s1}{\PYGZsq{}} \PYG{o}{+} \PYG{l+s+s1}{\PYGZsq{}}\PYG{l+s+s1}{Test}\PYG{l+s+s1}{\PYGZsq{}}
\end{sphinxVerbatim}

\end{sphinxuseclass}\end{sphinxVerbatimInput}
\begin{sphinxVerbatimOutput}

\begin{sphinxuseclass}{cell_output}
\begin{sphinxVerbatim}[commandchars=\\\{\}]
\PYGZsq{}TestTestTest\PYGZsq{}
\end{sphinxVerbatim}

\end{sphinxuseclass}\end{sphinxVerbatimOutput}

\end{sphinxuseclass}
\begin{sphinxuseclass}{cell}\begin{sphinxVerbatimInput}

\begin{sphinxuseclass}{cell_input}
\begin{sphinxVerbatim}[commandchars=\\\{\}]
\PYG{l+s+s1}{\PYGZsq{}}\PYG{l+s+s1}{Test}\PYG{l+s+s1}{\PYGZsq{}} \PYG{o}{*} \PYG{l+m+mi}{3}
\end{sphinxVerbatim}

\end{sphinxuseclass}\end{sphinxVerbatimInput}
\begin{sphinxVerbatimOutput}

\begin{sphinxuseclass}{cell_output}
\begin{sphinxVerbatim}[commandchars=\\\{\}]
\PYGZsq{}TestTestTest\PYGZsq{}
\end{sphinxVerbatim}

\end{sphinxuseclass}\end{sphinxVerbatimOutput}

\end{sphinxuseclass}

\subsubsection{Two Sum}
\label{\detokenize{mckinney_02_practice_03:two-sum}}
\sphinxAtStartPar
Write a function \sphinxcode{\sphinxupquote{two\_sum}} that does the following.

\sphinxAtStartPar
Given an array of integers \sphinxcode{\sphinxupquote{nums}} and an integer \sphinxcode{\sphinxupquote{target}}, return the indices of the two numbers that add up to target.

\sphinxAtStartPar
You may assume that each input would have exactly one solution, and you may not use the same element twice.

\sphinxAtStartPar
You can return the answer in any order.

\sphinxAtStartPar
Here are some examples:

\sphinxAtStartPar
Example 1:

\sphinxAtStartPar
Input: \sphinxcode{\sphinxupquote{nums = {[}2,7,11,15{]}}}, \sphinxcode{\sphinxupquote{target = 9}} \\
Output: \sphinxcode{\sphinxupquote{{[}0,1{]}}} \\
Explanation: Because \sphinxcode{\sphinxupquote{nums{[}0{]} + nums{[}1{]} == 9}}, we return \sphinxcode{\sphinxupquote{{[}0, 1{]}}}.

\sphinxAtStartPar
Example 2:

\sphinxAtStartPar
Input: \sphinxcode{\sphinxupquote{nums = {[}3,2,4{]}}}, \sphinxcode{\sphinxupquote{target = 6}} \\
Output: \sphinxcode{\sphinxupquote{{[}1,2{]}}} \textbackslash{}

\sphinxAtStartPar
Example 3:

\sphinxAtStartPar
Input: \sphinxcode{\sphinxupquote{nums = {[}3,3{]}}}, \sphinxcode{\sphinxupquote{target = 6}} \\
Output: \sphinxcode{\sphinxupquote{{[}0,1{]}}} \textbackslash{}

\sphinxAtStartPar
I saw this question on \sphinxhref{https://leetcode.com/problems/two-sum/description/}{LeetCode}.

\begin{sphinxuseclass}{cell}\begin{sphinxVerbatimInput}

\begin{sphinxuseclass}{cell_input}
\begin{sphinxVerbatim}[commandchars=\\\{\}]
\PYG{k}{def} \PYG{n+nf}{two\PYGZus{}sum}\PYG{p}{(}\PYG{n}{nums}\PYG{p}{,} \PYG{n}{target}\PYG{p}{)}\PYG{p}{:}
    \PYG{k}{for} \PYG{n}{i} \PYG{o+ow}{in} \PYG{n+nb}{range}\PYG{p}{(}\PYG{l+m+mi}{1}\PYG{p}{,} \PYG{n+nb}{len}\PYG{p}{(}\PYG{n}{nums}\PYG{p}{)}\PYG{p}{)}\PYG{p}{:}
        \PYG{k}{for} \PYG{n}{j} \PYG{o+ow}{in} \PYG{n+nb}{range}\PYG{p}{(}\PYG{n}{i}\PYG{p}{)}\PYG{p}{:}
            \PYG{k}{if} \PYG{n}{nums}\PYG{p}{[}\PYG{n}{i}\PYG{p}{]} \PYG{o}{+} \PYG{n}{nums}\PYG{p}{[}\PYG{n}{j}\PYG{p}{]} \PYG{o}{==} \PYG{n}{target}\PYG{p}{:}
                \PYG{k}{return} \PYG{p}{[}\PYG{n}{j}\PYG{p}{,} \PYG{n}{i}\PYG{p}{]}
\end{sphinxVerbatim}

\end{sphinxuseclass}\end{sphinxVerbatimInput}

\end{sphinxuseclass}
\begin{sphinxuseclass}{cell}\begin{sphinxVerbatimInput}

\begin{sphinxuseclass}{cell_input}
\begin{sphinxVerbatim}[commandchars=\\\{\}]
\PYG{n}{two\PYGZus{}sum}\PYG{p}{(}\PYG{n}{nums} \PYG{o}{=} \PYG{p}{[}\PYG{l+m+mi}{2}\PYG{p}{,}\PYG{l+m+mi}{7}\PYG{p}{,}\PYG{l+m+mi}{11}\PYG{p}{,}\PYG{l+m+mi}{15}\PYG{p}{]}\PYG{p}{,} \PYG{n}{target} \PYG{o}{=} \PYG{l+m+mi}{9}\PYG{p}{)}
\end{sphinxVerbatim}

\end{sphinxuseclass}\end{sphinxVerbatimInput}
\begin{sphinxVerbatimOutput}

\begin{sphinxuseclass}{cell_output}
\begin{sphinxVerbatim}[commandchars=\\\{\}]
[0, 1]
\end{sphinxVerbatim}

\end{sphinxuseclass}\end{sphinxVerbatimOutput}

\end{sphinxuseclass}
\begin{sphinxuseclass}{cell}\begin{sphinxVerbatimInput}

\begin{sphinxuseclass}{cell_input}
\begin{sphinxVerbatim}[commandchars=\\\{\}]
\PYG{n}{two\PYGZus{}sum}\PYG{p}{(}\PYG{n}{nums} \PYG{o}{=} \PYG{p}{[}\PYG{l+m+mi}{3}\PYG{p}{,}\PYG{l+m+mi}{2}\PYG{p}{,}\PYG{l+m+mi}{4}\PYG{p}{]}\PYG{p}{,} \PYG{n}{target} \PYG{o}{=} \PYG{l+m+mi}{6}\PYG{p}{)}
\end{sphinxVerbatim}

\end{sphinxuseclass}\end{sphinxVerbatimInput}
\begin{sphinxVerbatimOutput}

\begin{sphinxuseclass}{cell_output}
\begin{sphinxVerbatim}[commandchars=\\\{\}]
[1, 2]
\end{sphinxVerbatim}

\end{sphinxuseclass}\end{sphinxVerbatimOutput}

\end{sphinxuseclass}
\begin{sphinxuseclass}{cell}\begin{sphinxVerbatimInput}

\begin{sphinxuseclass}{cell_input}
\begin{sphinxVerbatim}[commandchars=\\\{\}]
\PYG{n}{two\PYGZus{}sum}\PYG{p}{(}\PYG{n}{nums} \PYG{o}{=} \PYG{p}{[}\PYG{l+m+mi}{3}\PYG{p}{,}\PYG{l+m+mi}{3}\PYG{p}{]}\PYG{p}{,} \PYG{n}{target} \PYG{o}{=} \PYG{l+m+mi}{6}\PYG{p}{)}
\end{sphinxVerbatim}

\end{sphinxuseclass}\end{sphinxVerbatimInput}
\begin{sphinxVerbatimOutput}

\begin{sphinxuseclass}{cell_output}
\begin{sphinxVerbatim}[commandchars=\\\{\}]
[0, 1]
\end{sphinxVerbatim}

\end{sphinxuseclass}\end{sphinxVerbatimOutput}

\end{sphinxuseclass}
\sphinxAtStartPar
We can write more efficient code once we learn other data structures in chapter 3 of McKinney!


\subsubsection{Best Time}
\label{\detokenize{mckinney_02_practice_03:best-time}}
\sphinxAtStartPar
Write a function \sphinxcode{\sphinxupquote{best\_time}} that solves the following.

\sphinxAtStartPar
You are given an array \sphinxcode{\sphinxupquote{prices}} where \sphinxcode{\sphinxupquote{prices{[}i{]}}} is the price of a given stock on the \$i\textasciicircum{}\{th\}\$ day.

\sphinxAtStartPar
You want to maximize your profit by choosing a single day to buy one stock and choosing a different day in the future to sell that stock.

\sphinxAtStartPar
Return the maximum profit you can achieve from this transaction. If you cannot achieve any profit, return 0.

\sphinxAtStartPar
Here are some examples:

\sphinxAtStartPar
Example 1:

\sphinxAtStartPar
Input: \sphinxcode{\sphinxupquote{prices = {[}7,1,5,3,6,4{]}}} \\
Output: \sphinxcode{\sphinxupquote{5}} \\
Explanation: Buy on day 2 (price = 1) and sell on day 5 (price = 6), profit = 6\sphinxhyphen{}1 = 5.
Note that buying on day 2 and selling on day 1 is not allowed because you must buy before you sell.

\sphinxAtStartPar
Example 2:

\sphinxAtStartPar
Input: \sphinxcode{\sphinxupquote{prices = {[}7,6,4,3,1{]}}} \\
Output: \sphinxcode{\sphinxupquote{0}} \\
Explanation: In this case, no transactions are done and the max profit = 0.

\sphinxAtStartPar
I saw this question on \sphinxhref{https://leetcode.com/problems/best-time-to-buy-and-sell-stock/}{LeetCode}.

\begin{sphinxuseclass}{cell}\begin{sphinxVerbatimInput}

\begin{sphinxuseclass}{cell_input}
\begin{sphinxVerbatim}[commandchars=\\\{\}]
\PYG{k}{def} \PYG{n+nf}{max\PYGZus{}profit}\PYG{p}{(}\PYG{n}{prices}\PYG{p}{)}\PYG{p}{:}
        \PYG{n}{min\PYGZus{}price} \PYG{o}{=} \PYG{n}{prices}\PYG{p}{[}\PYG{l+m+mi}{0}\PYG{p}{]}
        \PYG{n}{max\PYGZus{}profit} \PYG{o}{=} \PYG{l+m+mi}{0}
        \PYG{k}{for} \PYG{n}{price} \PYG{o+ow}{in} \PYG{n}{prices}\PYG{p}{:}
            \PYG{n}{min\PYGZus{}price} \PYG{o}{=} \PYG{n}{price} \PYG{k}{if} \PYG{n}{price} \PYG{o}{\PYGZlt{}} \PYG{n}{min\PYGZus{}price} \PYG{k}{else} \PYG{n}{min\PYGZus{}price}
            \PYG{n}{profit} \PYG{o}{=} \PYG{n}{price} \PYG{o}{\PYGZhy{}} \PYG{n}{min\PYGZus{}price}
            \PYG{n}{max\PYGZus{}profit} \PYG{o}{=} \PYG{n}{profit} \PYG{k}{if} \PYG{n}{profit} \PYG{o}{\PYGZgt{}} \PYG{n}{max\PYGZus{}profit} \PYG{k}{else} \PYG{n}{max\PYGZus{}profit}
        \PYG{k}{return} \PYG{n}{max\PYGZus{}profit}
\end{sphinxVerbatim}

\end{sphinxuseclass}\end{sphinxVerbatimInput}

\end{sphinxuseclass}
\begin{sphinxuseclass}{cell}\begin{sphinxVerbatimInput}

\begin{sphinxuseclass}{cell_input}
\begin{sphinxVerbatim}[commandchars=\\\{\}]
\PYG{n}{max\PYGZus{}profit}\PYG{p}{(}\PYG{n}{prices}\PYG{o}{=}\PYG{p}{[}\PYG{l+m+mi}{7}\PYG{p}{,}\PYG{l+m+mi}{1}\PYG{p}{,}\PYG{l+m+mi}{5}\PYG{p}{,}\PYG{l+m+mi}{3}\PYG{p}{,}\PYG{l+m+mi}{6}\PYG{p}{,}\PYG{l+m+mi}{4}\PYG{p}{]}\PYG{p}{)}
\end{sphinxVerbatim}

\end{sphinxuseclass}\end{sphinxVerbatimInput}
\begin{sphinxVerbatimOutput}

\begin{sphinxuseclass}{cell_output}
\begin{sphinxVerbatim}[commandchars=\\\{\}]
5
\end{sphinxVerbatim}

\end{sphinxuseclass}\end{sphinxVerbatimOutput}

\end{sphinxuseclass}
\begin{sphinxuseclass}{cell}\begin{sphinxVerbatimInput}

\begin{sphinxuseclass}{cell_input}
\begin{sphinxVerbatim}[commandchars=\\\{\}]
\PYG{n}{max\PYGZus{}profit}\PYG{p}{(}\PYG{n}{prices}\PYG{o}{=}\PYG{p}{[}\PYG{l+m+mi}{7}\PYG{p}{,}\PYG{l+m+mi}{6}\PYG{p}{,}\PYG{l+m+mi}{4}\PYG{p}{,}\PYG{l+m+mi}{3}\PYG{p}{,}\PYG{l+m+mi}{1}\PYG{p}{]}\PYG{p}{)}
\end{sphinxVerbatim}

\end{sphinxuseclass}\end{sphinxVerbatimInput}
\begin{sphinxVerbatimOutput}

\begin{sphinxuseclass}{cell_output}
\begin{sphinxVerbatim}[commandchars=\\\{\}]
0
\end{sphinxVerbatim}

\end{sphinxuseclass}\end{sphinxVerbatimOutput}

\end{sphinxuseclass}
\sphinxstepscope


\section{McKinney Chapter 2 \sphinxhyphen{} Practice (Section 4, Wednesday 11:45 AM)}
\label{\detokenize{mckinney_02_practice_04:mckinney-chapter-2-practice-section-4-wednesday-11-45-am}}\label{\detokenize{mckinney_02_practice_04::doc}}

\subsection{Practice}
\label{\detokenize{mckinney_02_practice_04:practice}}

\subsubsection{Extract the year, month, and day from an integer 8\sphinxhyphen{}digit date (i.e., YYYYMMDD format) using \sphinxstyleliteralintitle{\sphinxupquote{//}} (integer division) and \sphinxstyleliteralintitle{\sphinxupquote{\%}} (modulo division).}
\label{\detokenize{mckinney_02_practice_04:extract-the-year-month-and-day-from-an-integer-8-digit-date-i-e-yyyymmdd-format-using-integer-division-and-modulo-division}}
\sphinxAtStartPar
Try \sphinxcode{\sphinxupquote{20080915}}.

\begin{sphinxuseclass}{cell}\begin{sphinxVerbatimInput}

\begin{sphinxuseclass}{cell_input}
\begin{sphinxVerbatim}[commandchars=\\\{\}]
\PYG{l+m+mi}{1234} \PYG{o}{/} \PYG{l+m+mi}{100} \PYG{c+c1}{\PYGZsh{} \PYGZdq{}regular\PYGZdq{} division}
\end{sphinxVerbatim}

\end{sphinxuseclass}\end{sphinxVerbatimInput}
\begin{sphinxVerbatimOutput}

\begin{sphinxuseclass}{cell_output}
\begin{sphinxVerbatim}[commandchars=\\\{\}]
12.34
\end{sphinxVerbatim}

\end{sphinxuseclass}\end{sphinxVerbatimOutput}

\end{sphinxuseclass}
\begin{sphinxuseclass}{cell}\begin{sphinxVerbatimInput}

\begin{sphinxuseclass}{cell_input}
\begin{sphinxVerbatim}[commandchars=\\\{\}]
\PYG{l+m+mi}{1234} \PYG{o}{/}\PYG{o}{/} \PYG{l+m+mi}{100} \PYG{c+c1}{\PYGZsh{} integer or floor division gives us the integer portion}
\end{sphinxVerbatim}

\end{sphinxuseclass}\end{sphinxVerbatimInput}
\begin{sphinxVerbatimOutput}

\begin{sphinxuseclass}{cell_output}
\begin{sphinxVerbatim}[commandchars=\\\{\}]
12
\end{sphinxVerbatim}

\end{sphinxuseclass}\end{sphinxVerbatimOutput}

\end{sphinxuseclass}
\begin{sphinxuseclass}{cell}\begin{sphinxVerbatimInput}

\begin{sphinxuseclass}{cell_input}
\begin{sphinxVerbatim}[commandchars=\\\{\}]
\PYG{l+m+mi}{1234} \PYG{o}{\PYGZpc{}} \PYG{l+m+mi}{100} \PYG{c+c1}{\PYGZsh{} modulo division gives us the remainder}
\end{sphinxVerbatim}

\end{sphinxuseclass}\end{sphinxVerbatimInput}
\begin{sphinxVerbatimOutput}

\begin{sphinxuseclass}{cell_output}
\begin{sphinxVerbatim}[commandchars=\\\{\}]
34
\end{sphinxVerbatim}

\end{sphinxuseclass}\end{sphinxVerbatimOutput}

\end{sphinxuseclass}
\begin{sphinxuseclass}{cell}\begin{sphinxVerbatimInput}

\begin{sphinxuseclass}{cell_input}
\begin{sphinxVerbatim}[commandchars=\\\{\}]
\PYG{n}{ymd} \PYG{o}{=} \PYG{l+m+mi}{20080915}
\end{sphinxVerbatim}

\end{sphinxuseclass}\end{sphinxVerbatimInput}

\end{sphinxuseclass}
\begin{sphinxuseclass}{cell}\begin{sphinxVerbatimInput}

\begin{sphinxuseclass}{cell_input}
\begin{sphinxVerbatim}[commandchars=\\\{\}]
\PYG{n}{year} \PYG{o}{=} \PYG{n}{ymd} \PYG{o}{/}\PYG{o}{/} \PYG{l+m+mi}{10\PYGZus{}000}
\end{sphinxVerbatim}

\end{sphinxuseclass}\end{sphinxVerbatimInput}

\end{sphinxuseclass}
\begin{sphinxuseclass}{cell}\begin{sphinxVerbatimInput}

\begin{sphinxuseclass}{cell_input}
\begin{sphinxVerbatim}[commandchars=\\\{\}]
\PYG{n}{month} \PYG{o}{=} \PYG{p}{(}\PYG{n}{ymd} \PYG{o}{\PYGZpc{}} \PYG{l+m+mi}{10\PYGZus{}000}\PYG{p}{)} \PYG{o}{/}\PYG{o}{/} \PYG{l+m+mi}{100}
\end{sphinxVerbatim}

\end{sphinxuseclass}\end{sphinxVerbatimInput}

\end{sphinxuseclass}
\begin{sphinxuseclass}{cell}\begin{sphinxVerbatimInput}

\begin{sphinxuseclass}{cell_input}
\begin{sphinxVerbatim}[commandchars=\\\{\}]
\PYG{n}{day} \PYG{o}{=} \PYG{n}{ymd} \PYG{o}{\PYGZpc{}} \PYG{l+m+mi}{100}
\end{sphinxVerbatim}

\end{sphinxuseclass}\end{sphinxVerbatimInput}

\end{sphinxuseclass}
\begin{sphinxuseclass}{cell}\begin{sphinxVerbatimInput}

\begin{sphinxuseclass}{cell_input}
\begin{sphinxVerbatim}[commandchars=\\\{\}]
\PYG{p}{(}\PYG{n}{year}\PYG{p}{,} \PYG{n}{month}\PYG{p}{,} \PYG{n}{day}\PYG{p}{)}
\end{sphinxVerbatim}

\end{sphinxuseclass}\end{sphinxVerbatimInput}
\begin{sphinxVerbatimOutput}

\begin{sphinxuseclass}{cell_output}
\begin{sphinxVerbatim}[commandchars=\\\{\}]
(2008, 9, 15)
\end{sphinxVerbatim}

\end{sphinxuseclass}\end{sphinxVerbatimOutput}

\end{sphinxuseclass}

\subsubsection{Use your answer above to write a function \sphinxstyleliteralintitle{\sphinxupquote{date}} that accepts an integer 8\sphinxhyphen{}digit date argument and returns a tuple of the year, month, and date (e.g., \sphinxstyleliteralintitle{\sphinxupquote{return (year, month, date)}}).}
\label{\detokenize{mckinney_02_practice_04:use-your-answer-above-to-write-a-function-date-that-accepts-an-integer-8-digit-date-argument-and-returns-a-tuple-of-the-year-month-and-date-e-g-return-year-month-date}}
\begin{sphinxuseclass}{cell}\begin{sphinxVerbatimInput}

\begin{sphinxuseclass}{cell_input}
\begin{sphinxVerbatim}[commandchars=\\\{\}]
\PYG{k}{def} \PYG{n+nf}{date}\PYG{p}{(}\PYG{n}{ymd}\PYG{p}{)}\PYG{p}{:}
    \PYG{n}{year} \PYG{o}{=} \PYG{n}{ymd} \PYG{o}{/}\PYG{o}{/} \PYG{l+m+mi}{10\PYGZus{}000}
    \PYG{n}{month} \PYG{o}{=} \PYG{p}{(}\PYG{n}{ymd} \PYG{o}{\PYGZpc{}} \PYG{l+m+mi}{10\PYGZus{}000}\PYG{p}{)} \PYG{o}{/}\PYG{o}{/} \PYG{l+m+mi}{100}
    \PYG{n}{day} \PYG{o}{=} \PYG{n}{ymd} \PYG{o}{\PYGZpc{}} \PYG{l+m+mi}{100}
    \PYG{k}{return} \PYG{p}{(}\PYG{n}{year}\PYG{p}{,} \PYG{n}{month}\PYG{p}{,} \PYG{n}{day}\PYG{p}{)}
\end{sphinxVerbatim}

\end{sphinxuseclass}\end{sphinxVerbatimInput}

\end{sphinxuseclass}
\begin{sphinxuseclass}{cell}\begin{sphinxVerbatimInput}

\begin{sphinxuseclass}{cell_input}
\begin{sphinxVerbatim}[commandchars=\\\{\}]
\PYG{n}{date}\PYG{p}{(}\PYG{n}{ymd}\PYG{o}{=}\PYG{l+m+mi}{20080915}\PYG{p}{)}
\end{sphinxVerbatim}

\end{sphinxuseclass}\end{sphinxVerbatimInput}
\begin{sphinxVerbatimOutput}

\begin{sphinxuseclass}{cell_output}
\begin{sphinxVerbatim}[commandchars=\\\{\}]
(2008, 9, 15)
\end{sphinxVerbatim}

\end{sphinxuseclass}\end{sphinxVerbatimOutput}

\end{sphinxuseclass}

\subsubsection{Rewrite \sphinxstyleliteralintitle{\sphinxupquote{date}} to accept an 8\sphinxhyphen{}digit date as an integer or string.}
\label{\detokenize{mckinney_02_practice_04:rewrite-date-to-accept-an-8-digit-date-as-an-integer-or-string}}
\begin{sphinxuseclass}{cell}\begin{sphinxVerbatimInput}

\begin{sphinxuseclass}{cell_input}
\begin{sphinxVerbatim}[commandchars=\\\{\}]
\PYG{k}{def} \PYG{n+nf}{date}\PYG{p}{(}\PYG{n}{ymd}\PYG{p}{)}\PYG{p}{:}
    \PYG{k}{if} \PYG{n+nb}{type}\PYG{p}{(}\PYG{n}{ymd}\PYG{p}{)} \PYG{o+ow}{is} \PYG{n+nb}{str}\PYG{p}{:}
        \PYG{n}{ymd} \PYG{o}{=} \PYG{n+nb}{int}\PYG{p}{(}\PYG{n}{ymd}\PYG{p}{)}
    \PYG{n}{year} \PYG{o}{=} \PYG{n}{ymd} \PYG{o}{/}\PYG{o}{/} \PYG{l+m+mi}{10\PYGZus{}000}
    \PYG{n}{month} \PYG{o}{=} \PYG{p}{(}\PYG{n}{ymd} \PYG{o}{\PYGZpc{}} \PYG{l+m+mi}{10\PYGZus{}000}\PYG{p}{)} \PYG{o}{/}\PYG{o}{/} \PYG{l+m+mi}{100}
    \PYG{n}{day} \PYG{o}{=} \PYG{n}{ymd} \PYG{o}{\PYGZpc{}} \PYG{l+m+mi}{100}
    \PYG{k}{return} \PYG{p}{(}\PYG{n}{year}\PYG{p}{,} \PYG{n}{month}\PYG{p}{,} \PYG{n}{day}\PYG{p}{)}
\end{sphinxVerbatim}

\end{sphinxuseclass}\end{sphinxVerbatimInput}

\end{sphinxuseclass}
\begin{sphinxuseclass}{cell}\begin{sphinxVerbatimInput}

\begin{sphinxuseclass}{cell_input}
\begin{sphinxVerbatim}[commandchars=\\\{\}]
\PYG{n}{date}\PYG{p}{(}\PYG{n}{ymd}\PYG{o}{=}\PYG{l+s+s1}{\PYGZsq{}}\PYG{l+s+s1}{20080915}\PYG{l+s+s1}{\PYGZsq{}}\PYG{p}{)}
\end{sphinxVerbatim}

\end{sphinxuseclass}\end{sphinxVerbatimInput}
\begin{sphinxVerbatimOutput}

\begin{sphinxuseclass}{cell_output}
\begin{sphinxVerbatim}[commandchars=\\\{\}]
(2008, 9, 15)
\end{sphinxVerbatim}

\end{sphinxuseclass}\end{sphinxVerbatimOutput}

\end{sphinxuseclass}

\subsubsection{Finally, rewrite \sphinxstyleliteralintitle{\sphinxupquote{date}} to accept a list of 8\sphinxhyphen{}digit dates as integers or strings.}
\label{\detokenize{mckinney_02_practice_04:finally-rewrite-date-to-accept-a-list-of-8-digit-dates-as-integers-or-strings}}
\begin{sphinxuseclass}{cell}\begin{sphinxVerbatimInput}

\begin{sphinxuseclass}{cell_input}
\begin{sphinxVerbatim}[commandchars=\\\{\}]
\PYG{n}{ymds} \PYG{o}{=} \PYG{p}{[}\PYG{l+m+mi}{20080915}\PYG{p}{,} \PYG{l+s+s1}{\PYGZsq{}}\PYG{l+s+s1}{20080915}\PYG{l+s+s1}{\PYGZsq{}}\PYG{p}{]}
\end{sphinxVerbatim}

\end{sphinxuseclass}\end{sphinxVerbatimInput}

\end{sphinxuseclass}
\begin{sphinxuseclass}{cell}\begin{sphinxVerbatimInput}

\begin{sphinxuseclass}{cell_input}
\begin{sphinxVerbatim}[commandchars=\\\{\}]
\PYG{k}{def} \PYG{n+nf}{date}\PYG{p}{(}\PYG{n}{ymds}\PYG{p}{)}\PYG{p}{:}
    \PYG{n}{ymds\PYGZus{}out} \PYG{o}{=} \PYG{p}{[}\PYG{p}{]}
    \PYG{k}{if} \PYG{n+nb}{type}\PYG{p}{(}\PYG{n}{ymds}\PYG{p}{)} \PYG{o+ow}{is} \PYG{o+ow}{not} \PYG{n+nb}{list}\PYG{p}{:}
        \PYG{n}{ymds} \PYG{o}{=} \PYG{p}{[}\PYG{n}{ymds}\PYG{p}{]}
    \PYG{k}{for} \PYG{n}{ymd} \PYG{o+ow}{in} \PYG{n}{ymds}\PYG{p}{:}
        \PYG{k}{if} \PYG{n+nb}{type}\PYG{p}{(}\PYG{n}{ymd}\PYG{p}{)} \PYG{o+ow}{is} \PYG{n+nb}{str}\PYG{p}{:}
            \PYG{n}{ymd} \PYG{o}{=} \PYG{n+nb}{int}\PYG{p}{(}\PYG{n}{ymd}\PYG{p}{)}
        \PYG{n}{year} \PYG{o}{=} \PYG{n}{ymd} \PYG{o}{/}\PYG{o}{/} \PYG{l+m+mi}{10\PYGZus{}000}
        \PYG{n}{month} \PYG{o}{=} \PYG{p}{(}\PYG{n}{ymd} \PYG{o}{\PYGZpc{}} \PYG{l+m+mi}{10\PYGZus{}000}\PYG{p}{)} \PYG{o}{/}\PYG{o}{/} \PYG{l+m+mi}{100}
        \PYG{n}{day} \PYG{o}{=} \PYG{n}{ymd} \PYG{o}{\PYGZpc{}} \PYG{l+m+mi}{100}
        \PYG{n}{ymds\PYGZus{}out}\PYG{o}{.}\PYG{n}{append}\PYG{p}{(}\PYG{p}{(}\PYG{n}{year}\PYG{p}{,} \PYG{n}{month}\PYG{p}{,} \PYG{n}{day}\PYG{p}{)}\PYG{p}{)}
    \PYG{k}{return} \PYG{n}{ymds\PYGZus{}out}
\end{sphinxVerbatim}

\end{sphinxuseclass}\end{sphinxVerbatimInput}

\end{sphinxuseclass}
\begin{sphinxuseclass}{cell}\begin{sphinxVerbatimInput}

\begin{sphinxuseclass}{cell_input}
\begin{sphinxVerbatim}[commandchars=\\\{\}]
\PYG{n}{date}\PYG{p}{(}\PYG{n}{ymds}\PYG{p}{)}
\end{sphinxVerbatim}

\end{sphinxuseclass}\end{sphinxVerbatimInput}
\begin{sphinxVerbatimOutput}

\begin{sphinxuseclass}{cell_output}
\begin{sphinxVerbatim}[commandchars=\\\{\}]
[(2008, 9, 15), (2008, 9, 15)]
\end{sphinxVerbatim}

\end{sphinxuseclass}\end{sphinxVerbatimOutput}

\end{sphinxuseclass}
\begin{sphinxuseclass}{cell}\begin{sphinxVerbatimInput}

\begin{sphinxuseclass}{cell_input}
\begin{sphinxVerbatim}[commandchars=\\\{\}]
\PYG{n}{date}\PYG{p}{(}\PYG{l+m+mi}{20080915}\PYG{p}{)}
\end{sphinxVerbatim}

\end{sphinxuseclass}\end{sphinxVerbatimInput}
\begin{sphinxVerbatimOutput}

\begin{sphinxuseclass}{cell_output}
\begin{sphinxVerbatim}[commandchars=\\\{\}]
[(2008, 9, 15)]
\end{sphinxVerbatim}

\end{sphinxuseclass}\end{sphinxVerbatimOutput}

\end{sphinxuseclass}

\subsubsection{Write a for loop that prints the squares of integers from 1 to 10.}
\label{\detokenize{mckinney_02_practice_04:write-a-for-loop-that-prints-the-squares-of-integers-from-1-to-10}}
\begin{sphinxuseclass}{cell}\begin{sphinxVerbatimInput}

\begin{sphinxuseclass}{cell_input}
\begin{sphinxVerbatim}[commandchars=\\\{\}]
\PYG{k}{for} \PYG{n}{i} \PYG{o+ow}{in} \PYG{n+nb}{range}\PYG{p}{(}\PYG{l+m+mi}{1}\PYG{p}{,} \PYG{l+m+mi}{11}\PYG{p}{)}\PYG{p}{:}
    \PYG{n+nb}{print}\PYG{p}{(}\PYG{n}{i}\PYG{o}{*}\PYG{o}{*}\PYG{l+m+mi}{2}\PYG{p}{,} \PYG{n}{end}\PYG{o}{=}\PYG{l+s+s1}{\PYGZsq{}}\PYG{l+s+s1}{ }\PYG{l+s+s1}{\PYGZsq{}}\PYG{p}{)}
\end{sphinxVerbatim}

\end{sphinxuseclass}\end{sphinxVerbatimInput}
\begin{sphinxVerbatimOutput}

\begin{sphinxuseclass}{cell_output}
\begin{sphinxVerbatim}[commandchars=\\\{\}]
1 4 9 16 25 36 49 64 81 100 
\end{sphinxVerbatim}

\end{sphinxuseclass}\end{sphinxVerbatimOutput}

\end{sphinxuseclass}
\sphinxAtStartPar
Above, I change the \sphinxcode{\sphinxupquote{end}} argument from the default ‘\textbackslash{}n’ to ‘ ‘.
The default ‘\textbackslash{}n’ inserts a new line after value, making the output too long.


\subsubsection{Write a for loop that prints the squares of \sphinxstyleemphasis{even} integers from 1 to 10.}
\label{\detokenize{mckinney_02_practice_04:write-a-for-loop-that-prints-the-squares-of-even-integers-from-1-to-10}}
\begin{sphinxuseclass}{cell}\begin{sphinxVerbatimInput}

\begin{sphinxuseclass}{cell_input}
\begin{sphinxVerbatim}[commandchars=\\\{\}]
\PYG{k}{for} \PYG{n}{i} \PYG{o+ow}{in} \PYG{n+nb}{range}\PYG{p}{(}\PYG{l+m+mi}{1}\PYG{p}{,} \PYG{l+m+mi}{11}\PYG{p}{)}\PYG{p}{:}
    \PYG{k}{if} \PYG{n}{i} \PYG{o}{\PYGZpc{}} \PYG{l+m+mi}{2} \PYG{o}{==} \PYG{l+m+mi}{0}\PYG{p}{:}
        \PYG{n+nb}{print}\PYG{p}{(}\PYG{n}{i}\PYG{o}{*}\PYG{o}{*}\PYG{l+m+mi}{2}\PYG{p}{,} \PYG{n}{end}\PYG{o}{=}\PYG{l+s+s1}{\PYGZsq{}}\PYG{l+s+s1}{ }\PYG{l+s+s1}{\PYGZsq{}}\PYG{p}{)}
\end{sphinxVerbatim}

\end{sphinxuseclass}\end{sphinxVerbatimInput}
\begin{sphinxVerbatimOutput}

\begin{sphinxuseclass}{cell_output}
\begin{sphinxVerbatim}[commandchars=\\\{\}]
4 16 36 64 100 
\end{sphinxVerbatim}

\end{sphinxuseclass}\end{sphinxVerbatimOutput}

\end{sphinxuseclass}

\subsubsection{Write a for loop that sums the squares of integers from 1 to 10.}
\label{\detokenize{mckinney_02_practice_04:write-a-for-loop-that-sums-the-squares-of-integers-from-1-to-10}}
\begin{sphinxuseclass}{cell}\begin{sphinxVerbatimInput}

\begin{sphinxuseclass}{cell_input}
\begin{sphinxVerbatim}[commandchars=\\\{\}]
\PYG{n}{total} \PYG{o}{=} \PYG{l+m+mi}{0}
\PYG{k}{for} \PYG{n}{i} \PYG{o+ow}{in} \PYG{n+nb}{range}\PYG{p}{(}\PYG{l+m+mi}{1}\PYG{p}{,} \PYG{l+m+mi}{11}\PYG{p}{)}\PYG{p}{:}
    \PYG{n}{total} \PYG{o}{+}\PYG{o}{=} \PYG{n}{i}\PYG{o}{*}\PYG{o}{*}\PYG{l+m+mi}{2}
\end{sphinxVerbatim}

\end{sphinxuseclass}\end{sphinxVerbatimInput}

\end{sphinxuseclass}
\begin{sphinxuseclass}{cell}\begin{sphinxVerbatimInput}

\begin{sphinxuseclass}{cell_input}
\begin{sphinxVerbatim}[commandchars=\\\{\}]
\PYG{n}{total}
\end{sphinxVerbatim}

\end{sphinxuseclass}\end{sphinxVerbatimInput}
\begin{sphinxVerbatimOutput}

\begin{sphinxuseclass}{cell_output}
\begin{sphinxVerbatim}[commandchars=\\\{\}]
385
\end{sphinxVerbatim}

\end{sphinxuseclass}\end{sphinxVerbatimOutput}

\end{sphinxuseclass}
\sphinxAtStartPar
Above, I use \sphinxcode{\sphinxupquote{total += i}}, which is equivalent to \sphinxcode{\sphinxupquote{total = total + i}}.


\subsubsection{Write a for loop that sums the squares of integers from 1 to 10 but stops before the sum exceeds 50.}
\label{\detokenize{mckinney_02_practice_04:write-a-for-loop-that-sums-the-squares-of-integers-from-1-to-10-but-stops-before-the-sum-exceeds-50}}
\begin{sphinxuseclass}{cell}\begin{sphinxVerbatimInput}

\begin{sphinxuseclass}{cell_input}
\begin{sphinxVerbatim}[commandchars=\\\{\}]
\PYG{n}{total} \PYG{o}{=} \PYG{l+m+mi}{0}
\PYG{k}{for} \PYG{n}{i} \PYG{o+ow}{in} \PYG{n+nb}{range}\PYG{p}{(}\PYG{l+m+mi}{1}\PYG{p}{,} \PYG{l+m+mi}{11}\PYG{p}{)}\PYG{p}{:}
    \PYG{k}{if} \PYG{p}{(}\PYG{n}{total} \PYG{o}{+} \PYG{n}{i}\PYG{o}{*}\PYG{o}{*}\PYG{l+m+mi}{2}\PYG{p}{)} \PYG{o}{\PYGZgt{}} \PYG{l+m+mi}{50}\PYG{p}{:}
        \PYG{k}{break}
    \PYG{n}{total} \PYG{o}{+}\PYG{o}{=} \PYG{n}{i}\PYG{o}{*}\PYG{o}{*}\PYG{l+m+mi}{2}
\end{sphinxVerbatim}

\end{sphinxuseclass}\end{sphinxVerbatimInput}

\end{sphinxuseclass}
\begin{sphinxuseclass}{cell}\begin{sphinxVerbatimInput}

\begin{sphinxuseclass}{cell_input}
\begin{sphinxVerbatim}[commandchars=\\\{\}]
\PYG{n}{total}
\end{sphinxVerbatim}

\end{sphinxuseclass}\end{sphinxVerbatimInput}
\begin{sphinxVerbatimOutput}

\begin{sphinxuseclass}{cell_output}
\begin{sphinxVerbatim}[commandchars=\\\{\}]
30
\end{sphinxVerbatim}

\end{sphinxuseclass}\end{sphinxVerbatimOutput}

\end{sphinxuseclass}

\subsubsection{FizzBuzz}
\label{\detokenize{mckinney_02_practice_04:fizzbuzz}}
\sphinxAtStartPar
Write a for loop that prints the numbers from 1 to 100.
For multiples of three print “Fizz” instead of the number.
For multiples of five print “Buzz”.
For numbers that are multiples of both three and five print “FizzBuzz”.
More \sphinxhref{https://blog.codinghorror.com/why-cant-programmers-program/}{here}.

\begin{sphinxuseclass}{cell}\begin{sphinxVerbatimInput}

\begin{sphinxuseclass}{cell_input}
\begin{sphinxVerbatim}[commandchars=\\\{\}]
\PYG{l+m+mi}{4} \PYG{o}{\PYGZpc{}} \PYG{l+m+mi}{3} \PYG{o}{==} \PYG{l+m+mi}{0}
\end{sphinxVerbatim}

\end{sphinxuseclass}\end{sphinxVerbatimInput}
\begin{sphinxVerbatimOutput}

\begin{sphinxuseclass}{cell_output}
\begin{sphinxVerbatim}[commandchars=\\\{\}]
False
\end{sphinxVerbatim}

\end{sphinxuseclass}\end{sphinxVerbatimOutput}

\end{sphinxuseclass}
\begin{sphinxuseclass}{cell}\begin{sphinxVerbatimInput}

\begin{sphinxuseclass}{cell_input}
\begin{sphinxVerbatim}[commandchars=\\\{\}]
\PYG{k}{for} \PYG{n}{i} \PYG{o+ow}{in} \PYG{n+nb}{range}\PYG{p}{(}\PYG{l+m+mi}{1}\PYG{p}{,} \PYG{l+m+mi}{101}\PYG{p}{)}\PYG{p}{:}
    \PYG{n}{is\PYGZus{}mult\PYGZus{}3} \PYG{o}{=} \PYG{p}{(}\PYG{n}{i} \PYG{o}{\PYGZpc{}} \PYG{l+m+mi}{3} \PYG{o}{==} \PYG{l+m+mi}{0}\PYG{p}{)}
    \PYG{n}{is\PYGZus{}mult\PYGZus{}5} \PYG{o}{=} \PYG{p}{(}\PYG{n}{i} \PYG{o}{\PYGZpc{}} \PYG{l+m+mi}{5} \PYG{o}{==} \PYG{l+m+mi}{0}\PYG{p}{)}
    \PYG{k}{if} \PYG{n}{is\PYGZus{}mult\PYGZus{}3} \PYG{o+ow}{and} \PYG{n}{is\PYGZus{}mult\PYGZus{}5}\PYG{p}{:}
        \PYG{n+nb}{print}\PYG{p}{(}\PYG{l+s+s1}{\PYGZsq{}}\PYG{l+s+s1}{FizzBuzz}\PYG{l+s+s1}{\PYGZsq{}}\PYG{p}{,} \PYG{n}{end}\PYG{o}{=}\PYG{l+s+s1}{\PYGZsq{}}\PYG{l+s+s1}{ }\PYG{l+s+s1}{\PYGZsq{}}\PYG{p}{)}
    \PYG{k}{elif} \PYG{n}{is\PYGZus{}mult\PYGZus{}3}\PYG{p}{:}
        \PYG{n+nb}{print}\PYG{p}{(}\PYG{l+s+s1}{\PYGZsq{}}\PYG{l+s+s1}{Fizz}\PYG{l+s+s1}{\PYGZsq{}}\PYG{p}{,} \PYG{n}{end}\PYG{o}{=}\PYG{l+s+s1}{\PYGZsq{}}\PYG{l+s+s1}{ }\PYG{l+s+s1}{\PYGZsq{}}\PYG{p}{)}
    \PYG{k}{elif} \PYG{n}{is\PYGZus{}mult\PYGZus{}5}\PYG{p}{:}
        \PYG{n+nb}{print}\PYG{p}{(}\PYG{l+s+s1}{\PYGZsq{}}\PYG{l+s+s1}{Buzz}\PYG{l+s+s1}{\PYGZsq{}}\PYG{p}{,} \PYG{n}{end}\PYG{o}{=}\PYG{l+s+s1}{\PYGZsq{}}\PYG{l+s+s1}{ }\PYG{l+s+s1}{\PYGZsq{}}\PYG{p}{)}
    \PYG{k}{else}\PYG{p}{:}
        \PYG{n+nb}{print}\PYG{p}{(}\PYG{n}{i}\PYG{p}{,} \PYG{n}{end}\PYG{o}{=}\PYG{l+s+s1}{\PYGZsq{}}\PYG{l+s+s1}{ }\PYG{l+s+s1}{\PYGZsq{}}\PYG{p}{)}
\end{sphinxVerbatim}

\end{sphinxuseclass}\end{sphinxVerbatimInput}
\begin{sphinxVerbatimOutput}

\begin{sphinxuseclass}{cell_output}
\begin{sphinxVerbatim}[commandchars=\\\{\}]
1 2 Fizz 4 Buzz Fizz 7 8 Fizz Buzz 11 Fizz 13 14 FizzBuzz 16 17 Fizz 19 Buzz Fizz 22 23 Fizz Buzz 26 Fizz 28 29 FizzBuzz 31 32 Fizz 34 Buzz Fizz 37 38 Fizz Buzz 41 Fizz 43 44 FizzBuzz 46 47 Fizz 49 Buzz Fizz 52 53 Fizz Buzz 56 Fizz 58 59 FizzBuzz 61 62 Fizz 64 Buzz Fizz 67 68 Fizz Buzz 71 Fizz 73 74 FizzBuzz 76 77 Fizz 79 Buzz Fizz 82 83 Fizz Buzz 86 Fizz 88 89 FizzBuzz 91 92 Fizz 94 Buzz Fizz 97 98 Fizz Buzz 
\end{sphinxVerbatim}

\end{sphinxuseclass}\end{sphinxVerbatimOutput}

\end{sphinxuseclass}

\subsubsection{Use ternary expressions to make your FizzBuzz code more compact.}
\label{\detokenize{mckinney_02_practice_04:use-ternary-expressions-to-make-your-fizzbuzz-code-more-compact}}
\begin{sphinxuseclass}{cell}\begin{sphinxVerbatimInput}

\begin{sphinxuseclass}{cell_input}
\begin{sphinxVerbatim}[commandchars=\\\{\}]
\PYG{k}{for} \PYG{n}{i} \PYG{o+ow}{in} \PYG{n+nb}{range}\PYG{p}{(}\PYG{l+m+mi}{1}\PYG{p}{,} \PYG{l+m+mi}{101}\PYG{p}{)}\PYG{p}{:}
    \PYG{n}{is\PYGZus{}mult\PYGZus{}3} \PYG{o}{=} \PYG{p}{(}\PYG{n}{i} \PYG{o}{\PYGZpc{}} \PYG{l+m+mi}{3} \PYG{o}{==} \PYG{l+m+mi}{0}\PYG{p}{)}
    \PYG{n}{is\PYGZus{}mult\PYGZus{}5} \PYG{o}{=} \PYG{p}{(}\PYG{n}{i} \PYG{o}{\PYGZpc{}} \PYG{l+m+mi}{5} \PYG{o}{==} \PYG{l+m+mi}{0}\PYG{p}{)}
    \PYG{n+nb}{print}\PYG{p}{(}\PYG{l+s+s1}{\PYGZsq{}}\PYG{l+s+s1}{Fizz}\PYG{l+s+s1}{\PYGZsq{}}\PYG{o}{*}\PYG{n}{is\PYGZus{}mult\PYGZus{}3} \PYG{o}{+} \PYG{l+s+s1}{\PYGZsq{}}\PYG{l+s+s1}{Buzz}\PYG{l+s+s1}{\PYGZsq{}}\PYG{o}{*}\PYG{n}{is\PYGZus{}mult\PYGZus{}5} \PYG{k}{if} \PYG{n}{is\PYGZus{}mult\PYGZus{}3} \PYG{o+ow}{or} \PYG{n}{is\PYGZus{}mult\PYGZus{}5} \PYG{k}{else} \PYG{n}{i}\PYG{p}{,} \PYG{n}{end}\PYG{o}{=}\PYG{l+s+s1}{\PYGZsq{}}\PYG{l+s+s1}{ }\PYG{l+s+s1}{\PYGZsq{}}\PYG{p}{)}
\end{sphinxVerbatim}

\end{sphinxuseclass}\end{sphinxVerbatimInput}
\begin{sphinxVerbatimOutput}

\begin{sphinxuseclass}{cell_output}
\begin{sphinxVerbatim}[commandchars=\\\{\}]
1 2 Fizz 4 Buzz Fizz 7 8 Fizz Buzz 11 Fizz 13 14 FizzBuzz 16 17 Fizz 19 Buzz Fizz 22 23 Fizz Buzz 26 Fizz 28 29 FizzBuzz 31 32 Fizz 34 Buzz Fizz 37 38 Fizz Buzz 41 Fizz 43 44 FizzBuzz 46 47 Fizz 49 Buzz Fizz 52 53 Fizz Buzz 56 Fizz 58 59 FizzBuzz 61 62 Fizz 64 Buzz Fizz 67 68 Fizz Buzz 71 Fizz 73 74 FizzBuzz 76 77 Fizz 79 Buzz Fizz 82 83 Fizz Buzz 86 Fizz 88 89 FizzBuzz 91 92 Fizz 94 Buzz Fizz 97 98 Fizz Buzz 
\end{sphinxVerbatim}

\end{sphinxuseclass}\end{sphinxVerbatimOutput}

\end{sphinxuseclass}
\sphinxAtStartPar
The solution above is shorter and uses from neat tricks, but I consider the previous solution easier to read and troubleshoot.


\subsubsection{Triangle}
\label{\detokenize{mckinney_02_practice_04:triangle}}
\sphinxAtStartPar
Write a function \sphinxcode{\sphinxupquote{triangle}} that accepts a positive integer \$N\$ and prints a numerical triangle of height \$N\sphinxhyphen{}1\$.
For example, \sphinxcode{\sphinxupquote{triangle(N=6)}} should print:

\begin{sphinxVerbatim}[commandchars=\\\{\}]
\PYG{l+m+mi}{1}
\PYG{l+m+mi}{22}
\PYG{l+m+mi}{333}
\PYG{l+m+mi}{4444}
\PYG{l+m+mi}{55555}
\end{sphinxVerbatim}

\begin{sphinxuseclass}{cell}\begin{sphinxVerbatimInput}

\begin{sphinxuseclass}{cell_input}
\begin{sphinxVerbatim}[commandchars=\\\{\}]
\PYG{k}{def} \PYG{n+nf}{triangle}\PYG{p}{(}\PYG{n}{N}\PYG{p}{)}\PYG{p}{:}
    \PYG{k}{for} \PYG{n}{i} \PYG{o+ow}{in} \PYG{n+nb}{range}\PYG{p}{(}\PYG{l+m+mi}{1}\PYG{p}{,} \PYG{n}{N}\PYG{p}{)}\PYG{p}{:}
        \PYG{n+nb}{print}\PYG{p}{(}\PYG{n+nb}{str}\PYG{p}{(}\PYG{n}{i}\PYG{p}{)} \PYG{o}{*} \PYG{n}{i}\PYG{p}{)}
\end{sphinxVerbatim}

\end{sphinxuseclass}\end{sphinxVerbatimInput}

\end{sphinxuseclass}
\begin{sphinxuseclass}{cell}\begin{sphinxVerbatimInput}

\begin{sphinxuseclass}{cell_input}
\begin{sphinxVerbatim}[commandchars=\\\{\}]
\PYG{n}{triangle}\PYG{p}{(}\PYG{l+m+mi}{6}\PYG{p}{)}
\end{sphinxVerbatim}

\end{sphinxuseclass}\end{sphinxVerbatimInput}
\begin{sphinxVerbatimOutput}

\begin{sphinxuseclass}{cell_output}
\begin{sphinxVerbatim}[commandchars=\\\{\}]
1
22
333
4444
55555
\end{sphinxVerbatim}

\end{sphinxuseclass}\end{sphinxVerbatimOutput}

\end{sphinxuseclass}
\sphinxAtStartPar
The solution above works because a multiplying a string by \sphinxcode{\sphinxupquote{i}} concatenates \sphinxcode{\sphinxupquote{i}} copies of the string.

\begin{sphinxuseclass}{cell}\begin{sphinxVerbatimInput}

\begin{sphinxuseclass}{cell_input}
\begin{sphinxVerbatim}[commandchars=\\\{\}]
\PYG{l+s+s1}{\PYGZsq{}}\PYG{l+s+s1}{Test}\PYG{l+s+s1}{\PYGZsq{}} \PYG{o}{+} \PYG{l+s+s1}{\PYGZsq{}}\PYG{l+s+s1}{Test}\PYG{l+s+s1}{\PYGZsq{}} \PYG{o}{+} \PYG{l+s+s1}{\PYGZsq{}}\PYG{l+s+s1}{Test}\PYG{l+s+s1}{\PYGZsq{}}
\end{sphinxVerbatim}

\end{sphinxuseclass}\end{sphinxVerbatimInput}
\begin{sphinxVerbatimOutput}

\begin{sphinxuseclass}{cell_output}
\begin{sphinxVerbatim}[commandchars=\\\{\}]
\PYGZsq{}TestTestTest\PYGZsq{}
\end{sphinxVerbatim}

\end{sphinxuseclass}\end{sphinxVerbatimOutput}

\end{sphinxuseclass}
\begin{sphinxuseclass}{cell}\begin{sphinxVerbatimInput}

\begin{sphinxuseclass}{cell_input}
\begin{sphinxVerbatim}[commandchars=\\\{\}]
\PYG{l+s+s1}{\PYGZsq{}}\PYG{l+s+s1}{Test}\PYG{l+s+s1}{\PYGZsq{}} \PYG{o}{*} \PYG{l+m+mi}{3}
\end{sphinxVerbatim}

\end{sphinxuseclass}\end{sphinxVerbatimInput}
\begin{sphinxVerbatimOutput}

\begin{sphinxuseclass}{cell_output}
\begin{sphinxVerbatim}[commandchars=\\\{\}]
\PYGZsq{}TestTestTest\PYGZsq{}
\end{sphinxVerbatim}

\end{sphinxuseclass}\end{sphinxVerbatimOutput}

\end{sphinxuseclass}

\subsubsection{Two Sum}
\label{\detokenize{mckinney_02_practice_04:two-sum}}
\sphinxAtStartPar
Write a function \sphinxcode{\sphinxupquote{two\_sum}} that does the following.

\sphinxAtStartPar
Given an array of integers \sphinxcode{\sphinxupquote{nums}} and an integer \sphinxcode{\sphinxupquote{target}}, return the indices of the two numbers that add up to target.

\sphinxAtStartPar
You may assume that each input would have exactly one solution, and you may not use the same element twice.

\sphinxAtStartPar
You can return the answer in any order.

\sphinxAtStartPar
Here are some examples:

\sphinxAtStartPar
Example 1:

\sphinxAtStartPar
Input: \sphinxcode{\sphinxupquote{nums = {[}2,7,11,15{]}}}, \sphinxcode{\sphinxupquote{target = 9}} \\
Output: \sphinxcode{\sphinxupquote{{[}0,1{]}}} \\
Explanation: Because \sphinxcode{\sphinxupquote{nums{[}0{]} + nums{[}1{]} == 9}}, we return \sphinxcode{\sphinxupquote{{[}0, 1{]}}}.

\sphinxAtStartPar
Example 2:

\sphinxAtStartPar
Input: \sphinxcode{\sphinxupquote{nums = {[}3,2,4{]}}}, \sphinxcode{\sphinxupquote{target = 6}} \\
Output: \sphinxcode{\sphinxupquote{{[}1,2{]}}} \textbackslash{}

\sphinxAtStartPar
Example 3:

\sphinxAtStartPar
Input: \sphinxcode{\sphinxupquote{nums = {[}3,3{]}}}, \sphinxcode{\sphinxupquote{target = 6}} \\
Output: \sphinxcode{\sphinxupquote{{[}0,1{]}}} \textbackslash{}

\sphinxAtStartPar
I saw this question on \sphinxhref{https://leetcode.com/problems/two-sum/description/}{LeetCode}.

\begin{sphinxuseclass}{cell}\begin{sphinxVerbatimInput}

\begin{sphinxuseclass}{cell_input}
\begin{sphinxVerbatim}[commandchars=\\\{\}]
\PYG{k}{def} \PYG{n+nf}{two\PYGZus{}sum}\PYG{p}{(}\PYG{n}{nums}\PYG{p}{,} \PYG{n}{target}\PYG{p}{)}\PYG{p}{:}
    \PYG{k}{for} \PYG{n}{i} \PYG{o+ow}{in} \PYG{n+nb}{range}\PYG{p}{(}\PYG{l+m+mi}{1}\PYG{p}{,} \PYG{n+nb}{len}\PYG{p}{(}\PYG{n}{nums}\PYG{p}{)}\PYG{p}{)}\PYG{p}{:}
        \PYG{k}{for} \PYG{n}{j} \PYG{o+ow}{in} \PYG{n+nb}{range}\PYG{p}{(}\PYG{n}{i}\PYG{p}{)}\PYG{p}{:}
            \PYG{k}{if} \PYG{n}{nums}\PYG{p}{[}\PYG{n}{i}\PYG{p}{]} \PYG{o}{+} \PYG{n}{nums}\PYG{p}{[}\PYG{n}{j}\PYG{p}{]} \PYG{o}{==} \PYG{n}{target}\PYG{p}{:}
                \PYG{k}{return} \PYG{p}{[}\PYG{n}{j}\PYG{p}{,} \PYG{n}{i}\PYG{p}{]}
\end{sphinxVerbatim}

\end{sphinxuseclass}\end{sphinxVerbatimInput}

\end{sphinxuseclass}
\begin{sphinxuseclass}{cell}\begin{sphinxVerbatimInput}

\begin{sphinxuseclass}{cell_input}
\begin{sphinxVerbatim}[commandchars=\\\{\}]
\PYG{n}{two\PYGZus{}sum}\PYG{p}{(}\PYG{n}{nums} \PYG{o}{=} \PYG{p}{[}\PYG{l+m+mi}{2}\PYG{p}{,}\PYG{l+m+mi}{7}\PYG{p}{,}\PYG{l+m+mi}{11}\PYG{p}{,}\PYG{l+m+mi}{15}\PYG{p}{]}\PYG{p}{,} \PYG{n}{target} \PYG{o}{=} \PYG{l+m+mi}{9}\PYG{p}{)}
\end{sphinxVerbatim}

\end{sphinxuseclass}\end{sphinxVerbatimInput}
\begin{sphinxVerbatimOutput}

\begin{sphinxuseclass}{cell_output}
\begin{sphinxVerbatim}[commandchars=\\\{\}]
[0, 1]
\end{sphinxVerbatim}

\end{sphinxuseclass}\end{sphinxVerbatimOutput}

\end{sphinxuseclass}
\begin{sphinxuseclass}{cell}\begin{sphinxVerbatimInput}

\begin{sphinxuseclass}{cell_input}
\begin{sphinxVerbatim}[commandchars=\\\{\}]
\PYG{n}{two\PYGZus{}sum}\PYG{p}{(}\PYG{n}{nums} \PYG{o}{=} \PYG{p}{[}\PYG{l+m+mi}{3}\PYG{p}{,}\PYG{l+m+mi}{2}\PYG{p}{,}\PYG{l+m+mi}{4}\PYG{p}{]}\PYG{p}{,} \PYG{n}{target} \PYG{o}{=} \PYG{l+m+mi}{6}\PYG{p}{)}
\end{sphinxVerbatim}

\end{sphinxuseclass}\end{sphinxVerbatimInput}
\begin{sphinxVerbatimOutput}

\begin{sphinxuseclass}{cell_output}
\begin{sphinxVerbatim}[commandchars=\\\{\}]
[1, 2]
\end{sphinxVerbatim}

\end{sphinxuseclass}\end{sphinxVerbatimOutput}

\end{sphinxuseclass}
\begin{sphinxuseclass}{cell}\begin{sphinxVerbatimInput}

\begin{sphinxuseclass}{cell_input}
\begin{sphinxVerbatim}[commandchars=\\\{\}]
\PYG{n}{two\PYGZus{}sum}\PYG{p}{(}\PYG{n}{nums} \PYG{o}{=} \PYG{p}{[}\PYG{l+m+mi}{3}\PYG{p}{,}\PYG{l+m+mi}{3}\PYG{p}{]}\PYG{p}{,} \PYG{n}{target} \PYG{o}{=} \PYG{l+m+mi}{6}\PYG{p}{)}
\end{sphinxVerbatim}

\end{sphinxuseclass}\end{sphinxVerbatimInput}
\begin{sphinxVerbatimOutput}

\begin{sphinxuseclass}{cell_output}
\begin{sphinxVerbatim}[commandchars=\\\{\}]
[0, 1]
\end{sphinxVerbatim}

\end{sphinxuseclass}\end{sphinxVerbatimOutput}

\end{sphinxuseclass}
\sphinxAtStartPar
We can write more efficient code once we learn other data structures in chapter 3 of McKinney!


\subsubsection{Best Time}
\label{\detokenize{mckinney_02_practice_04:best-time}}
\sphinxAtStartPar
Write a function \sphinxcode{\sphinxupquote{best\_time}} that solves the following.

\sphinxAtStartPar
You are given an array \sphinxcode{\sphinxupquote{prices}} where \sphinxcode{\sphinxupquote{prices{[}i{]}}} is the price of a given stock on the \$i\textasciicircum{}\{th\}\$ day.

\sphinxAtStartPar
You want to maximize your profit by choosing a single day to buy one stock and choosing a different day in the future to sell that stock.

\sphinxAtStartPar
Return the maximum profit you can achieve from this transaction. If you cannot achieve any profit, return 0.

\sphinxAtStartPar
Here are some examples:

\sphinxAtStartPar
Example 1:

\sphinxAtStartPar
Input: \sphinxcode{\sphinxupquote{prices = {[}7,1,5,3,6,4{]}}} \\
Output: \sphinxcode{\sphinxupquote{5}} \\
Explanation: Buy on day 2 (price = 1) and sell on day 5 (price = 6), profit = 6\sphinxhyphen{}1 = 5.
Note that buying on day 2 and selling on day 1 is not allowed because you must buy before you sell.

\sphinxAtStartPar
Example 2:

\sphinxAtStartPar
Input: \sphinxcode{\sphinxupquote{prices = {[}7,6,4,3,1{]}}} \\
Output: \sphinxcode{\sphinxupquote{0}} \\
Explanation: In this case, no transactions are done and the max profit = 0.

\sphinxAtStartPar
I saw this question on \sphinxhref{https://leetcode.com/problems/best-time-to-buy-and-sell-stock/}{LeetCode}.

\begin{sphinxuseclass}{cell}\begin{sphinxVerbatimInput}

\begin{sphinxuseclass}{cell_input}
\begin{sphinxVerbatim}[commandchars=\\\{\}]
\PYG{k}{def} \PYG{n+nf}{max\PYGZus{}profit}\PYG{p}{(}\PYG{n}{prices}\PYG{p}{)}\PYG{p}{:}
        \PYG{n}{min\PYGZus{}price} \PYG{o}{=} \PYG{n}{prices}\PYG{p}{[}\PYG{l+m+mi}{0}\PYG{p}{]}
        \PYG{n}{max\PYGZus{}profit} \PYG{o}{=} \PYG{l+m+mi}{0}
        \PYG{k}{for} \PYG{n}{price} \PYG{o+ow}{in} \PYG{n}{prices}\PYG{p}{:}
            \PYG{n}{min\PYGZus{}price} \PYG{o}{=} \PYG{n}{price} \PYG{k}{if} \PYG{n}{price} \PYG{o}{\PYGZlt{}} \PYG{n}{min\PYGZus{}price} \PYG{k}{else} \PYG{n}{min\PYGZus{}price}
            \PYG{n}{profit} \PYG{o}{=} \PYG{n}{price} \PYG{o}{\PYGZhy{}} \PYG{n}{min\PYGZus{}price}
            \PYG{n}{max\PYGZus{}profit} \PYG{o}{=} \PYG{n}{profit} \PYG{k}{if} \PYG{n}{profit} \PYG{o}{\PYGZgt{}} \PYG{n}{max\PYGZus{}profit} \PYG{k}{else} \PYG{n}{max\PYGZus{}profit}
        \PYG{k}{return} \PYG{n}{max\PYGZus{}profit}
\end{sphinxVerbatim}

\end{sphinxuseclass}\end{sphinxVerbatimInput}

\end{sphinxuseclass}
\begin{sphinxuseclass}{cell}\begin{sphinxVerbatimInput}

\begin{sphinxuseclass}{cell_input}
\begin{sphinxVerbatim}[commandchars=\\\{\}]
\PYG{n}{max\PYGZus{}profit}\PYG{p}{(}\PYG{n}{prices}\PYG{o}{=}\PYG{p}{[}\PYG{l+m+mi}{7}\PYG{p}{,}\PYG{l+m+mi}{1}\PYG{p}{,}\PYG{l+m+mi}{5}\PYG{p}{,}\PYG{l+m+mi}{3}\PYG{p}{,}\PYG{l+m+mi}{6}\PYG{p}{,}\PYG{l+m+mi}{4}\PYG{p}{]}\PYG{p}{)}
\end{sphinxVerbatim}

\end{sphinxuseclass}\end{sphinxVerbatimInput}
\begin{sphinxVerbatimOutput}

\begin{sphinxuseclass}{cell_output}
\begin{sphinxVerbatim}[commandchars=\\\{\}]
5
\end{sphinxVerbatim}

\end{sphinxuseclass}\end{sphinxVerbatimOutput}

\end{sphinxuseclass}
\begin{sphinxuseclass}{cell}\begin{sphinxVerbatimInput}

\begin{sphinxuseclass}{cell_input}
\begin{sphinxVerbatim}[commandchars=\\\{\}]
\PYG{n}{max\PYGZus{}profit}\PYG{p}{(}\PYG{n}{prices}\PYG{o}{=}\PYG{p}{[}\PYG{l+m+mi}{7}\PYG{p}{,}\PYG{l+m+mi}{6}\PYG{p}{,}\PYG{l+m+mi}{4}\PYG{p}{,}\PYG{l+m+mi}{3}\PYG{p}{,}\PYG{l+m+mi}{1}\PYG{p}{]}\PYG{p}{)}
\end{sphinxVerbatim}

\end{sphinxuseclass}\end{sphinxVerbatimInput}
\begin{sphinxVerbatimOutput}

\begin{sphinxuseclass}{cell_output}
\begin{sphinxVerbatim}[commandchars=\\\{\}]
0
\end{sphinxVerbatim}

\end{sphinxuseclass}\end{sphinxVerbatimOutput}

\end{sphinxuseclass}
\sphinxstepscope


\section{McKinney Chapter 2 \sphinxhyphen{} Practice (Section 2, Wednesday 2:45 PM)}
\label{\detokenize{mckinney_02_practice_02:mckinney-chapter-2-practice-section-2-wednesday-2-45-pm}}\label{\detokenize{mckinney_02_practice_02::doc}}

\subsection{Practice}
\label{\detokenize{mckinney_02_practice_02:practice}}

\subsubsection{Extract the year, month, and day from an integer 8\sphinxhyphen{}digit date (i.e., YYYYMMDD format) using \sphinxstyleliteralintitle{\sphinxupquote{//}} (integer division) and \sphinxstyleliteralintitle{\sphinxupquote{\%}} (modulo division).}
\label{\detokenize{mckinney_02_practice_02:extract-the-year-month-and-day-from-an-integer-8-digit-date-i-e-yyyymmdd-format-using-integer-division-and-modulo-division}}
\sphinxAtStartPar
Try \sphinxcode{\sphinxupquote{20080915}}.

\begin{sphinxuseclass}{cell}\begin{sphinxVerbatimInput}

\begin{sphinxuseclass}{cell_input}
\begin{sphinxVerbatim}[commandchars=\\\{\}]
\PYG{l+m+mi}{1234} \PYG{o}{/} \PYG{l+m+mi}{100}
\end{sphinxVerbatim}

\end{sphinxuseclass}\end{sphinxVerbatimInput}
\begin{sphinxVerbatimOutput}

\begin{sphinxuseclass}{cell_output}
\begin{sphinxVerbatim}[commandchars=\\\{\}]
12.34
\end{sphinxVerbatim}

\end{sphinxuseclass}\end{sphinxVerbatimOutput}

\end{sphinxuseclass}
\begin{sphinxuseclass}{cell}\begin{sphinxVerbatimInput}

\begin{sphinxuseclass}{cell_input}
\begin{sphinxVerbatim}[commandchars=\\\{\}]
\PYG{l+m+mi}{1234} \PYG{o}{/}\PYG{o}{/} \PYG{l+m+mi}{100} \PYG{c+c1}{\PYGZsh{} // is integer division}
\end{sphinxVerbatim}

\end{sphinxuseclass}\end{sphinxVerbatimInput}
\begin{sphinxVerbatimOutput}

\begin{sphinxuseclass}{cell_output}
\begin{sphinxVerbatim}[commandchars=\\\{\}]
12
\end{sphinxVerbatim}

\end{sphinxuseclass}\end{sphinxVerbatimOutput}

\end{sphinxuseclass}
\begin{sphinxuseclass}{cell}\begin{sphinxVerbatimInput}

\begin{sphinxuseclass}{cell_input}
\begin{sphinxVerbatim}[commandchars=\\\{\}]
\PYG{l+m+mi}{1234} \PYG{o}{\PYGZpc{}} \PYG{l+m+mi}{100} \PYG{c+c1}{\PYGZsh{} \PYGZpc{} is modulo (or remainder) division}
\end{sphinxVerbatim}

\end{sphinxuseclass}\end{sphinxVerbatimInput}
\begin{sphinxVerbatimOutput}

\begin{sphinxuseclass}{cell_output}
\begin{sphinxVerbatim}[commandchars=\\\{\}]
34
\end{sphinxVerbatim}

\end{sphinxuseclass}\end{sphinxVerbatimOutput}

\end{sphinxuseclass}
\begin{sphinxuseclass}{cell}\begin{sphinxVerbatimInput}

\begin{sphinxuseclass}{cell_input}
\begin{sphinxVerbatim}[commandchars=\\\{\}]
\PYG{n}{ymd} \PYG{o}{=} \PYG{l+m+mi}{20080915}
\end{sphinxVerbatim}

\end{sphinxuseclass}\end{sphinxVerbatimInput}

\end{sphinxuseclass}
\begin{sphinxuseclass}{cell}\begin{sphinxVerbatimInput}

\begin{sphinxuseclass}{cell_input}
\begin{sphinxVerbatim}[commandchars=\\\{\}]
\PYG{n}{year} \PYG{o}{=} \PYG{n}{ymd} \PYG{o}{/}\PYG{o}{/} \PYG{l+m+mi}{10\PYGZus{}000}
\end{sphinxVerbatim}

\end{sphinxuseclass}\end{sphinxVerbatimInput}

\end{sphinxuseclass}
\begin{sphinxuseclass}{cell}\begin{sphinxVerbatimInput}

\begin{sphinxuseclass}{cell_input}
\begin{sphinxVerbatim}[commandchars=\\\{\}]
\PYG{n}{month} \PYG{o}{=} \PYG{p}{(}\PYG{n}{ymd} \PYG{o}{\PYGZpc{}} \PYG{l+m+mi}{10\PYGZus{}000}\PYG{p}{)} \PYG{o}{/}\PYG{o}{/} \PYG{l+m+mi}{100}
\end{sphinxVerbatim}

\end{sphinxuseclass}\end{sphinxVerbatimInput}

\end{sphinxuseclass}
\begin{sphinxuseclass}{cell}\begin{sphinxVerbatimInput}

\begin{sphinxuseclass}{cell_input}
\begin{sphinxVerbatim}[commandchars=\\\{\}]
\PYG{n}{day} \PYG{o}{=} \PYG{n}{ymd} \PYG{o}{\PYGZpc{}} \PYG{l+m+mi}{100}
\end{sphinxVerbatim}

\end{sphinxuseclass}\end{sphinxVerbatimInput}

\end{sphinxuseclass}
\begin{sphinxuseclass}{cell}\begin{sphinxVerbatimInput}

\begin{sphinxuseclass}{cell_input}
\begin{sphinxVerbatim}[commandchars=\\\{\}]
\PYG{p}{(}\PYG{n}{year}\PYG{p}{,} \PYG{n}{month}\PYG{p}{,} \PYG{n}{day}\PYG{p}{)}
\end{sphinxVerbatim}

\end{sphinxuseclass}\end{sphinxVerbatimInput}
\begin{sphinxVerbatimOutput}

\begin{sphinxuseclass}{cell_output}
\begin{sphinxVerbatim}[commandchars=\\\{\}]
(2008, 9, 15)
\end{sphinxVerbatim}

\end{sphinxuseclass}\end{sphinxVerbatimOutput}

\end{sphinxuseclass}
\begin{sphinxuseclass}{cell}\begin{sphinxVerbatimInput}

\begin{sphinxuseclass}{cell_input}
\begin{sphinxVerbatim}[commandchars=\\\{\}]
\PYG{n+nb}{print}\PYG{p}{(}\PYG{n}{year}\PYG{p}{,} \PYG{n}{month}\PYG{p}{,} \PYG{n}{day}\PYG{p}{)}
\end{sphinxVerbatim}

\end{sphinxuseclass}\end{sphinxVerbatimInput}
\begin{sphinxVerbatimOutput}

\begin{sphinxuseclass}{cell_output}
\begin{sphinxVerbatim}[commandchars=\\\{\}]
2008 9 15
\end{sphinxVerbatim}

\end{sphinxuseclass}\end{sphinxVerbatimOutput}

\end{sphinxuseclass}
\begin{sphinxuseclass}{cell}\begin{sphinxVerbatimInput}

\begin{sphinxuseclass}{cell_input}
\begin{sphinxVerbatim}[commandchars=\\\{\}]
\PYG{n+nb}{print}\PYG{p}{(}\PYG{n}{year}\PYG{p}{,} \PYG{n}{month}\PYG{p}{,} \PYG{n}{day}\PYG{p}{,} \PYG{n}{sep}\PYG{o}{=}\PYG{l+s+s1}{\PYGZsq{}}\PYG{l+s+s1}{/}\PYG{l+s+s1}{\PYGZsq{}}\PYG{p}{)}
\end{sphinxVerbatim}

\end{sphinxuseclass}\end{sphinxVerbatimInput}
\begin{sphinxVerbatimOutput}

\begin{sphinxuseclass}{cell_output}
\begin{sphinxVerbatim}[commandchars=\\\{\}]
2008/9/15
\end{sphinxVerbatim}

\end{sphinxuseclass}\end{sphinxVerbatimOutput}

\end{sphinxuseclass}
\begin{sphinxuseclass}{cell}\begin{sphinxVerbatimInput}

\begin{sphinxuseclass}{cell_input}
\begin{sphinxVerbatim}[commandchars=\\\{\}]
\PYG{n+nb}{print}\PYG{p}{(}\PYG{l+s+sa}{f}\PYG{l+s+s1}{\PYGZsq{}}\PYG{l+s+s1}{year is }\PYG{l+s+si}{\PYGZob{}}\PYG{n}{year}\PYG{l+s+si}{\PYGZcb{}}\PYG{l+s+s1}{, month is }\PYG{l+s+si}{\PYGZob{}}\PYG{n}{month}\PYG{l+s+si}{\PYGZcb{}}\PYG{l+s+s1}{, day is }\PYG{l+s+si}{\PYGZob{}}\PYG{n}{day}\PYG{l+s+si}{\PYGZcb{}}\PYG{l+s+s1}{\PYGZsq{}}\PYG{p}{)}
\end{sphinxVerbatim}

\end{sphinxuseclass}\end{sphinxVerbatimInput}
\begin{sphinxVerbatimOutput}

\begin{sphinxuseclass}{cell_output}
\begin{sphinxVerbatim}[commandchars=\\\{\}]
year is 2008, month is 9, day is 15
\end{sphinxVerbatim}

\end{sphinxuseclass}\end{sphinxVerbatimOutput}

\end{sphinxuseclass}

\subsubsection{Use your answer above to write a function \sphinxstyleliteralintitle{\sphinxupquote{date}} that accepts an integer 8\sphinxhyphen{}digit date argument and returns a tuple of the year, month, and date (e.g., \sphinxstyleliteralintitle{\sphinxupquote{return (year, month, date)}}).}
\label{\detokenize{mckinney_02_practice_02:use-your-answer-above-to-write-a-function-date-that-accepts-an-integer-8-digit-date-argument-and-returns-a-tuple-of-the-year-month-and-date-e-g-return-year-month-date}}
\begin{sphinxuseclass}{cell}\begin{sphinxVerbatimInput}

\begin{sphinxuseclass}{cell_input}
\begin{sphinxVerbatim}[commandchars=\\\{\}]
\PYG{k}{def} \PYG{n+nf}{date}\PYG{p}{(}\PYG{n}{ymd}\PYG{p}{)}\PYG{p}{:}
    \PYG{n}{year} \PYG{o}{=} \PYG{n}{ymd} \PYG{o}{/}\PYG{o}{/} \PYG{l+m+mi}{10\PYGZus{}000}
    \PYG{n}{month} \PYG{o}{=} \PYG{p}{(}\PYG{n}{ymd} \PYG{o}{\PYGZpc{}} \PYG{l+m+mi}{10\PYGZus{}000}\PYG{p}{)} \PYG{o}{/}\PYG{o}{/} \PYG{l+m+mi}{100}
    \PYG{n}{day} \PYG{o}{=} \PYG{n}{ymd} \PYG{o}{\PYGZpc{}} \PYG{l+m+mi}{100}
    \PYG{k}{return} \PYG{p}{(}\PYG{n}{year}\PYG{p}{,} \PYG{n}{month}\PYG{p}{,} \PYG{n}{day}\PYG{p}{)}
\end{sphinxVerbatim}

\end{sphinxuseclass}\end{sphinxVerbatimInput}

\end{sphinxuseclass}
\begin{sphinxuseclass}{cell}\begin{sphinxVerbatimInput}

\begin{sphinxuseclass}{cell_input}
\begin{sphinxVerbatim}[commandchars=\\\{\}]
\PYG{n}{parsed\PYGZus{}date} \PYG{o}{=} \PYG{n}{date}\PYG{p}{(}\PYG{l+m+mi}{20080915}\PYG{p}{)}
\end{sphinxVerbatim}

\end{sphinxuseclass}\end{sphinxVerbatimInput}

\end{sphinxuseclass}
\begin{sphinxuseclass}{cell}\begin{sphinxVerbatimInput}

\begin{sphinxuseclass}{cell_input}
\begin{sphinxVerbatim}[commandchars=\\\{\}]
\PYG{n}{parsed\PYGZus{}date}
\end{sphinxVerbatim}

\end{sphinxuseclass}\end{sphinxVerbatimInput}
\begin{sphinxVerbatimOutput}

\begin{sphinxuseclass}{cell_output}
\begin{sphinxVerbatim}[commandchars=\\\{\}]
(2008, 9, 15)
\end{sphinxVerbatim}

\end{sphinxuseclass}\end{sphinxVerbatimOutput}

\end{sphinxuseclass}

\subsubsection{Rewrite \sphinxstyleliteralintitle{\sphinxupquote{date}} to accept an 8\sphinxhyphen{}digit date as an integer or string.}
\label{\detokenize{mckinney_02_practice_02:rewrite-date-to-accept-an-8-digit-date-as-an-integer-or-string}}
\begin{sphinxuseclass}{cell}\begin{sphinxVerbatimInput}

\begin{sphinxuseclass}{cell_input}
\begin{sphinxVerbatim}[commandchars=\\\{\}]
\PYG{k}{def} \PYG{n+nf}{date}\PYG{p}{(}\PYG{n}{ymd}\PYG{p}{)}\PYG{p}{:}
    \PYG{k}{if} \PYG{n+nb}{type}\PYG{p}{(}\PYG{n}{ymd}\PYG{p}{)} \PYG{o+ow}{is} \PYG{n+nb}{str}\PYG{p}{:}
        \PYG{n}{ymd} \PYG{o}{=} \PYG{n+nb}{int}\PYG{p}{(}\PYG{n}{ymd}\PYG{p}{)}
    \PYG{n}{year} \PYG{o}{=} \PYG{n}{ymd} \PYG{o}{/}\PYG{o}{/} \PYG{l+m+mi}{10\PYGZus{}000}
    \PYG{n}{month} \PYG{o}{=} \PYG{p}{(}\PYG{n}{ymd} \PYG{o}{\PYGZpc{}} \PYG{l+m+mi}{10\PYGZus{}000}\PYG{p}{)} \PYG{o}{/}\PYG{o}{/} \PYG{l+m+mi}{100}
    \PYG{n}{day} \PYG{o}{=} \PYG{n}{ymd} \PYG{o}{\PYGZpc{}} \PYG{l+m+mi}{100}
    \PYG{k}{return} \PYG{p}{(}\PYG{n}{year}\PYG{p}{,} \PYG{n}{month}\PYG{p}{,} \PYG{n}{day}\PYG{p}{)}
\end{sphinxVerbatim}

\end{sphinxuseclass}\end{sphinxVerbatimInput}

\end{sphinxuseclass}
\begin{sphinxuseclass}{cell}\begin{sphinxVerbatimInput}

\begin{sphinxuseclass}{cell_input}
\begin{sphinxVerbatim}[commandchars=\\\{\}]
\PYG{n}{date}\PYG{p}{(}\PYG{l+s+s1}{\PYGZsq{}}\PYG{l+s+s1}{20080915}\PYG{l+s+s1}{\PYGZsq{}}\PYG{p}{)}
\end{sphinxVerbatim}

\end{sphinxuseclass}\end{sphinxVerbatimInput}
\begin{sphinxVerbatimOutput}

\begin{sphinxuseclass}{cell_output}
\begin{sphinxVerbatim}[commandchars=\\\{\}]
(2008, 9, 15)
\end{sphinxVerbatim}

\end{sphinxuseclass}\end{sphinxVerbatimOutput}

\end{sphinxuseclass}
\begin{sphinxuseclass}{cell}\begin{sphinxVerbatimInput}

\begin{sphinxuseclass}{cell_input}
\begin{sphinxVerbatim}[commandchars=\\\{\}]
\PYG{n}{date}\PYG{p}{(}\PYG{l+m+mi}{20080915}\PYG{p}{)}
\end{sphinxVerbatim}

\end{sphinxuseclass}\end{sphinxVerbatimInput}
\begin{sphinxVerbatimOutput}

\begin{sphinxuseclass}{cell_output}
\begin{sphinxVerbatim}[commandchars=\\\{\}]
(2008, 9, 15)
\end{sphinxVerbatim}

\end{sphinxuseclass}\end{sphinxVerbatimOutput}

\end{sphinxuseclass}

\subsubsection{Finally, rewrite \sphinxstyleliteralintitle{\sphinxupquote{date}} to accept a list of 8\sphinxhyphen{}digit dates as integers or strings.}
\label{\detokenize{mckinney_02_practice_02:finally-rewrite-date-to-accept-a-list-of-8-digit-dates-as-integers-or-strings}}
\sphinxAtStartPar
Return a list of tuples of year, month, and date.

\begin{sphinxuseclass}{cell}\begin{sphinxVerbatimInput}

\begin{sphinxuseclass}{cell_input}
\begin{sphinxVerbatim}[commandchars=\\\{\}]
\PYG{n}{ymds} \PYG{o}{=} \PYG{p}{[}\PYG{l+m+mi}{20080915}\PYG{p}{,} \PYG{l+s+s1}{\PYGZsq{}}\PYG{l+s+s1}{20080916}\PYG{l+s+s1}{\PYGZsq{}}\PYG{p}{]}
\end{sphinxVerbatim}

\end{sphinxuseclass}\end{sphinxVerbatimInput}

\end{sphinxuseclass}
\begin{sphinxuseclass}{cell}\begin{sphinxVerbatimInput}

\begin{sphinxuseclass}{cell_input}
\begin{sphinxVerbatim}[commandchars=\\\{\}]
\PYG{n}{ymds}\PYG{p}{[}\PYG{l+m+mi}{0}\PYG{p}{]}
\end{sphinxVerbatim}

\end{sphinxuseclass}\end{sphinxVerbatimInput}
\begin{sphinxVerbatimOutput}

\begin{sphinxuseclass}{cell_output}
\begin{sphinxVerbatim}[commandchars=\\\{\}]
20080915
\end{sphinxVerbatim}

\end{sphinxuseclass}\end{sphinxVerbatimOutput}

\end{sphinxuseclass}
\begin{sphinxuseclass}{cell}\begin{sphinxVerbatimInput}

\begin{sphinxuseclass}{cell_input}
\begin{sphinxVerbatim}[commandchars=\\\{\}]
\PYG{n}{ymds}\PYG{p}{[}\PYG{l+m+mi}{1}\PYG{p}{]}
\end{sphinxVerbatim}

\end{sphinxuseclass}\end{sphinxVerbatimInput}
\begin{sphinxVerbatimOutput}

\begin{sphinxuseclass}{cell_output}
\begin{sphinxVerbatim}[commandchars=\\\{\}]
\PYGZsq{}20080916\PYGZsq{}
\end{sphinxVerbatim}

\end{sphinxuseclass}\end{sphinxVerbatimOutput}

\end{sphinxuseclass}
\begin{sphinxuseclass}{cell}\begin{sphinxVerbatimInput}

\begin{sphinxuseclass}{cell_input}
\begin{sphinxVerbatim}[commandchars=\\\{\}]
\PYG{k}{for} \PYG{n}{ymd} \PYG{o+ow}{in} \PYG{n}{ymds}\PYG{p}{:}
    \PYG{n+nb}{print}\PYG{p}{(}\PYG{n}{ymd}\PYG{p}{)}
\end{sphinxVerbatim}

\end{sphinxuseclass}\end{sphinxVerbatimInput}
\begin{sphinxVerbatimOutput}

\begin{sphinxuseclass}{cell_output}
\begin{sphinxVerbatim}[commandchars=\\\{\}]
20080915
20080916
\end{sphinxVerbatim}

\end{sphinxuseclass}\end{sphinxVerbatimOutput}

\end{sphinxuseclass}
\begin{sphinxuseclass}{cell}\begin{sphinxVerbatimInput}

\begin{sphinxuseclass}{cell_input}
\begin{sphinxVerbatim}[commandchars=\\\{\}]
\PYG{k}{def} \PYG{n+nf}{date}\PYG{p}{(}\PYG{n}{ymds}\PYG{p}{)}\PYG{p}{:}
    \PYG{n}{parsed\PYGZus{}dates} \PYG{o}{=} \PYG{p}{[}\PYG{p}{]}
    \PYG{k}{for} \PYG{n}{ymd} \PYG{o+ow}{in} \PYG{n}{ymds}\PYG{p}{:}
        \PYG{k}{if} \PYG{n+nb}{type}\PYG{p}{(}\PYG{n}{ymd}\PYG{p}{)} \PYG{o+ow}{is} \PYG{n+nb}{str}\PYG{p}{:}
            \PYG{n}{ymd} \PYG{o}{=} \PYG{n+nb}{int}\PYG{p}{(}\PYG{n}{ymd}\PYG{p}{)}
        \PYG{n}{year} \PYG{o}{=} \PYG{n}{ymd} \PYG{o}{/}\PYG{o}{/} \PYG{l+m+mi}{10\PYGZus{}000}
        \PYG{n}{month} \PYG{o}{=} \PYG{p}{(}\PYG{n}{ymd} \PYG{o}{\PYGZpc{}} \PYG{l+m+mi}{10\PYGZus{}000}\PYG{p}{)} \PYG{o}{/}\PYG{o}{/} \PYG{l+m+mi}{100}
        \PYG{n}{day} \PYG{o}{=} \PYG{n}{ymd} \PYG{o}{\PYGZpc{}} \PYG{l+m+mi}{100}
        \PYG{n}{parsed\PYGZus{}dates}\PYG{o}{.}\PYG{n}{append}\PYG{p}{(}\PYG{p}{(}\PYG{n}{year}\PYG{p}{,} \PYG{n}{month}\PYG{p}{,} \PYG{n}{day}\PYG{p}{)}\PYG{p}{)}
    \PYG{k}{return} \PYG{n}{parsed\PYGZus{}dates}
\end{sphinxVerbatim}

\end{sphinxuseclass}\end{sphinxVerbatimInput}

\end{sphinxuseclass}
\begin{sphinxuseclass}{cell}\begin{sphinxVerbatimInput}

\begin{sphinxuseclass}{cell_input}
\begin{sphinxVerbatim}[commandchars=\\\{\}]
\PYG{n}{date}\PYG{p}{(}\PYG{n}{ymds}\PYG{o}{=}\PYG{n}{ymds}\PYG{p}{)}
\end{sphinxVerbatim}

\end{sphinxuseclass}\end{sphinxVerbatimInput}
\begin{sphinxVerbatimOutput}

\begin{sphinxuseclass}{cell_output}
\begin{sphinxVerbatim}[commandchars=\\\{\}]
[(2008, 9, 15), (2008, 9, 16)]
\end{sphinxVerbatim}

\end{sphinxuseclass}\end{sphinxVerbatimOutput}

\end{sphinxuseclass}

\subsubsection{Write a for loop that prints the squares of integers from 1 to 10.}
\label{\detokenize{mckinney_02_practice_02:write-a-for-loop-that-prints-the-squares-of-integers-from-1-to-10}}
\begin{sphinxuseclass}{cell}\begin{sphinxVerbatimInput}

\begin{sphinxuseclass}{cell_input}
\begin{sphinxVerbatim}[commandchars=\\\{\}]
\PYG{k}{for} \PYG{n}{i} \PYG{o+ow}{in} \PYG{n+nb}{range}\PYG{p}{(}\PYG{l+m+mi}{1}\PYG{p}{,} \PYG{l+m+mi}{11}\PYG{p}{)}\PYG{p}{:}
    \PYG{n+nb}{print}\PYG{p}{(}\PYG{n}{i}\PYG{o}{*}\PYG{o}{*}\PYG{l+m+mi}{2}\PYG{p}{,} \PYG{n}{end}\PYG{o}{=}\PYG{l+s+s1}{\PYGZsq{}}\PYG{l+s+s1}{ }\PYG{l+s+s1}{\PYGZsq{}}\PYG{p}{)}
\end{sphinxVerbatim}

\end{sphinxuseclass}\end{sphinxVerbatimInput}
\begin{sphinxVerbatimOutput}

\begin{sphinxuseclass}{cell_output}
\begin{sphinxVerbatim}[commandchars=\\\{\}]
1 4 9 16 25 36 49 64 81 100 
\end{sphinxVerbatim}

\end{sphinxuseclass}\end{sphinxVerbatimOutput}

\end{sphinxuseclass}
\sphinxAtStartPar
Above, I change the \sphinxcode{\sphinxupquote{end}} argument from the default ‘\textbackslash{}n’ to ‘ ‘.
The default ‘\textbackslash{}n’ inserts a new line after value, making the output too long.


\subsubsection{Write a for loop that prints the squares of \sphinxstyleemphasis{even} integers from 1 to 10.}
\label{\detokenize{mckinney_02_practice_02:write-a-for-loop-that-prints-the-squares-of-even-integers-from-1-to-10}}
\begin{sphinxuseclass}{cell}\begin{sphinxVerbatimInput}

\begin{sphinxuseclass}{cell_input}
\begin{sphinxVerbatim}[commandchars=\\\{\}]
\PYG{k}{for} \PYG{n}{i} \PYG{o+ow}{in} \PYG{n+nb}{range}\PYG{p}{(}\PYG{l+m+mi}{1}\PYG{p}{,} \PYG{l+m+mi}{11}\PYG{p}{)}\PYG{p}{:}
    \PYG{k}{if} \PYG{n}{i} \PYG{o}{\PYGZpc{}} \PYG{l+m+mi}{2} \PYG{o}{==} \PYG{l+m+mi}{0}\PYG{p}{:}
        \PYG{n+nb}{print}\PYG{p}{(}\PYG{n}{i}\PYG{o}{*}\PYG{o}{*}\PYG{l+m+mi}{2}\PYG{p}{,} \PYG{n}{end}\PYG{o}{=}\PYG{l+s+s1}{\PYGZsq{}}\PYG{l+s+s1}{ }\PYG{l+s+s1}{\PYGZsq{}}\PYG{p}{)}
\end{sphinxVerbatim}

\end{sphinxuseclass}\end{sphinxVerbatimInput}
\begin{sphinxVerbatimOutput}

\begin{sphinxuseclass}{cell_output}
\begin{sphinxVerbatim}[commandchars=\\\{\}]
4 16 36 64 100 
\end{sphinxVerbatim}

\end{sphinxuseclass}\end{sphinxVerbatimOutput}

\end{sphinxuseclass}

\subsubsection{Write a for loop that sums the squares of integers from 1 to 10.}
\label{\detokenize{mckinney_02_practice_02:write-a-for-loop-that-sums-the-squares-of-integers-from-1-to-10}}
\begin{sphinxuseclass}{cell}\begin{sphinxVerbatimInput}

\begin{sphinxuseclass}{cell_input}
\begin{sphinxVerbatim}[commandchars=\\\{\}]
\PYG{n}{total} \PYG{o}{=} \PYG{l+m+mi}{0}
\PYG{k}{for} \PYG{n}{i} \PYG{o+ow}{in} \PYG{n+nb}{range}\PYG{p}{(}\PYG{l+m+mi}{1}\PYG{p}{,} \PYG{l+m+mi}{11}\PYG{p}{)}\PYG{p}{:}
    \PYG{n}{total} \PYG{o}{+}\PYG{o}{=} \PYG{n}{i}\PYG{o}{*}\PYG{o}{*}\PYG{l+m+mi}{2}
\end{sphinxVerbatim}

\end{sphinxuseclass}\end{sphinxVerbatimInput}

\end{sphinxuseclass}
\begin{sphinxuseclass}{cell}\begin{sphinxVerbatimInput}

\begin{sphinxuseclass}{cell_input}
\begin{sphinxVerbatim}[commandchars=\\\{\}]
\PYG{n}{total}
\end{sphinxVerbatim}

\end{sphinxuseclass}\end{sphinxVerbatimInput}
\begin{sphinxVerbatimOutput}

\begin{sphinxuseclass}{cell_output}
\begin{sphinxVerbatim}[commandchars=\\\{\}]
385
\end{sphinxVerbatim}

\end{sphinxuseclass}\end{sphinxVerbatimOutput}

\end{sphinxuseclass}
\sphinxAtStartPar
Above, I use \sphinxcode{\sphinxupquote{total += i}}, which is equivalent to \sphinxcode{\sphinxupquote{total = total + i}}.


\subsubsection{Write a for loop that sums the squares of integers from 1 to 10 but stops before the sum exceeds 50.}
\label{\detokenize{mckinney_02_practice_02:write-a-for-loop-that-sums-the-squares-of-integers-from-1-to-10-but-stops-before-the-sum-exceeds-50}}
\begin{sphinxuseclass}{cell}\begin{sphinxVerbatimInput}

\begin{sphinxuseclass}{cell_input}
\begin{sphinxVerbatim}[commandchars=\\\{\}]
\PYG{n}{total} \PYG{o}{=} \PYG{l+m+mi}{0}
\PYG{k}{for} \PYG{n}{i} \PYG{o+ow}{in} \PYG{n+nb}{range}\PYG{p}{(}\PYG{l+m+mi}{1}\PYG{p}{,} \PYG{l+m+mi}{11}\PYG{p}{)}\PYG{p}{:}
    \PYG{k}{if} \PYG{p}{(}\PYG{n}{total} \PYG{o}{+} \PYG{n}{i}\PYG{o}{*}\PYG{o}{*}\PYG{l+m+mi}{2}\PYG{p}{)} \PYG{o}{\PYGZgt{}} \PYG{l+m+mi}{50}\PYG{p}{:}
        \PYG{k}{break}
    \PYG{n}{total} \PYG{o}{+}\PYG{o}{=} \PYG{n}{i}\PYG{o}{*}\PYG{o}{*}\PYG{l+m+mi}{2}
\end{sphinxVerbatim}

\end{sphinxuseclass}\end{sphinxVerbatimInput}

\end{sphinxuseclass}
\begin{sphinxuseclass}{cell}\begin{sphinxVerbatimInput}

\begin{sphinxuseclass}{cell_input}
\begin{sphinxVerbatim}[commandchars=\\\{\}]
\PYG{n}{total}
\end{sphinxVerbatim}

\end{sphinxuseclass}\end{sphinxVerbatimInput}
\begin{sphinxVerbatimOutput}

\begin{sphinxuseclass}{cell_output}
\begin{sphinxVerbatim}[commandchars=\\\{\}]
30
\end{sphinxVerbatim}

\end{sphinxuseclass}\end{sphinxVerbatimOutput}

\end{sphinxuseclass}

\subsubsection{FizzBuzz}
\label{\detokenize{mckinney_02_practice_02:fizzbuzz}}
\sphinxAtStartPar
Write a for loop that prints the numbers from 1 to 100.
For multiples of three print “Fizz” instead of the number.
For multiples of five print “Buzz”.
For numbers that are multiples of both three and five print “FizzBuzz”.
More \sphinxhref{https://blog.codinghorror.com/why-cant-programmers-program/}{here}.

\begin{sphinxuseclass}{cell}\begin{sphinxVerbatimInput}

\begin{sphinxuseclass}{cell_input}
\begin{sphinxVerbatim}[commandchars=\\\{\}]
\PYG{l+m+mi}{6} \PYG{o}{\PYGZpc{}} \PYG{l+m+mi}{3} \PYG{o}{==} \PYG{l+m+mi}{0}
\end{sphinxVerbatim}

\end{sphinxuseclass}\end{sphinxVerbatimInput}
\begin{sphinxVerbatimOutput}

\begin{sphinxuseclass}{cell_output}
\begin{sphinxVerbatim}[commandchars=\\\{\}]
True
\end{sphinxVerbatim}

\end{sphinxuseclass}\end{sphinxVerbatimOutput}

\end{sphinxuseclass}
\begin{sphinxuseclass}{cell}\begin{sphinxVerbatimInput}

\begin{sphinxuseclass}{cell_input}
\begin{sphinxVerbatim}[commandchars=\\\{\}]
\PYG{k}{for} \PYG{n}{i} \PYG{o+ow}{in} \PYG{n+nb}{range}\PYG{p}{(}\PYG{l+m+mi}{1}\PYG{p}{,} \PYG{l+m+mi}{101}\PYG{p}{)}\PYG{p}{:}
    \PYG{n}{is\PYGZus{}mult\PYGZus{}3} \PYG{o}{=} \PYG{p}{(}\PYG{n}{i} \PYG{o}{\PYGZpc{}} \PYG{l+m+mi}{3} \PYG{o}{==} \PYG{l+m+mi}{0}\PYG{p}{)}
    \PYG{n}{is\PYGZus{}mult\PYGZus{}5} \PYG{o}{=} \PYG{p}{(}\PYG{n}{i} \PYG{o}{\PYGZpc{}} \PYG{l+m+mi}{5} \PYG{o}{==} \PYG{l+m+mi}{0}\PYG{p}{)}
    \PYG{k}{if} \PYG{n}{is\PYGZus{}mult\PYGZus{}3} \PYG{o+ow}{and} \PYG{n}{is\PYGZus{}mult\PYGZus{}5}\PYG{p}{:}
        \PYG{n+nb}{print}\PYG{p}{(}\PYG{l+s+s1}{\PYGZsq{}}\PYG{l+s+s1}{FizzBuzz}\PYG{l+s+s1}{\PYGZsq{}}\PYG{p}{,} \PYG{n}{end}\PYG{o}{=}\PYG{l+s+s1}{\PYGZsq{}}\PYG{l+s+s1}{ }\PYG{l+s+s1}{\PYGZsq{}}\PYG{p}{)}
    \PYG{k}{elif} \PYG{n}{is\PYGZus{}mult\PYGZus{}3}\PYG{p}{:}
        \PYG{n+nb}{print}\PYG{p}{(}\PYG{l+s+s1}{\PYGZsq{}}\PYG{l+s+s1}{Fizz}\PYG{l+s+s1}{\PYGZsq{}}\PYG{p}{,} \PYG{n}{end}\PYG{o}{=}\PYG{l+s+s1}{\PYGZsq{}}\PYG{l+s+s1}{ }\PYG{l+s+s1}{\PYGZsq{}}\PYG{p}{)}
    \PYG{k}{elif} \PYG{n}{is\PYGZus{}mult\PYGZus{}5}\PYG{p}{:}
        \PYG{n+nb}{print}\PYG{p}{(}\PYG{l+s+s1}{\PYGZsq{}}\PYG{l+s+s1}{Buzz}\PYG{l+s+s1}{\PYGZsq{}}\PYG{p}{,} \PYG{n}{end}\PYG{o}{=}\PYG{l+s+s1}{\PYGZsq{}}\PYG{l+s+s1}{ }\PYG{l+s+s1}{\PYGZsq{}}\PYG{p}{)}
    \PYG{k}{else}\PYG{p}{:}
        \PYG{n+nb}{print}\PYG{p}{(}\PYG{n}{i}\PYG{p}{,} \PYG{n}{end}\PYG{o}{=}\PYG{l+s+s1}{\PYGZsq{}}\PYG{l+s+s1}{ }\PYG{l+s+s1}{\PYGZsq{}}\PYG{p}{)}
\end{sphinxVerbatim}

\end{sphinxuseclass}\end{sphinxVerbatimInput}
\begin{sphinxVerbatimOutput}

\begin{sphinxuseclass}{cell_output}
\begin{sphinxVerbatim}[commandchars=\\\{\}]
1 2 Fizz 4 Buzz Fizz 7 8 Fizz Buzz 11 Fizz 13 14 FizzBuzz 16 17 Fizz 19 Buzz Fizz 22 23 Fizz Buzz 26 Fizz 28 29 FizzBuzz 31 32 Fizz 34 Buzz Fizz 37 38 Fizz Buzz 41 Fizz 43 44 FizzBuzz 46 47 Fizz 49 Buzz Fizz 52 53 Fizz Buzz 56 Fizz 58 59 FizzBuzz 61 62 Fizz 64 Buzz Fizz 67 68 Fizz Buzz 71 Fizz 73 74 FizzBuzz 76 77 Fizz 79 Buzz Fizz 82 83 Fizz Buzz 86 Fizz 88 89 FizzBuzz 91 92 Fizz 94 Buzz Fizz 97 98 Fizz Buzz 
\end{sphinxVerbatim}

\end{sphinxuseclass}\end{sphinxVerbatimOutput}

\end{sphinxuseclass}

\subsubsection{Use ternary expressions to make your FizzBuzz code more compact.}
\label{\detokenize{mckinney_02_practice_02:use-ternary-expressions-to-make-your-fizzbuzz-code-more-compact}}
\begin{sphinxuseclass}{cell}\begin{sphinxVerbatimInput}

\begin{sphinxuseclass}{cell_input}
\begin{sphinxVerbatim}[commandchars=\\\{\}]
\PYG{k}{for} \PYG{n}{i} \PYG{o+ow}{in} \PYG{n+nb}{range}\PYG{p}{(}\PYG{l+m+mi}{1}\PYG{p}{,} \PYG{l+m+mi}{101}\PYG{p}{)}\PYG{p}{:}
    \PYG{n}{is\PYGZus{}mult\PYGZus{}3} \PYG{o}{=} \PYG{p}{(}\PYG{n}{i} \PYG{o}{\PYGZpc{}} \PYG{l+m+mi}{3} \PYG{o}{==} \PYG{l+m+mi}{0}\PYG{p}{)}
    \PYG{n}{is\PYGZus{}mult\PYGZus{}5} \PYG{o}{=} \PYG{p}{(}\PYG{n}{i} \PYG{o}{\PYGZpc{}} \PYG{l+m+mi}{5} \PYG{o}{==} \PYG{l+m+mi}{0}\PYG{p}{)}
    \PYG{n+nb}{print}\PYG{p}{(}\PYG{l+s+s1}{\PYGZsq{}}\PYG{l+s+s1}{Fizz}\PYG{l+s+s1}{\PYGZsq{}}\PYG{o}{*}\PYG{n}{is\PYGZus{}mult\PYGZus{}3} \PYG{o}{+} \PYG{l+s+s1}{\PYGZsq{}}\PYG{l+s+s1}{Buzz}\PYG{l+s+s1}{\PYGZsq{}}\PYG{o}{*}\PYG{n}{is\PYGZus{}mult\PYGZus{}5} \PYG{k}{if} \PYG{n}{is\PYGZus{}mult\PYGZus{}3} \PYG{o+ow}{or} \PYG{n}{is\PYGZus{}mult\PYGZus{}5} \PYG{k}{else} \PYG{n}{i}\PYG{p}{,} \PYG{n}{end}\PYG{o}{=}\PYG{l+s+s1}{\PYGZsq{}}\PYG{l+s+s1}{ }\PYG{l+s+s1}{\PYGZsq{}}\PYG{p}{)}
\end{sphinxVerbatim}

\end{sphinxuseclass}\end{sphinxVerbatimInput}
\begin{sphinxVerbatimOutput}

\begin{sphinxuseclass}{cell_output}
\begin{sphinxVerbatim}[commandchars=\\\{\}]
1 2 Fizz 4 Buzz Fizz 7 8 Fizz Buzz 11 Fizz 13 14 FizzBuzz 16 17 Fizz 19 Buzz Fizz 22 23 Fizz Buzz 26 Fizz 28 29 FizzBuzz 31 32 Fizz 34 Buzz Fizz 37 38 Fizz Buzz 41 Fizz 43 44 FizzBuzz 46 47 Fizz 49 Buzz Fizz 52 53 Fizz Buzz 56 Fizz 58 59 FizzBuzz 61 62 Fizz 64 Buzz Fizz 67 68 Fizz Buzz 71 Fizz 73 74 FizzBuzz 76 77 Fizz 79 Buzz Fizz 82 83 Fizz Buzz 86 Fizz 88 89 FizzBuzz 91 92 Fizz 94 Buzz Fizz 97 98 Fizz Buzz 
\end{sphinxVerbatim}

\end{sphinxuseclass}\end{sphinxVerbatimOutput}

\end{sphinxuseclass}
\sphinxAtStartPar
The solution above is shorter and uses from neat tricks, but I consider the previous solution easier to read and troubleshoot.


\subsubsection{Triangle}
\label{\detokenize{mckinney_02_practice_02:triangle}}
\sphinxAtStartPar
Write a function \sphinxcode{\sphinxupquote{triangle}} that accepts a positive integer \$N\$ and prints a numerical triangle of height \$N\sphinxhyphen{}1\$.
For example, \sphinxcode{\sphinxupquote{triangle(N=6)}} should print:

\begin{sphinxVerbatim}[commandchars=\\\{\}]
\PYG{l+m+mi}{1}
\PYG{l+m+mi}{22}
\PYG{l+m+mi}{333}
\PYG{l+m+mi}{4444}
\PYG{l+m+mi}{55555}
\end{sphinxVerbatim}

\begin{sphinxuseclass}{cell}\begin{sphinxVerbatimInput}

\begin{sphinxuseclass}{cell_input}
\begin{sphinxVerbatim}[commandchars=\\\{\}]
\PYG{k}{def} \PYG{n+nf}{triangle}\PYG{p}{(}\PYG{n}{N}\PYG{p}{)}\PYG{p}{:}
    \PYG{k}{for} \PYG{n}{i} \PYG{o+ow}{in} \PYG{n+nb}{range}\PYG{p}{(}\PYG{l+m+mi}{1}\PYG{p}{,} \PYG{n}{N}\PYG{p}{)}\PYG{p}{:}
        \PYG{n+nb}{print}\PYG{p}{(}\PYG{n+nb}{str}\PYG{p}{(}\PYG{n}{i}\PYG{p}{)} \PYG{o}{*} \PYG{n}{i}\PYG{p}{)}
\end{sphinxVerbatim}

\end{sphinxuseclass}\end{sphinxVerbatimInput}

\end{sphinxuseclass}
\begin{sphinxuseclass}{cell}\begin{sphinxVerbatimInput}

\begin{sphinxuseclass}{cell_input}
\begin{sphinxVerbatim}[commandchars=\\\{\}]
\PYG{n}{triangle}\PYG{p}{(}\PYG{l+m+mi}{6}\PYG{p}{)}
\end{sphinxVerbatim}

\end{sphinxuseclass}\end{sphinxVerbatimInput}
\begin{sphinxVerbatimOutput}

\begin{sphinxuseclass}{cell_output}
\begin{sphinxVerbatim}[commandchars=\\\{\}]
1
22
333
4444
55555
\end{sphinxVerbatim}

\end{sphinxuseclass}\end{sphinxVerbatimOutput}

\end{sphinxuseclass}
\sphinxAtStartPar
The solution above works because a multiplying a string by \sphinxcode{\sphinxupquote{i}} concatenates \sphinxcode{\sphinxupquote{i}} copies of the string.

\begin{sphinxuseclass}{cell}\begin{sphinxVerbatimInput}

\begin{sphinxuseclass}{cell_input}
\begin{sphinxVerbatim}[commandchars=\\\{\}]
\PYG{l+s+s1}{\PYGZsq{}}\PYG{l+s+s1}{Test}\PYG{l+s+s1}{\PYGZsq{}} \PYG{o}{+} \PYG{l+s+s1}{\PYGZsq{}}\PYG{l+s+s1}{Test}\PYG{l+s+s1}{\PYGZsq{}} \PYG{o}{+} \PYG{l+s+s1}{\PYGZsq{}}\PYG{l+s+s1}{Test}\PYG{l+s+s1}{\PYGZsq{}}
\end{sphinxVerbatim}

\end{sphinxuseclass}\end{sphinxVerbatimInput}
\begin{sphinxVerbatimOutput}

\begin{sphinxuseclass}{cell_output}
\begin{sphinxVerbatim}[commandchars=\\\{\}]
\PYGZsq{}TestTestTest\PYGZsq{}
\end{sphinxVerbatim}

\end{sphinxuseclass}\end{sphinxVerbatimOutput}

\end{sphinxuseclass}
\begin{sphinxuseclass}{cell}\begin{sphinxVerbatimInput}

\begin{sphinxuseclass}{cell_input}
\begin{sphinxVerbatim}[commandchars=\\\{\}]
\PYG{l+s+s1}{\PYGZsq{}}\PYG{l+s+s1}{Test}\PYG{l+s+s1}{\PYGZsq{}} \PYG{o}{*} \PYG{l+m+mi}{3}
\end{sphinxVerbatim}

\end{sphinxuseclass}\end{sphinxVerbatimInput}
\begin{sphinxVerbatimOutput}

\begin{sphinxuseclass}{cell_output}
\begin{sphinxVerbatim}[commandchars=\\\{\}]
\PYGZsq{}TestTestTest\PYGZsq{}
\end{sphinxVerbatim}

\end{sphinxuseclass}\end{sphinxVerbatimOutput}

\end{sphinxuseclass}

\subsubsection{Two Sum}
\label{\detokenize{mckinney_02_practice_02:two-sum}}
\sphinxAtStartPar
Write a function \sphinxcode{\sphinxupquote{two\_sum}} that does the following.

\sphinxAtStartPar
Given an array of integers \sphinxcode{\sphinxupquote{nums}} and an integer \sphinxcode{\sphinxupquote{target}}, return the indices of the two numbers that add up to target.

\sphinxAtStartPar
You may assume that each input would have exactly one solution, and you may not use the same element twice.

\sphinxAtStartPar
You can return the answer in any order.

\sphinxAtStartPar
Here are some examples:

\sphinxAtStartPar
Example 1:

\sphinxAtStartPar
Input: \sphinxcode{\sphinxupquote{nums = {[}2,7,11,15{]}}}, \sphinxcode{\sphinxupquote{target = 9}} \\
Output: \sphinxcode{\sphinxupquote{{[}0,1{]}}} \\
Explanation: Because \sphinxcode{\sphinxupquote{nums{[}0{]} + nums{[}1{]} == 9}}, we return \sphinxcode{\sphinxupquote{{[}0, 1{]}}}.

\sphinxAtStartPar
Example 2:

\sphinxAtStartPar
Input: \sphinxcode{\sphinxupquote{nums = {[}3,2,4{]}}}, \sphinxcode{\sphinxupquote{target = 6}} \\
Output: \sphinxcode{\sphinxupquote{{[}1,2{]}}} \textbackslash{}

\sphinxAtStartPar
Example 3:

\sphinxAtStartPar
Input: \sphinxcode{\sphinxupquote{nums = {[}3,3{]}}}, \sphinxcode{\sphinxupquote{target = 6}} \\
Output: \sphinxcode{\sphinxupquote{{[}0,1{]}}} \textbackslash{}

\sphinxAtStartPar
I saw this question on \sphinxhref{https://leetcode.com/problems/two-sum/description/}{LeetCode}.

\begin{sphinxuseclass}{cell}\begin{sphinxVerbatimInput}

\begin{sphinxuseclass}{cell_input}
\begin{sphinxVerbatim}[commandchars=\\\{\}]
\PYG{k}{def} \PYG{n+nf}{two\PYGZus{}sum}\PYG{p}{(}\PYG{n}{nums}\PYG{p}{,} \PYG{n}{target}\PYG{p}{)}\PYG{p}{:}
    \PYG{k}{for} \PYG{n}{i} \PYG{o+ow}{in} \PYG{n+nb}{range}\PYG{p}{(}\PYG{l+m+mi}{1}\PYG{p}{,} \PYG{n+nb}{len}\PYG{p}{(}\PYG{n}{nums}\PYG{p}{)}\PYG{p}{)}\PYG{p}{:}
        \PYG{k}{for} \PYG{n}{j} \PYG{o+ow}{in} \PYG{n+nb}{range}\PYG{p}{(}\PYG{n}{i}\PYG{p}{)}\PYG{p}{:}
            \PYG{k}{if} \PYG{n}{nums}\PYG{p}{[}\PYG{n}{i}\PYG{p}{]} \PYG{o}{+} \PYG{n}{nums}\PYG{p}{[}\PYG{n}{j}\PYG{p}{]} \PYG{o}{==} \PYG{n}{target}\PYG{p}{:}
                \PYG{k}{return} \PYG{p}{[}\PYG{n}{j}\PYG{p}{,} \PYG{n}{i}\PYG{p}{]}
\end{sphinxVerbatim}

\end{sphinxuseclass}\end{sphinxVerbatimInput}

\end{sphinxuseclass}
\begin{sphinxuseclass}{cell}\begin{sphinxVerbatimInput}

\begin{sphinxuseclass}{cell_input}
\begin{sphinxVerbatim}[commandchars=\\\{\}]
\PYG{n}{two\PYGZus{}sum}\PYG{p}{(}\PYG{n}{nums} \PYG{o}{=} \PYG{p}{[}\PYG{l+m+mi}{2}\PYG{p}{,}\PYG{l+m+mi}{7}\PYG{p}{,}\PYG{l+m+mi}{11}\PYG{p}{,}\PYG{l+m+mi}{15}\PYG{p}{]}\PYG{p}{,} \PYG{n}{target} \PYG{o}{=} \PYG{l+m+mi}{9}\PYG{p}{)}
\end{sphinxVerbatim}

\end{sphinxuseclass}\end{sphinxVerbatimInput}
\begin{sphinxVerbatimOutput}

\begin{sphinxuseclass}{cell_output}
\begin{sphinxVerbatim}[commandchars=\\\{\}]
[0, 1]
\end{sphinxVerbatim}

\end{sphinxuseclass}\end{sphinxVerbatimOutput}

\end{sphinxuseclass}
\begin{sphinxuseclass}{cell}\begin{sphinxVerbatimInput}

\begin{sphinxuseclass}{cell_input}
\begin{sphinxVerbatim}[commandchars=\\\{\}]
\PYG{n}{two\PYGZus{}sum}\PYG{p}{(}\PYG{n}{nums} \PYG{o}{=} \PYG{p}{[}\PYG{l+m+mi}{3}\PYG{p}{,}\PYG{l+m+mi}{2}\PYG{p}{,}\PYG{l+m+mi}{4}\PYG{p}{]}\PYG{p}{,} \PYG{n}{target} \PYG{o}{=} \PYG{l+m+mi}{6}\PYG{p}{)}
\end{sphinxVerbatim}

\end{sphinxuseclass}\end{sphinxVerbatimInput}
\begin{sphinxVerbatimOutput}

\begin{sphinxuseclass}{cell_output}
\begin{sphinxVerbatim}[commandchars=\\\{\}]
[1, 2]
\end{sphinxVerbatim}

\end{sphinxuseclass}\end{sphinxVerbatimOutput}

\end{sphinxuseclass}
\begin{sphinxuseclass}{cell}\begin{sphinxVerbatimInput}

\begin{sphinxuseclass}{cell_input}
\begin{sphinxVerbatim}[commandchars=\\\{\}]
\PYG{n}{two\PYGZus{}sum}\PYG{p}{(}\PYG{n}{nums} \PYG{o}{=} \PYG{p}{[}\PYG{l+m+mi}{3}\PYG{p}{,}\PYG{l+m+mi}{3}\PYG{p}{]}\PYG{p}{,} \PYG{n}{target} \PYG{o}{=} \PYG{l+m+mi}{6}\PYG{p}{)}
\end{sphinxVerbatim}

\end{sphinxuseclass}\end{sphinxVerbatimInput}
\begin{sphinxVerbatimOutput}

\begin{sphinxuseclass}{cell_output}
\begin{sphinxVerbatim}[commandchars=\\\{\}]
[0, 1]
\end{sphinxVerbatim}

\end{sphinxuseclass}\end{sphinxVerbatimOutput}

\end{sphinxuseclass}
\sphinxAtStartPar
We can write more efficient code once we learn other data structures in chapter 3 of McKinney!


\subsubsection{Best Time}
\label{\detokenize{mckinney_02_practice_02:best-time}}
\sphinxAtStartPar
Write a function \sphinxcode{\sphinxupquote{best\_time}} that solves the following.

\sphinxAtStartPar
You are given an array \sphinxcode{\sphinxupquote{prices}} where \sphinxcode{\sphinxupquote{prices{[}i{]}}} is the price of a given stock on the \$i\textasciicircum{}\{th\}\$ day.

\sphinxAtStartPar
You want to maximize your profit by choosing a single day to buy one stock and choosing a different day in the future to sell that stock.

\sphinxAtStartPar
Return the maximum profit you can achieve from this transaction. If you cannot achieve any profit, return 0.

\sphinxAtStartPar
Here are some examples:

\sphinxAtStartPar
Example 1:

\sphinxAtStartPar
Input: \sphinxcode{\sphinxupquote{prices = {[}7,1,5,3,6,4{]}}} \\
Output: \sphinxcode{\sphinxupquote{5}} \\
Explanation: Buy on day 2 (price = 1) and sell on day 5 (price = 6), profit = 6\sphinxhyphen{}1 = 5.
Note that buying on day 2 and selling on day 1 is not allowed because you must buy before you sell.

\sphinxAtStartPar
Example 2:

\sphinxAtStartPar
Input: \sphinxcode{\sphinxupquote{prices = {[}7,6,4,3,1{]}}} \\
Output: \sphinxcode{\sphinxupquote{0}} \\
Explanation: In this case, no transactions are done and the max profit = 0.

\sphinxAtStartPar
I saw this question on \sphinxhref{https://leetcode.com/problems/best-time-to-buy-and-sell-stock/}{LeetCode}.

\begin{sphinxuseclass}{cell}\begin{sphinxVerbatimInput}

\begin{sphinxuseclass}{cell_input}
\begin{sphinxVerbatim}[commandchars=\\\{\}]
\PYG{k}{def} \PYG{n+nf}{max\PYGZus{}profit}\PYG{p}{(}\PYG{n}{prices}\PYG{p}{)}\PYG{p}{:}
        \PYG{n}{min\PYGZus{}price} \PYG{o}{=} \PYG{n}{prices}\PYG{p}{[}\PYG{l+m+mi}{0}\PYG{p}{]}
        \PYG{n}{max\PYGZus{}profit} \PYG{o}{=} \PYG{l+m+mi}{0}
        \PYG{k}{for} \PYG{n}{price} \PYG{o+ow}{in} \PYG{n}{prices}\PYG{p}{:}
            \PYG{n}{min\PYGZus{}price} \PYG{o}{=} \PYG{n}{price} \PYG{k}{if} \PYG{n}{price} \PYG{o}{\PYGZlt{}} \PYG{n}{min\PYGZus{}price} \PYG{k}{else} \PYG{n}{min\PYGZus{}price}
            \PYG{n}{profit} \PYG{o}{=} \PYG{n}{price} \PYG{o}{\PYGZhy{}} \PYG{n}{min\PYGZus{}price}
            \PYG{n}{max\PYGZus{}profit} \PYG{o}{=} \PYG{n}{profit} \PYG{k}{if} \PYG{n}{profit} \PYG{o}{\PYGZgt{}} \PYG{n}{max\PYGZus{}profit} \PYG{k}{else} \PYG{n}{max\PYGZus{}profit}
        \PYG{k}{return} \PYG{n}{max\PYGZus{}profit}
\end{sphinxVerbatim}

\end{sphinxuseclass}\end{sphinxVerbatimInput}

\end{sphinxuseclass}
\begin{sphinxuseclass}{cell}\begin{sphinxVerbatimInput}

\begin{sphinxuseclass}{cell_input}
\begin{sphinxVerbatim}[commandchars=\\\{\}]
\PYG{n}{max\PYGZus{}profit}\PYG{p}{(}\PYG{n}{prices}\PYG{o}{=}\PYG{p}{[}\PYG{l+m+mi}{7}\PYG{p}{,}\PYG{l+m+mi}{1}\PYG{p}{,}\PYG{l+m+mi}{5}\PYG{p}{,}\PYG{l+m+mi}{3}\PYG{p}{,}\PYG{l+m+mi}{6}\PYG{p}{,}\PYG{l+m+mi}{4}\PYG{p}{]}\PYG{p}{)}
\end{sphinxVerbatim}

\end{sphinxuseclass}\end{sphinxVerbatimInput}
\begin{sphinxVerbatimOutput}

\begin{sphinxuseclass}{cell_output}
\begin{sphinxVerbatim}[commandchars=\\\{\}]
5
\end{sphinxVerbatim}

\end{sphinxuseclass}\end{sphinxVerbatimOutput}

\end{sphinxuseclass}
\begin{sphinxuseclass}{cell}\begin{sphinxVerbatimInput}

\begin{sphinxuseclass}{cell_input}
\begin{sphinxVerbatim}[commandchars=\\\{\}]
\PYG{n}{max\PYGZus{}profit}\PYG{p}{(}\PYG{n}{prices}\PYG{o}{=}\PYG{p}{[}\PYG{l+m+mi}{7}\PYG{p}{,}\PYG{l+m+mi}{6}\PYG{p}{,}\PYG{l+m+mi}{4}\PYG{p}{,}\PYG{l+m+mi}{3}\PYG{p}{,}\PYG{l+m+mi}{1}\PYG{p}{]}\PYG{p}{)}
\end{sphinxVerbatim}

\end{sphinxuseclass}\end{sphinxVerbatimInput}
\begin{sphinxVerbatimOutput}

\begin{sphinxuseclass}{cell_output}
\begin{sphinxVerbatim}[commandchars=\\\{\}]
0
\end{sphinxVerbatim}

\end{sphinxuseclass}\end{sphinxVerbatimOutput}

\end{sphinxuseclass}
\sphinxstepscope


\chapter{McKinney Chapter 3 \sphinxhyphen{} Built\sphinxhyphen{}In Data Structures, Functions, and Files}
\label{\detokenize{mckinney_03_lecture:mckinney-chapter-3-built-in-data-structures-functions-and-files}}\label{\detokenize{mckinney_03_lecture::doc}}

\section{Introduction}
\label{\detokenize{mckinney_03_lecture:introduction}}
\sphinxAtStartPar
We must understand Python’s core functionality to fully use NumPy and pandas.
Chapter 3 of Wes McKinney’s \sphinxhref{https://wesmckinney.com/book/}{\sphinxstyleemphasis{Python for Data Analysis}} discusses Python’s core functionality.
We will focus on the following:
\begin{enumerate}
\sphinxsetlistlabels{\arabic}{enumi}{enumii}{}{.}%
\item {} 
\sphinxAtStartPar
Data structures
\begin{enumerate}
\sphinxsetlistlabels{\arabic}{enumii}{enumiii}{}{.}%
\item {} 
\sphinxAtStartPar
tuples

\item {} 
\sphinxAtStartPar
lists

\item {} 
\sphinxAtStartPar
dicts (also known as dictionaries)

\item {} 
\sphinxAtStartPar
\sphinxstyleemphasis{we will ignore sets}

\end{enumerate}

\item {} 
\sphinxAtStartPar
List comprehensions

\item {} 
\sphinxAtStartPar
Functions
\begin{enumerate}
\sphinxsetlistlabels{\arabic}{enumii}{enumiii}{}{.}%
\item {} 
\sphinxAtStartPar
Returning multiple values

\item {} 
\sphinxAtStartPar
Using anonymous functions

\end{enumerate}

\end{enumerate}

\sphinxAtStartPar
\sphinxstyleemphasis{\sphinxstylestrong{Note:}}
Indented block quotes are from McKinney unless otherwise indicated.
The section numbers here differ from McKinney because we will only discuss some topics.


\section{Data Structures and Sequences}
\label{\detokenize{mckinney_03_lecture:data-structures-and-sequences}}\begin{quote}

\sphinxAtStartPar
Python’s data structures are simple but powerful. Mastering their use is a critical part
of becoming a proficient Python programmer.
\end{quote}


\subsection{Tuple}
\label{\detokenize{mckinney_03_lecture:tuple}}\begin{quote}

\sphinxAtStartPar
A tuple is a fixed\sphinxhyphen{}length, immutable sequence of Python objects.
\end{quote}

\sphinxAtStartPar
We cannot change a tuple after we create it because tuples are immutable.
A tuple is ordered, so we can subset or slice it with a numerical index.
We will surround tuples with parentheses but the parentheses are not always required.

\begin{sphinxuseclass}{cell}\begin{sphinxVerbatimInput}

\begin{sphinxuseclass}{cell_input}
\begin{sphinxVerbatim}[commandchars=\\\{\}]
\PYG{n}{tup} \PYG{o}{=} \PYG{p}{(}\PYG{l+m+mi}{4}\PYG{p}{,} \PYG{l+m+mi}{5}\PYG{p}{,} \PYG{l+m+mi}{6}\PYG{p}{)}
\end{sphinxVerbatim}

\end{sphinxuseclass}\end{sphinxVerbatimInput}

\end{sphinxuseclass}
\sphinxAtStartPar
\sphinxstyleemphasis{\sphinxstylestrong{Python is zero\sphinxhyphen{}indexed, so zero accesses the first element in \sphinxcode{\sphinxupquote{tup}}!}}

\begin{sphinxuseclass}{cell}\begin{sphinxVerbatimInput}

\begin{sphinxuseclass}{cell_input}
\begin{sphinxVerbatim}[commandchars=\\\{\}]
\PYG{n}{tup}\PYG{p}{[}\PYG{l+m+mi}{0}\PYG{p}{]}
\end{sphinxVerbatim}

\end{sphinxuseclass}\end{sphinxVerbatimInput}
\begin{sphinxVerbatimOutput}

\begin{sphinxuseclass}{cell_output}
\begin{sphinxVerbatim}[commandchars=\\\{\}]
4
\end{sphinxVerbatim}

\end{sphinxuseclass}\end{sphinxVerbatimOutput}

\end{sphinxuseclass}
\begin{sphinxuseclass}{cell}\begin{sphinxVerbatimInput}

\begin{sphinxuseclass}{cell_input}
\begin{sphinxVerbatim}[commandchars=\\\{\}]
\PYG{n}{tup}\PYG{p}{[}\PYG{l+m+mi}{1}\PYG{p}{]}
\end{sphinxVerbatim}

\end{sphinxuseclass}\end{sphinxVerbatimInput}
\begin{sphinxVerbatimOutput}

\begin{sphinxuseclass}{cell_output}
\begin{sphinxVerbatim}[commandchars=\\\{\}]
5
\end{sphinxVerbatim}

\end{sphinxuseclass}\end{sphinxVerbatimOutput}

\end{sphinxuseclass}
\begin{sphinxuseclass}{cell}\begin{sphinxVerbatimInput}

\begin{sphinxuseclass}{cell_input}
\begin{sphinxVerbatim}[commandchars=\\\{\}]
\PYG{n}{nested\PYGZus{}tup} \PYG{o}{=} \PYG{p}{(}\PYG{p}{(}\PYG{l+m+mi}{4}\PYG{p}{,} \PYG{l+m+mi}{5}\PYG{p}{,} \PYG{l+m+mi}{6}\PYG{p}{)}\PYG{p}{,} \PYG{p}{(}\PYG{l+m+mi}{7}\PYG{p}{,} \PYG{l+m+mi}{8}\PYG{p}{)}\PYG{p}{)}
\end{sphinxVerbatim}

\end{sphinxuseclass}\end{sphinxVerbatimInput}

\end{sphinxuseclass}
\sphinxAtStartPar
\sphinxstyleemphasis{\sphinxstylestrong{Python is zero\sphinxhyphen{}indexed!}}

\begin{sphinxuseclass}{cell}\begin{sphinxVerbatimInput}

\begin{sphinxuseclass}{cell_input}
\begin{sphinxVerbatim}[commandchars=\\\{\}]
\PYG{n}{nested\PYGZus{}tup}\PYG{p}{[}\PYG{l+m+mi}{0}\PYG{p}{]}
\end{sphinxVerbatim}

\end{sphinxuseclass}\end{sphinxVerbatimInput}
\begin{sphinxVerbatimOutput}

\begin{sphinxuseclass}{cell_output}
\begin{sphinxVerbatim}[commandchars=\\\{\}]
(4, 5, 6)
\end{sphinxVerbatim}

\end{sphinxuseclass}\end{sphinxVerbatimOutput}

\end{sphinxuseclass}
\begin{sphinxuseclass}{cell}\begin{sphinxVerbatimInput}

\begin{sphinxuseclass}{cell_input}
\begin{sphinxVerbatim}[commandchars=\\\{\}]
\PYG{n}{nested\PYGZus{}tup}\PYG{p}{[}\PYG{l+m+mi}{0}\PYG{p}{]}\PYG{p}{[}\PYG{l+m+mi}{0}\PYG{p}{]}
\end{sphinxVerbatim}

\end{sphinxuseclass}\end{sphinxVerbatimInput}
\begin{sphinxVerbatimOutput}

\begin{sphinxuseclass}{cell_output}
\begin{sphinxVerbatim}[commandchars=\\\{\}]
4
\end{sphinxVerbatim}

\end{sphinxuseclass}\end{sphinxVerbatimOutput}

\end{sphinxuseclass}
\begin{sphinxuseclass}{cell}\begin{sphinxVerbatimInput}

\begin{sphinxuseclass}{cell_input}
\begin{sphinxVerbatim}[commandchars=\\\{\}]
\PYG{n}{tup} \PYG{o}{=} \PYG{n+nb}{tuple}\PYG{p}{(}\PYG{l+s+s1}{\PYGZsq{}}\PYG{l+s+s1}{string}\PYG{l+s+s1}{\PYGZsq{}}\PYG{p}{)}
\end{sphinxVerbatim}

\end{sphinxuseclass}\end{sphinxVerbatimInput}

\end{sphinxuseclass}
\begin{sphinxuseclass}{cell}\begin{sphinxVerbatimInput}

\begin{sphinxuseclass}{cell_input}
\begin{sphinxVerbatim}[commandchars=\\\{\}]
\PYG{n}{tup}
\end{sphinxVerbatim}

\end{sphinxuseclass}\end{sphinxVerbatimInput}
\begin{sphinxVerbatimOutput}

\begin{sphinxuseclass}{cell_output}
\begin{sphinxVerbatim}[commandchars=\\\{\}]
(\PYGZsq{}s\PYGZsq{}, \PYGZsq{}t\PYGZsq{}, \PYGZsq{}r\PYGZsq{}, \PYGZsq{}i\PYGZsq{}, \PYGZsq{}n\PYGZsq{}, \PYGZsq{}g\PYGZsq{})
\end{sphinxVerbatim}

\end{sphinxuseclass}\end{sphinxVerbatimOutput}

\end{sphinxuseclass}
\begin{sphinxuseclass}{cell}\begin{sphinxVerbatimInput}

\begin{sphinxuseclass}{cell_input}
\begin{sphinxVerbatim}[commandchars=\\\{\}]
\PYG{n}{tup}\PYG{p}{[}\PYG{l+m+mi}{0}\PYG{p}{]}
\end{sphinxVerbatim}

\end{sphinxuseclass}\end{sphinxVerbatimInput}
\begin{sphinxVerbatimOutput}

\begin{sphinxuseclass}{cell_output}
\begin{sphinxVerbatim}[commandchars=\\\{\}]
\PYGZsq{}s\PYGZsq{}
\end{sphinxVerbatim}

\end{sphinxuseclass}\end{sphinxVerbatimOutput}

\end{sphinxuseclass}
\begin{sphinxuseclass}{cell}\begin{sphinxVerbatimInput}

\begin{sphinxuseclass}{cell_input}
\begin{sphinxVerbatim}[commandchars=\\\{\}]
\PYG{n}{tup} \PYG{o}{=} \PYG{n+nb}{tuple}\PYG{p}{(}\PYG{p}{[}\PYG{l+s+s1}{\PYGZsq{}}\PYG{l+s+s1}{foo}\PYG{l+s+s1}{\PYGZsq{}}\PYG{p}{,} \PYG{p}{[}\PYG{l+m+mi}{1}\PYG{p}{,} \PYG{l+m+mi}{2}\PYG{p}{]}\PYG{p}{,} \PYG{k+kc}{True}\PYG{p}{]}\PYG{p}{)}
\end{sphinxVerbatim}

\end{sphinxuseclass}\end{sphinxVerbatimInput}

\end{sphinxuseclass}
\begin{sphinxuseclass}{cell}\begin{sphinxVerbatimInput}

\begin{sphinxuseclass}{cell_input}
\begin{sphinxVerbatim}[commandchars=\\\{\}]
\PYG{n}{tup}
\end{sphinxVerbatim}

\end{sphinxuseclass}\end{sphinxVerbatimInput}
\begin{sphinxVerbatimOutput}

\begin{sphinxuseclass}{cell_output}
\begin{sphinxVerbatim}[commandchars=\\\{\}]
(\PYGZsq{}foo\PYGZsq{}, [1, 2], True)
\end{sphinxVerbatim}

\end{sphinxuseclass}\end{sphinxVerbatimOutput}

\end{sphinxuseclass}
\begin{sphinxuseclass}{cell}\begin{sphinxVerbatimInput}

\begin{sphinxuseclass}{cell_input}
\begin{sphinxVerbatim}[commandchars=\\\{\}]
\PYG{c+c1}{\PYGZsh{} tup[2] = False \PYGZsh{} gives an error, because tuples are immutable (unchangeable)}
\end{sphinxVerbatim}

\end{sphinxuseclass}\end{sphinxVerbatimInput}

\end{sphinxuseclass}\begin{quote}

\sphinxAtStartPar
If an object inside a tuple is mutable, such as a list, you can modify it in\sphinxhyphen{}place.
\end{quote}

\begin{sphinxuseclass}{cell}\begin{sphinxVerbatimInput}

\begin{sphinxuseclass}{cell_input}
\begin{sphinxVerbatim}[commandchars=\\\{\}]
\PYG{n}{tup}
\end{sphinxVerbatim}

\end{sphinxuseclass}\end{sphinxVerbatimInput}
\begin{sphinxVerbatimOutput}

\begin{sphinxuseclass}{cell_output}
\begin{sphinxVerbatim}[commandchars=\\\{\}]
(\PYGZsq{}foo\PYGZsq{}, [1, 2], True)
\end{sphinxVerbatim}

\end{sphinxuseclass}\end{sphinxVerbatimOutput}

\end{sphinxuseclass}
\begin{sphinxuseclass}{cell}\begin{sphinxVerbatimInput}

\begin{sphinxuseclass}{cell_input}
\begin{sphinxVerbatim}[commandchars=\\\{\}]
\PYG{n}{tup}\PYG{p}{[}\PYG{l+m+mi}{1}\PYG{p}{]}\PYG{o}{.}\PYG{n}{append}\PYG{p}{(}\PYG{l+m+mi}{3}\PYG{p}{)}
\end{sphinxVerbatim}

\end{sphinxuseclass}\end{sphinxVerbatimInput}

\end{sphinxuseclass}
\begin{sphinxuseclass}{cell}\begin{sphinxVerbatimInput}

\begin{sphinxuseclass}{cell_input}
\begin{sphinxVerbatim}[commandchars=\\\{\}]
\PYG{n}{tup}
\end{sphinxVerbatim}

\end{sphinxuseclass}\end{sphinxVerbatimInput}
\begin{sphinxVerbatimOutput}

\begin{sphinxuseclass}{cell_output}
\begin{sphinxVerbatim}[commandchars=\\\{\}]
(\PYGZsq{}foo\PYGZsq{}, [1, 2, 3], True)
\end{sphinxVerbatim}

\end{sphinxuseclass}\end{sphinxVerbatimOutput}

\end{sphinxuseclass}\begin{quote}

\sphinxAtStartPar
You can concatenate tuples using the + operator to produce longer tuples:
\end{quote}

\sphinxAtStartPar
Tuples are immutable, but we can combine two tuples into a new tuple.

\begin{sphinxuseclass}{cell}\begin{sphinxVerbatimInput}

\begin{sphinxuseclass}{cell_input}
\begin{sphinxVerbatim}[commandchars=\\\{\}]
\PYG{p}{(}\PYG{l+m+mi}{1}\PYG{p}{,} \PYG{l+m+mi}{2}\PYG{p}{)} \PYG{o}{+} \PYG{p}{(}\PYG{l+m+mi}{1}\PYG{p}{,} \PYG{l+m+mi}{2}\PYG{p}{)}
\end{sphinxVerbatim}

\end{sphinxuseclass}\end{sphinxVerbatimInput}
\begin{sphinxVerbatimOutput}

\begin{sphinxuseclass}{cell_output}
\begin{sphinxVerbatim}[commandchars=\\\{\}]
(1, 2, 1, 2)
\end{sphinxVerbatim}

\end{sphinxuseclass}\end{sphinxVerbatimOutput}

\end{sphinxuseclass}
\begin{sphinxuseclass}{cell}\begin{sphinxVerbatimInput}

\begin{sphinxuseclass}{cell_input}
\begin{sphinxVerbatim}[commandchars=\\\{\}]
\PYG{p}{(}\PYG{l+m+mi}{4}\PYG{p}{,} \PYG{k+kc}{None}\PYG{p}{,} \PYG{l+s+s1}{\PYGZsq{}}\PYG{l+s+s1}{foo}\PYG{l+s+s1}{\PYGZsq{}}\PYG{p}{)} \PYG{o}{+} \PYG{p}{(}\PYG{l+m+mi}{6}\PYG{p}{,} \PYG{l+m+mi}{0}\PYG{p}{)} \PYG{o}{+} \PYG{p}{(}\PYG{l+s+s1}{\PYGZsq{}}\PYG{l+s+s1}{bar}\PYG{l+s+s1}{\PYGZsq{}}\PYG{p}{,}\PYG{p}{)}
\end{sphinxVerbatim}

\end{sphinxuseclass}\end{sphinxVerbatimInput}
\begin{sphinxVerbatimOutput}

\begin{sphinxuseclass}{cell_output}
\begin{sphinxVerbatim}[commandchars=\\\{\}]
(4, None, \PYGZsq{}foo\PYGZsq{}, 6, 0, \PYGZsq{}bar\PYGZsq{})
\end{sphinxVerbatim}

\end{sphinxuseclass}\end{sphinxVerbatimOutput}

\end{sphinxuseclass}\begin{quote}

\sphinxAtStartPar
Multiplying a tuple by an integer, as with lists, has the effect of concatenating together
that many copies of the tuple:
\end{quote}

\sphinxAtStartPar
This multiplication behavior is the logical extension of the addition behavior above.
The output of \sphinxcode{\sphinxupquote{tup + tup}} should be the same as the output of \sphinxcode{\sphinxupquote{2 * tup}}.

\begin{sphinxuseclass}{cell}\begin{sphinxVerbatimInput}

\begin{sphinxuseclass}{cell_input}
\begin{sphinxVerbatim}[commandchars=\\\{\}]
\PYG{p}{(}\PYG{l+s+s1}{\PYGZsq{}}\PYG{l+s+s1}{foo}\PYG{l+s+s1}{\PYGZsq{}}\PYG{p}{,} \PYG{l+s+s1}{\PYGZsq{}}\PYG{l+s+s1}{bar}\PYG{l+s+s1}{\PYGZsq{}}\PYG{p}{)} \PYG{o}{*} \PYG{l+m+mi}{4}
\end{sphinxVerbatim}

\end{sphinxuseclass}\end{sphinxVerbatimInput}
\begin{sphinxVerbatimOutput}

\begin{sphinxuseclass}{cell_output}
\begin{sphinxVerbatim}[commandchars=\\\{\}]
(\PYGZsq{}foo\PYGZsq{}, \PYGZsq{}bar\PYGZsq{}, \PYGZsq{}foo\PYGZsq{}, \PYGZsq{}bar\PYGZsq{}, \PYGZsq{}foo\PYGZsq{}, \PYGZsq{}bar\PYGZsq{}, \PYGZsq{}foo\PYGZsq{}, \PYGZsq{}bar\PYGZsq{})
\end{sphinxVerbatim}

\end{sphinxuseclass}\end{sphinxVerbatimOutput}

\end{sphinxuseclass}
\begin{sphinxuseclass}{cell}\begin{sphinxVerbatimInput}

\begin{sphinxuseclass}{cell_input}
\begin{sphinxVerbatim}[commandchars=\\\{\}]
\PYG{p}{(}\PYG{l+s+s1}{\PYGZsq{}}\PYG{l+s+s1}{foo}\PYG{l+s+s1}{\PYGZsq{}}\PYG{p}{,} \PYG{l+s+s1}{\PYGZsq{}}\PYG{l+s+s1}{bar}\PYG{l+s+s1}{\PYGZsq{}}\PYG{p}{)} \PYG{o}{+} \PYG{p}{(}\PYG{l+s+s1}{\PYGZsq{}}\PYG{l+s+s1}{foo}\PYG{l+s+s1}{\PYGZsq{}}\PYG{p}{,} \PYG{l+s+s1}{\PYGZsq{}}\PYG{l+s+s1}{bar}\PYG{l+s+s1}{\PYGZsq{}}\PYG{p}{)} \PYG{o}{+} \PYG{p}{(}\PYG{l+s+s1}{\PYGZsq{}}\PYG{l+s+s1}{foo}\PYG{l+s+s1}{\PYGZsq{}}\PYG{p}{,} \PYG{l+s+s1}{\PYGZsq{}}\PYG{l+s+s1}{bar}\PYG{l+s+s1}{\PYGZsq{}}\PYG{p}{)} \PYG{o}{+} \PYG{p}{(}\PYG{l+s+s1}{\PYGZsq{}}\PYG{l+s+s1}{foo}\PYG{l+s+s1}{\PYGZsq{}}\PYG{p}{,} \PYG{l+s+s1}{\PYGZsq{}}\PYG{l+s+s1}{bar}\PYG{l+s+s1}{\PYGZsq{}}\PYG{p}{)}
\end{sphinxVerbatim}

\end{sphinxuseclass}\end{sphinxVerbatimInput}
\begin{sphinxVerbatimOutput}

\begin{sphinxuseclass}{cell_output}
\begin{sphinxVerbatim}[commandchars=\\\{\}]
(\PYGZsq{}foo\PYGZsq{}, \PYGZsq{}bar\PYGZsq{}, \PYGZsq{}foo\PYGZsq{}, \PYGZsq{}bar\PYGZsq{}, \PYGZsq{}foo\PYGZsq{}, \PYGZsq{}bar\PYGZsq{}, \PYGZsq{}foo\PYGZsq{}, \PYGZsq{}bar\PYGZsq{})
\end{sphinxVerbatim}

\end{sphinxuseclass}\end{sphinxVerbatimOutput}

\end{sphinxuseclass}

\subsubsection{Unpacking tuples}
\label{\detokenize{mckinney_03_lecture:unpacking-tuples}}\begin{quote}

\sphinxAtStartPar
If you try to assign to a tuple\sphinxhyphen{}like expression of variables, Python will attempt to
unpack the value on the righthand side of the equals sign.
\end{quote}

\begin{sphinxuseclass}{cell}\begin{sphinxVerbatimInput}

\begin{sphinxuseclass}{cell_input}
\begin{sphinxVerbatim}[commandchars=\\\{\}]
\PYG{n}{tup} \PYG{o}{=} \PYG{p}{(}\PYG{l+m+mi}{4}\PYG{p}{,} \PYG{l+m+mi}{5}\PYG{p}{,} \PYG{l+m+mi}{6}\PYG{p}{)}
\PYG{n}{a}\PYG{p}{,} \PYG{n}{b}\PYG{p}{,} \PYG{n}{c} \PYG{o}{=} \PYG{n}{tup}
\end{sphinxVerbatim}

\end{sphinxuseclass}\end{sphinxVerbatimInput}

\end{sphinxuseclass}
\begin{sphinxuseclass}{cell}\begin{sphinxVerbatimInput}

\begin{sphinxuseclass}{cell_input}
\begin{sphinxVerbatim}[commandchars=\\\{\}]
\PYG{n}{a}
\end{sphinxVerbatim}

\end{sphinxuseclass}\end{sphinxVerbatimInput}
\begin{sphinxVerbatimOutput}

\begin{sphinxuseclass}{cell_output}
\begin{sphinxVerbatim}[commandchars=\\\{\}]
4
\end{sphinxVerbatim}

\end{sphinxuseclass}\end{sphinxVerbatimOutput}

\end{sphinxuseclass}
\begin{sphinxuseclass}{cell}\begin{sphinxVerbatimInput}

\begin{sphinxuseclass}{cell_input}
\begin{sphinxVerbatim}[commandchars=\\\{\}]
\PYG{n}{b}
\end{sphinxVerbatim}

\end{sphinxuseclass}\end{sphinxVerbatimInput}
\begin{sphinxVerbatimOutput}

\begin{sphinxuseclass}{cell_output}
\begin{sphinxVerbatim}[commandchars=\\\{\}]
5
\end{sphinxVerbatim}

\end{sphinxuseclass}\end{sphinxVerbatimOutput}

\end{sphinxuseclass}
\begin{sphinxuseclass}{cell}\begin{sphinxVerbatimInput}

\begin{sphinxuseclass}{cell_input}
\begin{sphinxVerbatim}[commandchars=\\\{\}]
\PYG{n}{c}
\end{sphinxVerbatim}

\end{sphinxuseclass}\end{sphinxVerbatimInput}
\begin{sphinxVerbatimOutput}

\begin{sphinxuseclass}{cell_output}
\begin{sphinxVerbatim}[commandchars=\\\{\}]
6
\end{sphinxVerbatim}

\end{sphinxuseclass}\end{sphinxVerbatimOutput}

\end{sphinxuseclass}
\begin{sphinxuseclass}{cell}\begin{sphinxVerbatimInput}

\begin{sphinxuseclass}{cell_input}
\begin{sphinxVerbatim}[commandchars=\\\{\}]
\PYG{p}{(}\PYG{n}{d}\PYG{p}{,} \PYG{n}{e}\PYG{p}{,} \PYG{n}{f}\PYG{p}{)} \PYG{o}{=} \PYG{p}{(}\PYG{l+m+mi}{7}\PYG{p}{,} \PYG{l+m+mi}{8}\PYG{p}{,} \PYG{l+m+mi}{9}\PYG{p}{)} \PYG{c+c1}{\PYGZsh{} the parentheses are optional but helpful!}
\end{sphinxVerbatim}

\end{sphinxuseclass}\end{sphinxVerbatimInput}

\end{sphinxuseclass}
\begin{sphinxuseclass}{cell}\begin{sphinxVerbatimInput}

\begin{sphinxuseclass}{cell_input}
\begin{sphinxVerbatim}[commandchars=\\\{\}]
\PYG{n}{d}
\end{sphinxVerbatim}

\end{sphinxuseclass}\end{sphinxVerbatimInput}
\begin{sphinxVerbatimOutput}

\begin{sphinxuseclass}{cell_output}
\begin{sphinxVerbatim}[commandchars=\\\{\}]
7
\end{sphinxVerbatim}

\end{sphinxuseclass}\end{sphinxVerbatimOutput}

\end{sphinxuseclass}
\begin{sphinxuseclass}{cell}\begin{sphinxVerbatimInput}

\begin{sphinxuseclass}{cell_input}
\begin{sphinxVerbatim}[commandchars=\\\{\}]
\PYG{n}{e}
\end{sphinxVerbatim}

\end{sphinxuseclass}\end{sphinxVerbatimInput}
\begin{sphinxVerbatimOutput}

\begin{sphinxuseclass}{cell_output}
\begin{sphinxVerbatim}[commandchars=\\\{\}]
8
\end{sphinxVerbatim}

\end{sphinxuseclass}\end{sphinxVerbatimOutput}

\end{sphinxuseclass}
\begin{sphinxuseclass}{cell}\begin{sphinxVerbatimInput}

\begin{sphinxuseclass}{cell_input}
\begin{sphinxVerbatim}[commandchars=\\\{\}]
\PYG{n}{f}
\end{sphinxVerbatim}

\end{sphinxuseclass}\end{sphinxVerbatimInput}
\begin{sphinxVerbatimOutput}

\begin{sphinxuseclass}{cell_output}
\begin{sphinxVerbatim}[commandchars=\\\{\}]
9
\end{sphinxVerbatim}

\end{sphinxuseclass}\end{sphinxVerbatimOutput}

\end{sphinxuseclass}
\begin{sphinxuseclass}{cell}\begin{sphinxVerbatimInput}

\begin{sphinxuseclass}{cell_input}
\begin{sphinxVerbatim}[commandchars=\\\{\}]
\PYG{c+c1}{\PYGZsh{} g, h = 10, 11, 12 \PYGZsh{} ValueError: too many values to unpack (expected 2)}
\end{sphinxVerbatim}

\end{sphinxuseclass}\end{sphinxVerbatimInput}

\end{sphinxuseclass}
\sphinxAtStartPar
We can unpack nested tuples!

\begin{sphinxuseclass}{cell}\begin{sphinxVerbatimInput}

\begin{sphinxuseclass}{cell_input}
\begin{sphinxVerbatim}[commandchars=\\\{\}]
\PYG{n}{tup} \PYG{o}{=} \PYG{l+m+mi}{4}\PYG{p}{,} \PYG{l+m+mi}{5}\PYG{p}{,} \PYG{p}{(}\PYG{l+m+mi}{6}\PYG{p}{,} \PYG{l+m+mi}{7}\PYG{p}{)}
\PYG{n}{a}\PYG{p}{,} \PYG{n}{b}\PYG{p}{,} \PYG{p}{(}\PYG{n}{c}\PYG{p}{,} \PYG{n}{d}\PYG{p}{)} \PYG{o}{=} \PYG{n}{tup}
\end{sphinxVerbatim}

\end{sphinxuseclass}\end{sphinxVerbatimInput}

\end{sphinxuseclass}
\begin{sphinxuseclass}{cell}\begin{sphinxVerbatimInput}

\begin{sphinxuseclass}{cell_input}
\begin{sphinxVerbatim}[commandchars=\\\{\}]
\PYG{n}{a}
\end{sphinxVerbatim}

\end{sphinxuseclass}\end{sphinxVerbatimInput}
\begin{sphinxVerbatimOutput}

\begin{sphinxuseclass}{cell_output}
\begin{sphinxVerbatim}[commandchars=\\\{\}]
4
\end{sphinxVerbatim}

\end{sphinxuseclass}\end{sphinxVerbatimOutput}

\end{sphinxuseclass}
\begin{sphinxuseclass}{cell}\begin{sphinxVerbatimInput}

\begin{sphinxuseclass}{cell_input}
\begin{sphinxVerbatim}[commandchars=\\\{\}]
\PYG{n}{b}
\end{sphinxVerbatim}

\end{sphinxuseclass}\end{sphinxVerbatimInput}
\begin{sphinxVerbatimOutput}

\begin{sphinxuseclass}{cell_output}
\begin{sphinxVerbatim}[commandchars=\\\{\}]
5
\end{sphinxVerbatim}

\end{sphinxuseclass}\end{sphinxVerbatimOutput}

\end{sphinxuseclass}
\begin{sphinxuseclass}{cell}\begin{sphinxVerbatimInput}

\begin{sphinxuseclass}{cell_input}
\begin{sphinxVerbatim}[commandchars=\\\{\}]
\PYG{n}{c}
\end{sphinxVerbatim}

\end{sphinxuseclass}\end{sphinxVerbatimInput}
\begin{sphinxVerbatimOutput}

\begin{sphinxuseclass}{cell_output}
\begin{sphinxVerbatim}[commandchars=\\\{\}]
6
\end{sphinxVerbatim}

\end{sphinxuseclass}\end{sphinxVerbatimOutput}

\end{sphinxuseclass}
\begin{sphinxuseclass}{cell}\begin{sphinxVerbatimInput}

\begin{sphinxuseclass}{cell_input}
\begin{sphinxVerbatim}[commandchars=\\\{\}]
\PYG{n}{d}
\end{sphinxVerbatim}

\end{sphinxuseclass}\end{sphinxVerbatimInput}
\begin{sphinxVerbatimOutput}

\begin{sphinxuseclass}{cell_output}
\begin{sphinxVerbatim}[commandchars=\\\{\}]
7
\end{sphinxVerbatim}

\end{sphinxuseclass}\end{sphinxVerbatimOutput}

\end{sphinxuseclass}

\subsubsection{Tuple methods}
\label{\detokenize{mckinney_03_lecture:tuple-methods}}\begin{quote}

\sphinxAtStartPar
Since the size and contents of a tuple cannot be modified, it is very light on instance
methods. A particularly useful one (also available on lists) is count, which counts the
number of occurrences of a value.
\end{quote}

\begin{sphinxuseclass}{cell}\begin{sphinxVerbatimInput}

\begin{sphinxuseclass}{cell_input}
\begin{sphinxVerbatim}[commandchars=\\\{\}]
\PYG{n}{a} \PYG{o}{=} \PYG{p}{(}\PYG{l+m+mi}{1}\PYG{p}{,} \PYG{l+m+mi}{2}\PYG{p}{,} \PYG{l+m+mi}{2}\PYG{p}{,} \PYG{l+m+mi}{2}\PYG{p}{,} \PYG{l+m+mi}{3}\PYG{p}{,} \PYG{l+m+mi}{4}\PYG{p}{,} \PYG{l+m+mi}{2}\PYG{p}{)}
\PYG{n}{a}\PYG{o}{.}\PYG{n}{count}\PYG{p}{(}\PYG{l+m+mi}{2}\PYG{p}{)}
\end{sphinxVerbatim}

\end{sphinxuseclass}\end{sphinxVerbatimInput}
\begin{sphinxVerbatimOutput}

\begin{sphinxuseclass}{cell_output}
\begin{sphinxVerbatim}[commandchars=\\\{\}]
4
\end{sphinxVerbatim}

\end{sphinxuseclass}\end{sphinxVerbatimOutput}

\end{sphinxuseclass}

\subsection{List}
\label{\detokenize{mckinney_03_lecture:list}}\begin{quote}

\sphinxAtStartPar
In contrast with tuples, lists are variable\sphinxhyphen{}length and their contents can be modified in\sphinxhyphen{}place. You can define them using square brackets {[} {]} or using the list type function.
\end{quote}

\begin{sphinxuseclass}{cell}\begin{sphinxVerbatimInput}

\begin{sphinxuseclass}{cell_input}
\begin{sphinxVerbatim}[commandchars=\\\{\}]
\PYG{n}{a\PYGZus{}list} \PYG{o}{=} \PYG{p}{[}\PYG{l+m+mi}{2}\PYG{p}{,} \PYG{l+m+mi}{3}\PYG{p}{,} \PYG{l+m+mi}{7}\PYG{p}{,} \PYG{k+kc}{None}\PYG{p}{]}
\PYG{n}{tup} \PYG{o}{=} \PYG{p}{(}\PYG{l+s+s1}{\PYGZsq{}}\PYG{l+s+s1}{foo}\PYG{l+s+s1}{\PYGZsq{}}\PYG{p}{,} \PYG{l+s+s1}{\PYGZsq{}}\PYG{l+s+s1}{bar}\PYG{l+s+s1}{\PYGZsq{}}\PYG{p}{,} \PYG{l+s+s1}{\PYGZsq{}}\PYG{l+s+s1}{baz}\PYG{l+s+s1}{\PYGZsq{}}\PYG{p}{)}
\PYG{n}{b\PYGZus{}list} \PYG{o}{=} \PYG{n+nb}{list}\PYG{p}{(}\PYG{n}{tup}\PYG{p}{)}
\end{sphinxVerbatim}

\end{sphinxuseclass}\end{sphinxVerbatimInput}

\end{sphinxuseclass}
\begin{sphinxuseclass}{cell}\begin{sphinxVerbatimInput}

\begin{sphinxuseclass}{cell_input}
\begin{sphinxVerbatim}[commandchars=\\\{\}]
\PYG{n}{a\PYGZus{}list}
\end{sphinxVerbatim}

\end{sphinxuseclass}\end{sphinxVerbatimInput}
\begin{sphinxVerbatimOutput}

\begin{sphinxuseclass}{cell_output}
\begin{sphinxVerbatim}[commandchars=\\\{\}]
[2, 3, 7, None]
\end{sphinxVerbatim}

\end{sphinxuseclass}\end{sphinxVerbatimOutput}

\end{sphinxuseclass}
\begin{sphinxuseclass}{cell}\begin{sphinxVerbatimInput}

\begin{sphinxuseclass}{cell_input}
\begin{sphinxVerbatim}[commandchars=\\\{\}]
\PYG{n}{b\PYGZus{}list}
\end{sphinxVerbatim}

\end{sphinxuseclass}\end{sphinxVerbatimInput}
\begin{sphinxVerbatimOutput}

\begin{sphinxuseclass}{cell_output}
\begin{sphinxVerbatim}[commandchars=\\\{\}]
[\PYGZsq{}foo\PYGZsq{}, \PYGZsq{}bar\PYGZsq{}, \PYGZsq{}baz\PYGZsq{}]
\end{sphinxVerbatim}

\end{sphinxuseclass}\end{sphinxVerbatimOutput}

\end{sphinxuseclass}
\sphinxAtStartPar
\sphinxstyleemphasis{\sphinxstylestrong{Pyhon is zero\sphinxhyphen{}indexed!}}

\begin{sphinxuseclass}{cell}\begin{sphinxVerbatimInput}

\begin{sphinxuseclass}{cell_input}
\begin{sphinxVerbatim}[commandchars=\\\{\}]
\PYG{n}{a\PYGZus{}list}\PYG{p}{[}\PYG{l+m+mi}{0}\PYG{p}{]}
\end{sphinxVerbatim}

\end{sphinxuseclass}\end{sphinxVerbatimInput}
\begin{sphinxVerbatimOutput}

\begin{sphinxuseclass}{cell_output}
\begin{sphinxVerbatim}[commandchars=\\\{\}]
2
\end{sphinxVerbatim}

\end{sphinxuseclass}\end{sphinxVerbatimOutput}

\end{sphinxuseclass}

\subsubsection{Adding and removing elements}
\label{\detokenize{mckinney_03_lecture:adding-and-removing-elements}}\begin{quote}

\sphinxAtStartPar
Elements can be appended to the end of the list with the append method.
\end{quote}

\sphinxAtStartPar
The \sphinxcode{\sphinxupquote{.append()}} method appends an element to the list \sphinxstyleemphasis{in place} without reassigning the list.

\begin{sphinxuseclass}{cell}\begin{sphinxVerbatimInput}

\begin{sphinxuseclass}{cell_input}
\begin{sphinxVerbatim}[commandchars=\\\{\}]
\PYG{n}{b\PYGZus{}list}\PYG{o}{.}\PYG{n}{append}\PYG{p}{(}\PYG{l+s+s1}{\PYGZsq{}}\PYG{l+s+s1}{dwarf}\PYG{l+s+s1}{\PYGZsq{}}\PYG{p}{)}
\end{sphinxVerbatim}

\end{sphinxuseclass}\end{sphinxVerbatimInput}

\end{sphinxuseclass}
\begin{sphinxuseclass}{cell}\begin{sphinxVerbatimInput}

\begin{sphinxuseclass}{cell_input}
\begin{sphinxVerbatim}[commandchars=\\\{\}]
\PYG{n}{b\PYGZus{}list}
\end{sphinxVerbatim}

\end{sphinxuseclass}\end{sphinxVerbatimInput}
\begin{sphinxVerbatimOutput}

\begin{sphinxuseclass}{cell_output}
\begin{sphinxVerbatim}[commandchars=\\\{\}]
[\PYGZsq{}foo\PYGZsq{}, \PYGZsq{}bar\PYGZsq{}, \PYGZsq{}baz\PYGZsq{}, \PYGZsq{}dwarf\PYGZsq{}]
\end{sphinxVerbatim}

\end{sphinxuseclass}\end{sphinxVerbatimOutput}

\end{sphinxuseclass}\begin{quote}

\sphinxAtStartPar
Using insert you can insert an element at a specific location in the list.
The insertion index must be between 0 and the length of the list, inclusive.
\end{quote}

\begin{sphinxuseclass}{cell}\begin{sphinxVerbatimInput}

\begin{sphinxuseclass}{cell_input}
\begin{sphinxVerbatim}[commandchars=\\\{\}]
\PYG{n}{b\PYGZus{}list}\PYG{o}{.}\PYG{n}{insert}\PYG{p}{(}\PYG{l+m+mi}{1}\PYG{p}{,} \PYG{l+s+s1}{\PYGZsq{}}\PYG{l+s+s1}{red}\PYG{l+s+s1}{\PYGZsq{}}\PYG{p}{)}
\end{sphinxVerbatim}

\end{sphinxuseclass}\end{sphinxVerbatimInput}

\end{sphinxuseclass}
\begin{sphinxuseclass}{cell}\begin{sphinxVerbatimInput}

\begin{sphinxuseclass}{cell_input}
\begin{sphinxVerbatim}[commandchars=\\\{\}]
\PYG{n}{b\PYGZus{}list}
\end{sphinxVerbatim}

\end{sphinxuseclass}\end{sphinxVerbatimInput}
\begin{sphinxVerbatimOutput}

\begin{sphinxuseclass}{cell_output}
\begin{sphinxVerbatim}[commandchars=\\\{\}]
[\PYGZsq{}foo\PYGZsq{}, \PYGZsq{}red\PYGZsq{}, \PYGZsq{}bar\PYGZsq{}, \PYGZsq{}baz\PYGZsq{}, \PYGZsq{}dwarf\PYGZsq{}]
\end{sphinxVerbatim}

\end{sphinxuseclass}\end{sphinxVerbatimOutput}

\end{sphinxuseclass}
\begin{sphinxuseclass}{cell}\begin{sphinxVerbatimInput}

\begin{sphinxuseclass}{cell_input}
\begin{sphinxVerbatim}[commandchars=\\\{\}]
\PYG{n}{b\PYGZus{}list}\PYG{o}{.}\PYG{n}{index}\PYG{p}{(}\PYG{l+s+s1}{\PYGZsq{}}\PYG{l+s+s1}{red}\PYG{l+s+s1}{\PYGZsq{}}\PYG{p}{)}
\end{sphinxVerbatim}

\end{sphinxuseclass}\end{sphinxVerbatimInput}
\begin{sphinxVerbatimOutput}

\begin{sphinxuseclass}{cell_output}
\begin{sphinxVerbatim}[commandchars=\\\{\}]
1
\end{sphinxVerbatim}

\end{sphinxuseclass}\end{sphinxVerbatimOutput}

\end{sphinxuseclass}
\begin{sphinxuseclass}{cell}\begin{sphinxVerbatimInput}

\begin{sphinxuseclass}{cell_input}
\begin{sphinxVerbatim}[commandchars=\\\{\}]
\PYG{n}{b\PYGZus{}list}\PYG{p}{[}\PYG{n}{b\PYGZus{}list}\PYG{o}{.}\PYG{n}{index}\PYG{p}{(}\PYG{l+s+s1}{\PYGZsq{}}\PYG{l+s+s1}{red}\PYG{l+s+s1}{\PYGZsq{}}\PYG{p}{)}\PYG{p}{]} \PYG{o}{=} \PYG{l+s+s1}{\PYGZsq{}}\PYG{l+s+s1}{blue}\PYG{l+s+s1}{\PYGZsq{}}
\end{sphinxVerbatim}

\end{sphinxuseclass}\end{sphinxVerbatimInput}

\end{sphinxuseclass}
\begin{sphinxuseclass}{cell}\begin{sphinxVerbatimInput}

\begin{sphinxuseclass}{cell_input}
\begin{sphinxVerbatim}[commandchars=\\\{\}]
\PYG{n}{b\PYGZus{}list}
\end{sphinxVerbatim}

\end{sphinxuseclass}\end{sphinxVerbatimInput}
\begin{sphinxVerbatimOutput}

\begin{sphinxuseclass}{cell_output}
\begin{sphinxVerbatim}[commandchars=\\\{\}]
[\PYGZsq{}foo\PYGZsq{}, \PYGZsq{}blue\PYGZsq{}, \PYGZsq{}bar\PYGZsq{}, \PYGZsq{}baz\PYGZsq{}, \PYGZsq{}dwarf\PYGZsq{}]
\end{sphinxVerbatim}

\end{sphinxuseclass}\end{sphinxVerbatimOutput}

\end{sphinxuseclass}\begin{quote}

\sphinxAtStartPar
The inverse operation to insert is pop, which removes and returns an element at a
particular index.
\end{quote}

\begin{sphinxuseclass}{cell}\begin{sphinxVerbatimInput}

\begin{sphinxuseclass}{cell_input}
\begin{sphinxVerbatim}[commandchars=\\\{\}]
\PYG{n}{b\PYGZus{}list}\PYG{o}{.}\PYG{n}{pop}\PYG{p}{(}\PYG{l+m+mi}{2}\PYG{p}{)}
\end{sphinxVerbatim}

\end{sphinxuseclass}\end{sphinxVerbatimInput}
\begin{sphinxVerbatimOutput}

\begin{sphinxuseclass}{cell_output}
\begin{sphinxVerbatim}[commandchars=\\\{\}]
\PYGZsq{}bar\PYGZsq{}
\end{sphinxVerbatim}

\end{sphinxuseclass}\end{sphinxVerbatimOutput}

\end{sphinxuseclass}
\begin{sphinxuseclass}{cell}\begin{sphinxVerbatimInput}

\begin{sphinxuseclass}{cell_input}
\begin{sphinxVerbatim}[commandchars=\\\{\}]
\PYG{n}{b\PYGZus{}list}
\end{sphinxVerbatim}

\end{sphinxuseclass}\end{sphinxVerbatimInput}
\begin{sphinxVerbatimOutput}

\begin{sphinxuseclass}{cell_output}
\begin{sphinxVerbatim}[commandchars=\\\{\}]
[\PYGZsq{}foo\PYGZsq{}, \PYGZsq{}blue\PYGZsq{}, \PYGZsq{}baz\PYGZsq{}, \PYGZsq{}dwarf\PYGZsq{}]
\end{sphinxVerbatim}

\end{sphinxuseclass}\end{sphinxVerbatimOutput}

\end{sphinxuseclass}
\sphinxAtStartPar
Note that \sphinxcode{\sphinxupquote{.pop(2)}} removes the 2 element.
If we do not want to remove the 2 element, we should use \sphinxcode{\sphinxupquote{{[}2{]}}} to access an element without removing it.
\begin{quote}

\sphinxAtStartPar
Elements can be removed by value with remove, which locates the first such value and removes it from the list.
\end{quote}

\begin{sphinxuseclass}{cell}\begin{sphinxVerbatimInput}

\begin{sphinxuseclass}{cell_input}
\begin{sphinxVerbatim}[commandchars=\\\{\}]
\PYG{n}{b\PYGZus{}list}\PYG{o}{.}\PYG{n}{append}\PYG{p}{(}\PYG{l+s+s1}{\PYGZsq{}}\PYG{l+s+s1}{foo}\PYG{l+s+s1}{\PYGZsq{}}\PYG{p}{)}
\end{sphinxVerbatim}

\end{sphinxuseclass}\end{sphinxVerbatimInput}

\end{sphinxuseclass}
\begin{sphinxuseclass}{cell}\begin{sphinxVerbatimInput}

\begin{sphinxuseclass}{cell_input}
\begin{sphinxVerbatim}[commandchars=\\\{\}]
\PYG{n}{b\PYGZus{}list}
\end{sphinxVerbatim}

\end{sphinxuseclass}\end{sphinxVerbatimInput}
\begin{sphinxVerbatimOutput}

\begin{sphinxuseclass}{cell_output}
\begin{sphinxVerbatim}[commandchars=\\\{\}]
[\PYGZsq{}foo\PYGZsq{}, \PYGZsq{}blue\PYGZsq{}, \PYGZsq{}baz\PYGZsq{}, \PYGZsq{}dwarf\PYGZsq{}, \PYGZsq{}foo\PYGZsq{}]
\end{sphinxVerbatim}

\end{sphinxuseclass}\end{sphinxVerbatimOutput}

\end{sphinxuseclass}
\begin{sphinxuseclass}{cell}\begin{sphinxVerbatimInput}

\begin{sphinxuseclass}{cell_input}
\begin{sphinxVerbatim}[commandchars=\\\{\}]
\PYG{n}{b\PYGZus{}list}\PYG{o}{.}\PYG{n}{remove}\PYG{p}{(}\PYG{l+s+s1}{\PYGZsq{}}\PYG{l+s+s1}{foo}\PYG{l+s+s1}{\PYGZsq{}}\PYG{p}{)}
\end{sphinxVerbatim}

\end{sphinxuseclass}\end{sphinxVerbatimInput}

\end{sphinxuseclass}
\begin{sphinxuseclass}{cell}\begin{sphinxVerbatimInput}

\begin{sphinxuseclass}{cell_input}
\begin{sphinxVerbatim}[commandchars=\\\{\}]
\PYG{n}{b\PYGZus{}list}
\end{sphinxVerbatim}

\end{sphinxuseclass}\end{sphinxVerbatimInput}
\begin{sphinxVerbatimOutput}

\begin{sphinxuseclass}{cell_output}
\begin{sphinxVerbatim}[commandchars=\\\{\}]
[\PYGZsq{}blue\PYGZsq{}, \PYGZsq{}baz\PYGZsq{}, \PYGZsq{}dwarf\PYGZsq{}, \PYGZsq{}foo\PYGZsq{}]
\end{sphinxVerbatim}

\end{sphinxuseclass}\end{sphinxVerbatimOutput}

\end{sphinxuseclass}
\begin{sphinxuseclass}{cell}\begin{sphinxVerbatimInput}

\begin{sphinxuseclass}{cell_input}
\begin{sphinxVerbatim}[commandchars=\\\{\}]
\PYG{l+s+s1}{\PYGZsq{}}\PYG{l+s+s1}{dwarf}\PYG{l+s+s1}{\PYGZsq{}} \PYG{o+ow}{in} \PYG{n}{b\PYGZus{}list}
\end{sphinxVerbatim}

\end{sphinxuseclass}\end{sphinxVerbatimInput}
\begin{sphinxVerbatimOutput}

\begin{sphinxuseclass}{cell_output}
\begin{sphinxVerbatim}[commandchars=\\\{\}]
True
\end{sphinxVerbatim}

\end{sphinxuseclass}\end{sphinxVerbatimOutput}

\end{sphinxuseclass}
\begin{sphinxuseclass}{cell}\begin{sphinxVerbatimInput}

\begin{sphinxuseclass}{cell_input}
\begin{sphinxVerbatim}[commandchars=\\\{\}]
\PYG{l+s+s1}{\PYGZsq{}}\PYG{l+s+s1}{dwarf}\PYG{l+s+s1}{\PYGZsq{}} \PYG{o+ow}{not} \PYG{o+ow}{in} \PYG{n}{b\PYGZus{}list}
\end{sphinxVerbatim}

\end{sphinxuseclass}\end{sphinxVerbatimInput}
\begin{sphinxVerbatimOutput}

\begin{sphinxuseclass}{cell_output}
\begin{sphinxVerbatim}[commandchars=\\\{\}]
False
\end{sphinxVerbatim}

\end{sphinxuseclass}\end{sphinxVerbatimOutput}

\end{sphinxuseclass}

\subsubsection{Concatenating and combining lists}
\label{\detokenize{mckinney_03_lecture:concatenating-and-combining-lists}}\begin{quote}

\sphinxAtStartPar
Similar to tuples, adding two lists together with + concatenates them.
\end{quote}

\begin{sphinxuseclass}{cell}\begin{sphinxVerbatimInput}

\begin{sphinxuseclass}{cell_input}
\begin{sphinxVerbatim}[commandchars=\\\{\}]
\PYG{p}{[}\PYG{l+m+mi}{4}\PYG{p}{,} \PYG{k+kc}{None}\PYG{p}{,} \PYG{l+s+s1}{\PYGZsq{}}\PYG{l+s+s1}{foo}\PYG{l+s+s1}{\PYGZsq{}}\PYG{p}{]} \PYG{o}{+} \PYG{p}{[}\PYG{l+m+mi}{7}\PYG{p}{,} \PYG{l+m+mi}{8}\PYG{p}{,} \PYG{p}{(}\PYG{l+m+mi}{2}\PYG{p}{,} \PYG{l+m+mi}{3}\PYG{p}{)}\PYG{p}{]}
\end{sphinxVerbatim}

\end{sphinxuseclass}\end{sphinxVerbatimInput}
\begin{sphinxVerbatimOutput}

\begin{sphinxuseclass}{cell_output}
\begin{sphinxVerbatim}[commandchars=\\\{\}]
[4, None, \PYGZsq{}foo\PYGZsq{}, 7, 8, (2, 3)]
\end{sphinxVerbatim}

\end{sphinxuseclass}\end{sphinxVerbatimOutput}

\end{sphinxuseclass}
\sphinxAtStartPar
The \sphinxcode{\sphinxupquote{.append()}} method adds its argument as the last element in a list.

\begin{sphinxuseclass}{cell}\begin{sphinxVerbatimInput}

\begin{sphinxuseclass}{cell_input}
\begin{sphinxVerbatim}[commandchars=\\\{\}]
\PYG{n}{xx} \PYG{o}{=} \PYG{p}{[}\PYG{l+m+mi}{4}\PYG{p}{,} \PYG{k+kc}{None}\PYG{p}{,} \PYG{l+s+s1}{\PYGZsq{}}\PYG{l+s+s1}{foo}\PYG{l+s+s1}{\PYGZsq{}}\PYG{p}{]}
\PYG{n}{xx}\PYG{o}{.}\PYG{n}{append}\PYG{p}{(}\PYG{p}{[}\PYG{l+m+mi}{7}\PYG{p}{,} \PYG{l+m+mi}{8}\PYG{p}{,} \PYG{p}{(}\PYG{l+m+mi}{2}\PYG{p}{,} \PYG{l+m+mi}{3}\PYG{p}{)}\PYG{p}{]}\PYG{p}{)}
\end{sphinxVerbatim}

\end{sphinxuseclass}\end{sphinxVerbatimInput}

\end{sphinxuseclass}
\begin{sphinxuseclass}{cell}\begin{sphinxVerbatimInput}

\begin{sphinxuseclass}{cell_input}
\begin{sphinxVerbatim}[commandchars=\\\{\}]
\PYG{n}{xx}
\end{sphinxVerbatim}

\end{sphinxuseclass}\end{sphinxVerbatimInput}
\begin{sphinxVerbatimOutput}

\begin{sphinxuseclass}{cell_output}
\begin{sphinxVerbatim}[commandchars=\\\{\}]
[4, None, \PYGZsq{}foo\PYGZsq{}, [7, 8, (2, 3)]]
\end{sphinxVerbatim}

\end{sphinxuseclass}\end{sphinxVerbatimOutput}

\end{sphinxuseclass}\begin{quote}

\sphinxAtStartPar
If you have a list already defined, you can append multiple elements to it using the extend method.
\end{quote}

\begin{sphinxuseclass}{cell}\begin{sphinxVerbatimInput}

\begin{sphinxuseclass}{cell_input}
\begin{sphinxVerbatim}[commandchars=\\\{\}]
\PYG{n}{x} \PYG{o}{=} \PYG{p}{[}\PYG{l+m+mi}{4}\PYG{p}{,} \PYG{k+kc}{None}\PYG{p}{,} \PYG{l+s+s1}{\PYGZsq{}}\PYG{l+s+s1}{foo}\PYG{l+s+s1}{\PYGZsq{}}\PYG{p}{]}
\PYG{n}{x}\PYG{o}{.}\PYG{n}{extend}\PYG{p}{(}\PYG{p}{[}\PYG{l+m+mi}{7}\PYG{p}{,} \PYG{l+m+mi}{8}\PYG{p}{,} \PYG{p}{(}\PYG{l+m+mi}{2}\PYG{p}{,} \PYG{l+m+mi}{3}\PYG{p}{)}\PYG{p}{]}\PYG{p}{)}
\end{sphinxVerbatim}

\end{sphinxuseclass}\end{sphinxVerbatimInput}

\end{sphinxuseclass}
\begin{sphinxuseclass}{cell}\begin{sphinxVerbatimInput}

\begin{sphinxuseclass}{cell_input}
\begin{sphinxVerbatim}[commandchars=\\\{\}]
\PYG{n}{x}
\end{sphinxVerbatim}

\end{sphinxuseclass}\end{sphinxVerbatimInput}
\begin{sphinxVerbatimOutput}

\begin{sphinxuseclass}{cell_output}
\begin{sphinxVerbatim}[commandchars=\\\{\}]
[4, None, \PYGZsq{}foo\PYGZsq{}, 7, 8, (2, 3)]
\end{sphinxVerbatim}

\end{sphinxuseclass}\end{sphinxVerbatimOutput}

\end{sphinxuseclass}
\sphinxAtStartPar
\sphinxstyleemphasis{\sphinxstylestrong{Check your output! It will take you time to understand all these methods!}}


\subsubsection{Sorting}
\label{\detokenize{mckinney_03_lecture:sorting}}\begin{quote}

\sphinxAtStartPar
You can sort a list in\sphinxhyphen{}place (without creating a new object) by calling its sort function.
\end{quote}

\begin{sphinxuseclass}{cell}\begin{sphinxVerbatimInput}

\begin{sphinxuseclass}{cell_input}
\begin{sphinxVerbatim}[commandchars=\\\{\}]
\PYG{n}{a} \PYG{o}{=} \PYG{p}{[}\PYG{l+m+mi}{7}\PYG{p}{,} \PYG{l+m+mi}{2}\PYG{p}{,} \PYG{l+m+mi}{5}\PYG{p}{,} \PYG{l+m+mi}{1}\PYG{p}{,} \PYG{l+m+mi}{3}\PYG{p}{]}
\PYG{n}{a}\PYG{o}{.}\PYG{n}{sort}\PYG{p}{(}\PYG{p}{)}
\end{sphinxVerbatim}

\end{sphinxuseclass}\end{sphinxVerbatimInput}

\end{sphinxuseclass}
\begin{sphinxuseclass}{cell}\begin{sphinxVerbatimInput}

\begin{sphinxuseclass}{cell_input}
\begin{sphinxVerbatim}[commandchars=\\\{\}]
\PYG{n}{a}
\end{sphinxVerbatim}

\end{sphinxuseclass}\end{sphinxVerbatimInput}
\begin{sphinxVerbatimOutput}

\begin{sphinxuseclass}{cell_output}
\begin{sphinxVerbatim}[commandchars=\\\{\}]
[1, 2, 3, 5, 7]
\end{sphinxVerbatim}

\end{sphinxuseclass}\end{sphinxVerbatimOutput}

\end{sphinxuseclass}\begin{quote}

\sphinxAtStartPar
sort has a few options that will occasionally come in handy. One is the ability to pass a secondary sort key—that is, a function that produces a value to use to sort the objects. For example, we could sort a collection of strings by their lengths.
\end{quote}

\sphinxAtStartPar
Before you write your own solution to a problem, read the docstring (help file) of the built\sphinxhyphen{}in function.
The built\sphinxhyphen{}in function may already solve your problem faster with fewer bugs.

\begin{sphinxuseclass}{cell}\begin{sphinxVerbatimInput}

\begin{sphinxuseclass}{cell_input}
\begin{sphinxVerbatim}[commandchars=\\\{\}]
\PYG{n}{b} \PYG{o}{=} \PYG{p}{[}\PYG{l+s+s1}{\PYGZsq{}}\PYG{l+s+s1}{saw}\PYG{l+s+s1}{\PYGZsq{}}\PYG{p}{,} \PYG{l+s+s1}{\PYGZsq{}}\PYG{l+s+s1}{small}\PYG{l+s+s1}{\PYGZsq{}}\PYG{p}{,} \PYG{l+s+s1}{\PYGZsq{}}\PYG{l+s+s1}{He}\PYG{l+s+s1}{\PYGZsq{}}\PYG{p}{,} \PYG{l+s+s1}{\PYGZsq{}}\PYG{l+s+s1}{foxes}\PYG{l+s+s1}{\PYGZsq{}}\PYG{p}{,} \PYG{l+s+s1}{\PYGZsq{}}\PYG{l+s+s1}{six}\PYG{l+s+s1}{\PYGZsq{}}\PYG{p}{]}
\PYG{n}{b}\PYG{o}{.}\PYG{n}{sort}\PYG{p}{(}\PYG{p}{)}
\end{sphinxVerbatim}

\end{sphinxuseclass}\end{sphinxVerbatimInput}

\end{sphinxuseclass}
\begin{sphinxuseclass}{cell}\begin{sphinxVerbatimInput}

\begin{sphinxuseclass}{cell_input}
\begin{sphinxVerbatim}[commandchars=\\\{\}]
\PYG{n}{b} \PYG{c+c1}{\PYGZsh{} Python is case sensitive, so \PYGZdq{}He\PYGZdq{} sorts before \PYGZdq{}foxes\PYGZdq{}}
\end{sphinxVerbatim}

\end{sphinxuseclass}\end{sphinxVerbatimInput}
\begin{sphinxVerbatimOutput}

\begin{sphinxuseclass}{cell_output}
\begin{sphinxVerbatim}[commandchars=\\\{\}]
[\PYGZsq{}He\PYGZsq{}, \PYGZsq{}foxes\PYGZsq{}, \PYGZsq{}saw\PYGZsq{}, \PYGZsq{}six\PYGZsq{}, \PYGZsq{}small\PYGZsq{}]
\end{sphinxVerbatim}

\end{sphinxuseclass}\end{sphinxVerbatimOutput}

\end{sphinxuseclass}
\begin{sphinxuseclass}{cell}\begin{sphinxVerbatimInput}

\begin{sphinxuseclass}{cell_input}
\begin{sphinxVerbatim}[commandchars=\\\{\}]
\PYG{n+nb}{len}\PYG{p}{(}\PYG{n}{b}\PYG{p}{[}\PYG{l+m+mi}{0}\PYG{p}{]}\PYG{p}{)}
\end{sphinxVerbatim}

\end{sphinxuseclass}\end{sphinxVerbatimInput}
\begin{sphinxVerbatimOutput}

\begin{sphinxuseclass}{cell_output}
\begin{sphinxVerbatim}[commandchars=\\\{\}]
2
\end{sphinxVerbatim}

\end{sphinxuseclass}\end{sphinxVerbatimOutput}

\end{sphinxuseclass}
\begin{sphinxuseclass}{cell}\begin{sphinxVerbatimInput}

\begin{sphinxuseclass}{cell_input}
\begin{sphinxVerbatim}[commandchars=\\\{\}]
\PYG{n+nb}{len}\PYG{p}{(}\PYG{n}{b}\PYG{p}{[}\PYG{l+m+mi}{1}\PYG{p}{]}\PYG{p}{)}
\end{sphinxVerbatim}

\end{sphinxuseclass}\end{sphinxVerbatimInput}
\begin{sphinxVerbatimOutput}

\begin{sphinxuseclass}{cell_output}
\begin{sphinxVerbatim}[commandchars=\\\{\}]
5
\end{sphinxVerbatim}

\end{sphinxuseclass}\end{sphinxVerbatimOutput}

\end{sphinxuseclass}
\begin{sphinxuseclass}{cell}\begin{sphinxVerbatimInput}

\begin{sphinxuseclass}{cell_input}
\begin{sphinxVerbatim}[commandchars=\\\{\}]
\PYG{n}{b}\PYG{o}{.}\PYG{n}{sort}\PYG{p}{(}\PYG{n}{key}\PYG{o}{=}\PYG{n+nb}{len}\PYG{p}{)}
\end{sphinxVerbatim}

\end{sphinxuseclass}\end{sphinxVerbatimInput}

\end{sphinxuseclass}
\begin{sphinxuseclass}{cell}\begin{sphinxVerbatimInput}

\begin{sphinxuseclass}{cell_input}
\begin{sphinxVerbatim}[commandchars=\\\{\}]
\PYG{n}{b}
\end{sphinxVerbatim}

\end{sphinxuseclass}\end{sphinxVerbatimInput}
\begin{sphinxVerbatimOutput}

\begin{sphinxuseclass}{cell_output}
\begin{sphinxVerbatim}[commandchars=\\\{\}]
[\PYGZsq{}He\PYGZsq{}, \PYGZsq{}saw\PYGZsq{}, \PYGZsq{}six\PYGZsq{}, \PYGZsq{}foxes\PYGZsq{}, \PYGZsq{}small\PYGZsq{}]
\end{sphinxVerbatim}

\end{sphinxuseclass}\end{sphinxVerbatimOutput}

\end{sphinxuseclass}

\subsubsection{Slicing}
\label{\detokenize{mckinney_03_lecture:slicing}}
\sphinxAtStartPar
\sphinxstyleemphasis{\sphinxstylestrong{Slicing is very important!}}
\begin{quote}

\sphinxAtStartPar
You can select sections of most sequence types by using slice notation, which in its basic form consists of start:stop passed to the indexing operator {[} {]}.
\end{quote}

\sphinxAtStartPar
Recall that Python is zero\sphinxhyphen{}indexed, so the first element has an index of 0.
The necessary consequence of zero\sphinxhyphen{}indexing is that start:stop is inclusive on the left edge (start) and exclusive on the right edge (stop).

\begin{sphinxuseclass}{cell}\begin{sphinxVerbatimInput}

\begin{sphinxuseclass}{cell_input}
\begin{sphinxVerbatim}[commandchars=\\\{\}]
\PYG{n}{seq} \PYG{o}{=} \PYG{p}{[}\PYG{l+m+mi}{7}\PYG{p}{,} \PYG{l+m+mi}{2}\PYG{p}{,} \PYG{l+m+mi}{3}\PYG{p}{,} \PYG{l+m+mi}{7}\PYG{p}{,} \PYG{l+m+mi}{5}\PYG{p}{,} \PYG{l+m+mi}{6}\PYG{p}{,} \PYG{l+m+mi}{0}\PYG{p}{,} \PYG{l+m+mi}{1}\PYG{p}{]}
\PYG{n}{seq}
\end{sphinxVerbatim}

\end{sphinxuseclass}\end{sphinxVerbatimInput}
\begin{sphinxVerbatimOutput}

\begin{sphinxuseclass}{cell_output}
\begin{sphinxVerbatim}[commandchars=\\\{\}]
[7, 2, 3, 7, 5, 6, 0, 1]
\end{sphinxVerbatim}

\end{sphinxuseclass}\end{sphinxVerbatimOutput}

\end{sphinxuseclass}
\begin{sphinxuseclass}{cell}\begin{sphinxVerbatimInput}

\begin{sphinxuseclass}{cell_input}
\begin{sphinxVerbatim}[commandchars=\\\{\}]
\PYG{n}{seq}\PYG{p}{[}\PYG{l+m+mi}{5}\PYG{p}{]}
\end{sphinxVerbatim}

\end{sphinxuseclass}\end{sphinxVerbatimInput}
\begin{sphinxVerbatimOutput}

\begin{sphinxuseclass}{cell_output}
\begin{sphinxVerbatim}[commandchars=\\\{\}]
6
\end{sphinxVerbatim}

\end{sphinxuseclass}\end{sphinxVerbatimOutput}

\end{sphinxuseclass}
\begin{sphinxuseclass}{cell}\begin{sphinxVerbatimInput}

\begin{sphinxuseclass}{cell_input}
\begin{sphinxVerbatim}[commandchars=\\\{\}]
\PYG{n}{seq}\PYG{p}{[}\PYG{p}{:}\PYG{l+m+mi}{5}\PYG{p}{]}
\end{sphinxVerbatim}

\end{sphinxuseclass}\end{sphinxVerbatimInput}
\begin{sphinxVerbatimOutput}

\begin{sphinxuseclass}{cell_output}
\begin{sphinxVerbatim}[commandchars=\\\{\}]
[7, 2, 3, 7, 5]
\end{sphinxVerbatim}

\end{sphinxuseclass}\end{sphinxVerbatimOutput}

\end{sphinxuseclass}
\begin{sphinxuseclass}{cell}\begin{sphinxVerbatimInput}

\begin{sphinxuseclass}{cell_input}
\begin{sphinxVerbatim}[commandchars=\\\{\}]
\PYG{n}{seq}\PYG{p}{[}\PYG{l+m+mi}{1}\PYG{p}{:}\PYG{l+m+mi}{5}\PYG{p}{]}
\end{sphinxVerbatim}

\end{sphinxuseclass}\end{sphinxVerbatimInput}
\begin{sphinxVerbatimOutput}

\begin{sphinxuseclass}{cell_output}
\begin{sphinxVerbatim}[commandchars=\\\{\}]
[2, 3, 7, 5]
\end{sphinxVerbatim}

\end{sphinxuseclass}\end{sphinxVerbatimOutput}

\end{sphinxuseclass}
\begin{sphinxuseclass}{cell}\begin{sphinxVerbatimInput}

\begin{sphinxuseclass}{cell_input}
\begin{sphinxVerbatim}[commandchars=\\\{\}]
\PYG{n}{seq}\PYG{p}{[}\PYG{l+m+mi}{3}\PYG{p}{:}\PYG{l+m+mi}{5}\PYG{p}{]}
\end{sphinxVerbatim}

\end{sphinxuseclass}\end{sphinxVerbatimInput}
\begin{sphinxVerbatimOutput}

\begin{sphinxuseclass}{cell_output}
\begin{sphinxVerbatim}[commandchars=\\\{\}]
[7, 5]
\end{sphinxVerbatim}

\end{sphinxuseclass}\end{sphinxVerbatimOutput}

\end{sphinxuseclass}\begin{quote}

\sphinxAtStartPar
Either the start or stop can be omitted, in which case they default to the start of the sequence and the end of the sequence, respectively.
\end{quote}

\begin{sphinxuseclass}{cell}\begin{sphinxVerbatimInput}

\begin{sphinxuseclass}{cell_input}
\begin{sphinxVerbatim}[commandchars=\\\{\}]
\PYG{n}{seq}\PYG{p}{[}\PYG{p}{:}\PYG{l+m+mi}{5}\PYG{p}{]}
\end{sphinxVerbatim}

\end{sphinxuseclass}\end{sphinxVerbatimInput}
\begin{sphinxVerbatimOutput}

\begin{sphinxuseclass}{cell_output}
\begin{sphinxVerbatim}[commandchars=\\\{\}]
[7, 2, 3, 7, 5]
\end{sphinxVerbatim}

\end{sphinxuseclass}\end{sphinxVerbatimOutput}

\end{sphinxuseclass}
\begin{sphinxuseclass}{cell}\begin{sphinxVerbatimInput}

\begin{sphinxuseclass}{cell_input}
\begin{sphinxVerbatim}[commandchars=\\\{\}]
\PYG{n}{seq}\PYG{p}{[}\PYG{l+m+mi}{3}\PYG{p}{:}\PYG{p}{]}
\end{sphinxVerbatim}

\end{sphinxuseclass}\end{sphinxVerbatimInput}
\begin{sphinxVerbatimOutput}

\begin{sphinxuseclass}{cell_output}
\begin{sphinxVerbatim}[commandchars=\\\{\}]
[7, 5, 6, 0, 1]
\end{sphinxVerbatim}

\end{sphinxuseclass}\end{sphinxVerbatimOutput}

\end{sphinxuseclass}\begin{quote}

\sphinxAtStartPar
Negative indices slice the sequence relative to the end.
\end{quote}

\begin{sphinxuseclass}{cell}\begin{sphinxVerbatimInput}

\begin{sphinxuseclass}{cell_input}
\begin{sphinxVerbatim}[commandchars=\\\{\}]
\PYG{n}{seq}
\end{sphinxVerbatim}

\end{sphinxuseclass}\end{sphinxVerbatimInput}
\begin{sphinxVerbatimOutput}

\begin{sphinxuseclass}{cell_output}
\begin{sphinxVerbatim}[commandchars=\\\{\}]
[7, 2, 3, 7, 5, 6, 0, 1]
\end{sphinxVerbatim}

\end{sphinxuseclass}\end{sphinxVerbatimOutput}

\end{sphinxuseclass}
\begin{sphinxuseclass}{cell}\begin{sphinxVerbatimInput}

\begin{sphinxuseclass}{cell_input}
\begin{sphinxVerbatim}[commandchars=\\\{\}]
\PYG{n}{seq}\PYG{p}{[}\PYG{o}{\PYGZhy{}}\PYG{l+m+mi}{1}\PYG{p}{:}\PYG{p}{]}
\end{sphinxVerbatim}

\end{sphinxuseclass}\end{sphinxVerbatimInput}
\begin{sphinxVerbatimOutput}

\begin{sphinxuseclass}{cell_output}
\begin{sphinxVerbatim}[commandchars=\\\{\}]
[1]
\end{sphinxVerbatim}

\end{sphinxuseclass}\end{sphinxVerbatimOutput}

\end{sphinxuseclass}
\begin{sphinxuseclass}{cell}\begin{sphinxVerbatimInput}

\begin{sphinxuseclass}{cell_input}
\begin{sphinxVerbatim}[commandchars=\\\{\}]
\PYG{n}{seq}\PYG{p}{[}\PYG{o}{\PYGZhy{}}\PYG{l+m+mi}{4}\PYG{p}{:}\PYG{p}{]}
\end{sphinxVerbatim}

\end{sphinxuseclass}\end{sphinxVerbatimInput}
\begin{sphinxVerbatimOutput}

\begin{sphinxuseclass}{cell_output}
\begin{sphinxVerbatim}[commandchars=\\\{\}]
[5, 6, 0, 1]
\end{sphinxVerbatim}

\end{sphinxuseclass}\end{sphinxVerbatimOutput}

\end{sphinxuseclass}
\begin{sphinxuseclass}{cell}\begin{sphinxVerbatimInput}

\begin{sphinxuseclass}{cell_input}
\begin{sphinxVerbatim}[commandchars=\\\{\}]
\PYG{n}{seq}\PYG{p}{[}\PYG{o}{\PYGZhy{}}\PYG{l+m+mi}{4}\PYG{p}{:}\PYG{o}{\PYGZhy{}}\PYG{l+m+mi}{1}\PYG{p}{]}
\end{sphinxVerbatim}

\end{sphinxuseclass}\end{sphinxVerbatimInput}
\begin{sphinxVerbatimOutput}

\begin{sphinxuseclass}{cell_output}
\begin{sphinxVerbatim}[commandchars=\\\{\}]
[5, 6, 0]
\end{sphinxVerbatim}

\end{sphinxuseclass}\end{sphinxVerbatimOutput}

\end{sphinxuseclass}
\begin{sphinxuseclass}{cell}\begin{sphinxVerbatimInput}

\begin{sphinxuseclass}{cell_input}
\begin{sphinxVerbatim}[commandchars=\\\{\}]
\PYG{n}{seq}\PYG{p}{[}\PYG{o}{\PYGZhy{}}\PYG{l+m+mi}{6}\PYG{p}{:}\PYG{o}{\PYGZhy{}}\PYG{l+m+mi}{2}\PYG{p}{]}
\end{sphinxVerbatim}

\end{sphinxuseclass}\end{sphinxVerbatimInput}
\begin{sphinxVerbatimOutput}

\begin{sphinxuseclass}{cell_output}
\begin{sphinxVerbatim}[commandchars=\\\{\}]
[3, 7, 5, 6]
\end{sphinxVerbatim}

\end{sphinxuseclass}\end{sphinxVerbatimOutput}

\end{sphinxuseclass}\begin{quote}

\sphinxAtStartPar
A step can also be used after a second colon to, say, take every other element.
\end{quote}

\begin{sphinxuseclass}{cell}\begin{sphinxVerbatimInput}

\begin{sphinxuseclass}{cell_input}
\begin{sphinxVerbatim}[commandchars=\\\{\}]
\PYG{n}{seq}\PYG{p}{[}\PYG{p}{:}\PYG{p}{]}
\end{sphinxVerbatim}

\end{sphinxuseclass}\end{sphinxVerbatimInput}
\begin{sphinxVerbatimOutput}

\begin{sphinxuseclass}{cell_output}
\begin{sphinxVerbatim}[commandchars=\\\{\}]
[7, 2, 3, 7, 5, 6, 0, 1]
\end{sphinxVerbatim}

\end{sphinxuseclass}\end{sphinxVerbatimOutput}

\end{sphinxuseclass}
\begin{sphinxuseclass}{cell}\begin{sphinxVerbatimInput}

\begin{sphinxuseclass}{cell_input}
\begin{sphinxVerbatim}[commandchars=\\\{\}]
\PYG{n}{seq}\PYG{p}{[}\PYG{p}{:}\PYG{p}{:}\PYG{l+m+mi}{2}\PYG{p}{]}
\end{sphinxVerbatim}

\end{sphinxuseclass}\end{sphinxVerbatimInput}
\begin{sphinxVerbatimOutput}

\begin{sphinxuseclass}{cell_output}
\begin{sphinxVerbatim}[commandchars=\\\{\}]
[7, 3, 5, 0]
\end{sphinxVerbatim}

\end{sphinxuseclass}\end{sphinxVerbatimOutput}

\end{sphinxuseclass}
\begin{sphinxuseclass}{cell}\begin{sphinxVerbatimInput}

\begin{sphinxuseclass}{cell_input}
\begin{sphinxVerbatim}[commandchars=\\\{\}]
\PYG{n}{seq}\PYG{p}{[}\PYG{l+m+mi}{1}\PYG{p}{:}\PYG{p}{:}\PYG{l+m+mi}{2}\PYG{p}{]}
\end{sphinxVerbatim}

\end{sphinxuseclass}\end{sphinxVerbatimInput}
\begin{sphinxVerbatimOutput}

\begin{sphinxuseclass}{cell_output}
\begin{sphinxVerbatim}[commandchars=\\\{\}]
[2, 7, 6, 1]
\end{sphinxVerbatim}

\end{sphinxuseclass}\end{sphinxVerbatimOutput}

\end{sphinxuseclass}
\sphinxAtStartPar
I remember the trick above as \sphinxcode{\sphinxupquote{:2}} is “count by 2”.
\begin{quote}

\sphinxAtStartPar
A clever use of this is to pass \sphinxhyphen{}1, which has the useful effect of reversing a list or tuple.
\end{quote}

\begin{sphinxuseclass}{cell}\begin{sphinxVerbatimInput}

\begin{sphinxuseclass}{cell_input}
\begin{sphinxVerbatim}[commandchars=\\\{\}]
\PYG{n}{seq}\PYG{p}{[}\PYG{p}{:}\PYG{p}{:}\PYG{o}{\PYGZhy{}}\PYG{l+m+mi}{1}\PYG{p}{]}
\end{sphinxVerbatim}

\end{sphinxuseclass}\end{sphinxVerbatimInput}
\begin{sphinxVerbatimOutput}

\begin{sphinxuseclass}{cell_output}
\begin{sphinxVerbatim}[commandchars=\\\{\}]
[1, 0, 6, 5, 7, 3, 2, 7]
\end{sphinxVerbatim}

\end{sphinxuseclass}\end{sphinxVerbatimOutput}

\end{sphinxuseclass}
\sphinxAtStartPar
We will use slicing (subsetting) all semester, so it is worth a few minutes to understand the examples above.


\subsection{dict}
\label{\detokenize{mckinney_03_lecture:dict}}\begin{quote}

\sphinxAtStartPar
dict is likely the most important built\sphinxhyphen{}in Python data structure. A more common
name for it is hash map or associative array. It is a flexibly sized collection of key\sphinxhyphen{}value
pairs, where key and value are Python objects. One approach for creating one is to use
curly braces \{\} and colons to separate keys and values.
\end{quote}

\sphinxAtStartPar
Elements in dictionaries have names, while elements in tuples and lists have numerical indices.
Dictionaries are handy for passing named arguments and returning named results.

\begin{sphinxuseclass}{cell}\begin{sphinxVerbatimInput}

\begin{sphinxuseclass}{cell_input}
\begin{sphinxVerbatim}[commandchars=\\\{\}]
\PYG{n}{empty\PYGZus{}dict} \PYG{o}{=} \PYG{p}{\PYGZob{}}\PYG{p}{\PYGZcb{}}
\PYG{n}{empty\PYGZus{}dict}
\end{sphinxVerbatim}

\end{sphinxuseclass}\end{sphinxVerbatimInput}
\begin{sphinxVerbatimOutput}

\begin{sphinxuseclass}{cell_output}
\begin{sphinxVerbatim}[commandchars=\\\{\}]
\PYGZob{}\PYGZcb{}
\end{sphinxVerbatim}

\end{sphinxuseclass}\end{sphinxVerbatimOutput}

\end{sphinxuseclass}
\sphinxAtStartPar
A dictionary is a set of key\sphinxhyphen{}value pairs.

\begin{sphinxuseclass}{cell}\begin{sphinxVerbatimInput}

\begin{sphinxuseclass}{cell_input}
\begin{sphinxVerbatim}[commandchars=\\\{\}]
\PYG{n}{d1} \PYG{o}{=} \PYG{p}{\PYGZob{}}\PYG{l+s+s1}{\PYGZsq{}}\PYG{l+s+s1}{a}\PYG{l+s+s1}{\PYGZsq{}}\PYG{p}{:} \PYG{l+s+s1}{\PYGZsq{}}\PYG{l+s+s1}{some value}\PYG{l+s+s1}{\PYGZsq{}}\PYG{p}{,} \PYG{l+s+s1}{\PYGZsq{}}\PYG{l+s+s1}{b}\PYG{l+s+s1}{\PYGZsq{}}\PYG{p}{:} \PYG{p}{[}\PYG{l+m+mi}{1}\PYG{p}{,} \PYG{l+m+mi}{2}\PYG{p}{,} \PYG{l+m+mi}{3}\PYG{p}{,} \PYG{l+m+mi}{4}\PYG{p}{]}\PYG{p}{\PYGZcb{}}
\end{sphinxVerbatim}

\end{sphinxuseclass}\end{sphinxVerbatimInput}

\end{sphinxuseclass}
\begin{sphinxuseclass}{cell}\begin{sphinxVerbatimInput}

\begin{sphinxuseclass}{cell_input}
\begin{sphinxVerbatim}[commandchars=\\\{\}]
\PYG{n}{d1}\PYG{p}{[}\PYG{l+s+s1}{\PYGZsq{}}\PYG{l+s+s1}{a}\PYG{l+s+s1}{\PYGZsq{}}\PYG{p}{]}
\end{sphinxVerbatim}

\end{sphinxuseclass}\end{sphinxVerbatimInput}
\begin{sphinxVerbatimOutput}

\begin{sphinxuseclass}{cell_output}
\begin{sphinxVerbatim}[commandchars=\\\{\}]
\PYGZsq{}some value\PYGZsq{}
\end{sphinxVerbatim}

\end{sphinxuseclass}\end{sphinxVerbatimOutput}

\end{sphinxuseclass}
\begin{sphinxuseclass}{cell}\begin{sphinxVerbatimInput}

\begin{sphinxuseclass}{cell_input}
\begin{sphinxVerbatim}[commandchars=\\\{\}]
\PYG{n}{d1}\PYG{p}{[}\PYG{l+m+mi}{7}\PYG{p}{]} \PYG{o}{=} \PYG{l+s+s1}{\PYGZsq{}}\PYG{l+s+s1}{an integer}\PYG{l+s+s1}{\PYGZsq{}}
\end{sphinxVerbatim}

\end{sphinxuseclass}\end{sphinxVerbatimInput}

\end{sphinxuseclass}
\begin{sphinxuseclass}{cell}\begin{sphinxVerbatimInput}

\begin{sphinxuseclass}{cell_input}
\begin{sphinxVerbatim}[commandchars=\\\{\}]
\PYG{n}{d1}
\end{sphinxVerbatim}

\end{sphinxuseclass}\end{sphinxVerbatimInput}
\begin{sphinxVerbatimOutput}

\begin{sphinxuseclass}{cell_output}
\begin{sphinxVerbatim}[commandchars=\\\{\}]
\PYGZob{}\PYGZsq{}a\PYGZsq{}: \PYGZsq{}some value\PYGZsq{}, \PYGZsq{}b\PYGZsq{}: [1, 2, 3, 4], 7: \PYGZsq{}an integer\PYGZsq{}\PYGZcb{}
\end{sphinxVerbatim}

\end{sphinxuseclass}\end{sphinxVerbatimOutput}

\end{sphinxuseclass}
\sphinxAtStartPar
We access dictionary values by key names instead of key positions.

\begin{sphinxuseclass}{cell}\begin{sphinxVerbatimInput}

\begin{sphinxuseclass}{cell_input}
\begin{sphinxVerbatim}[commandchars=\\\{\}]
\PYG{n}{d1}\PYG{p}{[}\PYG{l+s+s1}{\PYGZsq{}}\PYG{l+s+s1}{b}\PYG{l+s+s1}{\PYGZsq{}}\PYG{p}{]}
\end{sphinxVerbatim}

\end{sphinxuseclass}\end{sphinxVerbatimInput}
\begin{sphinxVerbatimOutput}

\begin{sphinxuseclass}{cell_output}
\begin{sphinxVerbatim}[commandchars=\\\{\}]
[1, 2, 3, 4]
\end{sphinxVerbatim}

\end{sphinxuseclass}\end{sphinxVerbatimOutput}

\end{sphinxuseclass}
\begin{sphinxuseclass}{cell}\begin{sphinxVerbatimInput}

\begin{sphinxuseclass}{cell_input}
\begin{sphinxVerbatim}[commandchars=\\\{\}]
\PYG{l+s+s1}{\PYGZsq{}}\PYG{l+s+s1}{b}\PYG{l+s+s1}{\PYGZsq{}} \PYG{o+ow}{in} \PYG{n}{d1}
\end{sphinxVerbatim}

\end{sphinxuseclass}\end{sphinxVerbatimInput}
\begin{sphinxVerbatimOutput}

\begin{sphinxuseclass}{cell_output}
\begin{sphinxVerbatim}[commandchars=\\\{\}]
True
\end{sphinxVerbatim}

\end{sphinxuseclass}\end{sphinxVerbatimOutput}

\end{sphinxuseclass}\begin{quote}

\sphinxAtStartPar
You can delete values either using the del keyword or the pop method (which simultaneously returns the value and deletes the key).
\end{quote}

\begin{sphinxuseclass}{cell}\begin{sphinxVerbatimInput}

\begin{sphinxuseclass}{cell_input}
\begin{sphinxVerbatim}[commandchars=\\\{\}]
\PYG{n}{d1}\PYG{p}{[}\PYG{l+m+mi}{5}\PYG{p}{]} \PYG{o}{=} \PYG{l+s+s1}{\PYGZsq{}}\PYG{l+s+s1}{some value}\PYG{l+s+s1}{\PYGZsq{}}
\end{sphinxVerbatim}

\end{sphinxuseclass}\end{sphinxVerbatimInput}

\end{sphinxuseclass}
\begin{sphinxuseclass}{cell}\begin{sphinxVerbatimInput}

\begin{sphinxuseclass}{cell_input}
\begin{sphinxVerbatim}[commandchars=\\\{\}]
\PYG{n}{d1}\PYG{p}{[}\PYG{l+s+s1}{\PYGZsq{}}\PYG{l+s+s1}{dummy}\PYG{l+s+s1}{\PYGZsq{}}\PYG{p}{]} \PYG{o}{=} \PYG{l+s+s1}{\PYGZsq{}}\PYG{l+s+s1}{another value}\PYG{l+s+s1}{\PYGZsq{}}
\end{sphinxVerbatim}

\end{sphinxuseclass}\end{sphinxVerbatimInput}

\end{sphinxuseclass}
\begin{sphinxuseclass}{cell}\begin{sphinxVerbatimInput}

\begin{sphinxuseclass}{cell_input}
\begin{sphinxVerbatim}[commandchars=\\\{\}]
\PYG{n}{d1}
\end{sphinxVerbatim}

\end{sphinxuseclass}\end{sphinxVerbatimInput}
\begin{sphinxVerbatimOutput}

\begin{sphinxuseclass}{cell_output}
\begin{sphinxVerbatim}[commandchars=\\\{\}]
\PYGZob{}\PYGZsq{}a\PYGZsq{}: \PYGZsq{}some value\PYGZsq{},
 \PYGZsq{}b\PYGZsq{}: [1, 2, 3, 4],
 7: \PYGZsq{}an integer\PYGZsq{},
 5: \PYGZsq{}some value\PYGZsq{},
 \PYGZsq{}dummy\PYGZsq{}: \PYGZsq{}another value\PYGZsq{}\PYGZcb{}
\end{sphinxVerbatim}

\end{sphinxuseclass}\end{sphinxVerbatimOutput}

\end{sphinxuseclass}
\begin{sphinxuseclass}{cell}\begin{sphinxVerbatimInput}

\begin{sphinxuseclass}{cell_input}
\begin{sphinxVerbatim}[commandchars=\\\{\}]
\PYG{k}{del} \PYG{n}{d1}\PYG{p}{[}\PYG{l+m+mi}{5}\PYG{p}{]}
\end{sphinxVerbatim}

\end{sphinxuseclass}\end{sphinxVerbatimInput}

\end{sphinxuseclass}
\begin{sphinxuseclass}{cell}\begin{sphinxVerbatimInput}

\begin{sphinxuseclass}{cell_input}
\begin{sphinxVerbatim}[commandchars=\\\{\}]
\PYG{n}{d1}
\end{sphinxVerbatim}

\end{sphinxuseclass}\end{sphinxVerbatimInput}
\begin{sphinxVerbatimOutput}

\begin{sphinxuseclass}{cell_output}
\begin{sphinxVerbatim}[commandchars=\\\{\}]
\PYGZob{}\PYGZsq{}a\PYGZsq{}: \PYGZsq{}some value\PYGZsq{},
 \PYGZsq{}b\PYGZsq{}: [1, 2, 3, 4],
 7: \PYGZsq{}an integer\PYGZsq{},
 \PYGZsq{}dummy\PYGZsq{}: \PYGZsq{}another value\PYGZsq{}\PYGZcb{}
\end{sphinxVerbatim}

\end{sphinxuseclass}\end{sphinxVerbatimOutput}

\end{sphinxuseclass}
\begin{sphinxuseclass}{cell}\begin{sphinxVerbatimInput}

\begin{sphinxuseclass}{cell_input}
\begin{sphinxVerbatim}[commandchars=\\\{\}]
\PYG{n}{ret} \PYG{o}{=} \PYG{n}{d1}\PYG{o}{.}\PYG{n}{pop}\PYG{p}{(}\PYG{l+s+s1}{\PYGZsq{}}\PYG{l+s+s1}{dummy}\PYG{l+s+s1}{\PYGZsq{}}\PYG{p}{)}
\end{sphinxVerbatim}

\end{sphinxuseclass}\end{sphinxVerbatimInput}

\end{sphinxuseclass}
\begin{sphinxuseclass}{cell}\begin{sphinxVerbatimInput}

\begin{sphinxuseclass}{cell_input}
\begin{sphinxVerbatim}[commandchars=\\\{\}]
\PYG{n}{ret}
\end{sphinxVerbatim}

\end{sphinxuseclass}\end{sphinxVerbatimInput}
\begin{sphinxVerbatimOutput}

\begin{sphinxuseclass}{cell_output}
\begin{sphinxVerbatim}[commandchars=\\\{\}]
\PYGZsq{}another value\PYGZsq{}
\end{sphinxVerbatim}

\end{sphinxuseclass}\end{sphinxVerbatimOutput}

\end{sphinxuseclass}
\begin{sphinxuseclass}{cell}\begin{sphinxVerbatimInput}

\begin{sphinxuseclass}{cell_input}
\begin{sphinxVerbatim}[commandchars=\\\{\}]
\PYG{n}{d1}
\end{sphinxVerbatim}

\end{sphinxuseclass}\end{sphinxVerbatimInput}
\begin{sphinxVerbatimOutput}

\begin{sphinxuseclass}{cell_output}
\begin{sphinxVerbatim}[commandchars=\\\{\}]
\PYGZob{}\PYGZsq{}a\PYGZsq{}: \PYGZsq{}some value\PYGZsq{}, \PYGZsq{}b\PYGZsq{}: [1, 2, 3, 4], 7: \PYGZsq{}an integer\PYGZsq{}\PYGZcb{}
\end{sphinxVerbatim}

\end{sphinxuseclass}\end{sphinxVerbatimOutput}

\end{sphinxuseclass}\begin{quote}

\sphinxAtStartPar
The keys and values method give you iterators of the dict’s keys and values, respectively. While the key\sphinxhyphen{}value pairs are not in any particular order, these functions output the keys and values in the same order.
\end{quote}

\begin{sphinxuseclass}{cell}\begin{sphinxVerbatimInput}

\begin{sphinxuseclass}{cell_input}
\begin{sphinxVerbatim}[commandchars=\\\{\}]
\PYG{n}{d1}\PYG{o}{.}\PYG{n}{keys}\PYG{p}{(}\PYG{p}{)}
\end{sphinxVerbatim}

\end{sphinxuseclass}\end{sphinxVerbatimInput}
\begin{sphinxVerbatimOutput}

\begin{sphinxuseclass}{cell_output}
\begin{sphinxVerbatim}[commandchars=\\\{\}]
dict\PYGZus{}keys([\PYGZsq{}a\PYGZsq{}, \PYGZsq{}b\PYGZsq{}, 7])
\end{sphinxVerbatim}

\end{sphinxuseclass}\end{sphinxVerbatimOutput}

\end{sphinxuseclass}
\begin{sphinxuseclass}{cell}\begin{sphinxVerbatimInput}

\begin{sphinxuseclass}{cell_input}
\begin{sphinxVerbatim}[commandchars=\\\{\}]
\PYG{n}{d1}\PYG{o}{.}\PYG{n}{values}\PYG{p}{(}\PYG{p}{)}
\end{sphinxVerbatim}

\end{sphinxuseclass}\end{sphinxVerbatimInput}
\begin{sphinxVerbatimOutput}

\begin{sphinxuseclass}{cell_output}
\begin{sphinxVerbatim}[commandchars=\\\{\}]
dict\PYGZus{}values([\PYGZsq{}some value\PYGZsq{}, [1, 2, 3, 4], \PYGZsq{}an integer\PYGZsq{}])
\end{sphinxVerbatim}

\end{sphinxuseclass}\end{sphinxVerbatimOutput}

\end{sphinxuseclass}\begin{quote}

\sphinxAtStartPar
You can merge one dict into another using the update method.
\end{quote}

\begin{sphinxuseclass}{cell}\begin{sphinxVerbatimInput}

\begin{sphinxuseclass}{cell_input}
\begin{sphinxVerbatim}[commandchars=\\\{\}]
\PYG{n}{d1}
\end{sphinxVerbatim}

\end{sphinxuseclass}\end{sphinxVerbatimInput}
\begin{sphinxVerbatimOutput}

\begin{sphinxuseclass}{cell_output}
\begin{sphinxVerbatim}[commandchars=\\\{\}]
\PYGZob{}\PYGZsq{}a\PYGZsq{}: \PYGZsq{}some value\PYGZsq{}, \PYGZsq{}b\PYGZsq{}: [1, 2, 3, 4], 7: \PYGZsq{}an integer\PYGZsq{}\PYGZcb{}
\end{sphinxVerbatim}

\end{sphinxuseclass}\end{sphinxVerbatimOutput}

\end{sphinxuseclass}
\begin{sphinxuseclass}{cell}\begin{sphinxVerbatimInput}

\begin{sphinxuseclass}{cell_input}
\begin{sphinxVerbatim}[commandchars=\\\{\}]
\PYG{n}{d1}\PYG{o}{.}\PYG{n}{update}\PYG{p}{(}\PYG{p}{\PYGZob{}}\PYG{l+s+s1}{\PYGZsq{}}\PYG{l+s+s1}{b}\PYG{l+s+s1}{\PYGZsq{}}\PYG{p}{:} \PYG{l+s+s1}{\PYGZsq{}}\PYG{l+s+s1}{foo}\PYG{l+s+s1}{\PYGZsq{}}\PYG{p}{,} \PYG{l+s+s1}{\PYGZsq{}}\PYG{l+s+s1}{c}\PYG{l+s+s1}{\PYGZsq{}}\PYG{p}{:} \PYG{l+m+mi}{12}\PYG{p}{\PYGZcb{}}\PYG{p}{)}
\end{sphinxVerbatim}

\end{sphinxuseclass}\end{sphinxVerbatimInput}

\end{sphinxuseclass}
\begin{sphinxuseclass}{cell}\begin{sphinxVerbatimInput}

\begin{sphinxuseclass}{cell_input}
\begin{sphinxVerbatim}[commandchars=\\\{\}]
\PYG{n}{d1}
\end{sphinxVerbatim}

\end{sphinxuseclass}\end{sphinxVerbatimInput}
\begin{sphinxVerbatimOutput}

\begin{sphinxuseclass}{cell_output}
\begin{sphinxVerbatim}[commandchars=\\\{\}]
\PYGZob{}\PYGZsq{}a\PYGZsq{}: \PYGZsq{}some value\PYGZsq{}, \PYGZsq{}b\PYGZsq{}: \PYGZsq{}foo\PYGZsq{}, 7: \PYGZsq{}an integer\PYGZsq{}, \PYGZsq{}c\PYGZsq{}: 12\PYGZcb{}
\end{sphinxVerbatim}

\end{sphinxuseclass}\end{sphinxVerbatimOutput}

\end{sphinxuseclass}

\section{List, Set, and Dict Comprehensions}
\label{\detokenize{mckinney_03_lecture:list-set-and-dict-comprehensions}}
\sphinxAtStartPar
We will focus on list comprehensions.
\begin{quote}

\sphinxAtStartPar
List comprehensions are one of the most\sphinxhyphen{}loved Python language features. They allow you to concisely form a new list by filtering the elements of a collection, transforming the elements passing the filter in one concise expression. They take the basic form:

\begin{sphinxVerbatim}[commandchars=\\\{\}]
\PYG{p}{[}\PYG{n}{expr} \PYG{k}{for} \PYG{n}{val} \PYG{o+ow}{in} \PYG{n}{collection} \PYG{k}{if} \PYG{n}{condition}\PYG{p}{]}
\end{sphinxVerbatim}

\sphinxAtStartPar
This is equivalent to the following for loop:

\begin{sphinxVerbatim}[commandchars=\\\{\}]
\PYG{n}{result} \PYG{o}{=} \PYG{p}{[}\PYG{p}{]}
\PYG{k}{for} \PYG{n}{val} \PYG{o+ow}{in} \PYG{n}{collection}\PYG{p}{:}
    \PYG{k}{if} \PYG{n}{condition}\PYG{p}{:}
        \PYG{n}{result}\PYG{o}{.}\PYG{n}{append}\PYG{p}{(}\PYG{n}{expr}\PYG{p}{)}
\end{sphinxVerbatim}

\sphinxAtStartPar
The filter condition can be omitted, leaving only the expression.
\end{quote}

\sphinxAtStartPar
List comprehensions are very \sphinxhref{https://blog.startifact.com/posts/older/what-is-pythonic.html}{Pythonic}.

\begin{sphinxuseclass}{cell}\begin{sphinxVerbatimInput}

\begin{sphinxuseclass}{cell_input}
\begin{sphinxVerbatim}[commandchars=\\\{\}]
\PYG{n}{strings} \PYG{o}{=} \PYG{p}{[}\PYG{l+s+s1}{\PYGZsq{}}\PYG{l+s+s1}{a}\PYG{l+s+s1}{\PYGZsq{}}\PYG{p}{,} \PYG{l+s+s1}{\PYGZsq{}}\PYG{l+s+s1}{as}\PYG{l+s+s1}{\PYGZsq{}}\PYG{p}{,} \PYG{l+s+s1}{\PYGZsq{}}\PYG{l+s+s1}{bat}\PYG{l+s+s1}{\PYGZsq{}}\PYG{p}{,} \PYG{l+s+s1}{\PYGZsq{}}\PYG{l+s+s1}{car}\PYG{l+s+s1}{\PYGZsq{}}\PYG{p}{,} \PYG{l+s+s1}{\PYGZsq{}}\PYG{l+s+s1}{dove}\PYG{l+s+s1}{\PYGZsq{}}\PYG{p}{,} \PYG{l+s+s1}{\PYGZsq{}}\PYG{l+s+s1}{python}\PYG{l+s+s1}{\PYGZsq{}}\PYG{p}{]}
\end{sphinxVerbatim}

\end{sphinxuseclass}\end{sphinxVerbatimInput}

\end{sphinxuseclass}
\sphinxAtStartPar
We could use a for loop to capitalize the strings in \sphinxcode{\sphinxupquote{strings}} and keep only strings with lengths greater than two.

\begin{sphinxuseclass}{cell}\begin{sphinxVerbatimInput}

\begin{sphinxuseclass}{cell_input}
\begin{sphinxVerbatim}[commandchars=\\\{\}]
\PYG{n}{caps} \PYG{o}{=} \PYG{p}{[}\PYG{p}{]}
\PYG{k}{for} \PYG{n}{x} \PYG{o+ow}{in} \PYG{n}{strings}\PYG{p}{:}
    \PYG{k}{if} \PYG{n+nb}{len}\PYG{p}{(}\PYG{n}{x}\PYG{p}{)} \PYG{o}{\PYGZgt{}} \PYG{l+m+mi}{2}\PYG{p}{:}
        \PYG{n}{caps}\PYG{o}{.}\PYG{n}{append}\PYG{p}{(}\PYG{n}{x}\PYG{o}{.}\PYG{n}{upper}\PYG{p}{(}\PYG{p}{)}\PYG{p}{)}

\PYG{n}{caps}
\end{sphinxVerbatim}

\end{sphinxuseclass}\end{sphinxVerbatimInput}
\begin{sphinxVerbatimOutput}

\begin{sphinxuseclass}{cell_output}
\begin{sphinxVerbatim}[commandchars=\\\{\}]
[\PYGZsq{}BAT\PYGZsq{}, \PYGZsq{}CAR\PYGZsq{}, \PYGZsq{}DOVE\PYGZsq{}, \PYGZsq{}PYTHON\PYGZsq{}]
\end{sphinxVerbatim}

\end{sphinxuseclass}\end{sphinxVerbatimOutput}

\end{sphinxuseclass}
\sphinxAtStartPar
A list comprehension is a more Pythonic solution and replaces four lines of code with one.
The general format for a list comprehension is \sphinxcode{\sphinxupquote{{[}operation on x for x in list if condition{]}}}

\begin{sphinxuseclass}{cell}\begin{sphinxVerbatimInput}

\begin{sphinxuseclass}{cell_input}
\begin{sphinxVerbatim}[commandchars=\\\{\}]
\PYG{p}{[}\PYG{n}{x}\PYG{o}{.}\PYG{n}{upper}\PYG{p}{(}\PYG{p}{)} \PYG{k}{for} \PYG{n}{x} \PYG{o+ow}{in} \PYG{n}{strings} \PYG{k}{if} \PYG{n+nb}{len}\PYG{p}{(}\PYG{n}{x}\PYG{p}{)} \PYG{o}{\PYGZgt{}} \PYG{l+m+mi}{2}\PYG{p}{]}
\end{sphinxVerbatim}

\end{sphinxuseclass}\end{sphinxVerbatimInput}
\begin{sphinxVerbatimOutput}

\begin{sphinxuseclass}{cell_output}
\begin{sphinxVerbatim}[commandchars=\\\{\}]
[\PYGZsq{}BAT\PYGZsq{}, \PYGZsq{}CAR\PYGZsq{}, \PYGZsq{}DOVE\PYGZsq{}, \PYGZsq{}PYTHON\PYGZsq{}]
\end{sphinxVerbatim}

\end{sphinxuseclass}\end{sphinxVerbatimOutput}

\end{sphinxuseclass}
\sphinxAtStartPar
Here is another example.
Write a for\sphinxhyphen{}loop and the equivalent list comprehension that squares the integers from 1 to 10.

\begin{sphinxuseclass}{cell}\begin{sphinxVerbatimInput}

\begin{sphinxuseclass}{cell_input}
\begin{sphinxVerbatim}[commandchars=\\\{\}]
\PYG{n}{squares} \PYG{o}{=} \PYG{p}{[}\PYG{p}{]}
\PYG{k}{for} \PYG{n}{i} \PYG{o+ow}{in} \PYG{n+nb}{range}\PYG{p}{(}\PYG{l+m+mi}{1}\PYG{p}{,} \PYG{l+m+mi}{11}\PYG{p}{)}\PYG{p}{:}
    \PYG{n}{squares}\PYG{o}{.}\PYG{n}{append}\PYG{p}{(}\PYG{n}{i} \PYG{o}{*}\PYG{o}{*} \PYG{l+m+mi}{2}\PYG{p}{)}
    
\PYG{n}{squares}
\end{sphinxVerbatim}

\end{sphinxuseclass}\end{sphinxVerbatimInput}
\begin{sphinxVerbatimOutput}

\begin{sphinxuseclass}{cell_output}
\begin{sphinxVerbatim}[commandchars=\\\{\}]
[1, 4, 9, 16, 25, 36, 49, 64, 81, 100]
\end{sphinxVerbatim}

\end{sphinxuseclass}\end{sphinxVerbatimOutput}

\end{sphinxuseclass}
\begin{sphinxuseclass}{cell}\begin{sphinxVerbatimInput}

\begin{sphinxuseclass}{cell_input}
\begin{sphinxVerbatim}[commandchars=\\\{\}]
\PYG{p}{[}\PYG{n}{i}\PYG{o}{*}\PYG{o}{*}\PYG{l+m+mi}{2} \PYG{k}{for} \PYG{n}{i} \PYG{o+ow}{in} \PYG{n+nb}{range}\PYG{p}{(}\PYG{l+m+mi}{1}\PYG{p}{,} \PYG{l+m+mi}{11}\PYG{p}{)}\PYG{p}{]}
\end{sphinxVerbatim}

\end{sphinxuseclass}\end{sphinxVerbatimInput}
\begin{sphinxVerbatimOutput}

\begin{sphinxuseclass}{cell_output}
\begin{sphinxVerbatim}[commandchars=\\\{\}]
[1, 4, 9, 16, 25, 36, 49, 64, 81, 100]
\end{sphinxVerbatim}

\end{sphinxuseclass}\end{sphinxVerbatimOutput}

\end{sphinxuseclass}

\section{Functions}
\label{\detokenize{mckinney_03_lecture:functions}}\begin{quote}

\sphinxAtStartPar
Functions are the primary and most important method of code organization and reuse in Python. As a rule of thumb, if you anticipate needing to repeat the same or very similar code more than once, it may be worth writing a reusable function. Functions can also help make your code more readable by giving a name to a group of Python statements.

\sphinxAtStartPar
Functions are declared with the def keyword and returned from with the return keyword:

\begin{sphinxVerbatim}[commandchars=\\\{\}]
\PYG{k}{def} \PYG{n+nf}{my\PYGZus{}function}\PYG{p}{(}\PYG{n}{x}\PYG{p}{,} \PYG{n}{y}\PYG{p}{,} \PYG{n}{z}\PYG{o}{=}\PYG{l+m+mf}{1.5}\PYG{p}{)}\PYG{p}{:}
    \PYG{k}{if} \PYG{n}{z} \PYG{o}{\PYGZgt{}} \PYG{l+m+mi}{1}\PYG{p}{:}
         \PYG{k}{return} \PYG{n}{z} \PYG{o}{*} \PYG{p}{(}\PYG{n}{x} \PYG{o}{+} \PYG{n}{y}\PYG{p}{)}
     \PYG{k}{else}\PYG{p}{:}
         \PYG{k}{return} \PYG{n}{z} \PYG{o}{/} \PYG{p}{(}\PYG{n}{x} \PYG{o}{+} \PYG{n}{y}\PYG{p}{)}
\end{sphinxVerbatim}

\sphinxAtStartPar
There is no issue with having multiple return statements. If Python reaches the end of a function without encountering a return statement, None is returned automatically.

\sphinxAtStartPar
Each function can have positional arguments and keyword arguments. Keyword arguments are most commonly used to specify default values or optional arguments. In the preceding function, x and y are positional arguments while z is a keyword argument. This means that the function can be called in any of these ways:

\begin{sphinxVerbatim}[commandchars=\\\{\}]
 \PYG{n}{my\PYGZus{}function}\PYG{p}{(}\PYG{l+m+mi}{5}\PYG{p}{,} \PYG{l+m+mi}{6}\PYG{p}{,} \PYG{n}{z}\PYG{o}{=}\PYG{l+m+mf}{0.7}\PYG{p}{)}
 \PYG{n}{my\PYGZus{}function}\PYG{p}{(}\PYG{l+m+mf}{3.14}\PYG{p}{,} \PYG{l+m+mi}{7}\PYG{p}{,} \PYG{l+m+mf}{3.5}\PYG{p}{)}
 \PYG{n}{my\PYGZus{}function}\PYG{p}{(}\PYG{l+m+mi}{10}\PYG{p}{,} \PYG{l+m+mi}{20}\PYG{p}{)}
\end{sphinxVerbatim}

\sphinxAtStartPar
The main restriction on function arguments is that the keyword arguments must follow the positional arguments (if any). You can specify keyword arguments in any order; this frees you from having to remember which order the function arguments were specified in and only what their names are.
\end{quote}

\sphinxAtStartPar
Here is the basic syntax for a function:

\begin{sphinxuseclass}{cell}\begin{sphinxVerbatimInput}

\begin{sphinxuseclass}{cell_input}
\begin{sphinxVerbatim}[commandchars=\\\{\}]
\PYG{k}{def} \PYG{n+nf}{mult\PYGZus{}by\PYGZus{}two}\PYG{p}{(}\PYG{n}{x}\PYG{p}{)}\PYG{p}{:}
    \PYG{k}{return} \PYG{l+m+mi}{2}\PYG{o}{*}\PYG{n}{x}
\end{sphinxVerbatim}

\end{sphinxuseclass}\end{sphinxVerbatimInput}

\end{sphinxuseclass}

\subsection{Returning Multiple Values}
\label{\detokenize{mckinney_03_lecture:returning-multiple-values}}
\sphinxAtStartPar
We can write Python functions that return multiple objects.
In reality, the function \sphinxcode{\sphinxupquote{f()}} below returns one object, a tuple, that we can unpack to multiple objects.

\begin{sphinxuseclass}{cell}\begin{sphinxVerbatimInput}

\begin{sphinxuseclass}{cell_input}
\begin{sphinxVerbatim}[commandchars=\\\{\}]
\PYG{k}{def} \PYG{n+nf}{f}\PYG{p}{(}\PYG{p}{)}\PYG{p}{:}
    \PYG{n}{a} \PYG{o}{=} \PYG{l+m+mi}{5}
    \PYG{n}{b} \PYG{o}{=} \PYG{l+m+mi}{6}
    \PYG{n}{c} \PYG{o}{=} \PYG{l+m+mi}{7}
    \PYG{k}{return} \PYG{p}{(}\PYG{n}{a}\PYG{p}{,} \PYG{n}{b}\PYG{p}{,} \PYG{n}{c}\PYG{p}{)}
\end{sphinxVerbatim}

\end{sphinxuseclass}\end{sphinxVerbatimInput}

\end{sphinxuseclass}
\begin{sphinxuseclass}{cell}\begin{sphinxVerbatimInput}

\begin{sphinxuseclass}{cell_input}
\begin{sphinxVerbatim}[commandchars=\\\{\}]
\PYG{n}{f}\PYG{p}{(}\PYG{p}{)}
\end{sphinxVerbatim}

\end{sphinxuseclass}\end{sphinxVerbatimInput}
\begin{sphinxVerbatimOutput}

\begin{sphinxuseclass}{cell_output}
\begin{sphinxVerbatim}[commandchars=\\\{\}]
(5, 6, 7)
\end{sphinxVerbatim}

\end{sphinxuseclass}\end{sphinxVerbatimOutput}

\end{sphinxuseclass}
\sphinxAtStartPar
If we want to return multiple objects with names or labels, we can return a dictionary.

\begin{sphinxuseclass}{cell}\begin{sphinxVerbatimInput}

\begin{sphinxuseclass}{cell_input}
\begin{sphinxVerbatim}[commandchars=\\\{\}]
\PYG{k}{def} \PYG{n+nf}{f}\PYG{p}{(}\PYG{p}{)}\PYG{p}{:}
    \PYG{n}{a} \PYG{o}{=} \PYG{l+m+mi}{5}
    \PYG{n}{b} \PYG{o}{=} \PYG{l+m+mi}{6}
    \PYG{n}{c} \PYG{o}{=} \PYG{l+m+mi}{7}
    \PYG{k}{return} \PYG{p}{\PYGZob{}}\PYG{l+s+s1}{\PYGZsq{}}\PYG{l+s+s1}{a}\PYG{l+s+s1}{\PYGZsq{}} \PYG{p}{:} \PYG{n}{a}\PYG{p}{,} \PYG{l+s+s1}{\PYGZsq{}}\PYG{l+s+s1}{b}\PYG{l+s+s1}{\PYGZsq{}} \PYG{p}{:} \PYG{n}{b}\PYG{p}{,} \PYG{l+s+s1}{\PYGZsq{}}\PYG{l+s+s1}{c}\PYG{l+s+s1}{\PYGZsq{}} \PYG{p}{:} \PYG{n}{c}\PYG{p}{\PYGZcb{}}
\end{sphinxVerbatim}

\end{sphinxuseclass}\end{sphinxVerbatimInput}

\end{sphinxuseclass}
\begin{sphinxuseclass}{cell}\begin{sphinxVerbatimInput}

\begin{sphinxuseclass}{cell_input}
\begin{sphinxVerbatim}[commandchars=\\\{\}]
\PYG{n}{f}\PYG{p}{(}\PYG{p}{)}
\end{sphinxVerbatim}

\end{sphinxuseclass}\end{sphinxVerbatimInput}
\begin{sphinxVerbatimOutput}

\begin{sphinxuseclass}{cell_output}
\begin{sphinxVerbatim}[commandchars=\\\{\}]
\PYGZob{}\PYGZsq{}a\PYGZsq{}: 5, \PYGZsq{}b\PYGZsq{}: 6, \PYGZsq{}c\PYGZsq{}: 7\PYGZcb{}
\end{sphinxVerbatim}

\end{sphinxuseclass}\end{sphinxVerbatimOutput}

\end{sphinxuseclass}
\begin{sphinxuseclass}{cell}\begin{sphinxVerbatimInput}

\begin{sphinxuseclass}{cell_input}
\begin{sphinxVerbatim}[commandchars=\\\{\}]
\PYG{n}{f}\PYG{p}{(}\PYG{p}{)}\PYG{p}{[}\PYG{l+s+s1}{\PYGZsq{}}\PYG{l+s+s1}{a}\PYG{l+s+s1}{\PYGZsq{}}\PYG{p}{]}
\end{sphinxVerbatim}

\end{sphinxuseclass}\end{sphinxVerbatimInput}
\begin{sphinxVerbatimOutput}

\begin{sphinxuseclass}{cell_output}
\begin{sphinxVerbatim}[commandchars=\\\{\}]
5
\end{sphinxVerbatim}

\end{sphinxuseclass}\end{sphinxVerbatimOutput}

\end{sphinxuseclass}

\subsection{Anonymous (Lambda) Functions}
\label{\detokenize{mckinney_03_lecture:anonymous-lambda-functions}}\begin{quote}

\sphinxAtStartPar
Python has support for so\sphinxhyphen{}called anonymous or lambda functions, which are a way of writing functions consisting of a single statement, the result of which is the return value. They are defined with the lambda keyword, which has no meaning other than “we are declaring an anonymous function.”
\end{quote}
\begin{quote}

\sphinxAtStartPar
I usually refer to these as lambda functions in the rest of the book. They are especially convenient in data analysis because, as you’ll see, there are many cases where data transformation functions will take functions as arguments. It’s often less typing (and clearer) to pass a lambda function as opposed to writing a full\sphinxhyphen{}out function declaration or even assigning the lambda function to a local variable.
\end{quote}

\sphinxAtStartPar
Lambda functions are very Pythonic and let us to write simple functions on the fly.
For example, we could use a lambda function to sort \sphinxcode{\sphinxupquote{strings}} by the number of unique letters.

\begin{sphinxuseclass}{cell}\begin{sphinxVerbatimInput}

\begin{sphinxuseclass}{cell_input}
\begin{sphinxVerbatim}[commandchars=\\\{\}]
\PYG{n}{strings} \PYG{o}{=} \PYG{p}{[}\PYG{l+s+s1}{\PYGZsq{}}\PYG{l+s+s1}{foo}\PYG{l+s+s1}{\PYGZsq{}}\PYG{p}{,} \PYG{l+s+s1}{\PYGZsq{}}\PYG{l+s+s1}{card}\PYG{l+s+s1}{\PYGZsq{}}\PYG{p}{,} \PYG{l+s+s1}{\PYGZsq{}}\PYG{l+s+s1}{bar}\PYG{l+s+s1}{\PYGZsq{}}\PYG{p}{,} \PYG{l+s+s1}{\PYGZsq{}}\PYG{l+s+s1}{aaaa}\PYG{l+s+s1}{\PYGZsq{}}\PYG{p}{,} \PYG{l+s+s1}{\PYGZsq{}}\PYG{l+s+s1}{abab}\PYG{l+s+s1}{\PYGZsq{}}\PYG{p}{]}
\end{sphinxVerbatim}

\end{sphinxuseclass}\end{sphinxVerbatimInput}

\end{sphinxuseclass}
\begin{sphinxuseclass}{cell}\begin{sphinxVerbatimInput}

\begin{sphinxuseclass}{cell_input}
\begin{sphinxVerbatim}[commandchars=\\\{\}]
\PYG{n}{strings}\PYG{o}{.}\PYG{n}{sort}\PYG{p}{(}\PYG{p}{)}
\PYG{n}{strings}
\end{sphinxVerbatim}

\end{sphinxuseclass}\end{sphinxVerbatimInput}
\begin{sphinxVerbatimOutput}

\begin{sphinxuseclass}{cell_output}
\begin{sphinxVerbatim}[commandchars=\\\{\}]
[\PYGZsq{}aaaa\PYGZsq{}, \PYGZsq{}abab\PYGZsq{}, \PYGZsq{}bar\PYGZsq{}, \PYGZsq{}card\PYGZsq{}, \PYGZsq{}foo\PYGZsq{}]
\end{sphinxVerbatim}

\end{sphinxuseclass}\end{sphinxVerbatimOutput}

\end{sphinxuseclass}
\begin{sphinxuseclass}{cell}\begin{sphinxVerbatimInput}

\begin{sphinxuseclass}{cell_input}
\begin{sphinxVerbatim}[commandchars=\\\{\}]
\PYG{n}{strings}\PYG{o}{.}\PYG{n}{sort}\PYG{p}{(}\PYG{n}{key}\PYG{o}{=}\PYG{n+nb}{len}\PYG{p}{)}
\PYG{n}{strings}
\end{sphinxVerbatim}

\end{sphinxuseclass}\end{sphinxVerbatimInput}
\begin{sphinxVerbatimOutput}

\begin{sphinxuseclass}{cell_output}
\begin{sphinxVerbatim}[commandchars=\\\{\}]
[\PYGZsq{}bar\PYGZsq{}, \PYGZsq{}foo\PYGZsq{}, \PYGZsq{}aaaa\PYGZsq{}, \PYGZsq{}abab\PYGZsq{}, \PYGZsq{}card\PYGZsq{}]
\end{sphinxVerbatim}

\end{sphinxuseclass}\end{sphinxVerbatimOutput}

\end{sphinxuseclass}
\begin{sphinxuseclass}{cell}\begin{sphinxVerbatimInput}

\begin{sphinxuseclass}{cell_input}
\begin{sphinxVerbatim}[commandchars=\\\{\}]
\PYG{n}{strings}\PYG{o}{.}\PYG{n}{sort}\PYG{p}{(}\PYG{n}{key}\PYG{o}{=}\PYG{k}{lambda} \PYG{n}{x}\PYG{p}{:} \PYG{n}{x}\PYG{p}{[}\PYG{o}{\PYGZhy{}}\PYG{l+m+mi}{1}\PYG{p}{]}\PYG{p}{)}
\PYG{n}{strings}
\end{sphinxVerbatim}

\end{sphinxuseclass}\end{sphinxVerbatimInput}
\begin{sphinxVerbatimOutput}

\begin{sphinxuseclass}{cell_output}
\begin{sphinxVerbatim}[commandchars=\\\{\}]
[\PYGZsq{}aaaa\PYGZsq{}, \PYGZsq{}abab\PYGZsq{}, \PYGZsq{}card\PYGZsq{}, \PYGZsq{}foo\PYGZsq{}, \PYGZsq{}bar\PYGZsq{}]
\end{sphinxVerbatim}

\end{sphinxuseclass}\end{sphinxVerbatimOutput}

\end{sphinxuseclass}
\sphinxAtStartPar
How can I sort by the \sphinxstyleemphasis{second} letter in each string?

\begin{sphinxuseclass}{cell}\begin{sphinxVerbatimInput}

\begin{sphinxuseclass}{cell_input}
\begin{sphinxVerbatim}[commandchars=\\\{\}]
\PYG{n}{strings}
\end{sphinxVerbatim}

\end{sphinxuseclass}\end{sphinxVerbatimInput}
\begin{sphinxVerbatimOutput}

\begin{sphinxuseclass}{cell_output}
\begin{sphinxVerbatim}[commandchars=\\\{\}]
[\PYGZsq{}aaaa\PYGZsq{}, \PYGZsq{}abab\PYGZsq{}, \PYGZsq{}card\PYGZsq{}, \PYGZsq{}foo\PYGZsq{}, \PYGZsq{}bar\PYGZsq{}]
\end{sphinxVerbatim}

\end{sphinxuseclass}\end{sphinxVerbatimOutput}

\end{sphinxuseclass}
\begin{sphinxuseclass}{cell}\begin{sphinxVerbatimInput}

\begin{sphinxuseclass}{cell_input}
\begin{sphinxVerbatim}[commandchars=\\\{\}]
\PYG{n}{strings}\PYG{p}{[}\PYG{l+m+mi}{2}\PYG{p}{]}
\end{sphinxVerbatim}

\end{sphinxuseclass}\end{sphinxVerbatimInput}
\begin{sphinxVerbatimOutput}

\begin{sphinxuseclass}{cell_output}
\begin{sphinxVerbatim}[commandchars=\\\{\}]
\PYGZsq{}card\PYGZsq{}
\end{sphinxVerbatim}

\end{sphinxuseclass}\end{sphinxVerbatimOutput}

\end{sphinxuseclass}
\begin{sphinxuseclass}{cell}\begin{sphinxVerbatimInput}

\begin{sphinxuseclass}{cell_input}
\begin{sphinxVerbatim}[commandchars=\\\{\}]
\PYG{n}{strings}\PYG{p}{[}\PYG{l+m+mi}{2}\PYG{p}{]}\PYG{p}{[}\PYG{l+m+mi}{1}\PYG{p}{]}
\end{sphinxVerbatim}

\end{sphinxuseclass}\end{sphinxVerbatimInput}
\begin{sphinxVerbatimOutput}

\begin{sphinxuseclass}{cell_output}
\begin{sphinxVerbatim}[commandchars=\\\{\}]
\PYGZsq{}a\PYGZsq{}
\end{sphinxVerbatim}

\end{sphinxuseclass}\end{sphinxVerbatimOutput}

\end{sphinxuseclass}
\begin{sphinxuseclass}{cell}\begin{sphinxVerbatimInput}

\begin{sphinxuseclass}{cell_input}
\begin{sphinxVerbatim}[commandchars=\\\{\}]
\PYG{n}{strings}\PYG{o}{.}\PYG{n}{sort}\PYG{p}{(}\PYG{n}{key}\PYG{o}{=}\PYG{k}{lambda} \PYG{n}{x}\PYG{p}{:} \PYG{n}{x}\PYG{p}{[}\PYG{l+m+mi}{1}\PYG{p}{]}\PYG{p}{)}
\PYG{n}{strings}
\end{sphinxVerbatim}

\end{sphinxuseclass}\end{sphinxVerbatimInput}
\begin{sphinxVerbatimOutput}

\begin{sphinxuseclass}{cell_output}
\begin{sphinxVerbatim}[commandchars=\\\{\}]
[\PYGZsq{}aaaa\PYGZsq{}, \PYGZsq{}card\PYGZsq{}, \PYGZsq{}bar\PYGZsq{}, \PYGZsq{}abab\PYGZsq{}, \PYGZsq{}foo\PYGZsq{}]
\end{sphinxVerbatim}

\end{sphinxuseclass}\end{sphinxVerbatimOutput}

\end{sphinxuseclass}
\sphinxstepscope


\section{McKinney Chapter 3 \sphinxhyphen{} Practice (Blank)}
\label{\detokenize{mckinney_03_practice:mckinney-chapter-3-practice-blank}}\label{\detokenize{mckinney_03_practice::doc}}

\subsection{Practice}
\label{\detokenize{mckinney_03_practice:practice}}

\subsubsection{Swap the values assigned to \sphinxstyleliteralintitle{\sphinxupquote{a}} and \sphinxstyleliteralintitle{\sphinxupquote{b}} using a third variable \sphinxstyleliteralintitle{\sphinxupquote{c}}.}
\label{\detokenize{mckinney_03_practice:swap-the-values-assigned-to-a-and-b-using-a-third-variable-c}}
\begin{sphinxuseclass}{cell}\begin{sphinxVerbatimInput}

\begin{sphinxuseclass}{cell_input}
\begin{sphinxVerbatim}[commandchars=\\\{\}]
\PYG{n}{a} \PYG{o}{=} \PYG{l+m+mi}{1}
\end{sphinxVerbatim}

\end{sphinxuseclass}\end{sphinxVerbatimInput}

\end{sphinxuseclass}
\begin{sphinxuseclass}{cell}\begin{sphinxVerbatimInput}

\begin{sphinxuseclass}{cell_input}
\begin{sphinxVerbatim}[commandchars=\\\{\}]
\PYG{n}{b} \PYG{o}{=} \PYG{l+m+mi}{2}
\end{sphinxVerbatim}

\end{sphinxuseclass}\end{sphinxVerbatimInput}

\end{sphinxuseclass}

\subsubsection{Swap the values assigned to \sphinxstyleliteralintitle{\sphinxupquote{a}} and \sphinxstyleliteralintitle{\sphinxupquote{b}} \sphinxstyleemphasis{\sphinxstylestrong{without}} using a third variable \sphinxstyleliteralintitle{\sphinxupquote{c}}.}
\label{\detokenize{mckinney_03_practice:swap-the-values-assigned-to-a-and-b-without-using-a-third-variable-c}}
\begin{sphinxuseclass}{cell}\begin{sphinxVerbatimInput}

\begin{sphinxuseclass}{cell_input}
\begin{sphinxVerbatim}[commandchars=\\\{\}]
\PYG{n}{a} \PYG{o}{=} \PYG{l+m+mi}{1}
\end{sphinxVerbatim}

\end{sphinxuseclass}\end{sphinxVerbatimInput}

\end{sphinxuseclass}
\begin{sphinxuseclass}{cell}\begin{sphinxVerbatimInput}

\begin{sphinxuseclass}{cell_input}
\begin{sphinxVerbatim}[commandchars=\\\{\}]
\PYG{n}{b} \PYG{o}{=} \PYG{l+m+mi}{2}
\end{sphinxVerbatim}

\end{sphinxuseclass}\end{sphinxVerbatimInput}

\end{sphinxuseclass}

\subsubsection{What is the output of the following code and why?}
\label{\detokenize{mckinney_03_practice:what-is-the-output-of-the-following-code-and-why}}
\begin{sphinxuseclass}{cell}\begin{sphinxVerbatimInput}

\begin{sphinxuseclass}{cell_input}
\begin{sphinxVerbatim}[commandchars=\\\{\}]
\PYG{l+m+mi}{1}\PYG{p}{,} \PYG{l+m+mi}{1}\PYG{p}{,} \PYG{l+m+mi}{1} \PYG{o}{==} \PYG{p}{(}\PYG{l+m+mi}{1}\PYG{p}{,} \PYG{l+m+mi}{1}\PYG{p}{,} \PYG{l+m+mi}{1}\PYG{p}{)}
\end{sphinxVerbatim}

\end{sphinxuseclass}\end{sphinxVerbatimInput}
\begin{sphinxVerbatimOutput}

\begin{sphinxuseclass}{cell_output}
\begin{sphinxVerbatim}[commandchars=\\\{\}]
(1, 1, False)
\end{sphinxVerbatim}

\end{sphinxuseclass}\end{sphinxVerbatimOutput}

\end{sphinxuseclass}

\subsubsection{Create a list \sphinxstyleliteralintitle{\sphinxupquote{l1}} of integers from 1 to 100.}
\label{\detokenize{mckinney_03_practice:create-a-list-l1-of-integers-from-1-to-100}}

\subsubsection{Slice \sphinxstyleliteralintitle{\sphinxupquote{l1}} to create a list of integers from 60 to 50 (inclusive).}
\label{\detokenize{mckinney_03_practice:slice-l1-to-create-a-list-of-integers-from-60-to-50-inclusive}}
\sphinxAtStartPar
Name this list \sphinxcode{\sphinxupquote{l2}}.


\subsubsection{Create a list \sphinxstyleliteralintitle{\sphinxupquote{l3}} of odd integers from 1 to 21.}
\label{\detokenize{mckinney_03_practice:create-a-list-l3-of-odd-integers-from-1-to-21}}

\subsubsection{Create a list \sphinxstyleliteralintitle{\sphinxupquote{l4}} of the squares of integers from 1 to 100.}
\label{\detokenize{mckinney_03_practice:create-a-list-l4-of-the-squares-of-integers-from-1-to-100}}

\subsubsection{Create a list \sphinxstyleliteralintitle{\sphinxupquote{l5}} that contains the squares of \sphinxstyleemphasis{\sphinxstylestrong{odd}} integers from 1 to 100.}
\label{\detokenize{mckinney_03_practice:create-a-list-l5-that-contains-the-squares-of-odd-integers-from-1-to-100}}

\subsubsection{Use a lambda function to sort \sphinxstyleliteralintitle{\sphinxupquote{strings}} by the last letter.}
\label{\detokenize{mckinney_03_practice:use-a-lambda-function-to-sort-strings-by-the-last-letter}}
\begin{sphinxuseclass}{cell}\begin{sphinxVerbatimInput}

\begin{sphinxuseclass}{cell_input}
\begin{sphinxVerbatim}[commandchars=\\\{\}]
\PYG{n}{strings} \PYG{o}{=} \PYG{p}{[}\PYG{l+s+s1}{\PYGZsq{}}\PYG{l+s+s1}{card}\PYG{l+s+s1}{\PYGZsq{}}\PYG{p}{,} \PYG{l+s+s1}{\PYGZsq{}}\PYG{l+s+s1}{aaaa}\PYG{l+s+s1}{\PYGZsq{}}\PYG{p}{,} \PYG{l+s+s1}{\PYGZsq{}}\PYG{l+s+s1}{foo}\PYG{l+s+s1}{\PYGZsq{}}\PYG{p}{,} \PYG{l+s+s1}{\PYGZsq{}}\PYG{l+s+s1}{bar}\PYG{l+s+s1}{\PYGZsq{}}\PYG{p}{,} \PYG{l+s+s1}{\PYGZsq{}}\PYG{l+s+s1}{abab}\PYG{l+s+s1}{\PYGZsq{}}\PYG{p}{]}
\end{sphinxVerbatim}

\end{sphinxuseclass}\end{sphinxVerbatimInput}

\end{sphinxuseclass}

\subsubsection{Given an integer array \sphinxstyleliteralintitle{\sphinxupquote{nums}} and an integer \sphinxstyleliteralintitle{\sphinxupquote{k}}, return the \$k\textasciicircum{}\{th\}\$ largest element in the array.}
\label{\detokenize{mckinney_03_practice:given-an-integer-array-nums-and-an-integer-k-return-the-k-th-largest-element-in-the-array}}
\sphinxAtStartPar
Note that it is the \$k\textasciicircum{}\{th\}\$ largest element in the sorted order, not the \$k\textasciicircum{}\{th\}\$ distinct element.

\sphinxAtStartPar
Example 1:

\sphinxAtStartPar
Input: \sphinxcode{\sphinxupquote{nums = {[}3,2,1,5,6,4{]}}}, \sphinxcode{\sphinxupquote{k = 2}} \\
Output: \sphinxcode{\sphinxupquote{5}}

\sphinxAtStartPar
Example 2:

\sphinxAtStartPar
Input: \sphinxcode{\sphinxupquote{nums = {[}3,2,3,1,2,4,5,5,6{]}}}, \sphinxcode{\sphinxupquote{k = 4}} \\
Output: \sphinxcode{\sphinxupquote{4}}

\sphinxAtStartPar
I saw this question on \sphinxhref{https://leetcode.com/problems/kth-largest-element-in-an-array/}{LeetCode}.


\subsubsection{Given an integer array \sphinxstyleliteralintitle{\sphinxupquote{nums}} and an integer \sphinxstyleliteralintitle{\sphinxupquote{k}}, return the \sphinxstyleliteralintitle{\sphinxupquote{k}} most frequent elements.}
\label{\detokenize{mckinney_03_practice:given-an-integer-array-nums-and-an-integer-k-return-the-k-most-frequent-elements}}
\sphinxAtStartPar
You may return the answer in any order.

\sphinxAtStartPar
Example 1:

\sphinxAtStartPar
Input: \sphinxcode{\sphinxupquote{nums = {[}1,1,1,2,2,3{]}}}, \sphinxcode{\sphinxupquote{k = 2}} \\
Output: \sphinxcode{\sphinxupquote{{[}1,2{]}}}

\sphinxAtStartPar
Example 2:

\sphinxAtStartPar
Input: \sphinxcode{\sphinxupquote{nums = {[}1{]}}}, \sphinxcode{\sphinxupquote{k = 1}} \\
Output: \sphinxcode{\sphinxupquote{{[}1{]}}}

\sphinxAtStartPar
I saw this question on \sphinxhref{https://leetcode.com/problems/top-k-frequent-elements/}{LeetCode}.


\subsubsection{Test whether the given strings are palindromes.}
\label{\detokenize{mckinney_03_practice:test-whether-the-given-strings-are-palindromes}}
\sphinxAtStartPar
Input: \sphinxcode{\sphinxupquote{{[}"aba", "no"{]}}} \\
Output: \sphinxcode{\sphinxupquote{{[}True, False{]}}}


\subsubsection{Write a function \sphinxstyleliteralintitle{\sphinxupquote{returns()}} that accepts lists of prices and dividends and returns a list of returns.}
\label{\detokenize{mckinney_03_practice:write-a-function-returns-that-accepts-lists-of-prices-and-dividends-and-returns-a-list-of-returns}}
\begin{sphinxuseclass}{cell}\begin{sphinxVerbatimInput}

\begin{sphinxuseclass}{cell_input}
\begin{sphinxVerbatim}[commandchars=\\\{\}]
\PYG{n}{prices} \PYG{o}{=} \PYG{p}{[}\PYG{l+m+mi}{100}\PYG{p}{,} \PYG{l+m+mi}{150}\PYG{p}{,} \PYG{l+m+mi}{100}\PYG{p}{,} \PYG{l+m+mi}{50}\PYG{p}{,} \PYG{l+m+mi}{100}\PYG{p}{,} \PYG{l+m+mi}{150}\PYG{p}{,} \PYG{l+m+mi}{100}\PYG{p}{,} \PYG{l+m+mi}{150}\PYG{p}{]}
\PYG{n}{dividends} \PYG{o}{=} \PYG{p}{[}\PYG{l+m+mi}{1}\PYG{p}{,} \PYG{l+m+mi}{1}\PYG{p}{,} \PYG{l+m+mi}{1}\PYG{p}{,} \PYG{l+m+mi}{1}\PYG{p}{,} \PYG{l+m+mi}{2}\PYG{p}{,} \PYG{l+m+mi}{2}\PYG{p}{,} \PYG{l+m+mi}{2}\PYG{p}{,} \PYG{l+m+mi}{2}\PYG{p}{]}
\end{sphinxVerbatim}

\end{sphinxuseclass}\end{sphinxVerbatimInput}

\end{sphinxuseclass}

\subsubsection{Rewrite the function \sphinxstyleliteralintitle{\sphinxupquote{returns()}} so it returns lists of returns, capital gains yields, and dividend yields.}
\label{\detokenize{mckinney_03_practice:rewrite-the-function-returns-so-it-returns-lists-of-returns-capital-gains-yields-and-dividend-yields}}

\subsubsection{Rescale and shift numbers so that they cover the range \sphinxstyleliteralintitle{\sphinxupquote{{[}0, 1{]}}}.}
\label{\detokenize{mckinney_03_practice:rescale-and-shift-numbers-so-that-they-cover-the-range-0-1}}
\sphinxAtStartPar
Input: \sphinxcode{\sphinxupquote{{[}18.5, 17.0, 18.0, 19.0, 18.0{]}}} \\
Output: \sphinxcode{\sphinxupquote{{[}0.75, 0.0, 0.5, 1.0, 0.5{]}}}

\begin{sphinxuseclass}{cell}\begin{sphinxVerbatimInput}

\begin{sphinxuseclass}{cell_input}
\begin{sphinxVerbatim}[commandchars=\\\{\}]
\PYG{n}{numbers} \PYG{o}{=} \PYG{p}{[}\PYG{l+m+mf}{18.5}\PYG{p}{,} \PYG{l+m+mf}{17.0}\PYG{p}{,} \PYG{l+m+mf}{18.0}\PYG{p}{,} \PYG{l+m+mf}{19.0}\PYG{p}{,} \PYG{l+m+mf}{18.0}\PYG{p}{]}
\end{sphinxVerbatim}

\end{sphinxuseclass}\end{sphinxVerbatimInput}

\end{sphinxuseclass}
\sphinxstepscope


\section{McKinney Chapter 3 \sphinxhyphen{} Practice (Section 4, Wednesday 11:45 AM)}
\label{\detokenize{mckinney_03_practice_04:mckinney-chapter-3-practice-section-4-wednesday-11-45-am}}\label{\detokenize{mckinney_03_practice_04::doc}}

\subsection{Practice}
\label{\detokenize{mckinney_03_practice_04:practice}}

\subsubsection{Swap the values assigned to \sphinxstyleliteralintitle{\sphinxupquote{a}} and \sphinxstyleliteralintitle{\sphinxupquote{b}} using a third variable \sphinxstyleliteralintitle{\sphinxupquote{c}}.}
\label{\detokenize{mckinney_03_practice_04:swap-the-values-assigned-to-a-and-b-using-a-third-variable-c}}
\begin{sphinxuseclass}{cell}\begin{sphinxVerbatimInput}

\begin{sphinxuseclass}{cell_input}
\begin{sphinxVerbatim}[commandchars=\\\{\}]
\PYG{n}{a} \PYG{o}{=} \PYG{l+m+mi}{1}
\end{sphinxVerbatim}

\end{sphinxuseclass}\end{sphinxVerbatimInput}

\end{sphinxuseclass}
\begin{sphinxuseclass}{cell}\begin{sphinxVerbatimInput}

\begin{sphinxuseclass}{cell_input}
\begin{sphinxVerbatim}[commandchars=\\\{\}]
\PYG{n}{b} \PYG{o}{=} \PYG{l+m+mi}{2}
\end{sphinxVerbatim}

\end{sphinxuseclass}\end{sphinxVerbatimInput}

\end{sphinxuseclass}
\begin{sphinxuseclass}{cell}\begin{sphinxVerbatimInput}

\begin{sphinxuseclass}{cell_input}
\begin{sphinxVerbatim}[commandchars=\\\{\}]
\PYG{n+nb}{print}\PYG{p}{(}\PYG{l+s+sa}{f}\PYG{l+s+s1}{\PYGZsq{}}\PYG{l+s+s1}{a is }\PYG{l+s+si}{\PYGZob{}}\PYG{n}{a}\PYG{l+s+si}{\PYGZcb{}}\PYG{l+s+s1}{ and b is }\PYG{l+s+si}{\PYGZob{}}\PYG{n}{b}\PYG{l+s+si}{\PYGZcb{}}\PYG{l+s+s1}{\PYGZsq{}}\PYG{p}{)}
\end{sphinxVerbatim}

\end{sphinxuseclass}\end{sphinxVerbatimInput}
\begin{sphinxVerbatimOutput}

\begin{sphinxuseclass}{cell_output}
\begin{sphinxVerbatim}[commandchars=\\\{\}]
a is 1 and b is 2
\end{sphinxVerbatim}

\end{sphinxuseclass}\end{sphinxVerbatimOutput}

\end{sphinxuseclass}
\begin{sphinxuseclass}{cell}\begin{sphinxVerbatimInput}

\begin{sphinxuseclass}{cell_input}
\begin{sphinxVerbatim}[commandchars=\\\{\}]
\PYG{n}{c} \PYG{o}{=} \PYG{n}{a}
\end{sphinxVerbatim}

\end{sphinxuseclass}\end{sphinxVerbatimInput}

\end{sphinxuseclass}
\begin{sphinxuseclass}{cell}\begin{sphinxVerbatimInput}

\begin{sphinxuseclass}{cell_input}
\begin{sphinxVerbatim}[commandchars=\\\{\}]
\PYG{n}{c} \PYG{o}{==} \PYG{n}{a}
\end{sphinxVerbatim}

\end{sphinxuseclass}\end{sphinxVerbatimInput}
\begin{sphinxVerbatimOutput}

\begin{sphinxuseclass}{cell_output}
\begin{sphinxVerbatim}[commandchars=\\\{\}]
True
\end{sphinxVerbatim}

\end{sphinxuseclass}\end{sphinxVerbatimOutput}

\end{sphinxuseclass}
\begin{sphinxuseclass}{cell}\begin{sphinxVerbatimInput}

\begin{sphinxuseclass}{cell_input}
\begin{sphinxVerbatim}[commandchars=\\\{\}]
\PYG{n}{c} \PYG{o+ow}{is} \PYG{n}{a}
\end{sphinxVerbatim}

\end{sphinxuseclass}\end{sphinxVerbatimInput}
\begin{sphinxVerbatimOutput}

\begin{sphinxuseclass}{cell_output}
\begin{sphinxVerbatim}[commandchars=\\\{\}]
True
\end{sphinxVerbatim}

\end{sphinxuseclass}\end{sphinxVerbatimOutput}

\end{sphinxuseclass}
\begin{sphinxuseclass}{cell}\begin{sphinxVerbatimInput}

\begin{sphinxuseclass}{cell_input}
\begin{sphinxVerbatim}[commandchars=\\\{\}]
\PYG{n}{a} \PYG{o}{=} \PYG{n}{b}
\end{sphinxVerbatim}

\end{sphinxuseclass}\end{sphinxVerbatimInput}

\end{sphinxuseclass}
\begin{sphinxuseclass}{cell}\begin{sphinxVerbatimInput}

\begin{sphinxuseclass}{cell_input}
\begin{sphinxVerbatim}[commandchars=\\\{\}]
\PYG{n}{b} \PYG{o}{=} \PYG{n}{c}
\end{sphinxVerbatim}

\end{sphinxuseclass}\end{sphinxVerbatimInput}

\end{sphinxuseclass}
\begin{sphinxuseclass}{cell}\begin{sphinxVerbatimInput}

\begin{sphinxuseclass}{cell_input}
\begin{sphinxVerbatim}[commandchars=\\\{\}]
\PYG{n+nb}{print}\PYG{p}{(}\PYG{l+s+sa}{f}\PYG{l+s+s1}{\PYGZsq{}}\PYG{l+s+s1}{a is }\PYG{l+s+si}{\PYGZob{}}\PYG{n}{a}\PYG{l+s+si}{\PYGZcb{}}\PYG{l+s+s1}{ and b is }\PYG{l+s+si}{\PYGZob{}}\PYG{n}{b}\PYG{l+s+si}{\PYGZcb{}}\PYG{l+s+s1}{\PYGZsq{}}\PYG{p}{)}
\end{sphinxVerbatim}

\end{sphinxuseclass}\end{sphinxVerbatimInput}
\begin{sphinxVerbatimOutput}

\begin{sphinxuseclass}{cell_output}
\begin{sphinxVerbatim}[commandchars=\\\{\}]
a is 2 and b is 1
\end{sphinxVerbatim}

\end{sphinxuseclass}\end{sphinxVerbatimOutput}

\end{sphinxuseclass}
\begin{sphinxuseclass}{cell}\begin{sphinxVerbatimInput}

\begin{sphinxuseclass}{cell_input}
\begin{sphinxVerbatim}[commandchars=\\\{\}]
\PYG{o}{\PYGZpc{}}\PYG{k}{who}
\end{sphinxVerbatim}

\end{sphinxuseclass}\end{sphinxVerbatimInput}
\begin{sphinxVerbatimOutput}

\begin{sphinxuseclass}{cell_output}
\begin{sphinxVerbatim}[commandchars=\\\{\}]
a	 b	 c	 
\end{sphinxVerbatim}

\end{sphinxuseclass}\end{sphinxVerbatimOutput}

\end{sphinxuseclass}
\begin{sphinxuseclass}{cell}\begin{sphinxVerbatimInput}

\begin{sphinxuseclass}{cell_input}
\begin{sphinxVerbatim}[commandchars=\\\{\}]
\PYG{k}{del} \PYG{n}{c}
\end{sphinxVerbatim}

\end{sphinxuseclass}\end{sphinxVerbatimInput}

\end{sphinxuseclass}
\begin{sphinxuseclass}{cell}\begin{sphinxVerbatimInput}

\begin{sphinxuseclass}{cell_input}
\begin{sphinxVerbatim}[commandchars=\\\{\}]
\PYG{o}{\PYGZpc{}}\PYG{k}{who}
\end{sphinxVerbatim}

\end{sphinxuseclass}\end{sphinxVerbatimInput}
\begin{sphinxVerbatimOutput}

\begin{sphinxuseclass}{cell_output}
\begin{sphinxVerbatim}[commandchars=\\\{\}]
a	 b	 
\end{sphinxVerbatim}

\end{sphinxuseclass}\end{sphinxVerbatimOutput}

\end{sphinxuseclass}

\subsubsection{Swap the values assigned to \sphinxstyleliteralintitle{\sphinxupquote{a}} and \sphinxstyleliteralintitle{\sphinxupquote{b}} \sphinxstyleemphasis{\sphinxstylestrong{without}} using a third variable \sphinxstyleliteralintitle{\sphinxupquote{c}}.}
\label{\detokenize{mckinney_03_practice_04:swap-the-values-assigned-to-a-and-b-without-using-a-third-variable-c}}
\begin{sphinxuseclass}{cell}\begin{sphinxVerbatimInput}

\begin{sphinxuseclass}{cell_input}
\begin{sphinxVerbatim}[commandchars=\\\{\}]
\PYG{n}{a} \PYG{o}{=} \PYG{l+m+mi}{1}
\end{sphinxVerbatim}

\end{sphinxuseclass}\end{sphinxVerbatimInput}

\end{sphinxuseclass}
\begin{sphinxuseclass}{cell}\begin{sphinxVerbatimInput}

\begin{sphinxuseclass}{cell_input}
\begin{sphinxVerbatim}[commandchars=\\\{\}]
\PYG{n}{b} \PYG{o}{=} \PYG{l+m+mi}{2}
\end{sphinxVerbatim}

\end{sphinxuseclass}\end{sphinxVerbatimInput}

\end{sphinxuseclass}
\begin{sphinxuseclass}{cell}\begin{sphinxVerbatimInput}

\begin{sphinxuseclass}{cell_input}
\begin{sphinxVerbatim}[commandchars=\\\{\}]
\PYG{n+nb}{print}\PYG{p}{(}\PYG{l+s+sa}{f}\PYG{l+s+s1}{\PYGZsq{}}\PYG{l+s+s1}{a is }\PYG{l+s+si}{\PYGZob{}}\PYG{n}{a}\PYG{l+s+si}{\PYGZcb{}}\PYG{l+s+s1}{ and b is }\PYG{l+s+si}{\PYGZob{}}\PYG{n}{b}\PYG{l+s+si}{\PYGZcb{}}\PYG{l+s+s1}{\PYGZsq{}}\PYG{p}{)}
\end{sphinxVerbatim}

\end{sphinxuseclass}\end{sphinxVerbatimInput}
\begin{sphinxVerbatimOutput}

\begin{sphinxuseclass}{cell_output}
\begin{sphinxVerbatim}[commandchars=\\\{\}]
a is 1 and b is 2
\end{sphinxVerbatim}

\end{sphinxuseclass}\end{sphinxVerbatimOutput}

\end{sphinxuseclass}
\begin{sphinxuseclass}{cell}\begin{sphinxVerbatimInput}

\begin{sphinxuseclass}{cell_input}
\begin{sphinxVerbatim}[commandchars=\\\{\}]
\PYG{n+nb}{type}\PYG{p}{(}\PYG{p}{(}\PYG{n}{a}\PYG{p}{,} \PYG{n}{b}\PYG{p}{)}\PYG{p}{)}
\end{sphinxVerbatim}

\end{sphinxuseclass}\end{sphinxVerbatimInput}
\begin{sphinxVerbatimOutput}

\begin{sphinxuseclass}{cell_output}
\begin{sphinxVerbatim}[commandchars=\\\{\}]
tuple
\end{sphinxVerbatim}

\end{sphinxuseclass}\end{sphinxVerbatimOutput}

\end{sphinxuseclass}
\begin{sphinxuseclass}{cell}\begin{sphinxVerbatimInput}

\begin{sphinxuseclass}{cell_input}
\begin{sphinxVerbatim}[commandchars=\\\{\}]
\PYG{n}{b}\PYG{p}{,} \PYG{n}{a} \PYG{o}{=} \PYG{n}{a}\PYG{p}{,} \PYG{n}{b}
\end{sphinxVerbatim}

\end{sphinxuseclass}\end{sphinxVerbatimInput}

\end{sphinxuseclass}
\begin{sphinxuseclass}{cell}\begin{sphinxVerbatimInput}

\begin{sphinxuseclass}{cell_input}
\begin{sphinxVerbatim}[commandchars=\\\{\}]
\PYG{n+nb}{print}\PYG{p}{(}\PYG{l+s+sa}{f}\PYG{l+s+s1}{\PYGZsq{}}\PYG{l+s+s1}{a is }\PYG{l+s+si}{\PYGZob{}}\PYG{n}{a}\PYG{l+s+si}{\PYGZcb{}}\PYG{l+s+s1}{ and b is }\PYG{l+s+si}{\PYGZob{}}\PYG{n}{b}\PYG{l+s+si}{\PYGZcb{}}\PYG{l+s+s1}{\PYGZsq{}}\PYG{p}{)}
\end{sphinxVerbatim}

\end{sphinxuseclass}\end{sphinxVerbatimInput}
\begin{sphinxVerbatimOutput}

\begin{sphinxuseclass}{cell_output}
\begin{sphinxVerbatim}[commandchars=\\\{\}]
a is 2 and b is 1
\end{sphinxVerbatim}

\end{sphinxuseclass}\end{sphinxVerbatimOutput}

\end{sphinxuseclass}

\subsubsection{What is the output of the following code and why?}
\label{\detokenize{mckinney_03_practice_04:what-is-the-output-of-the-following-code-and-why}}
\begin{sphinxuseclass}{cell}\begin{sphinxVerbatimInput}

\begin{sphinxuseclass}{cell_input}
\begin{sphinxVerbatim}[commandchars=\\\{\}]
\PYG{l+m+mi}{1} \PYG{o}{==} \PYG{p}{(}\PYG{l+m+mi}{1}\PYG{p}{,} \PYG{l+m+mi}{1}\PYG{p}{,} \PYG{l+m+mi}{1}\PYG{p}{)}
\end{sphinxVerbatim}

\end{sphinxuseclass}\end{sphinxVerbatimInput}
\begin{sphinxVerbatimOutput}

\begin{sphinxuseclass}{cell_output}
\begin{sphinxVerbatim}[commandchars=\\\{\}]
False
\end{sphinxVerbatim}

\end{sphinxuseclass}\end{sphinxVerbatimOutput}

\end{sphinxuseclass}
\begin{sphinxuseclass}{cell}\begin{sphinxVerbatimInput}

\begin{sphinxuseclass}{cell_input}
\begin{sphinxVerbatim}[commandchars=\\\{\}]
\PYG{l+m+mi}{1}\PYG{p}{,} \PYG{l+m+mi}{1}\PYG{p}{,} \PYG{l+m+mi}{1} \PYG{o}{==} \PYG{p}{(}\PYG{l+m+mi}{1}\PYG{p}{,} \PYG{l+m+mi}{1}\PYG{p}{,} \PYG{l+m+mi}{1}\PYG{p}{)}
\end{sphinxVerbatim}

\end{sphinxuseclass}\end{sphinxVerbatimInput}
\begin{sphinxVerbatimOutput}

\begin{sphinxuseclass}{cell_output}
\begin{sphinxVerbatim}[commandchars=\\\{\}]
(1, 1, False)
\end{sphinxVerbatim}

\end{sphinxuseclass}\end{sphinxVerbatimOutput}

\end{sphinxuseclass}
\begin{sphinxuseclass}{cell}\begin{sphinxVerbatimInput}

\begin{sphinxuseclass}{cell_input}
\begin{sphinxVerbatim}[commandchars=\\\{\}]
\PYG{l+m+mi}{1}\PYG{p}{,} \PYG{l+m+mi}{1} \PYG{o}{==} \PYG{p}{(}\PYG{l+m+mi}{1}\PYG{p}{,} \PYG{l+m+mi}{1}\PYG{p}{,} \PYG{l+m+mi}{1}\PYG{p}{)}
\end{sphinxVerbatim}

\end{sphinxuseclass}\end{sphinxVerbatimInput}
\begin{sphinxVerbatimOutput}

\begin{sphinxuseclass}{cell_output}
\begin{sphinxVerbatim}[commandchars=\\\{\}]
(1, False)
\end{sphinxVerbatim}

\end{sphinxuseclass}\end{sphinxVerbatimOutput}

\end{sphinxuseclass}
\begin{sphinxuseclass}{cell}\begin{sphinxVerbatimInput}

\begin{sphinxuseclass}{cell_input}
\begin{sphinxVerbatim}[commandchars=\\\{\}]
\PYG{p}{(}\PYG{l+m+mi}{1}\PYG{p}{,} \PYG{l+m+mi}{1}\PYG{p}{,} \PYG{l+m+mi}{1}\PYG{p}{)} \PYG{o}{==} \PYG{p}{(}\PYG{l+m+mi}{1}\PYG{p}{,} \PYG{l+m+mi}{1}\PYG{p}{,} \PYG{l+m+mi}{1}\PYG{p}{)}
\end{sphinxVerbatim}

\end{sphinxuseclass}\end{sphinxVerbatimInput}
\begin{sphinxVerbatimOutput}

\begin{sphinxuseclass}{cell_output}
\begin{sphinxVerbatim}[commandchars=\\\{\}]
True
\end{sphinxVerbatim}

\end{sphinxuseclass}\end{sphinxVerbatimOutput}

\end{sphinxuseclass}
\begin{sphinxuseclass}{cell}\begin{sphinxVerbatimInput}

\begin{sphinxuseclass}{cell_input}
\begin{sphinxVerbatim}[commandchars=\\\{\}]
\PYG{p}{(}\PYG{l+m+mi}{1}\PYG{p}{,} \PYG{l+m+mi}{1}\PYG{p}{,} \PYG{l+m+mi}{1}\PYG{p}{)} \PYG{o}{==} \PYG{l+m+mi}{1}\PYG{p}{,} \PYG{l+m+mi}{1}\PYG{p}{,} \PYG{l+m+mi}{1}
\end{sphinxVerbatim}

\end{sphinxuseclass}\end{sphinxVerbatimInput}
\begin{sphinxVerbatimOutput}

\begin{sphinxuseclass}{cell_output}
\begin{sphinxVerbatim}[commandchars=\\\{\}]
(False, 1, 1)
\end{sphinxVerbatim}

\end{sphinxuseclass}\end{sphinxVerbatimOutput}

\end{sphinxuseclass}
\begin{sphinxuseclass}{cell}\begin{sphinxVerbatimInput}

\begin{sphinxuseclass}{cell_input}
\begin{sphinxVerbatim}[commandchars=\\\{\}]
\PYG{l+m+mi}{1}\PYG{p}{,} \PYG{l+m+mi}{1}\PYG{p}{,} \PYG{l+m+mi}{1} \PYG{o}{==} \PYG{l+m+mi}{1}\PYG{p}{,} \PYG{l+m+mi}{1}\PYG{p}{,} \PYG{l+m+mi}{1}
\end{sphinxVerbatim}

\end{sphinxuseclass}\end{sphinxVerbatimInput}
\begin{sphinxVerbatimOutput}

\begin{sphinxuseclass}{cell_output}
\begin{sphinxVerbatim}[commandchars=\\\{\}]
(1, 1, True, 1, 1)
\end{sphinxVerbatim}

\end{sphinxuseclass}\end{sphinxVerbatimOutput}

\end{sphinxuseclass}

\subsubsection{Create a list \sphinxstyleliteralintitle{\sphinxupquote{l1}} of integers from 1 to 100.}
\label{\detokenize{mckinney_03_practice_04:create-a-list-l1-of-integers-from-1-to-100}}
\begin{sphinxuseclass}{cell}\begin{sphinxVerbatimInput}

\begin{sphinxuseclass}{cell_input}
\begin{sphinxVerbatim}[commandchars=\\\{\}]
\PYG{n}{l1} \PYG{o}{=} \PYG{n+nb}{list}\PYG{p}{(}\PYG{n+nb}{range}\PYG{p}{(}\PYG{l+m+mi}{1}\PYG{p}{,} \PYG{l+m+mi}{101}\PYG{p}{)}\PYG{p}{)}
\PYG{n+nb}{print}\PYG{p}{(}\PYG{n}{l1}\PYG{p}{)}
\end{sphinxVerbatim}

\end{sphinxuseclass}\end{sphinxVerbatimInput}
\begin{sphinxVerbatimOutput}

\begin{sphinxuseclass}{cell_output}
\begin{sphinxVerbatim}[commandchars=\\\{\}]
[1, 2, 3, 4, 5, 6, 7, 8, 9, 10, 11, 12, 13, 14, 15, 16, 17, 18, 19, 20, 21, 22, 23, 24, 25, 26, 27, 28, 29, 30, 31, 32, 33, 34, 35, 36, 37, 38, 39, 40, 41, 42, 43, 44, 45, 46, 47, 48, 49, 50, 51, 52, 53, 54, 55, 56, 57, 58, 59, 60, 61, 62, 63, 64, 65, 66, 67, 68, 69, 70, 71, 72, 73, 74, 75, 76, 77, 78, 79, 80, 81, 82, 83, 84, 85, 86, 87, 88, 89, 90, 91, 92, 93, 94, 95, 96, 97, 98, 99, 100]
\end{sphinxVerbatim}

\end{sphinxuseclass}\end{sphinxVerbatimOutput}

\end{sphinxuseclass}
\begin{sphinxuseclass}{cell}\begin{sphinxVerbatimInput}

\begin{sphinxuseclass}{cell_input}
\begin{sphinxVerbatim}[commandchars=\\\{\}]
\PYG{n}{l1}\PYG{p}{[}\PYG{p}{:}\PYG{l+m+mi}{5}\PYG{p}{]} 
\end{sphinxVerbatim}

\end{sphinxuseclass}\end{sphinxVerbatimInput}
\begin{sphinxVerbatimOutput}

\begin{sphinxuseclass}{cell_output}
\begin{sphinxVerbatim}[commandchars=\\\{\}]
[1, 2, 3, 4, 5]
\end{sphinxVerbatim}

\end{sphinxuseclass}\end{sphinxVerbatimOutput}

\end{sphinxuseclass}
\begin{sphinxuseclass}{cell}\begin{sphinxVerbatimInput}

\begin{sphinxuseclass}{cell_input}
\begin{sphinxVerbatim}[commandchars=\\\{\}]
\PYG{n}{l1}\PYG{p}{[}\PYG{l+m+mi}{5}\PYG{p}{:}\PYG{l+m+mi}{10}\PYG{p}{]}
\end{sphinxVerbatim}

\end{sphinxuseclass}\end{sphinxVerbatimInput}
\begin{sphinxVerbatimOutput}

\begin{sphinxuseclass}{cell_output}
\begin{sphinxVerbatim}[commandchars=\\\{\}]
[6, 7, 8, 9, 10]
\end{sphinxVerbatim}

\end{sphinxuseclass}\end{sphinxVerbatimOutput}

\end{sphinxuseclass}

\subsubsection{Slice \sphinxstyleliteralintitle{\sphinxupquote{l1}} to create a list of integers from 60 to 50 (inclusive).}
\label{\detokenize{mckinney_03_practice_04:slice-l1-to-create-a-list-of-integers-from-60-to-50-inclusive}}
\sphinxAtStartPar
Name this list \sphinxcode{\sphinxupquote{l2}}.

\begin{sphinxuseclass}{cell}\begin{sphinxVerbatimInput}

\begin{sphinxuseclass}{cell_input}
\begin{sphinxVerbatim}[commandchars=\\\{\}]
\PYG{n}{l2} \PYG{o}{=} \PYG{n}{l1}\PYG{p}{[}\PYG{l+m+mi}{49}\PYG{p}{:}\PYG{l+m+mi}{60}\PYG{p}{]}
\PYG{n}{l2}\PYG{o}{.}\PYG{n}{sort}\PYG{p}{(}\PYG{n}{reverse}\PYG{o}{=}\PYG{k+kc}{True}\PYG{p}{)}
\PYG{n+nb}{print}\PYG{p}{(}\PYG{n}{l2}\PYG{p}{)}
\end{sphinxVerbatim}

\end{sphinxuseclass}\end{sphinxVerbatimInput}
\begin{sphinxVerbatimOutput}

\begin{sphinxuseclass}{cell_output}
\begin{sphinxVerbatim}[commandchars=\\\{\}]
[60, 59, 58, 57, 56, 55, 54, 53, 52, 51, 50]
\end{sphinxVerbatim}

\end{sphinxuseclass}\end{sphinxVerbatimOutput}

\end{sphinxuseclass}
\begin{sphinxuseclass}{cell}\begin{sphinxVerbatimInput}

\begin{sphinxuseclass}{cell_input}
\begin{sphinxVerbatim}[commandchars=\\\{\}]
\PYG{n}{l1}\PYG{p}{[}\PYG{l+m+mi}{59}\PYG{p}{:}\PYG{l+m+mi}{48}\PYG{p}{:}\PYG{o}{\PYGZhy{}}\PYG{l+m+mi}{1}\PYG{p}{]}
\end{sphinxVerbatim}

\end{sphinxuseclass}\end{sphinxVerbatimInput}
\begin{sphinxVerbatimOutput}

\begin{sphinxuseclass}{cell_output}
\begin{sphinxVerbatim}[commandchars=\\\{\}]
[60, 59, 58, 57, 56, 55, 54, 53, 52, 51, 50]
\end{sphinxVerbatim}

\end{sphinxuseclass}\end{sphinxVerbatimOutput}

\end{sphinxuseclass}
\begin{sphinxuseclass}{cell}\begin{sphinxVerbatimInput}

\begin{sphinxuseclass}{cell_input}
\begin{sphinxVerbatim}[commandchars=\\\{\}]
\PYG{n}{l1}\PYG{p}{[}\PYG{l+m+mi}{49}\PYG{p}{:}\PYG{l+m+mi}{60}\PYG{p}{]}\PYG{p}{[}\PYG{p}{:}\PYG{p}{:}\PYG{o}{\PYGZhy{}}\PYG{l+m+mi}{1}\PYG{p}{]}
\end{sphinxVerbatim}

\end{sphinxuseclass}\end{sphinxVerbatimInput}
\begin{sphinxVerbatimOutput}

\begin{sphinxuseclass}{cell_output}
\begin{sphinxVerbatim}[commandchars=\\\{\}]
[60, 59, 58, 57, 56, 55, 54, 53, 52, 51, 50]
\end{sphinxVerbatim}

\end{sphinxuseclass}\end{sphinxVerbatimOutput}

\end{sphinxuseclass}
\begin{sphinxuseclass}{cell}\begin{sphinxVerbatimInput}

\begin{sphinxuseclass}{cell_input}
\begin{sphinxVerbatim}[commandchars=\\\{\}]
\PYG{n}{l2} \PYG{o}{=} \PYG{n}{l1}\PYG{p}{[}\PYG{l+m+mi}{49}\PYG{p}{:}\PYG{l+m+mi}{60}\PYG{p}{]}
\PYG{n}{l2}\PYG{o}{.}\PYG{n}{reverse}\PYG{p}{(}\PYG{p}{)}
\PYG{n}{l2}
\end{sphinxVerbatim}

\end{sphinxuseclass}\end{sphinxVerbatimInput}
\begin{sphinxVerbatimOutput}

\begin{sphinxuseclass}{cell_output}
\begin{sphinxVerbatim}[commandchars=\\\{\}]
[60, 59, 58, 57, 56, 55, 54, 53, 52, 51, 50]
\end{sphinxVerbatim}

\end{sphinxuseclass}\end{sphinxVerbatimOutput}

\end{sphinxuseclass}

\subsubsection{Create a list \sphinxstyleliteralintitle{\sphinxupquote{l3}} of odd integers from 1 to 21.}
\label{\detokenize{mckinney_03_practice_04:create-a-list-l3-of-odd-integers-from-1-to-21}}
\begin{sphinxuseclass}{cell}\begin{sphinxVerbatimInput}

\begin{sphinxuseclass}{cell_input}
\begin{sphinxVerbatim}[commandchars=\\\{\}]
\PYG{n+nb}{list}\PYG{p}{(}\PYG{n+nb}{range}\PYG{p}{(}\PYG{l+m+mi}{1}\PYG{p}{,} \PYG{l+m+mi}{22}\PYG{p}{,} \PYG{l+m+mi}{2}\PYG{p}{)}\PYG{p}{)}
\end{sphinxVerbatim}

\end{sphinxuseclass}\end{sphinxVerbatimInput}
\begin{sphinxVerbatimOutput}

\begin{sphinxuseclass}{cell_output}
\begin{sphinxVerbatim}[commandchars=\\\{\}]
[1, 3, 5, 7, 9, 11, 13, 15, 17, 19, 21]
\end{sphinxVerbatim}

\end{sphinxuseclass}\end{sphinxVerbatimOutput}

\end{sphinxuseclass}
\begin{sphinxuseclass}{cell}\begin{sphinxVerbatimInput}

\begin{sphinxuseclass}{cell_input}
\begin{sphinxVerbatim}[commandchars=\\\{\}]
\PYG{n}{l1}\PYG{p}{[}\PYG{l+m+mi}{0}\PYG{p}{:}\PYG{l+m+mi}{21}\PYG{p}{:}\PYG{l+m+mi}{2}\PYG{p}{]}
\end{sphinxVerbatim}

\end{sphinxuseclass}\end{sphinxVerbatimInput}
\begin{sphinxVerbatimOutput}

\begin{sphinxuseclass}{cell_output}
\begin{sphinxVerbatim}[commandchars=\\\{\}]
[1, 3, 5, 7, 9, 11, 13, 15, 17, 19, 21]
\end{sphinxVerbatim}

\end{sphinxuseclass}\end{sphinxVerbatimOutput}

\end{sphinxuseclass}

\subsubsection{Create a list \sphinxstyleliteralintitle{\sphinxupquote{l4}} of the squares of integers from 1 to 100.}
\label{\detokenize{mckinney_03_practice_04:create-a-list-l4-of-the-squares-of-integers-from-1-to-100}}
\begin{sphinxuseclass}{cell}\begin{sphinxVerbatimInput}

\begin{sphinxuseclass}{cell_input}
\begin{sphinxVerbatim}[commandchars=\\\{\}]
\PYG{n}{l4} \PYG{o}{=} \PYG{p}{[}\PYG{n}{x}\PYG{o}{*}\PYG{o}{*}\PYG{l+m+mi}{2} \PYG{k}{for} \PYG{n}{x} \PYG{o+ow}{in} \PYG{n+nb}{range}\PYG{p}{(}\PYG{l+m+mi}{1}\PYG{p}{,} \PYG{l+m+mi}{101}\PYG{p}{)}\PYG{p}{]}
\end{sphinxVerbatim}

\end{sphinxuseclass}\end{sphinxVerbatimInput}

\end{sphinxuseclass}
\begin{sphinxuseclass}{cell}\begin{sphinxVerbatimInput}

\begin{sphinxuseclass}{cell_input}
\begin{sphinxVerbatim}[commandchars=\\\{\}]
\PYG{n}{l4\PYGZus{}quiz1} \PYG{o}{=} \PYG{p}{[}\PYG{p}{]}
\PYG{k}{for} \PYG{n}{x} \PYG{o+ow}{in} \PYG{n+nb}{range}\PYG{p}{(}\PYG{l+m+mi}{1}\PYG{p}{,} \PYG{l+m+mi}{101}\PYG{p}{)}\PYG{p}{:}
    \PYG{n}{l4\PYGZus{}quiz1}\PYG{o}{.}\PYG{n}{append}\PYG{p}{(}\PYG{n}{x}\PYG{o}{*}\PYG{o}{*}\PYG{l+m+mi}{2}\PYG{p}{)}
\end{sphinxVerbatim}

\end{sphinxuseclass}\end{sphinxVerbatimInput}

\end{sphinxuseclass}
\begin{sphinxuseclass}{cell}\begin{sphinxVerbatimInput}

\begin{sphinxuseclass}{cell_input}
\begin{sphinxVerbatim}[commandchars=\\\{\}]
\PYG{n}{l4} \PYG{o}{==} \PYG{n}{l4\PYGZus{}quiz1}
\end{sphinxVerbatim}

\end{sphinxuseclass}\end{sphinxVerbatimInput}
\begin{sphinxVerbatimOutput}

\begin{sphinxuseclass}{cell_output}
\begin{sphinxVerbatim}[commandchars=\\\{\}]
True
\end{sphinxVerbatim}

\end{sphinxuseclass}\end{sphinxVerbatimOutput}

\end{sphinxuseclass}

\bigskip\hrule\bigskip


\sphinxAtStartPar
New question: Can we use ternary expressions (i.e., inline \sphinxcode{\sphinxupquote{if else}}) with a list comprehension?

\sphinxAtStartPar
I suggest writing a function to make this more obvious, but here is an example.
We can try squaring \sphinxcode{\sphinxupquote{x}} if \sphinxcode{\sphinxupquote{x}} is even, but cubing \sphinxcode{\sphinxupquote{x}} if \sphinxcode{\sphinxupquote{x}} is odd.
This case is an excellent application of “ternary” statements, which have the form “TrueValue if True else FalseValue”.

\begin{sphinxuseclass}{cell}\begin{sphinxVerbatimInput}

\begin{sphinxuseclass}{cell_input}
\begin{sphinxVerbatim}[commandchars=\\\{\}]
\PYG{n}{l4\PYGZus{}new\PYGZus{}question} \PYG{o}{=} \PYG{p}{[}\PYG{n}{x}\PYG{o}{*}\PYG{o}{*}\PYG{l+m+mi}{2} \PYG{k}{if} \PYG{p}{(}\PYG{n}{x}\PYG{o}{\PYGZpc{}}\PYG{k}{2} == 0) else x**3 for x in range(1, 101)]
\end{sphinxVerbatim}

\end{sphinxuseclass}\end{sphinxVerbatimInput}

\end{sphinxuseclass}
\begin{sphinxuseclass}{cell}\begin{sphinxVerbatimInput}

\begin{sphinxuseclass}{cell_input}
\begin{sphinxVerbatim}[commandchars=\\\{\}]
\PYG{n+nb}{print}\PYG{p}{(}\PYG{n}{l4\PYGZus{}new\PYGZus{}question}\PYG{p}{)}
\end{sphinxVerbatim}

\end{sphinxuseclass}\end{sphinxVerbatimInput}
\begin{sphinxVerbatimOutput}

\begin{sphinxuseclass}{cell_output}
\begin{sphinxVerbatim}[commandchars=\\\{\}]
[1, 4, 27, 16, 125, 36, 343, 64, 729, 100, 1331, 144, 2197, 196, 3375, 256, 4913, 324, 6859, 400, 9261, 484, 12167, 576, 15625, 676, 19683, 784, 24389, 900, 29791, 1024, 35937, 1156, 42875, 1296, 50653, 1444, 59319, 1600, 68921, 1764, 79507, 1936, 91125, 2116, 103823, 2304, 117649, 2500, 132651, 2704, 148877, 2916, 166375, 3136, 185193, 3364, 205379, 3600, 226981, 3844, 250047, 4096, 274625, 4356, 300763, 4624, 328509, 4900, 357911, 5184, 389017, 5476, 421875, 5776, 456533, 6084, 493039, 6400, 531441, 6724, 571787, 7056, 614125, 7396, 658503, 7744, 704969, 8100, 753571, 8464, 804357, 8836, 857375, 9216, 912673, 9604, 970299, 10000]
\end{sphinxVerbatim}

\end{sphinxuseclass}\end{sphinxVerbatimOutput}

\end{sphinxuseclass}

\bigskip\hrule\bigskip



\subsubsection{Create a list \sphinxstyleliteralintitle{\sphinxupquote{l5}} that contains the squares of \sphinxstyleemphasis{\sphinxstylestrong{odd}} integers from 1 to 100.}
\label{\detokenize{mckinney_03_practice_04:create-a-list-l5-that-contains-the-squares-of-odd-integers-from-1-to-100}}
\begin{sphinxuseclass}{cell}\begin{sphinxVerbatimInput}

\begin{sphinxuseclass}{cell_input}
\begin{sphinxVerbatim}[commandchars=\\\{\}]
\PYG{n}{l5} \PYG{o}{=} \PYG{p}{[}\PYG{n}{x}\PYG{o}{*}\PYG{o}{*}\PYG{l+m+mi}{2} \PYG{k}{for} \PYG{n}{x} \PYG{o+ow}{in} \PYG{n+nb}{range}\PYG{p}{(}\PYG{l+m+mi}{1}\PYG{p}{,} \PYG{l+m+mi}{101}\PYG{p}{,} \PYG{l+m+mi}{2}\PYG{p}{)}\PYG{p}{]}
\end{sphinxVerbatim}

\end{sphinxuseclass}\end{sphinxVerbatimInput}

\end{sphinxuseclass}
\begin{sphinxuseclass}{cell}\begin{sphinxVerbatimInput}

\begin{sphinxuseclass}{cell_input}
\begin{sphinxVerbatim}[commandchars=\\\{\}]
\PYG{n+nb}{print}\PYG{p}{(}\PYG{n}{l5}\PYG{p}{)}
\end{sphinxVerbatim}

\end{sphinxuseclass}\end{sphinxVerbatimInput}
\begin{sphinxVerbatimOutput}

\begin{sphinxuseclass}{cell_output}
\begin{sphinxVerbatim}[commandchars=\\\{\}]
[1, 9, 25, 49, 81, 121, 169, 225, 289, 361, 441, 529, 625, 729, 841, 961, 1089, 1225, 1369, 1521, 1681, 1849, 2025, 2209, 2401, 2601, 2809, 3025, 3249, 3481, 3721, 3969, 4225, 4489, 4761, 5041, 5329, 5625, 5929, 6241, 6561, 6889, 7225, 7569, 7921, 8281, 8649, 9025, 9409, 9801]
\end{sphinxVerbatim}

\end{sphinxuseclass}\end{sphinxVerbatimOutput}

\end{sphinxuseclass}
\begin{sphinxuseclass}{cell}\begin{sphinxVerbatimInput}

\begin{sphinxuseclass}{cell_input}
\begin{sphinxVerbatim}[commandchars=\\\{\}]
\PYG{n}{l5\PYGZus{}alt} \PYG{o}{=} \PYG{p}{[}\PYG{n}{x}\PYG{o}{*}\PYG{o}{*}\PYG{l+m+mi}{2} \PYG{k}{for} \PYG{n}{x} \PYG{o+ow}{in} \PYG{n+nb}{range}\PYG{p}{(}\PYG{l+m+mi}{1}\PYG{p}{,} \PYG{l+m+mi}{101}\PYG{p}{)} \PYG{k}{if} \PYG{n}{x}\PYG{o}{\PYGZpc{}}\PYG{k}{2} != 0]
\end{sphinxVerbatim}

\end{sphinxuseclass}\end{sphinxVerbatimInput}

\end{sphinxuseclass}
\begin{sphinxuseclass}{cell}\begin{sphinxVerbatimInput}

\begin{sphinxuseclass}{cell_input}
\begin{sphinxVerbatim}[commandchars=\\\{\}]
\PYG{n}{l5} \PYG{o}{==} \PYG{n}{l5\PYGZus{}alt}\PYG{p}{[}\PYG{o}{\PYGZhy{}}\PYG{l+m+mi}{41}\PYG{p}{]}
\end{sphinxVerbatim}

\end{sphinxuseclass}\end{sphinxVerbatimInput}
\begin{sphinxVerbatimOutput}

\begin{sphinxuseclass}{cell_output}
\begin{sphinxVerbatim}[commandchars=\\\{\}]
False
\end{sphinxVerbatim}

\end{sphinxuseclass}\end{sphinxVerbatimOutput}

\end{sphinxuseclass}

\subsubsection{Use a lambda function to sort \sphinxstyleliteralintitle{\sphinxupquote{strings}} by the last letter.}
\label{\detokenize{mckinney_03_practice_04:use-a-lambda-function-to-sort-strings-by-the-last-letter}}
\begin{sphinxuseclass}{cell}\begin{sphinxVerbatimInput}

\begin{sphinxuseclass}{cell_input}
\begin{sphinxVerbatim}[commandchars=\\\{\}]
\PYG{n}{strings} \PYG{o}{=} \PYG{p}{[}\PYG{l+s+s1}{\PYGZsq{}}\PYG{l+s+s1}{card}\PYG{l+s+s1}{\PYGZsq{}}\PYG{p}{,} \PYG{l+s+s1}{\PYGZsq{}}\PYG{l+s+s1}{aaaa}\PYG{l+s+s1}{\PYGZsq{}}\PYG{p}{,} \PYG{l+s+s1}{\PYGZsq{}}\PYG{l+s+s1}{foo}\PYG{l+s+s1}{\PYGZsq{}}\PYG{p}{,} \PYG{l+s+s1}{\PYGZsq{}}\PYG{l+s+s1}{bar}\PYG{l+s+s1}{\PYGZsq{}}\PYG{p}{,} \PYG{l+s+s1}{\PYGZsq{}}\PYG{l+s+s1}{abab}\PYG{l+s+s1}{\PYGZsq{}}\PYG{p}{]}
\end{sphinxVerbatim}

\end{sphinxuseclass}\end{sphinxVerbatimInput}

\end{sphinxuseclass}
\begin{sphinxuseclass}{cell}\begin{sphinxVerbatimInput}

\begin{sphinxuseclass}{cell_input}
\begin{sphinxVerbatim}[commandchars=\\\{\}]
\PYG{n}{strings}\PYG{o}{.}\PYG{n}{sort}\PYG{p}{(}\PYG{n}{key}\PYG{o}{=}\PYG{k}{lambda} \PYG{n}{x}\PYG{p}{:} \PYG{n}{x}\PYG{p}{[}\PYG{o}{\PYGZhy{}}\PYG{l+m+mi}{1}\PYG{p}{]}\PYG{p}{)}
\PYG{n}{strings}
\end{sphinxVerbatim}

\end{sphinxuseclass}\end{sphinxVerbatimInput}
\begin{sphinxVerbatimOutput}

\begin{sphinxuseclass}{cell_output}
\begin{sphinxVerbatim}[commandchars=\\\{\}]
[\PYGZsq{}aaaa\PYGZsq{}, \PYGZsq{}abab\PYGZsq{}, \PYGZsq{}card\PYGZsq{}, \PYGZsq{}foo\PYGZsq{}, \PYGZsq{}bar\PYGZsq{}]
\end{sphinxVerbatim}

\end{sphinxuseclass}\end{sphinxVerbatimOutput}

\end{sphinxuseclass}

\subsubsection{Given an integer array \sphinxstyleliteralintitle{\sphinxupquote{nums}} and an integer \sphinxstyleliteralintitle{\sphinxupquote{k}}, return the \$k\textasciicircum{}\{th\}\$ largest element in the array.}
\label{\detokenize{mckinney_03_practice_04:given-an-integer-array-nums-and-an-integer-k-return-the-k-th-largest-element-in-the-array}}
\sphinxAtStartPar
Note that it is the \$k\textasciicircum{}\{th\}\$ largest element in the sorted order, not the \$k\textasciicircum{}\{th\}\$ distinct element.

\sphinxAtStartPar
Example 1:

\sphinxAtStartPar
Input: \sphinxcode{\sphinxupquote{nums = {[}3,2,1,5,6,4{]}}}, \sphinxcode{\sphinxupquote{k = 2}} \\
Output: \sphinxcode{\sphinxupquote{5}}

\sphinxAtStartPar
Example 2:

\sphinxAtStartPar
Input: \sphinxcode{\sphinxupquote{nums = {[}3,2,3,1,2,4,5,5,6{]}}}, \sphinxcode{\sphinxupquote{k = 4}} \\
Output: \sphinxcode{\sphinxupquote{4}}

\sphinxAtStartPar
I saw this question on \sphinxhref{https://leetcode.com/problems/kth-largest-element-in-an-array/}{LeetCode}.

\begin{sphinxuseclass}{cell}\begin{sphinxVerbatimInput}

\begin{sphinxuseclass}{cell_input}
\begin{sphinxVerbatim}[commandchars=\\\{\}]
\PYG{k}{def} \PYG{n+nf}{nums}\PYG{p}{(}\PYG{n}{x}\PYG{p}{,} \PYG{n}{k}\PYG{p}{)}\PYG{p}{:}
    \PYG{n}{x\PYGZus{}copy} \PYG{o}{=} \PYG{n}{x}\PYG{o}{.}\PYG{n}{copy}\PYG{p}{(}\PYG{p}{)}
    \PYG{n}{x\PYGZus{}copy}\PYG{o}{.}\PYG{n}{sort}\PYG{p}{(}\PYG{p}{)}
    \PYG{k}{return} \PYG{n}{x\PYGZus{}copy}\PYG{p}{[}\PYG{o}{\PYGZhy{}}\PYG{n}{k}\PYG{p}{]}
\end{sphinxVerbatim}

\end{sphinxuseclass}\end{sphinxVerbatimInput}

\end{sphinxuseclass}
\begin{sphinxuseclass}{cell}\begin{sphinxVerbatimInput}

\begin{sphinxuseclass}{cell_input}
\begin{sphinxVerbatim}[commandchars=\\\{\}]
\PYG{n}{nums}\PYG{p}{(}\PYG{n}{x}\PYG{o}{=}\PYG{p}{[}\PYG{l+m+mi}{3}\PYG{p}{,}\PYG{l+m+mi}{2}\PYG{p}{,}\PYG{l+m+mi}{1}\PYG{p}{,}\PYG{l+m+mi}{5}\PYG{p}{,}\PYG{l+m+mi}{6}\PYG{p}{,}\PYG{l+m+mi}{4}\PYG{p}{]}\PYG{p}{,} \PYG{n}{k}\PYG{o}{=}\PYG{l+m+mi}{2}\PYG{p}{)}
\end{sphinxVerbatim}

\end{sphinxuseclass}\end{sphinxVerbatimInput}
\begin{sphinxVerbatimOutput}

\begin{sphinxuseclass}{cell_output}
\begin{sphinxVerbatim}[commandchars=\\\{\}]
5
\end{sphinxVerbatim}

\end{sphinxuseclass}\end{sphinxVerbatimOutput}

\end{sphinxuseclass}

\subsubsection{Given an integer array \sphinxstyleliteralintitle{\sphinxupquote{nums}} and an integer \sphinxstyleliteralintitle{\sphinxupquote{k}}, return the \sphinxstyleliteralintitle{\sphinxupquote{k}} most frequent elements.}
\label{\detokenize{mckinney_03_practice_04:given-an-integer-array-nums-and-an-integer-k-return-the-k-most-frequent-elements}}
\sphinxAtStartPar
You may return the answer in any order.

\sphinxAtStartPar
Example 1:

\sphinxAtStartPar
Input: \sphinxcode{\sphinxupquote{nums = {[}1,1,1,2,2,3{]}}}, \sphinxcode{\sphinxupquote{k = 2}} \\
Output: \sphinxcode{\sphinxupquote{{[}1,2{]}}}

\sphinxAtStartPar
Example 2:

\sphinxAtStartPar
Input: \sphinxcode{\sphinxupquote{nums = {[}1{]}}}, \sphinxcode{\sphinxupquote{k = 1}} \\
Output: \sphinxcode{\sphinxupquote{{[}1{]}}}

\sphinxAtStartPar
I saw this question on \sphinxhref{https://leetcode.com/problems/top-k-frequent-elements/}{LeetCode}.

\begin{sphinxuseclass}{cell}\begin{sphinxVerbatimInput}

\begin{sphinxuseclass}{cell_input}
\begin{sphinxVerbatim}[commandchars=\\\{\}]
\PYG{k}{def} \PYG{n+nf}{nums}\PYG{p}{(}\PYG{n}{nums}\PYG{p}{,} \PYG{n}{k}\PYG{p}{)}\PYG{p}{:}
    \PYG{n}{counts} \PYG{o}{=} \PYG{p}{\PYGZob{}}\PYG{p}{\PYGZcb{}}
    \PYG{k}{for} \PYG{n}{n} \PYG{o+ow}{in} \PYG{n}{nums}\PYG{p}{:}
        \PYG{k}{if} \PYG{n}{n} \PYG{o+ow}{in} \PYG{n}{counts}\PYG{p}{:}
            \PYG{n}{counts}\PYG{p}{[}\PYG{n}{n}\PYG{p}{]} \PYG{o}{+}\PYG{o}{=} \PYG{l+m+mi}{1}
        \PYG{k}{else}\PYG{p}{:}
            \PYG{n}{counts}\PYG{p}{[}\PYG{n}{n}\PYG{p}{]} \PYG{o}{=} \PYG{l+m+mi}{1}
    \PYG{k}{return} \PYG{p}{[}\PYG{n}{x}\PYG{p}{[}\PYG{l+m+mi}{0}\PYG{p}{]} \PYG{k}{for} \PYG{n}{x} \PYG{o+ow}{in} \PYG{n+nb}{sorted}\PYG{p}{(}\PYG{n}{counts}\PYG{o}{.}\PYG{n}{items}\PYG{p}{(}\PYG{p}{)}\PYG{p}{,} \PYG{n}{key}\PYG{o}{=}\PYG{k}{lambda} \PYG{n}{x}\PYG{p}{:} \PYG{n}{x}\PYG{p}{[}\PYG{l+m+mi}{1}\PYG{p}{]}\PYG{p}{,} \PYG{n}{reverse}\PYG{o}{=}\PYG{k+kc}{True}\PYG{p}{)}\PYG{p}{[}\PYG{p}{:}\PYG{n}{k}\PYG{p}{]}\PYG{p}{]}
\end{sphinxVerbatim}

\end{sphinxuseclass}\end{sphinxVerbatimInput}

\end{sphinxuseclass}
\begin{sphinxuseclass}{cell}\begin{sphinxVerbatimInput}

\begin{sphinxuseclass}{cell_input}
\begin{sphinxVerbatim}[commandchars=\\\{\}]
\PYG{n}{nums}\PYG{p}{(}\PYG{n}{nums}\PYG{o}{=}\PYG{p}{[}\PYG{l+m+mi}{1}\PYG{p}{,}\PYG{l+m+mi}{1}\PYG{p}{,}\PYG{l+m+mi}{1}\PYG{p}{,}\PYG{l+m+mi}{2}\PYG{p}{,}\PYG{l+m+mi}{2}\PYG{p}{,}\PYG{l+m+mi}{3}\PYG{p}{]}\PYG{p}{,} \PYG{n}{k}\PYG{o}{=}\PYG{l+m+mi}{2}\PYG{p}{)}
\end{sphinxVerbatim}

\end{sphinxuseclass}\end{sphinxVerbatimInput}
\begin{sphinxVerbatimOutput}

\begin{sphinxuseclass}{cell_output}
\begin{sphinxVerbatim}[commandchars=\\\{\}]
[1, 2]
\end{sphinxVerbatim}

\end{sphinxuseclass}\end{sphinxVerbatimOutput}

\end{sphinxuseclass}

\subsubsection{Test whether the given strings are palindromes.}
\label{\detokenize{mckinney_03_practice_04:test-whether-the-given-strings-are-palindromes}}
\sphinxAtStartPar
Input: \sphinxcode{\sphinxupquote{{[}"aba", "no"{]}}} \\
Output: \sphinxcode{\sphinxupquote{{[}True, False{]}}}

\begin{sphinxuseclass}{cell}\begin{sphinxVerbatimInput}

\begin{sphinxuseclass}{cell_input}
\begin{sphinxVerbatim}[commandchars=\\\{\}]
\PYG{k}{def} \PYG{n+nf}{is\PYGZus{}palindrome}\PYG{p}{(}\PYG{n}{x}\PYG{p}{)}\PYG{p}{:}
    \PYG{k}{return} \PYG{p}{[}\PYG{n+nb}{list}\PYG{p}{(}\PYG{n}{y}\PYG{p}{)} \PYG{o}{==} \PYG{n+nb}{list}\PYG{p}{(}\PYG{n}{y}\PYG{p}{)}\PYG{p}{[}\PYG{p}{:}\PYG{p}{:}\PYG{o}{\PYGZhy{}}\PYG{l+m+mi}{1}\PYG{p}{]} \PYG{k}{for} \PYG{n}{y} \PYG{o+ow}{in} \PYG{n}{x}\PYG{p}{]}
\end{sphinxVerbatim}

\end{sphinxuseclass}\end{sphinxVerbatimInput}

\end{sphinxuseclass}
\begin{sphinxuseclass}{cell}\begin{sphinxVerbatimInput}

\begin{sphinxuseclass}{cell_input}
\begin{sphinxVerbatim}[commandchars=\\\{\}]
\PYG{n}{is\PYGZus{}palindrome}\PYG{p}{(}\PYG{p}{[}\PYG{l+s+s2}{\PYGZdq{}}\PYG{l+s+s2}{aba}\PYG{l+s+s2}{\PYGZdq{}}\PYG{p}{,} \PYG{l+s+s2}{\PYGZdq{}}\PYG{l+s+s2}{no}\PYG{l+s+s2}{\PYGZdq{}}\PYG{p}{]}\PYG{p}{)}
\end{sphinxVerbatim}

\end{sphinxuseclass}\end{sphinxVerbatimInput}
\begin{sphinxVerbatimOutput}

\begin{sphinxuseclass}{cell_output}
\begin{sphinxVerbatim}[commandchars=\\\{\}]
[True, False]
\end{sphinxVerbatim}

\end{sphinxuseclass}\end{sphinxVerbatimOutput}

\end{sphinxuseclass}

\subsubsection{Write a function \sphinxstyleliteralintitle{\sphinxupquote{returns()}} that accepts lists of prices and dividends and returns a list of returns.}
\label{\detokenize{mckinney_03_practice_04:write-a-function-returns-that-accepts-lists-of-prices-and-dividends-and-returns-a-list-of-returns}}
\begin{sphinxuseclass}{cell}\begin{sphinxVerbatimInput}

\begin{sphinxuseclass}{cell_input}
\begin{sphinxVerbatim}[commandchars=\\\{\}]
\PYG{n}{prices} \PYG{o}{=} \PYG{p}{[}\PYG{l+m+mi}{100}\PYG{p}{,} \PYG{l+m+mi}{150}\PYG{p}{,} \PYG{l+m+mi}{100}\PYG{p}{,} \PYG{l+m+mi}{50}\PYG{p}{,} \PYG{l+m+mi}{100}\PYG{p}{,} \PYG{l+m+mi}{150}\PYG{p}{,} \PYG{l+m+mi}{100}\PYG{p}{,} \PYG{l+m+mi}{150}\PYG{p}{]}
\PYG{n}{dividends} \PYG{o}{=} \PYG{p}{[}\PYG{l+m+mi}{1}\PYG{p}{,} \PYG{l+m+mi}{1}\PYG{p}{,} \PYG{l+m+mi}{1}\PYG{p}{,} \PYG{l+m+mi}{1}\PYG{p}{,} \PYG{l+m+mi}{2}\PYG{p}{,} \PYG{l+m+mi}{2}\PYG{p}{,} \PYG{l+m+mi}{2}\PYG{p}{,} \PYG{l+m+mi}{2}\PYG{p}{]}
\end{sphinxVerbatim}

\end{sphinxuseclass}\end{sphinxVerbatimInput}

\end{sphinxuseclass}
\begin{sphinxuseclass}{cell}\begin{sphinxVerbatimInput}

\begin{sphinxuseclass}{cell_input}
\begin{sphinxVerbatim}[commandchars=\\\{\}]
\PYG{k}{def} \PYG{n+nf}{returns}\PYG{p}{(}\PYG{n}{p}\PYG{p}{,} \PYG{n}{d}\PYG{p}{)}\PYG{p}{:}
    \PYG{n}{rts} \PYG{o}{=} \PYG{p}{[}\PYG{k+kc}{None}\PYG{p}{]}
    \PYG{k}{for} \PYG{n}{t} \PYG{o+ow}{in} \PYG{n+nb}{range}\PYG{p}{(}\PYG{l+m+mi}{1}\PYG{p}{,} \PYG{n+nb}{len}\PYG{p}{(}\PYG{n}{p}\PYG{p}{)}\PYG{p}{)}\PYG{p}{:} \PYG{c+c1}{\PYGZsh{} start at t=1 b/c need 2 prices to calculate a return}
        \PYG{n}{pt} \PYG{o}{=} \PYG{n}{p}\PYG{p}{[}\PYG{n}{t}\PYG{p}{]}
        \PYG{n}{ptm1} \PYG{o}{=} \PYG{n}{p}\PYG{p}{[}\PYG{n}{t}\PYG{o}{\PYGZhy{}}\PYG{l+m+mi}{1}\PYG{p}{]}
        \PYG{n}{dt} \PYG{o}{=} \PYG{n}{d}\PYG{p}{[}\PYG{n}{t}\PYG{p}{]}
        \PYG{n}{rt} \PYG{o}{=} \PYG{p}{(}\PYG{n}{pt} \PYG{o}{\PYGZhy{}} \PYG{n}{ptm1} \PYG{o}{+} \PYG{n}{dt}\PYG{p}{)} \PYG{o}{/} \PYG{n}{ptm1}
        \PYG{n}{rts}\PYG{o}{.}\PYG{n}{append}\PYG{p}{(}\PYG{n}{rt}\PYG{p}{)}

    \PYG{k}{return} \PYG{n}{rts}
\end{sphinxVerbatim}

\end{sphinxuseclass}\end{sphinxVerbatimInput}

\end{sphinxuseclass}
\begin{sphinxuseclass}{cell}\begin{sphinxVerbatimInput}

\begin{sphinxuseclass}{cell_input}
\begin{sphinxVerbatim}[commandchars=\\\{\}]
\PYG{n}{returns}\PYG{p}{(}\PYG{n}{p}\PYG{o}{=}\PYG{n}{prices}\PYG{p}{,} \PYG{n}{d}\PYG{o}{=}\PYG{n}{dividends}\PYG{p}{)}
\end{sphinxVerbatim}

\end{sphinxuseclass}\end{sphinxVerbatimInput}
\begin{sphinxVerbatimOutput}

\begin{sphinxuseclass}{cell_output}
\begin{sphinxVerbatim}[commandchars=\\\{\}]
[None, 0.51, \PYGZhy{}0.32666666666666666, \PYGZhy{}0.49, 1.04, 0.52, \PYGZhy{}0.32, 0.52]
\end{sphinxVerbatim}

\end{sphinxuseclass}\end{sphinxVerbatimOutput}

\end{sphinxuseclass}

\subsubsection{Rewrite the function \sphinxstyleliteralintitle{\sphinxupquote{returns()}} so it returns lists of returns, capital gains yields, and dividend yields.}
\label{\detokenize{mckinney_03_practice_04:rewrite-the-function-returns-so-it-returns-lists-of-returns-capital-gains-yields-and-dividend-yields}}
\begin{sphinxuseclass}{cell}\begin{sphinxVerbatimInput}

\begin{sphinxuseclass}{cell_input}
\begin{sphinxVerbatim}[commandchars=\\\{\}]
\PYG{k}{def} \PYG{n+nf}{returns}\PYG{p}{(}\PYG{n}{p}\PYG{p}{,} \PYG{n}{d}\PYG{p}{)}\PYG{p}{:}
    \PYG{n}{rts}\PYG{p}{,} \PYG{n}{cgs}\PYG{p}{,} \PYG{n}{dys} \PYG{o}{=} \PYG{p}{[}\PYG{k+kc}{None}\PYG{p}{]}\PYG{p}{,} \PYG{p}{[}\PYG{k+kc}{None}\PYG{p}{]}\PYG{p}{,} \PYG{p}{[}\PYG{k+kc}{None}\PYG{p}{]}
    \PYG{k}{for} \PYG{n}{t} \PYG{o+ow}{in} \PYG{n+nb}{range}\PYG{p}{(}\PYG{l+m+mi}{1}\PYG{p}{,} \PYG{n+nb}{len}\PYG{p}{(}\PYG{n}{p}\PYG{p}{)}\PYG{p}{)}\PYG{p}{:} \PYG{c+c1}{\PYGZsh{} start at t=1 b/c need 2 prices to calculate a return}
        \PYG{n}{pt} \PYG{o}{=} \PYG{n}{p}\PYG{p}{[}\PYG{n}{t}\PYG{p}{]}
        \PYG{n}{ptm1} \PYG{o}{=} \PYG{n}{p}\PYG{p}{[}\PYG{n}{t}\PYG{o}{\PYGZhy{}}\PYG{l+m+mi}{1}\PYG{p}{]}
        \PYG{n}{dt} \PYG{o}{=} \PYG{n}{d}\PYG{p}{[}\PYG{n}{t}\PYG{p}{]}
        \PYG{n}{dy} \PYG{o}{=} \PYG{n}{dt} \PYG{o}{/} \PYG{n}{ptm1}
        \PYG{n}{cg} \PYG{o}{=} \PYG{p}{(}\PYG{n}{pt} \PYG{o}{\PYGZhy{}} \PYG{n}{ptm1}\PYG{p}{)} \PYG{o}{/} \PYG{n}{ptm1}
        \PYG{n}{rt} \PYG{o}{=} \PYG{n}{dy} \PYG{o}{+} \PYG{n}{cg}
        \PYG{n}{rts}\PYG{o}{.}\PYG{n}{append}\PYG{p}{(}\PYG{n}{rt}\PYG{p}{)}
        \PYG{n}{dys}\PYG{o}{.}\PYG{n}{append}\PYG{p}{(}\PYG{n}{dy}\PYG{p}{)}
        \PYG{n}{cgs}\PYG{o}{.}\PYG{n}{append}\PYG{p}{(}\PYG{n}{cg}\PYG{p}{)}

    \PYG{k}{return} \PYG{p}{\PYGZob{}}\PYG{l+s+s1}{\PYGZsq{}}\PYG{l+s+s1}{rt}\PYG{l+s+s1}{\PYGZsq{}}\PYG{p}{:}\PYG{n}{rts}\PYG{p}{,} \PYG{l+s+s1}{\PYGZsq{}}\PYG{l+s+s1}{dy}\PYG{l+s+s1}{\PYGZsq{}}\PYG{p}{:}\PYG{n}{dys}\PYG{p}{,} \PYG{l+s+s1}{\PYGZsq{}}\PYG{l+s+s1}{cg}\PYG{l+s+s1}{\PYGZsq{}}\PYG{p}{:}\PYG{n}{cgs}\PYG{p}{\PYGZcb{}}
\end{sphinxVerbatim}

\end{sphinxuseclass}\end{sphinxVerbatimInput}

\end{sphinxuseclass}
\begin{sphinxuseclass}{cell}\begin{sphinxVerbatimInput}

\begin{sphinxuseclass}{cell_input}
\begin{sphinxVerbatim}[commandchars=\\\{\}]
\PYG{n}{returns}\PYG{p}{(}\PYG{n}{p}\PYG{o}{=}\PYG{n}{prices}\PYG{p}{,} \PYG{n}{d}\PYG{o}{=}\PYG{n}{dividends}\PYG{p}{)}
\end{sphinxVerbatim}

\end{sphinxuseclass}\end{sphinxVerbatimInput}
\begin{sphinxVerbatimOutput}

\begin{sphinxuseclass}{cell_output}
\begin{sphinxVerbatim}[commandchars=\\\{\}]
\PYGZob{}\PYGZsq{}rt\PYGZsq{}: [None, 0.51, \PYGZhy{}0.32666666666666666, \PYGZhy{}0.49, 1.04, 0.52, \PYGZhy{}0.32, 0.52],
 \PYGZsq{}dy\PYGZsq{}: [None,
  0.01,
  0.006666666666666667,
  0.01,
  0.04,
  0.02,
  0.013333333333333334,
  0.02],
 \PYGZsq{}cg\PYGZsq{}: [None,
  0.5,
  \PYGZhy{}0.3333333333333333,
  \PYGZhy{}0.5,
  1.0,
  0.5,
  \PYGZhy{}0.3333333333333333,
  0.5]\PYGZcb{}
\end{sphinxVerbatim}

\end{sphinxuseclass}\end{sphinxVerbatimOutput}

\end{sphinxuseclass}

\subsubsection{Rescale and shift numbers so that they cover the range \sphinxstyleliteralintitle{\sphinxupquote{{[}0, 1{]}}}.}
\label{\detokenize{mckinney_03_practice_04:rescale-and-shift-numbers-so-that-they-cover-the-range-0-1}}
\sphinxAtStartPar
Input: \sphinxcode{\sphinxupquote{{[}18.5, 17.0, 18.0, 19.0, 18.0{]}}} \\
Output: \sphinxcode{\sphinxupquote{{[}0.75, 0.0, 0.5, 1.0, 0.5{]}}}

\begin{sphinxuseclass}{cell}\begin{sphinxVerbatimInput}

\begin{sphinxuseclass}{cell_input}
\begin{sphinxVerbatim}[commandchars=\\\{\}]
\PYG{n}{numbers} \PYG{o}{=} \PYG{p}{[}\PYG{l+m+mf}{18.5}\PYG{p}{,} \PYG{l+m+mf}{17.0}\PYG{p}{,} \PYG{l+m+mf}{18.0}\PYG{p}{,} \PYG{l+m+mf}{19.0}\PYG{p}{,} \PYG{l+m+mf}{18.0}\PYG{p}{]}
\end{sphinxVerbatim}

\end{sphinxuseclass}\end{sphinxVerbatimInput}

\end{sphinxuseclass}
\begin{sphinxuseclass}{cell}\begin{sphinxVerbatimInput}

\begin{sphinxuseclass}{cell_input}
\begin{sphinxVerbatim}[commandchars=\\\{\}]
\PYG{k}{def} \PYG{n+nf}{rescale}\PYG{p}{(}\PYG{n}{x}\PYG{p}{)}\PYG{p}{:}
    \PYG{n}{x\PYGZus{}min} \PYG{o}{=} \PYG{n+nb}{min}\PYG{p}{(}\PYG{n}{x}\PYG{p}{)}
    \PYG{n}{x\PYGZus{}max} \PYG{o}{=} \PYG{n+nb}{max}\PYG{p}{(}\PYG{n}{x}\PYG{p}{)}
    \PYG{k}{return} \PYG{p}{[}\PYG{p}{(}\PYG{n}{x} \PYG{o}{\PYGZhy{}} \PYG{n}{x\PYGZus{}min}\PYG{p}{)} \PYG{o}{/} \PYG{p}{(}\PYG{n}{x\PYGZus{}max} \PYG{o}{\PYGZhy{}} \PYG{n}{x\PYGZus{}min}\PYG{p}{)} \PYG{k}{for} \PYG{n}{x} \PYG{o+ow}{in} \PYG{n}{x}\PYG{p}{]}
\end{sphinxVerbatim}

\end{sphinxuseclass}\end{sphinxVerbatimInput}

\end{sphinxuseclass}
\begin{sphinxuseclass}{cell}\begin{sphinxVerbatimInput}

\begin{sphinxuseclass}{cell_input}
\begin{sphinxVerbatim}[commandchars=\\\{\}]
\PYG{n}{rescale}\PYG{p}{(}\PYG{n}{numbers}\PYG{p}{)}
\end{sphinxVerbatim}

\end{sphinxuseclass}\end{sphinxVerbatimInput}
\begin{sphinxVerbatimOutput}

\begin{sphinxuseclass}{cell_output}
\begin{sphinxVerbatim}[commandchars=\\\{\}]
[0.75, 0.0, 0.5, 1.0, 0.5]
\end{sphinxVerbatim}

\end{sphinxuseclass}\end{sphinxVerbatimOutput}

\end{sphinxuseclass}
\sphinxstepscope


\section{McKinney Chapter 3 \sphinxhyphen{} Practice (Section 2, Wednesday 2:45 PM)}
\label{\detokenize{mckinney_03_practice_02:mckinney-chapter-3-practice-section-2-wednesday-2-45-pm}}\label{\detokenize{mckinney_03_practice_02::doc}}

\subsection{Practice}
\label{\detokenize{mckinney_03_practice_02:practice}}

\subsubsection{Swap the values assigned to \sphinxstyleliteralintitle{\sphinxupquote{a}} and \sphinxstyleliteralintitle{\sphinxupquote{b}} using a third variable \sphinxstyleliteralintitle{\sphinxupquote{c}}.}
\label{\detokenize{mckinney_03_practice_02:swap-the-values-assigned-to-a-and-b-using-a-third-variable-c}}
\begin{sphinxuseclass}{cell}\begin{sphinxVerbatimInput}

\begin{sphinxuseclass}{cell_input}
\begin{sphinxVerbatim}[commandchars=\\\{\}]
\PYG{n}{a} \PYG{o}{=} \PYG{l+m+mi}{1}
\end{sphinxVerbatim}

\end{sphinxuseclass}\end{sphinxVerbatimInput}

\end{sphinxuseclass}
\begin{sphinxuseclass}{cell}\begin{sphinxVerbatimInput}

\begin{sphinxuseclass}{cell_input}
\begin{sphinxVerbatim}[commandchars=\\\{\}]
\PYG{n}{b} \PYG{o}{=} \PYG{l+m+mi}{2}
\end{sphinxVerbatim}

\end{sphinxuseclass}\end{sphinxVerbatimInput}

\end{sphinxuseclass}
\begin{sphinxuseclass}{cell}\begin{sphinxVerbatimInput}

\begin{sphinxuseclass}{cell_input}
\begin{sphinxVerbatim}[commandchars=\\\{\}]
\PYG{n+nb}{print}\PYG{p}{(}\PYG{l+s+sa}{f}\PYG{l+s+s1}{\PYGZsq{}}\PYG{l+s+s1}{a is }\PYG{l+s+si}{\PYGZob{}}\PYG{n}{a}\PYG{l+s+si}{\PYGZcb{}}\PYG{l+s+s1}{ and b is }\PYG{l+s+si}{\PYGZob{}}\PYG{n}{b}\PYG{l+s+si}{\PYGZcb{}}\PYG{l+s+s1}{\PYGZsq{}}\PYG{p}{)}
\end{sphinxVerbatim}

\end{sphinxuseclass}\end{sphinxVerbatimInput}
\begin{sphinxVerbatimOutput}

\begin{sphinxuseclass}{cell_output}
\begin{sphinxVerbatim}[commandchars=\\\{\}]
a is 1 and b is 2
\end{sphinxVerbatim}

\end{sphinxuseclass}\end{sphinxVerbatimOutput}

\end{sphinxuseclass}
\begin{sphinxuseclass}{cell}\begin{sphinxVerbatimInput}

\begin{sphinxuseclass}{cell_input}
\begin{sphinxVerbatim}[commandchars=\\\{\}]
\PYG{n}{c} \PYG{o}{=} \PYG{n}{a}
\end{sphinxVerbatim}

\end{sphinxuseclass}\end{sphinxVerbatimInput}

\end{sphinxuseclass}
\begin{sphinxuseclass}{cell}\begin{sphinxVerbatimInput}

\begin{sphinxuseclass}{cell_input}
\begin{sphinxVerbatim}[commandchars=\\\{\}]
\PYG{n}{c} \PYG{o}{==} \PYG{n}{a}
\end{sphinxVerbatim}

\end{sphinxuseclass}\end{sphinxVerbatimInput}
\begin{sphinxVerbatimOutput}

\begin{sphinxuseclass}{cell_output}
\begin{sphinxVerbatim}[commandchars=\\\{\}]
True
\end{sphinxVerbatim}

\end{sphinxuseclass}\end{sphinxVerbatimOutput}

\end{sphinxuseclass}
\begin{sphinxuseclass}{cell}\begin{sphinxVerbatimInput}

\begin{sphinxuseclass}{cell_input}
\begin{sphinxVerbatim}[commandchars=\\\{\}]
\PYG{n}{c} \PYG{o+ow}{is} \PYG{n}{a}
\end{sphinxVerbatim}

\end{sphinxuseclass}\end{sphinxVerbatimInput}
\begin{sphinxVerbatimOutput}

\begin{sphinxuseclass}{cell_output}
\begin{sphinxVerbatim}[commandchars=\\\{\}]
True
\end{sphinxVerbatim}

\end{sphinxuseclass}\end{sphinxVerbatimOutput}

\end{sphinxuseclass}
\begin{sphinxuseclass}{cell}\begin{sphinxVerbatimInput}

\begin{sphinxuseclass}{cell_input}
\begin{sphinxVerbatim}[commandchars=\\\{\}]
\PYG{n}{a} \PYG{o}{=} \PYG{n}{b}
\end{sphinxVerbatim}

\end{sphinxuseclass}\end{sphinxVerbatimInput}

\end{sphinxuseclass}
\begin{sphinxuseclass}{cell}\begin{sphinxVerbatimInput}

\begin{sphinxuseclass}{cell_input}
\begin{sphinxVerbatim}[commandchars=\\\{\}]
\PYG{n}{b} \PYG{o}{=} \PYG{n}{c}
\end{sphinxVerbatim}

\end{sphinxuseclass}\end{sphinxVerbatimInput}

\end{sphinxuseclass}
\begin{sphinxuseclass}{cell}\begin{sphinxVerbatimInput}

\begin{sphinxuseclass}{cell_input}
\begin{sphinxVerbatim}[commandchars=\\\{\}]
\PYG{n+nb}{print}\PYG{p}{(}\PYG{l+s+sa}{f}\PYG{l+s+s1}{\PYGZsq{}}\PYG{l+s+s1}{a is }\PYG{l+s+si}{\PYGZob{}}\PYG{n}{a}\PYG{l+s+si}{\PYGZcb{}}\PYG{l+s+s1}{ and b is }\PYG{l+s+si}{\PYGZob{}}\PYG{n}{b}\PYG{l+s+si}{\PYGZcb{}}\PYG{l+s+s1}{\PYGZsq{}}\PYG{p}{)}
\end{sphinxVerbatim}

\end{sphinxuseclass}\end{sphinxVerbatimInput}
\begin{sphinxVerbatimOutput}

\begin{sphinxuseclass}{cell_output}
\begin{sphinxVerbatim}[commandchars=\\\{\}]
a is 2 and b is 1
\end{sphinxVerbatim}

\end{sphinxuseclass}\end{sphinxVerbatimOutput}

\end{sphinxuseclass}
\begin{sphinxuseclass}{cell}\begin{sphinxVerbatimInput}

\begin{sphinxuseclass}{cell_input}
\begin{sphinxVerbatim}[commandchars=\\\{\}]
\PYG{o}{\PYGZpc{}}\PYG{k}{who}
\end{sphinxVerbatim}

\end{sphinxuseclass}\end{sphinxVerbatimInput}
\begin{sphinxVerbatimOutput}

\begin{sphinxuseclass}{cell_output}
\begin{sphinxVerbatim}[commandchars=\\\{\}]
a	 b	 c	 
\end{sphinxVerbatim}

\end{sphinxuseclass}\end{sphinxVerbatimOutput}

\end{sphinxuseclass}
\begin{sphinxuseclass}{cell}\begin{sphinxVerbatimInput}

\begin{sphinxuseclass}{cell_input}
\begin{sphinxVerbatim}[commandchars=\\\{\}]
\PYG{k}{del} \PYG{n}{c}
\end{sphinxVerbatim}

\end{sphinxuseclass}\end{sphinxVerbatimInput}

\end{sphinxuseclass}
\begin{sphinxuseclass}{cell}\begin{sphinxVerbatimInput}

\begin{sphinxuseclass}{cell_input}
\begin{sphinxVerbatim}[commandchars=\\\{\}]
\PYG{o}{\PYGZpc{}}\PYG{k}{who}
\end{sphinxVerbatim}

\end{sphinxuseclass}\end{sphinxVerbatimInput}
\begin{sphinxVerbatimOutput}

\begin{sphinxuseclass}{cell_output}
\begin{sphinxVerbatim}[commandchars=\\\{\}]
a	 b	 
\end{sphinxVerbatim}

\end{sphinxuseclass}\end{sphinxVerbatimOutput}

\end{sphinxuseclass}

\subsubsection{Swap the values assigned to \sphinxstyleliteralintitle{\sphinxupquote{a}} and \sphinxstyleliteralintitle{\sphinxupquote{b}} \sphinxstyleemphasis{\sphinxstylestrong{without}} using a third variable \sphinxstyleliteralintitle{\sphinxupquote{c}}.}
\label{\detokenize{mckinney_03_practice_02:swap-the-values-assigned-to-a-and-b-without-using-a-third-variable-c}}
\begin{sphinxuseclass}{cell}\begin{sphinxVerbatimInput}

\begin{sphinxuseclass}{cell_input}
\begin{sphinxVerbatim}[commandchars=\\\{\}]
\PYG{n}{a} \PYG{o}{=} \PYG{l+m+mi}{1}
\end{sphinxVerbatim}

\end{sphinxuseclass}\end{sphinxVerbatimInput}

\end{sphinxuseclass}
\begin{sphinxuseclass}{cell}\begin{sphinxVerbatimInput}

\begin{sphinxuseclass}{cell_input}
\begin{sphinxVerbatim}[commandchars=\\\{\}]
\PYG{n}{b} \PYG{o}{=} \PYG{l+m+mi}{2}
\end{sphinxVerbatim}

\end{sphinxuseclass}\end{sphinxVerbatimInput}

\end{sphinxuseclass}
\begin{sphinxuseclass}{cell}\begin{sphinxVerbatimInput}

\begin{sphinxuseclass}{cell_input}
\begin{sphinxVerbatim}[commandchars=\\\{\}]
\PYG{n+nb}{print}\PYG{p}{(}\PYG{l+s+sa}{f}\PYG{l+s+s1}{\PYGZsq{}}\PYG{l+s+s1}{a is }\PYG{l+s+si}{\PYGZob{}}\PYG{n}{a}\PYG{l+s+si}{\PYGZcb{}}\PYG{l+s+s1}{ and b is }\PYG{l+s+si}{\PYGZob{}}\PYG{n}{b}\PYG{l+s+si}{\PYGZcb{}}\PYG{l+s+s1}{\PYGZsq{}}\PYG{p}{)}
\end{sphinxVerbatim}

\end{sphinxuseclass}\end{sphinxVerbatimInput}
\begin{sphinxVerbatimOutput}

\begin{sphinxuseclass}{cell_output}
\begin{sphinxVerbatim}[commandchars=\\\{\}]
a is 1 and b is 2
\end{sphinxVerbatim}

\end{sphinxuseclass}\end{sphinxVerbatimOutput}

\end{sphinxuseclass}
\begin{sphinxuseclass}{cell}\begin{sphinxVerbatimInput}

\begin{sphinxuseclass}{cell_input}
\begin{sphinxVerbatim}[commandchars=\\\{\}]
\PYG{n+nb}{type}\PYG{p}{(}\PYG{p}{(}\PYG{n}{a}\PYG{p}{,} \PYG{n}{b}\PYG{p}{)}\PYG{p}{)}
\end{sphinxVerbatim}

\end{sphinxuseclass}\end{sphinxVerbatimInput}
\begin{sphinxVerbatimOutput}

\begin{sphinxuseclass}{cell_output}
\begin{sphinxVerbatim}[commandchars=\\\{\}]
tuple
\end{sphinxVerbatim}

\end{sphinxuseclass}\end{sphinxVerbatimOutput}

\end{sphinxuseclass}
\begin{sphinxuseclass}{cell}\begin{sphinxVerbatimInput}

\begin{sphinxuseclass}{cell_input}
\begin{sphinxVerbatim}[commandchars=\\\{\}]
\PYG{n}{b}\PYG{p}{,} \PYG{n}{a} \PYG{o}{=} \PYG{n}{a}\PYG{p}{,} \PYG{n}{b}
\end{sphinxVerbatim}

\end{sphinxuseclass}\end{sphinxVerbatimInput}

\end{sphinxuseclass}
\begin{sphinxuseclass}{cell}\begin{sphinxVerbatimInput}

\begin{sphinxuseclass}{cell_input}
\begin{sphinxVerbatim}[commandchars=\\\{\}]
\PYG{n+nb}{print}\PYG{p}{(}\PYG{l+s+sa}{f}\PYG{l+s+s1}{\PYGZsq{}}\PYG{l+s+s1}{a is }\PYG{l+s+si}{\PYGZob{}}\PYG{n}{a}\PYG{l+s+si}{\PYGZcb{}}\PYG{l+s+s1}{ and b is }\PYG{l+s+si}{\PYGZob{}}\PYG{n}{b}\PYG{l+s+si}{\PYGZcb{}}\PYG{l+s+s1}{\PYGZsq{}}\PYG{p}{)}
\end{sphinxVerbatim}

\end{sphinxuseclass}\end{sphinxVerbatimInput}
\begin{sphinxVerbatimOutput}

\begin{sphinxuseclass}{cell_output}
\begin{sphinxVerbatim}[commandchars=\\\{\}]
a is 2 and b is 1
\end{sphinxVerbatim}

\end{sphinxuseclass}\end{sphinxVerbatimOutput}

\end{sphinxuseclass}

\bigskip\hrule\bigskip


\sphinxAtStartPar
Here is a trap:

\begin{sphinxuseclass}{cell}\begin{sphinxVerbatimInput}

\begin{sphinxuseclass}{cell_input}
\begin{sphinxVerbatim}[commandchars=\\\{\}]
\PYG{n}{d} \PYG{o}{=} \PYG{n}{a}\PYG{p}{,} \PYG{n}{b}
\end{sphinxVerbatim}

\end{sphinxuseclass}\end{sphinxVerbatimInput}

\end{sphinxuseclass}
\begin{sphinxuseclass}{cell}\begin{sphinxVerbatimInput}

\begin{sphinxuseclass}{cell_input}
\begin{sphinxVerbatim}[commandchars=\\\{\}]
\PYG{n+nb}{type}\PYG{p}{(}\PYG{n}{d}\PYG{p}{)}
\end{sphinxVerbatim}

\end{sphinxuseclass}\end{sphinxVerbatimInput}
\begin{sphinxVerbatimOutput}

\begin{sphinxuseclass}{cell_output}
\begin{sphinxVerbatim}[commandchars=\\\{\}]
tuple
\end{sphinxVerbatim}

\end{sphinxuseclass}\end{sphinxVerbatimOutput}

\end{sphinxuseclass}
\begin{sphinxuseclass}{cell}\begin{sphinxVerbatimInput}

\begin{sphinxuseclass}{cell_input}
\begin{sphinxVerbatim}[commandchars=\\\{\}]
\PYG{n}{e}\PYG{p}{,} \PYG{n}{f} \PYG{o}{=} \PYG{n}{d}
\end{sphinxVerbatim}

\end{sphinxuseclass}\end{sphinxVerbatimInput}

\end{sphinxuseclass}
\begin{sphinxuseclass}{cell}\begin{sphinxVerbatimInput}

\begin{sphinxuseclass}{cell_input}
\begin{sphinxVerbatim}[commandchars=\\\{\}]
\PYG{n+nb}{print}\PYG{p}{(}\PYG{l+s+sa}{f}\PYG{l+s+s1}{\PYGZsq{}}\PYG{l+s+s1}{e is }\PYG{l+s+si}{\PYGZob{}}\PYG{n}{e}\PYG{l+s+si}{\PYGZcb{}}\PYG{l+s+s1}{ and f is }\PYG{l+s+si}{\PYGZob{}}\PYG{n}{f}\PYG{l+s+si}{\PYGZcb{}}\PYG{l+s+s1}{\PYGZsq{}}\PYG{p}{)}
\end{sphinxVerbatim}

\end{sphinxuseclass}\end{sphinxVerbatimInput}
\begin{sphinxVerbatimOutput}

\begin{sphinxuseclass}{cell_output}
\begin{sphinxVerbatim}[commandchars=\\\{\}]
e is 2 and f is 1
\end{sphinxVerbatim}

\end{sphinxuseclass}\end{sphinxVerbatimOutput}

\end{sphinxuseclass}
\begin{sphinxuseclass}{cell}\begin{sphinxVerbatimInput}

\begin{sphinxuseclass}{cell_input}
\begin{sphinxVerbatim}[commandchars=\\\{\}]
\PYG{c+c1}{\PYGZsh{} ValueError: not enough values to unpack (expected 3, got 2)}
\PYG{c+c1}{\PYGZsh{} g, h, i = d}
\end{sphinxVerbatim}

\end{sphinxuseclass}\end{sphinxVerbatimInput}

\end{sphinxuseclass}

\bigskip\hrule\bigskip



\subsubsection{What is the output of the following code and why?}
\label{\detokenize{mckinney_03_practice_02:what-is-the-output-of-the-following-code-and-why}}
\begin{sphinxuseclass}{cell}\begin{sphinxVerbatimInput}

\begin{sphinxuseclass}{cell_input}
\begin{sphinxVerbatim}[commandchars=\\\{\}]
\PYG{l+m+mi}{1} \PYG{o}{==} \PYG{p}{(}\PYG{l+m+mi}{1}\PYG{p}{,} \PYG{l+m+mi}{1}\PYG{p}{,} \PYG{l+m+mi}{1}\PYG{p}{)}
\end{sphinxVerbatim}

\end{sphinxuseclass}\end{sphinxVerbatimInput}
\begin{sphinxVerbatimOutput}

\begin{sphinxuseclass}{cell_output}
\begin{sphinxVerbatim}[commandchars=\\\{\}]
False
\end{sphinxVerbatim}

\end{sphinxuseclass}\end{sphinxVerbatimOutput}

\end{sphinxuseclass}
\begin{sphinxuseclass}{cell}\begin{sphinxVerbatimInput}

\begin{sphinxuseclass}{cell_input}
\begin{sphinxVerbatim}[commandchars=\\\{\}]
\PYG{l+m+mi}{1}\PYG{p}{,} \PYG{l+m+mi}{1}\PYG{p}{,} \PYG{l+m+mi}{1} \PYG{o}{==} \PYG{p}{(}\PYG{l+m+mi}{1}\PYG{p}{,} \PYG{l+m+mi}{1}\PYG{p}{,} \PYG{l+m+mi}{1}\PYG{p}{)}
\end{sphinxVerbatim}

\end{sphinxuseclass}\end{sphinxVerbatimInput}
\begin{sphinxVerbatimOutput}

\begin{sphinxuseclass}{cell_output}
\begin{sphinxVerbatim}[commandchars=\\\{\}]
(1, 1, False)
\end{sphinxVerbatim}

\end{sphinxuseclass}\end{sphinxVerbatimOutput}

\end{sphinxuseclass}
\begin{sphinxuseclass}{cell}\begin{sphinxVerbatimInput}

\begin{sphinxuseclass}{cell_input}
\begin{sphinxVerbatim}[commandchars=\\\{\}]
\PYG{l+m+mi}{1}\PYG{p}{,} \PYG{l+m+mi}{1} \PYG{o}{==} \PYG{p}{(}\PYG{l+m+mi}{1}\PYG{p}{,} \PYG{l+m+mi}{1}\PYG{p}{,} \PYG{l+m+mi}{1}\PYG{p}{)}
\end{sphinxVerbatim}

\end{sphinxuseclass}\end{sphinxVerbatimInput}
\begin{sphinxVerbatimOutput}

\begin{sphinxuseclass}{cell_output}
\begin{sphinxVerbatim}[commandchars=\\\{\}]
(1, False)
\end{sphinxVerbatim}

\end{sphinxuseclass}\end{sphinxVerbatimOutput}

\end{sphinxuseclass}
\begin{sphinxuseclass}{cell}\begin{sphinxVerbatimInput}

\begin{sphinxuseclass}{cell_input}
\begin{sphinxVerbatim}[commandchars=\\\{\}]
\PYG{p}{(}\PYG{l+m+mi}{1}\PYG{p}{,} \PYG{l+m+mi}{1}\PYG{p}{,} \PYG{l+m+mi}{1}\PYG{p}{)} \PYG{o}{==} \PYG{p}{(}\PYG{l+m+mi}{1}\PYG{p}{,} \PYG{l+m+mi}{1}\PYG{p}{,} \PYG{l+m+mi}{1}\PYG{p}{)}
\end{sphinxVerbatim}

\end{sphinxuseclass}\end{sphinxVerbatimInput}
\begin{sphinxVerbatimOutput}

\begin{sphinxuseclass}{cell_output}
\begin{sphinxVerbatim}[commandchars=\\\{\}]
True
\end{sphinxVerbatim}

\end{sphinxuseclass}\end{sphinxVerbatimOutput}

\end{sphinxuseclass}
\begin{sphinxuseclass}{cell}\begin{sphinxVerbatimInput}

\begin{sphinxuseclass}{cell_input}
\begin{sphinxVerbatim}[commandchars=\\\{\}]
\PYG{p}{(}\PYG{l+m+mi}{1}\PYG{p}{,} \PYG{l+m+mi}{1}\PYG{p}{,} \PYG{l+m+mi}{1}\PYG{p}{)} \PYG{o}{==} \PYG{l+m+mi}{1}\PYG{p}{,} \PYG{l+m+mi}{1}\PYG{p}{,} \PYG{l+m+mi}{1}
\end{sphinxVerbatim}

\end{sphinxuseclass}\end{sphinxVerbatimInput}
\begin{sphinxVerbatimOutput}

\begin{sphinxuseclass}{cell_output}
\begin{sphinxVerbatim}[commandchars=\\\{\}]
(False, 1, 1)
\end{sphinxVerbatim}

\end{sphinxuseclass}\end{sphinxVerbatimOutput}

\end{sphinxuseclass}
\begin{sphinxuseclass}{cell}\begin{sphinxVerbatimInput}

\begin{sphinxuseclass}{cell_input}
\begin{sphinxVerbatim}[commandchars=\\\{\}]
\PYG{l+m+mi}{1}\PYG{p}{,} \PYG{l+m+mi}{1}\PYG{p}{,} \PYG{l+m+mi}{1} \PYG{o}{==} \PYG{l+m+mi}{1}\PYG{p}{,} \PYG{l+m+mi}{1}\PYG{p}{,} \PYG{l+m+mi}{1}
\end{sphinxVerbatim}

\end{sphinxuseclass}\end{sphinxVerbatimInput}
\begin{sphinxVerbatimOutput}

\begin{sphinxuseclass}{cell_output}
\begin{sphinxVerbatim}[commandchars=\\\{\}]
(1, 1, True, 1, 1)
\end{sphinxVerbatim}

\end{sphinxuseclass}\end{sphinxVerbatimOutput}

\end{sphinxuseclass}

\subsubsection{Create a list \sphinxstyleliteralintitle{\sphinxupquote{l1}} of integers from 1 to 100.}
\label{\detokenize{mckinney_03_practice_02:create-a-list-l1-of-integers-from-1-to-100}}
\begin{sphinxuseclass}{cell}\begin{sphinxVerbatimInput}

\begin{sphinxuseclass}{cell_input}
\begin{sphinxVerbatim}[commandchars=\\\{\}]
\PYG{n}{l1} \PYG{o}{=} \PYG{n+nb}{list}\PYG{p}{(}\PYG{n+nb}{range}\PYG{p}{(}\PYG{l+m+mi}{1}\PYG{p}{,} \PYG{l+m+mi}{101}\PYG{p}{)}\PYG{p}{)}
\PYG{n+nb}{print}\PYG{p}{(}\PYG{n}{l1}\PYG{p}{)}
\end{sphinxVerbatim}

\end{sphinxuseclass}\end{sphinxVerbatimInput}
\begin{sphinxVerbatimOutput}

\begin{sphinxuseclass}{cell_output}
\begin{sphinxVerbatim}[commandchars=\\\{\}]
[1, 2, 3, 4, 5, 6, 7, 8, 9, 10, 11, 12, 13, 14, 15, 16, 17, 18, 19, 20, 21, 22, 23, 24, 25, 26, 27, 28, 29, 30, 31, 32, 33, 34, 35, 36, 37, 38, 39, 40, 41, 42, 43, 44, 45, 46, 47, 48, 49, 50, 51, 52, 53, 54, 55, 56, 57, 58, 59, 60, 61, 62, 63, 64, 65, 66, 67, 68, 69, 70, 71, 72, 73, 74, 75, 76, 77, 78, 79, 80, 81, 82, 83, 84, 85, 86, 87, 88, 89, 90, 91, 92, 93, 94, 95, 96, 97, 98, 99, 100]
\end{sphinxVerbatim}

\end{sphinxuseclass}\end{sphinxVerbatimOutput}

\end{sphinxuseclass}
\begin{sphinxuseclass}{cell}\begin{sphinxVerbatimInput}

\begin{sphinxuseclass}{cell_input}
\begin{sphinxVerbatim}[commandchars=\\\{\}]
\PYG{n}{l1}\PYG{p}{[}\PYG{p}{:}\PYG{l+m+mi}{5}\PYG{p}{]} 
\end{sphinxVerbatim}

\end{sphinxuseclass}\end{sphinxVerbatimInput}
\begin{sphinxVerbatimOutput}

\begin{sphinxuseclass}{cell_output}
\begin{sphinxVerbatim}[commandchars=\\\{\}]
[1, 2, 3, 4, 5]
\end{sphinxVerbatim}

\end{sphinxuseclass}\end{sphinxVerbatimOutput}

\end{sphinxuseclass}
\begin{sphinxuseclass}{cell}\begin{sphinxVerbatimInput}

\begin{sphinxuseclass}{cell_input}
\begin{sphinxVerbatim}[commandchars=\\\{\}]
\PYG{n}{l1}\PYG{p}{[}\PYG{l+m+mi}{5}\PYG{p}{:}\PYG{l+m+mi}{10}\PYG{p}{]}
\end{sphinxVerbatim}

\end{sphinxuseclass}\end{sphinxVerbatimInput}
\begin{sphinxVerbatimOutput}

\begin{sphinxuseclass}{cell_output}
\begin{sphinxVerbatim}[commandchars=\\\{\}]
[6, 7, 8, 9, 10]
\end{sphinxVerbatim}

\end{sphinxuseclass}\end{sphinxVerbatimOutput}

\end{sphinxuseclass}

\subsubsection{Slice \sphinxstyleliteralintitle{\sphinxupquote{l1}} to create a list of integers from 60 to 50 (inclusive).}
\label{\detokenize{mckinney_03_practice_02:slice-l1-to-create-a-list-of-integers-from-60-to-50-inclusive}}
\sphinxAtStartPar
Name this list \sphinxcode{\sphinxupquote{l2}}.

\begin{sphinxuseclass}{cell}\begin{sphinxVerbatimInput}

\begin{sphinxuseclass}{cell_input}
\begin{sphinxVerbatim}[commandchars=\\\{\}]
\PYG{n}{l2} \PYG{o}{=} \PYG{n}{l1}\PYG{p}{[}\PYG{l+m+mi}{49}\PYG{p}{:}\PYG{l+m+mi}{60}\PYG{p}{]}
\PYG{n}{l2}\PYG{o}{.}\PYG{n}{sort}\PYG{p}{(}\PYG{n}{reverse}\PYG{o}{=}\PYG{k+kc}{True}\PYG{p}{)}
\PYG{n+nb}{print}\PYG{p}{(}\PYG{n}{l2}\PYG{p}{)}
\end{sphinxVerbatim}

\end{sphinxuseclass}\end{sphinxVerbatimInput}
\begin{sphinxVerbatimOutput}

\begin{sphinxuseclass}{cell_output}
\begin{sphinxVerbatim}[commandchars=\\\{\}]
[60, 59, 58, 57, 56, 55, 54, 53, 52, 51, 50]
\end{sphinxVerbatim}

\end{sphinxuseclass}\end{sphinxVerbatimOutput}

\end{sphinxuseclass}
\begin{sphinxuseclass}{cell}\begin{sphinxVerbatimInput}

\begin{sphinxuseclass}{cell_input}
\begin{sphinxVerbatim}[commandchars=\\\{\}]
\PYG{n}{l1}\PYG{p}{[}\PYG{l+m+mi}{59}\PYG{p}{:}\PYG{l+m+mi}{48}\PYG{p}{:}\PYG{o}{\PYGZhy{}}\PYG{l+m+mi}{1}\PYG{p}{]}
\end{sphinxVerbatim}

\end{sphinxuseclass}\end{sphinxVerbatimInput}
\begin{sphinxVerbatimOutput}

\begin{sphinxuseclass}{cell_output}
\begin{sphinxVerbatim}[commandchars=\\\{\}]
[60, 59, 58, 57, 56, 55, 54, 53, 52, 51, 50]
\end{sphinxVerbatim}

\end{sphinxuseclass}\end{sphinxVerbatimOutput}

\end{sphinxuseclass}
\begin{sphinxuseclass}{cell}\begin{sphinxVerbatimInput}

\begin{sphinxuseclass}{cell_input}
\begin{sphinxVerbatim}[commandchars=\\\{\}]
\PYG{n}{l1}\PYG{p}{[}\PYG{l+m+mi}{49}\PYG{p}{:}\PYG{l+m+mi}{60}\PYG{p}{]}\PYG{p}{[}\PYG{p}{:}\PYG{p}{:}\PYG{o}{\PYGZhy{}}\PYG{l+m+mi}{1}\PYG{p}{]}
\end{sphinxVerbatim}

\end{sphinxuseclass}\end{sphinxVerbatimInput}
\begin{sphinxVerbatimOutput}

\begin{sphinxuseclass}{cell_output}
\begin{sphinxVerbatim}[commandchars=\\\{\}]
[60, 59, 58, 57, 56, 55, 54, 53, 52, 51, 50]
\end{sphinxVerbatim}

\end{sphinxuseclass}\end{sphinxVerbatimOutput}

\end{sphinxuseclass}
\begin{sphinxuseclass}{cell}\begin{sphinxVerbatimInput}

\begin{sphinxuseclass}{cell_input}
\begin{sphinxVerbatim}[commandchars=\\\{\}]
\PYG{n}{l2} \PYG{o}{=} \PYG{n}{l1}\PYG{p}{[}\PYG{l+m+mi}{49}\PYG{p}{:}\PYG{l+m+mi}{60}\PYG{p}{]}
\PYG{n}{l2}\PYG{o}{.}\PYG{n}{reverse}\PYG{p}{(}\PYG{p}{)}
\PYG{n}{l2}
\end{sphinxVerbatim}

\end{sphinxuseclass}\end{sphinxVerbatimInput}
\begin{sphinxVerbatimOutput}

\begin{sphinxuseclass}{cell_output}
\begin{sphinxVerbatim}[commandchars=\\\{\}]
[60, 59, 58, 57, 56, 55, 54, 53, 52, 51, 50]
\end{sphinxVerbatim}

\end{sphinxuseclass}\end{sphinxVerbatimOutput}

\end{sphinxuseclass}

\subsubsection{Create a list \sphinxstyleliteralintitle{\sphinxupquote{l3}} of odd integers from 1 to 21.}
\label{\detokenize{mckinney_03_practice_02:create-a-list-l3-of-odd-integers-from-1-to-21}}
\begin{sphinxuseclass}{cell}\begin{sphinxVerbatimInput}

\begin{sphinxuseclass}{cell_input}
\begin{sphinxVerbatim}[commandchars=\\\{\}]
\PYG{n}{l3} \PYG{o}{=} \PYG{n+nb}{list}\PYG{p}{(}\PYG{n+nb}{range}\PYG{p}{(}\PYG{l+m+mi}{1}\PYG{p}{,} \PYG{l+m+mi}{22}\PYG{p}{,} \PYG{l+m+mi}{2}\PYG{p}{)}\PYG{p}{)}
\PYG{n}{l3}
\end{sphinxVerbatim}

\end{sphinxuseclass}\end{sphinxVerbatimInput}
\begin{sphinxVerbatimOutput}

\begin{sphinxuseclass}{cell_output}
\begin{sphinxVerbatim}[commandchars=\\\{\}]
[1, 3, 5, 7, 9, 11, 13, 15, 17, 19, 21]
\end{sphinxVerbatim}

\end{sphinxuseclass}\end{sphinxVerbatimOutput}

\end{sphinxuseclass}

\subsubsection{Create a list \sphinxstyleliteralintitle{\sphinxupquote{l4}} of the squares of integers from 1 to 100.}
\label{\detokenize{mckinney_03_practice_02:create-a-list-l4-of-the-squares-of-integers-from-1-to-100}}
\begin{sphinxuseclass}{cell}\begin{sphinxVerbatimInput}

\begin{sphinxuseclass}{cell_input}
\begin{sphinxVerbatim}[commandchars=\\\{\}]
\PYG{o}{\PYGZpc{}\PYGZpc{}time}it
\PYG{n}{l4} \PYG{o}{=} \PYG{p}{[}\PYG{p}{]}
\PYG{k}{for} \PYG{n}{i} \PYG{o+ow}{in} \PYG{n+nb}{range}\PYG{p}{(}\PYG{l+m+mi}{1}\PYG{p}{,} \PYG{l+m+mi}{101}\PYG{p}{)}\PYG{p}{:}
    \PYG{n}{l4}\PYG{o}{.}\PYG{n}{append}\PYG{p}{(}\PYG{n}{i}\PYG{o}{*}\PYG{o}{*}\PYG{l+m+mi}{2}\PYG{p}{)}
\end{sphinxVerbatim}

\end{sphinxuseclass}\end{sphinxVerbatimInput}
\begin{sphinxVerbatimOutput}

\begin{sphinxuseclass}{cell_output}
\begin{sphinxVerbatim}[commandchars=\\\{\}]
16 µs ± 91.2 ns per loop (mean ± std. dev. of 7 runs, 100000 loops each)
\end{sphinxVerbatim}

\end{sphinxuseclass}\end{sphinxVerbatimOutput}

\end{sphinxuseclass}
\sphinxAtStartPar
A list comprehension is easier to write and read and considered “more Pythonic”.
Here is the pseudo code for a list comprehension:

\begin{sphinxVerbatim}[commandchars=\\\{\}]
\PYG{p}{[}\PYG{n}{f}\PYG{p}{(}\PYG{n}{x}\PYG{p}{)} \PYG{k}{for} \PYG{n}{x} \PYG{o+ow}{in} \PYG{n}{some} \PYG{n+nb}{range} \PYG{k}{if} \PYG{n}{some} \PYG{n}{condition}\PYG{p}{]}
\end{sphinxVerbatim}

\begin{sphinxuseclass}{cell}\begin{sphinxVerbatimInput}

\begin{sphinxuseclass}{cell_input}
\begin{sphinxVerbatim}[commandchars=\\\{\}]
\PYG{o}{\PYGZpc{}\PYGZpc{}time}it
\PYG{n}{l4} \PYG{o}{=} \PYG{p}{[}\PYG{n}{x}\PYG{o}{*}\PYG{o}{*}\PYG{l+m+mi}{2} \PYG{k}{for} \PYG{n}{x} \PYG{o+ow}{in} \PYG{n+nb}{range}\PYG{p}{(}\PYG{l+m+mi}{1}\PYG{p}{,} \PYG{l+m+mi}{101}\PYG{p}{)}\PYG{p}{]}
\end{sphinxVerbatim}

\end{sphinxuseclass}\end{sphinxVerbatimInput}
\begin{sphinxVerbatimOutput}

\begin{sphinxuseclass}{cell_output}
\begin{sphinxVerbatim}[commandchars=\\\{\}]
15.1 µs ± 24.8 ns per loop (mean ± std. dev. of 7 runs, 100000 loops each)
\end{sphinxVerbatim}

\end{sphinxuseclass}\end{sphinxVerbatimOutput}

\end{sphinxuseclass}

\subsubsection{Create a list \sphinxstyleliteralintitle{\sphinxupquote{l5}} that contains the squares of \sphinxstyleemphasis{\sphinxstylestrong{odd}} integers from 1 to 100.}
\label{\detokenize{mckinney_03_practice_02:create-a-list-l5-that-contains-the-squares-of-odd-integers-from-1-to-100}}
\begin{sphinxuseclass}{cell}\begin{sphinxVerbatimInput}

\begin{sphinxuseclass}{cell_input}
\begin{sphinxVerbatim}[commandchars=\\\{\}]
\PYG{n}{l5} \PYG{o}{=} \PYG{p}{[}\PYG{n}{x}\PYG{o}{*}\PYG{o}{*}\PYG{l+m+mi}{2} \PYG{k}{for} \PYG{n}{x} \PYG{o+ow}{in} \PYG{n+nb}{range}\PYG{p}{(}\PYG{l+m+mi}{1}\PYG{p}{,} \PYG{l+m+mi}{101}\PYG{p}{,} \PYG{l+m+mi}{2}\PYG{p}{)}\PYG{p}{]}
\PYG{n+nb}{print}\PYG{p}{(}\PYG{n}{l5}\PYG{p}{)}
\end{sphinxVerbatim}

\end{sphinxuseclass}\end{sphinxVerbatimInput}
\begin{sphinxVerbatimOutput}

\begin{sphinxuseclass}{cell_output}
\begin{sphinxVerbatim}[commandchars=\\\{\}]
[1, 9, 25, 49, 81, 121, 169, 225, 289, 361, 441, 529, 625, 729, 841, 961, 1089, 1225, 1369, 1521, 1681, 1849, 2025, 2209, 2401, 2601, 2809, 3025, 3249, 3481, 3721, 3969, 4225, 4489, 4761, 5041, 5329, 5625, 5929, 6241, 6561, 6889, 7225, 7569, 7921, 8281, 8649, 9025, 9409, 9801]
\end{sphinxVerbatim}

\end{sphinxuseclass}\end{sphinxVerbatimOutput}

\end{sphinxuseclass}
\sphinxAtStartPar
Here is a less efficient solution that uses the \sphinxcode{\sphinxupquote{if some condition}} part of list comprehensions:

\begin{sphinxuseclass}{cell}\begin{sphinxVerbatimInput}

\begin{sphinxuseclass}{cell_input}
\begin{sphinxVerbatim}[commandchars=\\\{\}]
\PYG{n+nb}{print}\PYG{p}{(}\PYG{p}{[}\PYG{n}{x}\PYG{o}{*}\PYG{o}{*}\PYG{l+m+mi}{2} \PYG{k}{for} \PYG{n}{x} \PYG{o+ow}{in} \PYG{n+nb}{range}\PYG{p}{(}\PYG{l+m+mi}{1}\PYG{p}{,} \PYG{l+m+mi}{101}\PYG{p}{)} \PYG{k}{if} \PYG{n}{x}\PYG{o}{\PYGZpc{}}\PYG{k}{2} != 0])
\end{sphinxVerbatim}

\end{sphinxuseclass}\end{sphinxVerbatimInput}
\begin{sphinxVerbatimOutput}

\begin{sphinxuseclass}{cell_output}
\begin{sphinxVerbatim}[commandchars=\\\{\}]
[1, 9, 25, 49, 81, 121, 169, 225, 289, 361, 441, 529, 625, 729, 841, 961, 1089, 1225, 1369, 1521, 1681, 1849, 2025, 2209, 2401, 2601, 2809, 3025, 3249, 3481, 3721, 3969, 4225, 4489, 4761, 5041, 5329, 5625, 5929, 6241, 6561, 6889, 7225, 7569, 7921, 8281, 8649, 9025, 9409, 9801]
\end{sphinxVerbatim}

\end{sphinxuseclass}\end{sphinxVerbatimOutput}

\end{sphinxuseclass}

\subsubsection{Use a lambda function to sort \sphinxstyleliteralintitle{\sphinxupquote{strings}} by the last letter.}
\label{\detokenize{mckinney_03_practice_02:use-a-lambda-function-to-sort-strings-by-the-last-letter}}
\begin{sphinxuseclass}{cell}\begin{sphinxVerbatimInput}

\begin{sphinxuseclass}{cell_input}
\begin{sphinxVerbatim}[commandchars=\\\{\}]
\PYG{n}{strings} \PYG{o}{=} \PYG{p}{[}\PYG{l+s+s1}{\PYGZsq{}}\PYG{l+s+s1}{card}\PYG{l+s+s1}{\PYGZsq{}}\PYG{p}{,} \PYG{l+s+s1}{\PYGZsq{}}\PYG{l+s+s1}{aaaa}\PYG{l+s+s1}{\PYGZsq{}}\PYG{p}{,} \PYG{l+s+s1}{\PYGZsq{}}\PYG{l+s+s1}{foo}\PYG{l+s+s1}{\PYGZsq{}}\PYG{p}{,} \PYG{l+s+s1}{\PYGZsq{}}\PYG{l+s+s1}{bar}\PYG{l+s+s1}{\PYGZsq{}}\PYG{p}{,} \PYG{l+s+s1}{\PYGZsq{}}\PYG{l+s+s1}{abab}\PYG{l+s+s1}{\PYGZsq{}}\PYG{p}{]}
\end{sphinxVerbatim}

\end{sphinxuseclass}\end{sphinxVerbatimInput}

\end{sphinxuseclass}
\begin{sphinxuseclass}{cell}\begin{sphinxVerbatimInput}

\begin{sphinxuseclass}{cell_input}
\begin{sphinxVerbatim}[commandchars=\\\{\}]
\PYG{n}{strings}\PYG{o}{.}\PYG{n}{sort}\PYG{p}{(}\PYG{n}{key}\PYG{o}{=}\PYG{k}{lambda} \PYG{n}{x}\PYG{p}{:} \PYG{n}{x}\PYG{p}{[}\PYG{o}{\PYGZhy{}}\PYG{l+m+mi}{1}\PYG{p}{]}\PYG{p}{)}
\end{sphinxVerbatim}

\end{sphinxuseclass}\end{sphinxVerbatimInput}

\end{sphinxuseclass}
\begin{sphinxuseclass}{cell}\begin{sphinxVerbatimInput}

\begin{sphinxuseclass}{cell_input}
\begin{sphinxVerbatim}[commandchars=\\\{\}]
\PYG{n}{strings}
\end{sphinxVerbatim}

\end{sphinxuseclass}\end{sphinxVerbatimInput}
\begin{sphinxVerbatimOutput}

\begin{sphinxuseclass}{cell_output}
\begin{sphinxVerbatim}[commandchars=\\\{\}]
[\PYGZsq{}aaaa\PYGZsq{}, \PYGZsq{}abab\PYGZsq{}, \PYGZsq{}card\PYGZsq{}, \PYGZsq{}foo\PYGZsq{}, \PYGZsq{}bar\PYGZsq{}]
\end{sphinxVerbatim}

\end{sphinxuseclass}\end{sphinxVerbatimOutput}

\end{sphinxuseclass}

\subsubsection{Given an integer array \sphinxstyleliteralintitle{\sphinxupquote{nums}} and an integer \sphinxstyleliteralintitle{\sphinxupquote{k}}, return the \$k\textasciicircum{}\{th\}\$ largest element in the array.}
\label{\detokenize{mckinney_03_practice_02:given-an-integer-array-nums-and-an-integer-k-return-the-k-th-largest-element-in-the-array}}
\sphinxAtStartPar
Note that it is the \$k\textasciicircum{}\{th\}\$ largest element in the sorted order, not the \$k\textasciicircum{}\{th\}\$ distinct element.

\sphinxAtStartPar
Example 1:

\sphinxAtStartPar
Input: \sphinxcode{\sphinxupquote{nums = {[}3,2,1,5,6,4{]}}}, \sphinxcode{\sphinxupquote{k = 2}} \\
Output: \sphinxcode{\sphinxupquote{5}}

\sphinxAtStartPar
Example 2:

\sphinxAtStartPar
Input: \sphinxcode{\sphinxupquote{nums = {[}3,2,3,1,2,4,5,5,6{]}}}, \sphinxcode{\sphinxupquote{k = 4}} \\
Output: \sphinxcode{\sphinxupquote{4}}

\sphinxAtStartPar
I saw this question on \sphinxhref{https://leetcode.com/problems/kth-largest-element-in-an-array/}{LeetCode}.

\begin{sphinxuseclass}{cell}\begin{sphinxVerbatimInput}

\begin{sphinxuseclass}{cell_input}
\begin{sphinxVerbatim}[commandchars=\\\{\}]
\PYG{k}{def} \PYG{n+nf}{nums}\PYG{p}{(}\PYG{n}{x}\PYG{p}{,} \PYG{n}{k}\PYG{p}{)}\PYG{p}{:}
    \PYG{n}{x\PYGZus{}copy} \PYG{o}{=} \PYG{n}{x}\PYG{o}{.}\PYG{n}{copy}\PYG{p}{(}\PYG{p}{)}
    \PYG{n}{x\PYGZus{}copy}\PYG{o}{.}\PYG{n}{sort}\PYG{p}{(}\PYG{p}{)}
    \PYG{k}{return} \PYG{n}{x\PYGZus{}copy}\PYG{p}{[}\PYG{o}{\PYGZhy{}}\PYG{n}{k}\PYG{p}{]}
\end{sphinxVerbatim}

\end{sphinxuseclass}\end{sphinxVerbatimInput}

\end{sphinxuseclass}
\begin{sphinxuseclass}{cell}\begin{sphinxVerbatimInput}

\begin{sphinxuseclass}{cell_input}
\begin{sphinxVerbatim}[commandchars=\\\{\}]
\PYG{n}{nums}\PYG{p}{(}\PYG{n}{x}\PYG{o}{=}\PYG{p}{[}\PYG{l+m+mi}{3}\PYG{p}{,}\PYG{l+m+mi}{2}\PYG{p}{,}\PYG{l+m+mi}{1}\PYG{p}{,}\PYG{l+m+mi}{5}\PYG{p}{,}\PYG{l+m+mi}{6}\PYG{p}{,}\PYG{l+m+mi}{4}\PYG{p}{]}\PYG{p}{,} \PYG{n}{k}\PYG{o}{=}\PYG{l+m+mi}{2}\PYG{p}{)}
\end{sphinxVerbatim}

\end{sphinxuseclass}\end{sphinxVerbatimInput}
\begin{sphinxVerbatimOutput}

\begin{sphinxuseclass}{cell_output}
\begin{sphinxVerbatim}[commandchars=\\\{\}]
5
\end{sphinxVerbatim}

\end{sphinxuseclass}\end{sphinxVerbatimOutput}

\end{sphinxuseclass}

\subsubsection{Given an integer array \sphinxstyleliteralintitle{\sphinxupquote{nums}} and an integer \sphinxstyleliteralintitle{\sphinxupquote{k}}, return the \sphinxstyleliteralintitle{\sphinxupquote{k}} most frequent elements.}
\label{\detokenize{mckinney_03_practice_02:given-an-integer-array-nums-and-an-integer-k-return-the-k-most-frequent-elements}}
\sphinxAtStartPar
You may return the answer in any order.

\sphinxAtStartPar
Example 1:

\sphinxAtStartPar
Input: \sphinxcode{\sphinxupquote{nums = {[}1,1,1,2,2,3{]}}}, \sphinxcode{\sphinxupquote{k = 2}} \\
Output: \sphinxcode{\sphinxupquote{{[}1,2{]}}}

\sphinxAtStartPar
Example 2:

\sphinxAtStartPar
Input: \sphinxcode{\sphinxupquote{nums = {[}1{]}}}, \sphinxcode{\sphinxupquote{k = 1}} \\
Output: \sphinxcode{\sphinxupquote{{[}1{]}}}

\sphinxAtStartPar
I saw this question on \sphinxhref{https://leetcode.com/problems/top-k-frequent-elements/}{LeetCode}.

\begin{sphinxuseclass}{cell}\begin{sphinxVerbatimInput}

\begin{sphinxuseclass}{cell_input}
\begin{sphinxVerbatim}[commandchars=\\\{\}]
\PYG{k}{def} \PYG{n+nf}{nums}\PYG{p}{(}\PYG{n}{nums}\PYG{p}{,} \PYG{n}{k}\PYG{p}{)}\PYG{p}{:}
    \PYG{n}{counts} \PYG{o}{=} \PYG{p}{\PYGZob{}}\PYG{p}{\PYGZcb{}}
    \PYG{k}{for} \PYG{n}{n} \PYG{o+ow}{in} \PYG{n}{nums}\PYG{p}{:}
        \PYG{k}{if} \PYG{n}{n} \PYG{o+ow}{in} \PYG{n}{counts}\PYG{p}{:}
            \PYG{n}{counts}\PYG{p}{[}\PYG{n}{n}\PYG{p}{]} \PYG{o}{+}\PYG{o}{=} \PYG{l+m+mi}{1}
        \PYG{k}{else}\PYG{p}{:}
            \PYG{n}{counts}\PYG{p}{[}\PYG{n}{n}\PYG{p}{]} \PYG{o}{=} \PYG{l+m+mi}{1}
    \PYG{k}{return} \PYG{p}{[}\PYG{n}{x}\PYG{p}{[}\PYG{l+m+mi}{0}\PYG{p}{]} \PYG{k}{for} \PYG{n}{x} \PYG{o+ow}{in} \PYG{n+nb}{sorted}\PYG{p}{(}\PYG{n}{counts}\PYG{o}{.}\PYG{n}{items}\PYG{p}{(}\PYG{p}{)}\PYG{p}{,} \PYG{n}{key}\PYG{o}{=}\PYG{k}{lambda} \PYG{n}{x}\PYG{p}{:} \PYG{n}{x}\PYG{p}{[}\PYG{l+m+mi}{1}\PYG{p}{]}\PYG{p}{,} \PYG{n}{reverse}\PYG{o}{=}\PYG{k+kc}{True}\PYG{p}{)}\PYG{p}{[}\PYG{p}{:}\PYG{n}{k}\PYG{p}{]}\PYG{p}{]}
\end{sphinxVerbatim}

\end{sphinxuseclass}\end{sphinxVerbatimInput}

\end{sphinxuseclass}
\begin{sphinxuseclass}{cell}\begin{sphinxVerbatimInput}

\begin{sphinxuseclass}{cell_input}
\begin{sphinxVerbatim}[commandchars=\\\{\}]
\PYG{n}{nums}\PYG{p}{(}\PYG{n}{nums}\PYG{o}{=}\PYG{p}{[}\PYG{l+m+mi}{1}\PYG{p}{,}\PYG{l+m+mi}{1}\PYG{p}{,}\PYG{l+m+mi}{1}\PYG{p}{,}\PYG{l+m+mi}{2}\PYG{p}{,}\PYG{l+m+mi}{2}\PYG{p}{,}\PYG{l+m+mi}{3}\PYG{p}{]}\PYG{p}{,} \PYG{n}{k}\PYG{o}{=}\PYG{l+m+mi}{2}\PYG{p}{)}
\end{sphinxVerbatim}

\end{sphinxuseclass}\end{sphinxVerbatimInput}
\begin{sphinxVerbatimOutput}

\begin{sphinxuseclass}{cell_output}
\begin{sphinxVerbatim}[commandchars=\\\{\}]
[1, 2]
\end{sphinxVerbatim}

\end{sphinxuseclass}\end{sphinxVerbatimOutput}

\end{sphinxuseclass}

\subsubsection{Test whether the given strings are palindromes.}
\label{\detokenize{mckinney_03_practice_02:test-whether-the-given-strings-are-palindromes}}
\sphinxAtStartPar
Input: \sphinxcode{\sphinxupquote{{[}"aba", "no"{]}}} \\
Output: \sphinxcode{\sphinxupquote{{[}True, False{]}}}

\begin{sphinxuseclass}{cell}\begin{sphinxVerbatimInput}

\begin{sphinxuseclass}{cell_input}
\begin{sphinxVerbatim}[commandchars=\\\{\}]
\PYG{k}{def} \PYG{n+nf}{is\PYGZus{}palindrome}\PYG{p}{(}\PYG{n}{x}\PYG{p}{)}\PYG{p}{:}
    \PYG{k}{return} \PYG{p}{[}\PYG{n+nb}{list}\PYG{p}{(}\PYG{n}{y}\PYG{p}{)} \PYG{o}{==} \PYG{n+nb}{list}\PYG{p}{(}\PYG{n}{y}\PYG{p}{)}\PYG{p}{[}\PYG{p}{:}\PYG{p}{:}\PYG{o}{\PYGZhy{}}\PYG{l+m+mi}{1}\PYG{p}{]} \PYG{k}{for} \PYG{n}{y} \PYG{o+ow}{in} \PYG{n}{x}\PYG{p}{]}
\end{sphinxVerbatim}

\end{sphinxuseclass}\end{sphinxVerbatimInput}

\end{sphinxuseclass}
\begin{sphinxuseclass}{cell}\begin{sphinxVerbatimInput}

\begin{sphinxuseclass}{cell_input}
\begin{sphinxVerbatim}[commandchars=\\\{\}]
\PYG{n}{is\PYGZus{}palindrome}\PYG{p}{(}\PYG{p}{[}\PYG{l+s+s2}{\PYGZdq{}}\PYG{l+s+s2}{aba}\PYG{l+s+s2}{\PYGZdq{}}\PYG{p}{,} \PYG{l+s+s2}{\PYGZdq{}}\PYG{l+s+s2}{no}\PYG{l+s+s2}{\PYGZdq{}}\PYG{p}{]}\PYG{p}{)}
\end{sphinxVerbatim}

\end{sphinxuseclass}\end{sphinxVerbatimInput}
\begin{sphinxVerbatimOutput}

\begin{sphinxuseclass}{cell_output}
\begin{sphinxVerbatim}[commandchars=\\\{\}]
[True, False]
\end{sphinxVerbatim}

\end{sphinxuseclass}\end{sphinxVerbatimOutput}

\end{sphinxuseclass}

\subsubsection{Write a function \sphinxstyleliteralintitle{\sphinxupquote{returns()}} that accepts lists of prices and dividends and returns a list of returns.}
\label{\detokenize{mckinney_03_practice_02:write-a-function-returns-that-accepts-lists-of-prices-and-dividends-and-returns-a-list-of-returns}}
\begin{sphinxuseclass}{cell}\begin{sphinxVerbatimInput}

\begin{sphinxuseclass}{cell_input}
\begin{sphinxVerbatim}[commandchars=\\\{\}]
\PYG{n}{prices} \PYG{o}{=} \PYG{p}{[}\PYG{l+m+mi}{100}\PYG{p}{,} \PYG{l+m+mi}{150}\PYG{p}{,} \PYG{l+m+mi}{100}\PYG{p}{,} \PYG{l+m+mi}{50}\PYG{p}{,} \PYG{l+m+mi}{100}\PYG{p}{,} \PYG{l+m+mi}{150}\PYG{p}{,} \PYG{l+m+mi}{100}\PYG{p}{,} \PYG{l+m+mi}{150}\PYG{p}{]}
\PYG{n}{dividends} \PYG{o}{=} \PYG{p}{[}\PYG{l+m+mi}{1}\PYG{p}{,} \PYG{l+m+mi}{1}\PYG{p}{,} \PYG{l+m+mi}{1}\PYG{p}{,} \PYG{l+m+mi}{1}\PYG{p}{,} \PYG{l+m+mi}{2}\PYG{p}{,} \PYG{l+m+mi}{2}\PYG{p}{,} \PYG{l+m+mi}{2}\PYG{p}{,} \PYG{l+m+mi}{2}\PYG{p}{]}
\end{sphinxVerbatim}

\end{sphinxuseclass}\end{sphinxVerbatimInput}

\end{sphinxuseclass}
\begin{sphinxuseclass}{cell}\begin{sphinxVerbatimInput}

\begin{sphinxuseclass}{cell_input}
\begin{sphinxVerbatim}[commandchars=\\\{\}]
\PYG{k}{def} \PYG{n+nf}{returns}\PYG{p}{(}\PYG{n}{p}\PYG{p}{,} \PYG{n}{d}\PYG{p}{)}\PYG{p}{:}
    \PYG{n}{r} \PYG{o}{=} \PYG{p}{[}\PYG{k+kc}{None}\PYG{p}{]}
    \PYG{k}{for} \PYG{n}{t} \PYG{o+ow}{in} \PYG{n+nb}{range}\PYG{p}{(}\PYG{l+m+mi}{1}\PYG{p}{,} \PYG{n+nb}{len}\PYG{p}{(}\PYG{n}{p}\PYG{p}{)}\PYG{p}{)}\PYG{p}{:}
        \PYG{n}{pt} \PYG{o}{=} \PYG{n}{p}\PYG{p}{[}\PYG{n}{t}\PYG{p}{]}
        \PYG{n}{ptm1} \PYG{o}{=} \PYG{n}{p}\PYG{p}{[}\PYG{n}{t}\PYG{o}{\PYGZhy{}}\PYG{l+m+mi}{1}\PYG{p}{]}
        \PYG{n}{dt} \PYG{o}{=} \PYG{n}{d}\PYG{p}{[}\PYG{n}{t}\PYG{p}{]}
        \PYG{n}{rt} \PYG{o}{=} \PYG{p}{(}\PYG{n}{pt} \PYG{o}{\PYGZhy{}} \PYG{n}{ptm1} \PYG{o}{+} \PYG{n}{dt}\PYG{p}{)} \PYG{o}{/} \PYG{n}{ptm1}
        \PYG{n}{r}\PYG{o}{.}\PYG{n}{append}\PYG{p}{(}\PYG{n}{rt}\PYG{p}{)}
        
    \PYG{k}{return} \PYG{n}{r}
\end{sphinxVerbatim}

\end{sphinxuseclass}\end{sphinxVerbatimInput}

\end{sphinxuseclass}
\begin{sphinxuseclass}{cell}\begin{sphinxVerbatimInput}

\begin{sphinxuseclass}{cell_input}
\begin{sphinxVerbatim}[commandchars=\\\{\}]
\PYG{n}{returns}\PYG{p}{(}\PYG{n}{p}\PYG{o}{=}\PYG{n}{prices}\PYG{p}{,} \PYG{n}{d}\PYG{o}{=}\PYG{n}{dividends}\PYG{p}{)}
\end{sphinxVerbatim}

\end{sphinxuseclass}\end{sphinxVerbatimInput}
\begin{sphinxVerbatimOutput}

\begin{sphinxuseclass}{cell_output}
\begin{sphinxVerbatim}[commandchars=\\\{\}]
[None, 0.51, \PYGZhy{}0.32666666666666666, \PYGZhy{}0.49, 1.04, 0.52, \PYGZhy{}0.32, 0.52]
\end{sphinxVerbatim}

\end{sphinxuseclass}\end{sphinxVerbatimOutput}

\end{sphinxuseclass}

\subsubsection{Rewrite the function \sphinxstyleliteralintitle{\sphinxupquote{returns()}} so it returns lists of returns, capital gains yields, and dividend yields.}
\label{\detokenize{mckinney_03_practice_02:rewrite-the-function-returns-so-it-returns-lists-of-returns-capital-gains-yields-and-dividend-yields}}
\begin{sphinxuseclass}{cell}\begin{sphinxVerbatimInput}

\begin{sphinxuseclass}{cell_input}
\begin{sphinxVerbatim}[commandchars=\\\{\}]
\PYG{k}{def} \PYG{n+nf}{returns}\PYG{p}{(}\PYG{n}{p}\PYG{p}{,} \PYG{n}{d}\PYG{p}{)}\PYG{p}{:}
    \PYG{n}{r}\PYG{p}{,} \PYG{n}{cg}\PYG{p}{,} \PYG{n}{dy} \PYG{o}{=} \PYG{p}{[}\PYG{k+kc}{None}\PYG{p}{]}\PYG{p}{,} \PYG{p}{[}\PYG{k+kc}{None}\PYG{p}{]}\PYG{p}{,} \PYG{p}{[}\PYG{k+kc}{None}\PYG{p}{]}
    \PYG{k}{for} \PYG{n}{t} \PYG{o+ow}{in} \PYG{n+nb}{range}\PYG{p}{(}\PYG{l+m+mi}{1}\PYG{p}{,} \PYG{n+nb}{len}\PYG{p}{(}\PYG{n}{p}\PYG{p}{)}\PYG{p}{)}\PYG{p}{:}
        \PYG{n}{pt} \PYG{o}{=} \PYG{n}{p}\PYG{p}{[}\PYG{n}{t}\PYG{p}{]}
        \PYG{n}{ptm1} \PYG{o}{=} \PYG{n}{p}\PYG{p}{[}\PYG{n}{t}\PYG{o}{\PYGZhy{}}\PYG{l+m+mi}{1}\PYG{p}{]}
        \PYG{n}{dt} \PYG{o}{=} \PYG{n}{d}\PYG{p}{[}\PYG{n}{t}\PYG{p}{]}
        \PYG{n}{cgt} \PYG{o}{=} \PYG{p}{(}\PYG{n}{pt} \PYG{o}{\PYGZhy{}} \PYG{n}{ptm1}\PYG{p}{)} \PYG{o}{/} \PYG{n}{ptm1}
        \PYG{n}{dyt} \PYG{o}{=} \PYG{n}{dt} \PYG{o}{/} \PYG{n}{ptm1}
        \PYG{n}{rt} \PYG{o}{=} \PYG{n}{cgt} \PYG{o}{+} \PYG{n}{dyt}
        \PYG{n}{r}\PYG{o}{.}\PYG{n}{append}\PYG{p}{(}\PYG{n}{rt}\PYG{p}{)}
        \PYG{n}{cg}\PYG{o}{.}\PYG{n}{append}\PYG{p}{(}\PYG{n}{cgt}\PYG{p}{)}
        \PYG{n}{dy}\PYG{o}{.}\PYG{n}{append}\PYG{p}{(}\PYG{n}{dyt}\PYG{p}{)}
        
    \PYG{k}{return} \PYG{p}{\PYGZob{}}\PYG{l+s+s1}{\PYGZsq{}}\PYG{l+s+s1}{r}\PYG{l+s+s1}{\PYGZsq{}}\PYG{p}{:} \PYG{n}{r}\PYG{p}{,} \PYG{l+s+s1}{\PYGZsq{}}\PYG{l+s+s1}{cg}\PYG{l+s+s1}{\PYGZsq{}}\PYG{p}{:} \PYG{n}{cg}\PYG{p}{,} \PYG{l+s+s1}{\PYGZsq{}}\PYG{l+s+s1}{dy}\PYG{l+s+s1}{\PYGZsq{}}\PYG{p}{:} \PYG{n}{dy}\PYG{p}{\PYGZcb{}}
\end{sphinxVerbatim}

\end{sphinxuseclass}\end{sphinxVerbatimInput}

\end{sphinxuseclass}
\begin{sphinxuseclass}{cell}\begin{sphinxVerbatimInput}

\begin{sphinxuseclass}{cell_input}
\begin{sphinxVerbatim}[commandchars=\\\{\}]
\PYG{n}{returns}\PYG{p}{(}\PYG{n}{p}\PYG{o}{=}\PYG{n}{prices}\PYG{p}{,} \PYG{n}{d}\PYG{o}{=}\PYG{n}{dividends}\PYG{p}{)}
\end{sphinxVerbatim}

\end{sphinxuseclass}\end{sphinxVerbatimInput}
\begin{sphinxVerbatimOutput}

\begin{sphinxuseclass}{cell_output}
\begin{sphinxVerbatim}[commandchars=\\\{\}]
\PYGZob{}\PYGZsq{}r\PYGZsq{}: [None, 0.51, \PYGZhy{}0.32666666666666666, \PYGZhy{}0.49, 1.04, 0.52, \PYGZhy{}0.32, 0.52],
 \PYGZsq{}cg\PYGZsq{}: [None,
  0.5,
  \PYGZhy{}0.3333333333333333,
  \PYGZhy{}0.5,
  1.0,
  0.5,
  \PYGZhy{}0.3333333333333333,
  0.5],
 \PYGZsq{}dy\PYGZsq{}: [None,
  0.01,
  0.006666666666666667,
  0.01,
  0.04,
  0.02,
  0.013333333333333334,
  0.02]\PYGZcb{}
\end{sphinxVerbatim}

\end{sphinxuseclass}\end{sphinxVerbatimOutput}

\end{sphinxuseclass}

\subsubsection{Rescale and shift numbers so that they cover the range \sphinxstyleliteralintitle{\sphinxupquote{{[}0, 1{]}}}.}
\label{\detokenize{mckinney_03_practice_02:rescale-and-shift-numbers-so-that-they-cover-the-range-0-1}}
\sphinxAtStartPar
Input: \sphinxcode{\sphinxupquote{{[}18.5, 17.0, 18.0, 19.0, 18.0{]}}} \\
Output: \sphinxcode{\sphinxupquote{{[}0.75, 0.0, 0.5, 1.0, 0.5{]}}}

\begin{sphinxuseclass}{cell}\begin{sphinxVerbatimInput}

\begin{sphinxuseclass}{cell_input}
\begin{sphinxVerbatim}[commandchars=\\\{\}]
\PYG{n}{numbers} \PYG{o}{=} \PYG{p}{[}\PYG{l+m+mf}{18.5}\PYG{p}{,} \PYG{l+m+mf}{17.0}\PYG{p}{,} \PYG{l+m+mf}{18.0}\PYG{p}{,} \PYG{l+m+mf}{19.0}\PYG{p}{,} \PYG{l+m+mf}{18.0}\PYG{p}{]}
\end{sphinxVerbatim}

\end{sphinxuseclass}\end{sphinxVerbatimInput}

\end{sphinxuseclass}
\begin{sphinxuseclass}{cell}\begin{sphinxVerbatimInput}

\begin{sphinxuseclass}{cell_input}
\begin{sphinxVerbatim}[commandchars=\\\{\}]
\PYG{k}{def} \PYG{n+nf}{rescale}\PYG{p}{(}\PYG{n}{x}\PYG{p}{)}\PYG{p}{:}
    \PYG{n}{x\PYGZus{}min} \PYG{o}{=} \PYG{n+nb}{min}\PYG{p}{(}\PYG{n}{x}\PYG{p}{)}
    \PYG{n}{x\PYGZus{}max} \PYG{o}{=} \PYG{n+nb}{max}\PYG{p}{(}\PYG{n}{x}\PYG{p}{)}
    \PYG{k}{return} \PYG{p}{[}\PYG{p}{(}\PYG{n}{x} \PYG{o}{\PYGZhy{}} \PYG{n}{x\PYGZus{}min}\PYG{p}{)} \PYG{o}{/} \PYG{p}{(}\PYG{n}{x\PYGZus{}max} \PYG{o}{\PYGZhy{}} \PYG{n}{x\PYGZus{}min}\PYG{p}{)} \PYG{k}{for} \PYG{n}{x} \PYG{o+ow}{in} \PYG{n}{x}\PYG{p}{]}
\end{sphinxVerbatim}

\end{sphinxuseclass}\end{sphinxVerbatimInput}

\end{sphinxuseclass}
\begin{sphinxuseclass}{cell}\begin{sphinxVerbatimInput}

\begin{sphinxuseclass}{cell_input}
\begin{sphinxVerbatim}[commandchars=\\\{\}]
\PYG{n}{rescale}\PYG{p}{(}\PYG{n}{numbers}\PYG{p}{)}
\end{sphinxVerbatim}

\end{sphinxuseclass}\end{sphinxVerbatimInput}
\begin{sphinxVerbatimOutput}

\begin{sphinxuseclass}{cell_output}
\begin{sphinxVerbatim}[commandchars=\\\{\}]
[0.75, 0.0, 0.5, 1.0, 0.5]
\end{sphinxVerbatim}

\end{sphinxuseclass}\end{sphinxVerbatimOutput}

\end{sphinxuseclass}
\sphinxstepscope


\chapter{McKinney Chapter 4 \sphinxhyphen{} NumPy Basics: Arrays and Vectorized Computation}
\label{\detokenize{mckinney_04_lecture:mckinney-chapter-4-numpy-basics-arrays-and-vectorized-computation}}\label{\detokenize{mckinney_04_lecture::doc}}

\section{Introduction}
\label{\detokenize{mckinney_04_lecture:introduction}}
\sphinxAtStartPar
Chapter 4 of Wes McKinney’s \sphinxhref{https://wesmckinney.com/book/}{\sphinxstyleemphasis{Python for Data Analysis}} discusses the NumPy package (an abbreviation of numerical Python), which is the foundation for numerical computing in Python, including pandas.

\sphinxAtStartPar
We will focus on:
\begin{enumerate}
\sphinxsetlistlabels{\arabic}{enumi}{enumii}{}{.}%
\item {} 
\sphinxAtStartPar
Creating arrays

\item {} 
\sphinxAtStartPar
Slicing arrays

\item {} 
\sphinxAtStartPar
Applying functions and methods to arrays

\item {} 
\sphinxAtStartPar
Using conditional logic with arrays (i.e., \sphinxcode{\sphinxupquote{np.where()}} and \sphinxcode{\sphinxupquote{np.select()}})

\end{enumerate}

\sphinxAtStartPar
\sphinxstyleemphasis{\sphinxstylestrong{Note:}}
Indented block quotes are from McKinney unless otherwise indicated.
The section numbers here differ from McKinney because we will only discuss some topics.

\sphinxAtStartPar
The typical abbreviation for NumPy is \sphinxcode{\sphinxupquote{np}}.

\begin{sphinxuseclass}{cell}\begin{sphinxVerbatimInput}

\begin{sphinxuseclass}{cell_input}
\begin{sphinxVerbatim}[commandchars=\\\{\}]
\PYG{k+kn}{import} \PYG{n+nn}{numpy} \PYG{k}{as} \PYG{n+nn}{np}
\end{sphinxVerbatim}

\end{sphinxuseclass}\end{sphinxVerbatimInput}

\end{sphinxuseclass}
\sphinxAtStartPar
The “magic” function \sphinxcode{\sphinxupquote{\%precision 4}} tells JupyterLab to print NumPy arrays to 4 decimals.
This magic function only changes the printed precision and does not change the stored precision of the underlying values.

\begin{sphinxuseclass}{cell}\begin{sphinxVerbatimInput}

\begin{sphinxuseclass}{cell_input}
\begin{sphinxVerbatim}[commandchars=\\\{\}]
\PYG{o}{\PYGZpc{}}\PYG{k}{precision} 4
\end{sphinxVerbatim}

\end{sphinxuseclass}\end{sphinxVerbatimInput}
\begin{sphinxVerbatimOutput}

\begin{sphinxuseclass}{cell_output}
\begin{sphinxVerbatim}[commandchars=\\\{\}]
\PYGZsq{}\PYGZpc{}.4f\PYGZsq{}
\end{sphinxVerbatim}

\end{sphinxuseclass}\end{sphinxVerbatimOutput}

\end{sphinxuseclass}
\sphinxAtStartPar
McKinney thoroughly discuesses the history on NumPy, as well as its technical advantages.
But here is a simple illustration of the speed and syntax advantages of NumPy of Python’s built\sphinxhyphen{}in data structures.
First, we create a list and array with values from 0 to 999,999.

\begin{sphinxuseclass}{cell}\begin{sphinxVerbatimInput}

\begin{sphinxuseclass}{cell_input}
\begin{sphinxVerbatim}[commandchars=\\\{\}]
\PYG{n}{my\PYGZus{}list} \PYG{o}{=} \PYG{n+nb}{list}\PYG{p}{(}\PYG{n+nb}{range}\PYG{p}{(}\PYG{l+m+mi}{1\PYGZus{}000\PYGZus{}000}\PYG{p}{)}\PYG{p}{)}
\end{sphinxVerbatim}

\end{sphinxuseclass}\end{sphinxVerbatimInput}

\end{sphinxuseclass}
\begin{sphinxuseclass}{cell}\begin{sphinxVerbatimInput}

\begin{sphinxuseclass}{cell_input}
\begin{sphinxVerbatim}[commandchars=\\\{\}]
\PYG{n}{my\PYGZus{}arr} \PYG{o}{=} \PYG{n}{np}\PYG{o}{.}\PYG{n}{arange}\PYG{p}{(}\PYG{l+m+mi}{1\PYGZus{}000\PYGZus{}000}\PYG{p}{)}
\end{sphinxVerbatim}

\end{sphinxuseclass}\end{sphinxVerbatimInput}

\end{sphinxuseclass}
\begin{sphinxuseclass}{cell}\begin{sphinxVerbatimInput}

\begin{sphinxuseclass}{cell_input}
\begin{sphinxVerbatim}[commandchars=\\\{\}]
\PYG{n}{my\PYGZus{}list}\PYG{p}{[}\PYG{p}{:}\PYG{l+m+mi}{5}\PYG{p}{]}
\end{sphinxVerbatim}

\end{sphinxuseclass}\end{sphinxVerbatimInput}
\begin{sphinxVerbatimOutput}

\begin{sphinxuseclass}{cell_output}
\begin{sphinxVerbatim}[commandchars=\\\{\}]
[0, 1, 2, 3, 4]
\end{sphinxVerbatim}

\end{sphinxuseclass}\end{sphinxVerbatimOutput}

\end{sphinxuseclass}
\begin{sphinxuseclass}{cell}\begin{sphinxVerbatimInput}

\begin{sphinxuseclass}{cell_input}
\begin{sphinxVerbatim}[commandchars=\\\{\}]
\PYG{n}{my\PYGZus{}arr}\PYG{p}{[}\PYG{p}{:}\PYG{l+m+mi}{5}\PYG{p}{]}
\end{sphinxVerbatim}

\end{sphinxuseclass}\end{sphinxVerbatimInput}
\begin{sphinxVerbatimOutput}

\begin{sphinxuseclass}{cell_output}
\begin{sphinxVerbatim}[commandchars=\\\{\}]
array([0, 1, 2, 3, 4])
\end{sphinxVerbatim}

\end{sphinxuseclass}\end{sphinxVerbatimOutput}

\end{sphinxuseclass}
\sphinxAtStartPar
If we want to double each value in \sphinxcode{\sphinxupquote{my\_list}} we have to use a for loop or a list comprehension.

\begin{sphinxuseclass}{cell}\begin{sphinxVerbatimInput}

\begin{sphinxuseclass}{cell_input}
\begin{sphinxVerbatim}[commandchars=\\\{\}]
\PYG{n+nb}{len}\PYG{p}{(}\PYG{n}{my\PYGZus{}list} \PYG{o}{*} \PYG{l+m+mi}{2}\PYG{p}{)} \PYG{c+c1}{\PYGZsh{} concatenates two copies of my\PYGZus{}list}
\end{sphinxVerbatim}

\end{sphinxuseclass}\end{sphinxVerbatimInput}
\begin{sphinxVerbatimOutput}

\begin{sphinxuseclass}{cell_output}
\begin{sphinxVerbatim}[commandchars=\\\{\}]
2000000
\end{sphinxVerbatim}

\end{sphinxuseclass}\end{sphinxVerbatimOutput}

\end{sphinxuseclass}
\begin{sphinxuseclass}{cell}\begin{sphinxVerbatimInput}

\begin{sphinxuseclass}{cell_input}
\begin{sphinxVerbatim}[commandchars=\\\{\}]
\PYG{c+c1}{\PYGZsh{} [2 * x for x in my\PYGZus{}list] \PYGZsh{} list comprehension to double each value}
\end{sphinxVerbatim}

\end{sphinxuseclass}\end{sphinxVerbatimInput}

\end{sphinxuseclass}
\sphinxAtStartPar
However, we can multiply \sphinxcode{\sphinxupquote{my\_arr}} by two because math “just works” with NumPy.

\begin{sphinxuseclass}{cell}\begin{sphinxVerbatimInput}

\begin{sphinxuseclass}{cell_input}
\begin{sphinxVerbatim}[commandchars=\\\{\}]
\PYG{n}{my\PYGZus{}arr} \PYG{o}{*} \PYG{l+m+mi}{2}
\end{sphinxVerbatim}

\end{sphinxuseclass}\end{sphinxVerbatimInput}
\begin{sphinxVerbatimOutput}

\begin{sphinxuseclass}{cell_output}
\begin{sphinxVerbatim}[commandchars=\\\{\}]
array([      0,       2,       4, ..., 1999994, 1999996, 1999998])
\end{sphinxVerbatim}

\end{sphinxuseclass}\end{sphinxVerbatimOutput}

\end{sphinxuseclass}
\sphinxAtStartPar
We can use the “magic” function \sphinxcode{\sphinxupquote{\%timeit}} to time these two calculations.

\begin{sphinxuseclass}{cell}\begin{sphinxVerbatimInput}

\begin{sphinxuseclass}{cell_input}
\begin{sphinxVerbatim}[commandchars=\\\{\}]
\PYG{o}{\PYGZpc{}}\PYG{k}{timeit} [x * 2 for x in my\PYGZus{}list]
\end{sphinxVerbatim}

\end{sphinxuseclass}\end{sphinxVerbatimInput}
\begin{sphinxVerbatimOutput}

\begin{sphinxuseclass}{cell_output}
\begin{sphinxVerbatim}[commandchars=\\\{\}]
29.1 ms ± 213 µs per loop (mean ± std. dev. of 7 runs, 10 loops each)
\end{sphinxVerbatim}

\end{sphinxuseclass}\end{sphinxVerbatimOutput}

\end{sphinxuseclass}
\begin{sphinxuseclass}{cell}\begin{sphinxVerbatimInput}

\begin{sphinxuseclass}{cell_input}
\begin{sphinxVerbatim}[commandchars=\\\{\}]
\PYG{o}{\PYGZpc{}}\PYG{k}{timeit} my\PYGZus{}arr * 2
\end{sphinxVerbatim}

\end{sphinxuseclass}\end{sphinxVerbatimInput}
\begin{sphinxVerbatimOutput}

\begin{sphinxuseclass}{cell_output}
\begin{sphinxVerbatim}[commandchars=\\\{\}]
177 µs ± 3.28 µs per loop (mean ± std. dev. of 7 runs, 10,000 loops each)
\end{sphinxVerbatim}

\end{sphinxuseclass}\end{sphinxVerbatimOutput}

\end{sphinxuseclass}
\sphinxAtStartPar
The NumPy version is a hundred times faster than the list version.
The NumPy version is also faster to type, read, and troubleshoot, which are typically more important.
Our time is more valuable than the computer time!


\section{The NumPy ndarray: A Multidimensional Array Object}
\label{\detokenize{mckinney_04_lecture:the-numpy-ndarray-a-multidimensional-array-object}}\begin{quote}

\sphinxAtStartPar
One of the key features of NumPy is its N\sphinxhyphen{}dimensional array object, or ndarray, which is a fast, flexible container for large datasets in Python. Arrays enable you to perform mathematical operations on whole blocks of data using similar syntax to the equivalent operations between scalar elements.
\end{quote}

\sphinxAtStartPar
We generate random data to explore NumPy arrays.
Whenever we generate random data, we should set the random number seed with \sphinxcode{\sphinxupquote{np.random.seed(42)}}, which makes our random numbers repeatable.
If we use the same random number seed, our random numbers will be the same.

\begin{sphinxuseclass}{cell}\begin{sphinxVerbatimInput}

\begin{sphinxuseclass}{cell_input}
\begin{sphinxVerbatim}[commandchars=\\\{\}]
\PYG{n}{np}\PYG{o}{.}\PYG{n}{random}\PYG{o}{.}\PYG{n}{seed}\PYG{p}{(}\PYG{l+m+mi}{42}\PYG{p}{)}
\PYG{n}{data} \PYG{o}{=} \PYG{n}{np}\PYG{o}{.}\PYG{n}{random}\PYG{o}{.}\PYG{n}{randn}\PYG{p}{(}\PYG{l+m+mi}{2}\PYG{p}{,} \PYG{l+m+mi}{3}\PYG{p}{)}
\PYG{n}{data}
\end{sphinxVerbatim}

\end{sphinxuseclass}\end{sphinxVerbatimInput}
\begin{sphinxVerbatimOutput}

\begin{sphinxuseclass}{cell_output}
\begin{sphinxVerbatim}[commandchars=\\\{\}]
array([[ 0.4967, \PYGZhy{}0.1383,  0.6477],
       [ 1.523 , \PYGZhy{}0.2342, \PYGZhy{}0.2341]])
\end{sphinxVerbatim}

\end{sphinxuseclass}\end{sphinxVerbatimOutput}

\end{sphinxuseclass}
\sphinxAtStartPar
Multiplying \sphinxcode{\sphinxupquote{data}} by 10 multiplies each element in \sphinxcode{\sphinxupquote{data}} by 10, and adding \sphinxcode{\sphinxupquote{data}} to itself adds each element to itself (i.e., element\sphinxhyphen{}wise addition).
To achieve this common\sphinxhyphen{}sense behavior, NumPy arrays must contain homogeneous data types (e.g., all floats or all integers).

\begin{sphinxuseclass}{cell}\begin{sphinxVerbatimInput}

\begin{sphinxuseclass}{cell_input}
\begin{sphinxVerbatim}[commandchars=\\\{\}]
\PYG{n}{data} \PYG{o}{*} \PYG{l+m+mi}{10}
\end{sphinxVerbatim}

\end{sphinxuseclass}\end{sphinxVerbatimInput}
\begin{sphinxVerbatimOutput}

\begin{sphinxuseclass}{cell_output}
\begin{sphinxVerbatim}[commandchars=\\\{\}]
array([[ 4.9671, \PYGZhy{}1.3826,  6.4769],
       [15.2303, \PYGZhy{}2.3415, \PYGZhy{}2.3414]])
\end{sphinxVerbatim}

\end{sphinxuseclass}\end{sphinxVerbatimOutput}

\end{sphinxuseclass}
\begin{sphinxuseclass}{cell}\begin{sphinxVerbatimInput}

\begin{sphinxuseclass}{cell_input}
\begin{sphinxVerbatim}[commandchars=\\\{\}]
\PYG{n}{data\PYGZus{}2} \PYG{o}{=} \PYG{n}{data} \PYG{o}{+} \PYG{n}{data}
\PYG{n}{data\PYGZus{}2}
\end{sphinxVerbatim}

\end{sphinxuseclass}\end{sphinxVerbatimInput}
\begin{sphinxVerbatimOutput}

\begin{sphinxuseclass}{cell_output}
\begin{sphinxVerbatim}[commandchars=\\\{\}]
array([[ 0.9934, \PYGZhy{}0.2765,  1.2954],
       [ 3.0461, \PYGZhy{}0.4683, \PYGZhy{}0.4683]])
\end{sphinxVerbatim}

\end{sphinxuseclass}\end{sphinxVerbatimOutput}

\end{sphinxuseclass}
\sphinxAtStartPar
NumPy arrays have attributes.
Recall that Jupyter Notebooks provides tab completion.

\begin{sphinxuseclass}{cell}\begin{sphinxVerbatimInput}

\begin{sphinxuseclass}{cell_input}
\begin{sphinxVerbatim}[commandchars=\\\{\}]
\PYG{n}{data}\PYG{o}{.}\PYG{n}{ndim}
\end{sphinxVerbatim}

\end{sphinxuseclass}\end{sphinxVerbatimInput}
\begin{sphinxVerbatimOutput}

\begin{sphinxuseclass}{cell_output}
\begin{sphinxVerbatim}[commandchars=\\\{\}]
2
\end{sphinxVerbatim}

\end{sphinxuseclass}\end{sphinxVerbatimOutput}

\end{sphinxuseclass}
\begin{sphinxuseclass}{cell}\begin{sphinxVerbatimInput}

\begin{sphinxuseclass}{cell_input}
\begin{sphinxVerbatim}[commandchars=\\\{\}]
\PYG{n}{data}\PYG{o}{.}\PYG{n}{shape}
\end{sphinxVerbatim}

\end{sphinxuseclass}\end{sphinxVerbatimInput}
\begin{sphinxVerbatimOutput}

\begin{sphinxuseclass}{cell_output}
\begin{sphinxVerbatim}[commandchars=\\\{\}]
(2, 3)
\end{sphinxVerbatim}

\end{sphinxuseclass}\end{sphinxVerbatimOutput}

\end{sphinxuseclass}
\begin{sphinxuseclass}{cell}\begin{sphinxVerbatimInput}

\begin{sphinxuseclass}{cell_input}
\begin{sphinxVerbatim}[commandchars=\\\{\}]
\PYG{n}{data}\PYG{o}{.}\PYG{n}{dtype}
\end{sphinxVerbatim}

\end{sphinxuseclass}\end{sphinxVerbatimInput}
\begin{sphinxVerbatimOutput}

\begin{sphinxuseclass}{cell_output}
\begin{sphinxVerbatim}[commandchars=\\\{\}]
dtype(\PYGZsq{}float64\PYGZsq{})
\end{sphinxVerbatim}

\end{sphinxuseclass}\end{sphinxVerbatimOutput}

\end{sphinxuseclass}
\sphinxAtStartPar
We access or slice elements in a NumPy array using \sphinxcode{\sphinxupquote{{[}{]}}}, the same as we slice lists and tuples.

\begin{sphinxuseclass}{cell}\begin{sphinxVerbatimInput}

\begin{sphinxuseclass}{cell_input}
\begin{sphinxVerbatim}[commandchars=\\\{\}]
\PYG{n}{data}\PYG{p}{[}\PYG{l+m+mi}{0}\PYG{p}{]}
\end{sphinxVerbatim}

\end{sphinxuseclass}\end{sphinxVerbatimInput}
\begin{sphinxVerbatimOutput}

\begin{sphinxuseclass}{cell_output}
\begin{sphinxVerbatim}[commandchars=\\\{\}]
array([ 0.4967, \PYGZhy{}0.1383,  0.6477])
\end{sphinxVerbatim}

\end{sphinxuseclass}\end{sphinxVerbatimOutput}

\end{sphinxuseclass}
\sphinxAtStartPar
As with list and tuples, we can chain \sphinxcode{\sphinxupquote{{[}{]}}}s.

\begin{sphinxuseclass}{cell}\begin{sphinxVerbatimInput}

\begin{sphinxuseclass}{cell_input}
\begin{sphinxVerbatim}[commandchars=\\\{\}]
\PYG{n}{data}\PYG{p}{[}\PYG{l+m+mi}{0}\PYG{p}{]}\PYG{p}{[}\PYG{l+m+mi}{0}\PYG{p}{]}
\end{sphinxVerbatim}

\end{sphinxuseclass}\end{sphinxVerbatimInput}
\begin{sphinxVerbatimOutput}

\begin{sphinxuseclass}{cell_output}
\begin{sphinxVerbatim}[commandchars=\\\{\}]
0.4967
\end{sphinxVerbatim}

\end{sphinxuseclass}\end{sphinxVerbatimOutput}

\end{sphinxuseclass}
\sphinxAtStartPar
However, with NumPy arrays. we can replace \$n\$ chained \sphinxcode{\sphinxupquote{{[}{]}}}s with one pair of \sphinxcode{\sphinxupquote{{[}{]}}}s containing \$n\$ values separated by commas.
For example, \sphinxcode{\sphinxupquote{{[}i{]}{[}j{]}}} becomes \sphinxcode{\sphinxupquote{{[}i, j{]}}}, \sphinxcode{\sphinxupquote{{[}i{]}{[}j{]}{[}k{]}}} becomes \sphinxcode{\sphinxupquote{{[}i, j, k{]}}}.
This abbreviated notation is similar to what you see in your math and econometrics courses.

\begin{sphinxuseclass}{cell}\begin{sphinxVerbatimInput}

\begin{sphinxuseclass}{cell_input}
\begin{sphinxVerbatim}[commandchars=\\\{\}]
\PYG{n}{data}\PYG{p}{[}\PYG{l+m+mi}{0}\PYG{p}{,} \PYG{l+m+mi}{0}\PYG{p}{]} \PYG{c+c1}{\PYGZsh{} zero row, zero column}
\end{sphinxVerbatim}

\end{sphinxuseclass}\end{sphinxVerbatimInput}
\begin{sphinxVerbatimOutput}

\begin{sphinxuseclass}{cell_output}
\begin{sphinxVerbatim}[commandchars=\\\{\}]
0.4967
\end{sphinxVerbatim}

\end{sphinxuseclass}\end{sphinxVerbatimOutput}

\end{sphinxuseclass}
\begin{sphinxuseclass}{cell}\begin{sphinxVerbatimInput}

\begin{sphinxuseclass}{cell_input}
\begin{sphinxVerbatim}[commandchars=\\\{\}]
\PYG{n}{data}\PYG{p}{[}\PYG{l+m+mi}{0}\PYG{p}{]}\PYG{p}{[}\PYG{l+m+mi}{0}\PYG{p}{]} \PYG{o}{==} \PYG{n}{data}\PYG{p}{[}\PYG{l+m+mi}{0}\PYG{p}{,} \PYG{l+m+mi}{0}\PYG{p}{]}
\end{sphinxVerbatim}

\end{sphinxuseclass}\end{sphinxVerbatimInput}
\begin{sphinxVerbatimOutput}

\begin{sphinxuseclass}{cell_output}
\begin{sphinxVerbatim}[commandchars=\\\{\}]
True
\end{sphinxVerbatim}

\end{sphinxuseclass}\end{sphinxVerbatimOutput}

\end{sphinxuseclass}

\subsection{Creating ndarrays}
\label{\detokenize{mckinney_04_lecture:creating-ndarrays}}\begin{quote}

\sphinxAtStartPar
The easiest way to create an array is to use the array function. This accepts any sequence\sphinxhyphen{}like object (including other arrays) and produces a new NumPy array containing the passed data
\end{quote}

\begin{sphinxuseclass}{cell}\begin{sphinxVerbatimInput}

\begin{sphinxuseclass}{cell_input}
\begin{sphinxVerbatim}[commandchars=\\\{\}]
\PYG{n}{data1} \PYG{o}{=} \PYG{p}{[}\PYG{l+m+mi}{6}\PYG{p}{,} \PYG{l+m+mf}{7.5}\PYG{p}{,} \PYG{l+m+mi}{8}\PYG{p}{,} \PYG{l+m+mi}{0}\PYG{p}{,} \PYG{l+m+mi}{1}\PYG{p}{]}
\PYG{n}{arr1} \PYG{o}{=} \PYG{n}{np}\PYG{o}{.}\PYG{n}{array}\PYG{p}{(}\PYG{n}{data1}\PYG{p}{)}
\PYG{n}{arr1}
\end{sphinxVerbatim}

\end{sphinxuseclass}\end{sphinxVerbatimInput}
\begin{sphinxVerbatimOutput}

\begin{sphinxuseclass}{cell_output}
\begin{sphinxVerbatim}[commandchars=\\\{\}]
array([6. , 7.5, 8. , 0. , 1. ])
\end{sphinxVerbatim}

\end{sphinxuseclass}\end{sphinxVerbatimOutput}

\end{sphinxuseclass}
\begin{sphinxuseclass}{cell}\begin{sphinxVerbatimInput}

\begin{sphinxuseclass}{cell_input}
\begin{sphinxVerbatim}[commandchars=\\\{\}]
\PYG{n}{arr1}\PYG{o}{.}\PYG{n}{dtype}
\end{sphinxVerbatim}

\end{sphinxuseclass}\end{sphinxVerbatimInput}
\begin{sphinxVerbatimOutput}

\begin{sphinxuseclass}{cell_output}
\begin{sphinxVerbatim}[commandchars=\\\{\}]
dtype(\PYGZsq{}float64\PYGZsq{})
\end{sphinxVerbatim}

\end{sphinxuseclass}\end{sphinxVerbatimOutput}

\end{sphinxuseclass}
\sphinxAtStartPar
Here \sphinxcode{\sphinxupquote{np.array()}} casts the values in \sphinxcode{\sphinxupquote{data1}} to floats becuase NumPy arrays must have homogenous data types.
We could coerce these values to integers but would lose information.

\begin{sphinxuseclass}{cell}\begin{sphinxVerbatimInput}

\begin{sphinxuseclass}{cell_input}
\begin{sphinxVerbatim}[commandchars=\\\{\}]
\PYG{n}{np}\PYG{o}{.}\PYG{n}{array}\PYG{p}{(}\PYG{n}{data1}\PYG{p}{,} \PYG{n}{dtype}\PYG{o}{=}\PYG{n}{np}\PYG{o}{.}\PYG{n}{int64}\PYG{p}{)}
\end{sphinxVerbatim}

\end{sphinxuseclass}\end{sphinxVerbatimInput}
\begin{sphinxVerbatimOutput}

\begin{sphinxuseclass}{cell_output}
\begin{sphinxVerbatim}[commandchars=\\\{\}]
array([6, 7, 8, 0, 1])
\end{sphinxVerbatim}

\end{sphinxuseclass}\end{sphinxVerbatimOutput}

\end{sphinxuseclass}
\sphinxAtStartPar
We can coerce or cast a list\sphinxhyphen{}of\sphinxhyphen{}lists to a two\sphinxhyphen{}dimensional NumPy array.

\begin{sphinxuseclass}{cell}\begin{sphinxVerbatimInput}

\begin{sphinxuseclass}{cell_input}
\begin{sphinxVerbatim}[commandchars=\\\{\}]
\PYG{n}{data2} \PYG{o}{=} \PYG{p}{[}\PYG{p}{[}\PYG{l+m+mi}{1}\PYG{p}{,} \PYG{l+m+mi}{2}\PYG{p}{,} \PYG{l+m+mi}{3}\PYG{p}{,} \PYG{l+m+mi}{4}\PYG{p}{]}\PYG{p}{,} \PYG{p}{[}\PYG{l+m+mi}{5}\PYG{p}{,} \PYG{l+m+mi}{6}\PYG{p}{,} \PYG{l+m+mi}{7}\PYG{p}{,} \PYG{l+m+mi}{8}\PYG{p}{]}\PYG{p}{]}
\PYG{n}{arr2} \PYG{o}{=} \PYG{n}{np}\PYG{o}{.}\PYG{n}{array}\PYG{p}{(}\PYG{n}{data2}\PYG{p}{)}
\PYG{n}{arr2}
\end{sphinxVerbatim}

\end{sphinxuseclass}\end{sphinxVerbatimInput}
\begin{sphinxVerbatimOutput}

\begin{sphinxuseclass}{cell_output}
\begin{sphinxVerbatim}[commandchars=\\\{\}]
array([[1, 2, 3, 4],
       [5, 6, 7, 8]])
\end{sphinxVerbatim}

\end{sphinxuseclass}\end{sphinxVerbatimOutput}

\end{sphinxuseclass}
\begin{sphinxuseclass}{cell}\begin{sphinxVerbatimInput}

\begin{sphinxuseclass}{cell_input}
\begin{sphinxVerbatim}[commandchars=\\\{\}]
\PYG{n}{arr2}\PYG{o}{.}\PYG{n}{ndim}
\end{sphinxVerbatim}

\end{sphinxuseclass}\end{sphinxVerbatimInput}
\begin{sphinxVerbatimOutput}

\begin{sphinxuseclass}{cell_output}
\begin{sphinxVerbatim}[commandchars=\\\{\}]
2
\end{sphinxVerbatim}

\end{sphinxuseclass}\end{sphinxVerbatimOutput}

\end{sphinxuseclass}
\begin{sphinxuseclass}{cell}\begin{sphinxVerbatimInput}

\begin{sphinxuseclass}{cell_input}
\begin{sphinxVerbatim}[commandchars=\\\{\}]
\PYG{n}{arr2}\PYG{o}{.}\PYG{n}{shape}
\end{sphinxVerbatim}

\end{sphinxuseclass}\end{sphinxVerbatimInput}
\begin{sphinxVerbatimOutput}

\begin{sphinxuseclass}{cell_output}
\begin{sphinxVerbatim}[commandchars=\\\{\}]
(2, 4)
\end{sphinxVerbatim}

\end{sphinxuseclass}\end{sphinxVerbatimOutput}

\end{sphinxuseclass}
\begin{sphinxuseclass}{cell}\begin{sphinxVerbatimInput}

\begin{sphinxuseclass}{cell_input}
\begin{sphinxVerbatim}[commandchars=\\\{\}]
\PYG{n}{arr2}\PYG{o}{.}\PYG{n}{dtype}
\end{sphinxVerbatim}

\end{sphinxuseclass}\end{sphinxVerbatimInput}
\begin{sphinxVerbatimOutput}

\begin{sphinxuseclass}{cell_output}
\begin{sphinxVerbatim}[commandchars=\\\{\}]
dtype(\PYGZsq{}int64\PYGZsq{})
\end{sphinxVerbatim}

\end{sphinxuseclass}\end{sphinxVerbatimOutput}

\end{sphinxuseclass}
\sphinxAtStartPar
There are several other ways to create NumPy arrays.

\begin{sphinxuseclass}{cell}\begin{sphinxVerbatimInput}

\begin{sphinxuseclass}{cell_input}
\begin{sphinxVerbatim}[commandchars=\\\{\}]
\PYG{n}{np}\PYG{o}{.}\PYG{n}{zeros}\PYG{p}{(}\PYG{l+m+mi}{10}\PYG{p}{)}
\end{sphinxVerbatim}

\end{sphinxuseclass}\end{sphinxVerbatimInput}
\begin{sphinxVerbatimOutput}

\begin{sphinxuseclass}{cell_output}
\begin{sphinxVerbatim}[commandchars=\\\{\}]
array([0., 0., 0., 0., 0., 0., 0., 0., 0., 0.])
\end{sphinxVerbatim}

\end{sphinxuseclass}\end{sphinxVerbatimOutput}

\end{sphinxuseclass}
\begin{sphinxuseclass}{cell}\begin{sphinxVerbatimInput}

\begin{sphinxuseclass}{cell_input}
\begin{sphinxVerbatim}[commandchars=\\\{\}]
\PYG{n}{np}\PYG{o}{.}\PYG{n}{zeros}\PYG{p}{(}\PYG{p}{(}\PYG{l+m+mi}{3}\PYG{p}{,} \PYG{l+m+mi}{6}\PYG{p}{)}\PYG{p}{)}
\end{sphinxVerbatim}

\end{sphinxuseclass}\end{sphinxVerbatimInput}
\begin{sphinxVerbatimOutput}

\begin{sphinxuseclass}{cell_output}
\begin{sphinxVerbatim}[commandchars=\\\{\}]
array([[0., 0., 0., 0., 0., 0.],
       [0., 0., 0., 0., 0., 0.],
       [0., 0., 0., 0., 0., 0.]])
\end{sphinxVerbatim}

\end{sphinxuseclass}\end{sphinxVerbatimOutput}

\end{sphinxuseclass}
\sphinxAtStartPar
The \sphinxcode{\sphinxupquote{np.arange()}} function is similar to Python’s built\sphinxhyphen{}in \sphinxcode{\sphinxupquote{range()}} but creates an array directly.

\begin{sphinxuseclass}{cell}\begin{sphinxVerbatimInput}

\begin{sphinxuseclass}{cell_input}
\begin{sphinxVerbatim}[commandchars=\\\{\}]
\PYG{n+nb}{list}\PYG{p}{(}\PYG{n+nb}{range}\PYG{p}{(}\PYG{l+m+mi}{15}\PYG{p}{)}\PYG{p}{)}
\end{sphinxVerbatim}

\end{sphinxuseclass}\end{sphinxVerbatimInput}
\begin{sphinxVerbatimOutput}

\begin{sphinxuseclass}{cell_output}
\begin{sphinxVerbatim}[commandchars=\\\{\}]
[0, 1, 2, 3, 4, 5, 6, 7, 8, 9, 10, 11, 12, 13, 14]
\end{sphinxVerbatim}

\end{sphinxuseclass}\end{sphinxVerbatimOutput}

\end{sphinxuseclass}
\begin{sphinxuseclass}{cell}\begin{sphinxVerbatimInput}

\begin{sphinxuseclass}{cell_input}
\begin{sphinxVerbatim}[commandchars=\\\{\}]
\PYG{n}{np}\PYG{o}{.}\PYG{n}{array}\PYG{p}{(}\PYG{n+nb}{range}\PYG{p}{(}\PYG{l+m+mi}{15}\PYG{p}{)}\PYG{p}{)}
\end{sphinxVerbatim}

\end{sphinxuseclass}\end{sphinxVerbatimInput}
\begin{sphinxVerbatimOutput}

\begin{sphinxuseclass}{cell_output}
\begin{sphinxVerbatim}[commandchars=\\\{\}]
array([ 0,  1,  2,  3,  4,  5,  6,  7,  8,  9, 10, 11, 12, 13, 14])
\end{sphinxVerbatim}

\end{sphinxuseclass}\end{sphinxVerbatimOutput}

\end{sphinxuseclass}
\begin{sphinxuseclass}{cell}\begin{sphinxVerbatimInput}

\begin{sphinxuseclass}{cell_input}
\begin{sphinxVerbatim}[commandchars=\\\{\}]
\PYG{n}{np}\PYG{o}{.}\PYG{n}{arange}\PYG{p}{(}\PYG{l+m+mi}{15}\PYG{p}{)}
\end{sphinxVerbatim}

\end{sphinxuseclass}\end{sphinxVerbatimInput}
\begin{sphinxVerbatimOutput}

\begin{sphinxuseclass}{cell_output}
\begin{sphinxVerbatim}[commandchars=\\\{\}]
array([ 0,  1,  2,  3,  4,  5,  6,  7,  8,  9, 10, 11, 12, 13, 14])
\end{sphinxVerbatim}

\end{sphinxuseclass}\end{sphinxVerbatimOutput}

\end{sphinxuseclass}
\sphinxAtStartPar
\sphinxstyleemphasis{\sphinxstylestrong{Table 4\sphinxhyphen{}1}} from McKinney lists some NumPy array creation functions.
\begin{itemize}
\item {} 
\sphinxAtStartPar
\sphinxcode{\sphinxupquote{array}}: Convert input data (list, tuple, array, or other sequence type) to an ndarray either by inferring a dtype or explicitly specifying a dtype; copies the input data by default

\item {} 
\sphinxAtStartPar
\sphinxcode{\sphinxupquote{asarray}}:  Convert input to ndarray, but do not copy if the input is already an ndarray

\item {} 
\sphinxAtStartPar
\sphinxcode{\sphinxupquote{arange}}:  Like the built\sphinxhyphen{}in range but returns an ndarray instead of a list

\item {} 
\sphinxAtStartPar
\sphinxcode{\sphinxupquote{ones}}, \sphinxcode{\sphinxupquote{ones\_like}}:  Produce an array of all 1s with the given shape and dtype; \sphinxcode{\sphinxupquote{ones\_like}} takes another array and produces a \sphinxcode{\sphinxupquote{ones}} array of the \sphinxhyphen{} same shape and dtype

\item {} 
\sphinxAtStartPar
\sphinxcode{\sphinxupquote{zeros}}, \sphinxcode{\sphinxupquote{zeros\_like}}:  Like \sphinxcode{\sphinxupquote{ones}} and \sphinxcode{\sphinxupquote{ones\_like}} but producing arrays of 0s instead

\item {} 
\sphinxAtStartPar
\sphinxcode{\sphinxupquote{empty}}, \sphinxcode{\sphinxupquote{empty\_like}}:  Create new arrays by allocating new memory, but do not populate with any values like ones and zeros

\item {} 
\sphinxAtStartPar
\sphinxcode{\sphinxupquote{full}}, \sphinxcode{\sphinxupquote{full\_like}}:  Produce an array of the given shape and dtype with all values set to the indicated “fill value”

\item {} 
\sphinxAtStartPar
\sphinxcode{\sphinxupquote{eye}}, \sphinxcode{\sphinxupquote{identity}}:  Create a square N\sphinxhyphen{}by\sphinxhyphen{}N identity matrix (1s on the diagonal and 0s elsewhere)

\end{itemize}


\subsection{Arithmetic with NumPy Arrays}
\label{\detokenize{mckinney_04_lecture:arithmetic-with-numpy-arrays}}\begin{quote}

\sphinxAtStartPar
Arrays are important because they enable you to express batch operations on data without writing any for loops. NumPy users call this vectorization. Any arithmetic operations between equal\sphinxhyphen{}size arrays applies the operation element\sphinxhyphen{}wise
\end{quote}

\begin{sphinxuseclass}{cell}\begin{sphinxVerbatimInput}

\begin{sphinxuseclass}{cell_input}
\begin{sphinxVerbatim}[commandchars=\\\{\}]
\PYG{n}{arr} \PYG{o}{=} \PYG{n}{np}\PYG{o}{.}\PYG{n}{array}\PYG{p}{(}\PYG{p}{[}\PYG{p}{[}\PYG{l+m+mf}{1.}\PYG{p}{,} \PYG{l+m+mf}{2.}\PYG{p}{,} \PYG{l+m+mf}{3.}\PYG{p}{]}\PYG{p}{,} \PYG{p}{[}\PYG{l+m+mf}{4.}\PYG{p}{,} \PYG{l+m+mf}{5.}\PYG{p}{,} \PYG{l+m+mf}{6.}\PYG{p}{]}\PYG{p}{]}\PYG{p}{)}
\PYG{n}{arr}
\end{sphinxVerbatim}

\end{sphinxuseclass}\end{sphinxVerbatimInput}
\begin{sphinxVerbatimOutput}

\begin{sphinxuseclass}{cell_output}
\begin{sphinxVerbatim}[commandchars=\\\{\}]
array([[1., 2., 3.],
       [4., 5., 6.]])
\end{sphinxVerbatim}

\end{sphinxuseclass}\end{sphinxVerbatimOutput}

\end{sphinxuseclass}
\begin{sphinxuseclass}{cell}\begin{sphinxVerbatimInput}

\begin{sphinxuseclass}{cell_input}
\begin{sphinxVerbatim}[commandchars=\\\{\}]
\PYG{n}{arr}\PYG{o}{.}\PYG{n}{shape}
\end{sphinxVerbatim}

\end{sphinxuseclass}\end{sphinxVerbatimInput}
\begin{sphinxVerbatimOutput}

\begin{sphinxuseclass}{cell_output}
\begin{sphinxVerbatim}[commandchars=\\\{\}]
(2, 3)
\end{sphinxVerbatim}

\end{sphinxuseclass}\end{sphinxVerbatimOutput}

\end{sphinxuseclass}
\sphinxAtStartPar
NumPy array addition is elementwise.

\begin{sphinxuseclass}{cell}\begin{sphinxVerbatimInput}

\begin{sphinxuseclass}{cell_input}
\begin{sphinxVerbatim}[commandchars=\\\{\}]
\PYG{n}{arr} \PYG{o}{+} \PYG{n}{arr}
\end{sphinxVerbatim}

\end{sphinxuseclass}\end{sphinxVerbatimInput}
\begin{sphinxVerbatimOutput}

\begin{sphinxuseclass}{cell_output}
\begin{sphinxVerbatim}[commandchars=\\\{\}]
array([[ 2.,  4.,  6.],
       [ 8., 10., 12.]])
\end{sphinxVerbatim}

\end{sphinxuseclass}\end{sphinxVerbatimOutput}

\end{sphinxuseclass}
\sphinxAtStartPar
NumPy array multiplication is elementwise.

\begin{sphinxuseclass}{cell}\begin{sphinxVerbatimInput}

\begin{sphinxuseclass}{cell_input}
\begin{sphinxVerbatim}[commandchars=\\\{\}]
\PYG{n}{arr} \PYG{o}{*} \PYG{n}{arr}
\end{sphinxVerbatim}

\end{sphinxuseclass}\end{sphinxVerbatimInput}
\begin{sphinxVerbatimOutput}

\begin{sphinxuseclass}{cell_output}
\begin{sphinxVerbatim}[commandchars=\\\{\}]
array([[ 1.,  4.,  9.],
       [16., 25., 36.]])
\end{sphinxVerbatim}

\end{sphinxuseclass}\end{sphinxVerbatimOutput}

\end{sphinxuseclass}
\sphinxAtStartPar
NumPy array division is elementwise.

\begin{sphinxuseclass}{cell}\begin{sphinxVerbatimInput}

\begin{sphinxuseclass}{cell_input}
\begin{sphinxVerbatim}[commandchars=\\\{\}]
\PYG{l+m+mi}{1} \PYG{o}{/} \PYG{n}{arr}
\end{sphinxVerbatim}

\end{sphinxuseclass}\end{sphinxVerbatimInput}
\begin{sphinxVerbatimOutput}

\begin{sphinxuseclass}{cell_output}
\begin{sphinxVerbatim}[commandchars=\\\{\}]
array([[1.    , 0.5   , 0.3333],
       [0.25  , 0.2   , 0.1667]])
\end{sphinxVerbatim}

\end{sphinxuseclass}\end{sphinxVerbatimOutput}

\end{sphinxuseclass}
\sphinxAtStartPar
NumPy powers are elementwise, too.

\begin{sphinxuseclass}{cell}\begin{sphinxVerbatimInput}

\begin{sphinxuseclass}{cell_input}
\begin{sphinxVerbatim}[commandchars=\\\{\}]
\PYG{n}{arr} \PYG{o}{*}\PYG{o}{*} \PYG{l+m+mi}{2}
\end{sphinxVerbatim}

\end{sphinxuseclass}\end{sphinxVerbatimInput}
\begin{sphinxVerbatimOutput}

\begin{sphinxuseclass}{cell_output}
\begin{sphinxVerbatim}[commandchars=\\\{\}]
array([[ 1.,  4.,  9.],
       [16., 25., 36.]])
\end{sphinxVerbatim}

\end{sphinxuseclass}\end{sphinxVerbatimOutput}

\end{sphinxuseclass}
\sphinxAtStartPar
We can also raise a single value to an array!

\begin{sphinxuseclass}{cell}\begin{sphinxVerbatimInput}

\begin{sphinxuseclass}{cell_input}
\begin{sphinxVerbatim}[commandchars=\\\{\}]
\PYG{l+m+mi}{2} \PYG{o}{*}\PYG{o}{*} \PYG{n}{arr}
\end{sphinxVerbatim}

\end{sphinxuseclass}\end{sphinxVerbatimInput}
\begin{sphinxVerbatimOutput}

\begin{sphinxuseclass}{cell_output}
\begin{sphinxVerbatim}[commandchars=\\\{\}]
array([[ 2.,  4.,  8.],
       [16., 32., 64.]])
\end{sphinxVerbatim}

\end{sphinxuseclass}\end{sphinxVerbatimOutput}

\end{sphinxuseclass}

\subsection{Basic Indexing and Slicing}
\label{\detokenize{mckinney_04_lecture:basic-indexing-and-slicing}}
\sphinxAtStartPar
One\sphinxhyphen{}dimensional array index and slice the same as lists.

\begin{sphinxuseclass}{cell}\begin{sphinxVerbatimInput}

\begin{sphinxuseclass}{cell_input}
\begin{sphinxVerbatim}[commandchars=\\\{\}]
\PYG{n}{arr} \PYG{o}{=} \PYG{n}{np}\PYG{o}{.}\PYG{n}{arange}\PYG{p}{(}\PYG{l+m+mi}{10}\PYG{p}{)}
\PYG{n}{arr}
\end{sphinxVerbatim}

\end{sphinxuseclass}\end{sphinxVerbatimInput}
\begin{sphinxVerbatimOutput}

\begin{sphinxuseclass}{cell_output}
\begin{sphinxVerbatim}[commandchars=\\\{\}]
array([0, 1, 2, 3, 4, 5, 6, 7, 8, 9])
\end{sphinxVerbatim}

\end{sphinxuseclass}\end{sphinxVerbatimOutput}

\end{sphinxuseclass}
\begin{sphinxuseclass}{cell}\begin{sphinxVerbatimInput}

\begin{sphinxuseclass}{cell_input}
\begin{sphinxVerbatim}[commandchars=\\\{\}]
\PYG{n}{arr}\PYG{p}{[}\PYG{l+m+mi}{5}\PYG{p}{]}
\end{sphinxVerbatim}

\end{sphinxuseclass}\end{sphinxVerbatimInput}
\begin{sphinxVerbatimOutput}

\begin{sphinxuseclass}{cell_output}
\begin{sphinxVerbatim}[commandchars=\\\{\}]
5
\end{sphinxVerbatim}

\end{sphinxuseclass}\end{sphinxVerbatimOutput}

\end{sphinxuseclass}
\begin{sphinxuseclass}{cell}\begin{sphinxVerbatimInput}

\begin{sphinxuseclass}{cell_input}
\begin{sphinxVerbatim}[commandchars=\\\{\}]
\PYG{n}{arr}\PYG{p}{[}\PYG{l+m+mi}{5}\PYG{p}{:}\PYG{l+m+mi}{8}\PYG{p}{]}
\end{sphinxVerbatim}

\end{sphinxuseclass}\end{sphinxVerbatimInput}
\begin{sphinxVerbatimOutput}

\begin{sphinxuseclass}{cell_output}
\begin{sphinxVerbatim}[commandchars=\\\{\}]
array([5, 6, 7])
\end{sphinxVerbatim}

\end{sphinxuseclass}\end{sphinxVerbatimOutput}

\end{sphinxuseclass}
\begin{sphinxuseclass}{cell}\begin{sphinxVerbatimInput}

\begin{sphinxuseclass}{cell_input}
\begin{sphinxVerbatim}[commandchars=\\\{\}]
\PYG{n}{equiv\PYGZus{}list} \PYG{o}{=} \PYG{n+nb}{list}\PYG{p}{(}\PYG{n+nb}{range}\PYG{p}{(}\PYG{l+m+mi}{10}\PYG{p}{)}\PYG{p}{)}
\PYG{n}{equiv\PYGZus{}list}
\end{sphinxVerbatim}

\end{sphinxuseclass}\end{sphinxVerbatimInput}
\begin{sphinxVerbatimOutput}

\begin{sphinxuseclass}{cell_output}
\begin{sphinxVerbatim}[commandchars=\\\{\}]
[0, 1, 2, 3, 4, 5, 6, 7, 8, 9]
\end{sphinxVerbatim}

\end{sphinxuseclass}\end{sphinxVerbatimOutput}

\end{sphinxuseclass}
\begin{sphinxuseclass}{cell}\begin{sphinxVerbatimInput}

\begin{sphinxuseclass}{cell_input}
\begin{sphinxVerbatim}[commandchars=\\\{\}]
\PYG{n}{equiv\PYGZus{}list}\PYG{p}{[}\PYG{l+m+mi}{5}\PYG{p}{:}\PYG{l+m+mi}{8}\PYG{p}{]}
\end{sphinxVerbatim}

\end{sphinxuseclass}\end{sphinxVerbatimInput}
\begin{sphinxVerbatimOutput}

\begin{sphinxuseclass}{cell_output}
\begin{sphinxVerbatim}[commandchars=\\\{\}]
[5, 6, 7]
\end{sphinxVerbatim}

\end{sphinxuseclass}\end{sphinxVerbatimOutput}

\end{sphinxuseclass}
\sphinxAtStartPar
We have to jump through some hoops if we want to replace elements 5, 6, and 7 in \sphinxcode{\sphinxupquote{equiv\_list}} with the value 12.

\begin{sphinxuseclass}{cell}\begin{sphinxVerbatimInput}

\begin{sphinxuseclass}{cell_input}
\begin{sphinxVerbatim}[commandchars=\\\{\}]
\PYG{c+c1}{\PYGZsh{} TypeError: can only assign an iterable}
\PYG{c+c1}{\PYGZsh{} equiv\PYGZus{}list[5:8] = 12}
\end{sphinxVerbatim}

\end{sphinxuseclass}\end{sphinxVerbatimInput}

\end{sphinxuseclass}
\begin{sphinxuseclass}{cell}\begin{sphinxVerbatimInput}

\begin{sphinxuseclass}{cell_input}
\begin{sphinxVerbatim}[commandchars=\\\{\}]
\PYG{n}{equiv\PYGZus{}list}\PYG{p}{[}\PYG{l+m+mi}{5}\PYG{p}{:}\PYG{l+m+mi}{8}\PYG{p}{]} \PYG{o}{=} \PYG{p}{[}\PYG{l+m+mi}{12}\PYG{p}{]} \PYG{o}{*} \PYG{l+m+mi}{3}
\PYG{n}{equiv\PYGZus{}list}
\end{sphinxVerbatim}

\end{sphinxuseclass}\end{sphinxVerbatimInput}
\begin{sphinxVerbatimOutput}

\begin{sphinxuseclass}{cell_output}
\begin{sphinxVerbatim}[commandchars=\\\{\}]
[0, 1, 2, 3, 4, 12, 12, 12, 8, 9]
\end{sphinxVerbatim}

\end{sphinxuseclass}\end{sphinxVerbatimOutput}

\end{sphinxuseclass}
\sphinxAtStartPar
However, this operation is easy with the NumPy array \sphinxcode{\sphinxupquote{arr}}!

\begin{sphinxuseclass}{cell}\begin{sphinxVerbatimInput}

\begin{sphinxuseclass}{cell_input}
\begin{sphinxVerbatim}[commandchars=\\\{\}]
\PYG{n}{arr}\PYG{p}{[}\PYG{l+m+mi}{5}\PYG{p}{:}\PYG{l+m+mi}{8}\PYG{p}{]} \PYG{o}{=} \PYG{l+m+mi}{12}
\PYG{n}{arr}
\end{sphinxVerbatim}

\end{sphinxuseclass}\end{sphinxVerbatimInput}
\begin{sphinxVerbatimOutput}

\begin{sphinxuseclass}{cell_output}
\begin{sphinxVerbatim}[commandchars=\\\{\}]
array([ 0,  1,  2,  3,  4, 12, 12, 12,  8,  9])
\end{sphinxVerbatim}

\end{sphinxuseclass}\end{sphinxVerbatimOutput}

\end{sphinxuseclass}
\sphinxAtStartPar
“Broadcasting” is the name for this behavior.
\begin{quote}

\sphinxAtStartPar
As you can see, if you assign a scalar value to a slice, as in \sphinxcode{\sphinxupquote{arr{[}5:8{]} = 12}}, the value is propagated (or broadcasted henceforth) to the entire selection. An important first distinction from Python’s built\sphinxhyphen{}in lists is that array slices are views on the original array. This means that the data is not copied, and any modifications to the view will be reflected in the source array.
\end{quote}

\begin{sphinxuseclass}{cell}\begin{sphinxVerbatimInput}

\begin{sphinxuseclass}{cell_input}
\begin{sphinxVerbatim}[commandchars=\\\{\}]
\PYG{n}{arr\PYGZus{}slice} \PYG{o}{=} \PYG{n}{arr}\PYG{p}{[}\PYG{l+m+mi}{5}\PYG{p}{:}\PYG{l+m+mi}{8}\PYG{p}{]}
\PYG{n}{arr\PYGZus{}slice}
\end{sphinxVerbatim}

\end{sphinxuseclass}\end{sphinxVerbatimInput}
\begin{sphinxVerbatimOutput}

\begin{sphinxuseclass}{cell_output}
\begin{sphinxVerbatim}[commandchars=\\\{\}]
array([12, 12, 12])
\end{sphinxVerbatim}

\end{sphinxuseclass}\end{sphinxVerbatimOutput}

\end{sphinxuseclass}
\begin{sphinxuseclass}{cell}\begin{sphinxVerbatimInput}

\begin{sphinxuseclass}{cell_input}
\begin{sphinxVerbatim}[commandchars=\\\{\}]
\PYG{n}{x} \PYG{o}{=} \PYG{n}{arr\PYGZus{}slice}
\PYG{n}{x}
\end{sphinxVerbatim}

\end{sphinxuseclass}\end{sphinxVerbatimInput}
\begin{sphinxVerbatimOutput}

\begin{sphinxuseclass}{cell_output}
\begin{sphinxVerbatim}[commandchars=\\\{\}]
array([12, 12, 12])
\end{sphinxVerbatim}

\end{sphinxuseclass}\end{sphinxVerbatimOutput}

\end{sphinxuseclass}
\begin{sphinxuseclass}{cell}\begin{sphinxVerbatimInput}

\begin{sphinxuseclass}{cell_input}
\begin{sphinxVerbatim}[commandchars=\\\{\}]
\PYG{n}{x} \PYG{o+ow}{is} \PYG{n}{arr\PYGZus{}slice}
\end{sphinxVerbatim}

\end{sphinxuseclass}\end{sphinxVerbatimInput}
\begin{sphinxVerbatimOutput}

\begin{sphinxuseclass}{cell_output}
\begin{sphinxVerbatim}[commandchars=\\\{\}]
True
\end{sphinxVerbatim}

\end{sphinxuseclass}\end{sphinxVerbatimOutput}

\end{sphinxuseclass}
\begin{sphinxuseclass}{cell}\begin{sphinxVerbatimInput}

\begin{sphinxuseclass}{cell_input}
\begin{sphinxVerbatim}[commandchars=\\\{\}]
\PYG{n}{y} \PYG{o}{=} \PYG{n}{x}\PYG{o}{.}\PYG{n}{copy}\PYG{p}{(}\PYG{p}{)}
\end{sphinxVerbatim}

\end{sphinxuseclass}\end{sphinxVerbatimInput}

\end{sphinxuseclass}
\begin{sphinxuseclass}{cell}\begin{sphinxVerbatimInput}

\begin{sphinxuseclass}{cell_input}
\begin{sphinxVerbatim}[commandchars=\\\{\}]
\PYG{n}{y} \PYG{o+ow}{is} \PYG{n}{arr\PYGZus{}slice}
\end{sphinxVerbatim}

\end{sphinxuseclass}\end{sphinxVerbatimInput}
\begin{sphinxVerbatimOutput}

\begin{sphinxuseclass}{cell_output}
\begin{sphinxVerbatim}[commandchars=\\\{\}]
False
\end{sphinxVerbatim}

\end{sphinxuseclass}\end{sphinxVerbatimOutput}

\end{sphinxuseclass}
\begin{sphinxuseclass}{cell}\begin{sphinxVerbatimInput}

\begin{sphinxuseclass}{cell_input}
\begin{sphinxVerbatim}[commandchars=\\\{\}]
\PYG{n}{arr\PYGZus{}slice}\PYG{p}{[}\PYG{l+m+mi}{1}\PYG{p}{]} \PYG{o}{=} \PYG{l+m+mi}{12345}
\PYG{n}{arr\PYGZus{}slice}
\end{sphinxVerbatim}

\end{sphinxuseclass}\end{sphinxVerbatimInput}
\begin{sphinxVerbatimOutput}

\begin{sphinxuseclass}{cell_output}
\begin{sphinxVerbatim}[commandchars=\\\{\}]
array([   12, 12345,    12])
\end{sphinxVerbatim}

\end{sphinxuseclass}\end{sphinxVerbatimOutput}

\end{sphinxuseclass}
\begin{sphinxuseclass}{cell}\begin{sphinxVerbatimInput}

\begin{sphinxuseclass}{cell_input}
\begin{sphinxVerbatim}[commandchars=\\\{\}]
\PYG{n}{arr}
\end{sphinxVerbatim}

\end{sphinxuseclass}\end{sphinxVerbatimInput}
\begin{sphinxVerbatimOutput}

\begin{sphinxuseclass}{cell_output}
\begin{sphinxVerbatim}[commandchars=\\\{\}]
array([    0,     1,     2,     3,     4,    12, 12345,    12,     8,
           9])
\end{sphinxVerbatim}

\end{sphinxuseclass}\end{sphinxVerbatimOutput}

\end{sphinxuseclass}
\sphinxAtStartPar
The \sphinxcode{\sphinxupquote{:}} slices every element in \sphinxcode{\sphinxupquote{arr\_slice}}.

\begin{sphinxuseclass}{cell}\begin{sphinxVerbatimInput}

\begin{sphinxuseclass}{cell_input}
\begin{sphinxVerbatim}[commandchars=\\\{\}]
\PYG{n}{arr\PYGZus{}slice}\PYG{p}{[}\PYG{p}{:}\PYG{p}{]} \PYG{o}{=} \PYG{l+m+mi}{64}
\PYG{n}{arr\PYGZus{}slice}
\end{sphinxVerbatim}

\end{sphinxuseclass}\end{sphinxVerbatimInput}
\begin{sphinxVerbatimOutput}

\begin{sphinxuseclass}{cell_output}
\begin{sphinxVerbatim}[commandchars=\\\{\}]
array([64, 64, 64])
\end{sphinxVerbatim}

\end{sphinxuseclass}\end{sphinxVerbatimOutput}

\end{sphinxuseclass}
\begin{sphinxuseclass}{cell}\begin{sphinxVerbatimInput}

\begin{sphinxuseclass}{cell_input}
\begin{sphinxVerbatim}[commandchars=\\\{\}]
\PYG{n}{arr}
\end{sphinxVerbatim}

\end{sphinxuseclass}\end{sphinxVerbatimInput}
\begin{sphinxVerbatimOutput}

\begin{sphinxuseclass}{cell_output}
\begin{sphinxVerbatim}[commandchars=\\\{\}]
array([ 0,  1,  2,  3,  4, 64, 64, 64,  8,  9])
\end{sphinxVerbatim}

\end{sphinxuseclass}\end{sphinxVerbatimOutput}

\end{sphinxuseclass}\begin{quote}

\sphinxAtStartPar
If you want a copy of a slice of an ndarray instead of a view, you will need to explicitly copy the array\sphinxhyphen{}for example, \sphinxcode{\sphinxupquote{arr{[}5:8{]}.copy()}}.
\end{quote}

\begin{sphinxuseclass}{cell}\begin{sphinxVerbatimInput}

\begin{sphinxuseclass}{cell_input}
\begin{sphinxVerbatim}[commandchars=\\\{\}]
\PYG{n}{arr\PYGZus{}slice\PYGZus{}2} \PYG{o}{=} \PYG{n}{arr}\PYG{p}{[}\PYG{l+m+mi}{5}\PYG{p}{:}\PYG{l+m+mi}{8}\PYG{p}{]}\PYG{o}{.}\PYG{n}{copy}\PYG{p}{(}\PYG{p}{)}
\PYG{n}{arr\PYGZus{}slice\PYGZus{}2}
\end{sphinxVerbatim}

\end{sphinxuseclass}\end{sphinxVerbatimInput}
\begin{sphinxVerbatimOutput}

\begin{sphinxuseclass}{cell_output}
\begin{sphinxVerbatim}[commandchars=\\\{\}]
array([64, 64, 64])
\end{sphinxVerbatim}

\end{sphinxuseclass}\end{sphinxVerbatimOutput}

\end{sphinxuseclass}
\begin{sphinxuseclass}{cell}\begin{sphinxVerbatimInput}

\begin{sphinxuseclass}{cell_input}
\begin{sphinxVerbatim}[commandchars=\\\{\}]
\PYG{n}{arr\PYGZus{}slice\PYGZus{}2}\PYG{p}{[}\PYG{p}{:}\PYG{p}{]} \PYG{o}{=} \PYG{l+m+mi}{2001}
\PYG{n}{arr\PYGZus{}slice\PYGZus{}2}
\end{sphinxVerbatim}

\end{sphinxuseclass}\end{sphinxVerbatimInput}
\begin{sphinxVerbatimOutput}

\begin{sphinxuseclass}{cell_output}
\begin{sphinxVerbatim}[commandchars=\\\{\}]
array([2001, 2001, 2001])
\end{sphinxVerbatim}

\end{sphinxuseclass}\end{sphinxVerbatimOutput}

\end{sphinxuseclass}
\begin{sphinxuseclass}{cell}\begin{sphinxVerbatimInput}

\begin{sphinxuseclass}{cell_input}
\begin{sphinxVerbatim}[commandchars=\\\{\}]
\PYG{n}{arr}
\end{sphinxVerbatim}

\end{sphinxuseclass}\end{sphinxVerbatimInput}
\begin{sphinxVerbatimOutput}

\begin{sphinxuseclass}{cell_output}
\begin{sphinxVerbatim}[commandchars=\\\{\}]
array([ 0,  1,  2,  3,  4, 64, 64, 64,  8,  9])
\end{sphinxVerbatim}

\end{sphinxuseclass}\end{sphinxVerbatimOutput}

\end{sphinxuseclass}

\subsection{Indexing with slices}
\label{\detokenize{mckinney_04_lecture:indexing-with-slices}}
\sphinxAtStartPar
We can slice across two or more dimensions, including the \sphinxcode{\sphinxupquote{{[}i, j{]}}} notation.

\begin{sphinxuseclass}{cell}\begin{sphinxVerbatimInput}

\begin{sphinxuseclass}{cell_input}
\begin{sphinxVerbatim}[commandchars=\\\{\}]
\PYG{n}{arr2d} \PYG{o}{=} \PYG{n}{np}\PYG{o}{.}\PYG{n}{array}\PYG{p}{(}\PYG{p}{[}\PYG{p}{[}\PYG{l+m+mi}{1}\PYG{p}{,}\PYG{l+m+mi}{2}\PYG{p}{,}\PYG{l+m+mi}{3}\PYG{p}{]}\PYG{p}{,} \PYG{p}{[}\PYG{l+m+mi}{4}\PYG{p}{,}\PYG{l+m+mi}{5}\PYG{p}{,}\PYG{l+m+mi}{6}\PYG{p}{]}\PYG{p}{,} \PYG{p}{[}\PYG{l+m+mi}{7}\PYG{p}{,}\PYG{l+m+mi}{8}\PYG{p}{,}\PYG{l+m+mi}{9}\PYG{p}{]}\PYG{p}{]}\PYG{p}{)}
\PYG{n}{arr2d}
\end{sphinxVerbatim}

\end{sphinxuseclass}\end{sphinxVerbatimInput}
\begin{sphinxVerbatimOutput}

\begin{sphinxuseclass}{cell_output}
\begin{sphinxVerbatim}[commandchars=\\\{\}]
array([[1, 2, 3],
       [4, 5, 6],
       [7, 8, 9]])
\end{sphinxVerbatim}

\end{sphinxuseclass}\end{sphinxVerbatimOutput}

\end{sphinxuseclass}
\begin{sphinxuseclass}{cell}\begin{sphinxVerbatimInput}

\begin{sphinxuseclass}{cell_input}
\begin{sphinxVerbatim}[commandchars=\\\{\}]
\PYG{n}{arr2d}\PYG{p}{[}\PYG{p}{:}\PYG{l+m+mi}{2}\PYG{p}{]}
\end{sphinxVerbatim}

\end{sphinxuseclass}\end{sphinxVerbatimInput}
\begin{sphinxVerbatimOutput}

\begin{sphinxuseclass}{cell_output}
\begin{sphinxVerbatim}[commandchars=\\\{\}]
array([[1, 2, 3],
       [4, 5, 6]])
\end{sphinxVerbatim}

\end{sphinxuseclass}\end{sphinxVerbatimOutput}

\end{sphinxuseclass}
\begin{sphinxuseclass}{cell}\begin{sphinxVerbatimInput}

\begin{sphinxuseclass}{cell_input}
\begin{sphinxVerbatim}[commandchars=\\\{\}]
\PYG{n}{arr2d}\PYG{p}{[}\PYG{p}{:}\PYG{l+m+mi}{2}\PYG{p}{,} \PYG{l+m+mi}{1}\PYG{p}{:}\PYG{p}{]}
\end{sphinxVerbatim}

\end{sphinxuseclass}\end{sphinxVerbatimInput}
\begin{sphinxVerbatimOutput}

\begin{sphinxuseclass}{cell_output}
\begin{sphinxVerbatim}[commandchars=\\\{\}]
array([[2, 3],
       [5, 6]])
\end{sphinxVerbatim}

\end{sphinxuseclass}\end{sphinxVerbatimOutput}

\end{sphinxuseclass}
\sphinxAtStartPar
A colon (\sphinxcode{\sphinxupquote{:}}) by itself selects the entire dimension and is necessary to slice higher dimensions.

\begin{sphinxuseclass}{cell}\begin{sphinxVerbatimInput}

\begin{sphinxuseclass}{cell_input}
\begin{sphinxVerbatim}[commandchars=\\\{\}]
\PYG{n}{arr2d}\PYG{p}{[}\PYG{p}{:}\PYG{p}{,} \PYG{p}{:}\PYG{l+m+mi}{1}\PYG{p}{]}
\end{sphinxVerbatim}

\end{sphinxuseclass}\end{sphinxVerbatimInput}
\begin{sphinxVerbatimOutput}

\begin{sphinxuseclass}{cell_output}
\begin{sphinxVerbatim}[commandchars=\\\{\}]
array([[1],
       [4],
       [7]])
\end{sphinxVerbatim}

\end{sphinxuseclass}\end{sphinxVerbatimOutput}

\end{sphinxuseclass}
\begin{sphinxuseclass}{cell}\begin{sphinxVerbatimInput}

\begin{sphinxuseclass}{cell_input}
\begin{sphinxVerbatim}[commandchars=\\\{\}]
\PYG{n}{arr2d}\PYG{p}{[}\PYG{p}{:}\PYG{l+m+mi}{2}\PYG{p}{,} \PYG{l+m+mi}{1}\PYG{p}{:}\PYG{p}{]} \PYG{o}{=} \PYG{l+m+mi}{0}
\PYG{n}{arr2d}
\end{sphinxVerbatim}

\end{sphinxuseclass}\end{sphinxVerbatimInput}
\begin{sphinxVerbatimOutput}

\begin{sphinxuseclass}{cell_output}
\begin{sphinxVerbatim}[commandchars=\\\{\}]
array([[1, 0, 0],
       [4, 0, 0],
       [7, 8, 9]])
\end{sphinxVerbatim}

\end{sphinxuseclass}\end{sphinxVerbatimOutput}

\end{sphinxuseclass}
\sphinxAtStartPar
Slicing multi\sphinxhyphen{}dimension arrays is tricky!
\sphinxstyleemphasis{\sphinxstylestrong{Always check your output!}}


\subsection{Boolean Indexing}
\label{\detokenize{mckinney_04_lecture:boolean-indexing}}
\sphinxAtStartPar
We can use Booleans (\sphinxcode{\sphinxupquote{True}}s and \sphinxcode{\sphinxupquote{False}}s) to slice arrays, too.
Boolean indexing in Python is like combining \sphinxcode{\sphinxupquote{index()}} and \sphinxcode{\sphinxupquote{match()}} in Excel.

\begin{sphinxuseclass}{cell}\begin{sphinxVerbatimInput}

\begin{sphinxuseclass}{cell_input}
\begin{sphinxVerbatim}[commandchars=\\\{\}]
\PYG{n}{names} \PYG{o}{=} \PYG{n}{np}\PYG{o}{.}\PYG{n}{array}\PYG{p}{(}\PYG{p}{[}\PYG{l+s+s1}{\PYGZsq{}}\PYG{l+s+s1}{Bob}\PYG{l+s+s1}{\PYGZsq{}}\PYG{p}{,} \PYG{l+s+s1}{\PYGZsq{}}\PYG{l+s+s1}{Joe}\PYG{l+s+s1}{\PYGZsq{}}\PYG{p}{,} \PYG{l+s+s1}{\PYGZsq{}}\PYG{l+s+s1}{Will}\PYG{l+s+s1}{\PYGZsq{}}\PYG{p}{,} \PYG{l+s+s1}{\PYGZsq{}}\PYG{l+s+s1}{Bob}\PYG{l+s+s1}{\PYGZsq{}}\PYG{p}{,} \PYG{l+s+s1}{\PYGZsq{}}\PYG{l+s+s1}{Will}\PYG{l+s+s1}{\PYGZsq{}}\PYG{p}{,} \PYG{l+s+s1}{\PYGZsq{}}\PYG{l+s+s1}{Joe}\PYG{l+s+s1}{\PYGZsq{}}\PYG{p}{,} \PYG{l+s+s1}{\PYGZsq{}}\PYG{l+s+s1}{Joe}\PYG{l+s+s1}{\PYGZsq{}}\PYG{p}{]}\PYG{p}{)}
\PYG{n}{np}\PYG{o}{.}\PYG{n}{random}\PYG{o}{.}\PYG{n}{seed}\PYG{p}{(}\PYG{l+m+mi}{42}\PYG{p}{)}
\PYG{n}{data} \PYG{o}{=} \PYG{n}{np}\PYG{o}{.}\PYG{n}{random}\PYG{o}{.}\PYG{n}{randn}\PYG{p}{(}\PYG{l+m+mi}{7}\PYG{p}{,} \PYG{l+m+mi}{4}\PYG{p}{)}
\end{sphinxVerbatim}

\end{sphinxuseclass}\end{sphinxVerbatimInput}

\end{sphinxuseclass}
\begin{sphinxuseclass}{cell}\begin{sphinxVerbatimInput}

\begin{sphinxuseclass}{cell_input}
\begin{sphinxVerbatim}[commandchars=\\\{\}]
\PYG{n}{names}
\end{sphinxVerbatim}

\end{sphinxuseclass}\end{sphinxVerbatimInput}
\begin{sphinxVerbatimOutput}

\begin{sphinxuseclass}{cell_output}
\begin{sphinxVerbatim}[commandchars=\\\{\}]
array([\PYGZsq{}Bob\PYGZsq{}, \PYGZsq{}Joe\PYGZsq{}, \PYGZsq{}Will\PYGZsq{}, \PYGZsq{}Bob\PYGZsq{}, \PYGZsq{}Will\PYGZsq{}, \PYGZsq{}Joe\PYGZsq{}, \PYGZsq{}Joe\PYGZsq{}], dtype=\PYGZsq{}\PYGZlt{}U4\PYGZsq{})
\end{sphinxVerbatim}

\end{sphinxuseclass}\end{sphinxVerbatimOutput}

\end{sphinxuseclass}
\begin{sphinxuseclass}{cell}\begin{sphinxVerbatimInput}

\begin{sphinxuseclass}{cell_input}
\begin{sphinxVerbatim}[commandchars=\\\{\}]
\PYG{n}{data}
\end{sphinxVerbatim}

\end{sphinxuseclass}\end{sphinxVerbatimInput}
\begin{sphinxVerbatimOutput}

\begin{sphinxuseclass}{cell_output}
\begin{sphinxVerbatim}[commandchars=\\\{\}]
array([[ 0.4967, \PYGZhy{}0.1383,  0.6477,  1.523 ],
       [\PYGZhy{}0.2342, \PYGZhy{}0.2341,  1.5792,  0.7674],
       [\PYGZhy{}0.4695,  0.5426, \PYGZhy{}0.4634, \PYGZhy{}0.4657],
       [ 0.242 , \PYGZhy{}1.9133, \PYGZhy{}1.7249, \PYGZhy{}0.5623],
       [\PYGZhy{}1.0128,  0.3142, \PYGZhy{}0.908 , \PYGZhy{}1.4123],
       [ 1.4656, \PYGZhy{}0.2258,  0.0675, \PYGZhy{}1.4247],
       [\PYGZhy{}0.5444,  0.1109, \PYGZhy{}1.151 ,  0.3757]])
\end{sphinxVerbatim}

\end{sphinxuseclass}\end{sphinxVerbatimOutput}

\end{sphinxuseclass}
\sphinxAtStartPar
Here \sphinxcode{\sphinxupquote{names}} provides seven names for the seven rows in \sphinxcode{\sphinxupquote{data}}.

\begin{sphinxuseclass}{cell}\begin{sphinxVerbatimInput}

\begin{sphinxuseclass}{cell_input}
\begin{sphinxVerbatim}[commandchars=\\\{\}]
\PYG{n}{names} \PYG{o}{==} \PYG{l+s+s1}{\PYGZsq{}}\PYG{l+s+s1}{Bob}\PYG{l+s+s1}{\PYGZsq{}}
\end{sphinxVerbatim}

\end{sphinxuseclass}\end{sphinxVerbatimInput}
\begin{sphinxVerbatimOutput}

\begin{sphinxuseclass}{cell_output}
\begin{sphinxVerbatim}[commandchars=\\\{\}]
array([ True, False, False,  True, False, False, False])
\end{sphinxVerbatim}

\end{sphinxuseclass}\end{sphinxVerbatimOutput}

\end{sphinxuseclass}
\begin{sphinxuseclass}{cell}\begin{sphinxVerbatimInput}

\begin{sphinxuseclass}{cell_input}
\begin{sphinxVerbatim}[commandchars=\\\{\}]
\PYG{n}{data}\PYG{p}{[}\PYG{n}{names} \PYG{o}{==} \PYG{l+s+s1}{\PYGZsq{}}\PYG{l+s+s1}{Bob}\PYG{l+s+s1}{\PYGZsq{}}\PYG{p}{]}
\end{sphinxVerbatim}

\end{sphinxuseclass}\end{sphinxVerbatimInput}
\begin{sphinxVerbatimOutput}

\begin{sphinxuseclass}{cell_output}
\begin{sphinxVerbatim}[commandchars=\\\{\}]
array([[ 0.4967, \PYGZhy{}0.1383,  0.6477,  1.523 ],
       [ 0.242 , \PYGZhy{}1.9133, \PYGZhy{}1.7249, \PYGZhy{}0.5623]])
\end{sphinxVerbatim}

\end{sphinxuseclass}\end{sphinxVerbatimOutput}

\end{sphinxuseclass}
\sphinxAtStartPar
We can combine Boolean slicing with \sphinxcode{\sphinxupquote{:}} slicing.

\begin{sphinxuseclass}{cell}\begin{sphinxVerbatimInput}

\begin{sphinxuseclass}{cell_input}
\begin{sphinxVerbatim}[commandchars=\\\{\}]
\PYG{n}{data}\PYG{p}{[}\PYG{n}{names} \PYG{o}{==} \PYG{l+s+s1}{\PYGZsq{}}\PYG{l+s+s1}{Bob}\PYG{l+s+s1}{\PYGZsq{}}\PYG{p}{,} \PYG{l+m+mi}{2}\PYG{p}{:}\PYG{p}{]}
\end{sphinxVerbatim}

\end{sphinxuseclass}\end{sphinxVerbatimInput}
\begin{sphinxVerbatimOutput}

\begin{sphinxuseclass}{cell_output}
\begin{sphinxVerbatim}[commandchars=\\\{\}]
array([[ 0.6477,  1.523 ],
       [\PYGZhy{}1.7249, \PYGZhy{}0.5623]])
\end{sphinxVerbatim}

\end{sphinxuseclass}\end{sphinxVerbatimOutput}

\end{sphinxuseclass}
\sphinxAtStartPar
We can use \sphinxcode{\sphinxupquote{\textasciitilde{}}} to invert a Boolean.

\begin{sphinxuseclass}{cell}\begin{sphinxVerbatimInput}

\begin{sphinxuseclass}{cell_input}
\begin{sphinxVerbatim}[commandchars=\\\{\}]
\PYG{n}{cond} \PYG{o}{=} \PYG{n}{names} \PYG{o}{==} \PYG{l+s+s1}{\PYGZsq{}}\PYG{l+s+s1}{Bob}\PYG{l+s+s1}{\PYGZsq{}}
\PYG{n}{data}\PYG{p}{[}\PYG{o}{\PYGZti{}}\PYG{n}{cond}\PYG{p}{]}
\end{sphinxVerbatim}

\end{sphinxuseclass}\end{sphinxVerbatimInput}
\begin{sphinxVerbatimOutput}

\begin{sphinxuseclass}{cell_output}
\begin{sphinxVerbatim}[commandchars=\\\{\}]
array([[\PYGZhy{}0.2342, \PYGZhy{}0.2341,  1.5792,  0.7674],
       [\PYGZhy{}0.4695,  0.5426, \PYGZhy{}0.4634, \PYGZhy{}0.4657],
       [\PYGZhy{}1.0128,  0.3142, \PYGZhy{}0.908 , \PYGZhy{}1.4123],
       [ 1.4656, \PYGZhy{}0.2258,  0.0675, \PYGZhy{}1.4247],
       [\PYGZhy{}0.5444,  0.1109, \PYGZhy{}1.151 ,  0.3757]])
\end{sphinxVerbatim}

\end{sphinxuseclass}\end{sphinxVerbatimOutput}

\end{sphinxuseclass}
\sphinxAtStartPar
For NumPy arrays, we must use \sphinxcode{\sphinxupquote{\&}} and \sphinxcode{\sphinxupquote{|}} instead of \sphinxcode{\sphinxupquote{and}} and \sphinxcode{\sphinxupquote{or}}.

\begin{sphinxuseclass}{cell}\begin{sphinxVerbatimInput}

\begin{sphinxuseclass}{cell_input}
\begin{sphinxVerbatim}[commandchars=\\\{\}]
\PYG{n}{cond} \PYG{o}{=} \PYG{p}{(}\PYG{n}{names} \PYG{o}{==} \PYG{l+s+s1}{\PYGZsq{}}\PYG{l+s+s1}{Bob}\PYG{l+s+s1}{\PYGZsq{}}\PYG{p}{)} \PYG{o}{|} \PYG{p}{(}\PYG{n}{names} \PYG{o}{==} \PYG{l+s+s1}{\PYGZsq{}}\PYG{l+s+s1}{Will}\PYG{l+s+s1}{\PYGZsq{}}\PYG{p}{)}
\PYG{n}{data}\PYG{p}{[}\PYG{n}{cond}\PYG{p}{]}
\end{sphinxVerbatim}

\end{sphinxuseclass}\end{sphinxVerbatimInput}
\begin{sphinxVerbatimOutput}

\begin{sphinxuseclass}{cell_output}
\begin{sphinxVerbatim}[commandchars=\\\{\}]
array([[ 0.4967, \PYGZhy{}0.1383,  0.6477,  1.523 ],
       [\PYGZhy{}0.4695,  0.5426, \PYGZhy{}0.4634, \PYGZhy{}0.4657],
       [ 0.242 , \PYGZhy{}1.9133, \PYGZhy{}1.7249, \PYGZhy{}0.5623],
       [\PYGZhy{}1.0128,  0.3142, \PYGZhy{}0.908 , \PYGZhy{}1.4123]])
\end{sphinxVerbatim}

\end{sphinxuseclass}\end{sphinxVerbatimOutput}

\end{sphinxuseclass}
\sphinxAtStartPar
We can also create a Boolean for each element.

\begin{sphinxuseclass}{cell}\begin{sphinxVerbatimInput}

\begin{sphinxuseclass}{cell_input}
\begin{sphinxVerbatim}[commandchars=\\\{\}]
\PYG{n}{data} \PYG{o}{\PYGZlt{}} \PYG{l+m+mi}{0}
\end{sphinxVerbatim}

\end{sphinxuseclass}\end{sphinxVerbatimInput}
\begin{sphinxVerbatimOutput}

\begin{sphinxuseclass}{cell_output}
\begin{sphinxVerbatim}[commandchars=\\\{\}]
array([[False,  True, False, False],
       [ True,  True, False, False],
       [ True, False,  True,  True],
       [False,  True,  True,  True],
       [ True, False,  True,  True],
       [False,  True, False,  True],
       [ True, False,  True, False]])
\end{sphinxVerbatim}

\end{sphinxuseclass}\end{sphinxVerbatimOutput}

\end{sphinxuseclass}
\begin{sphinxuseclass}{cell}\begin{sphinxVerbatimInput}

\begin{sphinxuseclass}{cell_input}
\begin{sphinxVerbatim}[commandchars=\\\{\}]
\PYG{n}{data}\PYG{p}{[}\PYG{n}{data} \PYG{o}{\PYGZlt{}} \PYG{l+m+mi}{0}\PYG{p}{]} \PYG{o}{=} \PYG{l+m+mi}{0}
\PYG{n}{data}
\end{sphinxVerbatim}

\end{sphinxuseclass}\end{sphinxVerbatimInput}
\begin{sphinxVerbatimOutput}

\begin{sphinxuseclass}{cell_output}
\begin{sphinxVerbatim}[commandchars=\\\{\}]
array([[0.4967, 0.    , 0.6477, 1.523 ],
       [0.    , 0.    , 1.5792, 0.7674],
       [0.    , 0.5426, 0.    , 0.    ],
       [0.242 , 0.    , 0.    , 0.    ],
       [0.    , 0.3142, 0.    , 0.    ],
       [1.4656, 0.    , 0.0675, 0.    ],
       [0.    , 0.1109, 0.    , 0.3757]])
\end{sphinxVerbatim}

\end{sphinxuseclass}\end{sphinxVerbatimOutput}

\end{sphinxuseclass}

\section{Universal Functions: Fast Element\sphinxhyphen{}Wise Array Functions}
\label{\detokenize{mckinney_04_lecture:universal-functions-fast-element-wise-array-functions}}\begin{quote}

\sphinxAtStartPar
A universal function, or ufunc, is a function that performs element\sphinxhyphen{}wise operations on data in ndarrays. You can think of them as fast vectorized wrappers for simple functions that take one or more scalar values and produce one or more scalar results.
\end{quote}

\begin{sphinxuseclass}{cell}\begin{sphinxVerbatimInput}

\begin{sphinxuseclass}{cell_input}
\begin{sphinxVerbatim}[commandchars=\\\{\}]
\PYG{n}{arr} \PYG{o}{=} \PYG{n}{np}\PYG{o}{.}\PYG{n}{arange}\PYG{p}{(}\PYG{l+m+mi}{10}\PYG{p}{)}
\PYG{n}{arr}
\end{sphinxVerbatim}

\end{sphinxuseclass}\end{sphinxVerbatimInput}
\begin{sphinxVerbatimOutput}

\begin{sphinxuseclass}{cell_output}
\begin{sphinxVerbatim}[commandchars=\\\{\}]
array([0, 1, 2, 3, 4, 5, 6, 7, 8, 9])
\end{sphinxVerbatim}

\end{sphinxuseclass}\end{sphinxVerbatimOutput}

\end{sphinxuseclass}
\begin{sphinxuseclass}{cell}\begin{sphinxVerbatimInput}

\begin{sphinxuseclass}{cell_input}
\begin{sphinxVerbatim}[commandchars=\\\{\}]
\PYG{n}{np}\PYG{o}{.}\PYG{n}{sqrt}\PYG{p}{(}\PYG{n}{arr}\PYG{p}{)}
\end{sphinxVerbatim}

\end{sphinxuseclass}\end{sphinxVerbatimInput}
\begin{sphinxVerbatimOutput}

\begin{sphinxuseclass}{cell_output}
\begin{sphinxVerbatim}[commandchars=\\\{\}]
array([0.    , 1.    , 1.4142, 1.7321, 2.    , 2.2361, 2.4495, 2.6458,
       2.8284, 3.    ])
\end{sphinxVerbatim}

\end{sphinxuseclass}\end{sphinxVerbatimOutput}

\end{sphinxuseclass}
\sphinxAtStartPar
Like above, we can raise a single value to a NumPy array of powers.

\begin{sphinxuseclass}{cell}\begin{sphinxVerbatimInput}

\begin{sphinxuseclass}{cell_input}
\begin{sphinxVerbatim}[commandchars=\\\{\}]
\PYG{l+m+mi}{2}\PYG{o}{*}\PYG{o}{*}\PYG{n}{arr}
\end{sphinxVerbatim}

\end{sphinxuseclass}\end{sphinxVerbatimInput}
\begin{sphinxVerbatimOutput}

\begin{sphinxuseclass}{cell_output}
\begin{sphinxVerbatim}[commandchars=\\\{\}]
array([  1,   2,   4,   8,  16,  32,  64, 128, 256, 512])
\end{sphinxVerbatim}

\end{sphinxuseclass}\end{sphinxVerbatimOutput}

\end{sphinxuseclass}
\sphinxAtStartPar
\sphinxcode{\sphinxupquote{np.exp(x)}} is \$e\textasciicircum{}x\$.

\begin{sphinxuseclass}{cell}\begin{sphinxVerbatimInput}

\begin{sphinxuseclass}{cell_input}
\begin{sphinxVerbatim}[commandchars=\\\{\}]
\PYG{n}{np}\PYG{o}{.}\PYG{n}{exp}\PYG{p}{(}\PYG{n}{arr}\PYG{p}{)}
\end{sphinxVerbatim}

\end{sphinxuseclass}\end{sphinxVerbatimInput}
\begin{sphinxVerbatimOutput}

\begin{sphinxuseclass}{cell_output}
\begin{sphinxVerbatim}[commandchars=\\\{\}]
array([1.0000e+00, 2.7183e+00, 7.3891e+00, 2.0086e+01, 5.4598e+01,
       1.4841e+02, 4.0343e+02, 1.0966e+03, 2.9810e+03, 8.1031e+03])
\end{sphinxVerbatim}

\end{sphinxuseclass}\end{sphinxVerbatimOutput}

\end{sphinxuseclass}
\sphinxAtStartPar
The functions above accept one argument.
These “unary” functions operate on one array and return a new array with the same shape.
There are also “binary” functions that operate on two arrays and return one array.

\begin{sphinxuseclass}{cell}\begin{sphinxVerbatimInput}

\begin{sphinxuseclass}{cell_input}
\begin{sphinxVerbatim}[commandchars=\\\{\}]
\PYG{n}{np}\PYG{o}{.}\PYG{n}{random}\PYG{o}{.}\PYG{n}{seed}\PYG{p}{(}\PYG{l+m+mi}{42}\PYG{p}{)}
\PYG{n}{x} \PYG{o}{=} \PYG{n}{np}\PYG{o}{.}\PYG{n}{random}\PYG{o}{.}\PYG{n}{randn}\PYG{p}{(}\PYG{l+m+mi}{8}\PYG{p}{)}
\PYG{n}{y} \PYG{o}{=} \PYG{n}{np}\PYG{o}{.}\PYG{n}{random}\PYG{o}{.}\PYG{n}{randn}\PYG{p}{(}\PYG{l+m+mi}{8}\PYG{p}{)}
\end{sphinxVerbatim}

\end{sphinxuseclass}\end{sphinxVerbatimInput}

\end{sphinxuseclass}
\begin{sphinxuseclass}{cell}\begin{sphinxVerbatimInput}

\begin{sphinxuseclass}{cell_input}
\begin{sphinxVerbatim}[commandchars=\\\{\}]
\PYG{n}{x}
\end{sphinxVerbatim}

\end{sphinxuseclass}\end{sphinxVerbatimInput}
\begin{sphinxVerbatimOutput}

\begin{sphinxuseclass}{cell_output}
\begin{sphinxVerbatim}[commandchars=\\\{\}]
array([ 0.4967, \PYGZhy{}0.1383,  0.6477,  1.523 , \PYGZhy{}0.2342, \PYGZhy{}0.2341,  1.5792,
        0.7674])
\end{sphinxVerbatim}

\end{sphinxuseclass}\end{sphinxVerbatimOutput}

\end{sphinxuseclass}
\begin{sphinxuseclass}{cell}\begin{sphinxVerbatimInput}

\begin{sphinxuseclass}{cell_input}
\begin{sphinxVerbatim}[commandchars=\\\{\}]
\PYG{n}{y}
\end{sphinxVerbatim}

\end{sphinxuseclass}\end{sphinxVerbatimInput}
\begin{sphinxVerbatimOutput}

\begin{sphinxuseclass}{cell_output}
\begin{sphinxVerbatim}[commandchars=\\\{\}]
array([\PYGZhy{}0.4695,  0.5426, \PYGZhy{}0.4634, \PYGZhy{}0.4657,  0.242 , \PYGZhy{}1.9133, \PYGZhy{}1.7249,
       \PYGZhy{}0.5623])
\end{sphinxVerbatim}

\end{sphinxuseclass}\end{sphinxVerbatimOutput}

\end{sphinxuseclass}
\begin{sphinxuseclass}{cell}\begin{sphinxVerbatimInput}

\begin{sphinxuseclass}{cell_input}
\begin{sphinxVerbatim}[commandchars=\\\{\}]
\PYG{n}{np}\PYG{o}{.}\PYG{n}{maximum}\PYG{p}{(}\PYG{n}{x}\PYG{p}{,} \PYG{n}{y}\PYG{p}{)}
\end{sphinxVerbatim}

\end{sphinxuseclass}\end{sphinxVerbatimInput}
\begin{sphinxVerbatimOutput}

\begin{sphinxuseclass}{cell_output}
\begin{sphinxVerbatim}[commandchars=\\\{\}]
array([ 0.4967,  0.5426,  0.6477,  1.523 ,  0.242 , \PYGZhy{}0.2341,  1.5792,
        0.7674])
\end{sphinxVerbatim}

\end{sphinxuseclass}\end{sphinxVerbatimOutput}

\end{sphinxuseclass}
\sphinxAtStartPar
\sphinxstyleemphasis{\sphinxstylestrong{Be careful!
Function names are not the whole story.
Check your output and read the docstring!}}
For example, \sphinxcode{\sphinxupquote{np.max()}} returns the maximum of an array, instead of the elementwise maximum of two arrays for \sphinxcode{\sphinxupquote{np.maximum()}}.

\begin{sphinxuseclass}{cell}\begin{sphinxVerbatimInput}

\begin{sphinxuseclass}{cell_input}
\begin{sphinxVerbatim}[commandchars=\\\{\}]
\PYG{n}{np}\PYG{o}{.}\PYG{n}{max}\PYG{p}{(}\PYG{n}{x}\PYG{p}{)}
\end{sphinxVerbatim}

\end{sphinxuseclass}\end{sphinxVerbatimInput}
\begin{sphinxVerbatimOutput}

\begin{sphinxuseclass}{cell_output}
\begin{sphinxVerbatim}[commandchars=\\\{\}]
1.5792
\end{sphinxVerbatim}

\end{sphinxuseclass}\end{sphinxVerbatimOutput}

\end{sphinxuseclass}
\sphinxAtStartPar
\sphinxstyleemphasis{\sphinxstylestrong{Table 4\sphinxhyphen{}4}} from McKinney lists some fast, element\sphinxhyphen{}wise unary functions:
\begin{itemize}
\item {} 
\sphinxAtStartPar
\sphinxcode{\sphinxupquote{abs}}, \sphinxcode{\sphinxupquote{fabs}}: Compute the absolute value element\sphinxhyphen{}wise for integer, „oating\sphinxhyphen{}point, or complex values

\item {} 
\sphinxAtStartPar
\sphinxcode{\sphinxupquote{sqrt}}: Compute the square root of each element (equivalent to arr ** 0.5)

\item {} 
\sphinxAtStartPar
\sphinxcode{\sphinxupquote{square}}: Compute the square of each element (equivalent to arr ** 2)

\item {} 
\sphinxAtStartPar
\sphinxcode{\sphinxupquote{exp}}: Compute the exponent \$e\textasciicircum{}x\$ of each element

\item {} 
\sphinxAtStartPar
\sphinxcode{\sphinxupquote{log}}, \sphinxcode{\sphinxupquote{log10}}, \sphinxcode{\sphinxupquote{log2}}, \sphinxcode{\sphinxupquote{log1p}}: Natural logarithm (base e), log base 10, log base 2, and log(1 + x), respectively

\item {} 
\sphinxAtStartPar
\sphinxcode{\sphinxupquote{sign}}: Compute the sign of each element: 1 (positive), 0 (zero), or –1 (negative)

\item {} 
\sphinxAtStartPar
\sphinxcode{\sphinxupquote{ceil}}: Compute the ceiling of each element (i.e., the smallest integer greater than or equal to thatnumber)

\item {} 
\sphinxAtStartPar
\sphinxcode{\sphinxupquote{floor}}: Compute the „oor of each element (i.e., the largest integer less than or equal to each element)

\item {} 
\sphinxAtStartPar
\sphinxcode{\sphinxupquote{rint}}: Round elements to the nearest integer, preserving the dtype

\item {} 
\sphinxAtStartPar
\sphinxcode{\sphinxupquote{modf}}: Return fractional and integral parts of array as a separate array

\item {} 
\sphinxAtStartPar
\sphinxcode{\sphinxupquote{isnan}}: Return boolean array indicating whether each value is NaN (Not a Number)

\item {} 
\sphinxAtStartPar
\sphinxcode{\sphinxupquote{isfinite}}, \sphinxcode{\sphinxupquote{isinf}}: Return boolean array indicating whether each element is finite (non\sphinxhyphen{}inf, non\sphinxhyphen{}NaN) or infinite, respectively

\item {} 
\sphinxAtStartPar
\sphinxcode{\sphinxupquote{cos}}, \sphinxcode{\sphinxupquote{cosh}}, \sphinxcode{\sphinxupquote{sin}}, \sphinxcode{\sphinxupquote{sinh}}, \sphinxcode{\sphinxupquote{tan}}, \sphinxcode{\sphinxupquote{tanh}}: Regular and hyperbolic trigonometric functions

\item {} 
\sphinxAtStartPar
\sphinxcode{\sphinxupquote{arccos}}, \sphinxcode{\sphinxupquote{arccosh}}, \sphinxcode{\sphinxupquote{arcsin}}, \sphinxcode{\sphinxupquote{arcsinh}}, \sphinxcode{\sphinxupquote{arctan}}, \sphinxcode{\sphinxupquote{arctanh}}: Inverse trigonometric functions

\item {} 
\sphinxAtStartPar
\sphinxcode{\sphinxupquote{logical\_not}}: Compute truth value of not x element\sphinxhyphen{}wise (equivalent to \textasciitilde{}arr).

\end{itemize}

\sphinxAtStartPar
\sphinxstyleemphasis{\sphinxstylestrong{Table 4\sphinxhyphen{}5}} from McKinney lists some fast, element\sphinxhyphen{}wise binary functions:
\begin{itemize}
\item {} 
\sphinxAtStartPar
\sphinxcode{\sphinxupquote{add}}: Add corresponding elements in arrays

\item {} 
\sphinxAtStartPar
\sphinxcode{\sphinxupquote{subtract}}: Subtract elements in second array from first array

\item {} 
\sphinxAtStartPar
\sphinxcode{\sphinxupquote{multiply}}: Multiply array elements

\item {} 
\sphinxAtStartPar
\sphinxcode{\sphinxupquote{divide}}, \sphinxcode{\sphinxupquote{floor\_divide}}: Divide or floor divide (truncating the remainder)

\item {} 
\sphinxAtStartPar
\sphinxcode{\sphinxupquote{power}}: Raise elements in first array to powers indicated in second array

\item {} 
\sphinxAtStartPar
\sphinxcode{\sphinxupquote{maximum}}, \sphinxcode{\sphinxupquote{fmax}}: Element\sphinxhyphen{}wise maximum; \sphinxcode{\sphinxupquote{fmax}} ignores \sphinxcode{\sphinxupquote{NaN}}

\item {} 
\sphinxAtStartPar
\sphinxcode{\sphinxupquote{minimum}}, \sphinxcode{\sphinxupquote{fmin}}: Element\sphinxhyphen{}wise minimum; \sphinxcode{\sphinxupquote{fmin}} ignores \sphinxcode{\sphinxupquote{NaN}}

\item {} 
\sphinxAtStartPar
\sphinxcode{\sphinxupquote{mod}}: Element\sphinxhyphen{}wise modulus (remainder of division)

\item {} 
\sphinxAtStartPar
\sphinxcode{\sphinxupquote{copysign}}: Copy sign of values in second argument to values in first argument

\item {} 
\sphinxAtStartPar
\sphinxcode{\sphinxupquote{greater}}, \sphinxcode{\sphinxupquote{greater\_equal}}, \sphinxcode{\sphinxupquote{less}}, \sphinxcode{\sphinxupquote{less\_equal}}, \sphinxcode{\sphinxupquote{equal}}, \sphinxcode{\sphinxupquote{not\_equal}}: Perform element\sphinxhyphen{}wise comparison, yielding boolean array (equivalent to infix operators >, >=, <, <=, ==, !=)

\item {} 
\sphinxAtStartPar
\sphinxcode{\sphinxupquote{logical\_and}}, \sphinxcode{\sphinxupquote{logical\_or}}, \sphinxcode{\sphinxupquote{logical\_xor}}: Compute element\sphinxhyphen{}wise truth value of logical operation (equivalent to infix operators \& |, \textasciicircum{})

\end{itemize}


\section{Array\sphinxhyphen{}Oriented Programming with Arrays}
\label{\detokenize{mckinney_04_lecture:array-oriented-programming-with-arrays}}\begin{quote}

\sphinxAtStartPar
Using NumPy arrays enables you to express many kinds of data processing tasks as concise array expressions that might otherwise require writing loops. This practice of replacing explicit loops with array expressions is commonly referred to as vectorization. In general, vectorized array operations will often be one or two (or more) orders of magnitude faster than their pure Python equivalents, with the biggest impact in any kind of numerical computations. Later, in Appendix A, I explain broadcasting, a powerful method for vectorizing computations.
\end{quote}


\subsection{Expressing Conditional Logic as Array Operations}
\label{\detokenize{mckinney_04_lecture:expressing-conditional-logic-as-array-operations}}\begin{quote}

\sphinxAtStartPar
The numpy.where function is a vectorized version of the ternary expression x if condition else y.
\end{quote}

\sphinxAtStartPar
NumPy’s \sphinxcode{\sphinxupquote{where()}} is an if\sphinxhyphen{}else statement that operates like Excel’s \sphinxcode{\sphinxupquote{if()}}.

\begin{sphinxuseclass}{cell}\begin{sphinxVerbatimInput}

\begin{sphinxuseclass}{cell_input}
\begin{sphinxVerbatim}[commandchars=\\\{\}]
\PYG{n}{xarr} \PYG{o}{=} \PYG{n}{np}\PYG{o}{.}\PYG{n}{array}\PYG{p}{(}\PYG{p}{[}\PYG{l+m+mf}{1.1}\PYG{p}{,} \PYG{l+m+mf}{1.2}\PYG{p}{,} \PYG{l+m+mf}{1.3}\PYG{p}{,} \PYG{l+m+mf}{1.4}\PYG{p}{,} \PYG{l+m+mf}{1.5}\PYG{p}{]}\PYG{p}{)}
\PYG{n}{yarr} \PYG{o}{=} \PYG{n}{np}\PYG{o}{.}\PYG{n}{array}\PYG{p}{(}\PYG{p}{[}\PYG{l+m+mf}{2.1}\PYG{p}{,} \PYG{l+m+mf}{2.2}\PYG{p}{,} \PYG{l+m+mf}{2.3}\PYG{p}{,} \PYG{l+m+mf}{2.4}\PYG{p}{,} \PYG{l+m+mf}{2.5}\PYG{p}{]}\PYG{p}{)}
\PYG{n}{cond} \PYG{o}{=} \PYG{n}{np}\PYG{o}{.}\PYG{n}{array}\PYG{p}{(}\PYG{p}{[}\PYG{k+kc}{True}\PYG{p}{,} \PYG{k+kc}{False}\PYG{p}{,} \PYG{k+kc}{True}\PYG{p}{,} \PYG{k+kc}{True}\PYG{p}{,} \PYG{k+kc}{False}\PYG{p}{]}\PYG{p}{)}
\end{sphinxVerbatim}

\end{sphinxuseclass}\end{sphinxVerbatimInput}

\end{sphinxuseclass}
\begin{sphinxuseclass}{cell}\begin{sphinxVerbatimInput}

\begin{sphinxuseclass}{cell_input}
\begin{sphinxVerbatim}[commandchars=\\\{\}]
\PYG{n}{np}\PYG{o}{.}\PYG{n}{where}\PYG{p}{(}\PYG{n}{cond}\PYG{p}{,} \PYG{n}{xarr}\PYG{p}{,} \PYG{n}{yarr}\PYG{p}{)}
\end{sphinxVerbatim}

\end{sphinxuseclass}\end{sphinxVerbatimInput}
\begin{sphinxVerbatimOutput}

\begin{sphinxuseclass}{cell_output}
\begin{sphinxVerbatim}[commandchars=\\\{\}]
array([1.1, 2.2, 1.3, 1.4, 2.5])
\end{sphinxVerbatim}

\end{sphinxuseclass}\end{sphinxVerbatimOutput}

\end{sphinxuseclass}
\sphinxAtStartPar
We could use a list comprehension, instead, but the list comprehension is takes longer to type, read, and troubleshoot.

\begin{sphinxuseclass}{cell}\begin{sphinxVerbatimInput}

\begin{sphinxuseclass}{cell_input}
\begin{sphinxVerbatim}[commandchars=\\\{\}]
\PYG{n}{np}\PYG{o}{.}\PYG{n}{array}\PYG{p}{(}\PYG{p}{[}\PYG{p}{(}\PYG{n}{x} \PYG{k}{if} \PYG{n}{c} \PYG{k}{else} \PYG{n}{y}\PYG{p}{)} \PYG{k}{for} \PYG{n}{x}\PYG{p}{,} \PYG{n}{y}\PYG{p}{,} \PYG{n}{c} \PYG{o+ow}{in} \PYG{n+nb}{zip}\PYG{p}{(}\PYG{n}{xarr}\PYG{p}{,} \PYG{n}{yarr}\PYG{p}{,} \PYG{n}{cond}\PYG{p}{)}\PYG{p}{]}\PYG{p}{)}
\end{sphinxVerbatim}

\end{sphinxuseclass}\end{sphinxVerbatimInput}
\begin{sphinxVerbatimOutput}

\begin{sphinxuseclass}{cell_output}
\begin{sphinxVerbatim}[commandchars=\\\{\}]
array([1.1, 2.2, 1.3, 1.4, 2.5])
\end{sphinxVerbatim}

\end{sphinxuseclass}\end{sphinxVerbatimOutput}

\end{sphinxuseclass}
\sphinxAtStartPar
We could also use \sphinxcode{\sphinxupquote{np.select()}} here, but it is overkill to test one condition.
\sphinxcode{\sphinxupquote{np.select()}} lets us test more more than one condition and provides a default value if no condition is met.

\begin{sphinxuseclass}{cell}\begin{sphinxVerbatimInput}

\begin{sphinxuseclass}{cell_input}
\begin{sphinxVerbatim}[commandchars=\\\{\}]
\PYG{n}{np}\PYG{o}{.}\PYG{n}{select}\PYG{p}{(}
    \PYG{n}{condlist}\PYG{o}{=}\PYG{p}{[}\PYG{n}{cond}\PYG{o}{==}\PYG{k+kc}{True}\PYG{p}{,} \PYG{n}{cond}\PYG{o}{==}\PYG{k+kc}{False}\PYG{p}{]}\PYG{p}{,}
    \PYG{n}{choicelist}\PYG{o}{=}\PYG{p}{[}\PYG{n}{xarr}\PYG{p}{,} \PYG{n}{yarr}\PYG{p}{]}
\PYG{p}{)}
\end{sphinxVerbatim}

\end{sphinxuseclass}\end{sphinxVerbatimInput}
\begin{sphinxVerbatimOutput}

\begin{sphinxuseclass}{cell_output}
\begin{sphinxVerbatim}[commandchars=\\\{\}]
array([1.1, 2.2, 1.3, 1.4, 2.5])
\end{sphinxVerbatim}

\end{sphinxuseclass}\end{sphinxVerbatimOutput}

\end{sphinxuseclass}

\subsection{Mathematical and Statistical Methods}
\label{\detokenize{mckinney_04_lecture:mathematical-and-statistical-methods}}\begin{quote}

\sphinxAtStartPar
A set of mathematical functions that compute statistics about an entire array or about the data along an axis are accessible as methods of the array class. You can use aggregations (often called reductions) like sum, mean, and std (standard deviation) either by calling the array instance method or using the top\sphinxhyphen{}level NumPy function.
\end{quote}

\sphinxAtStartPar
We will use these aggregations extensively in pandas.

\begin{sphinxuseclass}{cell}\begin{sphinxVerbatimInput}

\begin{sphinxuseclass}{cell_input}
\begin{sphinxVerbatim}[commandchars=\\\{\}]
\PYG{n}{np}\PYG{o}{.}\PYG{n}{random}\PYG{o}{.}\PYG{n}{seed}\PYG{p}{(}\PYG{l+m+mi}{42}\PYG{p}{)}
\PYG{n}{arr} \PYG{o}{=} \PYG{n}{np}\PYG{o}{.}\PYG{n}{random}\PYG{o}{.}\PYG{n}{randn}\PYG{p}{(}\PYG{l+m+mi}{5}\PYG{p}{,} \PYG{l+m+mi}{4}\PYG{p}{)}
\PYG{n}{arr}
\end{sphinxVerbatim}

\end{sphinxuseclass}\end{sphinxVerbatimInput}
\begin{sphinxVerbatimOutput}

\begin{sphinxuseclass}{cell_output}
\begin{sphinxVerbatim}[commandchars=\\\{\}]
array([[ 0.4967, \PYGZhy{}0.1383,  0.6477,  1.523 ],
       [\PYGZhy{}0.2342, \PYGZhy{}0.2341,  1.5792,  0.7674],
       [\PYGZhy{}0.4695,  0.5426, \PYGZhy{}0.4634, \PYGZhy{}0.4657],
       [ 0.242 , \PYGZhy{}1.9133, \PYGZhy{}1.7249, \PYGZhy{}0.5623],
       [\PYGZhy{}1.0128,  0.3142, \PYGZhy{}0.908 , \PYGZhy{}1.4123]])
\end{sphinxVerbatim}

\end{sphinxuseclass}\end{sphinxVerbatimOutput}

\end{sphinxuseclass}
\begin{sphinxuseclass}{cell}\begin{sphinxVerbatimInput}

\begin{sphinxuseclass}{cell_input}
\begin{sphinxVerbatim}[commandchars=\\\{\}]
\PYG{n}{arr}\PYG{o}{.}\PYG{n}{mean}\PYG{p}{(}\PYG{p}{)}
\end{sphinxVerbatim}

\end{sphinxuseclass}\end{sphinxVerbatimInput}
\begin{sphinxVerbatimOutput}

\begin{sphinxuseclass}{cell_output}
\begin{sphinxVerbatim}[commandchars=\\\{\}]
\PYGZhy{}0.1713
\end{sphinxVerbatim}

\end{sphinxuseclass}\end{sphinxVerbatimOutput}

\end{sphinxuseclass}
\begin{sphinxuseclass}{cell}\begin{sphinxVerbatimInput}

\begin{sphinxuseclass}{cell_input}
\begin{sphinxVerbatim}[commandchars=\\\{\}]
\PYG{n}{arr}\PYG{o}{.}\PYG{n}{sum}\PYG{p}{(}\PYG{p}{)}
\end{sphinxVerbatim}

\end{sphinxuseclass}\end{sphinxVerbatimInput}
\begin{sphinxVerbatimOutput}

\begin{sphinxuseclass}{cell_output}
\begin{sphinxVerbatim}[commandchars=\\\{\}]
\PYGZhy{}3.4260
\end{sphinxVerbatim}

\end{sphinxuseclass}\end{sphinxVerbatimOutput}

\end{sphinxuseclass}
\sphinxAtStartPar
The aggregation methods above aggregated the whole array.
We can use the \sphinxcode{\sphinxupquote{axis}} argument to aggregate columns (\sphinxcode{\sphinxupquote{axis=0}}) and rows (\sphinxcode{\sphinxupquote{axis=1}}).

\begin{sphinxuseclass}{cell}\begin{sphinxVerbatimInput}

\begin{sphinxuseclass}{cell_input}
\begin{sphinxVerbatim}[commandchars=\\\{\}]
\PYG{n}{arr}\PYG{o}{.}\PYG{n}{mean}\PYG{p}{(}\PYG{n}{axis}\PYG{o}{=}\PYG{l+m+mi}{1}\PYG{p}{)}
\end{sphinxVerbatim}

\end{sphinxuseclass}\end{sphinxVerbatimInput}
\begin{sphinxVerbatimOutput}

\begin{sphinxuseclass}{cell_output}
\begin{sphinxVerbatim}[commandchars=\\\{\}]
array([ 0.6323,  0.4696, \PYGZhy{}0.214 , \PYGZhy{}0.9896, \PYGZhy{}0.7547])
\end{sphinxVerbatim}

\end{sphinxuseclass}\end{sphinxVerbatimOutput}

\end{sphinxuseclass}
\begin{sphinxuseclass}{cell}\begin{sphinxVerbatimInput}

\begin{sphinxuseclass}{cell_input}
\begin{sphinxVerbatim}[commandchars=\\\{\}]
\PYG{n}{arr}\PYG{p}{[}\PYG{l+m+mi}{0}\PYG{p}{]}\PYG{o}{.}\PYG{n}{mean}\PYG{p}{(}\PYG{p}{)}
\end{sphinxVerbatim}

\end{sphinxuseclass}\end{sphinxVerbatimInput}
\begin{sphinxVerbatimOutput}

\begin{sphinxuseclass}{cell_output}
\begin{sphinxVerbatim}[commandchars=\\\{\}]
0.6323
\end{sphinxVerbatim}

\end{sphinxuseclass}\end{sphinxVerbatimOutput}

\end{sphinxuseclass}
\begin{sphinxuseclass}{cell}\begin{sphinxVerbatimInput}

\begin{sphinxuseclass}{cell_input}
\begin{sphinxVerbatim}[commandchars=\\\{\}]
\PYG{n}{arr}\PYG{o}{.}\PYG{n}{mean}\PYG{p}{(}\PYG{n}{axis}\PYG{o}{=}\PYG{l+m+mi}{0}\PYG{p}{)}
\end{sphinxVerbatim}

\end{sphinxuseclass}\end{sphinxVerbatimInput}
\begin{sphinxVerbatimOutput}

\begin{sphinxuseclass}{cell_output}
\begin{sphinxVerbatim}[commandchars=\\\{\}]
array([\PYGZhy{}0.1956, \PYGZhy{}0.2858, \PYGZhy{}0.1739, \PYGZhy{}0.03  ])
\end{sphinxVerbatim}

\end{sphinxuseclass}\end{sphinxVerbatimOutput}

\end{sphinxuseclass}
\begin{sphinxuseclass}{cell}\begin{sphinxVerbatimInput}

\begin{sphinxuseclass}{cell_input}
\begin{sphinxVerbatim}[commandchars=\\\{\}]
\PYG{n}{arr}\PYG{p}{[}\PYG{p}{:}\PYG{p}{,} \PYG{l+m+mi}{0}\PYG{p}{]}\PYG{o}{.}\PYG{n}{mean}\PYG{p}{(}\PYG{p}{)}
\end{sphinxVerbatim}

\end{sphinxuseclass}\end{sphinxVerbatimInput}
\begin{sphinxVerbatimOutput}

\begin{sphinxuseclass}{cell_output}
\begin{sphinxVerbatim}[commandchars=\\\{\}]
\PYGZhy{}0.1956
\end{sphinxVerbatim}

\end{sphinxuseclass}\end{sphinxVerbatimOutput}

\end{sphinxuseclass}
\sphinxAtStartPar
The \sphinxcode{\sphinxupquote{.cumsum()}} method returns the sum of all previous elements.

\begin{sphinxuseclass}{cell}\begin{sphinxVerbatimInput}

\begin{sphinxuseclass}{cell_input}
\begin{sphinxVerbatim}[commandchars=\\\{\}]
\PYG{n}{arr} \PYG{o}{=} \PYG{n}{np}\PYG{o}{.}\PYG{n}{array}\PYG{p}{(}\PYG{p}{[}\PYG{l+m+mi}{0}\PYG{p}{,} \PYG{l+m+mi}{1}\PYG{p}{,} \PYG{l+m+mi}{2}\PYG{p}{,} \PYG{l+m+mi}{3}\PYG{p}{,} \PYG{l+m+mi}{4}\PYG{p}{,} \PYG{l+m+mi}{5}\PYG{p}{,} \PYG{l+m+mi}{6}\PYG{p}{,} \PYG{l+m+mi}{7}\PYG{p}{]}\PYG{p}{)}
\PYG{n}{arr}\PYG{o}{.}\PYG{n}{cumsum}\PYG{p}{(}\PYG{p}{)}
\end{sphinxVerbatim}

\end{sphinxuseclass}\end{sphinxVerbatimInput}
\begin{sphinxVerbatimOutput}

\begin{sphinxuseclass}{cell_output}
\begin{sphinxVerbatim}[commandchars=\\\{\}]
array([ 0,  1,  3,  6, 10, 15, 21, 28])
\end{sphinxVerbatim}

\end{sphinxuseclass}\end{sphinxVerbatimOutput}

\end{sphinxuseclass}
\sphinxAtStartPar
We can use the \sphinxcode{\sphinxupquote{.cumsum()}} method along the axis of a multi\sphinxhyphen{}dimension array, too.

\begin{sphinxuseclass}{cell}\begin{sphinxVerbatimInput}

\begin{sphinxuseclass}{cell_input}
\begin{sphinxVerbatim}[commandchars=\\\{\}]
\PYG{n}{arr} \PYG{o}{=} \PYG{n}{np}\PYG{o}{.}\PYG{n}{array}\PYG{p}{(}\PYG{p}{[}\PYG{p}{[}\PYG{l+m+mi}{0}\PYG{p}{,} \PYG{l+m+mi}{1}\PYG{p}{,} \PYG{l+m+mi}{2}\PYG{p}{]}\PYG{p}{,} \PYG{p}{[}\PYG{l+m+mi}{3}\PYG{p}{,} \PYG{l+m+mi}{4}\PYG{p}{,} \PYG{l+m+mi}{5}\PYG{p}{]}\PYG{p}{,} \PYG{p}{[}\PYG{l+m+mi}{6}\PYG{p}{,} \PYG{l+m+mi}{7}\PYG{p}{,} \PYG{l+m+mi}{8}\PYG{p}{]}\PYG{p}{]}\PYG{p}{)}
\PYG{n}{arr}
\end{sphinxVerbatim}

\end{sphinxuseclass}\end{sphinxVerbatimInput}
\begin{sphinxVerbatimOutput}

\begin{sphinxuseclass}{cell_output}
\begin{sphinxVerbatim}[commandchars=\\\{\}]
array([[0, 1, 2],
       [3, 4, 5],
       [6, 7, 8]])
\end{sphinxVerbatim}

\end{sphinxuseclass}\end{sphinxVerbatimOutput}

\end{sphinxuseclass}
\begin{sphinxuseclass}{cell}\begin{sphinxVerbatimInput}

\begin{sphinxuseclass}{cell_input}
\begin{sphinxVerbatim}[commandchars=\\\{\}]
\PYG{n}{arr}\PYG{o}{.}\PYG{n}{cumsum}\PYG{p}{(}\PYG{n}{axis}\PYG{o}{=}\PYG{l+m+mi}{0}\PYG{p}{)}
\end{sphinxVerbatim}

\end{sphinxuseclass}\end{sphinxVerbatimInput}
\begin{sphinxVerbatimOutput}

\begin{sphinxuseclass}{cell_output}
\begin{sphinxVerbatim}[commandchars=\\\{\}]
array([[ 0,  1,  2],
       [ 3,  5,  7],
       [ 9, 12, 15]])
\end{sphinxVerbatim}

\end{sphinxuseclass}\end{sphinxVerbatimOutput}

\end{sphinxuseclass}
\begin{sphinxuseclass}{cell}\begin{sphinxVerbatimInput}

\begin{sphinxuseclass}{cell_input}
\begin{sphinxVerbatim}[commandchars=\\\{\}]
\PYG{n}{arr}\PYG{o}{.}\PYG{n}{cumprod}\PYG{p}{(}\PYG{n}{axis}\PYG{o}{=}\PYG{l+m+mi}{1}\PYG{p}{)}
\end{sphinxVerbatim}

\end{sphinxuseclass}\end{sphinxVerbatimInput}
\begin{sphinxVerbatimOutput}

\begin{sphinxuseclass}{cell_output}
\begin{sphinxVerbatim}[commandchars=\\\{\}]
array([[  0,   0,   0],
       [  3,  12,  60],
       [  6,  42, 336]])
\end{sphinxVerbatim}

\end{sphinxuseclass}\end{sphinxVerbatimOutput}

\end{sphinxuseclass}
\sphinxAtStartPar
\sphinxstyleemphasis{\sphinxstylestrong{Table 4\sphinxhyphen{}6}} from McKinney lists some basic statistical methods:
\begin{itemize}
\item {} 
\sphinxAtStartPar
\sphinxcode{\sphinxupquote{sum}}: Sum of all the elements in the array or along an axis; zero\sphinxhyphen{}length arrays have sum 0

\item {} 
\sphinxAtStartPar
\sphinxcode{\sphinxupquote{mean}}: Arithmetic mean; zero\sphinxhyphen{}length arrays have NaN mean

\item {} 
\sphinxAtStartPar
\sphinxcode{\sphinxupquote{std}}, \sphinxcode{\sphinxupquote{var}}: Standard deviation and variance, respectively, with optional degrees of freedom adjustment (default denominator \$n\$)

\item {} 
\sphinxAtStartPar
\sphinxcode{\sphinxupquote{min}}, \sphinxcode{\sphinxupquote{max}}: Minimum and maximum

\item {} 
\sphinxAtStartPar
\sphinxcode{\sphinxupquote{argmin}}, \sphinxcode{\sphinxupquote{argmax}}: Indices of minimum and maximum elements, respectively

\item {} 
\sphinxAtStartPar
\sphinxcode{\sphinxupquote{cumsum}}: Cumulative sum of elements starting from 0

\item {} 
\sphinxAtStartPar
\sphinxcode{\sphinxupquote{cumprod}}: Cumulative product of elements starting from 1

\end{itemize}


\subsection{Methods for Boolean Arrays}
\label{\detokenize{mckinney_04_lecture:methods-for-boolean-arrays}}
\begin{sphinxuseclass}{cell}\begin{sphinxVerbatimInput}

\begin{sphinxuseclass}{cell_input}
\begin{sphinxVerbatim}[commandchars=\\\{\}]
\PYG{n}{np}\PYG{o}{.}\PYG{n}{random}\PYG{o}{.}\PYG{n}{seed}\PYG{p}{(}\PYG{l+m+mi}{42}\PYG{p}{)}
\PYG{n}{arr} \PYG{o}{=} \PYG{n}{np}\PYG{o}{.}\PYG{n}{random}\PYG{o}{.}\PYG{n}{randn}\PYG{p}{(}\PYG{l+m+mi}{100}\PYG{p}{)}
\PYG{n}{arr}
\end{sphinxVerbatim}

\end{sphinxuseclass}\end{sphinxVerbatimInput}
\begin{sphinxVerbatimOutput}

\begin{sphinxuseclass}{cell_output}
\begin{sphinxVerbatim}[commandchars=\\\{\}]
array([ 0.4967, \PYGZhy{}0.1383,  0.6477,  1.523 , \PYGZhy{}0.2342, \PYGZhy{}0.2341,  1.5792,
        0.7674, \PYGZhy{}0.4695,  0.5426, \PYGZhy{}0.4634, \PYGZhy{}0.4657,  0.242 , \PYGZhy{}1.9133,
       \PYGZhy{}1.7249, \PYGZhy{}0.5623, \PYGZhy{}1.0128,  0.3142, \PYGZhy{}0.908 , \PYGZhy{}1.4123,  1.4656,
       \PYGZhy{}0.2258,  0.0675, \PYGZhy{}1.4247, \PYGZhy{}0.5444,  0.1109, \PYGZhy{}1.151 ,  0.3757,
       \PYGZhy{}0.6006, \PYGZhy{}0.2917, \PYGZhy{}0.6017,  1.8523, \PYGZhy{}0.0135, \PYGZhy{}1.0577,  0.8225,
       \PYGZhy{}1.2208,  0.2089, \PYGZhy{}1.9597, \PYGZhy{}1.3282,  0.1969,  0.7385,  0.1714,
       \PYGZhy{}0.1156, \PYGZhy{}0.3011, \PYGZhy{}1.4785, \PYGZhy{}0.7198, \PYGZhy{}0.4606,  1.0571,  0.3436,
       \PYGZhy{}1.763 ,  0.3241, \PYGZhy{}0.3851, \PYGZhy{}0.6769,  0.6117,  1.031 ,  0.9313,
       \PYGZhy{}0.8392, \PYGZhy{}0.3092,  0.3313,  0.9755, \PYGZhy{}0.4792, \PYGZhy{}0.1857, \PYGZhy{}1.1063,
       \PYGZhy{}1.1962,  0.8125,  1.3562, \PYGZhy{}0.072 ,  1.0035,  0.3616, \PYGZhy{}0.6451,
        0.3614,  1.538 , \PYGZhy{}0.0358,  1.5646, \PYGZhy{}2.6197,  0.8219,  0.087 ,
       \PYGZhy{}0.299 ,  0.0918, \PYGZhy{}1.9876, \PYGZhy{}0.2197,  0.3571,  1.4779, \PYGZhy{}0.5183,
       \PYGZhy{}0.8085, \PYGZhy{}0.5018,  0.9154,  0.3288, \PYGZhy{}0.5298,  0.5133,  0.0971,
        0.9686, \PYGZhy{}0.7021, \PYGZhy{}0.3277, \PYGZhy{}0.3921, \PYGZhy{}1.4635,  0.2961,  0.2611,
        0.0051, \PYGZhy{}0.2346])
\end{sphinxVerbatim}

\end{sphinxuseclass}\end{sphinxVerbatimOutput}

\end{sphinxuseclass}
\begin{sphinxuseclass}{cell}\begin{sphinxVerbatimInput}

\begin{sphinxuseclass}{cell_input}
\begin{sphinxVerbatim}[commandchars=\\\{\}]
\PYG{n}{arr} \PYG{o}{\PYGZgt{}} \PYG{l+m+mi}{0}
\end{sphinxVerbatim}

\end{sphinxuseclass}\end{sphinxVerbatimInput}
\begin{sphinxVerbatimOutput}

\begin{sphinxuseclass}{cell_output}
\begin{sphinxVerbatim}[commandchars=\\\{\}]
array([ True, False,  True,  True, False, False,  True,  True, False,
        True, False, False,  True, False, False, False, False,  True,
       False, False,  True, False,  True, False, False,  True, False,
        True, False, False, False,  True, False, False,  True, False,
        True, False, False,  True,  True,  True, False, False, False,
       False, False,  True,  True, False,  True, False, False,  True,
        True,  True, False, False,  True,  True, False, False, False,
       False,  True,  True, False,  True,  True, False,  True,  True,
       False,  True, False,  True,  True, False,  True, False, False,
        True,  True, False, False, False,  True,  True, False,  True,
        True,  True, False, False, False, False,  True,  True,  True,
       False])
\end{sphinxVerbatim}

\end{sphinxuseclass}\end{sphinxVerbatimOutput}

\end{sphinxuseclass}
\begin{sphinxuseclass}{cell}\begin{sphinxVerbatimInput}

\begin{sphinxuseclass}{cell_input}
\begin{sphinxVerbatim}[commandchars=\\\{\}]
\PYG{p}{(}\PYG{n}{arr} \PYG{o}{\PYGZgt{}} \PYG{l+m+mi}{0}\PYG{p}{)}\PYG{o}{.}\PYG{n}{sum}\PYG{p}{(}\PYG{p}{)} \PYG{c+c1}{\PYGZsh{} Number of positive values}
\end{sphinxVerbatim}

\end{sphinxuseclass}\end{sphinxVerbatimInput}
\begin{sphinxVerbatimOutput}

\begin{sphinxuseclass}{cell_output}
\begin{sphinxVerbatim}[commandchars=\\\{\}]
46
\end{sphinxVerbatim}

\end{sphinxuseclass}\end{sphinxVerbatimOutput}

\end{sphinxuseclass}
\begin{sphinxuseclass}{cell}\begin{sphinxVerbatimInput}

\begin{sphinxuseclass}{cell_input}
\begin{sphinxVerbatim}[commandchars=\\\{\}]
\PYG{p}{(}\PYG{n}{arr} \PYG{o}{\PYGZgt{}} \PYG{l+m+mi}{0}\PYG{p}{)}\PYG{o}{.}\PYG{n}{mean}\PYG{p}{(}\PYG{p}{)} \PYG{c+c1}{\PYGZsh{} percentage of positive values}
\end{sphinxVerbatim}

\end{sphinxuseclass}\end{sphinxVerbatimInput}
\begin{sphinxVerbatimOutput}

\begin{sphinxuseclass}{cell_output}
\begin{sphinxVerbatim}[commandchars=\\\{\}]
0.4600
\end{sphinxVerbatim}

\end{sphinxuseclass}\end{sphinxVerbatimOutput}

\end{sphinxuseclass}
\begin{sphinxuseclass}{cell}\begin{sphinxVerbatimInput}

\begin{sphinxuseclass}{cell_input}
\begin{sphinxVerbatim}[commandchars=\\\{\}]
\PYG{n}{bools} \PYG{o}{=} \PYG{n}{np}\PYG{o}{.}\PYG{n}{array}\PYG{p}{(}\PYG{p}{[}\PYG{k+kc}{False}\PYG{p}{,} \PYG{k+kc}{False}\PYG{p}{,} \PYG{k+kc}{True}\PYG{p}{,} \PYG{k+kc}{False}\PYG{p}{]}\PYG{p}{)}
\PYG{n}{bools}
\end{sphinxVerbatim}

\end{sphinxuseclass}\end{sphinxVerbatimInput}
\begin{sphinxVerbatimOutput}

\begin{sphinxuseclass}{cell_output}
\begin{sphinxVerbatim}[commandchars=\\\{\}]
array([False, False,  True, False])
\end{sphinxVerbatim}

\end{sphinxuseclass}\end{sphinxVerbatimOutput}

\end{sphinxuseclass}
\begin{sphinxuseclass}{cell}\begin{sphinxVerbatimInput}

\begin{sphinxuseclass}{cell_input}
\begin{sphinxVerbatim}[commandchars=\\\{\}]
\PYG{n}{bools}\PYG{o}{.}\PYG{n}{any}\PYG{p}{(}\PYG{p}{)}
\end{sphinxVerbatim}

\end{sphinxuseclass}\end{sphinxVerbatimInput}
\begin{sphinxVerbatimOutput}

\begin{sphinxuseclass}{cell_output}
\begin{sphinxVerbatim}[commandchars=\\\{\}]
True
\end{sphinxVerbatim}

\end{sphinxuseclass}\end{sphinxVerbatimOutput}

\end{sphinxuseclass}
\begin{sphinxuseclass}{cell}\begin{sphinxVerbatimInput}

\begin{sphinxuseclass}{cell_input}
\begin{sphinxVerbatim}[commandchars=\\\{\}]
\PYG{n}{bools}\PYG{o}{.}\PYG{n}{all}\PYG{p}{(}\PYG{p}{)}
\end{sphinxVerbatim}

\end{sphinxuseclass}\end{sphinxVerbatimInput}
\begin{sphinxVerbatimOutput}

\begin{sphinxuseclass}{cell_output}
\begin{sphinxVerbatim}[commandchars=\\\{\}]
False
\end{sphinxVerbatim}

\end{sphinxuseclass}\end{sphinxVerbatimOutput}

\end{sphinxuseclass}
\sphinxstepscope


\section{McKinney Chapter 4 \sphinxhyphen{} Practice (Blank)}
\label{\detokenize{mckinney_04_practice:mckinney-chapter-4-practice-blank}}\label{\detokenize{mckinney_04_practice::doc}}
\begin{sphinxuseclass}{cell}\begin{sphinxVerbatimInput}

\begin{sphinxuseclass}{cell_input}
\begin{sphinxVerbatim}[commandchars=\\\{\}]
\PYG{k+kn}{import} \PYG{n+nn}{numpy} \PYG{k}{as} \PYG{n+nn}{np}
\PYG{o}{\PYGZpc{}}\PYG{k}{precision} 4
\end{sphinxVerbatim}

\end{sphinxuseclass}\end{sphinxVerbatimInput}
\begin{sphinxVerbatimOutput}

\begin{sphinxuseclass}{cell_output}
\begin{sphinxVerbatim}[commandchars=\\\{\}]
\PYGZsq{}\PYGZpc{}.4f\PYGZsq{}
\end{sphinxVerbatim}

\end{sphinxuseclass}\end{sphinxVerbatimOutput}

\end{sphinxuseclass}

\subsection{Practice}
\label{\detokenize{mckinney_04_practice:practice}}

\subsubsection{Create a 1\sphinxhyphen{}dimensional array named \sphinxstyleliteralintitle{\sphinxupquote{a1}} that counts from 0 to 24 by 1.}
\label{\detokenize{mckinney_04_practice:create-a-1-dimensional-array-named-a1-that-counts-from-0-to-24-by-1}}

\subsubsection{Create a 1\sphinxhyphen{}dimentional array named \sphinxstyleliteralintitle{\sphinxupquote{a2}} that counts from 0 to 24 by 3.}
\label{\detokenize{mckinney_04_practice:create-a-1-dimentional-array-named-a2-that-counts-from-0-to-24-by-3}}

\subsubsection{Create a 1\sphinxhyphen{}dimentional array named \sphinxstyleliteralintitle{\sphinxupquote{a3}} that counts from 0 to 100 by multiples of 3 and 5.}
\label{\detokenize{mckinney_04_practice:create-a-1-dimentional-array-named-a3-that-counts-from-0-to-100-by-multiples-of-3-and-5}}

\subsubsection{Create a 1\sphinxhyphen{}dimensional array \sphinxstyleliteralintitle{\sphinxupquote{a3}} that contains the squares of the even integers through 100,000.}
\label{\detokenize{mckinney_04_practice:create-a-1-dimensional-array-a3-that-contains-the-squares-of-the-even-integers-through-100-000}}
\sphinxAtStartPar
How much faster is the NumPy version than the list comprehension version?


\subsubsection{Write a function that mimic Excel’s \sphinxstyleliteralintitle{\sphinxupquote{pv}} function.}
\label{\detokenize{mckinney_04_practice:write-a-function-that-mimic-excel-s-pv-function}}

\subsubsection{Write a function that mimic Excel’s \sphinxstyleliteralintitle{\sphinxupquote{fv}} function.}
\label{\detokenize{mckinney_04_practice:write-a-function-that-mimic-excel-s-fv-function}}

\subsubsection{Replace the negative values in \sphinxstyleliteralintitle{\sphinxupquote{data}} with \sphinxhyphen{}1 and positive values with +1.}
\label{\detokenize{mckinney_04_practice:replace-the-negative-values-in-data-with-1-and-positive-values-with-1}}
\begin{sphinxuseclass}{cell}\begin{sphinxVerbatimInput}

\begin{sphinxuseclass}{cell_input}
\begin{sphinxVerbatim}[commandchars=\\\{\}]
\PYG{n}{np}\PYG{o}{.}\PYG{n}{random}\PYG{o}{.}\PYG{n}{seed}\PYG{p}{(}\PYG{l+m+mi}{42}\PYG{p}{)}
\PYG{n}{data} \PYG{o}{=} \PYG{n}{np}\PYG{o}{.}\PYG{n}{random}\PYG{o}{.}\PYG{n}{randn}\PYG{p}{(}\PYG{l+m+mi}{7}\PYG{p}{,} \PYG{l+m+mi}{7}\PYG{p}{)}
\PYG{n}{data}
\end{sphinxVerbatim}

\end{sphinxuseclass}\end{sphinxVerbatimInput}
\begin{sphinxVerbatimOutput}

\begin{sphinxuseclass}{cell_output}
\begin{sphinxVerbatim}[commandchars=\\\{\}]
array([[ 0.4967, \PYGZhy{}0.1383,  0.6477,  1.523 , \PYGZhy{}0.2342, \PYGZhy{}0.2341,  1.5792],
       [ 0.7674, \PYGZhy{}0.4695,  0.5426, \PYGZhy{}0.4634, \PYGZhy{}0.4657,  0.242 , \PYGZhy{}1.9133],
       [\PYGZhy{}1.7249, \PYGZhy{}0.5623, \PYGZhy{}1.0128,  0.3142, \PYGZhy{}0.908 , \PYGZhy{}1.4123,  1.4656],
       [\PYGZhy{}0.2258,  0.0675, \PYGZhy{}1.4247, \PYGZhy{}0.5444,  0.1109, \PYGZhy{}1.151 ,  0.3757],
       [\PYGZhy{}0.6006, \PYGZhy{}0.2917, \PYGZhy{}0.6017,  1.8523, \PYGZhy{}0.0135, \PYGZhy{}1.0577,  0.8225],
       [\PYGZhy{}1.2208,  0.2089, \PYGZhy{}1.9597, \PYGZhy{}1.3282,  0.1969,  0.7385,  0.1714],
       [\PYGZhy{}0.1156, \PYGZhy{}0.3011, \PYGZhy{}1.4785, \PYGZhy{}0.7198, \PYGZhy{}0.4606,  1.0571,  0.3436]])
\end{sphinxVerbatim}

\end{sphinxuseclass}\end{sphinxVerbatimOutput}

\end{sphinxuseclass}

\subsubsection{Write a function \sphinxstyleliteralintitle{\sphinxupquote{npmts()}} that calculates the number of payments that generate \$x\%\$ of the present value of a perpetuity.}
\label{\detokenize{mckinney_04_practice:write-a-function-npmts-that-calculates-the-number-of-payments-that-generate-x-of-the-present-value-of-a-perpetuity}}
\sphinxAtStartPar
Your \sphinxcode{\sphinxupquote{npmts()}} should accept arguments \sphinxcode{\sphinxupquote{c1}}, \sphinxcode{\sphinxupquote{r}}, and \sphinxcode{\sphinxupquote{g}} that represent  \$C\_1\$, \$r\$, and \$g\$.
The present value of a growing perpetuity is \$PV = \textbackslash{}frac\{C\_1\}\{r \sphinxhyphen{} g\}\$, and the present value of a growing annuity is \$PV = \textbackslash{}frac\{C\_1\}\{r \sphinxhyphen{} g\}\textbackslash{}left{[} 1 \sphinxhyphen{} \textbackslash{}left( \textbackslash{}frac\{1 + g\}\{1 + r\} \textbackslash{}right)\textasciicircum{}t \textbackslash{}right{]}\$.


\subsubsection{Write a function that calculates the internal rate of return given a NumPy array of cash flows.}
\label{\detokenize{mckinney_04_practice:write-a-function-that-calculates-the-internal-rate-of-return-given-a-numpy-array-of-cash-flows}}

\subsubsection{Write a function \sphinxstyleliteralintitle{\sphinxupquote{returns()}} that accepts \sphinxstyleemphasis{NumPy arrays} of prices and dividends and returns a \sphinxstyleemphasis{NumPy array} of returns.}
\label{\detokenize{mckinney_04_practice:write-a-function-returns-that-accepts-numpy-arrays-of-prices-and-dividends-and-returns-a-numpy-array-of-returns}}
\begin{sphinxuseclass}{cell}\begin{sphinxVerbatimInput}

\begin{sphinxuseclass}{cell_input}
\begin{sphinxVerbatim}[commandchars=\\\{\}]
\PYG{n}{prices} \PYG{o}{=} \PYG{n}{np}\PYG{o}{.}\PYG{n}{array}\PYG{p}{(}\PYG{p}{[}\PYG{l+m+mi}{100}\PYG{p}{,} \PYG{l+m+mi}{150}\PYG{p}{,} \PYG{l+m+mi}{100}\PYG{p}{,} \PYG{l+m+mi}{50}\PYG{p}{,} \PYG{l+m+mi}{100}\PYG{p}{,} \PYG{l+m+mi}{150}\PYG{p}{,} \PYG{l+m+mi}{100}\PYG{p}{,} \PYG{l+m+mi}{150}\PYG{p}{]}\PYG{p}{)}
\PYG{n}{dividends} \PYG{o}{=} \PYG{n}{np}\PYG{o}{.}\PYG{n}{array}\PYG{p}{(}\PYG{p}{[}\PYG{l+m+mi}{1}\PYG{p}{,} \PYG{l+m+mi}{1}\PYG{p}{,} \PYG{l+m+mi}{1}\PYG{p}{,} \PYG{l+m+mi}{1}\PYG{p}{,} \PYG{l+m+mi}{2}\PYG{p}{,} \PYG{l+m+mi}{2}\PYG{p}{,} \PYG{l+m+mi}{2}\PYG{p}{,} \PYG{l+m+mi}{2}\PYG{p}{]}\PYG{p}{)}
\end{sphinxVerbatim}

\end{sphinxuseclass}\end{sphinxVerbatimInput}

\end{sphinxuseclass}

\subsubsection{Rewrite the function \sphinxstyleliteralintitle{\sphinxupquote{returns()}} so it returns \sphinxstyleemphasis{NumPy arrays} of returns, capital gains yields, and dividend yields.}
\label{\detokenize{mckinney_04_practice:rewrite-the-function-returns-so-it-returns-numpy-arrays-of-returns-capital-gains-yields-and-dividend-yields}}

\subsubsection{Rescale and shift numbers so that they cover the range {[}0, 1{]}}
\label{\detokenize{mckinney_04_practice:rescale-and-shift-numbers-so-that-they-cover-the-range-0-1}}
\sphinxAtStartPar
Input: \sphinxcode{\sphinxupquote{np.array({[}18.5, 17.0, 18.0, 19.0, 18.0{]})}} \\
Output: \sphinxcode{\sphinxupquote{np.array({[}0.75, 0.0, 0.5, 1.0, 0.5{]})}}

\begin{sphinxuseclass}{cell}\begin{sphinxVerbatimInput}

\begin{sphinxuseclass}{cell_input}
\begin{sphinxVerbatim}[commandchars=\\\{\}]
\PYG{n}{numbers} \PYG{o}{=} \PYG{n}{np}\PYG{o}{.}\PYG{n}{array}\PYG{p}{(}\PYG{p}{[}\PYG{l+m+mf}{18.5}\PYG{p}{,} \PYG{l+m+mf}{17.0}\PYG{p}{,} \PYG{l+m+mf}{18.0}\PYG{p}{,} \PYG{l+m+mf}{19.0}\PYG{p}{,} \PYG{l+m+mf}{18.0}\PYG{p}{]}\PYG{p}{)}
\end{sphinxVerbatim}

\end{sphinxuseclass}\end{sphinxVerbatimInput}

\end{sphinxuseclass}

\subsubsection{Write functions \sphinxstyleliteralintitle{\sphinxupquote{var()}} and \sphinxstyleliteralintitle{\sphinxupquote{std()}} that calculate variance and standard deviation.}
\label{\detokenize{mckinney_04_practice:write-functions-var-and-std-that-calculate-variance-and-standard-deviation}}
\sphinxAtStartPar
NumPy’s \sphinxcode{\sphinxupquote{.var()}} and \sphinxcode{\sphinxupquote{.std()}} methods return \sphinxstyleemphasis{population} statistics (i.e., denominators of \$n\$).
The pandas equivalents return \sphinxstyleemphasis{sample} statistics (denominators of \$n\sphinxhyphen{}1\$), which are more appropriate for financial data analysis where we have a sample instead of a population.

\sphinxAtStartPar
Both function should have an argument \sphinxcode{\sphinxupquote{sample}} that is \sphinxcode{\sphinxupquote{True}} by default so both functions return sample statistics by default.

\sphinxAtStartPar
Use the output of \sphinxcode{\sphinxupquote{returns()}} to compare your functions with NumPy’s \sphinxcode{\sphinxupquote{.var()}} and \sphinxcode{\sphinxupquote{.std()}} methods.

\sphinxstepscope


\section{McKinney Chapter 4 \sphinxhyphen{} Practice (Section 3, Monday 2:45 PM)}
\label{\detokenize{mckinney_04_practice_03:mckinney-chapter-4-practice-section-3-monday-2-45-pm}}\label{\detokenize{mckinney_04_practice_03::doc}}
\begin{sphinxuseclass}{cell}\begin{sphinxVerbatimInput}

\begin{sphinxuseclass}{cell_input}
\begin{sphinxVerbatim}[commandchars=\\\{\}]
\PYG{k+kn}{import} \PYG{n+nn}{numpy} \PYG{k}{as} \PYG{n+nn}{np}
\PYG{o}{\PYGZpc{}}\PYG{k}{precision} 4
\end{sphinxVerbatim}

\end{sphinxuseclass}\end{sphinxVerbatimInput}
\begin{sphinxVerbatimOutput}

\begin{sphinxuseclass}{cell_output}
\begin{sphinxVerbatim}[commandchars=\\\{\}]
\PYGZsq{}\PYGZpc{}.4f\PYGZsq{}
\end{sphinxVerbatim}

\end{sphinxuseclass}\end{sphinxVerbatimOutput}

\end{sphinxuseclass}

\subsection{Practice}
\label{\detokenize{mckinney_04_practice_03:practice}}

\subsubsection{Create a 1\sphinxhyphen{}dimensional array named \sphinxstyleliteralintitle{\sphinxupquote{a1}} that counts from 0 to 24 by 1.}
\label{\detokenize{mckinney_04_practice_03:create-a-1-dimensional-array-named-a1-that-counts-from-0-to-24-by-1}}
\begin{sphinxuseclass}{cell}\begin{sphinxVerbatimInput}

\begin{sphinxuseclass}{cell_input}
\begin{sphinxVerbatim}[commandchars=\\\{\}]
\PYG{n}{np}\PYG{o}{.}\PYG{n}{array}\PYG{p}{(}\PYG{n+nb}{range}\PYG{p}{(}\PYG{l+m+mi}{25}\PYG{p}{)}\PYG{p}{)}
\end{sphinxVerbatim}

\end{sphinxuseclass}\end{sphinxVerbatimInput}
\begin{sphinxVerbatimOutput}

\begin{sphinxuseclass}{cell_output}
\begin{sphinxVerbatim}[commandchars=\\\{\}]
array([ 0,  1,  2,  3,  4,  5,  6,  7,  8,  9, 10, 11, 12, 13, 14, 15, 16,
       17, 18, 19, 20, 21, 22, 23, 24])
\end{sphinxVerbatim}

\end{sphinxuseclass}\end{sphinxVerbatimOutput}

\end{sphinxuseclass}
\begin{sphinxuseclass}{cell}\begin{sphinxVerbatimInput}

\begin{sphinxuseclass}{cell_input}
\begin{sphinxVerbatim}[commandchars=\\\{\}]
\PYG{n}{a1} \PYG{o}{=} \PYG{n}{np}\PYG{o}{.}\PYG{n}{arange}\PYG{p}{(}\PYG{l+m+mi}{25}\PYG{p}{)}
\PYG{n}{a1}
\end{sphinxVerbatim}

\end{sphinxuseclass}\end{sphinxVerbatimInput}
\begin{sphinxVerbatimOutput}

\begin{sphinxuseclass}{cell_output}
\begin{sphinxVerbatim}[commandchars=\\\{\}]
array([ 0,  1,  2,  3,  4,  5,  6,  7,  8,  9, 10, 11, 12, 13, 14, 15, 16,
       17, 18, 19, 20, 21, 22, 23, 24])
\end{sphinxVerbatim}

\end{sphinxuseclass}\end{sphinxVerbatimOutput}

\end{sphinxuseclass}
\begin{sphinxuseclass}{cell}\begin{sphinxVerbatimInput}

\begin{sphinxuseclass}{cell_input}
\begin{sphinxVerbatim}[commandchars=\\\{\}]
\PYG{n}{a1}\PYG{p}{[}\PYG{l+m+mi}{0}\PYG{p}{]}
\end{sphinxVerbatim}

\end{sphinxuseclass}\end{sphinxVerbatimInput}
\begin{sphinxVerbatimOutput}

\begin{sphinxuseclass}{cell_output}
\begin{sphinxVerbatim}[commandchars=\\\{\}]
0
\end{sphinxVerbatim}

\end{sphinxuseclass}\end{sphinxVerbatimOutput}

\end{sphinxuseclass}
\begin{sphinxuseclass}{cell}\begin{sphinxVerbatimInput}

\begin{sphinxuseclass}{cell_input}
\begin{sphinxVerbatim}[commandchars=\\\{\}]
\PYG{n}{a1}\PYG{p}{[}\PYG{o}{\PYGZhy{}}\PYG{l+m+mi}{1}\PYG{p}{]}
\end{sphinxVerbatim}

\end{sphinxuseclass}\end{sphinxVerbatimInput}
\begin{sphinxVerbatimOutput}

\begin{sphinxuseclass}{cell_output}
\begin{sphinxVerbatim}[commandchars=\\\{\}]
24
\end{sphinxVerbatim}

\end{sphinxuseclass}\end{sphinxVerbatimOutput}

\end{sphinxuseclass}
\sphinxAtStartPar
How can we quickly slice the first five elements in \sphinxcode{\sphinxupquote{a1}}? The next five elements in \sphinxcode{\sphinxupquote{a1}}?

\begin{sphinxuseclass}{cell}\begin{sphinxVerbatimInput}

\begin{sphinxuseclass}{cell_input}
\begin{sphinxVerbatim}[commandchars=\\\{\}]
\PYG{n}{a1}\PYG{p}{[}\PYG{p}{:}\PYG{l+m+mi}{5}\PYG{p}{]}
\end{sphinxVerbatim}

\end{sphinxuseclass}\end{sphinxVerbatimInput}
\begin{sphinxVerbatimOutput}

\begin{sphinxuseclass}{cell_output}
\begin{sphinxVerbatim}[commandchars=\\\{\}]
array([0, 1, 2, 3, 4])
\end{sphinxVerbatim}

\end{sphinxuseclass}\end{sphinxVerbatimOutput}

\end{sphinxuseclass}
\begin{sphinxuseclass}{cell}\begin{sphinxVerbatimInput}

\begin{sphinxuseclass}{cell_input}
\begin{sphinxVerbatim}[commandchars=\\\{\}]
\PYG{n}{a1}\PYG{p}{[}\PYG{l+m+mi}{5}\PYG{p}{:}\PYG{l+m+mi}{10}\PYG{p}{]}
\end{sphinxVerbatim}

\end{sphinxuseclass}\end{sphinxVerbatimInput}
\begin{sphinxVerbatimOutput}

\begin{sphinxuseclass}{cell_output}
\begin{sphinxVerbatim}[commandchars=\\\{\}]
array([5, 6, 7, 8, 9])
\end{sphinxVerbatim}

\end{sphinxuseclass}\end{sphinxVerbatimOutput}

\end{sphinxuseclass}

\subsubsection{Create a 1\sphinxhyphen{}dimentional array named \sphinxstyleliteralintitle{\sphinxupquote{a2}} that counts from 0 to 24 by 3.}
\label{\detokenize{mckinney_04_practice_03:create-a-1-dimentional-array-named-a2-that-counts-from-0-to-24-by-3}}
\begin{sphinxuseclass}{cell}\begin{sphinxVerbatimInput}

\begin{sphinxuseclass}{cell_input}
\begin{sphinxVerbatim}[commandchars=\\\{\}]
\PYG{n}{a2} \PYG{o}{=} \PYG{n}{np}\PYG{o}{.}\PYG{n}{arange}\PYG{p}{(}\PYG{l+m+mi}{0}\PYG{p}{,} \PYG{l+m+mi}{25}\PYG{p}{,} \PYG{l+m+mi}{3}\PYG{p}{)}
\end{sphinxVerbatim}

\end{sphinxuseclass}\end{sphinxVerbatimInput}

\end{sphinxuseclass}
\begin{sphinxuseclass}{cell}\begin{sphinxVerbatimInput}

\begin{sphinxuseclass}{cell_input}
\begin{sphinxVerbatim}[commandchars=\\\{\}]
\PYG{n}{np}\PYG{o}{.}\PYG{n}{arange}\PYG{p}{(}\PYG{l+m+mi}{0}\PYG{p}{,} \PYG{l+m+mi}{27}\PYG{p}{,} \PYG{l+m+mi}{3}\PYG{p}{)}
\end{sphinxVerbatim}

\end{sphinxuseclass}\end{sphinxVerbatimInput}
\begin{sphinxVerbatimOutput}

\begin{sphinxuseclass}{cell_output}
\begin{sphinxVerbatim}[commandchars=\\\{\}]
array([ 0,  3,  6,  9, 12, 15, 18, 21, 24])
\end{sphinxVerbatim}

\end{sphinxuseclass}\end{sphinxVerbatimOutput}

\end{sphinxuseclass}
\sphinxAtStartPar
Here is the “Zen of Python”, which is an easter egg about the philosophy of Python.

\begin{sphinxuseclass}{cell}\begin{sphinxVerbatimInput}

\begin{sphinxuseclass}{cell_input}
\begin{sphinxVerbatim}[commandchars=\\\{\}]
\PYG{k+kn}{import} \PYG{n+nn}{this}
\end{sphinxVerbatim}

\end{sphinxuseclass}\end{sphinxVerbatimInput}
\begin{sphinxVerbatimOutput}

\begin{sphinxuseclass}{cell_output}
\begin{sphinxVerbatim}[commandchars=\\\{\}]
The Zen of Python, by Tim Peters

Beautiful is better than ugly.
Explicit is better than implicit.
Simple is better than complex.
Complex is better than complicated.
Flat is better than nested.
Sparse is better than dense.
Readability counts.
Special cases aren\PYGZsq{}t special enough to break the rules.
Although practicality beats purity.
Errors should never pass silently.
Unless explicitly silenced.
In the face of ambiguity, refuse the temptation to guess.
There should be one\PYGZhy{}\PYGZhy{} and preferably only one \PYGZhy{}\PYGZhy{}obvious way to do it.
Although that way may not be obvious at first unless you\PYGZsq{}re Dutch.
Now is better than never.
Although never is often better than *right* now.
If the implementation is hard to explain, it\PYGZsq{}s a bad idea.
If the implementation is easy to explain, it may be a good idea.
Namespaces are one honking great idea \PYGZhy{}\PYGZhy{} let\PYGZsq{}s do more of those!
\end{sphinxVerbatim}

\end{sphinxuseclass}\end{sphinxVerbatimOutput}

\end{sphinxuseclass}

\subsubsection{Create a 1\sphinxhyphen{}dimentional array named \sphinxstyleliteralintitle{\sphinxupquote{a3}} that counts from 0 to 100 by multiples of 3 and 5.}
\label{\detokenize{mckinney_04_practice_03:create-a-1-dimentional-array-named-a3-that-counts-from-0-to-100-by-multiples-of-3-and-5}}
\begin{sphinxuseclass}{cell}\begin{sphinxVerbatimInput}

\begin{sphinxuseclass}{cell_input}
\begin{sphinxVerbatim}[commandchars=\\\{\}]
\PYG{n}{a3} \PYG{o}{=} \PYG{n}{np}\PYG{o}{.}\PYG{n}{arange}\PYG{p}{(}\PYG{l+m+mi}{100}\PYG{p}{)}
\end{sphinxVerbatim}

\end{sphinxuseclass}\end{sphinxVerbatimInput}

\end{sphinxuseclass}
\begin{sphinxuseclass}{cell}\begin{sphinxVerbatimInput}

\begin{sphinxuseclass}{cell_input}
\begin{sphinxVerbatim}[commandchars=\\\{\}]
\PYG{n}{a3} \PYG{o}{=} \PYG{n}{a3}\PYG{p}{[}\PYG{p}{(}\PYG{n}{a3}\PYG{o}{\PYGZpc{}}\PYG{k}{3}==0) | (a3\PYGZpc{}5==0)]
\PYG{n}{a3}
\end{sphinxVerbatim}

\end{sphinxuseclass}\end{sphinxVerbatimInput}
\begin{sphinxVerbatimOutput}

\begin{sphinxuseclass}{cell_output}
\begin{sphinxVerbatim}[commandchars=\\\{\}]
array([ 0,  3,  5,  6,  9, 10, 12, 15, 18, 20, 21, 24, 25, 27, 30, 33, 35,
       36, 39, 40, 42, 45, 48, 50, 51, 54, 55, 57, 60, 63, 65, 66, 69, 70,
       72, 75, 78, 80, 81, 84, 85, 87, 90, 93, 95, 96, 99])
\end{sphinxVerbatim}

\end{sphinxuseclass}\end{sphinxVerbatimOutput}

\end{sphinxuseclass}
\sphinxAtStartPar
We can also use a list comprehension, then cast the results to be an array.
List comprehensions look like: \sphinxcode{\sphinxupquote{{[}function(i) for i in range(x) if condition on i{]}}}

\begin{sphinxuseclass}{cell}\begin{sphinxVerbatimInput}

\begin{sphinxuseclass}{cell_input}
\begin{sphinxVerbatim}[commandchars=\\\{\}]
\PYG{n}{np}\PYG{o}{.}\PYG{n}{array}\PYG{p}{(}\PYG{p}{[}\PYG{n}{i} \PYG{k}{for} \PYG{n}{i} \PYG{o+ow}{in} \PYG{n+nb}{range}\PYG{p}{(}\PYG{l+m+mi}{100}\PYG{p}{)} \PYG{k}{if} \PYG{p}{(}\PYG{n}{i}\PYG{o}{\PYGZpc{}}\PYG{k}{3}==0) | (i\PYGZpc{}5==0)])
\end{sphinxVerbatim}

\end{sphinxuseclass}\end{sphinxVerbatimInput}
\begin{sphinxVerbatimOutput}

\begin{sphinxuseclass}{cell_output}
\begin{sphinxVerbatim}[commandchars=\\\{\}]
array([ 0,  3,  5,  6,  9, 10, 12, 15, 18, 20, 21, 24, 25, 27, 30, 33, 35,
       36, 39, 40, 42, 45, 48, 50, 51, 54, 55, 57, 60, 63, 65, 66, 69, 70,
       72, 75, 78, 80, 81, 84, 85, 87, 90, 93, 95, 96, 99])
\end{sphinxVerbatim}

\end{sphinxuseclass}\end{sphinxVerbatimOutput}

\end{sphinxuseclass}

\subsubsection{Create a 1\sphinxhyphen{}dimensional array \sphinxstyleliteralintitle{\sphinxupquote{a3}} that contains the squares of the even integers through 100,000.}
\label{\detokenize{mckinney_04_practice_03:create-a-1-dimensional-array-a3-that-contains-the-squares-of-the-even-integers-through-100-000}}
\sphinxAtStartPar
How much faster is the NumPy version than the list comprehension version?

\begin{sphinxuseclass}{cell}\begin{sphinxVerbatimInput}

\begin{sphinxuseclass}{cell_input}
\begin{sphinxVerbatim}[commandchars=\\\{\}]
\PYG{o}{\PYGZpc{}}\PYG{k}{timeit} np.array([i**2 for i in range(0, 100\PYGZus{}000, 2)])
\end{sphinxVerbatim}

\end{sphinxuseclass}\end{sphinxVerbatimInput}
\begin{sphinxVerbatimOutput}

\begin{sphinxuseclass}{cell_output}
\begin{sphinxVerbatim}[commandchars=\\\{\}]
9.91 ms ± 42.5 µs per loop (mean ± std. dev. of 7 runs, 100 loops each)
\end{sphinxVerbatim}

\end{sphinxuseclass}\end{sphinxVerbatimOutput}

\end{sphinxuseclass}
\begin{sphinxuseclass}{cell}\begin{sphinxVerbatimInput}

\begin{sphinxuseclass}{cell_input}
\begin{sphinxVerbatim}[commandchars=\\\{\}]
\PYG{o}{\PYGZpc{}}\PYG{k}{timeit} np.arange(0, 100\PYGZus{}000, 2)**2
\end{sphinxVerbatim}

\end{sphinxuseclass}\end{sphinxVerbatimInput}
\begin{sphinxVerbatimOutput}

\begin{sphinxuseclass}{cell_output}
\begin{sphinxVerbatim}[commandchars=\\\{\}]
19.5 µs ± 144 ns per loop (mean ± std. dev. of 7 runs, 100000 loops each)
\end{sphinxVerbatim}

\end{sphinxuseclass}\end{sphinxVerbatimOutput}

\end{sphinxuseclass}

\bigskip\hrule\bigskip


\sphinxAtStartPar
\sphinxstyleemphasis{\sphinxstylestrong{Note:}}
On some platforms, \sphinxcode{\sphinxupquote{np.arange(0, 100\_001, 2)}} returns an array of 32\sphinxhyphen{}bit integers.
If we square this array of 32\sphinxhyphen{}bit integers, we get the wrong answer because the large values (e.g., \$100,000\textasciicircum{}2\$) are too large to represent as 32\sphinxhyphen{}bit integers.
Since we know that we need 54\sphinxhyphen{}bit integers for this calculation, we should explcitly set either \sphinxcode{\sphinxupquote{dtype='int64'}} or \sphinxcode{\sphinxupquote{dtype=np.int64}}.

\begin{sphinxuseclass}{cell}\begin{sphinxVerbatimInput}

\begin{sphinxuseclass}{cell_input}
\begin{sphinxVerbatim}[commandchars=\\\{\}]
\PYG{n}{np}\PYG{o}{.}\PYG{n}{arange}\PYG{p}{(}\PYG{l+m+mi}{0}\PYG{p}{,} \PYG{l+m+mi}{100\PYGZus{}001}\PYG{p}{,} \PYG{l+m+mi}{2}\PYG{p}{,} \PYG{n}{dtype}\PYG{o}{=}\PYG{l+s+s1}{\PYGZsq{}}\PYG{l+s+s1}{int64}\PYG{l+s+s1}{\PYGZsq{}}\PYG{p}{)}\PYG{o}{*}\PYG{o}{*}\PYG{l+m+mi}{2}
\end{sphinxVerbatim}

\end{sphinxuseclass}\end{sphinxVerbatimInput}
\begin{sphinxVerbatimOutput}

\begin{sphinxuseclass}{cell_output}
\begin{sphinxVerbatim}[commandchars=\\\{\}]
array([          0,           4,          16, ...,  9999200016,
        9999600004, 10000000000])
\end{sphinxVerbatim}

\end{sphinxuseclass}\end{sphinxVerbatimOutput}

\end{sphinxuseclass}
\begin{sphinxuseclass}{cell}\begin{sphinxVerbatimInput}

\begin{sphinxuseclass}{cell_input}
\begin{sphinxVerbatim}[commandchars=\\\{\}]
\PYG{n}{np}\PYG{o}{.}\PYG{n}{arange}\PYG{p}{(}\PYG{l+m+mi}{0}\PYG{p}{,} \PYG{l+m+mi}{100\PYGZus{}001}\PYG{p}{,} \PYG{l+m+mi}{2}\PYG{p}{,} \PYG{n}{dtype}\PYG{o}{=}\PYG{n}{np}\PYG{o}{.}\PYG{n}{int64}\PYG{p}{)}\PYG{o}{*}\PYG{o}{*}\PYG{l+m+mi}{2}
\end{sphinxVerbatim}

\end{sphinxuseclass}\end{sphinxVerbatimInput}
\begin{sphinxVerbatimOutput}

\begin{sphinxuseclass}{cell_output}
\begin{sphinxVerbatim}[commandchars=\\\{\}]
array([          0,           4,          16, ...,  9999200016,
        9999600004, 10000000000])
\end{sphinxVerbatim}

\end{sphinxuseclass}\end{sphinxVerbatimOutput}

\end{sphinxuseclass}
\sphinxAtStartPar
This \sphinxhref{https://stackoverflow.com/a/1970697/334755}{StackOverflow answer} is this best explanation I have found of this behavior.


\bigskip\hrule\bigskip



\subsubsection{Write a function that mimic Excel’s \sphinxstyleliteralintitle{\sphinxupquote{pv}} function.}
\label{\detokenize{mckinney_04_practice_03:write-a-function-that-mimic-excel-s-pv-function}}
\sphinxAtStartPar
Here is how we call Excel’s \sphinxcode{\sphinxupquote{pv}} function:
\sphinxcode{\sphinxupquote{=PV(rate, nper, pmt, {[}fv{]}, {[}type{]})}}
We can use the annuity and lump sum present value formulas.

\sphinxAtStartPar
Present value of an annuity payment \sphinxcode{\sphinxupquote{pmt}}:
\$PV\_\{pmt\} = \textbackslash{}frac\{pmt\}\{rate\} \textbackslash{}times \textbackslash{}left(1 \sphinxhyphen{} \textbackslash{}frac\{1\}\{(1+rate)\textasciicircum{}\{nper\}\} \textbackslash{}right)\$

\sphinxAtStartPar
Present value of a lump sum \sphinxcode{\sphinxupquote{fv}}:
\$PV\_\{fv\} = \textbackslash{}frac\{fv\}\{(1+rate)\textasciicircum{}\{nper\}\}\$

\begin{sphinxuseclass}{cell}\begin{sphinxVerbatimInput}

\begin{sphinxuseclass}{cell_input}
\begin{sphinxVerbatim}[commandchars=\\\{\}]
\PYG{k}{def} \PYG{n+nf}{pv}\PYG{p}{(}\PYG{n}{rate}\PYG{p}{,} \PYG{n}{nper}\PYG{p}{,} \PYG{n}{pmt}\PYG{o}{=}\PYG{k+kc}{None}\PYG{p}{,} \PYG{n}{fv}\PYG{o}{=}\PYG{k+kc}{None}\PYG{p}{,} \PYG{n+nb}{type} \PYG{o}{=} \PYG{l+s+s1}{\PYGZsq{}}\PYG{l+s+s1}{END}\PYG{l+s+s1}{\PYGZsq{}}\PYG{p}{)}\PYG{p}{:}
    \PYG{k}{if} \PYG{n}{pmt} \PYG{o+ow}{is} \PYG{o+ow}{not} \PYG{k+kc}{None}\PYG{p}{:} \PYG{c+c1}{\PYGZsh{} calculate PV of pmt, if given}
        \PYG{n}{pv\PYGZus{}pmt} \PYG{o}{=} \PYG{p}{(}\PYG{n}{pmt} \PYG{o}{/} \PYG{n}{rate}\PYG{p}{)} \PYG{o}{*} \PYG{p}{(}\PYG{l+m+mi}{1} \PYG{o}{\PYGZhy{}} \PYG{p}{(}\PYG{l+m+mi}{1} \PYG{o}{+} \PYG{n}{rate}\PYG{p}{)}\PYG{o}{*}\PYG{o}{*}\PYG{p}{(}\PYG{o}{\PYGZhy{}}\PYG{n}{nper}\PYG{p}{)}\PYG{p}{)}
    \PYG{k}{else}\PYG{p}{:}
        \PYG{n}{pv\PYGZus{}pmt} \PYG{o}{=} \PYG{l+m+mi}{0}

    \PYG{k}{if} \PYG{n}{fv} \PYG{o+ow}{is} \PYG{o+ow}{not} \PYG{k+kc}{None}\PYG{p}{:} \PYG{c+c1}{\PYGZsh{} calculate PV of fv, if given}
        \PYG{n}{pv\PYGZus{}fv} \PYG{o}{=} \PYG{n}{fv} \PYG{o}{/} \PYG{p}{(}\PYG{l+m+mi}{1} \PYG{o}{+} \PYG{n}{rate}\PYG{p}{)}\PYG{o}{*}\PYG{o}{*}\PYG{n}{nper}
    \PYG{k}{else}\PYG{p}{:}
        \PYG{n}{pv\PYGZus{}fv} \PYG{o}{=} \PYG{l+m+mi}{0}
    
    \PYG{k}{if} \PYG{n+nb}{type}\PYG{o}{==}\PYG{l+s+s1}{\PYGZsq{}}\PYG{l+s+s1}{END}\PYG{l+s+s1}{\PYGZsq{}}\PYG{p}{:} \PYG{c+c1}{\PYGZsh{} as\PYGZhy{}is if end of period payments}
        \PYG{k}{return} \PYG{o}{\PYGZhy{}}\PYG{l+m+mi}{1} \PYG{o}{*} \PYG{p}{(}\PYG{n}{pv\PYGZus{}pmt} \PYG{o}{+} \PYG{n}{pv\PYGZus{}fv}\PYG{p}{)}
    \PYG{k}{elif} \PYG{n+nb}{type}\PYG{o}{==}\PYG{l+s+s1}{\PYGZsq{}}\PYG{l+s+s1}{BGN}\PYG{l+s+s1}{\PYGZsq{}}\PYG{p}{:} \PYG{c+c1}{\PYGZsh{} undo one period of discounting if bgn of period payments}
        \PYG{k}{return} \PYG{o}{\PYGZhy{}}\PYG{l+m+mi}{1} \PYG{o}{*} \PYG{p}{(}\PYG{n}{pv\PYGZus{}pmt} \PYG{o}{+} \PYG{n}{pv\PYGZus{}fv}\PYG{p}{)} \PYG{o}{*} \PYG{p}{(}\PYG{l+m+mi}{1} \PYG{o}{+} \PYG{n}{rate}\PYG{p}{)}
    \PYG{k}{else}\PYG{p}{:} \PYG{c+c1}{\PYGZsh{} otherwise, ask use to specify end or bgn of period payments}
        \PYG{n+nb}{print}\PYG{p}{(}\PYG{l+s+s1}{\PYGZsq{}}\PYG{l+s+s1}{Please enter END or BGN for the type argument}\PYG{l+s+s1}{\PYGZsq{}}\PYG{p}{)}
\end{sphinxVerbatim}

\end{sphinxuseclass}\end{sphinxVerbatimInput}

\end{sphinxuseclass}
\begin{sphinxuseclass}{cell}\begin{sphinxVerbatimInput}

\begin{sphinxuseclass}{cell_input}
\begin{sphinxVerbatim}[commandchars=\\\{\}]
\PYG{n}{pv}\PYG{p}{(}\PYG{n}{rate}\PYG{o}{=}\PYG{l+m+mf}{0.1}\PYG{p}{,} \PYG{n}{nper}\PYG{o}{=}\PYG{l+m+mi}{10}\PYG{p}{,} \PYG{n}{pmt}\PYG{o}{=}\PYG{l+m+mi}{100}\PYG{p}{,} \PYG{n}{fv}\PYG{o}{=}\PYG{l+m+mi}{1000}\PYG{p}{,} \PYG{n+nb}{type}\PYG{o}{=}\PYG{l+s+s1}{\PYGZsq{}}\PYG{l+s+s1}{END}\PYG{l+s+s1}{\PYGZsq{}}\PYG{p}{)}
\end{sphinxVerbatim}

\end{sphinxuseclass}\end{sphinxVerbatimInput}
\begin{sphinxVerbatimOutput}

\begin{sphinxuseclass}{cell_output}
\begin{sphinxVerbatim}[commandchars=\\\{\}]
\PYGZhy{}1000.0000
\end{sphinxVerbatim}

\end{sphinxuseclass}\end{sphinxVerbatimOutput}

\end{sphinxuseclass}

\subsubsection{Write a function that mimic Excel’s \sphinxstyleliteralintitle{\sphinxupquote{fv}} function.}
\label{\detokenize{mckinney_04_practice_03:write-a-function-that-mimic-excel-s-fv-function}}
\begin{sphinxuseclass}{cell}\begin{sphinxVerbatimInput}

\begin{sphinxuseclass}{cell_input}
\begin{sphinxVerbatim}[commandchars=\\\{\}]
\PYG{k}{def} \PYG{n+nf}{fv}\PYG{p}{(}\PYG{n}{rate}\PYG{p}{,} \PYG{n}{nper}\PYG{p}{,} \PYG{n}{pmt}\PYG{o}{=}\PYG{k+kc}{None}\PYG{p}{,} \PYG{n}{pv}\PYG{o}{=}\PYG{k+kc}{None}\PYG{p}{,} \PYG{n+nb}{type} \PYG{o}{=} \PYG{l+s+s1}{\PYGZsq{}}\PYG{l+s+s1}{END}\PYG{l+s+s1}{\PYGZsq{}}\PYG{p}{)}\PYG{p}{:}
    \PYG{k}{if} \PYG{n}{pmt} \PYG{o+ow}{is} \PYG{o+ow}{not} \PYG{k+kc}{None}\PYG{p}{:} \PYG{c+c1}{\PYGZsh{} calculate PV of pmt, if given}
        \PYG{n}{fv\PYGZus{}pmt} \PYG{o}{=} \PYG{p}{(}\PYG{n}{pmt} \PYG{o}{/} \PYG{n}{rate}\PYG{p}{)} \PYG{o}{*} \PYG{p}{(}\PYG{p}{(}\PYG{l+m+mi}{1} \PYG{o}{+} \PYG{n}{rate}\PYG{p}{)}\PYG{o}{*}\PYG{o}{*}\PYG{n}{nper} \PYG{o}{\PYGZhy{}} \PYG{l+m+mi}{1}\PYG{p}{)}
    \PYG{k}{else}\PYG{p}{:}
        \PYG{n}{fv\PYGZus{}pmt} \PYG{o}{=} \PYG{l+m+mi}{0}

    \PYG{k}{if} \PYG{n}{fv} \PYG{o+ow}{is} \PYG{o+ow}{not} \PYG{k+kc}{None}\PYG{p}{:} \PYG{c+c1}{\PYGZsh{} calculate PV of fv, if given}
        \PYG{n}{fv\PYGZus{}pv} \PYG{o}{=} \PYG{n}{pv} \PYG{o}{*} \PYG{p}{(}\PYG{l+m+mi}{1} \PYG{o}{+} \PYG{n}{rate}\PYG{p}{)}\PYG{o}{*}\PYG{o}{*}\PYG{n}{nper}
    \PYG{k}{else}\PYG{p}{:}
        \PYG{n}{fv\PYGZus{}fv} \PYG{o}{=} \PYG{l+m+mi}{0}
    
    \PYG{k}{if} \PYG{n+nb}{type}\PYG{o}{==}\PYG{l+s+s1}{\PYGZsq{}}\PYG{l+s+s1}{END}\PYG{l+s+s1}{\PYGZsq{}}\PYG{p}{:} \PYG{c+c1}{\PYGZsh{} as\PYGZhy{}is if end of period payments}
        \PYG{k}{return} \PYG{o}{\PYGZhy{}}\PYG{l+m+mi}{1} \PYG{o}{*} \PYG{p}{(}\PYG{n}{fv\PYGZus{}pmt} \PYG{o}{+} \PYG{n}{fv\PYGZus{}pv}\PYG{p}{)}
    \PYG{k}{elif} \PYG{n+nb}{type}\PYG{o}{==}\PYG{l+s+s1}{\PYGZsq{}}\PYG{l+s+s1}{BGN}\PYG{l+s+s1}{\PYGZsq{}}\PYG{p}{:} \PYG{c+c1}{\PYGZsh{} undo one period of discounting if bgn of period payments}
        \PYG{k}{return} \PYG{o}{\PYGZhy{}}\PYG{l+m+mi}{1} \PYG{o}{*} \PYG{p}{(}\PYG{n}{fv\PYGZus{}pmt} \PYG{o}{+} \PYG{n}{fv\PYGZus{}pv}\PYG{p}{)} \PYG{o}{*} \PYG{p}{(}\PYG{l+m+mi}{1} \PYG{o}{+} \PYG{n}{rate}\PYG{p}{)}
    \PYG{k}{else}\PYG{p}{:} \PYG{c+c1}{\PYGZsh{} otherwise, ask use to specify end or bgn of period payments}
        \PYG{n+nb}{print}\PYG{p}{(}\PYG{l+s+s1}{\PYGZsq{}}\PYG{l+s+s1}{Please enter END or BGN for the type argument}\PYG{l+s+s1}{\PYGZsq{}}\PYG{p}{)}
\end{sphinxVerbatim}

\end{sphinxuseclass}\end{sphinxVerbatimInput}

\end{sphinxuseclass}
\begin{sphinxuseclass}{cell}\begin{sphinxVerbatimInput}

\begin{sphinxuseclass}{cell_input}
\begin{sphinxVerbatim}[commandchars=\\\{\}]
\PYG{n}{fv}\PYG{p}{(}\PYG{n}{rate}\PYG{o}{=}\PYG{l+m+mf}{0.1}\PYG{p}{,} \PYG{n}{nper}\PYG{o}{=}\PYG{l+m+mi}{10}\PYG{p}{,} \PYG{n}{pmt}\PYG{o}{=}\PYG{l+m+mi}{100}\PYG{p}{,} \PYG{n}{pv}\PYG{o}{=}\PYG{o}{\PYGZhy{}}\PYG{l+m+mi}{1000}\PYG{p}{,} \PYG{n+nb}{type}\PYG{o}{=}\PYG{l+s+s1}{\PYGZsq{}}\PYG{l+s+s1}{END}\PYG{l+s+s1}{\PYGZsq{}}\PYG{p}{)}
\end{sphinxVerbatim}

\end{sphinxuseclass}\end{sphinxVerbatimInput}
\begin{sphinxVerbatimOutput}

\begin{sphinxuseclass}{cell_output}
\begin{sphinxVerbatim}[commandchars=\\\{\}]
1000.0000
\end{sphinxVerbatim}

\end{sphinxuseclass}\end{sphinxVerbatimOutput}

\end{sphinxuseclass}

\subsubsection{Replace the negative values in \sphinxstyleliteralintitle{\sphinxupquote{data}} with \sphinxhyphen{}1 and positive values with +1.}
\label{\detokenize{mckinney_04_practice_03:replace-the-negative-values-in-data-with-1-and-positive-values-with-1}}
\begin{sphinxuseclass}{cell}\begin{sphinxVerbatimInput}

\begin{sphinxuseclass}{cell_input}
\begin{sphinxVerbatim}[commandchars=\\\{\}]
\PYG{n}{np}\PYG{o}{.}\PYG{n}{random}\PYG{o}{.}\PYG{n}{seed}\PYG{p}{(}\PYG{l+m+mi}{42}\PYG{p}{)}
\PYG{n}{data} \PYG{o}{=} \PYG{n}{np}\PYG{o}{.}\PYG{n}{random}\PYG{o}{.}\PYG{n}{randn}\PYG{p}{(}\PYG{l+m+mi}{7}\PYG{p}{,} \PYG{l+m+mi}{7}\PYG{p}{)}
\PYG{n}{data}
\end{sphinxVerbatim}

\end{sphinxuseclass}\end{sphinxVerbatimInput}
\begin{sphinxVerbatimOutput}

\begin{sphinxuseclass}{cell_output}
\begin{sphinxVerbatim}[commandchars=\\\{\}]
array([[ 0.4967, \PYGZhy{}0.1383,  0.6477,  1.523 , \PYGZhy{}0.2342, \PYGZhy{}0.2341,  1.5792],
       [ 0.7674, \PYGZhy{}0.4695,  0.5426, \PYGZhy{}0.4634, \PYGZhy{}0.4657,  0.242 , \PYGZhy{}1.9133],
       [\PYGZhy{}1.7249, \PYGZhy{}0.5623, \PYGZhy{}1.0128,  0.3142, \PYGZhy{}0.908 , \PYGZhy{}1.4123,  1.4656],
       [\PYGZhy{}0.2258,  0.0675, \PYGZhy{}1.4247, \PYGZhy{}0.5444,  0.1109, \PYGZhy{}1.151 ,  0.3757],
       [\PYGZhy{}0.6006, \PYGZhy{}0.2917, \PYGZhy{}0.6017,  1.8523, \PYGZhy{}0.0135, \PYGZhy{}1.0577,  0.8225],
       [\PYGZhy{}1.2208,  0.2089, \PYGZhy{}1.9597, \PYGZhy{}1.3282,  0.1969,  0.7385,  0.1714],
       [\PYGZhy{}0.1156, \PYGZhy{}0.3011, \PYGZhy{}1.4785, \PYGZhy{}0.7198, \PYGZhy{}0.4606,  1.0571,  0.3436]])
\end{sphinxVerbatim}

\end{sphinxuseclass}\end{sphinxVerbatimOutput}

\end{sphinxuseclass}
\begin{sphinxuseclass}{cell}\begin{sphinxVerbatimInput}

\begin{sphinxuseclass}{cell_input}
\begin{sphinxVerbatim}[commandchars=\\\{\}]
\PYG{n}{data2} \PYG{o}{=} \PYG{n}{data}\PYG{o}{.}\PYG{n}{copy}\PYG{p}{(}\PYG{p}{)}
\PYG{n}{data2}\PYG{p}{[}\PYG{n}{data2} \PYG{o}{\PYGZlt{}} \PYG{l+m+mi}{0}\PYG{p}{]} \PYG{o}{=} \PYG{o}{\PYGZhy{}}\PYG{l+m+mi}{1}
\PYG{n}{data2}\PYG{p}{[}\PYG{n}{data2} \PYG{o}{\PYGZgt{}} \PYG{l+m+mi}{0}\PYG{p}{]} \PYG{o}{=} \PYG{o}{+}\PYG{l+m+mi}{1}
\PYG{n}{data2}
\end{sphinxVerbatim}

\end{sphinxuseclass}\end{sphinxVerbatimInput}
\begin{sphinxVerbatimOutput}

\begin{sphinxuseclass}{cell_output}
\begin{sphinxVerbatim}[commandchars=\\\{\}]
array([[ 1., \PYGZhy{}1.,  1.,  1., \PYGZhy{}1., \PYGZhy{}1.,  1.],
       [ 1., \PYGZhy{}1.,  1., \PYGZhy{}1., \PYGZhy{}1.,  1., \PYGZhy{}1.],
       [\PYGZhy{}1., \PYGZhy{}1., \PYGZhy{}1.,  1., \PYGZhy{}1., \PYGZhy{}1.,  1.],
       [\PYGZhy{}1.,  1., \PYGZhy{}1., \PYGZhy{}1.,  1., \PYGZhy{}1.,  1.],
       [\PYGZhy{}1., \PYGZhy{}1., \PYGZhy{}1.,  1., \PYGZhy{}1., \PYGZhy{}1.,  1.],
       [\PYGZhy{}1.,  1., \PYGZhy{}1., \PYGZhy{}1.,  1.,  1.,  1.],
       [\PYGZhy{}1., \PYGZhy{}1., \PYGZhy{}1., \PYGZhy{}1., \PYGZhy{}1.,  1.,  1.]])
\end{sphinxVerbatim}

\end{sphinxuseclass}\end{sphinxVerbatimOutput}

\end{sphinxuseclass}
\sphinxAtStartPar
We could also use \sphinxcode{\sphinxupquote{np.select()}}.

\begin{sphinxuseclass}{cell}\begin{sphinxVerbatimInput}

\begin{sphinxuseclass}{cell_input}
\begin{sphinxVerbatim}[commandchars=\\\{\}]
\PYG{n}{data3} \PYG{o}{=} \PYG{n}{np}\PYG{o}{.}\PYG{n}{select}\PYG{p}{(}
    \PYG{n}{condlist}\PYG{o}{=}\PYG{p}{[}\PYG{n}{data}\PYG{o}{\PYGZlt{}}\PYG{l+m+mi}{0}\PYG{p}{,} \PYG{n}{data}\PYG{o}{\PYGZgt{}}\PYG{l+m+mi}{0}\PYG{p}{]}\PYG{p}{,}
    \PYG{n}{choicelist}\PYG{o}{=}\PYG{p}{[}\PYG{o}{\PYGZhy{}}\PYG{l+m+mi}{1}\PYG{p}{,} \PYG{l+m+mi}{1}\PYG{p}{]}\PYG{p}{,}
    \PYG{n}{default}\PYG{o}{=}\PYG{l+m+mi}{0}
\PYG{p}{)}
\PYG{n}{data3}
\end{sphinxVerbatim}

\end{sphinxuseclass}\end{sphinxVerbatimInput}
\begin{sphinxVerbatimOutput}

\begin{sphinxuseclass}{cell_output}
\begin{sphinxVerbatim}[commandchars=\\\{\}]
array([[ 1, \PYGZhy{}1,  1,  1, \PYGZhy{}1, \PYGZhy{}1,  1],
       [ 1, \PYGZhy{}1,  1, \PYGZhy{}1, \PYGZhy{}1,  1, \PYGZhy{}1],
       [\PYGZhy{}1, \PYGZhy{}1, \PYGZhy{}1,  1, \PYGZhy{}1, \PYGZhy{}1,  1],
       [\PYGZhy{}1,  1, \PYGZhy{}1, \PYGZhy{}1,  1, \PYGZhy{}1,  1],
       [\PYGZhy{}1, \PYGZhy{}1, \PYGZhy{}1,  1, \PYGZhy{}1, \PYGZhy{}1,  1],
       [\PYGZhy{}1,  1, \PYGZhy{}1, \PYGZhy{}1,  1,  1,  1],
       [\PYGZhy{}1, \PYGZhy{}1, \PYGZhy{}1, \PYGZhy{}1, \PYGZhy{}1,  1,  1]])
\end{sphinxVerbatim}

\end{sphinxuseclass}\end{sphinxVerbatimOutput}

\end{sphinxuseclass}
\begin{sphinxuseclass}{cell}\begin{sphinxVerbatimInput}

\begin{sphinxuseclass}{cell_input}
\begin{sphinxVerbatim}[commandchars=\\\{\}]
\PYG{n}{np}\PYG{o}{.}\PYG{n}{allclose}\PYG{p}{(}\PYG{n}{data2}\PYG{p}{,} \PYG{n}{data3}\PYG{p}{)}
\end{sphinxVerbatim}

\end{sphinxuseclass}\end{sphinxVerbatimInput}
\begin{sphinxVerbatimOutput}

\begin{sphinxuseclass}{cell_output}
\begin{sphinxVerbatim}[commandchars=\\\{\}]
True
\end{sphinxVerbatim}

\end{sphinxuseclass}\end{sphinxVerbatimOutput}

\end{sphinxuseclass}

\subsubsection{Write a function \sphinxstyleliteralintitle{\sphinxupquote{npmts()}} that calculates the number of payments that generate \$x\%\$ of the present value of a perpetuity.}
\label{\detokenize{mckinney_04_practice_03:write-a-function-npmts-that-calculates-the-number-of-payments-that-generate-x-of-the-present-value-of-a-perpetuity}}
\sphinxAtStartPar
Your \sphinxcode{\sphinxupquote{npmts()}} should accept arguments \sphinxcode{\sphinxupquote{c1}}, \sphinxcode{\sphinxupquote{r}}, and \sphinxcode{\sphinxupquote{g}} that represent  \$C\_1\$, \$r\$, and \$g\$.
The present value of a growing perpetuity is \$PV = \textbackslash{}frac\{C\_1\}\{r \sphinxhyphen{} g\}\$, and the present value of a growing annuity is \$PV = \textbackslash{}frac\{C\_1\}\{r \sphinxhyphen{} g\}\textbackslash{}left{[} 1 \sphinxhyphen{} \textbackslash{}left( \textbackslash{}frac\{1 + g\}\{1 + r\} \textbackslash{}right)\textasciicircum{}t \textbackslash{}right{]}\$.

\sphinxAtStartPar
We can use the growing annuity and perpetuity formulas to show: \$x = \textbackslash{}left{[} 1 \sphinxhyphen{} \textbackslash{}left( \textbackslash{}frac\{1 + g\}\{1 + r\} \textbackslash{}right)\textasciicircum{}t \textbackslash{}right{]}\$.

\sphinxAtStartPar
Then: \$1 \sphinxhyphen{} x = \textbackslash{}left( \textbackslash{}frac\{1 + g\}\{1 + r\} \textbackslash{}right)\textasciicircum{}t\$.

\sphinxAtStartPar
Finally: \$t = \textbackslash{}frac\{\textbackslash{}log(1\sphinxhyphen{}x)\}\{\textbackslash{}log\textbackslash{}left(\textbackslash{}frac\{1 + g\}\{1 + r\}\textbackslash{}right)\}\$

\sphinxAtStartPar
\sphinxstyleemphasis{\sphinxstylestrong{We do not need to accept an argument \sphinxcode{\sphinxupquote{c1}} because \$C\_1\$ cancels out!}}

\begin{sphinxuseclass}{cell}\begin{sphinxVerbatimInput}

\begin{sphinxuseclass}{cell_input}
\begin{sphinxVerbatim}[commandchars=\\\{\}]
\PYG{k}{def} \PYG{n+nf}{npmts}\PYG{p}{(}\PYG{n}{x}\PYG{p}{,} \PYG{n}{r}\PYG{p}{,} \PYG{n}{g}\PYG{p}{)}\PYG{p}{:}
    \PYG{k}{return} \PYG{n}{np}\PYG{o}{.}\PYG{n}{log}\PYG{p}{(}\PYG{l+m+mi}{1}\PYG{o}{\PYGZhy{}}\PYG{n}{x}\PYG{p}{)} \PYG{o}{/} \PYG{n}{np}\PYG{o}{.}\PYG{n}{log}\PYG{p}{(}\PYG{p}{(}\PYG{l+m+mi}{1} \PYG{o}{+} \PYG{n}{g}\PYG{p}{)} \PYG{o}{/} \PYG{p}{(}\PYG{l+m+mi}{1} \PYG{o}{+} \PYG{n}{r}\PYG{p}{)}\PYG{p}{)}
\end{sphinxVerbatim}

\end{sphinxuseclass}\end{sphinxVerbatimInput}

\end{sphinxuseclass}
\begin{sphinxuseclass}{cell}\begin{sphinxVerbatimInput}

\begin{sphinxuseclass}{cell_input}
\begin{sphinxVerbatim}[commandchars=\\\{\}]
\PYG{n}{npmts}\PYG{p}{(}\PYG{l+m+mf}{0.5}\PYG{p}{,} \PYG{l+m+mf}{0.1}\PYG{p}{,} \PYG{l+m+mf}{0.05}\PYG{p}{)}
\end{sphinxVerbatim}

\end{sphinxuseclass}\end{sphinxVerbatimInput}
\begin{sphinxVerbatimOutput}

\begin{sphinxuseclass}{cell_output}
\begin{sphinxVerbatim}[commandchars=\\\{\}]
14.899977377480532
\end{sphinxVerbatim}

\end{sphinxuseclass}\end{sphinxVerbatimOutput}

\end{sphinxuseclass}

\subsubsection{Write a function that calculates the internal rate of return given a NumPy array of cash flows.}
\label{\detokenize{mckinney_04_practice_03:write-a-function-that-calculates-the-internal-rate-of-return-given-a-numpy-array-of-cash-flows}}
\sphinxAtStartPar
Here are some data where the \$IRR\$ is obvious!

\begin{sphinxuseclass}{cell}\begin{sphinxVerbatimInput}

\begin{sphinxuseclass}{cell_input}
\begin{sphinxVerbatim}[commandchars=\\\{\}]
\PYG{n}{c} \PYG{o}{=} \PYG{n}{np}\PYG{o}{.}\PYG{n}{array}\PYG{p}{(}\PYG{p}{[}\PYG{o}{\PYGZhy{}}\PYG{l+m+mi}{100}\PYG{p}{,} \PYG{l+m+mi}{110}\PYG{p}{]}\PYG{p}{)}
\PYG{n}{irr} \PYG{o}{=} \PYG{l+m+mf}{0.1}
\end{sphinxVerbatim}

\end{sphinxuseclass}\end{sphinxVerbatimInput}

\end{sphinxuseclass}
\sphinxAtStartPar
We want to replicate the following calculation with NumPy:

\begin{sphinxuseclass}{cell}\begin{sphinxVerbatimInput}

\begin{sphinxuseclass}{cell_input}
\begin{sphinxVerbatim}[commandchars=\\\{\}]
\PYG{n}{c}\PYG{p}{[}\PYG{l+m+mi}{0}\PYG{p}{]}\PYG{o}{/}\PYG{p}{(}\PYG{l+m+mi}{1}\PYG{o}{+}\PYG{n}{irr}\PYG{p}{)}\PYG{o}{*}\PYG{o}{*}\PYG{l+m+mi}{0} \PYG{o}{+} \PYG{n}{c}\PYG{p}{[}\PYG{l+m+mi}{1}\PYG{p}{]}\PYG{o}{/}\PYG{p}{(}\PYG{l+m+mi}{1}\PYG{o}{+}\PYG{n}{irr}\PYG{p}{)}\PYG{o}{*}\PYG{o}{*}\PYG{l+m+mi}{1}
\end{sphinxVerbatim}

\end{sphinxuseclass}\end{sphinxVerbatimInput}
\begin{sphinxVerbatimOutput}

\begin{sphinxuseclass}{cell_output}
\begin{sphinxVerbatim}[commandchars=\\\{\}]
\PYGZhy{}1.4210854715202004e\PYGZhy{}14
\end{sphinxVerbatim}

\end{sphinxuseclass}\end{sphinxVerbatimOutput}

\end{sphinxuseclass}
\sphinxAtStartPar
The following NumPy code calculates the present value interest factor for each cash flow:

\begin{sphinxuseclass}{cell}\begin{sphinxVerbatimInput}

\begin{sphinxuseclass}{cell_input}
\begin{sphinxVerbatim}[commandchars=\\\{\}]
\PYG{l+m+mi}{1} \PYG{o}{/} \PYG{p}{(}\PYG{l+m+mi}{1} \PYG{o}{+} \PYG{n}{irr}\PYG{p}{)}\PYG{o}{*}\PYG{o}{*}\PYG{n}{np}\PYG{o}{.}\PYG{n}{arange}\PYG{p}{(}\PYG{n+nb}{len}\PYG{p}{(}\PYG{n}{c}\PYG{p}{)}\PYG{p}{)} \PYG{c+c1}{\PYGZsh{} present value interest factor}
\end{sphinxVerbatim}

\end{sphinxuseclass}\end{sphinxVerbatimInput}
\begin{sphinxVerbatimOutput}

\begin{sphinxuseclass}{cell_output}
\begin{sphinxVerbatim}[commandchars=\\\{\}]
array([1.    , 0.9091])
\end{sphinxVerbatim}

\end{sphinxuseclass}\end{sphinxVerbatimOutput}

\end{sphinxuseclass}
\sphinxAtStartPar
The following NumPy code calculates the present value for each cash flow:

\begin{sphinxuseclass}{cell}\begin{sphinxVerbatimInput}

\begin{sphinxuseclass}{cell_input}
\begin{sphinxVerbatim}[commandchars=\\\{\}]
\PYG{n}{c} \PYG{o}{/} \PYG{p}{(}\PYG{l+m+mi}{1} \PYG{o}{+} \PYG{n}{irr}\PYG{p}{)}\PYG{o}{*}\PYG{o}{*}\PYG{n}{np}\PYG{o}{.}\PYG{n}{arange}\PYG{p}{(}\PYG{n+nb}{len}\PYG{p}{(}\PYG{n}{c}\PYG{p}{)}\PYG{p}{)} \PYG{c+c1}{\PYGZsh{} present value of each cash flow}
\end{sphinxVerbatim}

\end{sphinxuseclass}\end{sphinxVerbatimInput}
\begin{sphinxVerbatimOutput}

\begin{sphinxuseclass}{cell_output}
\begin{sphinxVerbatim}[commandchars=\\\{\}]
array([\PYGZhy{}100.,  100.])
\end{sphinxVerbatim}

\end{sphinxuseclass}\end{sphinxVerbatimOutput}

\end{sphinxuseclass}
\sphinxAtStartPar
We sum these present values of each cash flow to calculate the net present value:

\begin{sphinxuseclass}{cell}\begin{sphinxVerbatimInput}

\begin{sphinxuseclass}{cell_input}
\begin{sphinxVerbatim}[commandchars=\\\{\}]
\PYG{p}{(}\PYG{n}{c} \PYG{o}{/} \PYG{p}{(}\PYG{l+m+mi}{1} \PYG{o}{+} \PYG{n}{irr}\PYG{p}{)}\PYG{o}{*}\PYG{o}{*}\PYG{n}{np}\PYG{o}{.}\PYG{n}{arange}\PYG{p}{(}\PYG{n+nb}{len}\PYG{p}{(}\PYG{n}{c}\PYG{p}{)}\PYG{p}{)}\PYG{p}{)}\PYG{o}{.}\PYG{n}{sum}\PYG{p}{(}\PYG{p}{)}
\end{sphinxVerbatim}

\end{sphinxuseclass}\end{sphinxVerbatimInput}
\begin{sphinxVerbatimOutput}

\begin{sphinxuseclass}{cell_output}
\begin{sphinxVerbatim}[commandchars=\\\{\}]
\PYGZhy{}1.4210854715202004e\PYGZhy{}14
\end{sphinxVerbatim}

\end{sphinxuseclass}\end{sphinxVerbatimOutput}

\end{sphinxuseclass}
\sphinxAtStartPar
The \$IRR\$ is the discount rate where \$NPV=0\$.
We can can use the NumPy code above to try different discount rates until \$NPV=0\$.
The following code is crude, but sufficient for week three of our class and highlights two tools:
\begin{enumerate}
\sphinxsetlistlabels{\arabic}{enumi}{enumii}{}{.}%
\item {} 
\sphinxAtStartPar
Using a \sphinxcode{\sphinxupquote{while}} to try different discount rates until our answer is within some tolerance

\item {} 
\sphinxAtStartPar
Using NumPy to perform repetitive calculations without a \sphinxcode{\sphinxupquote{for}} loop

\end{enumerate}

\begin{sphinxuseclass}{cell}\begin{sphinxVerbatimInput}

\begin{sphinxuseclass}{cell_input}
\begin{sphinxVerbatim}[commandchars=\\\{\}]
\PYG{k}{def} \PYG{n+nf}{irr}\PYG{p}{(}\PYG{n}{c}\PYG{p}{,} \PYG{n}{guess}\PYG{o}{=}\PYG{l+m+mi}{0}\PYG{p}{,} \PYG{n}{tol}\PYG{o}{=}\PYG{l+m+mf}{0.0001}\PYG{p}{,} \PYG{n}{inc}\PYG{o}{=}\PYG{l+m+mf}{0.0001}\PYG{p}{)}\PYG{p}{:}
    \PYG{n}{npv} \PYG{o}{=} \PYG{l+m+mi}{42}
    \PYG{n}{irr} \PYG{o}{=} \PYG{n}{guess}
    \PYG{k}{while} \PYG{n}{np}\PYG{o}{.}\PYG{n}{abs}\PYG{p}{(}\PYG{n}{npv}\PYG{p}{)} \PYG{o}{\PYGZgt{}} \PYG{n}{tol}\PYG{p}{:}
        \PYG{n}{irr} \PYG{o}{+}\PYG{o}{=} \PYG{n}{inc}
        \PYG{n}{npv} \PYG{o}{=} \PYG{p}{(}\PYG{n}{c} \PYG{o}{/} \PYG{p}{(}\PYG{l+m+mi}{1} \PYG{o}{+} \PYG{n}{irr}\PYG{p}{)}\PYG{o}{*}\PYG{o}{*}\PYG{n}{np}\PYG{o}{.}\PYG{n}{arange}\PYG{p}{(}\PYG{n+nb}{len}\PYG{p}{(}\PYG{n}{c}\PYG{p}{)}\PYG{p}{)}\PYG{p}{)}\PYG{o}{.}\PYG{n}{sum}\PYG{p}{(}\PYG{p}{)}
        
    \PYG{k}{return} \PYG{n}{irr}
\end{sphinxVerbatim}

\end{sphinxuseclass}\end{sphinxVerbatimInput}

\end{sphinxuseclass}
\begin{sphinxuseclass}{cell}\begin{sphinxVerbatimInput}

\begin{sphinxuseclass}{cell_input}
\begin{sphinxVerbatim}[commandchars=\\\{\}]
\PYG{n}{c} \PYG{o}{=} \PYG{n}{np}\PYG{o}{.}\PYG{n}{array}\PYG{p}{(}\PYG{p}{[}\PYG{o}{\PYGZhy{}}\PYG{l+m+mi}{100}\PYG{p}{,} \PYG{l+m+mi}{110}\PYG{p}{]}\PYG{p}{)}
\end{sphinxVerbatim}

\end{sphinxuseclass}\end{sphinxVerbatimInput}

\end{sphinxuseclass}
\begin{sphinxuseclass}{cell}\begin{sphinxVerbatimInput}

\begin{sphinxuseclass}{cell_input}
\begin{sphinxVerbatim}[commandchars=\\\{\}]
\PYG{n}{irr}\PYG{p}{(}\PYG{n}{c}\PYG{p}{)}
\end{sphinxVerbatim}

\end{sphinxuseclass}\end{sphinxVerbatimInput}
\begin{sphinxVerbatimOutput}

\begin{sphinxuseclass}{cell_output}
\begin{sphinxVerbatim}[commandchars=\\\{\}]
0.1000
\end{sphinxVerbatim}

\end{sphinxuseclass}\end{sphinxVerbatimOutput}

\end{sphinxuseclass}

\subsubsection{Write a function \sphinxstyleliteralintitle{\sphinxupquote{returns()}} that accepts \sphinxstyleemphasis{NumPy arrays} of prices and dividends and returns a \sphinxstyleemphasis{NumPy array} of returns.}
\label{\detokenize{mckinney_04_practice_03:write-a-function-returns-that-accepts-numpy-arrays-of-prices-and-dividends-and-returns-a-numpy-array-of-returns}}
\begin{sphinxuseclass}{cell}\begin{sphinxVerbatimInput}

\begin{sphinxuseclass}{cell_input}
\begin{sphinxVerbatim}[commandchars=\\\{\}]
\PYG{n}{prices} \PYG{o}{=} \PYG{n}{np}\PYG{o}{.}\PYG{n}{array}\PYG{p}{(}\PYG{p}{[}\PYG{l+m+mi}{100}\PYG{p}{,} \PYG{l+m+mi}{150}\PYG{p}{,} \PYG{l+m+mi}{100}\PYG{p}{,} \PYG{l+m+mi}{50}\PYG{p}{,} \PYG{l+m+mi}{100}\PYG{p}{,} \PYG{l+m+mi}{150}\PYG{p}{,} \PYG{l+m+mi}{100}\PYG{p}{,} \PYG{l+m+mi}{150}\PYG{p}{]}\PYG{p}{)}
\PYG{n}{dividends} \PYG{o}{=} \PYG{n}{np}\PYG{o}{.}\PYG{n}{array}\PYG{p}{(}\PYG{p}{[}\PYG{l+m+mi}{1}\PYG{p}{,} \PYG{l+m+mi}{1}\PYG{p}{,} \PYG{l+m+mi}{1}\PYG{p}{,} \PYG{l+m+mi}{1}\PYG{p}{,} \PYG{l+m+mi}{2}\PYG{p}{,} \PYG{l+m+mi}{2}\PYG{p}{,} \PYG{l+m+mi}{2}\PYG{p}{,} \PYG{l+m+mi}{2}\PYG{p}{]}\PYG{p}{)}
\end{sphinxVerbatim}

\end{sphinxuseclass}\end{sphinxVerbatimInput}

\end{sphinxuseclass}
\begin{sphinxuseclass}{cell}\begin{sphinxVerbatimInput}

\begin{sphinxuseclass}{cell_input}
\begin{sphinxVerbatim}[commandchars=\\\{\}]
\PYG{k}{def} \PYG{n+nf}{returns}\PYG{p}{(}\PYG{n}{p}\PYG{p}{,} \PYG{n}{d}\PYG{p}{)}\PYG{p}{:}
    \PYG{k}{return} \PYG{p}{(}\PYG{n}{p}\PYG{p}{[}\PYG{l+m+mi}{1}\PYG{p}{:}\PYG{p}{]} \PYG{o}{\PYGZhy{}} \PYG{n}{p}\PYG{p}{[}\PYG{p}{:}\PYG{o}{\PYGZhy{}}\PYG{l+m+mi}{1}\PYG{p}{]} \PYG{o}{+} \PYG{n}{d}\PYG{p}{[}\PYG{l+m+mi}{1}\PYG{p}{:}\PYG{p}{]}\PYG{p}{)} \PYG{o}{/} \PYG{n}{p}\PYG{p}{[}\PYG{p}{:}\PYG{o}{\PYGZhy{}}\PYG{l+m+mi}{1}\PYG{p}{]}
\end{sphinxVerbatim}

\end{sphinxuseclass}\end{sphinxVerbatimInput}

\end{sphinxuseclass}
\begin{sphinxuseclass}{cell}\begin{sphinxVerbatimInput}

\begin{sphinxuseclass}{cell_input}
\begin{sphinxVerbatim}[commandchars=\\\{\}]
\PYG{n}{returns}\PYG{p}{(}\PYG{n}{p}\PYG{o}{=}\PYG{n}{prices}\PYG{p}{,} \PYG{n}{d}\PYG{o}{=}\PYG{n}{dividends}\PYG{p}{)}
\end{sphinxVerbatim}

\end{sphinxuseclass}\end{sphinxVerbatimInput}
\begin{sphinxVerbatimOutput}

\begin{sphinxuseclass}{cell_output}
\begin{sphinxVerbatim}[commandchars=\\\{\}]
array([ 0.51  , \PYGZhy{}0.3267, \PYGZhy{}0.49  ,  1.04  ,  0.52  , \PYGZhy{}0.32  ,  0.52  ])
\end{sphinxVerbatim}

\end{sphinxuseclass}\end{sphinxVerbatimOutput}

\end{sphinxuseclass}

\subsubsection{Rewrite the function \sphinxstyleliteralintitle{\sphinxupquote{returns()}} so it returns \sphinxstyleemphasis{NumPy arrays} of returns, capital gains yields, and dividend yields.}
\label{\detokenize{mckinney_04_practice_03:rewrite-the-function-returns-so-it-returns-numpy-arrays-of-returns-capital-gains-yields-and-dividend-yields}}
\begin{sphinxuseclass}{cell}\begin{sphinxVerbatimInput}

\begin{sphinxuseclass}{cell_input}
\begin{sphinxVerbatim}[commandchars=\\\{\}]
\PYG{k}{def} \PYG{n+nf}{returns}\PYG{p}{(}\PYG{n}{p}\PYG{p}{,} \PYG{n}{d}\PYG{p}{)}\PYG{p}{:}
    \PYG{n}{cg} \PYG{o}{=} \PYG{p}{(}\PYG{n}{p}\PYG{p}{[}\PYG{l+m+mi}{1}\PYG{p}{:}\PYG{p}{]} \PYG{o}{\PYGZhy{}} \PYG{n}{p}\PYG{p}{[}\PYG{p}{:}\PYG{o}{\PYGZhy{}}\PYG{l+m+mi}{1}\PYG{p}{]}\PYG{p}{)} \PYG{o}{/} \PYG{n}{p}\PYG{p}{[}\PYG{p}{:}\PYG{o}{\PYGZhy{}}\PYG{l+m+mi}{1}\PYG{p}{]}
    \PYG{n}{dy} \PYG{o}{=} \PYG{n}{d}\PYG{p}{[}\PYG{l+m+mi}{1}\PYG{p}{:}\PYG{p}{]} \PYG{o}{/} \PYG{n}{p}\PYG{p}{[}\PYG{p}{:}\PYG{o}{\PYGZhy{}}\PYG{l+m+mi}{1}\PYG{p}{]}
    \PYG{n}{r} \PYG{o}{=} \PYG{n}{cg} \PYG{o}{+} \PYG{n}{dy}
    \PYG{k}{return} \PYG{p}{\PYGZob{}}\PYG{l+s+s1}{\PYGZsq{}}\PYG{l+s+s1}{r}\PYG{l+s+s1}{\PYGZsq{}}\PYG{p}{:} \PYG{n}{r}\PYG{p}{,} \PYG{l+s+s1}{\PYGZsq{}}\PYG{l+s+s1}{cg}\PYG{l+s+s1}{\PYGZsq{}}\PYG{p}{:} \PYG{n}{cg}\PYG{p}{,} \PYG{l+s+s1}{\PYGZsq{}}\PYG{l+s+s1}{dy}\PYG{l+s+s1}{\PYGZsq{}}\PYG{p}{:} \PYG{n}{dy}\PYG{p}{\PYGZcb{}}
\end{sphinxVerbatim}

\end{sphinxuseclass}\end{sphinxVerbatimInput}

\end{sphinxuseclass}
\begin{sphinxuseclass}{cell}\begin{sphinxVerbatimInput}

\begin{sphinxuseclass}{cell_input}
\begin{sphinxVerbatim}[commandchars=\\\{\}]
\PYG{n}{returns}\PYG{p}{(}\PYG{n}{p}\PYG{o}{=}\PYG{n}{prices}\PYG{p}{,} \PYG{n}{d}\PYG{o}{=}\PYG{n}{dividends}\PYG{p}{)}
\end{sphinxVerbatim}

\end{sphinxuseclass}\end{sphinxVerbatimInput}
\begin{sphinxVerbatimOutput}

\begin{sphinxuseclass}{cell_output}
\begin{sphinxVerbatim}[commandchars=\\\{\}]
\PYGZob{}\PYGZsq{}r\PYGZsq{}: array([ 0.51  , \PYGZhy{}0.3267, \PYGZhy{}0.49  ,  1.04  ,  0.52  , \PYGZhy{}0.32  ,  0.52  ]),
 \PYGZsq{}cg\PYGZsq{}: array([ 0.5   , \PYGZhy{}0.3333, \PYGZhy{}0.5   ,  1.    ,  0.5   , \PYGZhy{}0.3333,  0.5   ]),
 \PYGZsq{}dy\PYGZsq{}: array([0.01  , 0.0067, 0.01  , 0.04  , 0.02  , 0.0133, 0.02  ])\PYGZcb{}
\end{sphinxVerbatim}

\end{sphinxuseclass}\end{sphinxVerbatimOutput}

\end{sphinxuseclass}

\subsubsection{Rescale and shift numbers so that they cover the range {[}0, 1{]}}
\label{\detokenize{mckinney_04_practice_03:rescale-and-shift-numbers-so-that-they-cover-the-range-0-1}}
\sphinxAtStartPar
Input: \sphinxcode{\sphinxupquote{np.array({[}18.5, 17.0, 18.0, 19.0, 18.0{]})}} \\
Output: \sphinxcode{\sphinxupquote{np.array({[}0.75, 0.0, 0.5, 1.0, 0.5{]})}}

\begin{sphinxuseclass}{cell}\begin{sphinxVerbatimInput}

\begin{sphinxuseclass}{cell_input}
\begin{sphinxVerbatim}[commandchars=\\\{\}]
\PYG{n}{numbers} \PYG{o}{=} \PYG{n}{np}\PYG{o}{.}\PYG{n}{array}\PYG{p}{(}\PYG{p}{[}\PYG{l+m+mf}{18.5}\PYG{p}{,} \PYG{l+m+mf}{17.0}\PYG{p}{,} \PYG{l+m+mf}{18.0}\PYG{p}{,} \PYG{l+m+mf}{19.0}\PYG{p}{,} \PYG{l+m+mf}{18.0}\PYG{p}{]}\PYG{p}{)}
\end{sphinxVerbatim}

\end{sphinxuseclass}\end{sphinxVerbatimInput}

\end{sphinxuseclass}
\begin{sphinxuseclass}{cell}\begin{sphinxVerbatimInput}

\begin{sphinxuseclass}{cell_input}
\begin{sphinxVerbatim}[commandchars=\\\{\}]
\PYG{p}{(}\PYG{n}{numbers} \PYG{o}{\PYGZhy{}} \PYG{n}{numbers}\PYG{o}{.}\PYG{n}{min}\PYG{p}{(}\PYG{p}{)}\PYG{p}{)} \PYG{o}{/} \PYG{p}{(}\PYG{n}{numbers}\PYG{o}{.}\PYG{n}{max}\PYG{p}{(}\PYG{p}{)} \PYG{o}{\PYGZhy{}} \PYG{n}{numbers}\PYG{o}{.}\PYG{n}{min}\PYG{p}{(}\PYG{p}{)}\PYG{p}{)}
\end{sphinxVerbatim}

\end{sphinxuseclass}\end{sphinxVerbatimInput}
\begin{sphinxVerbatimOutput}

\begin{sphinxuseclass}{cell_output}
\begin{sphinxVerbatim}[commandchars=\\\{\}]
array([0.75, 0.  , 0.5 , 1.  , 0.5 ])
\end{sphinxVerbatim}

\end{sphinxuseclass}\end{sphinxVerbatimOutput}

\end{sphinxuseclass}

\subsubsection{Write functions \sphinxstyleliteralintitle{\sphinxupquote{var()}} and \sphinxstyleliteralintitle{\sphinxupquote{std()}} that calculate variance and standard deviation.}
\label{\detokenize{mckinney_04_practice_03:write-functions-var-and-std-that-calculate-variance-and-standard-deviation}}
\sphinxAtStartPar
NumPy’s \sphinxcode{\sphinxupquote{.var()}} and \sphinxcode{\sphinxupquote{.std()}} methods return \sphinxstyleemphasis{population} statistics (i.e., denominators of \$n\$).
The pandas equivalents return \sphinxstyleemphasis{sample} statistics (denominators of \$n\sphinxhyphen{}1\$), which are more appropriate for financial data analysis where we have a sample instead of a population.

\sphinxAtStartPar
Both function should have an argument \sphinxcode{\sphinxupquote{sample}} that is \sphinxcode{\sphinxupquote{True}} by default so both functions return sample statistics by default.

\sphinxAtStartPar
Use the output of \sphinxcode{\sphinxupquote{returns()}} to compare your functions with NumPy’s \sphinxcode{\sphinxupquote{.var()}} and \sphinxcode{\sphinxupquote{.std()}} methods.

\begin{sphinxuseclass}{cell}\begin{sphinxVerbatimInput}

\begin{sphinxuseclass}{cell_input}
\begin{sphinxVerbatim}[commandchars=\\\{\}]
\PYG{n}{np}\PYG{o}{.}\PYG{n}{random}\PYG{o}{.}\PYG{n}{seed}\PYG{p}{(}\PYG{l+m+mi}{42}\PYG{p}{)}
\PYG{n}{r} \PYG{o}{=} \PYG{n}{np}\PYG{o}{.}\PYG{n}{random}\PYG{o}{.}\PYG{n}{randn}\PYG{p}{(}\PYG{l+m+mi}{1\PYGZus{}000\PYGZus{}000}\PYG{p}{)}
\PYG{n}{r}
\end{sphinxVerbatim}

\end{sphinxuseclass}\end{sphinxVerbatimInput}
\begin{sphinxVerbatimOutput}

\begin{sphinxuseclass}{cell_output}
\begin{sphinxVerbatim}[commandchars=\\\{\}]
array([ 0.4967, \PYGZhy{}0.1383,  0.6477, ..., \PYGZhy{}0.113 ,  1.4691,  0.4764])
\end{sphinxVerbatim}

\end{sphinxuseclass}\end{sphinxVerbatimOutput}

\end{sphinxuseclass}
\begin{sphinxuseclass}{cell}\begin{sphinxVerbatimInput}

\begin{sphinxuseclass}{cell_input}
\begin{sphinxVerbatim}[commandchars=\\\{\}]
\PYG{k}{def} \PYG{n+nf}{var}\PYG{p}{(}\PYG{n}{x}\PYG{p}{,} \PYG{n}{ddof}\PYG{o}{=}\PYG{l+m+mi}{0}\PYG{p}{)}\PYG{p}{:}
    \PYG{n}{N} \PYG{o}{=} \PYG{n+nb}{len}\PYG{p}{(}\PYG{n}{x}\PYG{p}{)}
    \PYG{k}{return} \PYG{p}{(}\PYG{p}{(}\PYG{n}{x} \PYG{o}{\PYGZhy{}} \PYG{n}{x}\PYG{o}{.}\PYG{n}{mean}\PYG{p}{(}\PYG{p}{)}\PYG{p}{)}\PYG{o}{*}\PYG{o}{*}\PYG{l+m+mi}{2}\PYG{p}{)}\PYG{o}{.}\PYG{n}{sum}\PYG{p}{(}\PYG{p}{)} \PYG{o}{/} \PYG{p}{(}\PYG{n}{N} \PYG{o}{\PYGZhy{}} \PYG{n}{ddof}\PYG{p}{)}
\end{sphinxVerbatim}

\end{sphinxuseclass}\end{sphinxVerbatimInput}

\end{sphinxuseclass}
\begin{sphinxuseclass}{cell}\begin{sphinxVerbatimInput}

\begin{sphinxuseclass}{cell_input}
\begin{sphinxVerbatim}[commandchars=\\\{\}]
\PYG{n}{r}\PYG{o}{.}\PYG{n}{var}\PYG{p}{(}\PYG{p}{)}
\end{sphinxVerbatim}

\end{sphinxuseclass}\end{sphinxVerbatimInput}
\begin{sphinxVerbatimOutput}

\begin{sphinxuseclass}{cell_output}
\begin{sphinxVerbatim}[commandchars=\\\{\}]
1.0003761089673018
\end{sphinxVerbatim}

\end{sphinxuseclass}\end{sphinxVerbatimOutput}

\end{sphinxuseclass}
\begin{sphinxuseclass}{cell}\begin{sphinxVerbatimInput}

\begin{sphinxuseclass}{cell_input}
\begin{sphinxVerbatim}[commandchars=\\\{\}]
\PYG{n}{var}\PYG{p}{(}\PYG{n}{r}\PYG{p}{)}
\end{sphinxVerbatim}

\end{sphinxuseclass}\end{sphinxVerbatimInput}
\begin{sphinxVerbatimOutput}

\begin{sphinxuseclass}{cell_output}
\begin{sphinxVerbatim}[commandchars=\\\{\}]
1.0003761089673018
\end{sphinxVerbatim}

\end{sphinxuseclass}\end{sphinxVerbatimOutput}

\end{sphinxuseclass}
\begin{sphinxuseclass}{cell}\begin{sphinxVerbatimInput}

\begin{sphinxuseclass}{cell_input}
\begin{sphinxVerbatim}[commandchars=\\\{\}]
\PYG{n}{np}\PYG{o}{.}\PYG{n}{allclose}\PYG{p}{(}\PYG{n}{r}\PYG{o}{.}\PYG{n}{var}\PYG{p}{(}\PYG{p}{)}\PYG{p}{,} \PYG{n}{var}\PYG{p}{(}\PYG{n}{r}\PYG{p}{)}\PYG{p}{)}
\end{sphinxVerbatim}

\end{sphinxuseclass}\end{sphinxVerbatimInput}
\begin{sphinxVerbatimOutput}

\begin{sphinxuseclass}{cell_output}
\begin{sphinxVerbatim}[commandchars=\\\{\}]
True
\end{sphinxVerbatim}

\end{sphinxuseclass}\end{sphinxVerbatimOutput}

\end{sphinxuseclass}
\begin{sphinxuseclass}{cell}\begin{sphinxVerbatimInput}

\begin{sphinxuseclass}{cell_input}
\begin{sphinxVerbatim}[commandchars=\\\{\}]
\PYG{n}{np}\PYG{o}{.}\PYG{n}{allclose}\PYG{p}{(}\PYG{n}{r}\PYG{o}{.}\PYG{n}{var}\PYG{p}{(}\PYG{n}{ddof}\PYG{o}{=}\PYG{l+m+mi}{1}\PYG{p}{)}\PYG{p}{,} \PYG{n}{var}\PYG{p}{(}\PYG{n}{r}\PYG{p}{,} \PYG{n}{ddof}\PYG{o}{=}\PYG{l+m+mi}{1}\PYG{p}{)}\PYG{p}{)}
\end{sphinxVerbatim}

\end{sphinxuseclass}\end{sphinxVerbatimInput}
\begin{sphinxVerbatimOutput}

\begin{sphinxuseclass}{cell_output}
\begin{sphinxVerbatim}[commandchars=\\\{\}]
True
\end{sphinxVerbatim}

\end{sphinxuseclass}\end{sphinxVerbatimOutput}

\end{sphinxuseclass}
\begin{sphinxuseclass}{cell}\begin{sphinxVerbatimInput}

\begin{sphinxuseclass}{cell_input}
\begin{sphinxVerbatim}[commandchars=\\\{\}]
\PYG{k}{def} \PYG{n+nf}{std}\PYG{p}{(}\PYG{n}{x}\PYG{p}{,} \PYG{n}{ddof}\PYG{o}{=}\PYG{l+m+mi}{0}\PYG{p}{)}\PYG{p}{:}
    \PYG{k}{return} \PYG{n}{np}\PYG{o}{.}\PYG{n}{sqrt}\PYG{p}{(}\PYG{n}{var}\PYG{p}{(}\PYG{n}{x}\PYG{o}{=}\PYG{n}{x}\PYG{p}{,} \PYG{n}{ddof}\PYG{o}{=}\PYG{n}{ddof}\PYG{p}{)}\PYG{p}{)}
\end{sphinxVerbatim}

\end{sphinxuseclass}\end{sphinxVerbatimInput}

\end{sphinxuseclass}
\begin{sphinxuseclass}{cell}\begin{sphinxVerbatimInput}

\begin{sphinxuseclass}{cell_input}
\begin{sphinxVerbatim}[commandchars=\\\{\}]
\PYG{n}{r}\PYG{o}{.}\PYG{n}{std}\PYG{p}{(}\PYG{p}{)}
\end{sphinxVerbatim}

\end{sphinxuseclass}\end{sphinxVerbatimInput}
\begin{sphinxVerbatimOutput}

\begin{sphinxuseclass}{cell_output}
\begin{sphinxVerbatim}[commandchars=\\\{\}]
1.000188036804731
\end{sphinxVerbatim}

\end{sphinxuseclass}\end{sphinxVerbatimOutput}

\end{sphinxuseclass}
\begin{sphinxuseclass}{cell}\begin{sphinxVerbatimInput}

\begin{sphinxuseclass}{cell_input}
\begin{sphinxVerbatim}[commandchars=\\\{\}]
\PYG{n}{std}\PYG{p}{(}\PYG{n}{r}\PYG{p}{)}
\end{sphinxVerbatim}

\end{sphinxuseclass}\end{sphinxVerbatimInput}
\begin{sphinxVerbatimOutput}

\begin{sphinxuseclass}{cell_output}
\begin{sphinxVerbatim}[commandchars=\\\{\}]
1.000188036804731
\end{sphinxVerbatim}

\end{sphinxuseclass}\end{sphinxVerbatimOutput}

\end{sphinxuseclass}
\begin{sphinxuseclass}{cell}\begin{sphinxVerbatimInput}

\begin{sphinxuseclass}{cell_input}
\begin{sphinxVerbatim}[commandchars=\\\{\}]
\PYG{n}{np}\PYG{o}{.}\PYG{n}{allclose}\PYG{p}{(}\PYG{n}{r}\PYG{o}{.}\PYG{n}{std}\PYG{p}{(}\PYG{p}{)}\PYG{p}{,} \PYG{n}{std}\PYG{p}{(}\PYG{n}{r}\PYG{p}{)}\PYG{p}{)}
\end{sphinxVerbatim}

\end{sphinxuseclass}\end{sphinxVerbatimInput}
\begin{sphinxVerbatimOutput}

\begin{sphinxuseclass}{cell_output}
\begin{sphinxVerbatim}[commandchars=\\\{\}]
True
\end{sphinxVerbatim}

\end{sphinxuseclass}\end{sphinxVerbatimOutput}

\end{sphinxuseclass}
\begin{sphinxuseclass}{cell}\begin{sphinxVerbatimInput}

\begin{sphinxuseclass}{cell_input}
\begin{sphinxVerbatim}[commandchars=\\\{\}]
\PYG{n}{np}\PYG{o}{.}\PYG{n}{allclose}\PYG{p}{(}\PYG{n}{r}\PYG{o}{.}\PYG{n}{std}\PYG{p}{(}\PYG{n}{ddof}\PYG{o}{=}\PYG{l+m+mi}{1}\PYG{p}{)}\PYG{p}{,} \PYG{n}{std}\PYG{p}{(}\PYG{n}{r}\PYG{p}{,} \PYG{n}{ddof}\PYG{o}{=}\PYG{l+m+mi}{1}\PYG{p}{)}\PYG{p}{)}
\end{sphinxVerbatim}

\end{sphinxuseclass}\end{sphinxVerbatimInput}
\begin{sphinxVerbatimOutput}

\begin{sphinxuseclass}{cell_output}
\begin{sphinxVerbatim}[commandchars=\\\{\}]
True
\end{sphinxVerbatim}

\end{sphinxuseclass}\end{sphinxVerbatimOutput}

\end{sphinxuseclass}
\sphinxstepscope


\section{McKinney Chapter 4 \sphinxhyphen{} Practice (Section 4, Wednesday 11:45 AM)}
\label{\detokenize{mckinney_04_practice_04:mckinney-chapter-4-practice-section-4-wednesday-11-45-am}}\label{\detokenize{mckinney_04_practice_04::doc}}
\begin{sphinxuseclass}{cell}\begin{sphinxVerbatimInput}

\begin{sphinxuseclass}{cell_input}
\begin{sphinxVerbatim}[commandchars=\\\{\}]
\PYG{k+kn}{import} \PYG{n+nn}{numpy} \PYG{k}{as} \PYG{n+nn}{np}
\PYG{o}{\PYGZpc{}}\PYG{k}{precision} 4
\end{sphinxVerbatim}

\end{sphinxuseclass}\end{sphinxVerbatimInput}
\begin{sphinxVerbatimOutput}

\begin{sphinxuseclass}{cell_output}
\begin{sphinxVerbatim}[commandchars=\\\{\}]
\PYGZsq{}\PYGZpc{}.4f\PYGZsq{}
\end{sphinxVerbatim}

\end{sphinxuseclass}\end{sphinxVerbatimOutput}

\end{sphinxuseclass}

\subsection{Practice}
\label{\detokenize{mckinney_04_practice_04:practice}}

\subsubsection{Create a 1\sphinxhyphen{}dimensional array named \sphinxstyleliteralintitle{\sphinxupquote{a1}} that counts from 0 to 24 by 1.}
\label{\detokenize{mckinney_04_practice_04:create-a-1-dimensional-array-named-a1-that-counts-from-0-to-24-by-1}}
\begin{sphinxuseclass}{cell}\begin{sphinxVerbatimInput}

\begin{sphinxuseclass}{cell_input}
\begin{sphinxVerbatim}[commandchars=\\\{\}]
\PYG{n}{a1} \PYG{o}{=} \PYG{n}{np}\PYG{o}{.}\PYG{n}{arange}\PYG{p}{(}\PYG{l+m+mi}{25}\PYG{p}{)}
\PYG{n}{a1}
\end{sphinxVerbatim}

\end{sphinxuseclass}\end{sphinxVerbatimInput}
\begin{sphinxVerbatimOutput}

\begin{sphinxuseclass}{cell_output}
\begin{sphinxVerbatim}[commandchars=\\\{\}]
array([ 0,  1,  2,  3,  4,  5,  6,  7,  8,  9, 10, 11, 12, 13, 14, 15, 16,
       17, 18, 19, 20, 21, 22, 23, 24])
\end{sphinxVerbatim}

\end{sphinxuseclass}\end{sphinxVerbatimOutput}

\end{sphinxuseclass}

\subsubsection{Create a 1\sphinxhyphen{}dimentional array named \sphinxstyleliteralintitle{\sphinxupquote{a2}} that counts from 0 to 24 by 3.}
\label{\detokenize{mckinney_04_practice_04:create-a-1-dimentional-array-named-a2-that-counts-from-0-to-24-by-3}}
\begin{sphinxuseclass}{cell}\begin{sphinxVerbatimInput}

\begin{sphinxuseclass}{cell_input}
\begin{sphinxVerbatim}[commandchars=\\\{\}]
\PYG{n}{a2} \PYG{o}{=} \PYG{n}{np}\PYG{o}{.}\PYG{n}{arange}\PYG{p}{(}\PYG{l+m+mi}{0}\PYG{p}{,} \PYG{l+m+mi}{25}\PYG{p}{,} \PYG{l+m+mi}{3}\PYG{p}{)}
\PYG{n}{a2}
\end{sphinxVerbatim}

\end{sphinxuseclass}\end{sphinxVerbatimInput}
\begin{sphinxVerbatimOutput}

\begin{sphinxuseclass}{cell_output}
\begin{sphinxVerbatim}[commandchars=\\\{\}]
array([ 0,  3,  6,  9, 12, 15, 18, 21, 24])
\end{sphinxVerbatim}

\end{sphinxuseclass}\end{sphinxVerbatimOutput}

\end{sphinxuseclass}
\begin{sphinxuseclass}{cell}\begin{sphinxVerbatimInput}

\begin{sphinxuseclass}{cell_input}
\begin{sphinxVerbatim}[commandchars=\\\{\}]
\PYG{n}{np}\PYG{o}{.}\PYG{n}{arange}\PYG{p}{(}\PYG{l+m+mi}{0}\PYG{p}{,} \PYG{l+m+mi}{26}\PYG{p}{,} \PYG{l+m+mi}{3}\PYG{p}{)}
\end{sphinxVerbatim}

\end{sphinxuseclass}\end{sphinxVerbatimInput}
\begin{sphinxVerbatimOutput}

\begin{sphinxuseclass}{cell_output}
\begin{sphinxVerbatim}[commandchars=\\\{\}]
array([ 0,  3,  6,  9, 12, 15, 18, 21, 24])
\end{sphinxVerbatim}

\end{sphinxuseclass}\end{sphinxVerbatimOutput}

\end{sphinxuseclass}
\begin{sphinxuseclass}{cell}\begin{sphinxVerbatimInput}

\begin{sphinxuseclass}{cell_input}
\begin{sphinxVerbatim}[commandchars=\\\{\}]
\PYG{n}{np}\PYG{o}{.}\PYG{n}{arange}\PYG{p}{(}\PYG{l+m+mi}{0}\PYG{p}{,} \PYG{l+m+mi}{27}\PYG{p}{,} \PYG{l+m+mi}{3}\PYG{p}{)}
\end{sphinxVerbatim}

\end{sphinxuseclass}\end{sphinxVerbatimInput}
\begin{sphinxVerbatimOutput}

\begin{sphinxuseclass}{cell_output}
\begin{sphinxVerbatim}[commandchars=\\\{\}]
array([ 0,  3,  6,  9, 12, 15, 18, 21, 24])
\end{sphinxVerbatim}

\end{sphinxuseclass}\end{sphinxVerbatimOutput}

\end{sphinxuseclass}

\subsubsection{Create a 1\sphinxhyphen{}dimentional array named \sphinxstyleliteralintitle{\sphinxupquote{a3}} that counts from 0 to 100 by multiples of 3 or 5.}
\label{\detokenize{mckinney_04_practice_04:create-a-1-dimentional-array-named-a3-that-counts-from-0-to-100-by-multiples-of-3-or-5}}
\begin{sphinxuseclass}{cell}\begin{sphinxVerbatimInput}

\begin{sphinxuseclass}{cell_input}
\begin{sphinxVerbatim}[commandchars=\\\{\}]
\PYG{n}{a3} \PYG{o}{=} \PYG{n}{np}\PYG{o}{.}\PYG{n}{arange}\PYG{p}{(}\PYG{l+m+mi}{101}\PYG{p}{)}
\PYG{n}{a3} \PYG{o}{=} \PYG{n}{a3}\PYG{p}{[}\PYG{p}{(}\PYG{n}{a3} \PYG{o}{\PYGZpc{}} \PYG{l+m+mi}{3} \PYG{o}{==} \PYG{l+m+mi}{0}\PYG{p}{)} \PYG{o}{|} \PYG{p}{(}\PYG{n}{a3} \PYG{o}{\PYGZpc{}} \PYG{l+m+mi}{5} \PYG{o}{==} \PYG{l+m+mi}{0}\PYG{p}{)}\PYG{p}{]}
\PYG{n}{a3}
\end{sphinxVerbatim}

\end{sphinxuseclass}\end{sphinxVerbatimInput}
\begin{sphinxVerbatimOutput}

\begin{sphinxuseclass}{cell_output}
\begin{sphinxVerbatim}[commandchars=\\\{\}]
array([  0,   3,   5,   6,   9,  10,  12,  15,  18,  20,  21,  24,  25,
        27,  30,  33,  35,  36,  39,  40,  42,  45,  48,  50,  51,  54,
        55,  57,  60,  63,  65,  66,  69,  70,  72,  75,  78,  80,  81,
        84,  85,  87,  90,  93,  95,  96,  99, 100])
\end{sphinxVerbatim}

\end{sphinxuseclass}\end{sphinxVerbatimOutput}

\end{sphinxuseclass}
\sphinxAtStartPar
We can also create \sphinxcode{\sphinxupquote{a3}} with a list comprehension.

\begin{sphinxuseclass}{cell}\begin{sphinxVerbatimInput}

\begin{sphinxuseclass}{cell_input}
\begin{sphinxVerbatim}[commandchars=\\\{\}]
\PYG{n}{a3\PYGZus{}lc} \PYG{o}{=} \PYG{n}{np}\PYG{o}{.}\PYG{n}{array}\PYG{p}{(}\PYG{p}{[}\PYG{n}{i} \PYG{k}{for} \PYG{n}{i} \PYG{o+ow}{in} \PYG{n+nb}{range}\PYG{p}{(}\PYG{l+m+mi}{101}\PYG{p}{)} \PYG{k}{if} \PYG{p}{(}\PYG{n}{i}\PYG{o}{\PYGZpc{}}\PYG{k}{3}==0) or (i\PYGZpc{}5==0)])
\PYG{n}{a3\PYGZus{}lc}
\end{sphinxVerbatim}

\end{sphinxuseclass}\end{sphinxVerbatimInput}
\begin{sphinxVerbatimOutput}

\begin{sphinxuseclass}{cell_output}
\begin{sphinxVerbatim}[commandchars=\\\{\}]
array([  0,   3,   5,   6,   9,  10,  12,  15,  18,  20,  21,  24,  25,
        27,  30,  33,  35,  36,  39,  40,  42,  45,  48,  50,  51,  54,
        55,  57,  60,  63,  65,  66,  69,  70,  72,  75,  78,  80,  81,
        84,  85,  87,  90,  93,  95,  96,  99, 100])
\end{sphinxVerbatim}

\end{sphinxuseclass}\end{sphinxVerbatimOutput}

\end{sphinxuseclass}
\begin{sphinxuseclass}{cell}\begin{sphinxVerbatimInput}

\begin{sphinxuseclass}{cell_input}
\begin{sphinxVerbatim}[commandchars=\\\{\}]
\PYG{n}{a3} \PYG{o}{==} \PYG{n}{a3\PYGZus{}lc}
\end{sphinxVerbatim}

\end{sphinxuseclass}\end{sphinxVerbatimInput}
\begin{sphinxVerbatimOutput}

\begin{sphinxuseclass}{cell_output}
\begin{sphinxVerbatim}[commandchars=\\\{\}]
array([ True,  True,  True,  True,  True,  True,  True,  True,  True,
        True,  True,  True,  True,  True,  True,  True,  True,  True,
        True,  True,  True,  True,  True,  True,  True,  True,  True,
        True,  True,  True,  True,  True,  True,  True,  True,  True,
        True,  True,  True,  True,  True,  True,  True,  True,  True,
        True,  True,  True])
\end{sphinxVerbatim}

\end{sphinxuseclass}\end{sphinxVerbatimOutput}

\end{sphinxuseclass}
\begin{sphinxuseclass}{cell}\begin{sphinxVerbatimInput}

\begin{sphinxuseclass}{cell_input}
\begin{sphinxVerbatim}[commandchars=\\\{\}]
\PYG{p}{(}\PYG{n}{a3} \PYG{o}{==} \PYG{n}{a3\PYGZus{}lc}\PYG{p}{)}\PYG{o}{.}\PYG{n}{all}\PYG{p}{(}\PYG{p}{)}
\end{sphinxVerbatim}

\end{sphinxuseclass}\end{sphinxVerbatimInput}
\begin{sphinxVerbatimOutput}

\begin{sphinxuseclass}{cell_output}
\begin{sphinxVerbatim}[commandchars=\\\{\}]
True
\end{sphinxVerbatim}

\end{sphinxuseclass}\end{sphinxVerbatimOutput}

\end{sphinxuseclass}
\begin{sphinxuseclass}{cell}\begin{sphinxVerbatimInput}

\begin{sphinxuseclass}{cell_input}
\begin{sphinxVerbatim}[commandchars=\\\{\}]
\PYG{n}{a3} \PYG{o+ow}{is} \PYG{n}{a3\PYGZus{}lc}
\end{sphinxVerbatim}

\end{sphinxuseclass}\end{sphinxVerbatimInput}
\begin{sphinxVerbatimOutput}

\begin{sphinxuseclass}{cell_output}
\begin{sphinxVerbatim}[commandchars=\\\{\}]
False
\end{sphinxVerbatim}

\end{sphinxuseclass}\end{sphinxVerbatimOutput}

\end{sphinxuseclass}

\subsubsection{Create a 1\sphinxhyphen{}dimensional array \sphinxstyleliteralintitle{\sphinxupquote{a3}} that contains the squares of the even integers through 100,000.}
\label{\detokenize{mckinney_04_practice_04:create-a-1-dimensional-array-a3-that-contains-the-squares-of-the-even-integers-through-100-000}}
\sphinxAtStartPar
How much faster is the NumPy version than the list comprehension version?

\begin{sphinxuseclass}{cell}\begin{sphinxVerbatimInput}

\begin{sphinxuseclass}{cell_input}
\begin{sphinxVerbatim}[commandchars=\\\{\}]
\PYG{o}{\PYGZpc{}}\PYG{k}{timeit} np.array([i**2 for i in range(100\PYGZus{}001) if i\PYGZpc{}2==0])
\end{sphinxVerbatim}

\end{sphinxuseclass}\end{sphinxVerbatimInput}
\begin{sphinxVerbatimOutput}

\begin{sphinxuseclass}{cell_output}
\begin{sphinxVerbatim}[commandchars=\\\{\}]
12.8 ms ± 21.5 µs per loop (mean ± std. dev. of 7 runs, 100 loops each)
\end{sphinxVerbatim}

\end{sphinxuseclass}\end{sphinxVerbatimOutput}

\end{sphinxuseclass}
\begin{sphinxuseclass}{cell}\begin{sphinxVerbatimInput}

\begin{sphinxuseclass}{cell_input}
\begin{sphinxVerbatim}[commandchars=\\\{\}]
\PYG{o}{\PYGZpc{}}\PYG{k}{timeit} np.array([i**2 for i in range(0, 100\PYGZus{}001, 2)])
\end{sphinxVerbatim}

\end{sphinxuseclass}\end{sphinxVerbatimInput}
\begin{sphinxVerbatimOutput}

\begin{sphinxuseclass}{cell_output}
\begin{sphinxVerbatim}[commandchars=\\\{\}]
10 ms ± 89 µs per loop (mean ± std. dev. of 7 runs, 100 loops each)
\end{sphinxVerbatim}

\end{sphinxuseclass}\end{sphinxVerbatimOutput}

\end{sphinxuseclass}
\begin{sphinxuseclass}{cell}\begin{sphinxVerbatimInput}

\begin{sphinxuseclass}{cell_input}
\begin{sphinxVerbatim}[commandchars=\\\{\}]
\PYG{o}{\PYGZpc{}}\PYG{k}{timeit} np.arange(0, 100\PYGZus{}001, 2, dtype=np.int64)**2
\end{sphinxVerbatim}

\end{sphinxuseclass}\end{sphinxVerbatimInput}
\begin{sphinxVerbatimOutput}

\begin{sphinxuseclass}{cell_output}
\begin{sphinxVerbatim}[commandchars=\\\{\}]
20.4 µs ± 262 ns per loop (mean ± std. dev. of 7 runs, 100000 loops each)
\end{sphinxVerbatim}

\end{sphinxuseclass}\end{sphinxVerbatimOutput}

\end{sphinxuseclass}

\bigskip\hrule\bigskip


\sphinxAtStartPar
\sphinxstyleemphasis{\sphinxstylestrong{Note:}}
On some platforms, \sphinxcode{\sphinxupquote{np.arange(0, 100\_001, 2)}} returns an array of 32\sphinxhyphen{}bit integers.
If we square this array of 32\sphinxhyphen{}bit integers, we get the wrong answer because the large values (e.g., \$100,000\textasciicircum{}2\$) are too large to represent as 32\sphinxhyphen{}bit integers.
Since we know that we need 54\sphinxhyphen{}bit integers for this calculation, we should explcitly set either \sphinxcode{\sphinxupquote{dtype='int64'}} or \sphinxcode{\sphinxupquote{dtype=np.int64}}.

\begin{sphinxuseclass}{cell}\begin{sphinxVerbatimInput}

\begin{sphinxuseclass}{cell_input}
\begin{sphinxVerbatim}[commandchars=\\\{\}]
\PYG{n}{np}\PYG{o}{.}\PYG{n}{arange}\PYG{p}{(}\PYG{l+m+mi}{0}\PYG{p}{,} \PYG{l+m+mi}{100\PYGZus{}001}\PYG{p}{,} \PYG{l+m+mi}{2}\PYG{p}{,} \PYG{n}{dtype}\PYG{o}{=}\PYG{l+s+s1}{\PYGZsq{}}\PYG{l+s+s1}{int64}\PYG{l+s+s1}{\PYGZsq{}}\PYG{p}{)}\PYG{o}{*}\PYG{o}{*}\PYG{l+m+mi}{2}
\end{sphinxVerbatim}

\end{sphinxuseclass}\end{sphinxVerbatimInput}
\begin{sphinxVerbatimOutput}

\begin{sphinxuseclass}{cell_output}
\begin{sphinxVerbatim}[commandchars=\\\{\}]
array([          0,           4,          16, ...,  9999200016,
        9999600004, 10000000000])
\end{sphinxVerbatim}

\end{sphinxuseclass}\end{sphinxVerbatimOutput}

\end{sphinxuseclass}
\begin{sphinxuseclass}{cell}\begin{sphinxVerbatimInput}

\begin{sphinxuseclass}{cell_input}
\begin{sphinxVerbatim}[commandchars=\\\{\}]
\PYG{n}{np}\PYG{o}{.}\PYG{n}{arange}\PYG{p}{(}\PYG{l+m+mi}{0}\PYG{p}{,} \PYG{l+m+mi}{100\PYGZus{}001}\PYG{p}{,} \PYG{l+m+mi}{2}\PYG{p}{,} \PYG{n}{dtype}\PYG{o}{=}\PYG{n}{np}\PYG{o}{.}\PYG{n}{int64}\PYG{p}{)}\PYG{o}{*}\PYG{o}{*}\PYG{l+m+mi}{2}
\end{sphinxVerbatim}

\end{sphinxuseclass}\end{sphinxVerbatimInput}
\begin{sphinxVerbatimOutput}

\begin{sphinxuseclass}{cell_output}
\begin{sphinxVerbatim}[commandchars=\\\{\}]
array([          0,           4,          16, ...,  9999200016,
        9999600004, 10000000000])
\end{sphinxVerbatim}

\end{sphinxuseclass}\end{sphinxVerbatimOutput}

\end{sphinxuseclass}
\sphinxAtStartPar
This \sphinxhref{https://stackoverflow.com/a/1970697/334755}{StackOverflow answer} is this best explanation I have found of this behavior.


\bigskip\hrule\bigskip



\subsubsection{Write a function that mimic Excel’s \sphinxstyleliteralintitle{\sphinxupquote{pv}} function.}
\label{\detokenize{mckinney_04_practice_04:write-a-function-that-mimic-excel-s-pv-function}}
\sphinxAtStartPar
Here is how we call Excel’s \sphinxcode{\sphinxupquote{pv}} function:
\sphinxcode{\sphinxupquote{=PV(rate, nper, pmt, {[}fv{]}, {[}type{]})}}
We can use the annuity and lump sum present value formulas.

\sphinxAtStartPar
Present value of an annuity payment \sphinxcode{\sphinxupquote{pmt}}:
\$PV\_\{pmt\} = \textbackslash{}frac\{pmt\}\{rate\} \textbackslash{}times \textbackslash{}left(1 \sphinxhyphen{} \textbackslash{}frac\{1\}\{(1+rate)\textasciicircum{}\{nper\}\} \textbackslash{}right)\$

\sphinxAtStartPar
Present value of a lump sum \sphinxcode{\sphinxupquote{fv}}:
\$PV\_\{fv\} = \textbackslash{}frac\{fv\}\{(1+rate)\textasciicircum{}\{nper\}\}\$

\begin{sphinxuseclass}{cell}\begin{sphinxVerbatimInput}

\begin{sphinxuseclass}{cell_input}
\begin{sphinxVerbatim}[commandchars=\\\{\}]
\PYG{k}{def} \PYG{n+nf}{pv}\PYG{p}{(}\PYG{n}{rate}\PYG{p}{,} \PYG{n}{nper}\PYG{p}{,} \PYG{n}{pmt}\PYG{o}{=}\PYG{k+kc}{None}\PYG{p}{,} \PYG{n}{fv}\PYG{o}{=}\PYG{k+kc}{None}\PYG{p}{,} \PYG{n+nb}{type}\PYG{o}{=}\PYG{l+s+s1}{\PYGZsq{}}\PYG{l+s+s1}{END}\PYG{l+s+s1}{\PYGZsq{}}\PYG{p}{)}\PYG{p}{:}
    \PYG{k}{if} \PYG{n}{pmt} \PYG{o+ow}{is} \PYG{k+kc}{None}\PYG{p}{:}
        \PYG{n}{pv\PYGZus{}pmt} \PYG{o}{=} \PYG{l+m+mi}{0}
    \PYG{k}{else}\PYG{p}{:}
        \PYG{n}{pv\PYGZus{}pmt} \PYG{o}{=} \PYG{p}{(}\PYG{n}{pmt} \PYG{o}{/} \PYG{n}{rate}\PYG{p}{)} \PYG{o}{*} \PYG{p}{(}\PYG{l+m+mi}{1} \PYG{o}{\PYGZhy{}} \PYG{p}{(}\PYG{l+m+mi}{1} \PYG{o}{+} \PYG{n}{rate}\PYG{p}{)}\PYG{o}{*}\PYG{o}{*}\PYG{p}{(}\PYG{o}{\PYGZhy{}}\PYG{l+m+mi}{1}\PYG{o}{*}\PYG{n}{nper}\PYG{p}{)}\PYG{p}{)}
    
    \PYG{k}{if} \PYG{n}{fv} \PYG{o+ow}{is} \PYG{k+kc}{None}\PYG{p}{:}
        \PYG{n}{pv\PYGZus{}fv} \PYG{o}{=} \PYG{l+m+mi}{0}
    \PYG{k}{else}\PYG{p}{:}
        \PYG{n}{pv\PYGZus{}fv} \PYG{o}{=} \PYG{n}{fv} \PYG{o}{/} \PYG{p}{(}\PYG{l+m+mi}{1} \PYG{o}{+} \PYG{n}{rate}\PYG{p}{)}\PYG{o}{*}\PYG{o}{*}\PYG{n}{nper}

    \PYG{n}{pv} \PYG{o}{=} \PYG{o}{\PYGZhy{}}\PYG{l+m+mi}{1} \PYG{o}{*} \PYG{p}{(}\PYG{n}{pv\PYGZus{}pmt} \PYG{o}{+} \PYG{n}{pv\PYGZus{}fv}\PYG{p}{)}
    
    \PYG{k}{if} \PYG{n+nb}{type} \PYG{o}{==} \PYG{l+s+s1}{\PYGZsq{}}\PYG{l+s+s1}{END}\PYG{l+s+s1}{\PYGZsq{}}\PYG{p}{:}
        \PYG{n}{pv} \PYG{o}{=} \PYG{n}{pv}
    \PYG{k}{elif} \PYG{n+nb}{type} \PYG{o}{==} \PYG{l+s+s1}{\PYGZsq{}}\PYG{l+s+s1}{BGN}\PYG{l+s+s1}{\PYGZsq{}}\PYG{p}{:}
        \PYG{n}{pv} \PYG{o}{*}\PYG{o}{=} \PYG{p}{(}\PYG{l+m+mi}{1} \PYG{o}{+} \PYG{n}{rate}\PYG{p}{)}
    \PYG{k}{else}\PYG{p}{:}
        \PYG{n+nb}{print}\PYG{p}{(}\PYG{l+s+s1}{\PYGZsq{}}\PYG{l+s+s1}{Enter END or BGN for type argument}\PYG{l+s+s1}{\PYGZsq{}}\PYG{p}{)}
        
    \PYG{k}{return} \PYG{n}{pv}
\end{sphinxVerbatim}

\end{sphinxuseclass}\end{sphinxVerbatimInput}

\end{sphinxuseclass}
\begin{sphinxuseclass}{cell}\begin{sphinxVerbatimInput}

\begin{sphinxuseclass}{cell_input}
\begin{sphinxVerbatim}[commandchars=\\\{\}]
\PYG{n}{pv}\PYG{p}{(}\PYG{n}{rate}\PYG{o}{=}\PYG{l+m+mf}{0.1}\PYG{p}{,} \PYG{n}{nper}\PYG{o}{=}\PYG{l+m+mi}{10}\PYG{p}{,} \PYG{n}{pmt}\PYG{o}{=}\PYG{l+m+mi}{100}\PYG{p}{,} \PYG{n}{fv}\PYG{o}{=}\PYG{l+m+mi}{1000}\PYG{p}{,} \PYG{n+nb}{type}\PYG{o}{=}\PYG{l+s+s1}{\PYGZsq{}}\PYG{l+s+s1}{END}\PYG{l+s+s1}{\PYGZsq{}}\PYG{p}{)}
\end{sphinxVerbatim}

\end{sphinxuseclass}\end{sphinxVerbatimInput}
\begin{sphinxVerbatimOutput}

\begin{sphinxuseclass}{cell_output}
\begin{sphinxVerbatim}[commandchars=\\\{\}]
\PYGZhy{}1000.0000
\end{sphinxVerbatim}

\end{sphinxuseclass}\end{sphinxVerbatimOutput}

\end{sphinxuseclass}

\subsubsection{Write a function that mimic Excel’s \sphinxstyleliteralintitle{\sphinxupquote{fv}} function.}
\label{\detokenize{mckinney_04_practice_04:write-a-function-that-mimic-excel-s-fv-function}}
\begin{sphinxuseclass}{cell}\begin{sphinxVerbatimInput}

\begin{sphinxuseclass}{cell_input}
\begin{sphinxVerbatim}[commandchars=\\\{\}]
\PYG{k}{def} \PYG{n+nf}{fv}\PYG{p}{(}\PYG{n}{rate}\PYG{p}{,} \PYG{n}{nper}\PYG{p}{,} \PYG{n}{pmt}\PYG{o}{=}\PYG{k+kc}{None}\PYG{p}{,} \PYG{n}{pv}\PYG{o}{=}\PYG{k+kc}{None}\PYG{p}{,} \PYG{n+nb}{type} \PYG{o}{=} \PYG{l+s+s1}{\PYGZsq{}}\PYG{l+s+s1}{END}\PYG{l+s+s1}{\PYGZsq{}}\PYG{p}{)}\PYG{p}{:}
    \PYG{k}{if} \PYG{n}{pmt} \PYG{o+ow}{is} \PYG{o+ow}{not} \PYG{k+kc}{None}\PYG{p}{:} \PYG{c+c1}{\PYGZsh{} calculate PV of pmt, if given}
        \PYG{n}{fv\PYGZus{}pmt} \PYG{o}{=} \PYG{p}{(}\PYG{n}{pmt} \PYG{o}{/} \PYG{n}{rate}\PYG{p}{)} \PYG{o}{*} \PYG{p}{(}\PYG{p}{(}\PYG{l+m+mi}{1} \PYG{o}{+} \PYG{n}{rate}\PYG{p}{)}\PYG{o}{*}\PYG{o}{*}\PYG{n}{nper} \PYG{o}{\PYGZhy{}} \PYG{l+m+mi}{1}\PYG{p}{)}
    \PYG{k}{else}\PYG{p}{:}
        \PYG{n}{fv\PYGZus{}pmt} \PYG{o}{=} \PYG{l+m+mi}{0}

    \PYG{k}{if} \PYG{n}{fv} \PYG{o+ow}{is} \PYG{o+ow}{not} \PYG{k+kc}{None}\PYG{p}{:} \PYG{c+c1}{\PYGZsh{} calculate PV of fv, if given}
        \PYG{n}{fv\PYGZus{}pv} \PYG{o}{=} \PYG{n}{pv} \PYG{o}{*} \PYG{p}{(}\PYG{l+m+mi}{1} \PYG{o}{+} \PYG{n}{rate}\PYG{p}{)}\PYG{o}{*}\PYG{o}{*}\PYG{n}{nper}
    \PYG{k}{else}\PYG{p}{:}
        \PYG{n}{fv\PYGZus{}fv} \PYG{o}{=} \PYG{l+m+mi}{0}
    
    \PYG{k}{if} \PYG{n+nb}{type}\PYG{o}{==}\PYG{l+s+s1}{\PYGZsq{}}\PYG{l+s+s1}{END}\PYG{l+s+s1}{\PYGZsq{}}\PYG{p}{:} \PYG{c+c1}{\PYGZsh{} as\PYGZhy{}is if end of period payments}
        \PYG{k}{return} \PYG{o}{\PYGZhy{}}\PYG{l+m+mi}{1} \PYG{o}{*} \PYG{p}{(}\PYG{n}{fv\PYGZus{}pmt} \PYG{o}{+} \PYG{n}{fv\PYGZus{}pv}\PYG{p}{)}
    \PYG{k}{elif} \PYG{n+nb}{type}\PYG{o}{==}\PYG{l+s+s1}{\PYGZsq{}}\PYG{l+s+s1}{BGN}\PYG{l+s+s1}{\PYGZsq{}}\PYG{p}{:} \PYG{c+c1}{\PYGZsh{} undo one period of discounting if bgn of period payments}
        \PYG{k}{return} \PYG{o}{\PYGZhy{}}\PYG{l+m+mi}{1} \PYG{o}{*} \PYG{p}{(}\PYG{n}{fv\PYGZus{}pmt} \PYG{o}{+} \PYG{n}{fv\PYGZus{}pv}\PYG{p}{)} \PYG{o}{*} \PYG{p}{(}\PYG{l+m+mi}{1} \PYG{o}{+} \PYG{n}{rate}\PYG{p}{)}
    \PYG{k}{else}\PYG{p}{:} \PYG{c+c1}{\PYGZsh{} otherwise, ask use to specify end or bgn of period payments}
        \PYG{n+nb}{print}\PYG{p}{(}\PYG{l+s+s1}{\PYGZsq{}}\PYG{l+s+s1}{Please enter END or BGN for the type argument}\PYG{l+s+s1}{\PYGZsq{}}\PYG{p}{)}
\end{sphinxVerbatim}

\end{sphinxuseclass}\end{sphinxVerbatimInput}

\end{sphinxuseclass}
\begin{sphinxuseclass}{cell}\begin{sphinxVerbatimInput}

\begin{sphinxuseclass}{cell_input}
\begin{sphinxVerbatim}[commandchars=\\\{\}]
\PYG{n}{fv}\PYG{p}{(}\PYG{n}{rate}\PYG{o}{=}\PYG{l+m+mf}{0.1}\PYG{p}{,} \PYG{n}{nper}\PYG{o}{=}\PYG{l+m+mi}{10}\PYG{p}{,} \PYG{n}{pmt}\PYG{o}{=}\PYG{l+m+mi}{100}\PYG{p}{,} \PYG{n}{pv}\PYG{o}{=}\PYG{o}{\PYGZhy{}}\PYG{l+m+mi}{1000}\PYG{p}{,} \PYG{n+nb}{type}\PYG{o}{=}\PYG{l+s+s1}{\PYGZsq{}}\PYG{l+s+s1}{END}\PYG{l+s+s1}{\PYGZsq{}}\PYG{p}{)}
\end{sphinxVerbatim}

\end{sphinxuseclass}\end{sphinxVerbatimInput}
\begin{sphinxVerbatimOutput}

\begin{sphinxuseclass}{cell_output}
\begin{sphinxVerbatim}[commandchars=\\\{\}]
1000.0000
\end{sphinxVerbatim}

\end{sphinxuseclass}\end{sphinxVerbatimOutput}

\end{sphinxuseclass}

\subsubsection{Replace the negative values in \sphinxstyleliteralintitle{\sphinxupquote{data}} with \sphinxhyphen{}1 and positive values with +1.}
\label{\detokenize{mckinney_04_practice_04:replace-the-negative-values-in-data-with-1-and-positive-values-with-1}}
\begin{sphinxuseclass}{cell}\begin{sphinxVerbatimInput}

\begin{sphinxuseclass}{cell_input}
\begin{sphinxVerbatim}[commandchars=\\\{\}]
\PYG{n}{np}\PYG{o}{.}\PYG{n}{random}\PYG{o}{.}\PYG{n}{seed}\PYG{p}{(}\PYG{l+m+mi}{42}\PYG{p}{)}
\PYG{n}{data} \PYG{o}{=} \PYG{n}{np}\PYG{o}{.}\PYG{n}{random}\PYG{o}{.}\PYG{n}{randn}\PYG{p}{(}\PYG{l+m+mi}{7}\PYG{p}{,} \PYG{l+m+mi}{7}\PYG{p}{)}
\PYG{n}{data}
\end{sphinxVerbatim}

\end{sphinxuseclass}\end{sphinxVerbatimInput}
\begin{sphinxVerbatimOutput}

\begin{sphinxuseclass}{cell_output}
\begin{sphinxVerbatim}[commandchars=\\\{\}]
array([[ 0.4967, \PYGZhy{}0.1383,  0.6477,  1.523 , \PYGZhy{}0.2342, \PYGZhy{}0.2341,  1.5792],
       [ 0.7674, \PYGZhy{}0.4695,  0.5426, \PYGZhy{}0.4634, \PYGZhy{}0.4657,  0.242 , \PYGZhy{}1.9133],
       [\PYGZhy{}1.7249, \PYGZhy{}0.5623, \PYGZhy{}1.0128,  0.3142, \PYGZhy{}0.908 , \PYGZhy{}1.4123,  1.4656],
       [\PYGZhy{}0.2258,  0.0675, \PYGZhy{}1.4247, \PYGZhy{}0.5444,  0.1109, \PYGZhy{}1.151 ,  0.3757],
       [\PYGZhy{}0.6006, \PYGZhy{}0.2917, \PYGZhy{}0.6017,  1.8523, \PYGZhy{}0.0135, \PYGZhy{}1.0577,  0.8225],
       [\PYGZhy{}1.2208,  0.2089, \PYGZhy{}1.9597, \PYGZhy{}1.3282,  0.1969,  0.7385,  0.1714],
       [\PYGZhy{}0.1156, \PYGZhy{}0.3011, \PYGZhy{}1.4785, \PYGZhy{}0.7198, \PYGZhy{}0.4606,  1.0571,  0.3436]])
\end{sphinxVerbatim}

\end{sphinxuseclass}\end{sphinxVerbatimOutput}

\end{sphinxuseclass}
\begin{sphinxuseclass}{cell}\begin{sphinxVerbatimInput}

\begin{sphinxuseclass}{cell_input}
\begin{sphinxVerbatim}[commandchars=\\\{\}]
\PYG{n}{data\PYGZus{}c} \PYG{o}{=} \PYG{n}{data}\PYG{o}{.}\PYG{n}{copy}\PYG{p}{(}\PYG{p}{)}
\PYG{n}{data\PYGZus{}c}\PYG{p}{[}\PYG{n}{data\PYGZus{}c} \PYG{o}{\PYGZlt{}} \PYG{l+m+mi}{0}\PYG{p}{]} \PYG{o}{=} \PYG{o}{\PYGZhy{}}\PYG{l+m+mi}{1}
\PYG{n}{data\PYGZus{}c}\PYG{p}{[}\PYG{n}{data\PYGZus{}c} \PYG{o}{\PYGZgt{}} \PYG{l+m+mi}{0}\PYG{p}{]} \PYG{o}{=} \PYG{o}{+}\PYG{l+m+mi}{1}

\PYG{n}{data\PYGZus{}c}
\end{sphinxVerbatim}

\end{sphinxuseclass}\end{sphinxVerbatimInput}
\begin{sphinxVerbatimOutput}

\begin{sphinxuseclass}{cell_output}
\begin{sphinxVerbatim}[commandchars=\\\{\}]
array([[ 1., \PYGZhy{}1.,  1.,  1., \PYGZhy{}1., \PYGZhy{}1.,  1.],
       [ 1., \PYGZhy{}1.,  1., \PYGZhy{}1., \PYGZhy{}1.,  1., \PYGZhy{}1.],
       [\PYGZhy{}1., \PYGZhy{}1., \PYGZhy{}1.,  1., \PYGZhy{}1., \PYGZhy{}1.,  1.],
       [\PYGZhy{}1.,  1., \PYGZhy{}1., \PYGZhy{}1.,  1., \PYGZhy{}1.,  1.],
       [\PYGZhy{}1., \PYGZhy{}1., \PYGZhy{}1.,  1., \PYGZhy{}1., \PYGZhy{}1.,  1.],
       [\PYGZhy{}1.,  1., \PYGZhy{}1., \PYGZhy{}1.,  1.,  1.,  1.],
       [\PYGZhy{}1., \PYGZhy{}1., \PYGZhy{}1., \PYGZhy{}1., \PYGZhy{}1.,  1.,  1.]])
\end{sphinxVerbatim}

\end{sphinxuseclass}\end{sphinxVerbatimOutput}

\end{sphinxuseclass}
\begin{sphinxuseclass}{cell}\begin{sphinxVerbatimInput}

\begin{sphinxuseclass}{cell_input}
\begin{sphinxVerbatim}[commandchars=\\\{\}]
\PYG{n}{data}
\end{sphinxVerbatim}

\end{sphinxuseclass}\end{sphinxVerbatimInput}
\begin{sphinxVerbatimOutput}

\begin{sphinxuseclass}{cell_output}
\begin{sphinxVerbatim}[commandchars=\\\{\}]
array([[ 0.4967, \PYGZhy{}0.1383,  0.6477,  1.523 , \PYGZhy{}0.2342, \PYGZhy{}0.2341,  1.5792],
       [ 0.7674, \PYGZhy{}0.4695,  0.5426, \PYGZhy{}0.4634, \PYGZhy{}0.4657,  0.242 , \PYGZhy{}1.9133],
       [\PYGZhy{}1.7249, \PYGZhy{}0.5623, \PYGZhy{}1.0128,  0.3142, \PYGZhy{}0.908 , \PYGZhy{}1.4123,  1.4656],
       [\PYGZhy{}0.2258,  0.0675, \PYGZhy{}1.4247, \PYGZhy{}0.5444,  0.1109, \PYGZhy{}1.151 ,  0.3757],
       [\PYGZhy{}0.6006, \PYGZhy{}0.2917, \PYGZhy{}0.6017,  1.8523, \PYGZhy{}0.0135, \PYGZhy{}1.0577,  0.8225],
       [\PYGZhy{}1.2208,  0.2089, \PYGZhy{}1.9597, \PYGZhy{}1.3282,  0.1969,  0.7385,  0.1714],
       [\PYGZhy{}0.1156, \PYGZhy{}0.3011, \PYGZhy{}1.4785, \PYGZhy{}0.7198, \PYGZhy{}0.4606,  1.0571,  0.3436]])
\end{sphinxVerbatim}

\end{sphinxuseclass}\end{sphinxVerbatimOutput}

\end{sphinxuseclass}
\begin{sphinxuseclass}{cell}\begin{sphinxVerbatimInput}

\begin{sphinxuseclass}{cell_input}
\begin{sphinxVerbatim}[commandchars=\\\{\}]
\PYG{n}{np}\PYG{o}{.}\PYG{n}{where}\PYG{p}{(}\PYG{n}{data} \PYG{o}{\PYGZlt{}} \PYG{l+m+mi}{0}\PYG{p}{,} \PYG{o}{\PYGZhy{}}\PYG{l+m+mi}{1}\PYG{p}{,} \PYG{n}{np}\PYG{o}{.}\PYG{n}{where}\PYG{p}{(}\PYG{n}{data} \PYG{o}{\PYGZgt{}} \PYG{l+m+mi}{0}\PYG{p}{,} \PYG{o}{+}\PYG{l+m+mi}{1}\PYG{p}{,} \PYG{n}{data}\PYG{p}{)}\PYG{p}{)}
\end{sphinxVerbatim}

\end{sphinxuseclass}\end{sphinxVerbatimInput}
\begin{sphinxVerbatimOutput}

\begin{sphinxuseclass}{cell_output}
\begin{sphinxVerbatim}[commandchars=\\\{\}]
array([[ 1., \PYGZhy{}1.,  1.,  1., \PYGZhy{}1., \PYGZhy{}1.,  1.],
       [ 1., \PYGZhy{}1.,  1., \PYGZhy{}1., \PYGZhy{}1.,  1., \PYGZhy{}1.],
       [\PYGZhy{}1., \PYGZhy{}1., \PYGZhy{}1.,  1., \PYGZhy{}1., \PYGZhy{}1.,  1.],
       [\PYGZhy{}1.,  1., \PYGZhy{}1., \PYGZhy{}1.,  1., \PYGZhy{}1.,  1.],
       [\PYGZhy{}1., \PYGZhy{}1., \PYGZhy{}1.,  1., \PYGZhy{}1., \PYGZhy{}1.,  1.],
       [\PYGZhy{}1.,  1., \PYGZhy{}1., \PYGZhy{}1.,  1.,  1.,  1.],
       [\PYGZhy{}1., \PYGZhy{}1., \PYGZhy{}1., \PYGZhy{}1., \PYGZhy{}1.,  1.,  1.]])
\end{sphinxVerbatim}

\end{sphinxuseclass}\end{sphinxVerbatimOutput}

\end{sphinxuseclass}
\begin{sphinxuseclass}{cell}\begin{sphinxVerbatimInput}

\begin{sphinxuseclass}{cell_input}
\begin{sphinxVerbatim}[commandchars=\\\{\}]
\PYG{n}{np}\PYG{o}{.}\PYG{n}{select}\PYG{p}{(}
    \PYG{n}{condlist}\PYG{o}{=}\PYG{p}{[}\PYG{n}{data}\PYG{o}{\PYGZlt{}}\PYG{l+m+mi}{0}\PYG{p}{,} \PYG{n}{data}\PYG{o}{\PYGZgt{}}\PYG{l+m+mi}{0}\PYG{p}{]}\PYG{p}{,}
    \PYG{n}{choicelist}\PYG{o}{=}\PYG{p}{[}\PYG{o}{\PYGZhy{}}\PYG{l+m+mi}{1}\PYG{p}{,} \PYG{o}{+}\PYG{l+m+mi}{1}\PYG{p}{]}\PYG{p}{,}
    \PYG{n}{default}\PYG{o}{=}\PYG{l+m+mi}{0}
\PYG{p}{)}
\end{sphinxVerbatim}

\end{sphinxuseclass}\end{sphinxVerbatimInput}
\begin{sphinxVerbatimOutput}

\begin{sphinxuseclass}{cell_output}
\begin{sphinxVerbatim}[commandchars=\\\{\}]
array([[ 1, \PYGZhy{}1,  1,  1, \PYGZhy{}1, \PYGZhy{}1,  1],
       [ 1, \PYGZhy{}1,  1, \PYGZhy{}1, \PYGZhy{}1,  1, \PYGZhy{}1],
       [\PYGZhy{}1, \PYGZhy{}1, \PYGZhy{}1,  1, \PYGZhy{}1, \PYGZhy{}1,  1],
       [\PYGZhy{}1,  1, \PYGZhy{}1, \PYGZhy{}1,  1, \PYGZhy{}1,  1],
       [\PYGZhy{}1, \PYGZhy{}1, \PYGZhy{}1,  1, \PYGZhy{}1, \PYGZhy{}1,  1],
       [\PYGZhy{}1,  1, \PYGZhy{}1, \PYGZhy{}1,  1,  1,  1],
       [\PYGZhy{}1, \PYGZhy{}1, \PYGZhy{}1, \PYGZhy{}1, \PYGZhy{}1,  1,  1]])
\end{sphinxVerbatim}

\end{sphinxuseclass}\end{sphinxVerbatimOutput}

\end{sphinxuseclass}

\subsubsection{Write a function \sphinxstyleliteralintitle{\sphinxupquote{npmts()}} that calculates the number of payments that generate \$x\%\$ of the present value of a perpetuity.}
\label{\detokenize{mckinney_04_practice_04:write-a-function-npmts-that-calculates-the-number-of-payments-that-generate-x-of-the-present-value-of-a-perpetuity}}
\sphinxAtStartPar
Your \sphinxcode{\sphinxupquote{npmts()}} should accept arguments \sphinxcode{\sphinxupquote{c1}}, \sphinxcode{\sphinxupquote{r}}, and \sphinxcode{\sphinxupquote{g}} that represent  \$C\_1\$, \$r\$, and \$g\$.
The present value of a growing perpetuity is \$PV = \textbackslash{}frac\{C\_1\}\{r \sphinxhyphen{} g\}\$, and the present value of a growing annuity is \$PV = \textbackslash{}frac\{C\_1\}\{r \sphinxhyphen{} g\}\textbackslash{}left{[} 1 \sphinxhyphen{} \textbackslash{}left( \textbackslash{}frac\{1 + g\}\{1 + r\} \textbackslash{}right)\textasciicircum{}t \textbackslash{}right{]}\$.

\sphinxAtStartPar
We can use the growing annuity and perpetuity formulas to show: \$x = \textbackslash{}left{[} 1 \sphinxhyphen{} \textbackslash{}left( \textbackslash{}frac\{1 + g\}\{1 + r\} \textbackslash{}right)\textasciicircum{}t \textbackslash{}right{]}\$.

\sphinxAtStartPar
Then: \$1 \sphinxhyphen{} x = \textbackslash{}left( \textbackslash{}frac\{1 + g\}\{1 + r\} \textbackslash{}right)\textasciicircum{}t\$.

\sphinxAtStartPar
Finally: \$t = \textbackslash{}frac\{\textbackslash{}log(1\sphinxhyphen{}x)\}\{\textbackslash{}log\textbackslash{}left(\textbackslash{}frac\{1 + g\}\{1 + r\}\textbackslash{}right)\}\$

\sphinxAtStartPar
\sphinxstyleemphasis{\sphinxstylestrong{We do not need to accept an argument \sphinxcode{\sphinxupquote{c1}} because \$C\_1\$ cancels out!}}

\begin{sphinxuseclass}{cell}\begin{sphinxVerbatimInput}

\begin{sphinxuseclass}{cell_input}
\begin{sphinxVerbatim}[commandchars=\\\{\}]
\PYG{k}{def} \PYG{n+nf}{npmts}\PYG{p}{(}\PYG{n}{x}\PYG{p}{,} \PYG{n}{r}\PYG{p}{,} \PYG{n}{g}\PYG{p}{)}\PYG{p}{:}
    \PYG{k}{return} \PYG{n}{np}\PYG{o}{.}\PYG{n}{log}\PYG{p}{(}\PYG{l+m+mi}{1}\PYG{o}{\PYGZhy{}}\PYG{n}{x}\PYG{p}{)} \PYG{o}{/} \PYG{n}{np}\PYG{o}{.}\PYG{n}{log}\PYG{p}{(}\PYG{p}{(}\PYG{l+m+mi}{1} \PYG{o}{+} \PYG{n}{g}\PYG{p}{)} \PYG{o}{/} \PYG{p}{(}\PYG{l+m+mi}{1} \PYG{o}{+} \PYG{n}{r}\PYG{p}{)}\PYG{p}{)}
\end{sphinxVerbatim}

\end{sphinxuseclass}\end{sphinxVerbatimInput}

\end{sphinxuseclass}
\begin{sphinxuseclass}{cell}\begin{sphinxVerbatimInput}

\begin{sphinxuseclass}{cell_input}
\begin{sphinxVerbatim}[commandchars=\\\{\}]
\PYG{n}{npmts}\PYG{p}{(}\PYG{l+m+mf}{0.5}\PYG{p}{,} \PYG{l+m+mf}{0.1}\PYG{p}{,} \PYG{l+m+mf}{0.05}\PYG{p}{)}
\end{sphinxVerbatim}

\end{sphinxuseclass}\end{sphinxVerbatimInput}
\begin{sphinxVerbatimOutput}

\begin{sphinxuseclass}{cell_output}
\begin{sphinxVerbatim}[commandchars=\\\{\}]
14.899977377480532
\end{sphinxVerbatim}

\end{sphinxuseclass}\end{sphinxVerbatimOutput}

\end{sphinxuseclass}

\subsubsection{Write a function that calculates the internal rate of return given a NumPy array of cash flows.}
\label{\detokenize{mckinney_04_practice_04:write-a-function-that-calculates-the-internal-rate-of-return-given-a-numpy-array-of-cash-flows}}
\sphinxAtStartPar
Here are some data where the \$IRR\$ is obvious!

\begin{sphinxuseclass}{cell}\begin{sphinxVerbatimInput}

\begin{sphinxuseclass}{cell_input}
\begin{sphinxVerbatim}[commandchars=\\\{\}]
\PYG{n}{c} \PYG{o}{=} \PYG{n}{np}\PYG{o}{.}\PYG{n}{array}\PYG{p}{(}\PYG{p}{[}\PYG{o}{\PYGZhy{}}\PYG{l+m+mi}{100}\PYG{p}{,} \PYG{l+m+mi}{110}\PYG{p}{]}\PYG{p}{)}
\PYG{n}{irr} \PYG{o}{=} \PYG{l+m+mf}{0.1}
\end{sphinxVerbatim}

\end{sphinxuseclass}\end{sphinxVerbatimInput}

\end{sphinxuseclass}
\sphinxAtStartPar
We want to replicate the following calculation with NumPy:

\begin{sphinxuseclass}{cell}\begin{sphinxVerbatimInput}

\begin{sphinxuseclass}{cell_input}
\begin{sphinxVerbatim}[commandchars=\\\{\}]
\PYG{n}{c}\PYG{p}{[}\PYG{l+m+mi}{0}\PYG{p}{]}\PYG{o}{/}\PYG{p}{(}\PYG{l+m+mi}{1}\PYG{o}{+}\PYG{n}{irr}\PYG{p}{)}\PYG{o}{*}\PYG{o}{*}\PYG{l+m+mi}{0} \PYG{o}{+} \PYG{n}{c}\PYG{p}{[}\PYG{l+m+mi}{1}\PYG{p}{]}\PYG{o}{/}\PYG{p}{(}\PYG{l+m+mi}{1}\PYG{o}{+}\PYG{n}{irr}\PYG{p}{)}\PYG{o}{*}\PYG{o}{*}\PYG{l+m+mi}{1}
\end{sphinxVerbatim}

\end{sphinxuseclass}\end{sphinxVerbatimInput}
\begin{sphinxVerbatimOutput}

\begin{sphinxuseclass}{cell_output}
\begin{sphinxVerbatim}[commandchars=\\\{\}]
\PYGZhy{}1.4210854715202004e\PYGZhy{}14
\end{sphinxVerbatim}

\end{sphinxuseclass}\end{sphinxVerbatimOutput}

\end{sphinxuseclass}
\sphinxAtStartPar
The following NumPy code calculates the present value interest factor for each cash flow:

\begin{sphinxuseclass}{cell}\begin{sphinxVerbatimInput}

\begin{sphinxuseclass}{cell_input}
\begin{sphinxVerbatim}[commandchars=\\\{\}]
\PYG{l+m+mi}{1} \PYG{o}{/} \PYG{p}{(}\PYG{l+m+mi}{1} \PYG{o}{+} \PYG{n}{irr}\PYG{p}{)}\PYG{o}{*}\PYG{o}{*}\PYG{n}{np}\PYG{o}{.}\PYG{n}{arange}\PYG{p}{(}\PYG{n+nb}{len}\PYG{p}{(}\PYG{n}{c}\PYG{p}{)}\PYG{p}{)} \PYG{c+c1}{\PYGZsh{} present value interest factor}
\end{sphinxVerbatim}

\end{sphinxuseclass}\end{sphinxVerbatimInput}
\begin{sphinxVerbatimOutput}

\begin{sphinxuseclass}{cell_output}
\begin{sphinxVerbatim}[commandchars=\\\{\}]
array([1.    , 0.9091])
\end{sphinxVerbatim}

\end{sphinxuseclass}\end{sphinxVerbatimOutput}

\end{sphinxuseclass}
\sphinxAtStartPar
The following NumPy code calculates the present value for each cash flow:

\begin{sphinxuseclass}{cell}\begin{sphinxVerbatimInput}

\begin{sphinxuseclass}{cell_input}
\begin{sphinxVerbatim}[commandchars=\\\{\}]
\PYG{n}{c} \PYG{o}{/} \PYG{p}{(}\PYG{l+m+mi}{1} \PYG{o}{+} \PYG{n}{irr}\PYG{p}{)}\PYG{o}{*}\PYG{o}{*}\PYG{n}{np}\PYG{o}{.}\PYG{n}{arange}\PYG{p}{(}\PYG{n+nb}{len}\PYG{p}{(}\PYG{n}{c}\PYG{p}{)}\PYG{p}{)} \PYG{c+c1}{\PYGZsh{} present value of each cash flow}
\end{sphinxVerbatim}

\end{sphinxuseclass}\end{sphinxVerbatimInput}
\begin{sphinxVerbatimOutput}

\begin{sphinxuseclass}{cell_output}
\begin{sphinxVerbatim}[commandchars=\\\{\}]
array([\PYGZhy{}100.,  100.])
\end{sphinxVerbatim}

\end{sphinxuseclass}\end{sphinxVerbatimOutput}

\end{sphinxuseclass}
\sphinxAtStartPar
We sum these present values of each cash flow to calculate the net present value:

\begin{sphinxuseclass}{cell}\begin{sphinxVerbatimInput}

\begin{sphinxuseclass}{cell_input}
\begin{sphinxVerbatim}[commandchars=\\\{\}]
\PYG{p}{(}\PYG{n}{c} \PYG{o}{/} \PYG{p}{(}\PYG{l+m+mi}{1} \PYG{o}{+} \PYG{n}{irr}\PYG{p}{)}\PYG{o}{*}\PYG{o}{*}\PYG{n}{np}\PYG{o}{.}\PYG{n}{arange}\PYG{p}{(}\PYG{n+nb}{len}\PYG{p}{(}\PYG{n}{c}\PYG{p}{)}\PYG{p}{)}\PYG{p}{)}\PYG{o}{.}\PYG{n}{sum}\PYG{p}{(}\PYG{p}{)}
\end{sphinxVerbatim}

\end{sphinxuseclass}\end{sphinxVerbatimInput}
\begin{sphinxVerbatimOutput}

\begin{sphinxuseclass}{cell_output}
\begin{sphinxVerbatim}[commandchars=\\\{\}]
\PYGZhy{}1.4210854715202004e\PYGZhy{}14
\end{sphinxVerbatim}

\end{sphinxuseclass}\end{sphinxVerbatimOutput}

\end{sphinxuseclass}
\sphinxAtStartPar
The \$IRR\$ is the discount rate where \$NPV=0\$.
We can can use the NumPy code above to try different discount rates until \$NPV=0\$.
The following code is crude, but sufficient for week three of our class and highlights two tools:
\begin{enumerate}
\sphinxsetlistlabels{\arabic}{enumi}{enumii}{}{.}%
\item {} 
\sphinxAtStartPar
Using a \sphinxcode{\sphinxupquote{while}} to try different discount rates until our answer is within some tolerance

\item {} 
\sphinxAtStartPar
Using NumPy to perform repetitive calculations without a \sphinxcode{\sphinxupquote{for}} loop

\end{enumerate}

\begin{sphinxuseclass}{cell}\begin{sphinxVerbatimInput}

\begin{sphinxuseclass}{cell_input}
\begin{sphinxVerbatim}[commandchars=\\\{\}]
\PYG{k}{def} \PYG{n+nf}{irr}\PYG{p}{(}\PYG{n}{c}\PYG{p}{,} \PYG{n}{guess}\PYG{o}{=}\PYG{l+m+mi}{0}\PYG{p}{,} \PYG{n}{tol}\PYG{o}{=}\PYG{l+m+mf}{0.0001}\PYG{p}{,} \PYG{n}{inc}\PYG{o}{=}\PYG{l+m+mf}{0.0001}\PYG{p}{)}\PYG{p}{:}
    \PYG{n}{npv} \PYG{o}{=} \PYG{l+m+mi}{42}
    \PYG{n}{irr} \PYG{o}{=} \PYG{n}{guess}
    \PYG{k}{while} \PYG{n}{np}\PYG{o}{.}\PYG{n}{abs}\PYG{p}{(}\PYG{n}{npv}\PYG{p}{)} \PYG{o}{\PYGZgt{}} \PYG{n}{tol}\PYG{p}{:}
        \PYG{n}{irr} \PYG{o}{+}\PYG{o}{=} \PYG{n}{inc}
        \PYG{n}{npv} \PYG{o}{=} \PYG{p}{(}\PYG{n}{c} \PYG{o}{/} \PYG{p}{(}\PYG{l+m+mi}{1} \PYG{o}{+} \PYG{n}{irr}\PYG{p}{)}\PYG{o}{*}\PYG{o}{*}\PYG{n}{np}\PYG{o}{.}\PYG{n}{arange}\PYG{p}{(}\PYG{n+nb}{len}\PYG{p}{(}\PYG{n}{c}\PYG{p}{)}\PYG{p}{)}\PYG{p}{)}\PYG{o}{.}\PYG{n}{sum}\PYG{p}{(}\PYG{p}{)}
        
    \PYG{k}{return} \PYG{n}{irr}
\end{sphinxVerbatim}

\end{sphinxuseclass}\end{sphinxVerbatimInput}

\end{sphinxuseclass}
\begin{sphinxuseclass}{cell}\begin{sphinxVerbatimInput}

\begin{sphinxuseclass}{cell_input}
\begin{sphinxVerbatim}[commandchars=\\\{\}]
\PYG{n}{c} \PYG{o}{=} \PYG{n}{np}\PYG{o}{.}\PYG{n}{array}\PYG{p}{(}\PYG{p}{[}\PYG{o}{\PYGZhy{}}\PYG{l+m+mi}{100}\PYG{p}{,} \PYG{l+m+mi}{110}\PYG{p}{]}\PYG{p}{)}
\end{sphinxVerbatim}

\end{sphinxuseclass}\end{sphinxVerbatimInput}

\end{sphinxuseclass}
\begin{sphinxuseclass}{cell}\begin{sphinxVerbatimInput}

\begin{sphinxuseclass}{cell_input}
\begin{sphinxVerbatim}[commandchars=\\\{\}]
\PYG{n}{irr}\PYG{p}{(}\PYG{n}{c}\PYG{p}{)}
\end{sphinxVerbatim}

\end{sphinxuseclass}\end{sphinxVerbatimInput}
\begin{sphinxVerbatimOutput}

\begin{sphinxuseclass}{cell_output}
\begin{sphinxVerbatim}[commandchars=\\\{\}]
0.1000
\end{sphinxVerbatim}

\end{sphinxuseclass}\end{sphinxVerbatimOutput}

\end{sphinxuseclass}

\subsubsection{Write a function \sphinxstyleliteralintitle{\sphinxupquote{returns()}} that accepts \sphinxstyleemphasis{NumPy arrays} of prices and dividends and returns a \sphinxstyleemphasis{NumPy array} of returns.}
\label{\detokenize{mckinney_04_practice_04:write-a-function-returns-that-accepts-numpy-arrays-of-prices-and-dividends-and-returns-a-numpy-array-of-returns}}
\begin{sphinxuseclass}{cell}\begin{sphinxVerbatimInput}

\begin{sphinxuseclass}{cell_input}
\begin{sphinxVerbatim}[commandchars=\\\{\}]
\PYG{n}{prices} \PYG{o}{=} \PYG{n}{np}\PYG{o}{.}\PYG{n}{array}\PYG{p}{(}\PYG{p}{[}\PYG{l+m+mi}{100}\PYG{p}{,} \PYG{l+m+mi}{150}\PYG{p}{,} \PYG{l+m+mi}{100}\PYG{p}{,} \PYG{l+m+mi}{50}\PYG{p}{,} \PYG{l+m+mi}{100}\PYG{p}{,} \PYG{l+m+mi}{150}\PYG{p}{,} \PYG{l+m+mi}{100}\PYG{p}{,} \PYG{l+m+mi}{150}\PYG{p}{]}\PYG{p}{)}
\PYG{n}{dividends} \PYG{o}{=} \PYG{n}{np}\PYG{o}{.}\PYG{n}{array}\PYG{p}{(}\PYG{p}{[}\PYG{l+m+mi}{1}\PYG{p}{,} \PYG{l+m+mi}{1}\PYG{p}{,} \PYG{l+m+mi}{1}\PYG{p}{,} \PYG{l+m+mi}{1}\PYG{p}{,} \PYG{l+m+mi}{2}\PYG{p}{,} \PYG{l+m+mi}{2}\PYG{p}{,} \PYG{l+m+mi}{2}\PYG{p}{,} \PYG{l+m+mi}{2}\PYG{p}{]}\PYG{p}{)}
\end{sphinxVerbatim}

\end{sphinxuseclass}\end{sphinxVerbatimInput}

\end{sphinxuseclass}
\begin{sphinxuseclass}{cell}\begin{sphinxVerbatimInput}

\begin{sphinxuseclass}{cell_input}
\begin{sphinxVerbatim}[commandchars=\\\{\}]
\PYG{k}{def} \PYG{n+nf}{returns}\PYG{p}{(}\PYG{n}{p}\PYG{p}{,} \PYG{n}{d}\PYG{p}{)}\PYG{p}{:}
    \PYG{k}{return} \PYG{p}{(}\PYG{n}{p}\PYG{p}{[}\PYG{l+m+mi}{1}\PYG{p}{:}\PYG{p}{]} \PYG{o}{\PYGZhy{}} \PYG{n}{p}\PYG{p}{[}\PYG{p}{:}\PYG{o}{\PYGZhy{}}\PYG{l+m+mi}{1}\PYG{p}{]} \PYG{o}{+} \PYG{n}{d}\PYG{p}{[}\PYG{l+m+mi}{1}\PYG{p}{:}\PYG{p}{]}\PYG{p}{)} \PYG{o}{/} \PYG{n}{p}\PYG{p}{[}\PYG{p}{:}\PYG{o}{\PYGZhy{}}\PYG{l+m+mi}{1}\PYG{p}{]}
\end{sphinxVerbatim}

\end{sphinxuseclass}\end{sphinxVerbatimInput}

\end{sphinxuseclass}
\begin{sphinxuseclass}{cell}\begin{sphinxVerbatimInput}

\begin{sphinxuseclass}{cell_input}
\begin{sphinxVerbatim}[commandchars=\\\{\}]
\PYG{n}{returns}\PYG{p}{(}\PYG{n}{p}\PYG{o}{=}\PYG{n}{prices}\PYG{p}{,} \PYG{n}{d}\PYG{o}{=}\PYG{n}{dividends}\PYG{p}{)}
\end{sphinxVerbatim}

\end{sphinxuseclass}\end{sphinxVerbatimInput}
\begin{sphinxVerbatimOutput}

\begin{sphinxuseclass}{cell_output}
\begin{sphinxVerbatim}[commandchars=\\\{\}]
array([ 0.51  , \PYGZhy{}0.3267, \PYGZhy{}0.49  ,  1.04  ,  0.52  , \PYGZhy{}0.32  ,  0.52  ])
\end{sphinxVerbatim}

\end{sphinxuseclass}\end{sphinxVerbatimOutput}

\end{sphinxuseclass}

\subsubsection{Rewrite the function \sphinxstyleliteralintitle{\sphinxupquote{returns()}} so it returns \sphinxstyleemphasis{NumPy arrays} of returns, capital gains yields, and dividend yields.}
\label{\detokenize{mckinney_04_practice_04:rewrite-the-function-returns-so-it-returns-numpy-arrays-of-returns-capital-gains-yields-and-dividend-yields}}
\begin{sphinxuseclass}{cell}\begin{sphinxVerbatimInput}

\begin{sphinxuseclass}{cell_input}
\begin{sphinxVerbatim}[commandchars=\\\{\}]
\PYG{k}{def} \PYG{n+nf}{returns}\PYG{p}{(}\PYG{n}{p}\PYG{p}{,} \PYG{n}{d}\PYG{p}{)}\PYG{p}{:}
    \PYG{n}{cg} \PYG{o}{=} \PYG{p}{(}\PYG{n}{p}\PYG{p}{[}\PYG{l+m+mi}{1}\PYG{p}{:}\PYG{p}{]} \PYG{o}{\PYGZhy{}} \PYG{n}{p}\PYG{p}{[}\PYG{p}{:}\PYG{o}{\PYGZhy{}}\PYG{l+m+mi}{1}\PYG{p}{]}\PYG{p}{)} \PYG{o}{/} \PYG{n}{p}\PYG{p}{[}\PYG{p}{:}\PYG{o}{\PYGZhy{}}\PYG{l+m+mi}{1}\PYG{p}{]}
    \PYG{n}{dy} \PYG{o}{=} \PYG{n}{d}\PYG{p}{[}\PYG{l+m+mi}{1}\PYG{p}{:}\PYG{p}{]} \PYG{o}{/} \PYG{n}{p}\PYG{p}{[}\PYG{p}{:}\PYG{o}{\PYGZhy{}}\PYG{l+m+mi}{1}\PYG{p}{]}
    \PYG{n}{r} \PYG{o}{=} \PYG{n}{cg} \PYG{o}{+} \PYG{n}{dy}
    \PYG{k}{return} \PYG{p}{\PYGZob{}}\PYG{l+s+s1}{\PYGZsq{}}\PYG{l+s+s1}{r}\PYG{l+s+s1}{\PYGZsq{}}\PYG{p}{:} \PYG{n}{r}\PYG{p}{,} \PYG{l+s+s1}{\PYGZsq{}}\PYG{l+s+s1}{cg}\PYG{l+s+s1}{\PYGZsq{}}\PYG{p}{:} \PYG{n}{cg}\PYG{p}{,} \PYG{l+s+s1}{\PYGZsq{}}\PYG{l+s+s1}{dy}\PYG{l+s+s1}{\PYGZsq{}}\PYG{p}{:} \PYG{n}{dy}\PYG{p}{\PYGZcb{}}
\end{sphinxVerbatim}

\end{sphinxuseclass}\end{sphinxVerbatimInput}

\end{sphinxuseclass}
\begin{sphinxuseclass}{cell}\begin{sphinxVerbatimInput}

\begin{sphinxuseclass}{cell_input}
\begin{sphinxVerbatim}[commandchars=\\\{\}]
\PYG{n}{returns}\PYG{p}{(}\PYG{n}{p}\PYG{o}{=}\PYG{n}{prices}\PYG{p}{,} \PYG{n}{d}\PYG{o}{=}\PYG{n}{dividends}\PYG{p}{)}
\end{sphinxVerbatim}

\end{sphinxuseclass}\end{sphinxVerbatimInput}
\begin{sphinxVerbatimOutput}

\begin{sphinxuseclass}{cell_output}
\begin{sphinxVerbatim}[commandchars=\\\{\}]
\PYGZob{}\PYGZsq{}r\PYGZsq{}: array([ 0.51  , \PYGZhy{}0.3267, \PYGZhy{}0.49  ,  1.04  ,  0.52  , \PYGZhy{}0.32  ,  0.52  ]),
 \PYGZsq{}cg\PYGZsq{}: array([ 0.5   , \PYGZhy{}0.3333, \PYGZhy{}0.5   ,  1.    ,  0.5   , \PYGZhy{}0.3333,  0.5   ]),
 \PYGZsq{}dy\PYGZsq{}: array([0.01  , 0.0067, 0.01  , 0.04  , 0.02  , 0.0133, 0.02  ])\PYGZcb{}
\end{sphinxVerbatim}

\end{sphinxuseclass}\end{sphinxVerbatimOutput}

\end{sphinxuseclass}

\subsubsection{Rescale and shift numbers so that they cover the range {[}0, 1{]}}
\label{\detokenize{mckinney_04_practice_04:rescale-and-shift-numbers-so-that-they-cover-the-range-0-1}}
\sphinxAtStartPar
Input: \sphinxcode{\sphinxupquote{np.array({[}18.5, 17.0, 18.0, 19.0, 18.0{]})}} \\
Output: \sphinxcode{\sphinxupquote{np.array({[}0.75, 0.0, 0.5, 1.0, 0.5{]})}}

\begin{sphinxuseclass}{cell}\begin{sphinxVerbatimInput}

\begin{sphinxuseclass}{cell_input}
\begin{sphinxVerbatim}[commandchars=\\\{\}]
\PYG{n}{numbers} \PYG{o}{=} \PYG{n}{np}\PYG{o}{.}\PYG{n}{array}\PYG{p}{(}\PYG{p}{[}\PYG{l+m+mf}{18.5}\PYG{p}{,} \PYG{l+m+mf}{17.0}\PYG{p}{,} \PYG{l+m+mf}{18.0}\PYG{p}{,} \PYG{l+m+mf}{19.0}\PYG{p}{,} \PYG{l+m+mf}{18.0}\PYG{p}{]}\PYG{p}{)}
\end{sphinxVerbatim}

\end{sphinxuseclass}\end{sphinxVerbatimInput}

\end{sphinxuseclass}
\begin{sphinxuseclass}{cell}\begin{sphinxVerbatimInput}

\begin{sphinxuseclass}{cell_input}
\begin{sphinxVerbatim}[commandchars=\\\{\}]
\PYG{p}{(}\PYG{n}{numbers} \PYG{o}{\PYGZhy{}} \PYG{n}{numbers}\PYG{o}{.}\PYG{n}{min}\PYG{p}{(}\PYG{p}{)}\PYG{p}{)} \PYG{o}{/} \PYG{p}{(}\PYG{n}{numbers}\PYG{o}{.}\PYG{n}{max}\PYG{p}{(}\PYG{p}{)} \PYG{o}{\PYGZhy{}} \PYG{n}{numbers}\PYG{o}{.}\PYG{n}{min}\PYG{p}{(}\PYG{p}{)}\PYG{p}{)}
\end{sphinxVerbatim}

\end{sphinxuseclass}\end{sphinxVerbatimInput}
\begin{sphinxVerbatimOutput}

\begin{sphinxuseclass}{cell_output}
\begin{sphinxVerbatim}[commandchars=\\\{\}]
array([0.75, 0.  , 0.5 , 1.  , 0.5 ])
\end{sphinxVerbatim}

\end{sphinxuseclass}\end{sphinxVerbatimOutput}

\end{sphinxuseclass}

\subsubsection{Write functions \sphinxstyleliteralintitle{\sphinxupquote{var()}} and \sphinxstyleliteralintitle{\sphinxupquote{std()}} that calculate variance and standard deviation.}
\label{\detokenize{mckinney_04_practice_04:write-functions-var-and-std-that-calculate-variance-and-standard-deviation}}
\sphinxAtStartPar
NumPy’s \sphinxcode{\sphinxupquote{.var()}} and \sphinxcode{\sphinxupquote{.std()}} methods return \sphinxstyleemphasis{population} statistics (i.e., denominators of \$n\$).
The pandas equivalents return \sphinxstyleemphasis{sample} statistics (denominators of \$n\sphinxhyphen{}1\$), which are more appropriate for financial data analysis where we have a sample instead of a population.

\sphinxAtStartPar
Both function should have an argument \sphinxcode{\sphinxupquote{sample}} that is \sphinxcode{\sphinxupquote{True}} by default so both functions return sample statistics by default.

\sphinxAtStartPar
Use the output of \sphinxcode{\sphinxupquote{returns()}} to compare your functions with NumPy’s \sphinxcode{\sphinxupquote{.var()}} and \sphinxcode{\sphinxupquote{.std()}} methods.

\begin{sphinxuseclass}{cell}\begin{sphinxVerbatimInput}

\begin{sphinxuseclass}{cell_input}
\begin{sphinxVerbatim}[commandchars=\\\{\}]
\PYG{n}{np}\PYG{o}{.}\PYG{n}{random}\PYG{o}{.}\PYG{n}{seed}\PYG{p}{(}\PYG{l+m+mi}{42}\PYG{p}{)}
\PYG{n}{r} \PYG{o}{=} \PYG{n}{np}\PYG{o}{.}\PYG{n}{random}\PYG{o}{.}\PYG{n}{randn}\PYG{p}{(}\PYG{l+m+mi}{1\PYGZus{}000\PYGZus{}000}\PYG{p}{)}
\PYG{n}{r}
\end{sphinxVerbatim}

\end{sphinxuseclass}\end{sphinxVerbatimInput}
\begin{sphinxVerbatimOutput}

\begin{sphinxuseclass}{cell_output}
\begin{sphinxVerbatim}[commandchars=\\\{\}]
array([ 0.4967, \PYGZhy{}0.1383,  0.6477, ..., \PYGZhy{}0.113 ,  1.4691,  0.4764])
\end{sphinxVerbatim}

\end{sphinxuseclass}\end{sphinxVerbatimOutput}

\end{sphinxuseclass}
\begin{sphinxuseclass}{cell}\begin{sphinxVerbatimInput}

\begin{sphinxuseclass}{cell_input}
\begin{sphinxVerbatim}[commandchars=\\\{\}]
\PYG{k}{def} \PYG{n+nf}{var}\PYG{p}{(}\PYG{n}{x}\PYG{p}{,} \PYG{n}{ddof}\PYG{o}{=}\PYG{l+m+mi}{0}\PYG{p}{)}\PYG{p}{:}
    \PYG{n}{N} \PYG{o}{=} \PYG{n+nb}{len}\PYG{p}{(}\PYG{n}{x}\PYG{p}{)}
    \PYG{k}{return} \PYG{p}{(}\PYG{p}{(}\PYG{n}{x} \PYG{o}{\PYGZhy{}} \PYG{n}{x}\PYG{o}{.}\PYG{n}{mean}\PYG{p}{(}\PYG{p}{)}\PYG{p}{)}\PYG{o}{*}\PYG{o}{*}\PYG{l+m+mi}{2}\PYG{p}{)}\PYG{o}{.}\PYG{n}{sum}\PYG{p}{(}\PYG{p}{)} \PYG{o}{/} \PYG{p}{(}\PYG{n}{N} \PYG{o}{\PYGZhy{}} \PYG{n}{ddof}\PYG{p}{)}
\end{sphinxVerbatim}

\end{sphinxuseclass}\end{sphinxVerbatimInput}

\end{sphinxuseclass}
\begin{sphinxuseclass}{cell}\begin{sphinxVerbatimInput}

\begin{sphinxuseclass}{cell_input}
\begin{sphinxVerbatim}[commandchars=\\\{\}]
\PYG{n}{r}\PYG{o}{.}\PYG{n}{var}\PYG{p}{(}\PYG{p}{)}
\end{sphinxVerbatim}

\end{sphinxuseclass}\end{sphinxVerbatimInput}
\begin{sphinxVerbatimOutput}

\begin{sphinxuseclass}{cell_output}
\begin{sphinxVerbatim}[commandchars=\\\{\}]
1.0003761089673018
\end{sphinxVerbatim}

\end{sphinxuseclass}\end{sphinxVerbatimOutput}

\end{sphinxuseclass}
\begin{sphinxuseclass}{cell}\begin{sphinxVerbatimInput}

\begin{sphinxuseclass}{cell_input}
\begin{sphinxVerbatim}[commandchars=\\\{\}]
\PYG{n}{var}\PYG{p}{(}\PYG{n}{r}\PYG{p}{)}
\end{sphinxVerbatim}

\end{sphinxuseclass}\end{sphinxVerbatimInput}
\begin{sphinxVerbatimOutput}

\begin{sphinxuseclass}{cell_output}
\begin{sphinxVerbatim}[commandchars=\\\{\}]
1.0003761089673018
\end{sphinxVerbatim}

\end{sphinxuseclass}\end{sphinxVerbatimOutput}

\end{sphinxuseclass}
\begin{sphinxuseclass}{cell}\begin{sphinxVerbatimInput}

\begin{sphinxuseclass}{cell_input}
\begin{sphinxVerbatim}[commandchars=\\\{\}]
\PYG{n}{np}\PYG{o}{.}\PYG{n}{allclose}\PYG{p}{(}\PYG{n}{r}\PYG{o}{.}\PYG{n}{var}\PYG{p}{(}\PYG{p}{)}\PYG{p}{,} \PYG{n}{var}\PYG{p}{(}\PYG{n}{r}\PYG{p}{)}\PYG{p}{)}
\end{sphinxVerbatim}

\end{sphinxuseclass}\end{sphinxVerbatimInput}
\begin{sphinxVerbatimOutput}

\begin{sphinxuseclass}{cell_output}
\begin{sphinxVerbatim}[commandchars=\\\{\}]
True
\end{sphinxVerbatim}

\end{sphinxuseclass}\end{sphinxVerbatimOutput}

\end{sphinxuseclass}
\begin{sphinxuseclass}{cell}\begin{sphinxVerbatimInput}

\begin{sphinxuseclass}{cell_input}
\begin{sphinxVerbatim}[commandchars=\\\{\}]
\PYG{n}{np}\PYG{o}{.}\PYG{n}{allclose}\PYG{p}{(}\PYG{n}{r}\PYG{o}{.}\PYG{n}{var}\PYG{p}{(}\PYG{n}{ddof}\PYG{o}{=}\PYG{l+m+mi}{1}\PYG{p}{)}\PYG{p}{,} \PYG{n}{var}\PYG{p}{(}\PYG{n}{r}\PYG{p}{,} \PYG{n}{ddof}\PYG{o}{=}\PYG{l+m+mi}{1}\PYG{p}{)}\PYG{p}{)}
\end{sphinxVerbatim}

\end{sphinxuseclass}\end{sphinxVerbatimInput}
\begin{sphinxVerbatimOutput}

\begin{sphinxuseclass}{cell_output}
\begin{sphinxVerbatim}[commandchars=\\\{\}]
True
\end{sphinxVerbatim}

\end{sphinxuseclass}\end{sphinxVerbatimOutput}

\end{sphinxuseclass}
\begin{sphinxuseclass}{cell}\begin{sphinxVerbatimInput}

\begin{sphinxuseclass}{cell_input}
\begin{sphinxVerbatim}[commandchars=\\\{\}]
\PYG{k}{def} \PYG{n+nf}{std}\PYG{p}{(}\PYG{n}{x}\PYG{p}{,} \PYG{n}{ddof}\PYG{o}{=}\PYG{l+m+mi}{0}\PYG{p}{)}\PYG{p}{:}
    \PYG{k}{return} \PYG{n}{np}\PYG{o}{.}\PYG{n}{sqrt}\PYG{p}{(}\PYG{n}{var}\PYG{p}{(}\PYG{n}{x}\PYG{o}{=}\PYG{n}{x}\PYG{p}{,} \PYG{n}{ddof}\PYG{o}{=}\PYG{n}{ddof}\PYG{p}{)}\PYG{p}{)}
\end{sphinxVerbatim}

\end{sphinxuseclass}\end{sphinxVerbatimInput}

\end{sphinxuseclass}
\begin{sphinxuseclass}{cell}\begin{sphinxVerbatimInput}

\begin{sphinxuseclass}{cell_input}
\begin{sphinxVerbatim}[commandchars=\\\{\}]
\PYG{n}{r}\PYG{o}{.}\PYG{n}{std}\PYG{p}{(}\PYG{p}{)}
\end{sphinxVerbatim}

\end{sphinxuseclass}\end{sphinxVerbatimInput}
\begin{sphinxVerbatimOutput}

\begin{sphinxuseclass}{cell_output}
\begin{sphinxVerbatim}[commandchars=\\\{\}]
1.000188036804731
\end{sphinxVerbatim}

\end{sphinxuseclass}\end{sphinxVerbatimOutput}

\end{sphinxuseclass}
\begin{sphinxuseclass}{cell}\begin{sphinxVerbatimInput}

\begin{sphinxuseclass}{cell_input}
\begin{sphinxVerbatim}[commandchars=\\\{\}]
\PYG{n}{std}\PYG{p}{(}\PYG{n}{r}\PYG{p}{)}
\end{sphinxVerbatim}

\end{sphinxuseclass}\end{sphinxVerbatimInput}
\begin{sphinxVerbatimOutput}

\begin{sphinxuseclass}{cell_output}
\begin{sphinxVerbatim}[commandchars=\\\{\}]
1.000188036804731
\end{sphinxVerbatim}

\end{sphinxuseclass}\end{sphinxVerbatimOutput}

\end{sphinxuseclass}
\begin{sphinxuseclass}{cell}\begin{sphinxVerbatimInput}

\begin{sphinxuseclass}{cell_input}
\begin{sphinxVerbatim}[commandchars=\\\{\}]
\PYG{n}{np}\PYG{o}{.}\PYG{n}{allclose}\PYG{p}{(}\PYG{n}{r}\PYG{o}{.}\PYG{n}{std}\PYG{p}{(}\PYG{p}{)}\PYG{p}{,} \PYG{n}{std}\PYG{p}{(}\PYG{n}{r}\PYG{p}{)}\PYG{p}{)}
\end{sphinxVerbatim}

\end{sphinxuseclass}\end{sphinxVerbatimInput}
\begin{sphinxVerbatimOutput}

\begin{sphinxuseclass}{cell_output}
\begin{sphinxVerbatim}[commandchars=\\\{\}]
True
\end{sphinxVerbatim}

\end{sphinxuseclass}\end{sphinxVerbatimOutput}

\end{sphinxuseclass}
\begin{sphinxuseclass}{cell}\begin{sphinxVerbatimInput}

\begin{sphinxuseclass}{cell_input}
\begin{sphinxVerbatim}[commandchars=\\\{\}]
\PYG{n}{np}\PYG{o}{.}\PYG{n}{allclose}\PYG{p}{(}\PYG{n}{r}\PYG{o}{.}\PYG{n}{std}\PYG{p}{(}\PYG{n}{ddof}\PYG{o}{=}\PYG{l+m+mi}{1}\PYG{p}{)}\PYG{p}{,} \PYG{n}{std}\PYG{p}{(}\PYG{n}{r}\PYG{p}{,} \PYG{n}{ddof}\PYG{o}{=}\PYG{l+m+mi}{1}\PYG{p}{)}\PYG{p}{)}
\end{sphinxVerbatim}

\end{sphinxuseclass}\end{sphinxVerbatimInput}
\begin{sphinxVerbatimOutput}

\begin{sphinxuseclass}{cell_output}
\begin{sphinxVerbatim}[commandchars=\\\{\}]
True
\end{sphinxVerbatim}

\end{sphinxuseclass}\end{sphinxVerbatimOutput}

\end{sphinxuseclass}
\sphinxstepscope


\section{McKinney Chapter 4 \sphinxhyphen{} Practice (Section 2, Wednesday 2:45 PM)}
\label{\detokenize{mckinney_04_practice_02:mckinney-chapter-4-practice-section-2-wednesday-2-45-pm}}\label{\detokenize{mckinney_04_practice_02::doc}}
\begin{sphinxuseclass}{cell}\begin{sphinxVerbatimInput}

\begin{sphinxuseclass}{cell_input}
\begin{sphinxVerbatim}[commandchars=\\\{\}]
\PYG{k+kn}{import} \PYG{n+nn}{numpy} \PYG{k}{as} \PYG{n+nn}{np}
\PYG{o}{\PYGZpc{}}\PYG{k}{precision} 4
\end{sphinxVerbatim}

\end{sphinxuseclass}\end{sphinxVerbatimInput}
\begin{sphinxVerbatimOutput}

\begin{sphinxuseclass}{cell_output}
\begin{sphinxVerbatim}[commandchars=\\\{\}]
\PYGZsq{}\PYGZpc{}.4f\PYGZsq{}
\end{sphinxVerbatim}

\end{sphinxuseclass}\end{sphinxVerbatimOutput}

\end{sphinxuseclass}

\subsection{Practice}
\label{\detokenize{mckinney_04_practice_02:practice}}

\subsubsection{Create a 1\sphinxhyphen{}dimensional array named \sphinxstyleliteralintitle{\sphinxupquote{a1}} that counts from 0 to 24 by 1.}
\label{\detokenize{mckinney_04_practice_02:create-a-1-dimensional-array-named-a1-that-counts-from-0-to-24-by-1}}
\begin{sphinxuseclass}{cell}\begin{sphinxVerbatimInput}

\begin{sphinxuseclass}{cell_input}
\begin{sphinxVerbatim}[commandchars=\\\{\}]
\PYG{n}{np}\PYG{o}{.}\PYG{n}{array}\PYG{p}{(}\PYG{n+nb}{range}\PYG{p}{(}\PYG{l+m+mi}{25}\PYG{p}{)}\PYG{p}{)}
\end{sphinxVerbatim}

\end{sphinxuseclass}\end{sphinxVerbatimInput}
\begin{sphinxVerbatimOutput}

\begin{sphinxuseclass}{cell_output}
\begin{sphinxVerbatim}[commandchars=\\\{\}]
array([ 0,  1,  2,  3,  4,  5,  6,  7,  8,  9, 10, 11, 12, 13, 14, 15, 16,
       17, 18, 19, 20, 21, 22, 23, 24])
\end{sphinxVerbatim}

\end{sphinxuseclass}\end{sphinxVerbatimOutput}

\end{sphinxuseclass}
\begin{sphinxuseclass}{cell}\begin{sphinxVerbatimInput}

\begin{sphinxuseclass}{cell_input}
\begin{sphinxVerbatim}[commandchars=\\\{\}]
\PYG{n}{np}\PYG{o}{.}\PYG{n}{arange}\PYG{p}{(}\PYG{l+m+mi}{25}\PYG{p}{)}
\end{sphinxVerbatim}

\end{sphinxuseclass}\end{sphinxVerbatimInput}
\begin{sphinxVerbatimOutput}

\begin{sphinxuseclass}{cell_output}
\begin{sphinxVerbatim}[commandchars=\\\{\}]
array([ 0,  1,  2,  3,  4,  5,  6,  7,  8,  9, 10, 11, 12, 13, 14, 15, 16,
       17, 18, 19, 20, 21, 22, 23, 24])
\end{sphinxVerbatim}

\end{sphinxuseclass}\end{sphinxVerbatimOutput}

\end{sphinxuseclass}
\begin{sphinxuseclass}{cell}\begin{sphinxVerbatimInput}

\begin{sphinxuseclass}{cell_input}
\begin{sphinxVerbatim}[commandchars=\\\{\}]
\PYG{n}{a1} \PYG{o}{=} \PYG{n}{np}\PYG{o}{.}\PYG{n}{arange}\PYG{p}{(}\PYG{l+m+mi}{25}\PYG{p}{)}
\end{sphinxVerbatim}

\end{sphinxuseclass}\end{sphinxVerbatimInput}

\end{sphinxuseclass}

\subsubsection{Create a 1\sphinxhyphen{}dimentional array named \sphinxstyleliteralintitle{\sphinxupquote{a2}} that counts from 0 to 24 by 3.}
\label{\detokenize{mckinney_04_practice_02:create-a-1-dimentional-array-named-a2-that-counts-from-0-to-24-by-3}}
\begin{sphinxuseclass}{cell}\begin{sphinxVerbatimInput}

\begin{sphinxuseclass}{cell_input}
\begin{sphinxVerbatim}[commandchars=\\\{\}]
\PYG{n}{a2} \PYG{o}{=} \PYG{n}{np}\PYG{o}{.}\PYG{n}{arange}\PYG{p}{(}\PYG{l+m+mi}{0}\PYG{p}{,} \PYG{l+m+mi}{25}\PYG{p}{,} \PYG{l+m+mi}{3}\PYG{p}{)}
\PYG{n}{a2}
\end{sphinxVerbatim}

\end{sphinxuseclass}\end{sphinxVerbatimInput}
\begin{sphinxVerbatimOutput}

\begin{sphinxuseclass}{cell_output}
\begin{sphinxVerbatim}[commandchars=\\\{\}]
array([ 0,  3,  6,  9, 12, 15, 18, 21, 24])
\end{sphinxVerbatim}

\end{sphinxuseclass}\end{sphinxVerbatimOutput}

\end{sphinxuseclass}

\subsubsection{Create a 1\sphinxhyphen{}dimentional array named \sphinxstyleliteralintitle{\sphinxupquote{a3}} that counts from 0 to 100 by multiples of 3 or 5.}
\label{\detokenize{mckinney_04_practice_02:create-a-1-dimentional-array-named-a3-that-counts-from-0-to-100-by-multiples-of-3-or-5}}
\begin{sphinxuseclass}{cell}\begin{sphinxVerbatimInput}

\begin{sphinxuseclass}{cell_input}
\begin{sphinxVerbatim}[commandchars=\\\{\}]
\PYG{n}{np}\PYG{o}{.}\PYG{n}{array}\PYG{p}{(}\PYG{p}{[}\PYG{n}{i} \PYG{k}{for} \PYG{n}{i} \PYG{o+ow}{in} \PYG{n+nb}{range}\PYG{p}{(}\PYG{l+m+mi}{101}\PYG{p}{)} \PYG{k}{if} \PYG{p}{(}\PYG{n}{i}\PYG{o}{\PYGZpc{}}\PYG{k}{3}==0) or (i\PYGZpc{}5==0)])
\end{sphinxVerbatim}

\end{sphinxuseclass}\end{sphinxVerbatimInput}
\begin{sphinxVerbatimOutput}

\begin{sphinxuseclass}{cell_output}
\begin{sphinxVerbatim}[commandchars=\\\{\}]
array([  0,   3,   5,   6,   9,  10,  12,  15,  18,  20,  21,  24,  25,
        27,  30,  33,  35,  36,  39,  40,  42,  45,  48,  50,  51,  54,
        55,  57,  60,  63,  65,  66,  69,  70,  72,  75,  78,  80,  81,
        84,  85,  87,  90,  93,  95,  96,  99, 100])
\end{sphinxVerbatim}

\end{sphinxuseclass}\end{sphinxVerbatimOutput}

\end{sphinxuseclass}
\begin{sphinxuseclass}{cell}\begin{sphinxVerbatimInput}

\begin{sphinxuseclass}{cell_input}
\begin{sphinxVerbatim}[commandchars=\\\{\}]
\PYG{n}{a3} \PYG{o}{=} \PYG{n}{np}\PYG{o}{.}\PYG{n}{arange}\PYG{p}{(}\PYG{l+m+mi}{101}\PYG{p}{)}
\PYG{n}{a3} \PYG{o}{=} \PYG{n}{a3}\PYG{p}{[}\PYG{p}{(}\PYG{n}{a3}\PYG{o}{\PYGZpc{}}\PYG{k}{3}==0) | (a3\PYGZpc{}5==0)] \PYGZsh{} NumPy only accepts \PYGZam{} for \PYGZdq{}and\PYGZdq{} and | for \PYGZdq{}or\PYGZdq{}
\end{sphinxVerbatim}

\end{sphinxuseclass}\end{sphinxVerbatimInput}

\end{sphinxuseclass}

\subsubsection{Create a 1\sphinxhyphen{}dimensional array \sphinxstyleliteralintitle{\sphinxupquote{a3}} that contains the squares of the even integers through 100,000.}
\label{\detokenize{mckinney_04_practice_02:create-a-1-dimensional-array-a3-that-contains-the-squares-of-the-even-integers-through-100-000}}
\sphinxAtStartPar
How much faster is the NumPy version than the list comprehension version?

\begin{sphinxuseclass}{cell}\begin{sphinxVerbatimInput}

\begin{sphinxuseclass}{cell_input}
\begin{sphinxVerbatim}[commandchars=\\\{\}]
\PYG{n}{np}\PYG{o}{.}\PYG{n}{array}\PYG{p}{(}\PYG{p}{[}\PYG{n}{i}\PYG{o}{*}\PYG{o}{*}\PYG{l+m+mi}{2} \PYG{k}{for} \PYG{n}{i} \PYG{o+ow}{in} \PYG{n+nb}{range}\PYG{p}{(}\PYG{l+m+mi}{100\PYGZus{}001}\PYG{p}{)} \PYG{k}{if} \PYG{p}{(}\PYG{n}{i}\PYG{o}{\PYGZpc{}}\PYG{k}{2}==0)])
\end{sphinxVerbatim}

\end{sphinxuseclass}\end{sphinxVerbatimInput}
\begin{sphinxVerbatimOutput}

\begin{sphinxuseclass}{cell_output}
\begin{sphinxVerbatim}[commandchars=\\\{\}]
array([          0,           4,          16, ...,  9999200016,
        9999600004, 10000000000])
\end{sphinxVerbatim}

\end{sphinxuseclass}\end{sphinxVerbatimOutput}

\end{sphinxuseclass}
\begin{sphinxuseclass}{cell}\begin{sphinxVerbatimInput}

\begin{sphinxuseclass}{cell_input}
\begin{sphinxVerbatim}[commandchars=\\\{\}]
\PYG{n}{np}\PYG{o}{.}\PYG{n}{array}\PYG{p}{(}\PYG{p}{[}\PYG{n}{i}\PYG{o}{*}\PYG{o}{*}\PYG{l+m+mi}{2} \PYG{k}{for} \PYG{n}{i} \PYG{o+ow}{in} \PYG{n+nb}{range}\PYG{p}{(}\PYG{l+m+mi}{0}\PYG{p}{,} \PYG{l+m+mi}{100\PYGZus{}001}\PYG{p}{,} \PYG{l+m+mi}{2}\PYG{p}{)}\PYG{p}{]}\PYG{p}{)}
\end{sphinxVerbatim}

\end{sphinxuseclass}\end{sphinxVerbatimInput}
\begin{sphinxVerbatimOutput}

\begin{sphinxuseclass}{cell_output}
\begin{sphinxVerbatim}[commandchars=\\\{\}]
array([          0,           4,          16, ...,  9999200016,
        9999600004, 10000000000])
\end{sphinxVerbatim}

\end{sphinxuseclass}\end{sphinxVerbatimOutput}

\end{sphinxuseclass}
\begin{sphinxuseclass}{cell}\begin{sphinxVerbatimInput}

\begin{sphinxuseclass}{cell_input}
\begin{sphinxVerbatim}[commandchars=\\\{\}]
\PYG{n}{np}\PYG{o}{.}\PYG{n}{arange}\PYG{p}{(}\PYG{l+m+mi}{0}\PYG{p}{,} \PYG{l+m+mi}{100\PYGZus{}001}\PYG{p}{,} \PYG{l+m+mi}{2}\PYG{p}{,} \PYG{n}{dtype}\PYG{o}{=}\PYG{n}{np}\PYG{o}{.}\PYG{n}{int64}\PYG{p}{)}\PYG{o}{*}\PYG{o}{*}\PYG{l+m+mi}{2}
\end{sphinxVerbatim}

\end{sphinxuseclass}\end{sphinxVerbatimInput}

\end{sphinxuseclass}

\bigskip\hrule\bigskip


\sphinxAtStartPar
\sphinxstyleemphasis{\sphinxstylestrong{Note:}}
On some platforms, \sphinxcode{\sphinxupquote{np.arange(0, 100\_001, 2)}} returns an array of 32\sphinxhyphen{}bit integers.
If we square this array of 32\sphinxhyphen{}bit integers, we get the wrong answer because the large values (e.g., \$100,000\textasciicircum{}2\$) are too large to represent as 32\sphinxhyphen{}bit integers.
Since we know that we need 54\sphinxhyphen{}bit integers for this calculation, we should explcitly set either \sphinxcode{\sphinxupquote{dtype='int64'}} or \sphinxcode{\sphinxupquote{dtype=np.int64}}.

\begin{sphinxuseclass}{cell}\begin{sphinxVerbatimInput}

\begin{sphinxuseclass}{cell_input}
\begin{sphinxVerbatim}[commandchars=\\\{\}]
\PYG{n}{np}\PYG{o}{.}\PYG{n}{arange}\PYG{p}{(}\PYG{l+m+mi}{0}\PYG{p}{,} \PYG{l+m+mi}{100\PYGZus{}001}\PYG{p}{,} \PYG{l+m+mi}{2}\PYG{p}{,} \PYG{n}{dtype}\PYG{o}{=}\PYG{l+s+s1}{\PYGZsq{}}\PYG{l+s+s1}{int64}\PYG{l+s+s1}{\PYGZsq{}}\PYG{p}{)}\PYG{o}{*}\PYG{o}{*}\PYG{l+m+mi}{2}
\end{sphinxVerbatim}

\end{sphinxuseclass}\end{sphinxVerbatimInput}
\begin{sphinxVerbatimOutput}

\begin{sphinxuseclass}{cell_output}
\begin{sphinxVerbatim}[commandchars=\\\{\}]
array([          0,           4,          16, ...,  9999200016,
        9999600004, 10000000000])
\end{sphinxVerbatim}

\end{sphinxuseclass}\end{sphinxVerbatimOutput}

\end{sphinxuseclass}
\begin{sphinxuseclass}{cell}\begin{sphinxVerbatimInput}

\begin{sphinxuseclass}{cell_input}
\begin{sphinxVerbatim}[commandchars=\\\{\}]
\PYG{n}{np}\PYG{o}{.}\PYG{n}{arange}\PYG{p}{(}\PYG{l+m+mi}{0}\PYG{p}{,} \PYG{l+m+mi}{100\PYGZus{}001}\PYG{p}{,} \PYG{l+m+mi}{2}\PYG{p}{,} \PYG{n}{dtype}\PYG{o}{=}\PYG{n}{np}\PYG{o}{.}\PYG{n}{int64}\PYG{p}{)}\PYG{o}{*}\PYG{o}{*}\PYG{l+m+mi}{2}
\end{sphinxVerbatim}

\end{sphinxuseclass}\end{sphinxVerbatimInput}
\begin{sphinxVerbatimOutput}

\begin{sphinxuseclass}{cell_output}
\begin{sphinxVerbatim}[commandchars=\\\{\}]
array([          0,           4,          16, ...,  9999200016,
        9999600004, 10000000000])
\end{sphinxVerbatim}

\end{sphinxuseclass}\end{sphinxVerbatimOutput}

\end{sphinxuseclass}
\sphinxAtStartPar
This \sphinxhref{https://stackoverflow.com/a/1970697/334755}{StackOverflow answer} is this best explanation I have found of this behavior.


\bigskip\hrule\bigskip



\subsubsection{Write a function that mimic Excel’s \sphinxstyleliteralintitle{\sphinxupquote{pv}} function.}
\label{\detokenize{mckinney_04_practice_02:write-a-function-that-mimic-excel-s-pv-function}}
\sphinxAtStartPar
Here is how we call Excel’s \sphinxcode{\sphinxupquote{pv}} function:
\sphinxcode{\sphinxupquote{=PV(rate, nper, pmt, {[}fv{]}, {[}type{]})}}
We can use the annuity and lump sum present value formulas.

\sphinxAtStartPar
Present value of an annuity payment \sphinxcode{\sphinxupquote{pmt}}:
\$PV\_\{pmt\} = \textbackslash{}frac\{pmt\}\{rate\} \textbackslash{}times \textbackslash{}left(1 \sphinxhyphen{} \textbackslash{}frac\{1\}\{(1+rate)\textasciicircum{}\{nper\}\} \textbackslash{}right)\$

\sphinxAtStartPar
Present value of a lump sum \sphinxcode{\sphinxupquote{fv}}:
\$PV\_\{fv\} = \textbackslash{}frac\{fv\}\{(1+rate)\textasciicircum{}\{nper\}\}\$

\begin{sphinxuseclass}{cell}\begin{sphinxVerbatimInput}

\begin{sphinxuseclass}{cell_input}
\begin{sphinxVerbatim}[commandchars=\\\{\}]
\PYG{k}{def} \PYG{n+nf}{pv}\PYG{p}{(}\PYG{n}{rate}\PYG{p}{,} \PYG{n}{nper}\PYG{p}{,} \PYG{n}{pmt}\PYG{o}{=}\PYG{k+kc}{None}\PYG{p}{,} \PYG{n}{fv}\PYG{o}{=}\PYG{k+kc}{None}\PYG{p}{,} \PYG{n+nb}{type} \PYG{o}{=} \PYG{l+s+s1}{\PYGZsq{}}\PYG{l+s+s1}{END}\PYG{l+s+s1}{\PYGZsq{}}\PYG{p}{)}\PYG{p}{:}
    \PYG{k}{if} \PYG{n}{pmt} \PYG{o+ow}{is} \PYG{o+ow}{not} \PYG{k+kc}{None}\PYG{p}{:} \PYG{c+c1}{\PYGZsh{} calculate PV of pmt, if given}
        \PYG{n}{pv\PYGZus{}pmt} \PYG{o}{=} \PYG{p}{(}\PYG{n}{pmt} \PYG{o}{/} \PYG{n}{rate}\PYG{p}{)} \PYG{o}{*} \PYG{p}{(}\PYG{l+m+mi}{1} \PYG{o}{\PYGZhy{}} \PYG{p}{(}\PYG{l+m+mi}{1} \PYG{o}{+} \PYG{n}{rate}\PYG{p}{)}\PYG{o}{*}\PYG{o}{*}\PYG{p}{(}\PYG{o}{\PYGZhy{}}\PYG{n}{nper}\PYG{p}{)}\PYG{p}{)}
    \PYG{k}{else}\PYG{p}{:}
        \PYG{n}{pv\PYGZus{}pmt} \PYG{o}{=} \PYG{l+m+mi}{0}

    \PYG{k}{if} \PYG{n}{fv} \PYG{o+ow}{is} \PYG{o+ow}{not} \PYG{k+kc}{None}\PYG{p}{:} \PYG{c+c1}{\PYGZsh{} calculate PV of fv, if given}
        \PYG{n}{pv\PYGZus{}fv} \PYG{o}{=} \PYG{n}{fv} \PYG{o}{/} \PYG{p}{(}\PYG{l+m+mi}{1} \PYG{o}{+} \PYG{n}{rate}\PYG{p}{)}\PYG{o}{*}\PYG{o}{*}\PYG{n}{nper}
    \PYG{k}{else}\PYG{p}{:}
        \PYG{n}{pv\PYGZus{}fv} \PYG{o}{=} \PYG{l+m+mi}{0}
    
    \PYG{k}{if} \PYG{n+nb}{type}\PYG{o}{==}\PYG{l+s+s1}{\PYGZsq{}}\PYG{l+s+s1}{END}\PYG{l+s+s1}{\PYGZsq{}}\PYG{p}{:} \PYG{c+c1}{\PYGZsh{} as\PYGZhy{}is if end of period payments}
        \PYG{k}{return} \PYG{o}{\PYGZhy{}}\PYG{l+m+mi}{1} \PYG{o}{*} \PYG{p}{(}\PYG{n}{pv\PYGZus{}pmt} \PYG{o}{+} \PYG{n}{pv\PYGZus{}fv}\PYG{p}{)}
    \PYG{k}{elif} \PYG{n+nb}{type}\PYG{o}{==}\PYG{l+s+s1}{\PYGZsq{}}\PYG{l+s+s1}{BGN}\PYG{l+s+s1}{\PYGZsq{}}\PYG{p}{:} \PYG{c+c1}{\PYGZsh{} undo one period of discounting if bgn of period payments}
        \PYG{k}{return} \PYG{o}{\PYGZhy{}}\PYG{l+m+mi}{1} \PYG{o}{*} \PYG{p}{(}\PYG{n}{pv\PYGZus{}pmt} \PYG{o}{+} \PYG{n}{pv\PYGZus{}fv}\PYG{p}{)} \PYG{o}{*} \PYG{p}{(}\PYG{l+m+mi}{1} \PYG{o}{+} \PYG{n}{rate}\PYG{p}{)}
    \PYG{k}{else}\PYG{p}{:} \PYG{c+c1}{\PYGZsh{} otherwise, ask use to specify end or bgn of period payments}
        \PYG{n+nb}{print}\PYG{p}{(}\PYG{l+s+s1}{\PYGZsq{}}\PYG{l+s+s1}{Please enter END or BGN for the type argument}\PYG{l+s+s1}{\PYGZsq{}}\PYG{p}{)}
\end{sphinxVerbatim}

\end{sphinxuseclass}\end{sphinxVerbatimInput}

\end{sphinxuseclass}
\begin{sphinxuseclass}{cell}\begin{sphinxVerbatimInput}

\begin{sphinxuseclass}{cell_input}
\begin{sphinxVerbatim}[commandchars=\\\{\}]
\PYG{n}{pv}\PYG{p}{(}\PYG{n}{rate}\PYG{o}{=}\PYG{l+m+mf}{0.1}\PYG{p}{,} \PYG{n}{nper}\PYG{o}{=}\PYG{l+m+mi}{10}\PYG{p}{,} \PYG{n}{pmt}\PYG{o}{=}\PYG{l+m+mi}{100}\PYG{p}{,} \PYG{n}{fv}\PYG{o}{=}\PYG{l+m+mi}{1000}\PYG{p}{,} \PYG{n+nb}{type}\PYG{o}{=}\PYG{l+s+s1}{\PYGZsq{}}\PYG{l+s+s1}{END}\PYG{l+s+s1}{\PYGZsq{}}\PYG{p}{)}
\end{sphinxVerbatim}

\end{sphinxuseclass}\end{sphinxVerbatimInput}
\begin{sphinxVerbatimOutput}

\begin{sphinxuseclass}{cell_output}
\begin{sphinxVerbatim}[commandchars=\\\{\}]
\PYGZhy{}1000.0000
\end{sphinxVerbatim}

\end{sphinxuseclass}\end{sphinxVerbatimOutput}

\end{sphinxuseclass}

\subsubsection{Write a function that mimic Excel’s \sphinxstyleliteralintitle{\sphinxupquote{fv}} function.}
\label{\detokenize{mckinney_04_practice_02:write-a-function-that-mimic-excel-s-fv-function}}
\begin{sphinxuseclass}{cell}\begin{sphinxVerbatimInput}

\begin{sphinxuseclass}{cell_input}
\begin{sphinxVerbatim}[commandchars=\\\{\}]
\PYG{k}{def} \PYG{n+nf}{fv}\PYG{p}{(}\PYG{n}{rate}\PYG{p}{,} \PYG{n}{nper}\PYG{p}{,} \PYG{n}{pmt}\PYG{o}{=}\PYG{k+kc}{None}\PYG{p}{,} \PYG{n}{pv}\PYG{o}{=}\PYG{k+kc}{None}\PYG{p}{,} \PYG{n+nb}{type} \PYG{o}{=} \PYG{l+s+s1}{\PYGZsq{}}\PYG{l+s+s1}{END}\PYG{l+s+s1}{\PYGZsq{}}\PYG{p}{)}\PYG{p}{:}
    \PYG{k}{if} \PYG{n}{pmt} \PYG{o+ow}{is} \PYG{o+ow}{not} \PYG{k+kc}{None}\PYG{p}{:} \PYG{c+c1}{\PYGZsh{} calculate PV of pmt, if given}
        \PYG{n}{fv\PYGZus{}pmt} \PYG{o}{=} \PYG{p}{(}\PYG{n}{pmt} \PYG{o}{/} \PYG{n}{rate}\PYG{p}{)} \PYG{o}{*} \PYG{p}{(}\PYG{p}{(}\PYG{l+m+mi}{1} \PYG{o}{+} \PYG{n}{rate}\PYG{p}{)}\PYG{o}{*}\PYG{o}{*}\PYG{n}{nper} \PYG{o}{\PYGZhy{}} \PYG{l+m+mi}{1}\PYG{p}{)}
    \PYG{k}{else}\PYG{p}{:}
        \PYG{n}{fv\PYGZus{}pmt} \PYG{o}{=} \PYG{l+m+mi}{0}

    \PYG{k}{if} \PYG{n}{fv} \PYG{o+ow}{is} \PYG{o+ow}{not} \PYG{k+kc}{None}\PYG{p}{:} \PYG{c+c1}{\PYGZsh{} calculate PV of fv, if given}
        \PYG{n}{fv\PYGZus{}pv} \PYG{o}{=} \PYG{n}{pv} \PYG{o}{*} \PYG{p}{(}\PYG{l+m+mi}{1} \PYG{o}{+} \PYG{n}{rate}\PYG{p}{)}\PYG{o}{*}\PYG{o}{*}\PYG{n}{nper}
    \PYG{k}{else}\PYG{p}{:}
        \PYG{n}{fv\PYGZus{}fv} \PYG{o}{=} \PYG{l+m+mi}{0}
    
    \PYG{k}{if} \PYG{n+nb}{type}\PYG{o}{==}\PYG{l+s+s1}{\PYGZsq{}}\PYG{l+s+s1}{END}\PYG{l+s+s1}{\PYGZsq{}}\PYG{p}{:} \PYG{c+c1}{\PYGZsh{} as\PYGZhy{}is if end of period payments}
        \PYG{k}{return} \PYG{o}{\PYGZhy{}}\PYG{l+m+mi}{1} \PYG{o}{*} \PYG{p}{(}\PYG{n}{fv\PYGZus{}pmt} \PYG{o}{+} \PYG{n}{fv\PYGZus{}pv}\PYG{p}{)}
    \PYG{k}{elif} \PYG{n+nb}{type}\PYG{o}{==}\PYG{l+s+s1}{\PYGZsq{}}\PYG{l+s+s1}{BGN}\PYG{l+s+s1}{\PYGZsq{}}\PYG{p}{:} \PYG{c+c1}{\PYGZsh{} undo one period of discounting if bgn of period payments}
        \PYG{k}{return} \PYG{o}{\PYGZhy{}}\PYG{l+m+mi}{1} \PYG{o}{*} \PYG{p}{(}\PYG{n}{fv\PYGZus{}pmt} \PYG{o}{+} \PYG{n}{fv\PYGZus{}pv}\PYG{p}{)} \PYG{o}{*} \PYG{p}{(}\PYG{l+m+mi}{1} \PYG{o}{+} \PYG{n}{rate}\PYG{p}{)}
    \PYG{k}{else}\PYG{p}{:} \PYG{c+c1}{\PYGZsh{} otherwise, ask use to specify end or bgn of period payments}
        \PYG{n+nb}{print}\PYG{p}{(}\PYG{l+s+s1}{\PYGZsq{}}\PYG{l+s+s1}{Please enter END or BGN for the type argument}\PYG{l+s+s1}{\PYGZsq{}}\PYG{p}{)}
\end{sphinxVerbatim}

\end{sphinxuseclass}\end{sphinxVerbatimInput}

\end{sphinxuseclass}
\begin{sphinxuseclass}{cell}\begin{sphinxVerbatimInput}

\begin{sphinxuseclass}{cell_input}
\begin{sphinxVerbatim}[commandchars=\\\{\}]
\PYG{n}{fv}\PYG{p}{(}\PYG{n}{rate}\PYG{o}{=}\PYG{l+m+mf}{0.1}\PYG{p}{,} \PYG{n}{nper}\PYG{o}{=}\PYG{l+m+mi}{10}\PYG{p}{,} \PYG{n}{pmt}\PYG{o}{=}\PYG{l+m+mi}{100}\PYG{p}{,} \PYG{n}{pv}\PYG{o}{=}\PYG{o}{\PYGZhy{}}\PYG{l+m+mi}{1000}\PYG{p}{,} \PYG{n+nb}{type}\PYG{o}{=}\PYG{l+s+s1}{\PYGZsq{}}\PYG{l+s+s1}{END}\PYG{l+s+s1}{\PYGZsq{}}\PYG{p}{)}
\end{sphinxVerbatim}

\end{sphinxuseclass}\end{sphinxVerbatimInput}
\begin{sphinxVerbatimOutput}

\begin{sphinxuseclass}{cell_output}
\begin{sphinxVerbatim}[commandchars=\\\{\}]
1000.0000
\end{sphinxVerbatim}

\end{sphinxuseclass}\end{sphinxVerbatimOutput}

\end{sphinxuseclass}

\subsubsection{Replace the negative values in \sphinxstyleliteralintitle{\sphinxupquote{data}} with \sphinxhyphen{}1 and positive values with +1.}
\label{\detokenize{mckinney_04_practice_02:replace-the-negative-values-in-data-with-1-and-positive-values-with-1}}
\begin{sphinxuseclass}{cell}\begin{sphinxVerbatimInput}

\begin{sphinxuseclass}{cell_input}
\begin{sphinxVerbatim}[commandchars=\\\{\}]
\PYG{n}{np}\PYG{o}{.}\PYG{n}{random}\PYG{o}{.}\PYG{n}{seed}\PYG{p}{(}\PYG{l+m+mi}{42}\PYG{p}{)}
\PYG{n}{data} \PYG{o}{=} \PYG{n}{np}\PYG{o}{.}\PYG{n}{random}\PYG{o}{.}\PYG{n}{randn}\PYG{p}{(}\PYG{l+m+mi}{7}\PYG{p}{,} \PYG{l+m+mi}{7}\PYG{p}{)}
\PYG{n}{data}
\end{sphinxVerbatim}

\end{sphinxuseclass}\end{sphinxVerbatimInput}
\begin{sphinxVerbatimOutput}

\begin{sphinxuseclass}{cell_output}
\begin{sphinxVerbatim}[commandchars=\\\{\}]
array([[ 0.4967, \PYGZhy{}0.1383,  0.6477,  1.523 , \PYGZhy{}0.2342, \PYGZhy{}0.2341,  1.5792],
       [ 0.7674, \PYGZhy{}0.4695,  0.5426, \PYGZhy{}0.4634, \PYGZhy{}0.4657,  0.242 , \PYGZhy{}1.9133],
       [\PYGZhy{}1.7249, \PYGZhy{}0.5623, \PYGZhy{}1.0128,  0.3142, \PYGZhy{}0.908 , \PYGZhy{}1.4123,  1.4656],
       [\PYGZhy{}0.2258,  0.0675, \PYGZhy{}1.4247, \PYGZhy{}0.5444,  0.1109, \PYGZhy{}1.151 ,  0.3757],
       [\PYGZhy{}0.6006, \PYGZhy{}0.2917, \PYGZhy{}0.6017,  1.8523, \PYGZhy{}0.0135, \PYGZhy{}1.0577,  0.8225],
       [\PYGZhy{}1.2208,  0.2089, \PYGZhy{}1.9597, \PYGZhy{}1.3282,  0.1969,  0.7385,  0.1714],
       [\PYGZhy{}0.1156, \PYGZhy{}0.3011, \PYGZhy{}1.4785, \PYGZhy{}0.7198, \PYGZhy{}0.4606,  1.0571,  0.3436]])
\end{sphinxVerbatim}

\end{sphinxuseclass}\end{sphinxVerbatimOutput}

\end{sphinxuseclass}
\begin{sphinxuseclass}{cell}\begin{sphinxVerbatimInput}

\begin{sphinxuseclass}{cell_input}
\begin{sphinxVerbatim}[commandchars=\\\{\}]
\PYG{n}{data\PYGZus{}c} \PYG{o}{=} \PYG{n}{data}\PYG{o}{.}\PYG{n}{copy}\PYG{p}{(}\PYG{p}{)}
\PYG{n}{data\PYGZus{}c}\PYG{p}{[}\PYG{n}{data\PYGZus{}c} \PYG{o}{\PYGZlt{}} \PYG{l+m+mi}{0}\PYG{p}{]} \PYG{o}{=} \PYG{o}{\PYGZhy{}}\PYG{l+m+mi}{1}
\PYG{n}{data\PYGZus{}c}\PYG{p}{[}\PYG{n}{data\PYGZus{}c} \PYG{o}{\PYGZgt{}} \PYG{l+m+mi}{0}\PYG{p}{]} \PYG{o}{=} \PYG{o}{+}\PYG{l+m+mi}{1}
\PYG{n}{data\PYGZus{}c}
\end{sphinxVerbatim}

\end{sphinxuseclass}\end{sphinxVerbatimInput}
\begin{sphinxVerbatimOutput}

\begin{sphinxuseclass}{cell_output}
\begin{sphinxVerbatim}[commandchars=\\\{\}]
array([[ 1., \PYGZhy{}1.,  1.,  1., \PYGZhy{}1., \PYGZhy{}1.,  1.],
       [ 1., \PYGZhy{}1.,  1., \PYGZhy{}1., \PYGZhy{}1.,  1., \PYGZhy{}1.],
       [\PYGZhy{}1., \PYGZhy{}1., \PYGZhy{}1.,  1., \PYGZhy{}1., \PYGZhy{}1.,  1.],
       [\PYGZhy{}1.,  1., \PYGZhy{}1., \PYGZhy{}1.,  1., \PYGZhy{}1.,  1.],
       [\PYGZhy{}1., \PYGZhy{}1., \PYGZhy{}1.,  1., \PYGZhy{}1., \PYGZhy{}1.,  1.],
       [\PYGZhy{}1.,  1., \PYGZhy{}1., \PYGZhy{}1.,  1.,  1.,  1.],
       [\PYGZhy{}1., \PYGZhy{}1., \PYGZhy{}1., \PYGZhy{}1., \PYGZhy{}1.,  1.,  1.]])
\end{sphinxVerbatim}

\end{sphinxuseclass}\end{sphinxVerbatimOutput}

\end{sphinxuseclass}
\begin{sphinxuseclass}{cell}\begin{sphinxVerbatimInput}

\begin{sphinxuseclass}{cell_input}
\begin{sphinxVerbatim}[commandchars=\\\{\}]
\PYG{n}{np}\PYG{o}{.}\PYG{n}{where}\PYG{p}{(}\PYG{n}{data} \PYG{o}{\PYGZlt{}} \PYG{l+m+mi}{0}\PYG{p}{,} \PYG{o}{\PYGZhy{}}\PYG{l+m+mi}{1}\PYG{p}{,} \PYG{n}{np}\PYG{o}{.}\PYG{n}{where}\PYG{p}{(}\PYG{n}{data} \PYG{o}{\PYGZgt{}} \PYG{l+m+mi}{0}\PYG{p}{,} \PYG{o}{+}\PYG{l+m+mi}{1}\PYG{p}{,} \PYG{n}{data}\PYG{p}{)}\PYG{p}{)}
\end{sphinxVerbatim}

\end{sphinxuseclass}\end{sphinxVerbatimInput}
\begin{sphinxVerbatimOutput}

\begin{sphinxuseclass}{cell_output}
\begin{sphinxVerbatim}[commandchars=\\\{\}]
array([[ 1., \PYGZhy{}1.,  1.,  1., \PYGZhy{}1., \PYGZhy{}1.,  1.],
       [ 1., \PYGZhy{}1.,  1., \PYGZhy{}1., \PYGZhy{}1.,  1., \PYGZhy{}1.],
       [\PYGZhy{}1., \PYGZhy{}1., \PYGZhy{}1.,  1., \PYGZhy{}1., \PYGZhy{}1.,  1.],
       [\PYGZhy{}1.,  1., \PYGZhy{}1., \PYGZhy{}1.,  1., \PYGZhy{}1.,  1.],
       [\PYGZhy{}1., \PYGZhy{}1., \PYGZhy{}1.,  1., \PYGZhy{}1., \PYGZhy{}1.,  1.],
       [\PYGZhy{}1.,  1., \PYGZhy{}1., \PYGZhy{}1.,  1.,  1.,  1.],
       [\PYGZhy{}1., \PYGZhy{}1., \PYGZhy{}1., \PYGZhy{}1., \PYGZhy{}1.,  1.,  1.]])
\end{sphinxVerbatim}

\end{sphinxuseclass}\end{sphinxVerbatimOutput}

\end{sphinxuseclass}
\begin{sphinxuseclass}{cell}\begin{sphinxVerbatimInput}

\begin{sphinxuseclass}{cell_input}
\begin{sphinxVerbatim}[commandchars=\\\{\}]
\PYG{n}{np}\PYG{o}{.}\PYG{n}{allclose}\PYG{p}{(}\PYG{n}{np}\PYG{o}{.}\PYG{n}{where}\PYG{p}{(}\PYG{n}{data} \PYG{o}{\PYGZlt{}} \PYG{l+m+mi}{0}\PYG{p}{,} \PYG{o}{\PYGZhy{}}\PYG{l+m+mi}{1}\PYG{p}{,} \PYG{n}{np}\PYG{o}{.}\PYG{n}{where}\PYG{p}{(}\PYG{n}{data} \PYG{o}{\PYGZgt{}} \PYG{l+m+mi}{0}\PYG{p}{,} \PYG{o}{+}\PYG{l+m+mi}{1}\PYG{p}{,} \PYG{n}{data}\PYG{p}{)}\PYG{p}{)}\PYG{p}{,} \PYG{n}{data\PYGZus{}c}\PYG{p}{)}
\end{sphinxVerbatim}

\end{sphinxuseclass}\end{sphinxVerbatimInput}
\begin{sphinxVerbatimOutput}

\begin{sphinxuseclass}{cell_output}
\begin{sphinxVerbatim}[commandchars=\\\{\}]
True
\end{sphinxVerbatim}

\end{sphinxuseclass}\end{sphinxVerbatimOutput}

\end{sphinxuseclass}
\begin{sphinxuseclass}{cell}\begin{sphinxVerbatimInput}

\begin{sphinxuseclass}{cell_input}
\begin{sphinxVerbatim}[commandchars=\\\{\}]
\PYG{n}{np}\PYG{o}{.}\PYG{n}{select}\PYG{p}{(}
    \PYG{n}{condlist}\PYG{o}{=}\PYG{p}{[}\PYG{n}{data}\PYG{o}{\PYGZlt{}}\PYG{l+m+mi}{0}\PYG{p}{,} \PYG{n}{data}\PYG{o}{\PYGZgt{}}\PYG{l+m+mi}{0}\PYG{p}{]}\PYG{p}{,}
    \PYG{n}{choicelist}\PYG{o}{=}\PYG{p}{[}\PYG{o}{\PYGZhy{}}\PYG{l+m+mi}{1}\PYG{p}{,} \PYG{o}{+}\PYG{l+m+mi}{1}\PYG{p}{]}\PYG{p}{,}
    \PYG{n}{default}\PYG{o}{=}\PYG{n}{data}
\PYG{p}{)}
\end{sphinxVerbatim}

\end{sphinxuseclass}\end{sphinxVerbatimInput}
\begin{sphinxVerbatimOutput}

\begin{sphinxuseclass}{cell_output}
\begin{sphinxVerbatim}[commandchars=\\\{\}]
array([[ 1., \PYGZhy{}1.,  1.,  1., \PYGZhy{}1., \PYGZhy{}1.,  1.],
       [ 1., \PYGZhy{}1.,  1., \PYGZhy{}1., \PYGZhy{}1.,  1., \PYGZhy{}1.],
       [\PYGZhy{}1., \PYGZhy{}1., \PYGZhy{}1.,  1., \PYGZhy{}1., \PYGZhy{}1.,  1.],
       [\PYGZhy{}1.,  1., \PYGZhy{}1., \PYGZhy{}1.,  1., \PYGZhy{}1.,  1.],
       [\PYGZhy{}1., \PYGZhy{}1., \PYGZhy{}1.,  1., \PYGZhy{}1., \PYGZhy{}1.,  1.],
       [\PYGZhy{}1.,  1., \PYGZhy{}1., \PYGZhy{}1.,  1.,  1.,  1.],
       [\PYGZhy{}1., \PYGZhy{}1., \PYGZhy{}1., \PYGZhy{}1., \PYGZhy{}1.,  1.,  1.]])
\end{sphinxVerbatim}

\end{sphinxuseclass}\end{sphinxVerbatimOutput}

\end{sphinxuseclass}

\subsubsection{Write a function \sphinxstyleliteralintitle{\sphinxupquote{npmts()}} that calculates the number of payments that generate \$x\%\$ of the present value of a perpetuity.}
\label{\detokenize{mckinney_04_practice_02:write-a-function-npmts-that-calculates-the-number-of-payments-that-generate-x-of-the-present-value-of-a-perpetuity}}
\sphinxAtStartPar
Your \sphinxcode{\sphinxupquote{npmts()}} should accept arguments \sphinxcode{\sphinxupquote{c1}}, \sphinxcode{\sphinxupquote{r}}, and \sphinxcode{\sphinxupquote{g}} that represent  \$C\_1\$, \$r\$, and \$g\$.
The present value of a growing perpetuity is \$PV = \textbackslash{}frac\{C\_1\}\{r \sphinxhyphen{} g\}\$, and the present value of a growing annuity is \$PV = \textbackslash{}frac\{C\_1\}\{r \sphinxhyphen{} g\}\textbackslash{}left{[} 1 \sphinxhyphen{} \textbackslash{}left( \textbackslash{}frac\{1 + g\}\{1 + r\} \textbackslash{}right)\textasciicircum{}t \textbackslash{}right{]}\$.

\sphinxAtStartPar
We can use the growing annuity and perpetuity formulas to show: \$x = \textbackslash{}left{[} 1 \sphinxhyphen{} \textbackslash{}left( \textbackslash{}frac\{1 + g\}\{1 + r\} \textbackslash{}right)\textasciicircum{}t \textbackslash{}right{]}\$.

\sphinxAtStartPar
Then: \$1 \sphinxhyphen{} x = \textbackslash{}left( \textbackslash{}frac\{1 + g\}\{1 + r\} \textbackslash{}right)\textasciicircum{}t\$.

\sphinxAtStartPar
Finally: \$t = \textbackslash{}frac\{\textbackslash{}log(1\sphinxhyphen{}x)\}\{\textbackslash{}log\textbackslash{}left(\textbackslash{}frac\{1 + g\}\{1 + r\}\textbackslash{}right)\}\$

\sphinxAtStartPar
\sphinxstyleemphasis{\sphinxstylestrong{We do not need to accept an argument \sphinxcode{\sphinxupquote{c1}} because \$C\_1\$ cancels out!}}

\begin{sphinxuseclass}{cell}\begin{sphinxVerbatimInput}

\begin{sphinxuseclass}{cell_input}
\begin{sphinxVerbatim}[commandchars=\\\{\}]
\PYG{k}{def} \PYG{n+nf}{npmts}\PYG{p}{(}\PYG{n}{x}\PYG{p}{,} \PYG{n}{r}\PYG{p}{,} \PYG{n}{g}\PYG{p}{)}\PYG{p}{:}
    \PYG{k}{return} \PYG{n}{np}\PYG{o}{.}\PYG{n}{log}\PYG{p}{(}\PYG{l+m+mi}{1}\PYG{o}{\PYGZhy{}}\PYG{n}{x}\PYG{p}{)} \PYG{o}{/} \PYG{n}{np}\PYG{o}{.}\PYG{n}{log}\PYG{p}{(}\PYG{p}{(}\PYG{l+m+mi}{1} \PYG{o}{+} \PYG{n}{g}\PYG{p}{)} \PYG{o}{/} \PYG{p}{(}\PYG{l+m+mi}{1} \PYG{o}{+} \PYG{n}{r}\PYG{p}{)}\PYG{p}{)}
\end{sphinxVerbatim}

\end{sphinxuseclass}\end{sphinxVerbatimInput}

\end{sphinxuseclass}
\begin{sphinxuseclass}{cell}\begin{sphinxVerbatimInput}

\begin{sphinxuseclass}{cell_input}
\begin{sphinxVerbatim}[commandchars=\\\{\}]
\PYG{n}{npmts}\PYG{p}{(}\PYG{l+m+mf}{0.5}\PYG{p}{,} \PYG{l+m+mf}{0.1}\PYG{p}{,} \PYG{l+m+mf}{0.05}\PYG{p}{)}
\end{sphinxVerbatim}

\end{sphinxuseclass}\end{sphinxVerbatimInput}
\begin{sphinxVerbatimOutput}

\begin{sphinxuseclass}{cell_output}
\begin{sphinxVerbatim}[commandchars=\\\{\}]
14.899977377480532
\end{sphinxVerbatim}

\end{sphinxuseclass}\end{sphinxVerbatimOutput}

\end{sphinxuseclass}

\subsubsection{Write a function that calculates the internal rate of return given a NumPy array of cash flows.}
\label{\detokenize{mckinney_04_practice_02:write-a-function-that-calculates-the-internal-rate-of-return-given-a-numpy-array-of-cash-flows}}
\sphinxAtStartPar
Here are some data where the \$IRR\$ is obvious!

\begin{sphinxuseclass}{cell}\begin{sphinxVerbatimInput}

\begin{sphinxuseclass}{cell_input}
\begin{sphinxVerbatim}[commandchars=\\\{\}]
\PYG{n}{c} \PYG{o}{=} \PYG{n}{np}\PYG{o}{.}\PYG{n}{array}\PYG{p}{(}\PYG{p}{[}\PYG{o}{\PYGZhy{}}\PYG{l+m+mi}{100}\PYG{p}{,} \PYG{l+m+mi}{110}\PYG{p}{]}\PYG{p}{)}
\PYG{n}{irr} \PYG{o}{=} \PYG{l+m+mf}{0.1}
\end{sphinxVerbatim}

\end{sphinxuseclass}\end{sphinxVerbatimInput}

\end{sphinxuseclass}
\sphinxAtStartPar
We want to replicate the following calculation with NumPy:

\begin{sphinxuseclass}{cell}\begin{sphinxVerbatimInput}

\begin{sphinxuseclass}{cell_input}
\begin{sphinxVerbatim}[commandchars=\\\{\}]
\PYG{n}{c}\PYG{p}{[}\PYG{l+m+mi}{0}\PYG{p}{]}\PYG{o}{/}\PYG{p}{(}\PYG{l+m+mi}{1}\PYG{o}{+}\PYG{n}{irr}\PYG{p}{)}\PYG{o}{*}\PYG{o}{*}\PYG{l+m+mi}{0} \PYG{o}{+} \PYG{n}{c}\PYG{p}{[}\PYG{l+m+mi}{1}\PYG{p}{]}\PYG{o}{/}\PYG{p}{(}\PYG{l+m+mi}{1}\PYG{o}{+}\PYG{n}{irr}\PYG{p}{)}\PYG{o}{*}\PYG{o}{*}\PYG{l+m+mi}{1}
\end{sphinxVerbatim}

\end{sphinxuseclass}\end{sphinxVerbatimInput}
\begin{sphinxVerbatimOutput}

\begin{sphinxuseclass}{cell_output}
\begin{sphinxVerbatim}[commandchars=\\\{\}]
\PYGZhy{}1.4210854715202004e\PYGZhy{}14
\end{sphinxVerbatim}

\end{sphinxuseclass}\end{sphinxVerbatimOutput}

\end{sphinxuseclass}
\sphinxAtStartPar
The following NumPy code calculates the present value interest factor for each cash flow:

\begin{sphinxuseclass}{cell}\begin{sphinxVerbatimInput}

\begin{sphinxuseclass}{cell_input}
\begin{sphinxVerbatim}[commandchars=\\\{\}]
\PYG{l+m+mi}{1} \PYG{o}{/} \PYG{p}{(}\PYG{l+m+mi}{1} \PYG{o}{+} \PYG{n}{irr}\PYG{p}{)}\PYG{o}{*}\PYG{o}{*}\PYG{n}{np}\PYG{o}{.}\PYG{n}{arange}\PYG{p}{(}\PYG{n+nb}{len}\PYG{p}{(}\PYG{n}{c}\PYG{p}{)}\PYG{p}{)} \PYG{c+c1}{\PYGZsh{} present value interest factor}
\end{sphinxVerbatim}

\end{sphinxuseclass}\end{sphinxVerbatimInput}
\begin{sphinxVerbatimOutput}

\begin{sphinxuseclass}{cell_output}
\begin{sphinxVerbatim}[commandchars=\\\{\}]
array([1.    , 0.9091])
\end{sphinxVerbatim}

\end{sphinxuseclass}\end{sphinxVerbatimOutput}

\end{sphinxuseclass}
\sphinxAtStartPar
The following NumPy code calculates the present value for each cash flow:

\begin{sphinxuseclass}{cell}\begin{sphinxVerbatimInput}

\begin{sphinxuseclass}{cell_input}
\begin{sphinxVerbatim}[commandchars=\\\{\}]
\PYG{n}{c} \PYG{o}{/} \PYG{p}{(}\PYG{l+m+mi}{1} \PYG{o}{+} \PYG{n}{irr}\PYG{p}{)}\PYG{o}{*}\PYG{o}{*}\PYG{n}{np}\PYG{o}{.}\PYG{n}{arange}\PYG{p}{(}\PYG{n+nb}{len}\PYG{p}{(}\PYG{n}{c}\PYG{p}{)}\PYG{p}{)} \PYG{c+c1}{\PYGZsh{} present value of each cash flow}
\end{sphinxVerbatim}

\end{sphinxuseclass}\end{sphinxVerbatimInput}
\begin{sphinxVerbatimOutput}

\begin{sphinxuseclass}{cell_output}
\begin{sphinxVerbatim}[commandchars=\\\{\}]
array([\PYGZhy{}100.,  100.])
\end{sphinxVerbatim}

\end{sphinxuseclass}\end{sphinxVerbatimOutput}

\end{sphinxuseclass}
\sphinxAtStartPar
We sum these present values of each cash flow to calculate the net present value:

\begin{sphinxuseclass}{cell}\begin{sphinxVerbatimInput}

\begin{sphinxuseclass}{cell_input}
\begin{sphinxVerbatim}[commandchars=\\\{\}]
\PYG{p}{(}\PYG{n}{c} \PYG{o}{/} \PYG{p}{(}\PYG{l+m+mi}{1} \PYG{o}{+} \PYG{n}{irr}\PYG{p}{)}\PYG{o}{*}\PYG{o}{*}\PYG{n}{np}\PYG{o}{.}\PYG{n}{arange}\PYG{p}{(}\PYG{n+nb}{len}\PYG{p}{(}\PYG{n}{c}\PYG{p}{)}\PYG{p}{)}\PYG{p}{)}\PYG{o}{.}\PYG{n}{sum}\PYG{p}{(}\PYG{p}{)}
\end{sphinxVerbatim}

\end{sphinxuseclass}\end{sphinxVerbatimInput}
\begin{sphinxVerbatimOutput}

\begin{sphinxuseclass}{cell_output}
\begin{sphinxVerbatim}[commandchars=\\\{\}]
\PYGZhy{}1.4210854715202004e\PYGZhy{}14
\end{sphinxVerbatim}

\end{sphinxuseclass}\end{sphinxVerbatimOutput}

\end{sphinxuseclass}
\sphinxAtStartPar
The \$IRR\$ is the discount rate where \$NPV=0\$.
We can can use the NumPy code above to try different discount rates until \$NPV=0\$.
The following code is crude, but sufficient for week three of our class and highlights two tools:
\begin{enumerate}
\sphinxsetlistlabels{\arabic}{enumi}{enumii}{}{.}%
\item {} 
\sphinxAtStartPar
Using a \sphinxcode{\sphinxupquote{while}} to try different discount rates until our answer is within some tolerance

\item {} 
\sphinxAtStartPar
Using NumPy to perform repetitive calculations without a \sphinxcode{\sphinxupquote{for}} loop

\end{enumerate}

\begin{sphinxuseclass}{cell}\begin{sphinxVerbatimInput}

\begin{sphinxuseclass}{cell_input}
\begin{sphinxVerbatim}[commandchars=\\\{\}]
\PYG{k}{def} \PYG{n+nf}{irr}\PYG{p}{(}\PYG{n}{c}\PYG{p}{,} \PYG{n}{guess}\PYG{o}{=}\PYG{l+m+mi}{0}\PYG{p}{,} \PYG{n}{tol}\PYG{o}{=}\PYG{l+m+mf}{0.0001}\PYG{p}{,} \PYG{n}{inc}\PYG{o}{=}\PYG{l+m+mf}{0.0001}\PYG{p}{)}\PYG{p}{:}
    \PYG{n}{npv} \PYG{o}{=} \PYG{l+m+mi}{42}
    \PYG{n}{irr} \PYG{o}{=} \PYG{n}{guess}
    \PYG{k}{while} \PYG{n}{np}\PYG{o}{.}\PYG{n}{abs}\PYG{p}{(}\PYG{n}{npv}\PYG{p}{)} \PYG{o}{\PYGZgt{}} \PYG{n}{tol}\PYG{p}{:}
        \PYG{n}{irr} \PYG{o}{+}\PYG{o}{=} \PYG{n}{inc}
        \PYG{n}{npv} \PYG{o}{=} \PYG{p}{(}\PYG{n}{c} \PYG{o}{/} \PYG{p}{(}\PYG{l+m+mi}{1} \PYG{o}{+} \PYG{n}{irr}\PYG{p}{)}\PYG{o}{*}\PYG{o}{*}\PYG{n}{np}\PYG{o}{.}\PYG{n}{arange}\PYG{p}{(}\PYG{n+nb}{len}\PYG{p}{(}\PYG{n}{c}\PYG{p}{)}\PYG{p}{)}\PYG{p}{)}\PYG{o}{.}\PYG{n}{sum}\PYG{p}{(}\PYG{p}{)}
        
    \PYG{k}{return} \PYG{n}{irr}
\end{sphinxVerbatim}

\end{sphinxuseclass}\end{sphinxVerbatimInput}

\end{sphinxuseclass}
\begin{sphinxuseclass}{cell}\begin{sphinxVerbatimInput}

\begin{sphinxuseclass}{cell_input}
\begin{sphinxVerbatim}[commandchars=\\\{\}]
\PYG{n}{c} \PYG{o}{=} \PYG{n}{np}\PYG{o}{.}\PYG{n}{array}\PYG{p}{(}\PYG{p}{[}\PYG{o}{\PYGZhy{}}\PYG{l+m+mi}{100}\PYG{p}{,} \PYG{l+m+mi}{110}\PYG{p}{]}\PYG{p}{)}
\end{sphinxVerbatim}

\end{sphinxuseclass}\end{sphinxVerbatimInput}

\end{sphinxuseclass}
\begin{sphinxuseclass}{cell}\begin{sphinxVerbatimInput}

\begin{sphinxuseclass}{cell_input}
\begin{sphinxVerbatim}[commandchars=\\\{\}]
\PYG{n}{irr}\PYG{p}{(}\PYG{n}{c}\PYG{p}{)}
\end{sphinxVerbatim}

\end{sphinxuseclass}\end{sphinxVerbatimInput}
\begin{sphinxVerbatimOutput}

\begin{sphinxuseclass}{cell_output}
\begin{sphinxVerbatim}[commandchars=\\\{\}]
0.1000
\end{sphinxVerbatim}

\end{sphinxuseclass}\end{sphinxVerbatimOutput}

\end{sphinxuseclass}

\subsubsection{Write a function \sphinxstyleliteralintitle{\sphinxupquote{returns()}} that accepts \sphinxstyleemphasis{NumPy arrays} of prices and dividends and returns a \sphinxstyleemphasis{NumPy array} of returns.}
\label{\detokenize{mckinney_04_practice_02:write-a-function-returns-that-accepts-numpy-arrays-of-prices-and-dividends-and-returns-a-numpy-array-of-returns}}
\begin{sphinxuseclass}{cell}\begin{sphinxVerbatimInput}

\begin{sphinxuseclass}{cell_input}
\begin{sphinxVerbatim}[commandchars=\\\{\}]
\PYG{n}{prices} \PYG{o}{=} \PYG{n}{np}\PYG{o}{.}\PYG{n}{array}\PYG{p}{(}\PYG{p}{[}\PYG{l+m+mi}{100}\PYG{p}{,} \PYG{l+m+mi}{150}\PYG{p}{,} \PYG{l+m+mi}{100}\PYG{p}{,} \PYG{l+m+mi}{50}\PYG{p}{,} \PYG{l+m+mi}{100}\PYG{p}{,} \PYG{l+m+mi}{150}\PYG{p}{,} \PYG{l+m+mi}{100}\PYG{p}{,} \PYG{l+m+mi}{150}\PYG{p}{]}\PYG{p}{)}
\PYG{n}{dividends} \PYG{o}{=} \PYG{n}{np}\PYG{o}{.}\PYG{n}{array}\PYG{p}{(}\PYG{p}{[}\PYG{l+m+mi}{1}\PYG{p}{,} \PYG{l+m+mi}{1}\PYG{p}{,} \PYG{l+m+mi}{1}\PYG{p}{,} \PYG{l+m+mi}{1}\PYG{p}{,} \PYG{l+m+mi}{2}\PYG{p}{,} \PYG{l+m+mi}{2}\PYG{p}{,} \PYG{l+m+mi}{2}\PYG{p}{,} \PYG{l+m+mi}{2}\PYG{p}{]}\PYG{p}{)}
\end{sphinxVerbatim}

\end{sphinxuseclass}\end{sphinxVerbatimInput}

\end{sphinxuseclass}
\begin{sphinxuseclass}{cell}\begin{sphinxVerbatimInput}

\begin{sphinxuseclass}{cell_input}
\begin{sphinxVerbatim}[commandchars=\\\{\}]
\PYG{k}{def} \PYG{n+nf}{returns}\PYG{p}{(}\PYG{n}{p}\PYG{p}{,} \PYG{n}{d}\PYG{p}{)}\PYG{p}{:}
    \PYG{k}{return} \PYG{p}{(}\PYG{n}{p}\PYG{p}{[}\PYG{l+m+mi}{1}\PYG{p}{:}\PYG{p}{]} \PYG{o}{\PYGZhy{}} \PYG{n}{p}\PYG{p}{[}\PYG{p}{:}\PYG{o}{\PYGZhy{}}\PYG{l+m+mi}{1}\PYG{p}{]} \PYG{o}{+} \PYG{n}{d}\PYG{p}{[}\PYG{l+m+mi}{1}\PYG{p}{:}\PYG{p}{]}\PYG{p}{)} \PYG{o}{/} \PYG{n}{p}\PYG{p}{[}\PYG{p}{:}\PYG{o}{\PYGZhy{}}\PYG{l+m+mi}{1}\PYG{p}{]}
\end{sphinxVerbatim}

\end{sphinxuseclass}\end{sphinxVerbatimInput}

\end{sphinxuseclass}
\begin{sphinxuseclass}{cell}\begin{sphinxVerbatimInput}

\begin{sphinxuseclass}{cell_input}
\begin{sphinxVerbatim}[commandchars=\\\{\}]
\PYG{n}{returns}\PYG{p}{(}\PYG{n}{p}\PYG{o}{=}\PYG{n}{prices}\PYG{p}{,} \PYG{n}{d}\PYG{o}{=}\PYG{n}{dividends}\PYG{p}{)}
\end{sphinxVerbatim}

\end{sphinxuseclass}\end{sphinxVerbatimInput}
\begin{sphinxVerbatimOutput}

\begin{sphinxuseclass}{cell_output}
\begin{sphinxVerbatim}[commandchars=\\\{\}]
array([ 0.51  , \PYGZhy{}0.3267, \PYGZhy{}0.49  ,  1.04  ,  0.52  , \PYGZhy{}0.32  ,  0.52  ])
\end{sphinxVerbatim}

\end{sphinxuseclass}\end{sphinxVerbatimOutput}

\end{sphinxuseclass}

\subsubsection{Rewrite the function \sphinxstyleliteralintitle{\sphinxupquote{returns()}} so it returns \sphinxstyleemphasis{NumPy arrays} of returns, capital gains yields, and dividend yields.}
\label{\detokenize{mckinney_04_practice_02:rewrite-the-function-returns-so-it-returns-numpy-arrays-of-returns-capital-gains-yields-and-dividend-yields}}
\begin{sphinxuseclass}{cell}\begin{sphinxVerbatimInput}

\begin{sphinxuseclass}{cell_input}
\begin{sphinxVerbatim}[commandchars=\\\{\}]
\PYG{k}{def} \PYG{n+nf}{returns}\PYG{p}{(}\PYG{n}{p}\PYG{p}{,} \PYG{n}{d}\PYG{p}{)}\PYG{p}{:}
    \PYG{n}{cg} \PYG{o}{=} \PYG{p}{(}\PYG{n}{p}\PYG{p}{[}\PYG{l+m+mi}{1}\PYG{p}{:}\PYG{p}{]} \PYG{o}{\PYGZhy{}} \PYG{n}{p}\PYG{p}{[}\PYG{p}{:}\PYG{o}{\PYGZhy{}}\PYG{l+m+mi}{1}\PYG{p}{]}\PYG{p}{)} \PYG{o}{/} \PYG{n}{p}\PYG{p}{[}\PYG{p}{:}\PYG{o}{\PYGZhy{}}\PYG{l+m+mi}{1}\PYG{p}{]}
    \PYG{n}{dy} \PYG{o}{=} \PYG{n}{d}\PYG{p}{[}\PYG{l+m+mi}{1}\PYG{p}{:}\PYG{p}{]} \PYG{o}{/} \PYG{n}{p}\PYG{p}{[}\PYG{p}{:}\PYG{o}{\PYGZhy{}}\PYG{l+m+mi}{1}\PYG{p}{]}
    \PYG{n}{r} \PYG{o}{=} \PYG{n}{cg} \PYG{o}{+} \PYG{n}{dy}
    \PYG{k}{return} \PYG{p}{\PYGZob{}}\PYG{l+s+s1}{\PYGZsq{}}\PYG{l+s+s1}{r}\PYG{l+s+s1}{\PYGZsq{}}\PYG{p}{:} \PYG{n}{r}\PYG{p}{,} \PYG{l+s+s1}{\PYGZsq{}}\PYG{l+s+s1}{cg}\PYG{l+s+s1}{\PYGZsq{}}\PYG{p}{:} \PYG{n}{cg}\PYG{p}{,} \PYG{l+s+s1}{\PYGZsq{}}\PYG{l+s+s1}{dy}\PYG{l+s+s1}{\PYGZsq{}}\PYG{p}{:} \PYG{n}{dy}\PYG{p}{\PYGZcb{}}
\end{sphinxVerbatim}

\end{sphinxuseclass}\end{sphinxVerbatimInput}

\end{sphinxuseclass}
\begin{sphinxuseclass}{cell}\begin{sphinxVerbatimInput}

\begin{sphinxuseclass}{cell_input}
\begin{sphinxVerbatim}[commandchars=\\\{\}]
\PYG{n}{returns}\PYG{p}{(}\PYG{n}{p}\PYG{o}{=}\PYG{n}{prices}\PYG{p}{,} \PYG{n}{d}\PYG{o}{=}\PYG{n}{dividends}\PYG{p}{)}
\end{sphinxVerbatim}

\end{sphinxuseclass}\end{sphinxVerbatimInput}
\begin{sphinxVerbatimOutput}

\begin{sphinxuseclass}{cell_output}
\begin{sphinxVerbatim}[commandchars=\\\{\}]
\PYGZob{}\PYGZsq{}r\PYGZsq{}: array([ 0.51  , \PYGZhy{}0.3267, \PYGZhy{}0.49  ,  1.04  ,  0.52  , \PYGZhy{}0.32  ,  0.52  ]),
 \PYGZsq{}cg\PYGZsq{}: array([ 0.5   , \PYGZhy{}0.3333, \PYGZhy{}0.5   ,  1.    ,  0.5   , \PYGZhy{}0.3333,  0.5   ]),
 \PYGZsq{}dy\PYGZsq{}: array([0.01  , 0.0067, 0.01  , 0.04  , 0.02  , 0.0133, 0.02  ])\PYGZcb{}
\end{sphinxVerbatim}

\end{sphinxuseclass}\end{sphinxVerbatimOutput}

\end{sphinxuseclass}

\subsubsection{Rescale and shift numbers so that they cover the range {[}0, 1{]}}
\label{\detokenize{mckinney_04_practice_02:rescale-and-shift-numbers-so-that-they-cover-the-range-0-1}}
\sphinxAtStartPar
Input: \sphinxcode{\sphinxupquote{np.array({[}18.5, 17.0, 18.0, 19.0, 18.0{]})}} \\
Output: \sphinxcode{\sphinxupquote{np.array({[}0.75, 0.0, 0.5, 1.0, 0.5{]})}}

\begin{sphinxuseclass}{cell}\begin{sphinxVerbatimInput}

\begin{sphinxuseclass}{cell_input}
\begin{sphinxVerbatim}[commandchars=\\\{\}]
\PYG{n}{numbers} \PYG{o}{=} \PYG{n}{np}\PYG{o}{.}\PYG{n}{array}\PYG{p}{(}\PYG{p}{[}\PYG{l+m+mf}{18.5}\PYG{p}{,} \PYG{l+m+mf}{17.0}\PYG{p}{,} \PYG{l+m+mf}{18.0}\PYG{p}{,} \PYG{l+m+mf}{19.0}\PYG{p}{,} \PYG{l+m+mf}{18.0}\PYG{p}{]}\PYG{p}{)}
\end{sphinxVerbatim}

\end{sphinxuseclass}\end{sphinxVerbatimInput}

\end{sphinxuseclass}
\begin{sphinxuseclass}{cell}\begin{sphinxVerbatimInput}

\begin{sphinxuseclass}{cell_input}
\begin{sphinxVerbatim}[commandchars=\\\{\}]
\PYG{p}{(}\PYG{n}{numbers} \PYG{o}{\PYGZhy{}} \PYG{n}{numbers}\PYG{o}{.}\PYG{n}{min}\PYG{p}{(}\PYG{p}{)}\PYG{p}{)} \PYG{o}{/} \PYG{p}{(}\PYG{n}{numbers}\PYG{o}{.}\PYG{n}{max}\PYG{p}{(}\PYG{p}{)} \PYG{o}{\PYGZhy{}} \PYG{n}{numbers}\PYG{o}{.}\PYG{n}{min}\PYG{p}{(}\PYG{p}{)}\PYG{p}{)}
\end{sphinxVerbatim}

\end{sphinxuseclass}\end{sphinxVerbatimInput}
\begin{sphinxVerbatimOutput}

\begin{sphinxuseclass}{cell_output}
\begin{sphinxVerbatim}[commandchars=\\\{\}]
array([0.75, 0.  , 0.5 , 1.  , 0.5 ])
\end{sphinxVerbatim}

\end{sphinxuseclass}\end{sphinxVerbatimOutput}

\end{sphinxuseclass}

\subsubsection{Write functions \sphinxstyleliteralintitle{\sphinxupquote{var()}} and \sphinxstyleliteralintitle{\sphinxupquote{std()}} that calculate variance and standard deviation.}
\label{\detokenize{mckinney_04_practice_02:write-functions-var-and-std-that-calculate-variance-and-standard-deviation}}
\sphinxAtStartPar
NumPy’s \sphinxcode{\sphinxupquote{.var()}} and \sphinxcode{\sphinxupquote{.std()}} methods return \sphinxstyleemphasis{population} statistics (i.e., denominators of \$n\$).
The pandas equivalents return \sphinxstyleemphasis{sample} statistics (denominators of \$n\sphinxhyphen{}1\$), which are more appropriate for financial data analysis where we have a sample instead of a population.

\sphinxAtStartPar
Both function should have an argument \sphinxcode{\sphinxupquote{sample}} that is \sphinxcode{\sphinxupquote{True}} by default so both functions return sample statistics by default.

\sphinxAtStartPar
Use the output of \sphinxcode{\sphinxupquote{returns()}} to compare your functions with NumPy’s \sphinxcode{\sphinxupquote{.var()}} and \sphinxcode{\sphinxupquote{.std()}} methods.

\begin{sphinxuseclass}{cell}\begin{sphinxVerbatimInput}

\begin{sphinxuseclass}{cell_input}
\begin{sphinxVerbatim}[commandchars=\\\{\}]
\PYG{n}{np}\PYG{o}{.}\PYG{n}{random}\PYG{o}{.}\PYG{n}{seed}\PYG{p}{(}\PYG{l+m+mi}{42}\PYG{p}{)}
\PYG{n}{r} \PYG{o}{=} \PYG{n}{np}\PYG{o}{.}\PYG{n}{random}\PYG{o}{.}\PYG{n}{randn}\PYG{p}{(}\PYG{l+m+mi}{1\PYGZus{}000\PYGZus{}000}\PYG{p}{)}
\PYG{n}{r}
\end{sphinxVerbatim}

\end{sphinxuseclass}\end{sphinxVerbatimInput}
\begin{sphinxVerbatimOutput}

\begin{sphinxuseclass}{cell_output}
\begin{sphinxVerbatim}[commandchars=\\\{\}]
array([ 0.4967, \PYGZhy{}0.1383,  0.6477, ..., \PYGZhy{}0.113 ,  1.4691,  0.4764])
\end{sphinxVerbatim}

\end{sphinxuseclass}\end{sphinxVerbatimOutput}

\end{sphinxuseclass}
\begin{sphinxuseclass}{cell}\begin{sphinxVerbatimInput}

\begin{sphinxuseclass}{cell_input}
\begin{sphinxVerbatim}[commandchars=\\\{\}]
\PYG{k}{def} \PYG{n+nf}{var}\PYG{p}{(}\PYG{n}{x}\PYG{p}{,} \PYG{n}{ddof}\PYG{o}{=}\PYG{l+m+mi}{0}\PYG{p}{)}\PYG{p}{:}
    \PYG{n}{N} \PYG{o}{=} \PYG{n+nb}{len}\PYG{p}{(}\PYG{n}{x}\PYG{p}{)}
    \PYG{k}{return} \PYG{p}{(}\PYG{p}{(}\PYG{n}{x} \PYG{o}{\PYGZhy{}} \PYG{n}{x}\PYG{o}{.}\PYG{n}{mean}\PYG{p}{(}\PYG{p}{)}\PYG{p}{)}\PYG{o}{*}\PYG{o}{*}\PYG{l+m+mi}{2}\PYG{p}{)}\PYG{o}{.}\PYG{n}{sum}\PYG{p}{(}\PYG{p}{)} \PYG{o}{/} \PYG{p}{(}\PYG{n}{N} \PYG{o}{\PYGZhy{}} \PYG{n}{ddof}\PYG{p}{)}
\end{sphinxVerbatim}

\end{sphinxuseclass}\end{sphinxVerbatimInput}

\end{sphinxuseclass}
\begin{sphinxuseclass}{cell}\begin{sphinxVerbatimInput}

\begin{sphinxuseclass}{cell_input}
\begin{sphinxVerbatim}[commandchars=\\\{\}]
\PYG{n}{r}\PYG{o}{.}\PYG{n}{var}\PYG{p}{(}\PYG{p}{)}
\end{sphinxVerbatim}

\end{sphinxuseclass}\end{sphinxVerbatimInput}
\begin{sphinxVerbatimOutput}

\begin{sphinxuseclass}{cell_output}
\begin{sphinxVerbatim}[commandchars=\\\{\}]
1.0003761089673018
\end{sphinxVerbatim}

\end{sphinxuseclass}\end{sphinxVerbatimOutput}

\end{sphinxuseclass}
\begin{sphinxuseclass}{cell}\begin{sphinxVerbatimInput}

\begin{sphinxuseclass}{cell_input}
\begin{sphinxVerbatim}[commandchars=\\\{\}]
\PYG{n}{var}\PYG{p}{(}\PYG{n}{r}\PYG{p}{)}
\end{sphinxVerbatim}

\end{sphinxuseclass}\end{sphinxVerbatimInput}
\begin{sphinxVerbatimOutput}

\begin{sphinxuseclass}{cell_output}
\begin{sphinxVerbatim}[commandchars=\\\{\}]
1.0003761089673018
\end{sphinxVerbatim}

\end{sphinxuseclass}\end{sphinxVerbatimOutput}

\end{sphinxuseclass}
\begin{sphinxuseclass}{cell}\begin{sphinxVerbatimInput}

\begin{sphinxuseclass}{cell_input}
\begin{sphinxVerbatim}[commandchars=\\\{\}]
\PYG{n}{np}\PYG{o}{.}\PYG{n}{allclose}\PYG{p}{(}\PYG{n}{r}\PYG{o}{.}\PYG{n}{var}\PYG{p}{(}\PYG{p}{)}\PYG{p}{,} \PYG{n}{var}\PYG{p}{(}\PYG{n}{r}\PYG{p}{)}\PYG{p}{)}
\end{sphinxVerbatim}

\end{sphinxuseclass}\end{sphinxVerbatimInput}
\begin{sphinxVerbatimOutput}

\begin{sphinxuseclass}{cell_output}
\begin{sphinxVerbatim}[commandchars=\\\{\}]
True
\end{sphinxVerbatim}

\end{sphinxuseclass}\end{sphinxVerbatimOutput}

\end{sphinxuseclass}
\begin{sphinxuseclass}{cell}\begin{sphinxVerbatimInput}

\begin{sphinxuseclass}{cell_input}
\begin{sphinxVerbatim}[commandchars=\\\{\}]
\PYG{n}{np}\PYG{o}{.}\PYG{n}{allclose}\PYG{p}{(}\PYG{n}{r}\PYG{o}{.}\PYG{n}{var}\PYG{p}{(}\PYG{n}{ddof}\PYG{o}{=}\PYG{l+m+mi}{1}\PYG{p}{)}\PYG{p}{,} \PYG{n}{var}\PYG{p}{(}\PYG{n}{r}\PYG{p}{,} \PYG{n}{ddof}\PYG{o}{=}\PYG{l+m+mi}{1}\PYG{p}{)}\PYG{p}{)}
\end{sphinxVerbatim}

\end{sphinxuseclass}\end{sphinxVerbatimInput}
\begin{sphinxVerbatimOutput}

\begin{sphinxuseclass}{cell_output}
\begin{sphinxVerbatim}[commandchars=\\\{\}]
True
\end{sphinxVerbatim}

\end{sphinxuseclass}\end{sphinxVerbatimOutput}

\end{sphinxuseclass}
\begin{sphinxuseclass}{cell}\begin{sphinxVerbatimInput}

\begin{sphinxuseclass}{cell_input}
\begin{sphinxVerbatim}[commandchars=\\\{\}]
\PYG{k}{def} \PYG{n+nf}{std}\PYG{p}{(}\PYG{n}{x}\PYG{p}{,} \PYG{n}{ddof}\PYG{o}{=}\PYG{l+m+mi}{0}\PYG{p}{)}\PYG{p}{:}
    \PYG{k}{return} \PYG{n}{np}\PYG{o}{.}\PYG{n}{sqrt}\PYG{p}{(}\PYG{n}{var}\PYG{p}{(}\PYG{n}{x}\PYG{o}{=}\PYG{n}{x}\PYG{p}{,} \PYG{n}{ddof}\PYG{o}{=}\PYG{n}{ddof}\PYG{p}{)}\PYG{p}{)}
\end{sphinxVerbatim}

\end{sphinxuseclass}\end{sphinxVerbatimInput}

\end{sphinxuseclass}
\begin{sphinxuseclass}{cell}\begin{sphinxVerbatimInput}

\begin{sphinxuseclass}{cell_input}
\begin{sphinxVerbatim}[commandchars=\\\{\}]
\PYG{n}{r}\PYG{o}{.}\PYG{n}{std}\PYG{p}{(}\PYG{p}{)}
\end{sphinxVerbatim}

\end{sphinxuseclass}\end{sphinxVerbatimInput}
\begin{sphinxVerbatimOutput}

\begin{sphinxuseclass}{cell_output}
\begin{sphinxVerbatim}[commandchars=\\\{\}]
1.000188036804731
\end{sphinxVerbatim}

\end{sphinxuseclass}\end{sphinxVerbatimOutput}

\end{sphinxuseclass}
\begin{sphinxuseclass}{cell}\begin{sphinxVerbatimInput}

\begin{sphinxuseclass}{cell_input}
\begin{sphinxVerbatim}[commandchars=\\\{\}]
\PYG{n}{std}\PYG{p}{(}\PYG{n}{r}\PYG{p}{)}
\end{sphinxVerbatim}

\end{sphinxuseclass}\end{sphinxVerbatimInput}
\begin{sphinxVerbatimOutput}

\begin{sphinxuseclass}{cell_output}
\begin{sphinxVerbatim}[commandchars=\\\{\}]
1.000188036804731
\end{sphinxVerbatim}

\end{sphinxuseclass}\end{sphinxVerbatimOutput}

\end{sphinxuseclass}
\begin{sphinxuseclass}{cell}\begin{sphinxVerbatimInput}

\begin{sphinxuseclass}{cell_input}
\begin{sphinxVerbatim}[commandchars=\\\{\}]
\PYG{n}{np}\PYG{o}{.}\PYG{n}{allclose}\PYG{p}{(}\PYG{n}{r}\PYG{o}{.}\PYG{n}{std}\PYG{p}{(}\PYG{p}{)}\PYG{p}{,} \PYG{n}{std}\PYG{p}{(}\PYG{n}{r}\PYG{p}{)}\PYG{p}{)}
\end{sphinxVerbatim}

\end{sphinxuseclass}\end{sphinxVerbatimInput}
\begin{sphinxVerbatimOutput}

\begin{sphinxuseclass}{cell_output}
\begin{sphinxVerbatim}[commandchars=\\\{\}]
True
\end{sphinxVerbatim}

\end{sphinxuseclass}\end{sphinxVerbatimOutput}

\end{sphinxuseclass}
\begin{sphinxuseclass}{cell}\begin{sphinxVerbatimInput}

\begin{sphinxuseclass}{cell_input}
\begin{sphinxVerbatim}[commandchars=\\\{\}]
\PYG{n}{np}\PYG{o}{.}\PYG{n}{allclose}\PYG{p}{(}\PYG{n}{r}\PYG{o}{.}\PYG{n}{std}\PYG{p}{(}\PYG{n}{ddof}\PYG{o}{=}\PYG{l+m+mi}{1}\PYG{p}{)}\PYG{p}{,} \PYG{n}{std}\PYG{p}{(}\PYG{n}{r}\PYG{p}{,} \PYG{n}{ddof}\PYG{o}{=}\PYG{l+m+mi}{1}\PYG{p}{)}\PYG{p}{)}
\end{sphinxVerbatim}

\end{sphinxuseclass}\end{sphinxVerbatimInput}
\begin{sphinxVerbatimOutput}

\begin{sphinxuseclass}{cell_output}
\begin{sphinxVerbatim}[commandchars=\\\{\}]
True
\end{sphinxVerbatim}

\end{sphinxuseclass}\end{sphinxVerbatimOutput}

\end{sphinxuseclass}
\sphinxstepscope


\chapter{McKinney Chapter 5 \sphinxhyphen{} Getting Started with pandas}
\label{\detokenize{mckinney_05_lecture:mckinney-chapter-5-getting-started-with-pandas}}\label{\detokenize{mckinney_05_lecture::doc}}

\section{Introduction}
\label{\detokenize{mckinney_05_lecture:introduction}}
\sphinxAtStartPar
Chapter 5 of Wes McKinney’s \sphinxhref{https://wesmckinney.com/book/}{\sphinxstyleemphasis{Python for Data Analysis}} discusses the fundamentals of pandas, which will be our main tool for the rest of the semester.
pandas is an abbrviation for \sphinxstyleemphasis{pan}el \sphinxstyleemphasis{da}ta, which provide time\sphinxhyphen{}stamped data for multiple individuals or firms.

\sphinxAtStartPar
\sphinxstyleemphasis{\sphinxstylestrong{Note:}}
Indented block quotes are from McKinney unless otherwise indicated.
The section numbers here differ from McKinney because we will only discuss some topics.

\begin{sphinxuseclass}{cell}\begin{sphinxVerbatimInput}

\begin{sphinxuseclass}{cell_input}
\begin{sphinxVerbatim}[commandchars=\\\{\}]
\PYG{k+kn}{import} \PYG{n+nn}{matplotlib}\PYG{n+nn}{.}\PYG{n+nn}{pyplot} \PYG{k}{as} \PYG{n+nn}{plt}
\PYG{k+kn}{import} \PYG{n+nn}{numpy} \PYG{k}{as} \PYG{n+nn}{np}
\PYG{k+kn}{import} \PYG{n+nn}{pandas} \PYG{k}{as} \PYG{n+nn}{pd}
\end{sphinxVerbatim}

\end{sphinxuseclass}\end{sphinxVerbatimInput}

\end{sphinxuseclass}
\begin{sphinxuseclass}{cell}\begin{sphinxVerbatimInput}

\begin{sphinxuseclass}{cell_input}
\begin{sphinxVerbatim}[commandchars=\\\{\}]
\PYG{o}{\PYGZpc{}}\PYG{k}{config} InlineBackend.figure\PYGZus{}format = \PYGZsq{}retina\PYGZsq{}
\PYG{o}{\PYGZpc{}}\PYG{k}{precision} 4
\PYG{n}{pd}\PYG{o}{.}\PYG{n}{options}\PYG{o}{.}\PYG{n}{display}\PYG{o}{.}\PYG{n}{float\PYGZus{}format} \PYG{o}{=} \PYG{l+s+s1}{\PYGZsq{}}\PYG{l+s+si}{\PYGZob{}:.4f\PYGZcb{}}\PYG{l+s+s1}{\PYGZsq{}}\PYG{o}{.}\PYG{n}{format}
\end{sphinxVerbatim}

\end{sphinxuseclass}\end{sphinxVerbatimInput}

\end{sphinxuseclass}\begin{quote}

\sphinxAtStartPar
pandas will be a major tool of interest throughout much of the rest of the book. It contains data structures and data manipulation tools designed to make data cleaning and analysis fast and easy in Python. pandas is often used in tandem with numerical computing tools like NumPy and SciPy, analytical libraries like statsmodels and scikit\sphinxhyphen{}learn, and data visualization libraries like matplotlib. pandas adopts significant parts of NumPy’s idiomatic style of array\sphinxhyphen{}based computing, especially array\sphinxhyphen{}based functions and a preference for data processing without for loops.

\sphinxAtStartPar
While pandas adopts many coding idioms from NumPy, the biggest difference is that pandas is designed for working with tabular or heterogeneous data. NumPy, by contrast, is best suited for working with homogeneous numerical array data.
\end{quote}

\sphinxAtStartPar
We will use pandas—a wrapper for NumPy that helps us manipulate and combine data—every day for the rest of the course.


\section{Introduction to pandas Data Structures}
\label{\detokenize{mckinney_05_lecture:introduction-to-pandas-data-structures}}\begin{quote}

\sphinxAtStartPar
To get started with pandas, you will need to get comfortable with its two workhorse data structures: Series and DataFrame. While they are not a universal solution for every problem, they provide a solid, easy\sphinxhyphen{}to\sphinxhyphen{}use basis for most applications.
\end{quote}


\subsection{Series}
\label{\detokenize{mckinney_05_lecture:series}}\begin{quote}

\sphinxAtStartPar
A Series is a one\sphinxhyphen{}dimensional array\sphinxhyphen{}like object containing a sequence of values (of similar types to NumPy types) and an associated array of data labels, called its index. The simplest Series is formed from only an array of data.
\end{quote}

\sphinxAtStartPar
The early examples use integer and string labels, but date\sphinxhyphen{}time labels are most useful.

\begin{sphinxuseclass}{cell}\begin{sphinxVerbatimInput}

\begin{sphinxuseclass}{cell_input}
\begin{sphinxVerbatim}[commandchars=\\\{\}]
\PYG{n}{obj} \PYG{o}{=} \PYG{n}{pd}\PYG{o}{.}\PYG{n}{Series}\PYG{p}{(}\PYG{p}{[}\PYG{l+m+mi}{4}\PYG{p}{,} \PYG{l+m+mi}{7}\PYG{p}{,} \PYG{o}{\PYGZhy{}}\PYG{l+m+mi}{5}\PYG{p}{,} \PYG{l+m+mi}{3}\PYG{p}{]}\PYG{p}{)}
\PYG{n}{obj}
\end{sphinxVerbatim}

\end{sphinxuseclass}\end{sphinxVerbatimInput}
\begin{sphinxVerbatimOutput}

\begin{sphinxuseclass}{cell_output}
\begin{sphinxVerbatim}[commandchars=\\\{\}]
0    4
1    7
2   \PYGZhy{}5
3    3
dtype: int64
\end{sphinxVerbatim}

\end{sphinxuseclass}\end{sphinxVerbatimOutput}

\end{sphinxuseclass}
\sphinxAtStartPar
Contrast \sphinxcode{\sphinxupquote{obj}} with a NumPy array equivalent:

\begin{sphinxuseclass}{cell}\begin{sphinxVerbatimInput}

\begin{sphinxuseclass}{cell_input}
\begin{sphinxVerbatim}[commandchars=\\\{\}]
\PYG{n}{np}\PYG{o}{.}\PYG{n}{array}\PYG{p}{(}\PYG{p}{[}\PYG{l+m+mi}{4}\PYG{p}{,} \PYG{l+m+mi}{7}\PYG{p}{,} \PYG{o}{\PYGZhy{}}\PYG{l+m+mi}{5}\PYG{p}{,} \PYG{l+m+mi}{3}\PYG{p}{]}\PYG{p}{)}
\end{sphinxVerbatim}

\end{sphinxuseclass}\end{sphinxVerbatimInput}
\begin{sphinxVerbatimOutput}

\begin{sphinxuseclass}{cell_output}
\begin{sphinxVerbatim}[commandchars=\\\{\}]
array([ 4,  7, \PYGZhy{}5,  3])
\end{sphinxVerbatim}

\end{sphinxuseclass}\end{sphinxVerbatimOutput}

\end{sphinxuseclass}
\begin{sphinxuseclass}{cell}\begin{sphinxVerbatimInput}

\begin{sphinxuseclass}{cell_input}
\begin{sphinxVerbatim}[commandchars=\\\{\}]
\PYG{n}{obj}\PYG{o}{.}\PYG{n}{values}
\end{sphinxVerbatim}

\end{sphinxuseclass}\end{sphinxVerbatimInput}
\begin{sphinxVerbatimOutput}

\begin{sphinxuseclass}{cell_output}
\begin{sphinxVerbatim}[commandchars=\\\{\}]
array([ 4,  7, \PYGZhy{}5,  3])
\end{sphinxVerbatim}

\end{sphinxuseclass}\end{sphinxVerbatimOutput}

\end{sphinxuseclass}
\begin{sphinxuseclass}{cell}\begin{sphinxVerbatimInput}

\begin{sphinxuseclass}{cell_input}
\begin{sphinxVerbatim}[commandchars=\\\{\}]
\PYG{n}{obj}\PYG{o}{.}\PYG{n}{index}  \PYG{c+c1}{\PYGZsh{} similar to range(4)}
\end{sphinxVerbatim}

\end{sphinxuseclass}\end{sphinxVerbatimInput}
\begin{sphinxVerbatimOutput}

\begin{sphinxuseclass}{cell_output}
\begin{sphinxVerbatim}[commandchars=\\\{\}]
RangeIndex(start=0, stop=4, step=1)
\end{sphinxVerbatim}

\end{sphinxuseclass}\end{sphinxVerbatimOutput}

\end{sphinxuseclass}
\sphinxAtStartPar
We did not explicitly assign an index to \sphinxcode{\sphinxupquote{obj}}, so \sphinxcode{\sphinxupquote{obj}} has an integer index that starts at 0.
We can explicitly assign an index with the \sphinxcode{\sphinxupquote{index=}} argument.

\begin{sphinxuseclass}{cell}\begin{sphinxVerbatimInput}

\begin{sphinxuseclass}{cell_input}
\begin{sphinxVerbatim}[commandchars=\\\{\}]
\PYG{n}{obj2} \PYG{o}{=} \PYG{n}{pd}\PYG{o}{.}\PYG{n}{Series}\PYG{p}{(}\PYG{p}{[}\PYG{l+m+mi}{4}\PYG{p}{,} \PYG{l+m+mi}{7}\PYG{p}{,} \PYG{o}{\PYGZhy{}}\PYG{l+m+mi}{5}\PYG{p}{,} \PYG{l+m+mi}{3}\PYG{p}{]}\PYG{p}{,} \PYG{n}{index}\PYG{o}{=}\PYG{p}{[}\PYG{l+s+s1}{\PYGZsq{}}\PYG{l+s+s1}{d}\PYG{l+s+s1}{\PYGZsq{}}\PYG{p}{,} \PYG{l+s+s1}{\PYGZsq{}}\PYG{l+s+s1}{b}\PYG{l+s+s1}{\PYGZsq{}}\PYG{p}{,} \PYG{l+s+s1}{\PYGZsq{}}\PYG{l+s+s1}{a}\PYG{l+s+s1}{\PYGZsq{}}\PYG{p}{,} \PYG{l+s+s1}{\PYGZsq{}}\PYG{l+s+s1}{c}\PYG{l+s+s1}{\PYGZsq{}}\PYG{p}{]}\PYG{p}{)}
\PYG{n}{obj2}
\end{sphinxVerbatim}

\end{sphinxuseclass}\end{sphinxVerbatimInput}
\begin{sphinxVerbatimOutput}

\begin{sphinxuseclass}{cell_output}
\begin{sphinxVerbatim}[commandchars=\\\{\}]
d    4
b    7
a   \PYGZhy{}5
c    3
dtype: int64
\end{sphinxVerbatim}

\end{sphinxuseclass}\end{sphinxVerbatimOutput}

\end{sphinxuseclass}
\begin{sphinxuseclass}{cell}\begin{sphinxVerbatimInput}

\begin{sphinxuseclass}{cell_input}
\begin{sphinxVerbatim}[commandchars=\\\{\}]
\PYG{n}{obj2}\PYG{o}{.}\PYG{n}{index}
\end{sphinxVerbatim}

\end{sphinxuseclass}\end{sphinxVerbatimInput}
\begin{sphinxVerbatimOutput}

\begin{sphinxuseclass}{cell_output}
\begin{sphinxVerbatim}[commandchars=\\\{\}]
Index([\PYGZsq{}d\PYGZsq{}, \PYGZsq{}b\PYGZsq{}, \PYGZsq{}a\PYGZsq{}, \PYGZsq{}c\PYGZsq{}], dtype=\PYGZsq{}object\PYGZsq{})
\end{sphinxVerbatim}

\end{sphinxuseclass}\end{sphinxVerbatimOutput}

\end{sphinxuseclass}
\begin{sphinxuseclass}{cell}\begin{sphinxVerbatimInput}

\begin{sphinxuseclass}{cell_input}
\begin{sphinxVerbatim}[commandchars=\\\{\}]
\PYG{n}{obj2}\PYG{p}{[}\PYG{l+s+s1}{\PYGZsq{}}\PYG{l+s+s1}{a}\PYG{l+s+s1}{\PYGZsq{}}\PYG{p}{]}
\end{sphinxVerbatim}

\end{sphinxuseclass}\end{sphinxVerbatimInput}
\begin{sphinxVerbatimOutput}

\begin{sphinxuseclass}{cell_output}
\begin{sphinxVerbatim}[commandchars=\\\{\}]
\PYGZhy{}5
\end{sphinxVerbatim}

\end{sphinxuseclass}\end{sphinxVerbatimOutput}

\end{sphinxuseclass}
\begin{sphinxuseclass}{cell}\begin{sphinxVerbatimInput}

\begin{sphinxuseclass}{cell_input}
\begin{sphinxVerbatim}[commandchars=\\\{\}]
\PYG{n}{obj2}\PYG{p}{[}\PYG{l+m+mi}{2}\PYG{p}{]}
\end{sphinxVerbatim}

\end{sphinxuseclass}\end{sphinxVerbatimInput}
\begin{sphinxVerbatimOutput}

\begin{sphinxuseclass}{cell_output}
\begin{sphinxVerbatim}[commandchars=\\\{\}]
\PYGZhy{}5
\end{sphinxVerbatim}

\end{sphinxuseclass}\end{sphinxVerbatimOutput}

\end{sphinxuseclass}
\begin{sphinxuseclass}{cell}\begin{sphinxVerbatimInput}

\begin{sphinxuseclass}{cell_input}
\begin{sphinxVerbatim}[commandchars=\\\{\}]
\PYG{n}{obj2}\PYG{p}{[}\PYG{l+s+s1}{\PYGZsq{}}\PYG{l+s+s1}{d}\PYG{l+s+s1}{\PYGZsq{}}\PYG{p}{]} \PYG{o}{=} \PYG{l+m+mi}{6}
\PYG{n}{obj2}
\end{sphinxVerbatim}

\end{sphinxuseclass}\end{sphinxVerbatimInput}
\begin{sphinxVerbatimOutput}

\begin{sphinxuseclass}{cell_output}
\begin{sphinxVerbatim}[commandchars=\\\{\}]
d    6
b    7
a   \PYGZhy{}5
c    3
dtype: int64
\end{sphinxVerbatim}

\end{sphinxuseclass}\end{sphinxVerbatimOutput}

\end{sphinxuseclass}
\begin{sphinxuseclass}{cell}\begin{sphinxVerbatimInput}

\begin{sphinxuseclass}{cell_input}
\begin{sphinxVerbatim}[commandchars=\\\{\}]
\PYG{n}{obj2}\PYG{p}{[}\PYG{p}{[}\PYG{l+s+s1}{\PYGZsq{}}\PYG{l+s+s1}{c}\PYG{l+s+s1}{\PYGZsq{}}\PYG{p}{,} \PYG{l+s+s1}{\PYGZsq{}}\PYG{l+s+s1}{a}\PYG{l+s+s1}{\PYGZsq{}}\PYG{p}{,} \PYG{l+s+s1}{\PYGZsq{}}\PYG{l+s+s1}{d}\PYG{l+s+s1}{\PYGZsq{}}\PYG{p}{]}\PYG{p}{]}
\end{sphinxVerbatim}

\end{sphinxuseclass}\end{sphinxVerbatimInput}
\begin{sphinxVerbatimOutput}

\begin{sphinxuseclass}{cell_output}
\begin{sphinxVerbatim}[commandchars=\\\{\}]
c    3
a   \PYGZhy{}5
d    6
dtype: int64
\end{sphinxVerbatim}

\end{sphinxuseclass}\end{sphinxVerbatimOutput}

\end{sphinxuseclass}
\sphinxAtStartPar
A pandas series behaves like a NumPy array.
We can use Boolean filters and perform vectorized mathematical operations.

\begin{sphinxuseclass}{cell}\begin{sphinxVerbatimInput}

\begin{sphinxuseclass}{cell_input}
\begin{sphinxVerbatim}[commandchars=\\\{\}]
\PYG{n}{obj2} \PYG{o}{\PYGZgt{}} \PYG{l+m+mi}{0}
\end{sphinxVerbatim}

\end{sphinxuseclass}\end{sphinxVerbatimInput}
\begin{sphinxVerbatimOutput}

\begin{sphinxuseclass}{cell_output}
\begin{sphinxVerbatim}[commandchars=\\\{\}]
d     True
b     True
a    False
c     True
dtype: bool
\end{sphinxVerbatim}

\end{sphinxuseclass}\end{sphinxVerbatimOutput}

\end{sphinxuseclass}
\begin{sphinxuseclass}{cell}\begin{sphinxVerbatimInput}

\begin{sphinxuseclass}{cell_input}
\begin{sphinxVerbatim}[commandchars=\\\{\}]
\PYG{n}{obj2}\PYG{p}{[}\PYG{n}{obj2} \PYG{o}{\PYGZgt{}} \PYG{l+m+mi}{0}\PYG{p}{]}
\end{sphinxVerbatim}

\end{sphinxuseclass}\end{sphinxVerbatimInput}
\begin{sphinxVerbatimOutput}

\begin{sphinxuseclass}{cell_output}
\begin{sphinxVerbatim}[commandchars=\\\{\}]
d    6
b    7
c    3
dtype: int64
\end{sphinxVerbatim}

\end{sphinxuseclass}\end{sphinxVerbatimOutput}

\end{sphinxuseclass}
\begin{sphinxuseclass}{cell}\begin{sphinxVerbatimInput}

\begin{sphinxuseclass}{cell_input}
\begin{sphinxVerbatim}[commandchars=\\\{\}]
\PYG{n}{obj2} \PYG{o}{*} \PYG{l+m+mi}{2}
\end{sphinxVerbatim}

\end{sphinxuseclass}\end{sphinxVerbatimInput}
\begin{sphinxVerbatimOutput}

\begin{sphinxuseclass}{cell_output}
\begin{sphinxVerbatim}[commandchars=\\\{\}]
d    12
b    14
a   \PYGZhy{}10
c     6
dtype: int64
\end{sphinxVerbatim}

\end{sphinxuseclass}\end{sphinxVerbatimOutput}

\end{sphinxuseclass}
\begin{sphinxuseclass}{cell}\begin{sphinxVerbatimInput}

\begin{sphinxuseclass}{cell_input}
\begin{sphinxVerbatim}[commandchars=\\\{\}]
\PYG{l+s+s1}{\PYGZsq{}}\PYG{l+s+s1}{b}\PYG{l+s+s1}{\PYGZsq{}} \PYG{o+ow}{in} \PYG{n}{obj2}
\end{sphinxVerbatim}

\end{sphinxuseclass}\end{sphinxVerbatimInput}
\begin{sphinxVerbatimOutput}

\begin{sphinxuseclass}{cell_output}
\begin{sphinxVerbatim}[commandchars=\\\{\}]
True
\end{sphinxVerbatim}

\end{sphinxuseclass}\end{sphinxVerbatimOutput}

\end{sphinxuseclass}
\begin{sphinxuseclass}{cell}\begin{sphinxVerbatimInput}

\begin{sphinxuseclass}{cell_input}
\begin{sphinxVerbatim}[commandchars=\\\{\}]
\PYG{l+s+s1}{\PYGZsq{}}\PYG{l+s+s1}{e}\PYG{l+s+s1}{\PYGZsq{}} \PYG{o+ow}{in} \PYG{n}{obj2}
\end{sphinxVerbatim}

\end{sphinxuseclass}\end{sphinxVerbatimInput}
\begin{sphinxVerbatimOutput}

\begin{sphinxuseclass}{cell_output}
\begin{sphinxVerbatim}[commandchars=\\\{\}]
False
\end{sphinxVerbatim}

\end{sphinxuseclass}\end{sphinxVerbatimOutput}

\end{sphinxuseclass}
\sphinxAtStartPar
We can create a pandas series from a dictionary.
The dictionary labels become the series index.

\begin{sphinxuseclass}{cell}\begin{sphinxVerbatimInput}

\begin{sphinxuseclass}{cell_input}
\begin{sphinxVerbatim}[commandchars=\\\{\}]
\PYG{n}{sdata} \PYG{o}{=} \PYG{p}{\PYGZob{}}\PYG{l+s+s1}{\PYGZsq{}}\PYG{l+s+s1}{Ohio}\PYG{l+s+s1}{\PYGZsq{}}\PYG{p}{:} \PYG{l+m+mi}{35000}\PYG{p}{,} \PYG{l+s+s1}{\PYGZsq{}}\PYG{l+s+s1}{Texas}\PYG{l+s+s1}{\PYGZsq{}}\PYG{p}{:} \PYG{l+m+mi}{71000}\PYG{p}{,} \PYG{l+s+s1}{\PYGZsq{}}\PYG{l+s+s1}{Oregon}\PYG{l+s+s1}{\PYGZsq{}}\PYG{p}{:} \PYG{l+m+mi}{16000}\PYG{p}{,} \PYG{l+s+s1}{\PYGZsq{}}\PYG{l+s+s1}{Utah}\PYG{l+s+s1}{\PYGZsq{}}\PYG{p}{:} \PYG{l+m+mi}{5000}\PYG{p}{\PYGZcb{}}
\PYG{n}{obj3} \PYG{o}{=} \PYG{n}{pd}\PYG{o}{.}\PYG{n}{Series}\PYG{p}{(}\PYG{n}{sdata}\PYG{p}{)}
\PYG{n}{obj3}
\end{sphinxVerbatim}

\end{sphinxuseclass}\end{sphinxVerbatimInput}
\begin{sphinxVerbatimOutput}

\begin{sphinxuseclass}{cell_output}
\begin{sphinxVerbatim}[commandchars=\\\{\}]
Ohio      35000
Texas     71000
Oregon    16000
Utah       5000
dtype: int64
\end{sphinxVerbatim}

\end{sphinxuseclass}\end{sphinxVerbatimOutput}

\end{sphinxuseclass}
\sphinxAtStartPar
We can create a pandas series from a list, too.
Note that pandas respects the order of the assigned index.
Also, pandas keeps California with \sphinxcode{\sphinxupquote{NaN}} (not a number or missing value) and drops Utah because it was not in the index.

\begin{sphinxuseclass}{cell}\begin{sphinxVerbatimInput}

\begin{sphinxuseclass}{cell_input}
\begin{sphinxVerbatim}[commandchars=\\\{\}]
\PYG{n}{states} \PYG{o}{=} \PYG{p}{[}\PYG{l+s+s1}{\PYGZsq{}}\PYG{l+s+s1}{California}\PYG{l+s+s1}{\PYGZsq{}}\PYG{p}{,} \PYG{l+s+s1}{\PYGZsq{}}\PYG{l+s+s1}{Ohio}\PYG{l+s+s1}{\PYGZsq{}}\PYG{p}{,} \PYG{l+s+s1}{\PYGZsq{}}\PYG{l+s+s1}{Oregon}\PYG{l+s+s1}{\PYGZsq{}}\PYG{p}{,} \PYG{l+s+s1}{\PYGZsq{}}\PYG{l+s+s1}{Texas}\PYG{l+s+s1}{\PYGZsq{}}\PYG{p}{]}
\PYG{n}{obj4} \PYG{o}{=} \PYG{n}{pd}\PYG{o}{.}\PYG{n}{Series}\PYG{p}{(}\PYG{n}{sdata}\PYG{p}{,} \PYG{n}{index}\PYG{o}{=}\PYG{n}{states}\PYG{p}{)}
\PYG{n}{obj4}
\end{sphinxVerbatim}

\end{sphinxuseclass}\end{sphinxVerbatimInput}
\begin{sphinxVerbatimOutput}

\begin{sphinxuseclass}{cell_output}
\begin{sphinxVerbatim}[commandchars=\\\{\}]
California          NaN
Ohio         35000.0000
Oregon       16000.0000
Texas        71000.0000
dtype: float64
\end{sphinxVerbatim}

\end{sphinxuseclass}\end{sphinxVerbatimOutput}

\end{sphinxuseclass}
\sphinxAtStartPar
Indices are one of pandas’ super powers.
When we perform mathematical operations, pandas aligns series by their indices.
Here \sphinxcode{\sphinxupquote{NaN}} is “not a number”, which indicates missing values.
\sphinxcode{\sphinxupquote{NaN}} is considered a float, so the data type switches from int64 to float64.

\begin{sphinxuseclass}{cell}\begin{sphinxVerbatimInput}

\begin{sphinxuseclass}{cell_input}
\begin{sphinxVerbatim}[commandchars=\\\{\}]
\PYG{n}{obj3} \PYG{o}{+} \PYG{n}{obj4}
\end{sphinxVerbatim}

\end{sphinxuseclass}\end{sphinxVerbatimInput}
\begin{sphinxVerbatimOutput}

\begin{sphinxuseclass}{cell_output}
\begin{sphinxVerbatim}[commandchars=\\\{\}]
California           NaN
Ohio          70000.0000
Oregon        32000.0000
Texas        142000.0000
Utah                 NaN
dtype: float64
\end{sphinxVerbatim}

\end{sphinxuseclass}\end{sphinxVerbatimOutput}

\end{sphinxuseclass}

\subsection{DataFrame}
\label{\detokenize{mckinney_05_lecture:dataframe}}
\sphinxAtStartPar
A pandas data frame is like a worksheet in an Excel workbook with row and columns that provide fast indexing.
\begin{quote}

\sphinxAtStartPar
A DataFrame represents a rectangular table of data and contains an ordered collection of columns, each of which can be a different value type (numeric, string, boolean, etc.). The DataFrame has both a row and column index; it can be thought of as a dict of Series all sharing the same index. Under the hood, the data is stored as one or more two\sphinxhyphen{}dimensional blocks rather than a list, dict, or some other collection of one\sphinxhyphen{}dimensional arrays. The exact details of DataFrame’s internals are outside the scope of this book.

\sphinxAtStartPar
There are many ways to construct a DataFrame, though one of the most common is from a dict of equal\sphinxhyphen{}length lists or NumPy arrays:
\end{quote}

\begin{sphinxuseclass}{cell}\begin{sphinxVerbatimInput}

\begin{sphinxuseclass}{cell_input}
\begin{sphinxVerbatim}[commandchars=\\\{\}]
\PYG{n}{data} \PYG{o}{=} \PYG{p}{\PYGZob{}}
    \PYG{l+s+s1}{\PYGZsq{}}\PYG{l+s+s1}{state}\PYG{l+s+s1}{\PYGZsq{}}\PYG{p}{:} \PYG{p}{[}\PYG{l+s+s1}{\PYGZsq{}}\PYG{l+s+s1}{Ohio}\PYG{l+s+s1}{\PYGZsq{}}\PYG{p}{,} \PYG{l+s+s1}{\PYGZsq{}}\PYG{l+s+s1}{Ohio}\PYG{l+s+s1}{\PYGZsq{}}\PYG{p}{,} \PYG{l+s+s1}{\PYGZsq{}}\PYG{l+s+s1}{Ohio}\PYG{l+s+s1}{\PYGZsq{}}\PYG{p}{,} \PYG{l+s+s1}{\PYGZsq{}}\PYG{l+s+s1}{Nevada}\PYG{l+s+s1}{\PYGZsq{}}\PYG{p}{,} \PYG{l+s+s1}{\PYGZsq{}}\PYG{l+s+s1}{Nevada}\PYG{l+s+s1}{\PYGZsq{}}\PYG{p}{,} \PYG{l+s+s1}{\PYGZsq{}}\PYG{l+s+s1}{Nevada}\PYG{l+s+s1}{\PYGZsq{}}\PYG{p}{]}\PYG{p}{,}
    \PYG{l+s+s1}{\PYGZsq{}}\PYG{l+s+s1}{year}\PYG{l+s+s1}{\PYGZsq{}}\PYG{p}{:} \PYG{p}{[}\PYG{l+m+mi}{2000}\PYG{p}{,} \PYG{l+m+mi}{2001}\PYG{p}{,} \PYG{l+m+mi}{2002}\PYG{p}{,} \PYG{l+m+mi}{2001}\PYG{p}{,} \PYG{l+m+mi}{2002}\PYG{p}{,} \PYG{l+m+mi}{2003}\PYG{p}{]}\PYG{p}{,}
    \PYG{l+s+s1}{\PYGZsq{}}\PYG{l+s+s1}{pop}\PYG{l+s+s1}{\PYGZsq{}}\PYG{p}{:} \PYG{p}{[}\PYG{l+m+mf}{1.5}\PYG{p}{,} \PYG{l+m+mf}{1.7}\PYG{p}{,} \PYG{l+m+mf}{3.6}\PYG{p}{,} \PYG{l+m+mf}{2.4}\PYG{p}{,} \PYG{l+m+mf}{2.9}\PYG{p}{,} \PYG{l+m+mf}{3.2}\PYG{p}{]}
\PYG{p}{\PYGZcb{}}
\PYG{n}{frame} \PYG{o}{=} \PYG{n}{pd}\PYG{o}{.}\PYG{n}{DataFrame}\PYG{p}{(}\PYG{n}{data}\PYG{p}{)}

\PYG{n}{frame}
\end{sphinxVerbatim}

\end{sphinxuseclass}\end{sphinxVerbatimInput}
\begin{sphinxVerbatimOutput}

\begin{sphinxuseclass}{cell_output}
\begin{sphinxVerbatim}[commandchars=\\\{\}]
    state  year    pop
0    Ohio  2000 1.5000
1    Ohio  2001 1.7000
2    Ohio  2002 3.6000
3  Nevada  2001 2.4000
4  Nevada  2002 2.9000
5  Nevada  2003 3.2000
\end{sphinxVerbatim}

\end{sphinxuseclass}\end{sphinxVerbatimOutput}

\end{sphinxuseclass}
\sphinxAtStartPar
We did not specify an index, so \sphinxcode{\sphinxupquote{frame}} has the default index of integers starting at 0.

\begin{sphinxuseclass}{cell}\begin{sphinxVerbatimInput}

\begin{sphinxuseclass}{cell_input}
\begin{sphinxVerbatim}[commandchars=\\\{\}]
\PYG{n}{frame2} \PYG{o}{=} \PYG{n}{pd}\PYG{o}{.}\PYG{n}{DataFrame}\PYG{p}{(}
    \PYG{n}{data}\PYG{p}{,} 
    \PYG{n}{columns}\PYG{o}{=}\PYG{p}{[}\PYG{l+s+s1}{\PYGZsq{}}\PYG{l+s+s1}{year}\PYG{l+s+s1}{\PYGZsq{}}\PYG{p}{,} \PYG{l+s+s1}{\PYGZsq{}}\PYG{l+s+s1}{state}\PYG{l+s+s1}{\PYGZsq{}}\PYG{p}{,} \PYG{l+s+s1}{\PYGZsq{}}\PYG{l+s+s1}{pop}\PYG{l+s+s1}{\PYGZsq{}}\PYG{p}{,} \PYG{l+s+s1}{\PYGZsq{}}\PYG{l+s+s1}{debt}\PYG{l+s+s1}{\PYGZsq{}}\PYG{p}{]}\PYG{p}{,}
    \PYG{n}{index}\PYG{o}{=}\PYG{p}{[}\PYG{l+s+s1}{\PYGZsq{}}\PYG{l+s+s1}{one}\PYG{l+s+s1}{\PYGZsq{}}\PYG{p}{,} \PYG{l+s+s1}{\PYGZsq{}}\PYG{l+s+s1}{two}\PYG{l+s+s1}{\PYGZsq{}}\PYG{p}{,} \PYG{l+s+s1}{\PYGZsq{}}\PYG{l+s+s1}{three}\PYG{l+s+s1}{\PYGZsq{}}\PYG{p}{,} \PYG{l+s+s1}{\PYGZsq{}}\PYG{l+s+s1}{four}\PYG{l+s+s1}{\PYGZsq{}}\PYG{p}{,} \PYG{l+s+s1}{\PYGZsq{}}\PYG{l+s+s1}{five}\PYG{l+s+s1}{\PYGZsq{}}\PYG{p}{,} \PYG{l+s+s1}{\PYGZsq{}}\PYG{l+s+s1}{six}\PYG{l+s+s1}{\PYGZsq{}}\PYG{p}{]}
\PYG{p}{)}

\PYG{n}{frame2}
\end{sphinxVerbatim}

\end{sphinxuseclass}\end{sphinxVerbatimInput}
\begin{sphinxVerbatimOutput}

\begin{sphinxuseclass}{cell_output}
\begin{sphinxVerbatim}[commandchars=\\\{\}]
       year   state    pop debt
one    2000    Ohio 1.5000  NaN
two    2001    Ohio 1.7000  NaN
three  2002    Ohio 3.6000  NaN
four   2001  Nevada 2.4000  NaN
five   2002  Nevada 2.9000  NaN
six    2003  Nevada 3.2000  NaN
\end{sphinxVerbatim}

\end{sphinxuseclass}\end{sphinxVerbatimOutput}

\end{sphinxuseclass}
\sphinxAtStartPar
If we extract one column, via either \sphinxcode{\sphinxupquote{df.column}} or \sphinxcode{\sphinxupquote{df{[}'column'{]}}}, the result is a series.
We can use either the \sphinxcode{\sphinxupquote{df.colname}} or the \sphinxcode{\sphinxupquote{df{[}'colname'{]}}} syntax to \sphinxstyleemphasis{extract} a column from a data frame as a series.
\sphinxstyleemphasis{\sphinxstylestrong{However, we must use the \sphinxcode{\sphinxupquote{df{[}'colname'{]}}} syntax to \sphinxstyleemphasis{add} a column to a data frame.}}
Also, we must use the \sphinxcode{\sphinxupquote{df{[}'colname'{]}}} syntax to extract or add a column whose name contains a whitespace.

\begin{sphinxuseclass}{cell}\begin{sphinxVerbatimInput}

\begin{sphinxuseclass}{cell_input}
\begin{sphinxVerbatim}[commandchars=\\\{\}]
\PYG{n}{frame2}\PYG{p}{[}\PYG{l+s+s1}{\PYGZsq{}}\PYG{l+s+s1}{state}\PYG{l+s+s1}{\PYGZsq{}}\PYG{p}{]}
\end{sphinxVerbatim}

\end{sphinxuseclass}\end{sphinxVerbatimInput}
\begin{sphinxVerbatimOutput}

\begin{sphinxuseclass}{cell_output}
\begin{sphinxVerbatim}[commandchars=\\\{\}]
one        Ohio
two        Ohio
three      Ohio
four     Nevada
five     Nevada
six      Nevada
Name: state, dtype: object
\end{sphinxVerbatim}

\end{sphinxuseclass}\end{sphinxVerbatimOutput}

\end{sphinxuseclass}
\begin{sphinxuseclass}{cell}\begin{sphinxVerbatimInput}

\begin{sphinxuseclass}{cell_input}
\begin{sphinxVerbatim}[commandchars=\\\{\}]
\PYG{n}{frame2}\PYG{o}{.}\PYG{n}{state}
\end{sphinxVerbatim}

\end{sphinxuseclass}\end{sphinxVerbatimInput}
\begin{sphinxVerbatimOutput}

\begin{sphinxuseclass}{cell_output}
\begin{sphinxVerbatim}[commandchars=\\\{\}]
one        Ohio
two        Ohio
three      Ohio
four     Nevada
five     Nevada
six      Nevada
Name: state, dtype: object
\end{sphinxVerbatim}

\end{sphinxuseclass}\end{sphinxVerbatimOutput}

\end{sphinxuseclass}
\sphinxAtStartPar
Similarly, if we extract one row. via either \sphinxcode{\sphinxupquote{df.loc{[}'rowlabel'{]}}} or \sphinxcode{\sphinxupquote{df.iloc{[}rownumber{]}}}, the result is a series.

\begin{sphinxuseclass}{cell}\begin{sphinxVerbatimInput}

\begin{sphinxuseclass}{cell_input}
\begin{sphinxVerbatim}[commandchars=\\\{\}]
\PYG{n}{frame2}\PYG{o}{.}\PYG{n}{loc}\PYG{p}{[}\PYG{l+s+s1}{\PYGZsq{}}\PYG{l+s+s1}{one}\PYG{l+s+s1}{\PYGZsq{}}\PYG{p}{]}
\end{sphinxVerbatim}

\end{sphinxuseclass}\end{sphinxVerbatimInput}
\begin{sphinxVerbatimOutput}

\begin{sphinxuseclass}{cell_output}
\begin{sphinxVerbatim}[commandchars=\\\{\}]
year      2000
state     Ohio
pop     1.5000
debt       NaN
Name: one, dtype: object
\end{sphinxVerbatim}

\end{sphinxuseclass}\end{sphinxVerbatimOutput}

\end{sphinxuseclass}
\sphinxAtStartPar
Data frame have two dimensions, so we have to slice data frames more precisely than series.
\begin{enumerate}
\sphinxsetlistlabels{\arabic}{enumi}{enumii}{}{.}%
\item {} 
\sphinxAtStartPar
The \sphinxcode{\sphinxupquote{.loc{[}{]}}} method slices by row labels and column names

\item {} 
\sphinxAtStartPar
The \sphinxcode{\sphinxupquote{.iloc{[}{]}}} method slices by \sphinxstyleemphasis{integer} row and label indices

\end{enumerate}

\begin{sphinxuseclass}{cell}\begin{sphinxVerbatimInput}

\begin{sphinxuseclass}{cell_input}
\begin{sphinxVerbatim}[commandchars=\\\{\}]
\PYG{n}{frame2}\PYG{o}{.}\PYG{n}{loc}\PYG{p}{[}\PYG{l+s+s1}{\PYGZsq{}}\PYG{l+s+s1}{three}\PYG{l+s+s1}{\PYGZsq{}}\PYG{p}{]}
\end{sphinxVerbatim}

\end{sphinxuseclass}\end{sphinxVerbatimInput}
\begin{sphinxVerbatimOutput}

\begin{sphinxuseclass}{cell_output}
\begin{sphinxVerbatim}[commandchars=\\\{\}]
year      2002
state     Ohio
pop     3.6000
debt       NaN
Name: three, dtype: object
\end{sphinxVerbatim}

\end{sphinxuseclass}\end{sphinxVerbatimOutput}

\end{sphinxuseclass}
\begin{sphinxuseclass}{cell}\begin{sphinxVerbatimInput}

\begin{sphinxuseclass}{cell_input}
\begin{sphinxVerbatim}[commandchars=\\\{\}]
\PYG{n}{frame2}\PYG{o}{.}\PYG{n}{iloc}\PYG{p}{[}\PYG{l+m+mi}{2}\PYG{p}{]}
\end{sphinxVerbatim}

\end{sphinxuseclass}\end{sphinxVerbatimInput}
\begin{sphinxVerbatimOutput}

\begin{sphinxuseclass}{cell_output}
\begin{sphinxVerbatim}[commandchars=\\\{\}]
year      2002
state     Ohio
pop     3.6000
debt       NaN
Name: three, dtype: object
\end{sphinxVerbatim}

\end{sphinxuseclass}\end{sphinxVerbatimOutput}

\end{sphinxuseclass}
\sphinxAtStartPar
We can use NumPy’s \sphinxcode{\sphinxupquote{{[}row, column{]}}} syntanx with \sphinxcode{\sphinxupquote{.loc{[}{]}}} and \sphinxcode{\sphinxupquote{.iloc{[}{]}}}.

\begin{sphinxuseclass}{cell}\begin{sphinxVerbatimInput}

\begin{sphinxuseclass}{cell_input}
\begin{sphinxVerbatim}[commandchars=\\\{\}]
\PYG{n}{frame2}\PYG{o}{.}\PYG{n}{loc}\PYG{p}{[}\PYG{l+s+s1}{\PYGZsq{}}\PYG{l+s+s1}{three}\PYG{l+s+s1}{\PYGZsq{}}\PYG{p}{,} \PYG{l+s+s1}{\PYGZsq{}}\PYG{l+s+s1}{state}\PYG{l+s+s1}{\PYGZsq{}}\PYG{p}{]} \PYG{c+c1}{\PYGZsh{} row, column}
\end{sphinxVerbatim}

\end{sphinxuseclass}\end{sphinxVerbatimInput}
\begin{sphinxVerbatimOutput}

\begin{sphinxuseclass}{cell_output}
\begin{sphinxVerbatim}[commandchars=\\\{\}]
\PYGZsq{}Ohio\PYGZsq{}
\end{sphinxVerbatim}

\end{sphinxuseclass}\end{sphinxVerbatimOutput}

\end{sphinxuseclass}
\begin{sphinxuseclass}{cell}\begin{sphinxVerbatimInput}

\begin{sphinxuseclass}{cell_input}
\begin{sphinxVerbatim}[commandchars=\\\{\}]
\PYG{n}{frame2}\PYG{o}{.}\PYG{n}{loc}\PYG{p}{[}\PYG{l+s+s1}{\PYGZsq{}}\PYG{l+s+s1}{three}\PYG{l+s+s1}{\PYGZsq{}}\PYG{p}{,} \PYG{p}{[}\PYG{l+s+s1}{\PYGZsq{}}\PYG{l+s+s1}{state}\PYG{l+s+s1}{\PYGZsq{}}\PYG{p}{,} \PYG{l+s+s1}{\PYGZsq{}}\PYG{l+s+s1}{pop}\PYG{l+s+s1}{\PYGZsq{}}\PYG{p}{]}\PYG{p}{]} \PYG{c+c1}{\PYGZsh{} row, column}
\end{sphinxVerbatim}

\end{sphinxuseclass}\end{sphinxVerbatimInput}
\begin{sphinxVerbatimOutput}

\begin{sphinxuseclass}{cell_output}
\begin{sphinxVerbatim}[commandchars=\\\{\}]
state     Ohio
pop     3.6000
Name: three, dtype: object
\end{sphinxVerbatim}

\end{sphinxuseclass}\end{sphinxVerbatimOutput}

\end{sphinxuseclass}
\sphinxAtStartPar
We can assign either scalars or arrays to data frame columns.
\begin{enumerate}
\sphinxsetlistlabels{\arabic}{enumi}{enumii}{}{.}%
\item {} 
\sphinxAtStartPar
Scalars will broadcast to every row in the data frame

\item {} 
\sphinxAtStartPar
Arrays must have the same length as the column

\end{enumerate}

\begin{sphinxuseclass}{cell}\begin{sphinxVerbatimInput}

\begin{sphinxuseclass}{cell_input}
\begin{sphinxVerbatim}[commandchars=\\\{\}]
\PYG{n}{frame2}\PYG{p}{[}\PYG{l+s+s1}{\PYGZsq{}}\PYG{l+s+s1}{debt}\PYG{l+s+s1}{\PYGZsq{}}\PYG{p}{]} \PYG{o}{=} \PYG{l+m+mf}{16.5}
\PYG{n}{frame2}
\end{sphinxVerbatim}

\end{sphinxuseclass}\end{sphinxVerbatimInput}
\begin{sphinxVerbatimOutput}

\begin{sphinxuseclass}{cell_output}
\begin{sphinxVerbatim}[commandchars=\\\{\}]
       year   state    pop    debt
one    2000    Ohio 1.5000 16.5000
two    2001    Ohio 1.7000 16.5000
three  2002    Ohio 3.6000 16.5000
four   2001  Nevada 2.4000 16.5000
five   2002  Nevada 2.9000 16.5000
six    2003  Nevada 3.2000 16.5000
\end{sphinxVerbatim}

\end{sphinxuseclass}\end{sphinxVerbatimOutput}

\end{sphinxuseclass}
\begin{sphinxuseclass}{cell}\begin{sphinxVerbatimInput}

\begin{sphinxuseclass}{cell_input}
\begin{sphinxVerbatim}[commandchars=\\\{\}]
\PYG{n}{frame2}\PYG{p}{[}\PYG{l+s+s1}{\PYGZsq{}}\PYG{l+s+s1}{debt}\PYG{l+s+s1}{\PYGZsq{}}\PYG{p}{]} \PYG{o}{=} \PYG{n}{np}\PYG{o}{.}\PYG{n}{arange}\PYG{p}{(}\PYG{l+m+mf}{6.}\PYG{p}{)}
\PYG{n}{frame2}
\end{sphinxVerbatim}

\end{sphinxuseclass}\end{sphinxVerbatimInput}
\begin{sphinxVerbatimOutput}

\begin{sphinxuseclass}{cell_output}
\begin{sphinxVerbatim}[commandchars=\\\{\}]
       year   state    pop   debt
one    2000    Ohio 1.5000 0.0000
two    2001    Ohio 1.7000 1.0000
three  2002    Ohio 3.6000 2.0000
four   2001  Nevada 2.4000 3.0000
five   2002  Nevada 2.9000 4.0000
six    2003  Nevada 3.2000 5.0000
\end{sphinxVerbatim}

\end{sphinxuseclass}\end{sphinxVerbatimOutput}

\end{sphinxuseclass}
\sphinxAtStartPar
If we assign a series to a data frame column, pandas will use the index to align it with the data frame.
Data frame rows not in the series will be missing values \sphinxcode{\sphinxupquote{NaN}}.

\begin{sphinxuseclass}{cell}\begin{sphinxVerbatimInput}

\begin{sphinxuseclass}{cell_input}
\begin{sphinxVerbatim}[commandchars=\\\{\}]
\PYG{n}{val} \PYG{o}{=} \PYG{n}{pd}\PYG{o}{.}\PYG{n}{Series}\PYG{p}{(}\PYG{p}{[}\PYG{o}{\PYGZhy{}}\PYG{l+m+mf}{1.2}\PYG{p}{,} \PYG{o}{\PYGZhy{}}\PYG{l+m+mf}{1.5}\PYG{p}{,} \PYG{o}{\PYGZhy{}}\PYG{l+m+mf}{1.7}\PYG{p}{]}\PYG{p}{,} \PYG{n}{index}\PYG{o}{=}\PYG{p}{[}\PYG{l+s+s1}{\PYGZsq{}}\PYG{l+s+s1}{two}\PYG{l+s+s1}{\PYGZsq{}}\PYG{p}{,} \PYG{l+s+s1}{\PYGZsq{}}\PYG{l+s+s1}{four}\PYG{l+s+s1}{\PYGZsq{}}\PYG{p}{,} \PYG{l+s+s1}{\PYGZsq{}}\PYG{l+s+s1}{five}\PYG{l+s+s1}{\PYGZsq{}}\PYG{p}{]}\PYG{p}{)}
\PYG{n}{val}
\end{sphinxVerbatim}

\end{sphinxuseclass}\end{sphinxVerbatimInput}
\begin{sphinxVerbatimOutput}

\begin{sphinxuseclass}{cell_output}
\begin{sphinxVerbatim}[commandchars=\\\{\}]
two    \PYGZhy{}1.2000
four   \PYGZhy{}1.5000
five   \PYGZhy{}1.7000
dtype: float64
\end{sphinxVerbatim}

\end{sphinxuseclass}\end{sphinxVerbatimOutput}

\end{sphinxuseclass}
\begin{sphinxuseclass}{cell}\begin{sphinxVerbatimInput}

\begin{sphinxuseclass}{cell_input}
\begin{sphinxVerbatim}[commandchars=\\\{\}]
\PYG{n}{frame2}\PYG{p}{[}\PYG{l+s+s1}{\PYGZsq{}}\PYG{l+s+s1}{debt}\PYG{l+s+s1}{\PYGZsq{}}\PYG{p}{]} \PYG{o}{=} \PYG{n}{val}
\PYG{n}{frame2}
\end{sphinxVerbatim}

\end{sphinxuseclass}\end{sphinxVerbatimInput}
\begin{sphinxVerbatimOutput}

\begin{sphinxuseclass}{cell_output}
\begin{sphinxVerbatim}[commandchars=\\\{\}]
       year   state    pop    debt
one    2000    Ohio 1.5000     NaN
two    2001    Ohio 1.7000 \PYGZhy{}1.2000
three  2002    Ohio 3.6000     NaN
four   2001  Nevada 2.4000 \PYGZhy{}1.5000
five   2002  Nevada 2.9000 \PYGZhy{}1.7000
six    2003  Nevada 3.2000     NaN
\end{sphinxVerbatim}

\end{sphinxuseclass}\end{sphinxVerbatimOutput}

\end{sphinxuseclass}
\sphinxAtStartPar
We can add columns to our data frame, then delete them with \sphinxcode{\sphinxupquote{del}}.

\begin{sphinxuseclass}{cell}\begin{sphinxVerbatimInput}

\begin{sphinxuseclass}{cell_input}
\begin{sphinxVerbatim}[commandchars=\\\{\}]
\PYG{n}{frame2}\PYG{p}{[}\PYG{l+s+s1}{\PYGZsq{}}\PYG{l+s+s1}{eastern}\PYG{l+s+s1}{\PYGZsq{}}\PYG{p}{]} \PYG{o}{=} \PYG{p}{(}\PYG{n}{frame2}\PYG{o}{.}\PYG{n}{state} \PYG{o}{==} \PYG{l+s+s1}{\PYGZsq{}}\PYG{l+s+s1}{Ohio}\PYG{l+s+s1}{\PYGZsq{}}\PYG{p}{)}
\PYG{n}{frame2}
\end{sphinxVerbatim}

\end{sphinxuseclass}\end{sphinxVerbatimInput}
\begin{sphinxVerbatimOutput}

\begin{sphinxuseclass}{cell_output}
\begin{sphinxVerbatim}[commandchars=\\\{\}]
       year   state    pop    debt  eastern
one    2000    Ohio 1.5000     NaN     True
two    2001    Ohio 1.7000 \PYGZhy{}1.2000     True
three  2002    Ohio 3.6000     NaN     True
four   2001  Nevada 2.4000 \PYGZhy{}1.5000    False
five   2002  Nevada 2.9000 \PYGZhy{}1.7000    False
six    2003  Nevada 3.2000     NaN    False
\end{sphinxVerbatim}

\end{sphinxuseclass}\end{sphinxVerbatimOutput}

\end{sphinxuseclass}
\begin{sphinxuseclass}{cell}\begin{sphinxVerbatimInput}

\begin{sphinxuseclass}{cell_input}
\begin{sphinxVerbatim}[commandchars=\\\{\}]
\PYG{k}{del} \PYG{n}{frame2}\PYG{p}{[}\PYG{l+s+s1}{\PYGZsq{}}\PYG{l+s+s1}{eastern}\PYG{l+s+s1}{\PYGZsq{}}\PYG{p}{]}
\PYG{n}{frame2}
\end{sphinxVerbatim}

\end{sphinxuseclass}\end{sphinxVerbatimInput}
\begin{sphinxVerbatimOutput}

\begin{sphinxuseclass}{cell_output}
\begin{sphinxVerbatim}[commandchars=\\\{\}]
       year   state    pop    debt
one    2000    Ohio 1.5000     NaN
two    2001    Ohio 1.7000 \PYGZhy{}1.2000
three  2002    Ohio 3.6000     NaN
four   2001  Nevada 2.4000 \PYGZhy{}1.5000
five   2002  Nevada 2.9000 \PYGZhy{}1.7000
six    2003  Nevada 3.2000     NaN
\end{sphinxVerbatim}

\end{sphinxuseclass}\end{sphinxVerbatimOutput}

\end{sphinxuseclass}

\subsection{Index Objects}
\label{\detokenize{mckinney_05_lecture:index-objects}}
\begin{sphinxuseclass}{cell}\begin{sphinxVerbatimInput}

\begin{sphinxuseclass}{cell_input}
\begin{sphinxVerbatim}[commandchars=\\\{\}]
\PYG{n}{obj} \PYG{o}{=} \PYG{n}{pd}\PYG{o}{.}\PYG{n}{Series}\PYG{p}{(}\PYG{n+nb}{range}\PYG{p}{(}\PYG{l+m+mi}{3}\PYG{p}{)}\PYG{p}{,} \PYG{n}{index}\PYG{o}{=}\PYG{p}{[}\PYG{l+s+s1}{\PYGZsq{}}\PYG{l+s+s1}{a}\PYG{l+s+s1}{\PYGZsq{}}\PYG{p}{,} \PYG{l+s+s1}{\PYGZsq{}}\PYG{l+s+s1}{b}\PYG{l+s+s1}{\PYGZsq{}}\PYG{p}{,} \PYG{l+s+s1}{\PYGZsq{}}\PYG{l+s+s1}{c}\PYG{l+s+s1}{\PYGZsq{}}\PYG{p}{]}\PYG{p}{)}
\PYG{n}{index} \PYG{o}{=} \PYG{n}{obj}\PYG{o}{.}\PYG{n}{index}
\end{sphinxVerbatim}

\end{sphinxuseclass}\end{sphinxVerbatimInput}

\end{sphinxuseclass}
\begin{sphinxuseclass}{cell}\begin{sphinxVerbatimInput}

\begin{sphinxuseclass}{cell_input}
\begin{sphinxVerbatim}[commandchars=\\\{\}]
\PYG{n}{index}\PYG{p}{[}\PYG{l+m+mi}{1}\PYG{p}{:}\PYG{p}{]}
\end{sphinxVerbatim}

\end{sphinxuseclass}\end{sphinxVerbatimInput}
\begin{sphinxVerbatimOutput}

\begin{sphinxuseclass}{cell_output}
\begin{sphinxVerbatim}[commandchars=\\\{\}]
Index([\PYGZsq{}b\PYGZsq{}, \PYGZsq{}c\PYGZsq{}], dtype=\PYGZsq{}object\PYGZsq{})
\end{sphinxVerbatim}

\end{sphinxuseclass}\end{sphinxVerbatimOutput}

\end{sphinxuseclass}
\sphinxAtStartPar
Index objects are immutable!

\begin{sphinxuseclass}{cell}\begin{sphinxVerbatimInput}

\begin{sphinxuseclass}{cell_input}
\begin{sphinxVerbatim}[commandchars=\\\{\}]
\PYG{c+c1}{\PYGZsh{} index[1] = \PYGZsq{}d\PYGZsq{}  \PYGZsh{} TypeError: Index does not support mutable operations}
\end{sphinxVerbatim}

\end{sphinxuseclass}\end{sphinxVerbatimInput}

\end{sphinxuseclass}
\sphinxAtStartPar
Indices can contain duplicates, so an index does not guarantee our data are duplicate\sphinxhyphen{}free.

\begin{sphinxuseclass}{cell}\begin{sphinxVerbatimInput}

\begin{sphinxuseclass}{cell_input}
\begin{sphinxVerbatim}[commandchars=\\\{\}]
\PYG{n}{dup\PYGZus{}labels} \PYG{o}{=} \PYG{n}{pd}\PYG{o}{.}\PYG{n}{Index}\PYG{p}{(}\PYG{p}{[}\PYG{l+s+s1}{\PYGZsq{}}\PYG{l+s+s1}{foo}\PYG{l+s+s1}{\PYGZsq{}}\PYG{p}{,} \PYG{l+s+s1}{\PYGZsq{}}\PYG{l+s+s1}{foo}\PYG{l+s+s1}{\PYGZsq{}}\PYG{p}{,} \PYG{l+s+s1}{\PYGZsq{}}\PYG{l+s+s1}{bar}\PYG{l+s+s1}{\PYGZsq{}}\PYG{p}{,} \PYG{l+s+s1}{\PYGZsq{}}\PYG{l+s+s1}{bar}\PYG{l+s+s1}{\PYGZsq{}}\PYG{p}{]}\PYG{p}{)}
\end{sphinxVerbatim}

\end{sphinxuseclass}\end{sphinxVerbatimInput}

\end{sphinxuseclass}

\section{Essential Functionality}
\label{\detokenize{mckinney_05_lecture:essential-functionality}}
\sphinxAtStartPar
This section provides the most import pandas operations.
It is difficult to provide an exhaustive reference, but this section provides a head start on the core pandas functionality.


\subsection{Dropping Entries from an Axis}
\label{\detokenize{mckinney_05_lecture:dropping-entries-from-an-axis}}\begin{quote}

\sphinxAtStartPar
Dropping one or more entries from an axis is easy if you already have an index array or list without those entries. As that can require a bit of munging and set logic, the  drop method will return a new object with the indicated value or values deleted from an axis.
\end{quote}

\begin{sphinxuseclass}{cell}\begin{sphinxVerbatimInput}

\begin{sphinxuseclass}{cell_input}
\begin{sphinxVerbatim}[commandchars=\\\{\}]
\PYG{n}{obj} \PYG{o}{=} \PYG{n}{pd}\PYG{o}{.}\PYG{n}{Series}\PYG{p}{(}\PYG{n}{np}\PYG{o}{.}\PYG{n}{arange}\PYG{p}{(}\PYG{l+m+mf}{5.}\PYG{p}{)}\PYG{p}{,} \PYG{n}{index}\PYG{o}{=}\PYG{p}{[}\PYG{l+s+s1}{\PYGZsq{}}\PYG{l+s+s1}{a}\PYG{l+s+s1}{\PYGZsq{}}\PYG{p}{,} \PYG{l+s+s1}{\PYGZsq{}}\PYG{l+s+s1}{b}\PYG{l+s+s1}{\PYGZsq{}}\PYG{p}{,} \PYG{l+s+s1}{\PYGZsq{}}\PYG{l+s+s1}{c}\PYG{l+s+s1}{\PYGZsq{}}\PYG{p}{,} \PYG{l+s+s1}{\PYGZsq{}}\PYG{l+s+s1}{d}\PYG{l+s+s1}{\PYGZsq{}}\PYG{p}{,} \PYG{l+s+s1}{\PYGZsq{}}\PYG{l+s+s1}{e}\PYG{l+s+s1}{\PYGZsq{}}\PYG{p}{]}\PYG{p}{)}
\PYG{n}{obj}
\end{sphinxVerbatim}

\end{sphinxuseclass}\end{sphinxVerbatimInput}
\begin{sphinxVerbatimOutput}

\begin{sphinxuseclass}{cell_output}
\begin{sphinxVerbatim}[commandchars=\\\{\}]
a   0.0000
b   1.0000
c   2.0000
d   3.0000
e   4.0000
dtype: float64
\end{sphinxVerbatim}

\end{sphinxuseclass}\end{sphinxVerbatimOutput}

\end{sphinxuseclass}
\begin{sphinxuseclass}{cell}\begin{sphinxVerbatimInput}

\begin{sphinxuseclass}{cell_input}
\begin{sphinxVerbatim}[commandchars=\\\{\}]
\PYG{n}{obj\PYGZus{}without\PYGZus{}d\PYGZus{}and\PYGZus{}c} \PYG{o}{=} \PYG{n}{obj}\PYG{o}{.}\PYG{n}{drop}\PYG{p}{(}\PYG{p}{[}\PYG{l+s+s1}{\PYGZsq{}}\PYG{l+s+s1}{d}\PYG{l+s+s1}{\PYGZsq{}}\PYG{p}{,} \PYG{l+s+s1}{\PYGZsq{}}\PYG{l+s+s1}{c}\PYG{l+s+s1}{\PYGZsq{}}\PYG{p}{]}\PYG{p}{)}
\PYG{n}{obj\PYGZus{}without\PYGZus{}d\PYGZus{}and\PYGZus{}c}
\end{sphinxVerbatim}

\end{sphinxuseclass}\end{sphinxVerbatimInput}
\begin{sphinxVerbatimOutput}

\begin{sphinxuseclass}{cell_output}
\begin{sphinxVerbatim}[commandchars=\\\{\}]
a   0.0000
b   1.0000
e   4.0000
dtype: float64
\end{sphinxVerbatim}

\end{sphinxuseclass}\end{sphinxVerbatimOutput}

\end{sphinxuseclass}
\sphinxAtStartPar
The \sphinxcode{\sphinxupquote{.drop()}} method works on data frames, too.

\begin{sphinxuseclass}{cell}\begin{sphinxVerbatimInput}

\begin{sphinxuseclass}{cell_input}
\begin{sphinxVerbatim}[commandchars=\\\{\}]
\PYG{n}{data} \PYG{o}{=} \PYG{n}{pd}\PYG{o}{.}\PYG{n}{DataFrame}\PYG{p}{(}
    \PYG{n}{np}\PYG{o}{.}\PYG{n}{arange}\PYG{p}{(}\PYG{l+m+mi}{16}\PYG{p}{)}\PYG{o}{.}\PYG{n}{reshape}\PYG{p}{(}\PYG{p}{(}\PYG{l+m+mi}{4}\PYG{p}{,} \PYG{l+m+mi}{4}\PYG{p}{)}\PYG{p}{)}\PYG{p}{,}
    \PYG{n}{index}\PYG{o}{=}\PYG{p}{[}\PYG{l+s+s1}{\PYGZsq{}}\PYG{l+s+s1}{Ohio}\PYG{l+s+s1}{\PYGZsq{}}\PYG{p}{,} \PYG{l+s+s1}{\PYGZsq{}}\PYG{l+s+s1}{Colorado}\PYG{l+s+s1}{\PYGZsq{}}\PYG{p}{,} \PYG{l+s+s1}{\PYGZsq{}}\PYG{l+s+s1}{Utah}\PYG{l+s+s1}{\PYGZsq{}}\PYG{p}{,} \PYG{l+s+s1}{\PYGZsq{}}\PYG{l+s+s1}{New York}\PYG{l+s+s1}{\PYGZsq{}}\PYG{p}{]}\PYG{p}{,}
    \PYG{n}{columns}\PYG{o}{=}\PYG{p}{[}\PYG{l+s+s1}{\PYGZsq{}}\PYG{l+s+s1}{one}\PYG{l+s+s1}{\PYGZsq{}}\PYG{p}{,} \PYG{l+s+s1}{\PYGZsq{}}\PYG{l+s+s1}{two}\PYG{l+s+s1}{\PYGZsq{}}\PYG{p}{,} \PYG{l+s+s1}{\PYGZsq{}}\PYG{l+s+s1}{three}\PYG{l+s+s1}{\PYGZsq{}}\PYG{p}{,} \PYG{l+s+s1}{\PYGZsq{}}\PYG{l+s+s1}{four}\PYG{l+s+s1}{\PYGZsq{}}\PYG{p}{]}
\PYG{p}{)}

\PYG{n}{data}
\end{sphinxVerbatim}

\end{sphinxuseclass}\end{sphinxVerbatimInput}
\begin{sphinxVerbatimOutput}

\begin{sphinxuseclass}{cell_output}
\begin{sphinxVerbatim}[commandchars=\\\{\}]
          one  two  three  four
Ohio        0    1      2     3
Colorado    4    5      6     7
Utah        8    9     10    11
New York   12   13     14    15
\end{sphinxVerbatim}

\end{sphinxuseclass}\end{sphinxVerbatimOutput}

\end{sphinxuseclass}
\begin{sphinxuseclass}{cell}\begin{sphinxVerbatimInput}

\begin{sphinxuseclass}{cell_input}
\begin{sphinxVerbatim}[commandchars=\\\{\}]
\PYG{n}{data}\PYG{o}{.}\PYG{n}{drop}\PYG{p}{(}\PYG{p}{[}\PYG{l+s+s1}{\PYGZsq{}}\PYG{l+s+s1}{Colorado}\PYG{l+s+s1}{\PYGZsq{}}\PYG{p}{,} \PYG{l+s+s1}{\PYGZsq{}}\PYG{l+s+s1}{Ohio}\PYG{l+s+s1}{\PYGZsq{}}\PYG{p}{]}\PYG{p}{)} \PYG{c+c1}{\PYGZsh{} implied \PYGZdq{}, axis=0\PYGZdq{}}
\end{sphinxVerbatim}

\end{sphinxuseclass}\end{sphinxVerbatimInput}
\begin{sphinxVerbatimOutput}

\begin{sphinxuseclass}{cell_output}
\begin{sphinxVerbatim}[commandchars=\\\{\}]
          one  two  three  four
Utah        8    9     10    11
New York   12   13     14    15
\end{sphinxVerbatim}

\end{sphinxuseclass}\end{sphinxVerbatimOutput}

\end{sphinxuseclass}
\begin{sphinxuseclass}{cell}\begin{sphinxVerbatimInput}

\begin{sphinxuseclass}{cell_input}
\begin{sphinxVerbatim}[commandchars=\\\{\}]
\PYG{n}{data}\PYG{o}{.}\PYG{n}{drop}\PYG{p}{(}\PYG{p}{[}\PYG{l+s+s1}{\PYGZsq{}}\PYG{l+s+s1}{Colorado}\PYG{l+s+s1}{\PYGZsq{}}\PYG{p}{,} \PYG{l+s+s1}{\PYGZsq{}}\PYG{l+s+s1}{Ohio}\PYG{l+s+s1}{\PYGZsq{}}\PYG{p}{]}\PYG{p}{,} \PYG{n}{axis}\PYG{o}{=}\PYG{l+m+mi}{0}\PYG{p}{)}
\end{sphinxVerbatim}

\end{sphinxuseclass}\end{sphinxVerbatimInput}
\begin{sphinxVerbatimOutput}

\begin{sphinxuseclass}{cell_output}
\begin{sphinxVerbatim}[commandchars=\\\{\}]
          one  two  three  four
Utah        8    9     10    11
New York   12   13     14    15
\end{sphinxVerbatim}

\end{sphinxuseclass}\end{sphinxVerbatimOutput}

\end{sphinxuseclass}
\begin{sphinxuseclass}{cell}\begin{sphinxVerbatimInput}

\begin{sphinxuseclass}{cell_input}
\begin{sphinxVerbatim}[commandchars=\\\{\}]
\PYG{n}{data}\PYG{o}{.}\PYG{n}{drop}\PYG{p}{(}\PYG{n}{index}\PYG{o}{=}\PYG{p}{[}\PYG{l+s+s1}{\PYGZsq{}}\PYG{l+s+s1}{Colorado}\PYG{l+s+s1}{\PYGZsq{}}\PYG{p}{,} \PYG{l+s+s1}{\PYGZsq{}}\PYG{l+s+s1}{Ohio}\PYG{l+s+s1}{\PYGZsq{}}\PYG{p}{]}\PYG{p}{)}
\end{sphinxVerbatim}

\end{sphinxuseclass}\end{sphinxVerbatimInput}
\begin{sphinxVerbatimOutput}

\begin{sphinxuseclass}{cell_output}
\begin{sphinxVerbatim}[commandchars=\\\{\}]
          one  two  three  four
Utah        8    9     10    11
New York   12   13     14    15
\end{sphinxVerbatim}

\end{sphinxuseclass}\end{sphinxVerbatimOutput}

\end{sphinxuseclass}
\sphinxAtStartPar
The \sphinxcode{\sphinxupquote{.drop()}} method accepts an \sphinxcode{\sphinxupquote{axis}} argument and the default is \sphinxcode{\sphinxupquote{axis=0}} to drop rows based on labels.
To drop columns, we use \sphinxcode{\sphinxupquote{axis=1}} or \sphinxcode{\sphinxupquote{axis='columns'}}.

\begin{sphinxuseclass}{cell}\begin{sphinxVerbatimInput}

\begin{sphinxuseclass}{cell_input}
\begin{sphinxVerbatim}[commandchars=\\\{\}]
\PYG{n}{data}\PYG{o}{.}\PYG{n}{drop}\PYG{p}{(}\PYG{l+s+s1}{\PYGZsq{}}\PYG{l+s+s1}{two}\PYG{l+s+s1}{\PYGZsq{}}\PYG{p}{,} \PYG{n}{axis}\PYG{o}{=}\PYG{l+m+mi}{1}\PYG{p}{)}
\end{sphinxVerbatim}

\end{sphinxuseclass}\end{sphinxVerbatimInput}
\begin{sphinxVerbatimOutput}

\begin{sphinxuseclass}{cell_output}
\begin{sphinxVerbatim}[commandchars=\\\{\}]
          one  three  four
Ohio        0      2     3
Colorado    4      6     7
Utah        8     10    11
New York   12     14    15
\end{sphinxVerbatim}

\end{sphinxuseclass}\end{sphinxVerbatimOutput}

\end{sphinxuseclass}
\begin{sphinxuseclass}{cell}\begin{sphinxVerbatimInput}

\begin{sphinxuseclass}{cell_input}
\begin{sphinxVerbatim}[commandchars=\\\{\}]
\PYG{n}{data}\PYG{o}{.}\PYG{n}{drop}\PYG{p}{(}\PYG{n}{columns}\PYG{o}{=}\PYG{l+s+s1}{\PYGZsq{}}\PYG{l+s+s1}{two}\PYG{l+s+s1}{\PYGZsq{}}\PYG{p}{)}
\end{sphinxVerbatim}

\end{sphinxuseclass}\end{sphinxVerbatimInput}
\begin{sphinxVerbatimOutput}

\begin{sphinxuseclass}{cell_output}
\begin{sphinxVerbatim}[commandchars=\\\{\}]
          one  three  four
Ohio        0      2     3
Colorado    4      6     7
Utah        8     10    11
New York   12     14    15
\end{sphinxVerbatim}

\end{sphinxuseclass}\end{sphinxVerbatimOutput}

\end{sphinxuseclass}

\subsection{Indexing, Selection, and Filtering}
\label{\detokenize{mckinney_05_lecture:indexing-selection-and-filtering}}
\sphinxAtStartPar
Indexing, selecting, and filtering will be among our most\sphinxhyphen{}used pandas features.
\begin{quote}

\sphinxAtStartPar
Series indexing (obj{[}…{]}) works analogously to NumPy array indexing, except you can use the Series’s index values instead of only integers.
\end{quote}

\begin{sphinxuseclass}{cell}\begin{sphinxVerbatimInput}

\begin{sphinxuseclass}{cell_input}
\begin{sphinxVerbatim}[commandchars=\\\{\}]
\PYG{n}{obj} \PYG{o}{=} \PYG{n}{pd}\PYG{o}{.}\PYG{n}{Series}\PYG{p}{(}\PYG{n}{np}\PYG{o}{.}\PYG{n}{arange}\PYG{p}{(}\PYG{l+m+mf}{4.}\PYG{p}{)}\PYG{p}{,} \PYG{n}{index}\PYG{o}{=}\PYG{p}{[}\PYG{l+s+s1}{\PYGZsq{}}\PYG{l+s+s1}{a}\PYG{l+s+s1}{\PYGZsq{}}\PYG{p}{,} \PYG{l+s+s1}{\PYGZsq{}}\PYG{l+s+s1}{b}\PYG{l+s+s1}{\PYGZsq{}}\PYG{p}{,} \PYG{l+s+s1}{\PYGZsq{}}\PYG{l+s+s1}{c}\PYG{l+s+s1}{\PYGZsq{}}\PYG{p}{,} \PYG{l+s+s1}{\PYGZsq{}}\PYG{l+s+s1}{d}\PYG{l+s+s1}{\PYGZsq{}}\PYG{p}{]}\PYG{p}{)}
\PYG{n}{obj}
\end{sphinxVerbatim}

\end{sphinxuseclass}\end{sphinxVerbatimInput}
\begin{sphinxVerbatimOutput}

\begin{sphinxuseclass}{cell_output}
\begin{sphinxVerbatim}[commandchars=\\\{\}]
a   0.0000
b   1.0000
c   2.0000
d   3.0000
dtype: float64
\end{sphinxVerbatim}

\end{sphinxuseclass}\end{sphinxVerbatimOutput}

\end{sphinxuseclass}
\begin{sphinxuseclass}{cell}\begin{sphinxVerbatimInput}

\begin{sphinxuseclass}{cell_input}
\begin{sphinxVerbatim}[commandchars=\\\{\}]
\PYG{n}{obj}\PYG{p}{[}\PYG{l+s+s1}{\PYGZsq{}}\PYG{l+s+s1}{b}\PYG{l+s+s1}{\PYGZsq{}}\PYG{p}{]}
\end{sphinxVerbatim}

\end{sphinxuseclass}\end{sphinxVerbatimInput}
\begin{sphinxVerbatimOutput}

\begin{sphinxuseclass}{cell_output}
\begin{sphinxVerbatim}[commandchars=\\\{\}]
1.0
\end{sphinxVerbatim}

\end{sphinxuseclass}\end{sphinxVerbatimOutput}

\end{sphinxuseclass}
\begin{sphinxuseclass}{cell}\begin{sphinxVerbatimInput}

\begin{sphinxuseclass}{cell_input}
\begin{sphinxVerbatim}[commandchars=\\\{\}]
\PYG{n}{obj}\PYG{p}{[}\PYG{l+m+mi}{1}\PYG{p}{]}
\end{sphinxVerbatim}

\end{sphinxuseclass}\end{sphinxVerbatimInput}
\begin{sphinxVerbatimOutput}

\begin{sphinxuseclass}{cell_output}
\begin{sphinxVerbatim}[commandchars=\\\{\}]
1.0
\end{sphinxVerbatim}

\end{sphinxuseclass}\end{sphinxVerbatimOutput}

\end{sphinxuseclass}
\sphinxAtStartPar
The code directly above works.
However, I prefer to be explicit and use \sphinxcode{\sphinxupquote{.iloc{[}{]}}} when I index or slice by integers.

\begin{sphinxuseclass}{cell}\begin{sphinxVerbatimInput}

\begin{sphinxuseclass}{cell_input}
\begin{sphinxVerbatim}[commandchars=\\\{\}]
\PYG{n}{obj}\PYG{o}{.}\PYG{n}{iloc}\PYG{p}{[}\PYG{l+m+mi}{1}\PYG{p}{]}
\end{sphinxVerbatim}

\end{sphinxuseclass}\end{sphinxVerbatimInput}
\begin{sphinxVerbatimOutput}

\begin{sphinxuseclass}{cell_output}
\begin{sphinxVerbatim}[commandchars=\\\{\}]
1.0
\end{sphinxVerbatim}

\end{sphinxuseclass}\end{sphinxVerbatimOutput}

\end{sphinxuseclass}
\begin{sphinxuseclass}{cell}\begin{sphinxVerbatimInput}

\begin{sphinxuseclass}{cell_input}
\begin{sphinxVerbatim}[commandchars=\\\{\}]
\PYG{n}{obj}\PYG{o}{.}\PYG{n}{iloc}\PYG{p}{[}\PYG{l+m+mi}{1}\PYG{p}{:}\PYG{l+m+mi}{3}\PYG{p}{]}
\end{sphinxVerbatim}

\end{sphinxuseclass}\end{sphinxVerbatimInput}
\begin{sphinxVerbatimOutput}

\begin{sphinxuseclass}{cell_output}
\begin{sphinxVerbatim}[commandchars=\\\{\}]
b   1.0000
c   2.0000
dtype: float64
\end{sphinxVerbatim}

\end{sphinxuseclass}\end{sphinxVerbatimOutput}

\end{sphinxuseclass}
\sphinxAtStartPar
\sphinxstyleemphasis{\sphinxstylestrong{When we slice with labels, the left and right endpoints are inclusive.}}

\begin{sphinxuseclass}{cell}\begin{sphinxVerbatimInput}

\begin{sphinxuseclass}{cell_input}
\begin{sphinxVerbatim}[commandchars=\\\{\}]
\PYG{n}{obj}\PYG{p}{[}\PYG{l+s+s1}{\PYGZsq{}}\PYG{l+s+s1}{b}\PYG{l+s+s1}{\PYGZsq{}}\PYG{p}{:}\PYG{l+s+s1}{\PYGZsq{}}\PYG{l+s+s1}{c}\PYG{l+s+s1}{\PYGZsq{}}\PYG{p}{]}
\end{sphinxVerbatim}

\end{sphinxuseclass}\end{sphinxVerbatimInput}
\begin{sphinxVerbatimOutput}

\begin{sphinxuseclass}{cell_output}
\begin{sphinxVerbatim}[commandchars=\\\{\}]
b   1.0000
c   2.0000
dtype: float64
\end{sphinxVerbatim}

\end{sphinxuseclass}\end{sphinxVerbatimOutput}

\end{sphinxuseclass}
\begin{sphinxuseclass}{cell}\begin{sphinxVerbatimInput}

\begin{sphinxuseclass}{cell_input}
\begin{sphinxVerbatim}[commandchars=\\\{\}]
\PYG{n}{obj}\PYG{p}{[}\PYG{l+s+s1}{\PYGZsq{}}\PYG{l+s+s1}{b}\PYG{l+s+s1}{\PYGZsq{}}\PYG{p}{:}\PYG{l+s+s1}{\PYGZsq{}}\PYG{l+s+s1}{c}\PYG{l+s+s1}{\PYGZsq{}}\PYG{p}{]} \PYG{o}{=} \PYG{l+m+mi}{5}
\PYG{n}{obj}
\end{sphinxVerbatim}

\end{sphinxuseclass}\end{sphinxVerbatimInput}
\begin{sphinxVerbatimOutput}

\begin{sphinxuseclass}{cell_output}
\begin{sphinxVerbatim}[commandchars=\\\{\}]
a   0.0000
b   5.0000
c   5.0000
d   3.0000
dtype: float64
\end{sphinxVerbatim}

\end{sphinxuseclass}\end{sphinxVerbatimOutput}

\end{sphinxuseclass}
\begin{sphinxuseclass}{cell}\begin{sphinxVerbatimInput}

\begin{sphinxuseclass}{cell_input}
\begin{sphinxVerbatim}[commandchars=\\\{\}]
\PYG{n}{data} \PYG{o}{=} \PYG{n}{pd}\PYG{o}{.}\PYG{n}{DataFrame}\PYG{p}{(}
    \PYG{n}{np}\PYG{o}{.}\PYG{n}{arange}\PYG{p}{(}\PYG{l+m+mi}{16}\PYG{p}{)}\PYG{o}{.}\PYG{n}{reshape}\PYG{p}{(}\PYG{p}{(}\PYG{l+m+mi}{4}\PYG{p}{,} \PYG{l+m+mi}{4}\PYG{p}{)}\PYG{p}{)}\PYG{p}{,}
    \PYG{n}{index}\PYG{o}{=}\PYG{p}{[}\PYG{l+s+s1}{\PYGZsq{}}\PYG{l+s+s1}{Ohio}\PYG{l+s+s1}{\PYGZsq{}}\PYG{p}{,} \PYG{l+s+s1}{\PYGZsq{}}\PYG{l+s+s1}{Colorado}\PYG{l+s+s1}{\PYGZsq{}}\PYG{p}{,} \PYG{l+s+s1}{\PYGZsq{}}\PYG{l+s+s1}{Utah}\PYG{l+s+s1}{\PYGZsq{}}\PYG{p}{,} \PYG{l+s+s1}{\PYGZsq{}}\PYG{l+s+s1}{New York}\PYG{l+s+s1}{\PYGZsq{}}\PYG{p}{]}\PYG{p}{,}
    \PYG{n}{columns}\PYG{o}{=}\PYG{p}{[}\PYG{l+s+s1}{\PYGZsq{}}\PYG{l+s+s1}{one}\PYG{l+s+s1}{\PYGZsq{}}\PYG{p}{,} \PYG{l+s+s1}{\PYGZsq{}}\PYG{l+s+s1}{two}\PYG{l+s+s1}{\PYGZsq{}}\PYG{p}{,} \PYG{l+s+s1}{\PYGZsq{}}\PYG{l+s+s1}{three}\PYG{l+s+s1}{\PYGZsq{}}\PYG{p}{,} \PYG{l+s+s1}{\PYGZsq{}}\PYG{l+s+s1}{four}\PYG{l+s+s1}{\PYGZsq{}}\PYG{p}{]}
\PYG{p}{)}

\PYG{n}{data}
\end{sphinxVerbatim}

\end{sphinxuseclass}\end{sphinxVerbatimInput}
\begin{sphinxVerbatimOutput}

\begin{sphinxuseclass}{cell_output}
\begin{sphinxVerbatim}[commandchars=\\\{\}]
          one  two  three  four
Ohio        0    1      2     3
Colorado    4    5      6     7
Utah        8    9     10    11
New York   12   13     14    15
\end{sphinxVerbatim}

\end{sphinxuseclass}\end{sphinxVerbatimOutput}

\end{sphinxuseclass}
\sphinxAtStartPar
Indexing one column returns a series.

\begin{sphinxuseclass}{cell}\begin{sphinxVerbatimInput}

\begin{sphinxuseclass}{cell_input}
\begin{sphinxVerbatim}[commandchars=\\\{\}]
\PYG{n}{data}\PYG{p}{[}\PYG{l+s+s1}{\PYGZsq{}}\PYG{l+s+s1}{two}\PYG{l+s+s1}{\PYGZsq{}}\PYG{p}{]}
\end{sphinxVerbatim}

\end{sphinxuseclass}\end{sphinxVerbatimInput}
\begin{sphinxVerbatimOutput}

\begin{sphinxuseclass}{cell_output}
\begin{sphinxVerbatim}[commandchars=\\\{\}]
Ohio         1
Colorado     5
Utah         9
New York    13
Name: two, dtype: int64
\end{sphinxVerbatim}

\end{sphinxuseclass}\end{sphinxVerbatimOutput}

\end{sphinxuseclass}
\sphinxAtStartPar
Indexing two or more columns returns a data frame.

\begin{sphinxuseclass}{cell}\begin{sphinxVerbatimInput}

\begin{sphinxuseclass}{cell_input}
\begin{sphinxVerbatim}[commandchars=\\\{\}]
\PYG{n}{data}\PYG{p}{[}\PYG{p}{[}\PYG{l+s+s1}{\PYGZsq{}}\PYG{l+s+s1}{three}\PYG{l+s+s1}{\PYGZsq{}}\PYG{p}{,} \PYG{l+s+s1}{\PYGZsq{}}\PYG{l+s+s1}{one}\PYG{l+s+s1}{\PYGZsq{}}\PYG{p}{]}\PYG{p}{]}
\end{sphinxVerbatim}

\end{sphinxuseclass}\end{sphinxVerbatimInput}
\begin{sphinxVerbatimOutput}

\begin{sphinxuseclass}{cell_output}
\begin{sphinxVerbatim}[commandchars=\\\{\}]
          three  one
Ohio          2    0
Colorado      6    4
Utah         10    8
New York     14   12
\end{sphinxVerbatim}

\end{sphinxuseclass}\end{sphinxVerbatimOutput}

\end{sphinxuseclass}
\sphinxAtStartPar
If we want a one\sphinxhyphen{}column data frame, we can use \sphinxcode{\sphinxupquote{{[}{[}{]}{]}}}:

\begin{sphinxuseclass}{cell}\begin{sphinxVerbatimInput}

\begin{sphinxuseclass}{cell_input}
\begin{sphinxVerbatim}[commandchars=\\\{\}]
\PYG{n}{data}\PYG{p}{[}\PYG{l+s+s1}{\PYGZsq{}}\PYG{l+s+s1}{three}\PYG{l+s+s1}{\PYGZsq{}}\PYG{p}{]}
\end{sphinxVerbatim}

\end{sphinxuseclass}\end{sphinxVerbatimInput}
\begin{sphinxVerbatimOutput}

\begin{sphinxuseclass}{cell_output}
\begin{sphinxVerbatim}[commandchars=\\\{\}]
Ohio         2
Colorado     6
Utah        10
New York    14
Name: three, dtype: int64
\end{sphinxVerbatim}

\end{sphinxuseclass}\end{sphinxVerbatimOutput}

\end{sphinxuseclass}
\begin{sphinxuseclass}{cell}\begin{sphinxVerbatimInput}

\begin{sphinxuseclass}{cell_input}
\begin{sphinxVerbatim}[commandchars=\\\{\}]
\PYG{n}{data}\PYG{p}{[}\PYG{p}{[}\PYG{l+s+s1}{\PYGZsq{}}\PYG{l+s+s1}{three}\PYG{l+s+s1}{\PYGZsq{}}\PYG{p}{]}\PYG{p}{]}
\end{sphinxVerbatim}

\end{sphinxuseclass}\end{sphinxVerbatimInput}
\begin{sphinxVerbatimOutput}

\begin{sphinxuseclass}{cell_output}
\begin{sphinxVerbatim}[commandchars=\\\{\}]
          three
Ohio          2
Colorado      6
Utah         10
New York     14
\end{sphinxVerbatim}

\end{sphinxuseclass}\end{sphinxVerbatimOutput}

\end{sphinxuseclass}
\sphinxAtStartPar
When we slice with integer indices with \sphinxcode{\sphinxupquote{{[}{]}}}, we slice rows.

\begin{sphinxuseclass}{cell}\begin{sphinxVerbatimInput}

\begin{sphinxuseclass}{cell_input}
\begin{sphinxVerbatim}[commandchars=\\\{\}]
\PYG{n}{data}\PYG{p}{[}\PYG{p}{:}\PYG{l+m+mi}{2}\PYG{p}{]}
\end{sphinxVerbatim}

\end{sphinxuseclass}\end{sphinxVerbatimInput}
\begin{sphinxVerbatimOutput}

\begin{sphinxuseclass}{cell_output}
\begin{sphinxVerbatim}[commandchars=\\\{\}]
          one  two  three  four
Ohio        0    1      2     3
Colorado    4    5      6     7
\end{sphinxVerbatim}

\end{sphinxuseclass}\end{sphinxVerbatimOutput}

\end{sphinxuseclass}
\sphinxAtStartPar
When I slice rows, I prefer to use \sphinxcode{\sphinxupquote{.loc{[}{]}}} or \sphinxcode{\sphinxupquote{.iloc{[}{]}}} to avoid confusion.

\begin{sphinxuseclass}{cell}\begin{sphinxVerbatimInput}

\begin{sphinxuseclass}{cell_input}
\begin{sphinxVerbatim}[commandchars=\\\{\}]
\PYG{n}{data}\PYG{o}{.}\PYG{n}{iloc}\PYG{p}{[}\PYG{p}{:}\PYG{l+m+mi}{2}\PYG{p}{]}
\end{sphinxVerbatim}

\end{sphinxuseclass}\end{sphinxVerbatimInput}
\begin{sphinxVerbatimOutput}

\begin{sphinxuseclass}{cell_output}
\begin{sphinxVerbatim}[commandchars=\\\{\}]
          one  two  three  four
Ohio        0    1      2     3
Colorado    4    5      6     7
\end{sphinxVerbatim}

\end{sphinxuseclass}\end{sphinxVerbatimOutput}

\end{sphinxuseclass}
\sphinxAtStartPar
We can index a data frame with Booleans, as we did with NumPy arrays.

\begin{sphinxuseclass}{cell}\begin{sphinxVerbatimInput}

\begin{sphinxuseclass}{cell_input}
\begin{sphinxVerbatim}[commandchars=\\\{\}]
\PYG{n}{data} \PYG{o}{\PYGZlt{}} \PYG{l+m+mi}{5}
\end{sphinxVerbatim}

\end{sphinxuseclass}\end{sphinxVerbatimInput}
\begin{sphinxVerbatimOutput}

\begin{sphinxuseclass}{cell_output}
\begin{sphinxVerbatim}[commandchars=\\\{\}]
            one    two  three   four
Ohio       True   True   True   True
Colorado   True  False  False  False
Utah      False  False  False  False
New York  False  False  False  False
\end{sphinxVerbatim}

\end{sphinxuseclass}\end{sphinxVerbatimOutput}

\end{sphinxuseclass}
\begin{sphinxuseclass}{cell}\begin{sphinxVerbatimInput}

\begin{sphinxuseclass}{cell_input}
\begin{sphinxVerbatim}[commandchars=\\\{\}]
\PYG{n}{data}\PYG{p}{[}\PYG{n}{data} \PYG{o}{\PYGZlt{}} \PYG{l+m+mi}{5}\PYG{p}{]} \PYG{o}{=} \PYG{l+m+mi}{0}
\PYG{n}{data}
\end{sphinxVerbatim}

\end{sphinxuseclass}\end{sphinxVerbatimInput}
\begin{sphinxVerbatimOutput}

\begin{sphinxuseclass}{cell_output}
\begin{sphinxVerbatim}[commandchars=\\\{\}]
          one  two  three  four
Ohio        0    0      0     0
Colorado    0    5      6     7
Utah        8    9     10    11
New York   12   13     14    15
\end{sphinxVerbatim}

\end{sphinxuseclass}\end{sphinxVerbatimOutput}

\end{sphinxuseclass}
\sphinxAtStartPar
Finally, we can chain slices.

\begin{sphinxuseclass}{cell}\begin{sphinxVerbatimInput}

\begin{sphinxuseclass}{cell_input}
\begin{sphinxVerbatim}[commandchars=\\\{\}]
\PYG{n}{data}\PYG{o}{.}\PYG{n}{iloc}\PYG{p}{[}\PYG{p}{:}\PYG{p}{,} \PYG{p}{:}\PYG{l+m+mi}{3}\PYG{p}{]}\PYG{p}{[}\PYG{n}{data}\PYG{o}{.}\PYG{n}{three} \PYG{o}{\PYGZgt{}} \PYG{l+m+mi}{5}\PYG{p}{]}
\end{sphinxVerbatim}

\end{sphinxuseclass}\end{sphinxVerbatimInput}
\begin{sphinxVerbatimOutput}

\begin{sphinxuseclass}{cell_output}
\begin{sphinxVerbatim}[commandchars=\\\{\}]
          one  two  three
Colorado    0    5      6
Utah        8    9     10
New York   12   13     14
\end{sphinxVerbatim}

\end{sphinxuseclass}\end{sphinxVerbatimOutput}

\end{sphinxuseclass}
\sphinxAtStartPar
\sphinxstyleemphasis{\sphinxstylestrong{Table 5\sphinxhyphen{}4}} summarizes data frame indexing and slicing options:
\begin{itemize}
\item {} 
\sphinxAtStartPar
\sphinxcode{\sphinxupquote{df{[}val{]}}}: Select single column or sequence of columns from the DataFrame; special case conveniences: boolean array (filter rows), slice (slice rows), or boolean DataFrame (set values based on some criterion)

\item {} 
\sphinxAtStartPar
\sphinxcode{\sphinxupquote{df.loc{[}val{]}}}: Selects single row or subset of rows from the DataFrame by label

\item {} 
\sphinxAtStartPar
\sphinxcode{\sphinxupquote{df.loc{[}:, val{]}}}: Selects single column or subset of columns by label

\item {} 
\sphinxAtStartPar
\sphinxcode{\sphinxupquote{df.loc{[}val1, val2{]}}}: Select both rows and columns by label

\item {} 
\sphinxAtStartPar
\sphinxcode{\sphinxupquote{df.iloc{[}where{]}}}: Selects single row or subset of rows from the DataFrame by integer position

\item {} 
\sphinxAtStartPar
\sphinxcode{\sphinxupquote{df.iloc{[}:, where{]}}}: Selects single column or subset of columns by integer position

\item {} 
\sphinxAtStartPar
\sphinxcode{\sphinxupquote{df.iloc{[}where\_i, where\_j{]}}}: Select both rows and columns by integer position

\item {} 
\sphinxAtStartPar
\sphinxcode{\sphinxupquote{df.at{[}label\_i, label\_j{]}}}: Select a single scalar value by row and column label

\item {} 
\sphinxAtStartPar
\sphinxcode{\sphinxupquote{df.iat{[}i, j{]}}}: Select a single scalar value by row and column position (integers) reindex method Select either rows or columns by labels

\item {} 
\sphinxAtStartPar
\sphinxcode{\sphinxupquote{get\_value}}, \sphinxcode{\sphinxupquote{set\_value}} methods: Select single value by row and column label

\end{itemize}

\sphinxAtStartPar
pandas is powerful and these options can be overwhelming!
We will typically use \sphinxcode{\sphinxupquote{df{[}val{]}}} to select columns (here \sphinxcode{\sphinxupquote{val}} is either a string or list of strings), \sphinxcode{\sphinxupquote{df.loc{[}val{]}}} to select rows (here \sphinxcode{\sphinxupquote{val}} is a row label), and \sphinxcode{\sphinxupquote{df.loc{[}val1, val2{]}}} to select both rows and columns.
The other options add flexibility, and we may occasionally use them.
However, our data will be large enough that counting row and column number will be tedious, making \sphinxcode{\sphinxupquote{.iloc{[}{]}}} impractical.


\subsection{Integer Indexes}
\label{\detokenize{mckinney_05_lecture:integer-indexes}}
\begin{sphinxuseclass}{cell}\begin{sphinxVerbatimInput}

\begin{sphinxuseclass}{cell_input}
\begin{sphinxVerbatim}[commandchars=\\\{\}]
\PYG{n}{ser} \PYG{o}{=} \PYG{n}{pd}\PYG{o}{.}\PYG{n}{Series}\PYG{p}{(}\PYG{n}{np}\PYG{o}{.}\PYG{n}{arange}\PYG{p}{(}\PYG{l+m+mf}{3.}\PYG{p}{)}\PYG{p}{)}
\PYG{n}{ser}
\end{sphinxVerbatim}

\end{sphinxuseclass}\end{sphinxVerbatimInput}
\begin{sphinxVerbatimOutput}

\begin{sphinxuseclass}{cell_output}
\begin{sphinxVerbatim}[commandchars=\\\{\}]
0   0.0000
1   1.0000
2   2.0000
dtype: float64
\end{sphinxVerbatim}

\end{sphinxuseclass}\end{sphinxVerbatimOutput}

\end{sphinxuseclass}
\sphinxAtStartPar
The following indexing yields an error because the series cannot fall back to NumPy array indexing.
Falling back to NumPy array indexing here would generate many subtle bugs elsewhere.

\begin{sphinxuseclass}{cell}\begin{sphinxVerbatimInput}

\begin{sphinxuseclass}{cell_input}
\begin{sphinxVerbatim}[commandchars=\\\{\}]
\PYG{c+c1}{\PYGZsh{} ser[\PYGZhy{}1]}
\end{sphinxVerbatim}

\end{sphinxuseclass}\end{sphinxVerbatimInput}

\end{sphinxuseclass}
\begin{sphinxuseclass}{cell}\begin{sphinxVerbatimInput}

\begin{sphinxuseclass}{cell_input}
\begin{sphinxVerbatim}[commandchars=\\\{\}]
\PYG{n}{ser}\PYG{o}{.}\PYG{n}{iloc}\PYG{p}{[}\PYG{o}{\PYGZhy{}}\PYG{l+m+mi}{1}\PYG{p}{]}
\end{sphinxVerbatim}

\end{sphinxuseclass}\end{sphinxVerbatimInput}
\begin{sphinxVerbatimOutput}

\begin{sphinxuseclass}{cell_output}
\begin{sphinxVerbatim}[commandchars=\\\{\}]
2.0
\end{sphinxVerbatim}

\end{sphinxuseclass}\end{sphinxVerbatimOutput}

\end{sphinxuseclass}
\sphinxAtStartPar
However, the following indexing works fine because with string labels there is no ambiguity.

\begin{sphinxuseclass}{cell}\begin{sphinxVerbatimInput}

\begin{sphinxuseclass}{cell_input}
\begin{sphinxVerbatim}[commandchars=\\\{\}]
\PYG{n}{ser2} \PYG{o}{=} \PYG{n}{pd}\PYG{o}{.}\PYG{n}{Series}\PYG{p}{(}\PYG{n}{np}\PYG{o}{.}\PYG{n}{arange}\PYG{p}{(}\PYG{l+m+mf}{3.}\PYG{p}{)}\PYG{p}{,} \PYG{n}{index}\PYG{o}{=}\PYG{p}{[}\PYG{l+s+s1}{\PYGZsq{}}\PYG{l+s+s1}{a}\PYG{l+s+s1}{\PYGZsq{}}\PYG{p}{,} \PYG{l+s+s1}{\PYGZsq{}}\PYG{l+s+s1}{b}\PYG{l+s+s1}{\PYGZsq{}}\PYG{p}{,} \PYG{l+s+s1}{\PYGZsq{}}\PYG{l+s+s1}{c}\PYG{l+s+s1}{\PYGZsq{}}\PYG{p}{]}\PYG{p}{)}
\PYG{n}{ser2}
\end{sphinxVerbatim}

\end{sphinxuseclass}\end{sphinxVerbatimInput}
\begin{sphinxVerbatimOutput}

\begin{sphinxuseclass}{cell_output}
\begin{sphinxVerbatim}[commandchars=\\\{\}]
a   0.0000
b   1.0000
c   2.0000
dtype: float64
\end{sphinxVerbatim}

\end{sphinxuseclass}\end{sphinxVerbatimOutput}

\end{sphinxuseclass}
\begin{sphinxuseclass}{cell}\begin{sphinxVerbatimInput}

\begin{sphinxuseclass}{cell_input}
\begin{sphinxVerbatim}[commandchars=\\\{\}]
\PYG{n}{ser2}\PYG{p}{[}\PYG{o}{\PYGZhy{}}\PYG{l+m+mi}{1}\PYG{p}{]}
\end{sphinxVerbatim}

\end{sphinxuseclass}\end{sphinxVerbatimInput}
\begin{sphinxVerbatimOutput}

\begin{sphinxuseclass}{cell_output}
\begin{sphinxVerbatim}[commandchars=\\\{\}]
2.0
\end{sphinxVerbatim}

\end{sphinxuseclass}\end{sphinxVerbatimOutput}

\end{sphinxuseclass}
\begin{sphinxuseclass}{cell}\begin{sphinxVerbatimInput}

\begin{sphinxuseclass}{cell_input}
\begin{sphinxVerbatim}[commandchars=\\\{\}]
\PYG{n}{ser2}\PYG{o}{.}\PYG{n}{iloc}\PYG{p}{[}\PYG{o}{\PYGZhy{}}\PYG{l+m+mi}{1}\PYG{p}{]}
\end{sphinxVerbatim}

\end{sphinxuseclass}\end{sphinxVerbatimInput}
\begin{sphinxVerbatimOutput}

\begin{sphinxuseclass}{cell_output}
\begin{sphinxVerbatim}[commandchars=\\\{\}]
2.0
\end{sphinxVerbatim}

\end{sphinxuseclass}\end{sphinxVerbatimOutput}

\end{sphinxuseclass}
\sphinxAtStartPar
In practice, these errors will not be an issue because we will index or slice with stock identifiers and dates instead of integers.
To avoid condusion, we should use \sphinxcode{\sphinxupquote{.iloc{[}{]}}} to index or slice with integers.


\subsection{Arithmetic and Data Alignment}
\label{\detokenize{mckinney_05_lecture:arithmetic-and-data-alignment}}\begin{quote}

\sphinxAtStartPar
An important pandas feature for some applications is the behavior of arithmetic between objects with different indexes. When you are adding together objects, if any index pairs are not the same, the respective index in the result will be the union of the index pairs. For users with database experience, this is similar to an automatic outer join on the index labels.
\end{quote}

\begin{sphinxuseclass}{cell}\begin{sphinxVerbatimInput}

\begin{sphinxuseclass}{cell_input}
\begin{sphinxVerbatim}[commandchars=\\\{\}]
\PYG{n}{s1} \PYG{o}{=} \PYG{n}{pd}\PYG{o}{.}\PYG{n}{Series}\PYG{p}{(}\PYG{p}{[}\PYG{l+m+mf}{7.3}\PYG{p}{,} \PYG{o}{\PYGZhy{}}\PYG{l+m+mf}{2.5}\PYG{p}{,} \PYG{l+m+mf}{3.4}\PYG{p}{,} \PYG{l+m+mf}{1.5}\PYG{p}{]}\PYG{p}{,} \PYG{n}{index}\PYG{o}{=}\PYG{p}{[}\PYG{l+s+s1}{\PYGZsq{}}\PYG{l+s+s1}{a}\PYG{l+s+s1}{\PYGZsq{}}\PYG{p}{,} \PYG{l+s+s1}{\PYGZsq{}}\PYG{l+s+s1}{c}\PYG{l+s+s1}{\PYGZsq{}}\PYG{p}{,} \PYG{l+s+s1}{\PYGZsq{}}\PYG{l+s+s1}{d}\PYG{l+s+s1}{\PYGZsq{}}\PYG{p}{,} \PYG{l+s+s1}{\PYGZsq{}}\PYG{l+s+s1}{e}\PYG{l+s+s1}{\PYGZsq{}}\PYG{p}{]}\PYG{p}{)}
\PYG{n}{s2} \PYG{o}{=} \PYG{n}{pd}\PYG{o}{.}\PYG{n}{Series}\PYG{p}{(}\PYG{p}{[}\PYG{o}{\PYGZhy{}}\PYG{l+m+mf}{2.1}\PYG{p}{,} \PYG{l+m+mf}{3.6}\PYG{p}{,} \PYG{o}{\PYGZhy{}}\PYG{l+m+mf}{1.5}\PYG{p}{,} \PYG{l+m+mi}{4}\PYG{p}{,} \PYG{l+m+mf}{3.1}\PYG{p}{]}\PYG{p}{,} \PYG{n}{index}\PYG{o}{=}\PYG{p}{[}\PYG{l+s+s1}{\PYGZsq{}}\PYG{l+s+s1}{a}\PYG{l+s+s1}{\PYGZsq{}}\PYG{p}{,} \PYG{l+s+s1}{\PYGZsq{}}\PYG{l+s+s1}{c}\PYG{l+s+s1}{\PYGZsq{}}\PYG{p}{,} \PYG{l+s+s1}{\PYGZsq{}}\PYG{l+s+s1}{e}\PYG{l+s+s1}{\PYGZsq{}}\PYG{p}{,} \PYG{l+s+s1}{\PYGZsq{}}\PYG{l+s+s1}{f}\PYG{l+s+s1}{\PYGZsq{}}\PYG{p}{,} \PYG{l+s+s1}{\PYGZsq{}}\PYG{l+s+s1}{g}\PYG{l+s+s1}{\PYGZsq{}}\PYG{p}{]}\PYG{p}{)}
\end{sphinxVerbatim}

\end{sphinxuseclass}\end{sphinxVerbatimInput}

\end{sphinxuseclass}
\begin{sphinxuseclass}{cell}\begin{sphinxVerbatimInput}

\begin{sphinxuseclass}{cell_input}
\begin{sphinxVerbatim}[commandchars=\\\{\}]
\PYG{n}{s1}
\end{sphinxVerbatim}

\end{sphinxuseclass}\end{sphinxVerbatimInput}
\begin{sphinxVerbatimOutput}

\begin{sphinxuseclass}{cell_output}
\begin{sphinxVerbatim}[commandchars=\\\{\}]
a    7.3000
c   \PYGZhy{}2.5000
d    3.4000
e    1.5000
dtype: float64
\end{sphinxVerbatim}

\end{sphinxuseclass}\end{sphinxVerbatimOutput}

\end{sphinxuseclass}
\begin{sphinxuseclass}{cell}\begin{sphinxVerbatimInput}

\begin{sphinxuseclass}{cell_input}
\begin{sphinxVerbatim}[commandchars=\\\{\}]
\PYG{n}{s2}
\end{sphinxVerbatim}

\end{sphinxuseclass}\end{sphinxVerbatimInput}
\begin{sphinxVerbatimOutput}

\begin{sphinxuseclass}{cell_output}
\begin{sphinxVerbatim}[commandchars=\\\{\}]
a   \PYGZhy{}2.1000
c    3.6000
e   \PYGZhy{}1.5000
f    4.0000
g    3.1000
dtype: float64
\end{sphinxVerbatim}

\end{sphinxuseclass}\end{sphinxVerbatimOutput}

\end{sphinxuseclass}
\begin{sphinxuseclass}{cell}\begin{sphinxVerbatimInput}

\begin{sphinxuseclass}{cell_input}
\begin{sphinxVerbatim}[commandchars=\\\{\}]
\PYG{n}{s1} \PYG{o}{+} \PYG{n}{s2}
\end{sphinxVerbatim}

\end{sphinxuseclass}\end{sphinxVerbatimInput}
\begin{sphinxVerbatimOutput}

\begin{sphinxuseclass}{cell_output}
\begin{sphinxVerbatim}[commandchars=\\\{\}]
a   5.2000
c   1.1000
d      NaN
e   0.0000
f      NaN
g      NaN
dtype: float64
\end{sphinxVerbatim}

\end{sphinxuseclass}\end{sphinxVerbatimOutput}

\end{sphinxuseclass}
\begin{sphinxuseclass}{cell}\begin{sphinxVerbatimInput}

\begin{sphinxuseclass}{cell_input}
\begin{sphinxVerbatim}[commandchars=\\\{\}]
\PYG{n}{df1} \PYG{o}{=} \PYG{n}{pd}\PYG{o}{.}\PYG{n}{DataFrame}\PYG{p}{(}\PYG{n}{np}\PYG{o}{.}\PYG{n}{arange}\PYG{p}{(}\PYG{l+m+mf}{9.}\PYG{p}{)}\PYG{o}{.}\PYG{n}{reshape}\PYG{p}{(}\PYG{p}{(}\PYG{l+m+mi}{3}\PYG{p}{,} \PYG{l+m+mi}{3}\PYG{p}{)}\PYG{p}{)}\PYG{p}{,} \PYG{n}{columns}\PYG{o}{=}\PYG{n+nb}{list}\PYG{p}{(}\PYG{l+s+s1}{\PYGZsq{}}\PYG{l+s+s1}{bcd}\PYG{l+s+s1}{\PYGZsq{}}\PYG{p}{)}\PYG{p}{,} \PYG{n}{index}\PYG{o}{=}\PYG{p}{[}\PYG{l+s+s1}{\PYGZsq{}}\PYG{l+s+s1}{Ohio}\PYG{l+s+s1}{\PYGZsq{}}\PYG{p}{,} \PYG{l+s+s1}{\PYGZsq{}}\PYG{l+s+s1}{Texas}\PYG{l+s+s1}{\PYGZsq{}}\PYG{p}{,} \PYG{l+s+s1}{\PYGZsq{}}\PYG{l+s+s1}{Colorado}\PYG{l+s+s1}{\PYGZsq{}}\PYG{p}{]}\PYG{p}{)}
\PYG{n}{df2} \PYG{o}{=} \PYG{n}{pd}\PYG{o}{.}\PYG{n}{DataFrame}\PYG{p}{(}\PYG{n}{np}\PYG{o}{.}\PYG{n}{arange}\PYG{p}{(}\PYG{l+m+mf}{12.}\PYG{p}{)}\PYG{o}{.}\PYG{n}{reshape}\PYG{p}{(}\PYG{p}{(}\PYG{l+m+mi}{4}\PYG{p}{,} \PYG{l+m+mi}{3}\PYG{p}{)}\PYG{p}{)}\PYG{p}{,} \PYG{n}{columns}\PYG{o}{=}\PYG{n+nb}{list}\PYG{p}{(}\PYG{l+s+s1}{\PYGZsq{}}\PYG{l+s+s1}{bde}\PYG{l+s+s1}{\PYGZsq{}}\PYG{p}{)}\PYG{p}{,} \PYG{n}{index}\PYG{o}{=}\PYG{p}{[}\PYG{l+s+s1}{\PYGZsq{}}\PYG{l+s+s1}{Utah}\PYG{l+s+s1}{\PYGZsq{}}\PYG{p}{,} \PYG{l+s+s1}{\PYGZsq{}}\PYG{l+s+s1}{Ohio}\PYG{l+s+s1}{\PYGZsq{}}\PYG{p}{,} \PYG{l+s+s1}{\PYGZsq{}}\PYG{l+s+s1}{Texas}\PYG{l+s+s1}{\PYGZsq{}}\PYG{p}{,} \PYG{l+s+s1}{\PYGZsq{}}\PYG{l+s+s1}{Oregon}\PYG{l+s+s1}{\PYGZsq{}}\PYG{p}{]}\PYG{p}{)}
\end{sphinxVerbatim}

\end{sphinxuseclass}\end{sphinxVerbatimInput}

\end{sphinxuseclass}
\begin{sphinxuseclass}{cell}\begin{sphinxVerbatimInput}

\begin{sphinxuseclass}{cell_input}
\begin{sphinxVerbatim}[commandchars=\\\{\}]
\PYG{n}{df1}
\end{sphinxVerbatim}

\end{sphinxuseclass}\end{sphinxVerbatimInput}
\begin{sphinxVerbatimOutput}

\begin{sphinxuseclass}{cell_output}
\begin{sphinxVerbatim}[commandchars=\\\{\}]
              b      c      d
Ohio     0.0000 1.0000 2.0000
Texas    3.0000 4.0000 5.0000
Colorado 6.0000 7.0000 8.0000
\end{sphinxVerbatim}

\end{sphinxuseclass}\end{sphinxVerbatimOutput}

\end{sphinxuseclass}
\begin{sphinxuseclass}{cell}\begin{sphinxVerbatimInput}

\begin{sphinxuseclass}{cell_input}
\begin{sphinxVerbatim}[commandchars=\\\{\}]
\PYG{n}{df2}
\end{sphinxVerbatim}

\end{sphinxuseclass}\end{sphinxVerbatimInput}
\begin{sphinxVerbatimOutput}

\begin{sphinxuseclass}{cell_output}
\begin{sphinxVerbatim}[commandchars=\\\{\}]
            b       d       e
Utah   0.0000  1.0000  2.0000
Ohio   3.0000  4.0000  5.0000
Texas  6.0000  7.0000  8.0000
Oregon 9.0000 10.0000 11.0000
\end{sphinxVerbatim}

\end{sphinxuseclass}\end{sphinxVerbatimOutput}

\end{sphinxuseclass}
\begin{sphinxuseclass}{cell}\begin{sphinxVerbatimInput}

\begin{sphinxuseclass}{cell_input}
\begin{sphinxVerbatim}[commandchars=\\\{\}]
\PYG{n}{df1} \PYG{o}{+} \PYG{n}{df2}
\end{sphinxVerbatim}

\end{sphinxuseclass}\end{sphinxVerbatimInput}
\begin{sphinxVerbatimOutput}

\begin{sphinxuseclass}{cell_output}
\begin{sphinxVerbatim}[commandchars=\\\{\}]
              b   c       d   e
Colorado    NaN NaN     NaN NaN
Ohio     3.0000 NaN  6.0000 NaN
Oregon      NaN NaN     NaN NaN
Texas    9.0000 NaN 12.0000 NaN
Utah        NaN NaN     NaN NaN
\end{sphinxVerbatim}

\end{sphinxuseclass}\end{sphinxVerbatimOutput}

\end{sphinxuseclass}
\begin{sphinxuseclass}{cell}\begin{sphinxVerbatimInput}

\begin{sphinxuseclass}{cell_input}
\begin{sphinxVerbatim}[commandchars=\\\{\}]
\PYG{n}{df1} \PYG{o}{=} \PYG{n}{pd}\PYG{o}{.}\PYG{n}{DataFrame}\PYG{p}{(}\PYG{p}{\PYGZob{}}\PYG{l+s+s1}{\PYGZsq{}}\PYG{l+s+s1}{A}\PYG{l+s+s1}{\PYGZsq{}}\PYG{p}{:} \PYG{p}{[}\PYG{l+m+mi}{1}\PYG{p}{,} \PYG{l+m+mi}{2}\PYG{p}{]}\PYG{p}{\PYGZcb{}}\PYG{p}{)}
\PYG{n}{df2} \PYG{o}{=} \PYG{n}{pd}\PYG{o}{.}\PYG{n}{DataFrame}\PYG{p}{(}\PYG{p}{\PYGZob{}}\PYG{l+s+s1}{\PYGZsq{}}\PYG{l+s+s1}{B}\PYG{l+s+s1}{\PYGZsq{}}\PYG{p}{:} \PYG{p}{[}\PYG{l+m+mi}{3}\PYG{p}{,} \PYG{l+m+mi}{4}\PYG{p}{]}\PYG{p}{\PYGZcb{}}\PYG{p}{)}
\end{sphinxVerbatim}

\end{sphinxuseclass}\end{sphinxVerbatimInput}

\end{sphinxuseclass}
\begin{sphinxuseclass}{cell}\begin{sphinxVerbatimInput}

\begin{sphinxuseclass}{cell_input}
\begin{sphinxVerbatim}[commandchars=\\\{\}]
\PYG{n}{df1}
\end{sphinxVerbatim}

\end{sphinxuseclass}\end{sphinxVerbatimInput}
\begin{sphinxVerbatimOutput}

\begin{sphinxuseclass}{cell_output}
\begin{sphinxVerbatim}[commandchars=\\\{\}]
   A
0  1
1  2
\end{sphinxVerbatim}

\end{sphinxuseclass}\end{sphinxVerbatimOutput}

\end{sphinxuseclass}
\begin{sphinxuseclass}{cell}\begin{sphinxVerbatimInput}

\begin{sphinxuseclass}{cell_input}
\begin{sphinxVerbatim}[commandchars=\\\{\}]
\PYG{n}{df2}
\end{sphinxVerbatim}

\end{sphinxuseclass}\end{sphinxVerbatimInput}
\begin{sphinxVerbatimOutput}

\begin{sphinxuseclass}{cell_output}
\begin{sphinxVerbatim}[commandchars=\\\{\}]
   B
0  3
1  4
\end{sphinxVerbatim}

\end{sphinxuseclass}\end{sphinxVerbatimOutput}

\end{sphinxuseclass}
\begin{sphinxuseclass}{cell}\begin{sphinxVerbatimInput}

\begin{sphinxuseclass}{cell_input}
\begin{sphinxVerbatim}[commandchars=\\\{\}]
\PYG{n}{df1} \PYG{o}{\PYGZhy{}} \PYG{n}{df2}
\end{sphinxVerbatim}

\end{sphinxuseclass}\end{sphinxVerbatimInput}
\begin{sphinxVerbatimOutput}

\begin{sphinxuseclass}{cell_output}
\begin{sphinxVerbatim}[commandchars=\\\{\}]
    A   B
0 NaN NaN
1 NaN NaN
\end{sphinxVerbatim}

\end{sphinxuseclass}\end{sphinxVerbatimOutput}

\end{sphinxuseclass}

\subsubsection{Arithmetic methods with fill values}
\label{\detokenize{mckinney_05_lecture:arithmetic-methods-with-fill-values}}
\begin{sphinxuseclass}{cell}\begin{sphinxVerbatimInput}

\begin{sphinxuseclass}{cell_input}
\begin{sphinxVerbatim}[commandchars=\\\{\}]
\PYG{n}{df1} \PYG{o}{=} \PYG{n}{pd}\PYG{o}{.}\PYG{n}{DataFrame}\PYG{p}{(}\PYG{n}{np}\PYG{o}{.}\PYG{n}{arange}\PYG{p}{(}\PYG{l+m+mf}{12.}\PYG{p}{)}\PYG{o}{.}\PYG{n}{reshape}\PYG{p}{(}\PYG{p}{(}\PYG{l+m+mi}{3}\PYG{p}{,} \PYG{l+m+mi}{4}\PYG{p}{)}\PYG{p}{)}\PYG{p}{,} \PYG{n}{columns}\PYG{o}{=}\PYG{n+nb}{list}\PYG{p}{(}\PYG{l+s+s1}{\PYGZsq{}}\PYG{l+s+s1}{abcd}\PYG{l+s+s1}{\PYGZsq{}}\PYG{p}{)}\PYG{p}{)}
\PYG{n}{df2} \PYG{o}{=} \PYG{n}{pd}\PYG{o}{.}\PYG{n}{DataFrame}\PYG{p}{(}\PYG{n}{np}\PYG{o}{.}\PYG{n}{arange}\PYG{p}{(}\PYG{l+m+mf}{20.}\PYG{p}{)}\PYG{o}{.}\PYG{n}{reshape}\PYG{p}{(}\PYG{p}{(}\PYG{l+m+mi}{4}\PYG{p}{,} \PYG{l+m+mi}{5}\PYG{p}{)}\PYG{p}{)}\PYG{p}{,} \PYG{n}{columns}\PYG{o}{=}\PYG{n+nb}{list}\PYG{p}{(}\PYG{l+s+s1}{\PYGZsq{}}\PYG{l+s+s1}{abcde}\PYG{l+s+s1}{\PYGZsq{}}\PYG{p}{)}\PYG{p}{)}
\PYG{n}{df2}\PYG{o}{.}\PYG{n}{loc}\PYG{p}{[}\PYG{l+m+mi}{1}\PYG{p}{,} \PYG{l+s+s1}{\PYGZsq{}}\PYG{l+s+s1}{b}\PYG{l+s+s1}{\PYGZsq{}}\PYG{p}{]} \PYG{o}{=} \PYG{n}{np}\PYG{o}{.}\PYG{n}{nan}
\end{sphinxVerbatim}

\end{sphinxuseclass}\end{sphinxVerbatimInput}

\end{sphinxuseclass}
\begin{sphinxuseclass}{cell}\begin{sphinxVerbatimInput}

\begin{sphinxuseclass}{cell_input}
\begin{sphinxVerbatim}[commandchars=\\\{\}]
\PYG{n}{df1}
\end{sphinxVerbatim}

\end{sphinxuseclass}\end{sphinxVerbatimInput}
\begin{sphinxVerbatimOutput}

\begin{sphinxuseclass}{cell_output}
\begin{sphinxVerbatim}[commandchars=\\\{\}]
       a      b       c       d
0 0.0000 1.0000  2.0000  3.0000
1 4.0000 5.0000  6.0000  7.0000
2 8.0000 9.0000 10.0000 11.0000
\end{sphinxVerbatim}

\end{sphinxuseclass}\end{sphinxVerbatimOutput}

\end{sphinxuseclass}
\begin{sphinxuseclass}{cell}\begin{sphinxVerbatimInput}

\begin{sphinxuseclass}{cell_input}
\begin{sphinxVerbatim}[commandchars=\\\{\}]
\PYG{n}{df2}
\end{sphinxVerbatim}

\end{sphinxuseclass}\end{sphinxVerbatimInput}
\begin{sphinxVerbatimOutput}

\begin{sphinxuseclass}{cell_output}
\begin{sphinxVerbatim}[commandchars=\\\{\}]
        a       b       c       d       e
0  0.0000  1.0000  2.0000  3.0000  4.0000
1  5.0000     NaN  7.0000  8.0000  9.0000
2 10.0000 11.0000 12.0000 13.0000 14.0000
3 15.0000 16.0000 17.0000 18.0000 19.0000
\end{sphinxVerbatim}

\end{sphinxuseclass}\end{sphinxVerbatimOutput}

\end{sphinxuseclass}
\begin{sphinxuseclass}{cell}\begin{sphinxVerbatimInput}

\begin{sphinxuseclass}{cell_input}
\begin{sphinxVerbatim}[commandchars=\\\{\}]
\PYG{n}{df1} \PYG{o}{+} \PYG{n}{df2}
\end{sphinxVerbatim}

\end{sphinxuseclass}\end{sphinxVerbatimInput}
\begin{sphinxVerbatimOutput}

\begin{sphinxuseclass}{cell_output}
\begin{sphinxVerbatim}[commandchars=\\\{\}]
        a       b       c       d   e
0  0.0000  2.0000  4.0000  6.0000 NaN
1  9.0000     NaN 13.0000 15.0000 NaN
2 18.0000 20.0000 22.0000 24.0000 NaN
3     NaN     NaN     NaN     NaN NaN
\end{sphinxVerbatim}

\end{sphinxuseclass}\end{sphinxVerbatimOutput}

\end{sphinxuseclass}
\sphinxAtStartPar
We can specify a fill value for \sphinxcode{\sphinxupquote{NaN}} values.
Note that pandas fills would\sphinxhyphen{}be \sphinxcode{\sphinxupquote{NaN}} values in each data frame \sphinxstyleemphasis{before} the arithmetic operation.

\begin{sphinxuseclass}{cell}\begin{sphinxVerbatimInput}

\begin{sphinxuseclass}{cell_input}
\begin{sphinxVerbatim}[commandchars=\\\{\}]
\PYG{n}{df1}\PYG{o}{.}\PYG{n}{add}\PYG{p}{(}\PYG{n}{df2}\PYG{p}{,} \PYG{n}{fill\PYGZus{}value}\PYG{o}{=}\PYG{l+m+mi}{0}\PYG{p}{)}
\end{sphinxVerbatim}

\end{sphinxuseclass}\end{sphinxVerbatimInput}
\begin{sphinxVerbatimOutput}

\begin{sphinxuseclass}{cell_output}
\begin{sphinxVerbatim}[commandchars=\\\{\}]
        a       b       c       d       e
0  0.0000  2.0000  4.0000  6.0000  4.0000
1  9.0000  5.0000 13.0000 15.0000  9.0000
2 18.0000 20.0000 22.0000 24.0000 14.0000
3 15.0000 16.0000 17.0000 18.0000 19.0000
\end{sphinxVerbatim}

\end{sphinxuseclass}\end{sphinxVerbatimOutput}

\end{sphinxuseclass}

\subsubsection{Operations between DataFrame and Series}
\label{\detokenize{mckinney_05_lecture:operations-between-dataframe-and-series}}
\begin{sphinxuseclass}{cell}\begin{sphinxVerbatimInput}

\begin{sphinxuseclass}{cell_input}
\begin{sphinxVerbatim}[commandchars=\\\{\}]
\PYG{n}{arr} \PYG{o}{=} \PYG{n}{np}\PYG{o}{.}\PYG{n}{arange}\PYG{p}{(}\PYG{l+m+mf}{12.}\PYG{p}{)}\PYG{o}{.}\PYG{n}{reshape}\PYG{p}{(}\PYG{p}{(}\PYG{l+m+mi}{3}\PYG{p}{,} \PYG{l+m+mi}{4}\PYG{p}{)}\PYG{p}{)}
\PYG{n}{arr}
\end{sphinxVerbatim}

\end{sphinxuseclass}\end{sphinxVerbatimInput}
\begin{sphinxVerbatimOutput}

\begin{sphinxuseclass}{cell_output}
\begin{sphinxVerbatim}[commandchars=\\\{\}]
array([[ 0.,  1.,  2.,  3.],
       [ 4.,  5.,  6.,  7.],
       [ 8.,  9., 10., 11.]])
\end{sphinxVerbatim}

\end{sphinxuseclass}\end{sphinxVerbatimOutput}

\end{sphinxuseclass}
\begin{sphinxuseclass}{cell}\begin{sphinxVerbatimInput}

\begin{sphinxuseclass}{cell_input}
\begin{sphinxVerbatim}[commandchars=\\\{\}]
\PYG{n}{arr}\PYG{p}{[}\PYG{l+m+mi}{0}\PYG{p}{]}
\end{sphinxVerbatim}

\end{sphinxuseclass}\end{sphinxVerbatimInput}
\begin{sphinxVerbatimOutput}

\begin{sphinxuseclass}{cell_output}
\begin{sphinxVerbatim}[commandchars=\\\{\}]
array([0., 1., 2., 3.])
\end{sphinxVerbatim}

\end{sphinxuseclass}\end{sphinxVerbatimOutput}

\end{sphinxuseclass}
\begin{sphinxuseclass}{cell}\begin{sphinxVerbatimInput}

\begin{sphinxuseclass}{cell_input}
\begin{sphinxVerbatim}[commandchars=\\\{\}]
\PYG{n}{arr} \PYG{o}{\PYGZhy{}} \PYG{n}{arr}\PYG{p}{[}\PYG{l+m+mi}{0}\PYG{p}{]}
\end{sphinxVerbatim}

\end{sphinxuseclass}\end{sphinxVerbatimInput}
\begin{sphinxVerbatimOutput}

\begin{sphinxuseclass}{cell_output}
\begin{sphinxVerbatim}[commandchars=\\\{\}]
array([[0., 0., 0., 0.],
       [4., 4., 4., 4.],
       [8., 8., 8., 8.]])
\end{sphinxVerbatim}

\end{sphinxuseclass}\end{sphinxVerbatimOutput}

\end{sphinxuseclass}
\sphinxAtStartPar
Arithmetic operations between series and data frames behave the same as the example above.

\begin{sphinxuseclass}{cell}\begin{sphinxVerbatimInput}

\begin{sphinxuseclass}{cell_input}
\begin{sphinxVerbatim}[commandchars=\\\{\}]
\PYG{n}{frame} \PYG{o}{=} \PYG{n}{pd}\PYG{o}{.}\PYG{n}{DataFrame}\PYG{p}{(}
    \PYG{n}{np}\PYG{o}{.}\PYG{n}{arange}\PYG{p}{(}\PYG{l+m+mf}{12.}\PYG{p}{)}\PYG{o}{.}\PYG{n}{reshape}\PYG{p}{(}\PYG{p}{(}\PYG{l+m+mi}{4}\PYG{p}{,} \PYG{l+m+mi}{3}\PYG{p}{)}\PYG{p}{)}\PYG{p}{,}
    \PYG{n}{columns}\PYG{o}{=}\PYG{n+nb}{list}\PYG{p}{(}\PYG{l+s+s1}{\PYGZsq{}}\PYG{l+s+s1}{bde}\PYG{l+s+s1}{\PYGZsq{}}\PYG{p}{)}\PYG{p}{,}
    \PYG{n}{index}\PYG{o}{=}\PYG{p}{[}\PYG{l+s+s1}{\PYGZsq{}}\PYG{l+s+s1}{Utah}\PYG{l+s+s1}{\PYGZsq{}}\PYG{p}{,} \PYG{l+s+s1}{\PYGZsq{}}\PYG{l+s+s1}{Ohio}\PYG{l+s+s1}{\PYGZsq{}}\PYG{p}{,} \PYG{l+s+s1}{\PYGZsq{}}\PYG{l+s+s1}{Texas}\PYG{l+s+s1}{\PYGZsq{}}\PYG{p}{,} \PYG{l+s+s1}{\PYGZsq{}}\PYG{l+s+s1}{Oregon}\PYG{l+s+s1}{\PYGZsq{}}\PYG{p}{]}
\PYG{p}{)}

\PYG{n}{series} \PYG{o}{=} \PYG{n}{frame}\PYG{o}{.}\PYG{n}{iloc}\PYG{p}{[}\PYG{l+m+mi}{0}\PYG{p}{]}
\end{sphinxVerbatim}

\end{sphinxuseclass}\end{sphinxVerbatimInput}

\end{sphinxuseclass}
\begin{sphinxuseclass}{cell}\begin{sphinxVerbatimInput}

\begin{sphinxuseclass}{cell_input}
\begin{sphinxVerbatim}[commandchars=\\\{\}]
\PYG{n}{frame}
\end{sphinxVerbatim}

\end{sphinxuseclass}\end{sphinxVerbatimInput}
\begin{sphinxVerbatimOutput}

\begin{sphinxuseclass}{cell_output}
\begin{sphinxVerbatim}[commandchars=\\\{\}]
            b       d       e
Utah   0.0000  1.0000  2.0000
Ohio   3.0000  4.0000  5.0000
Texas  6.0000  7.0000  8.0000
Oregon 9.0000 10.0000 11.0000
\end{sphinxVerbatim}

\end{sphinxuseclass}\end{sphinxVerbatimOutput}

\end{sphinxuseclass}
\begin{sphinxuseclass}{cell}\begin{sphinxVerbatimInput}

\begin{sphinxuseclass}{cell_input}
\begin{sphinxVerbatim}[commandchars=\\\{\}]
\PYG{n}{series}
\end{sphinxVerbatim}

\end{sphinxuseclass}\end{sphinxVerbatimInput}
\begin{sphinxVerbatimOutput}

\begin{sphinxuseclass}{cell_output}
\begin{sphinxVerbatim}[commandchars=\\\{\}]
b   0.0000
d   1.0000
e   2.0000
Name: Utah, dtype: float64
\end{sphinxVerbatim}

\end{sphinxuseclass}\end{sphinxVerbatimOutput}

\end{sphinxuseclass}
\begin{sphinxuseclass}{cell}\begin{sphinxVerbatimInput}

\begin{sphinxuseclass}{cell_input}
\begin{sphinxVerbatim}[commandchars=\\\{\}]
\PYG{n}{frame} \PYG{o}{\PYGZhy{}} \PYG{n}{series}
\end{sphinxVerbatim}

\end{sphinxuseclass}\end{sphinxVerbatimInput}
\begin{sphinxVerbatimOutput}

\begin{sphinxuseclass}{cell_output}
\begin{sphinxVerbatim}[commandchars=\\\{\}]
            b      d      e
Utah   0.0000 0.0000 0.0000
Ohio   3.0000 3.0000 3.0000
Texas  6.0000 6.0000 6.0000
Oregon 9.0000 9.0000 9.0000
\end{sphinxVerbatim}

\end{sphinxuseclass}\end{sphinxVerbatimOutput}

\end{sphinxuseclass}
\begin{sphinxuseclass}{cell}\begin{sphinxVerbatimInput}

\begin{sphinxuseclass}{cell_input}
\begin{sphinxVerbatim}[commandchars=\\\{\}]
\PYG{n}{series2} \PYG{o}{=} \PYG{n}{pd}\PYG{o}{.}\PYG{n}{Series}\PYG{p}{(}\PYG{n+nb}{range}\PYG{p}{(}\PYG{l+m+mi}{3}\PYG{p}{)}\PYG{p}{,} \PYG{n}{index}\PYG{o}{=}\PYG{p}{[}\PYG{l+s+s1}{\PYGZsq{}}\PYG{l+s+s1}{b}\PYG{l+s+s1}{\PYGZsq{}}\PYG{p}{,} \PYG{l+s+s1}{\PYGZsq{}}\PYG{l+s+s1}{e}\PYG{l+s+s1}{\PYGZsq{}}\PYG{p}{,} \PYG{l+s+s1}{\PYGZsq{}}\PYG{l+s+s1}{f}\PYG{l+s+s1}{\PYGZsq{}}\PYG{p}{]}\PYG{p}{)}
\end{sphinxVerbatim}

\end{sphinxuseclass}\end{sphinxVerbatimInput}

\end{sphinxuseclass}
\begin{sphinxuseclass}{cell}\begin{sphinxVerbatimInput}

\begin{sphinxuseclass}{cell_input}
\begin{sphinxVerbatim}[commandchars=\\\{\}]
\PYG{n}{frame}
\end{sphinxVerbatim}

\end{sphinxuseclass}\end{sphinxVerbatimInput}
\begin{sphinxVerbatimOutput}

\begin{sphinxuseclass}{cell_output}
\begin{sphinxVerbatim}[commandchars=\\\{\}]
            b       d       e
Utah   0.0000  1.0000  2.0000
Ohio   3.0000  4.0000  5.0000
Texas  6.0000  7.0000  8.0000
Oregon 9.0000 10.0000 11.0000
\end{sphinxVerbatim}

\end{sphinxuseclass}\end{sphinxVerbatimOutput}

\end{sphinxuseclass}
\begin{sphinxuseclass}{cell}\begin{sphinxVerbatimInput}

\begin{sphinxuseclass}{cell_input}
\begin{sphinxVerbatim}[commandchars=\\\{\}]
\PYG{n}{series2}
\end{sphinxVerbatim}

\end{sphinxuseclass}\end{sphinxVerbatimInput}
\begin{sphinxVerbatimOutput}

\begin{sphinxuseclass}{cell_output}
\begin{sphinxVerbatim}[commandchars=\\\{\}]
b    0
e    1
f    2
dtype: int64
\end{sphinxVerbatim}

\end{sphinxuseclass}\end{sphinxVerbatimOutput}

\end{sphinxuseclass}
\begin{sphinxuseclass}{cell}\begin{sphinxVerbatimInput}

\begin{sphinxuseclass}{cell_input}
\begin{sphinxVerbatim}[commandchars=\\\{\}]
\PYG{n}{frame} \PYG{o}{+} \PYG{n}{series2}
\end{sphinxVerbatim}

\end{sphinxuseclass}\end{sphinxVerbatimInput}
\begin{sphinxVerbatimOutput}

\begin{sphinxuseclass}{cell_output}
\begin{sphinxVerbatim}[commandchars=\\\{\}]
            b   d       e   f
Utah   0.0000 NaN  3.0000 NaN
Ohio   3.0000 NaN  6.0000 NaN
Texas  6.0000 NaN  9.0000 NaN
Oregon 9.0000 NaN 12.0000 NaN
\end{sphinxVerbatim}

\end{sphinxuseclass}\end{sphinxVerbatimOutput}

\end{sphinxuseclass}
\begin{sphinxuseclass}{cell}\begin{sphinxVerbatimInput}

\begin{sphinxuseclass}{cell_input}
\begin{sphinxVerbatim}[commandchars=\\\{\}]
\PYG{n}{series3} \PYG{o}{=} \PYG{n}{frame}\PYG{p}{[}\PYG{l+s+s1}{\PYGZsq{}}\PYG{l+s+s1}{d}\PYG{l+s+s1}{\PYGZsq{}}\PYG{p}{]}
\end{sphinxVerbatim}

\end{sphinxuseclass}\end{sphinxVerbatimInput}

\end{sphinxuseclass}
\begin{sphinxuseclass}{cell}\begin{sphinxVerbatimInput}

\begin{sphinxuseclass}{cell_input}
\begin{sphinxVerbatim}[commandchars=\\\{\}]
\PYG{n}{frame}\PYG{o}{.}\PYG{n}{sub}\PYG{p}{(}\PYG{n}{series3}\PYG{p}{,} \PYG{n}{axis}\PYG{o}{=}\PYG{l+s+s1}{\PYGZsq{}}\PYG{l+s+s1}{index}\PYG{l+s+s1}{\PYGZsq{}}\PYG{p}{)}
\end{sphinxVerbatim}

\end{sphinxuseclass}\end{sphinxVerbatimInput}
\begin{sphinxVerbatimOutput}

\begin{sphinxuseclass}{cell_output}
\begin{sphinxVerbatim}[commandchars=\\\{\}]
             b      d      e
Utah   \PYGZhy{}1.0000 0.0000 1.0000
Ohio   \PYGZhy{}1.0000 0.0000 1.0000
Texas  \PYGZhy{}1.0000 0.0000 1.0000
Oregon \PYGZhy{}1.0000 0.0000 1.0000
\end{sphinxVerbatim}

\end{sphinxuseclass}\end{sphinxVerbatimOutput}

\end{sphinxuseclass}

\subsection{Function Application and Mapping}
\label{\detokenize{mckinney_05_lecture:function-application-and-mapping}}
\begin{sphinxuseclass}{cell}\begin{sphinxVerbatimInput}

\begin{sphinxuseclass}{cell_input}
\begin{sphinxVerbatim}[commandchars=\\\{\}]
\PYG{n}{np}\PYG{o}{.}\PYG{n}{random}\PYG{o}{.}\PYG{n}{seed}\PYG{p}{(}\PYG{l+m+mi}{42}\PYG{p}{)}
\PYG{n}{frame} \PYG{o}{=} \PYG{n}{pd}\PYG{o}{.}\PYG{n}{DataFrame}\PYG{p}{(}
    \PYG{n}{np}\PYG{o}{.}\PYG{n}{random}\PYG{o}{.}\PYG{n}{randn}\PYG{p}{(}\PYG{l+m+mi}{4}\PYG{p}{,} \PYG{l+m+mi}{3}\PYG{p}{)}\PYG{p}{,} 
    \PYG{n}{columns}\PYG{o}{=}\PYG{n+nb}{list}\PYG{p}{(}\PYG{l+s+s1}{\PYGZsq{}}\PYG{l+s+s1}{bde}\PYG{l+s+s1}{\PYGZsq{}}\PYG{p}{)}\PYG{p}{,}
    \PYG{n}{index}\PYG{o}{=}\PYG{p}{[}\PYG{l+s+s1}{\PYGZsq{}}\PYG{l+s+s1}{Utah}\PYG{l+s+s1}{\PYGZsq{}}\PYG{p}{,} \PYG{l+s+s1}{\PYGZsq{}}\PYG{l+s+s1}{Ohio}\PYG{l+s+s1}{\PYGZsq{}}\PYG{p}{,} \PYG{l+s+s1}{\PYGZsq{}}\PYG{l+s+s1}{Texas}\PYG{l+s+s1}{\PYGZsq{}}\PYG{p}{,} \PYG{l+s+s1}{\PYGZsq{}}\PYG{l+s+s1}{Oregon}\PYG{l+s+s1}{\PYGZsq{}}\PYG{p}{]}
\PYG{p}{)}

\PYG{n}{frame}
\end{sphinxVerbatim}

\end{sphinxuseclass}\end{sphinxVerbatimInput}
\begin{sphinxVerbatimOutput}

\begin{sphinxuseclass}{cell_output}
\begin{sphinxVerbatim}[commandchars=\\\{\}]
            b       d       e
Utah   0.4967 \PYGZhy{}0.1383  0.6477
Ohio   1.5230 \PYGZhy{}0.2342 \PYGZhy{}0.2341
Texas  1.5792  0.7674 \PYGZhy{}0.4695
Oregon 0.5426 \PYGZhy{}0.4634 \PYGZhy{}0.4657
\end{sphinxVerbatim}

\end{sphinxuseclass}\end{sphinxVerbatimOutput}

\end{sphinxuseclass}
\begin{sphinxuseclass}{cell}\begin{sphinxVerbatimInput}

\begin{sphinxuseclass}{cell_input}
\begin{sphinxVerbatim}[commandchars=\\\{\}]
\PYG{n}{frame}\PYG{o}{.}\PYG{n}{abs}\PYG{p}{(}\PYG{p}{)}
\end{sphinxVerbatim}

\end{sphinxuseclass}\end{sphinxVerbatimInput}
\begin{sphinxVerbatimOutput}

\begin{sphinxuseclass}{cell_output}
\begin{sphinxVerbatim}[commandchars=\\\{\}]
            b      d      e
Utah   0.4967 0.1383 0.6477
Ohio   1.5230 0.2342 0.2341
Texas  1.5792 0.7674 0.4695
Oregon 0.5426 0.4634 0.4657
\end{sphinxVerbatim}

\end{sphinxuseclass}\end{sphinxVerbatimOutput}

\end{sphinxuseclass}\begin{quote}

\sphinxAtStartPar
Another frequent operation is applying a function on one\sphinxhyphen{}dimensional arrays to each column or row. DataFrame’s apply method does exactly this:
\end{quote}

\sphinxAtStartPar
Note that we can use anonymous (lambda) functions “on the fly”:

\begin{sphinxuseclass}{cell}\begin{sphinxVerbatimInput}

\begin{sphinxuseclass}{cell_input}
\begin{sphinxVerbatim}[commandchars=\\\{\}]
\PYG{n}{frame}\PYG{o}{.}\PYG{n}{apply}\PYG{p}{(}\PYG{k}{lambda} \PYG{n}{x}\PYG{p}{:} \PYG{n}{x}\PYG{o}{.}\PYG{n}{max}\PYG{p}{(}\PYG{p}{)} \PYG{o}{\PYGZhy{}} \PYG{n}{x}\PYG{o}{.}\PYG{n}{min}\PYG{p}{(}\PYG{p}{)}\PYG{p}{)}
\end{sphinxVerbatim}

\end{sphinxuseclass}\end{sphinxVerbatimInput}
\begin{sphinxVerbatimOutput}

\begin{sphinxuseclass}{cell_output}
\begin{sphinxVerbatim}[commandchars=\\\{\}]
b   1.0825
d   1.2309
e   1.1172
dtype: float64
\end{sphinxVerbatim}

\end{sphinxuseclass}\end{sphinxVerbatimOutput}

\end{sphinxuseclass}
\begin{sphinxuseclass}{cell}\begin{sphinxVerbatimInput}

\begin{sphinxuseclass}{cell_input}
\begin{sphinxVerbatim}[commandchars=\\\{\}]
\PYG{n}{frame}\PYG{o}{.}\PYG{n}{apply}\PYG{p}{(}\PYG{k}{lambda} \PYG{n}{x}\PYG{p}{:} \PYG{n}{x}\PYG{o}{.}\PYG{n}{max}\PYG{p}{(}\PYG{p}{)} \PYG{o}{\PYGZhy{}} \PYG{n}{x}\PYG{o}{.}\PYG{n}{min}\PYG{p}{(}\PYG{p}{)}\PYG{p}{,} \PYG{n}{axis}\PYG{o}{=}\PYG{l+m+mi}{1}\PYG{p}{)}
\end{sphinxVerbatim}

\end{sphinxuseclass}\end{sphinxVerbatimInput}
\begin{sphinxVerbatimOutput}

\begin{sphinxuseclass}{cell_output}
\begin{sphinxVerbatim}[commandchars=\\\{\}]
Utah     0.7860
Ohio     1.7572
Texas    2.0487
Oregon   1.0083
dtype: float64
\end{sphinxVerbatim}

\end{sphinxuseclass}\end{sphinxVerbatimOutput}

\end{sphinxuseclass}
\sphinxAtStartPar
However, under the hood, the \sphinxcode{\sphinxupquote{.apply()}} is basically a \sphinxcode{\sphinxupquote{for}} loop and much slowly than optimized, built\sphinxhyphen{}in methods.
Here is an example of the speed costs of \sphinxcode{\sphinxupquote{.apply()}}:

\begin{sphinxuseclass}{cell}\begin{sphinxVerbatimInput}

\begin{sphinxuseclass}{cell_input}
\begin{sphinxVerbatim}[commandchars=\\\{\}]
\PYG{o}{\PYGZpc{}}\PYG{k}{timeit} frame[\PYGZsq{}e\PYGZsq{}].abs()
\end{sphinxVerbatim}

\end{sphinxuseclass}\end{sphinxVerbatimInput}
\begin{sphinxVerbatimOutput}

\begin{sphinxuseclass}{cell_output}
\begin{sphinxVerbatim}[commandchars=\\\{\}]
9.13 µs ± 112 ns per loop (mean ± std. dev. of 7 runs, 100000 loops each)
\end{sphinxVerbatim}

\end{sphinxuseclass}\end{sphinxVerbatimOutput}

\end{sphinxuseclass}
\begin{sphinxuseclass}{cell}\begin{sphinxVerbatimInput}

\begin{sphinxuseclass}{cell_input}
\begin{sphinxVerbatim}[commandchars=\\\{\}]
\PYG{o}{\PYGZpc{}}\PYG{k}{timeit} frame[\PYGZsq{}e\PYGZsq{}].apply(np.abs)
\end{sphinxVerbatim}

\end{sphinxuseclass}\end{sphinxVerbatimInput}
\begin{sphinxVerbatimOutput}

\begin{sphinxuseclass}{cell_output}
\begin{sphinxVerbatim}[commandchars=\\\{\}]
26.3 µs ± 989 ns per loop (mean ± std. dev. of 7 runs, 10000 loops each)
\end{sphinxVerbatim}

\end{sphinxuseclass}\end{sphinxVerbatimOutput}

\end{sphinxuseclass}

\section{Summarizing and Computing Descriptive Statistics}
\label{\detokenize{mckinney_05_lecture:summarizing-and-computing-descriptive-statistics}}
\begin{sphinxuseclass}{cell}\begin{sphinxVerbatimInput}

\begin{sphinxuseclass}{cell_input}
\begin{sphinxVerbatim}[commandchars=\\\{\}]
\PYG{n}{df} \PYG{o}{=} \PYG{n}{pd}\PYG{o}{.}\PYG{n}{DataFrame}\PYG{p}{(}
    \PYG{p}{[}\PYG{p}{[}\PYG{l+m+mf}{1.4}\PYG{p}{,} \PYG{n}{np}\PYG{o}{.}\PYG{n}{nan}\PYG{p}{]}\PYG{p}{,} \PYG{p}{[}\PYG{l+m+mf}{7.1}\PYG{p}{,} \PYG{o}{\PYGZhy{}}\PYG{l+m+mf}{4.5}\PYG{p}{]}\PYG{p}{,} \PYG{p}{[}\PYG{n}{np}\PYG{o}{.}\PYG{n}{nan}\PYG{p}{,} \PYG{n}{np}\PYG{o}{.}\PYG{n}{nan}\PYG{p}{]}\PYG{p}{,} \PYG{p}{[}\PYG{l+m+mf}{0.75}\PYG{p}{,} \PYG{o}{\PYGZhy{}}\PYG{l+m+mf}{1.3}\PYG{p}{]}\PYG{p}{]}\PYG{p}{,}
    \PYG{n}{index}\PYG{o}{=}\PYG{p}{[}\PYG{l+s+s1}{\PYGZsq{}}\PYG{l+s+s1}{a}\PYG{l+s+s1}{\PYGZsq{}}\PYG{p}{,} \PYG{l+s+s1}{\PYGZsq{}}\PYG{l+s+s1}{b}\PYG{l+s+s1}{\PYGZsq{}}\PYG{p}{,} \PYG{l+s+s1}{\PYGZsq{}}\PYG{l+s+s1}{c}\PYG{l+s+s1}{\PYGZsq{}}\PYG{p}{,} \PYG{l+s+s1}{\PYGZsq{}}\PYG{l+s+s1}{d}\PYG{l+s+s1}{\PYGZsq{}}\PYG{p}{]}\PYG{p}{,}
    \PYG{n}{columns}\PYG{o}{=}\PYG{p}{[}\PYG{l+s+s1}{\PYGZsq{}}\PYG{l+s+s1}{one}\PYG{l+s+s1}{\PYGZsq{}}\PYG{p}{,} \PYG{l+s+s1}{\PYGZsq{}}\PYG{l+s+s1}{two}\PYG{l+s+s1}{\PYGZsq{}}\PYG{p}{]}
\PYG{p}{)}

\PYG{n}{df}
\end{sphinxVerbatim}

\end{sphinxuseclass}\end{sphinxVerbatimInput}
\begin{sphinxVerbatimOutput}

\begin{sphinxuseclass}{cell_output}
\begin{sphinxVerbatim}[commandchars=\\\{\}]
     one     two
a 1.4000     NaN
b 7.1000 \PYGZhy{}4.5000
c    NaN     NaN
d 0.7500 \PYGZhy{}1.3000
\end{sphinxVerbatim}

\end{sphinxuseclass}\end{sphinxVerbatimOutput}

\end{sphinxuseclass}
\begin{sphinxuseclass}{cell}\begin{sphinxVerbatimInput}

\begin{sphinxuseclass}{cell_input}
\begin{sphinxVerbatim}[commandchars=\\\{\}]
\PYG{n}{df}\PYG{o}{.}\PYG{n}{sum}\PYG{p}{(}\PYG{p}{)}
\end{sphinxVerbatim}

\end{sphinxuseclass}\end{sphinxVerbatimInput}
\begin{sphinxVerbatimOutput}

\begin{sphinxuseclass}{cell_output}
\begin{sphinxVerbatim}[commandchars=\\\{\}]
one    9.2500
two   \PYGZhy{}5.8000
dtype: float64
\end{sphinxVerbatim}

\end{sphinxuseclass}\end{sphinxVerbatimOutput}

\end{sphinxuseclass}
\begin{sphinxuseclass}{cell}\begin{sphinxVerbatimInput}

\begin{sphinxuseclass}{cell_input}
\begin{sphinxVerbatim}[commandchars=\\\{\}]
\PYG{n}{df}\PYG{o}{.}\PYG{n}{sum}\PYG{p}{(}\PYG{n}{axis}\PYG{o}{=}\PYG{l+m+mi}{1}\PYG{p}{)}
\end{sphinxVerbatim}

\end{sphinxuseclass}\end{sphinxVerbatimInput}
\begin{sphinxVerbatimOutput}

\begin{sphinxuseclass}{cell_output}
\begin{sphinxVerbatim}[commandchars=\\\{\}]
a    1.4000
b    2.6000
c    0.0000
d   \PYGZhy{}0.5500
dtype: float64
\end{sphinxVerbatim}

\end{sphinxuseclass}\end{sphinxVerbatimOutput}

\end{sphinxuseclass}
\begin{sphinxuseclass}{cell}\begin{sphinxVerbatimInput}

\begin{sphinxuseclass}{cell_input}
\begin{sphinxVerbatim}[commandchars=\\\{\}]
\PYG{n}{df}\PYG{o}{.}\PYG{n}{mean}\PYG{p}{(}\PYG{n}{axis}\PYG{o}{=}\PYG{l+m+mi}{1}\PYG{p}{,} \PYG{n}{skipna}\PYG{o}{=}\PYG{k+kc}{False}\PYG{p}{)}
\end{sphinxVerbatim}

\end{sphinxuseclass}\end{sphinxVerbatimInput}
\begin{sphinxVerbatimOutput}

\begin{sphinxuseclass}{cell_output}
\begin{sphinxVerbatim}[commandchars=\\\{\}]
a       NaN
b    1.3000
c       NaN
d   \PYGZhy{}0.2750
dtype: float64
\end{sphinxVerbatim}

\end{sphinxuseclass}\end{sphinxVerbatimOutput}

\end{sphinxuseclass}
\sphinxAtStartPar
The \sphinxcode{\sphinxupquote{.idxmax()}} method returns the label for the maximum observation.

\begin{sphinxuseclass}{cell}\begin{sphinxVerbatimInput}

\begin{sphinxuseclass}{cell_input}
\begin{sphinxVerbatim}[commandchars=\\\{\}]
\PYG{n}{df}\PYG{o}{.}\PYG{n}{idxmax}\PYG{p}{(}\PYG{p}{)}
\end{sphinxVerbatim}

\end{sphinxuseclass}\end{sphinxVerbatimInput}
\begin{sphinxVerbatimOutput}

\begin{sphinxuseclass}{cell_output}
\begin{sphinxVerbatim}[commandchars=\\\{\}]
one    b
two    d
dtype: object
\end{sphinxVerbatim}

\end{sphinxuseclass}\end{sphinxVerbatimOutput}

\end{sphinxuseclass}
\sphinxAtStartPar
The \sphinxcode{\sphinxupquote{.describe()}} returns summary statistics for each numerical column in a data frame.

\begin{sphinxuseclass}{cell}\begin{sphinxVerbatimInput}

\begin{sphinxuseclass}{cell_input}
\begin{sphinxVerbatim}[commandchars=\\\{\}]
\PYG{n}{df}\PYG{o}{.}\PYG{n}{describe}\PYG{p}{(}\PYG{p}{)}
\end{sphinxVerbatim}

\end{sphinxuseclass}\end{sphinxVerbatimInput}
\begin{sphinxVerbatimOutput}

\begin{sphinxuseclass}{cell_output}
\begin{sphinxVerbatim}[commandchars=\\\{\}]
         one     two
count 3.0000  2.0000
mean  3.0833 \PYGZhy{}2.9000
std   3.4937  2.2627
min   0.7500 \PYGZhy{}4.5000
25\PYGZpc{}   1.0750 \PYGZhy{}3.7000
50\PYGZpc{}   1.4000 \PYGZhy{}2.9000
75\PYGZpc{}   4.2500 \PYGZhy{}2.1000
max   7.1000 \PYGZhy{}1.3000
\end{sphinxVerbatim}

\end{sphinxuseclass}\end{sphinxVerbatimOutput}

\end{sphinxuseclass}
\sphinxAtStartPar
For non\sphinxhyphen{}numerical data, \sphinxcode{\sphinxupquote{.describe()}} returns alternative summary statistics.

\begin{sphinxuseclass}{cell}\begin{sphinxVerbatimInput}

\begin{sphinxuseclass}{cell_input}
\begin{sphinxVerbatim}[commandchars=\\\{\}]
\PYG{n}{obj} \PYG{o}{=} \PYG{n}{pd}\PYG{o}{.}\PYG{n}{Series}\PYG{p}{(}\PYG{p}{[}\PYG{l+s+s1}{\PYGZsq{}}\PYG{l+s+s1}{a}\PYG{l+s+s1}{\PYGZsq{}}\PYG{p}{,} \PYG{l+s+s1}{\PYGZsq{}}\PYG{l+s+s1}{a}\PYG{l+s+s1}{\PYGZsq{}}\PYG{p}{,} \PYG{l+s+s1}{\PYGZsq{}}\PYG{l+s+s1}{b}\PYG{l+s+s1}{\PYGZsq{}}\PYG{p}{,} \PYG{l+s+s1}{\PYGZsq{}}\PYG{l+s+s1}{c}\PYG{l+s+s1}{\PYGZsq{}}\PYG{p}{]} \PYG{o}{*} \PYG{l+m+mi}{4}\PYG{p}{)}
\PYG{n}{obj}\PYG{o}{.}\PYG{n}{describe}\PYG{p}{(}\PYG{p}{)}
\end{sphinxVerbatim}

\end{sphinxuseclass}\end{sphinxVerbatimInput}
\begin{sphinxVerbatimOutput}

\begin{sphinxuseclass}{cell_output}
\begin{sphinxVerbatim}[commandchars=\\\{\}]
count     16
unique     3
top        a
freq       8
dtype: object
\end{sphinxVerbatim}

\end{sphinxuseclass}\end{sphinxVerbatimOutput}

\end{sphinxuseclass}

\subsection{Correlation and Covariance}
\label{\detokenize{mckinney_05_lecture:correlation-and-covariance}}
\sphinxAtStartPar
To explore correlation and covariance methods, we can use Yahoo! Finance stock data.
We can use the yfinance package to import these data.
We can use the requests\sphinxhyphen{}cache package to cache our data requests, which avoid unnecessarily re\sphinxhyphen{}downloading data.

\sphinxAtStartPar
We can install these two functions with the \sphinxcode{\sphinxupquote{\%pip}} magic:

\begin{sphinxuseclass}{cell}\begin{sphinxVerbatimInput}

\begin{sphinxuseclass}{cell_input}
\begin{sphinxVerbatim}[commandchars=\\\{\}]
\PYG{c+c1}{\PYGZsh{} \PYGZpc{}pip install yfinance requests\PYGZhy{}cache}
\end{sphinxVerbatim}

\end{sphinxuseclass}\end{sphinxVerbatimInput}

\end{sphinxuseclass}
\sphinxAtStartPar
If we are running Python locally, we only need to run the \sphinxcode{\sphinxupquote{\%pip}} magic once.
If we are running Python in the cloud, we may need to run the \sphinxcode{\sphinxupquote{\%pip}} magic once \sphinxstyleemphasis{per login}.

\begin{sphinxuseclass}{cell}\begin{sphinxVerbatimInput}

\begin{sphinxuseclass}{cell_input}
\begin{sphinxVerbatim}[commandchars=\\\{\}]
\PYG{k+kn}{import} \PYG{n+nn}{yfinance} \PYG{k}{as} \PYG{n+nn}{yf}
\PYG{k+kn}{import} \PYG{n+nn}{requests\PYGZus{}cache}
\PYG{n}{session} \PYG{o}{=} \PYG{n}{requests\PYGZus{}cache}\PYG{o}{.}\PYG{n}{CachedSession}\PYG{p}{(}\PYG{n}{expire\PYGZus{}after}\PYG{o}{=}\PYG{l+s+s1}{\PYGZsq{}}\PYG{l+s+s1}{1D}\PYG{l+s+s1}{\PYGZsq{}}\PYG{p}{)}
\end{sphinxVerbatim}

\end{sphinxuseclass}\end{sphinxVerbatimInput}

\end{sphinxuseclass}
\begin{sphinxuseclass}{cell}\begin{sphinxVerbatimInput}

\begin{sphinxuseclass}{cell_input}
\begin{sphinxVerbatim}[commandchars=\\\{\}]
\PYG{n}{tickers} \PYG{o}{=} \PYG{n}{yf}\PYG{o}{.}\PYG{n}{Tickers}\PYG{p}{(}\PYG{l+s+s1}{\PYGZsq{}}\PYG{l+s+s1}{AAPL IBM MSFT GOOG}\PYG{l+s+s1}{\PYGZsq{}}\PYG{p}{,} \PYG{n}{session}\PYG{o}{=}\PYG{n}{session}\PYG{p}{)}
\end{sphinxVerbatim}

\end{sphinxuseclass}\end{sphinxVerbatimInput}

\end{sphinxuseclass}
\begin{sphinxuseclass}{cell}\begin{sphinxVerbatimInput}

\begin{sphinxuseclass}{cell_input}
\begin{sphinxVerbatim}[commandchars=\\\{\}]
\PYG{n}{prices} \PYG{o}{=} \PYG{n}{tickers}\PYG{o}{.}\PYG{n}{history}\PYG{p}{(}\PYG{n}{period}\PYG{o}{=}\PYG{l+s+s1}{\PYGZsq{}}\PYG{l+s+s1}{max}\PYG{l+s+s1}{\PYGZsq{}}\PYG{p}{,} \PYG{n}{auto\PYGZus{}adjust}\PYG{o}{=}\PYG{k+kc}{False}\PYG{p}{,} \PYG{n}{progress}\PYG{o}{=}\PYG{k+kc}{False}\PYG{p}{)}
\end{sphinxVerbatim}

\end{sphinxuseclass}\end{sphinxVerbatimInput}
\begin{sphinxVerbatimOutput}

\begin{sphinxuseclass}{cell_output}
\begin{sphinxVerbatim}[commandchars=\\\{\}]
[*********************100\PYGZpc{}***********************]  4 of 4 completed
\end{sphinxVerbatim}

\end{sphinxuseclass}\end{sphinxVerbatimOutput}

\end{sphinxuseclass}
\begin{sphinxuseclass}{cell}\begin{sphinxVerbatimInput}

\begin{sphinxuseclass}{cell_input}
\begin{sphinxVerbatim}[commandchars=\\\{\}]
\PYG{n}{prices}\PYG{o}{.}\PYG{n}{index} \PYG{o}{=} \PYG{n}{prices}\PYG{o}{.}\PYG{n}{index}\PYG{o}{.}\PYG{n}{tz\PYGZus{}localize}\PYG{p}{(}\PYG{k+kc}{None}\PYG{p}{)}
\end{sphinxVerbatim}

\end{sphinxuseclass}\end{sphinxVerbatimInput}

\end{sphinxuseclass}
\begin{sphinxuseclass}{cell}\begin{sphinxVerbatimInput}

\begin{sphinxuseclass}{cell_input}
\begin{sphinxVerbatim}[commandchars=\\\{\}]
\PYG{n}{prices}\PYG{p}{[}\PYG{l+s+s1}{\PYGZsq{}}\PYG{l+s+s1}{Adj Close}\PYG{l+s+s1}{\PYGZsq{}}\PYG{p}{]}
\end{sphinxVerbatim}

\end{sphinxuseclass}\end{sphinxVerbatimInput}
\begin{sphinxVerbatimOutput}

\begin{sphinxuseclass}{cell_output}
\begin{sphinxVerbatim}[commandchars=\\\{\}]
               AAPL     GOOG      IBM     MSFT
Date                                          
1962\PYGZhy{}01\PYGZhy{}02      NaN      NaN   1.6330      NaN
1962\PYGZhy{}01\PYGZhy{}03      NaN      NaN   1.6473      NaN
1962\PYGZhy{}01\PYGZhy{}04      NaN      NaN   1.6309      NaN
1962\PYGZhy{}01\PYGZhy{}05      NaN      NaN   1.5988      NaN
1962\PYGZhy{}01\PYGZhy{}08      NaN      NaN   1.5688      NaN
...             ...      ...      ...      ...
2023\PYGZhy{}01\PYGZhy{}20 137.8700  99.2800 141.2000 240.2200
2023\PYGZhy{}01\PYGZhy{}23 141.1100 101.2100 141.8600 242.5800
2023\PYGZhy{}01\PYGZhy{}24 142.5300  99.2100 141.4900 242.0400
2023\PYGZhy{}01\PYGZhy{}25 141.8600  96.7300 140.7600 240.6100
2023\PYGZhy{}01\PYGZhy{}26 143.6300  98.1800 135.1350 245.1400

[15373 rows x 4 columns]
\end{sphinxVerbatim}

\end{sphinxuseclass}\end{sphinxVerbatimOutput}

\end{sphinxuseclass}
\sphinxAtStartPar
The \sphinxcode{\sphinxupquote{prices}} data frames contains daily data for AAPL, IBM, MSFT, and GOOG.
The \sphinxcode{\sphinxupquote{Adj Close}} column provides a reverse\sphinxhyphen{}engineered daily closing price that accounts for dividends paid and stock splits (and reverse splits).
As a result, the \sphinxcode{\sphinxupquote{.pct\_change()}} in \sphinxcode{\sphinxupquote{Adj Close}} considers both price changes (i.e., capital gains) and dividends, so \$R\_t = \textbackslash{}frac\{(P\_t + D\_t) \sphinxhyphen{} P\_\{t\sphinxhyphen{}1\}\}\{P\_\{t\sphinxhyphen{}1\}\} = \textbackslash{}frac\{\textbackslash{}text\{Adj Close\}\sphinxstyleemphasis{t \sphinxhyphen{} \textbackslash{}text\{Adj Close\}}\{t\sphinxhyphen{}1\}\}\{\textbackslash{}text\{Adj Close\}\_\{t\sphinxhyphen{}1\}\}.\$

\begin{sphinxuseclass}{cell}\begin{sphinxVerbatimInput}

\begin{sphinxuseclass}{cell_input}
\begin{sphinxVerbatim}[commandchars=\\\{\}]
\PYG{n}{returns} \PYG{o}{=} \PYG{n}{prices}\PYG{p}{[}\PYG{l+s+s1}{\PYGZsq{}}\PYG{l+s+s1}{Adj Close}\PYG{l+s+s1}{\PYGZsq{}}\PYG{p}{]}\PYG{o}{.}\PYG{n}{pct\PYGZus{}change}\PYG{p}{(}\PYG{p}{)}\PYG{o}{.}\PYG{n}{dropna}\PYG{p}{(}\PYG{p}{)}
\PYG{n}{returns}
\end{sphinxVerbatim}

\end{sphinxuseclass}\end{sphinxVerbatimInput}
\begin{sphinxVerbatimOutput}

\begin{sphinxuseclass}{cell_output}
\begin{sphinxVerbatim}[commandchars=\\\{\}]
              AAPL    GOOG     IBM    MSFT
Date                                      
2004\PYGZhy{}08\PYGZhy{}20  0.0029  0.0794  0.0042  0.0029
2004\PYGZhy{}08\PYGZhy{}23  0.0091  0.0101 \PYGZhy{}0.0070  0.0044
2004\PYGZhy{}08\PYGZhy{}24  0.0280 \PYGZhy{}0.0414  0.0007  0.0000
2004\PYGZhy{}08\PYGZhy{}25  0.0344  0.0108  0.0042  0.0114
2004\PYGZhy{}08\PYGZhy{}26  0.0487  0.0180 \PYGZhy{}0.0045 \PYGZhy{}0.0040
...            ...     ...     ...     ...
2023\PYGZhy{}01\PYGZhy{}20  0.0192  0.0572  0.0041  0.0357
2023\PYGZhy{}01\PYGZhy{}23  0.0235  0.0194  0.0047  0.0098
2023\PYGZhy{}01\PYGZhy{}24  0.0101 \PYGZhy{}0.0198 \PYGZhy{}0.0026 \PYGZhy{}0.0022
2023\PYGZhy{}01\PYGZhy{}25 \PYGZhy{}0.0047 \PYGZhy{}0.0250 \PYGZhy{}0.0052 \PYGZhy{}0.0059
2023\PYGZhy{}01\PYGZhy{}26  0.0125  0.0150 \PYGZhy{}0.0400  0.0188

[4641 rows x 4 columns]
\end{sphinxVerbatim}

\end{sphinxuseclass}\end{sphinxVerbatimOutput}

\end{sphinxuseclass}
\sphinxAtStartPar
We multiply by 252 to annualize mean daily returns because means grow linearly with time and there are (about) 252 trading days per year.

\begin{sphinxuseclass}{cell}\begin{sphinxVerbatimInput}

\begin{sphinxuseclass}{cell_input}
\begin{sphinxVerbatim}[commandchars=\\\{\}]
\PYG{n}{returns}\PYG{o}{.}\PYG{n}{mean}\PYG{p}{(}\PYG{p}{)}\PYG{o}{.}\PYG{n}{mul}\PYG{p}{(}\PYG{l+m+mi}{252}\PYG{p}{)}
\end{sphinxVerbatim}

\end{sphinxuseclass}\end{sphinxVerbatimInput}
\begin{sphinxVerbatimOutput}

\begin{sphinxuseclass}{cell_output}
\begin{sphinxVerbatim}[commandchars=\\\{\}]
AAPL   0.3667
GOOG   0.2463
IBM    0.0822
MSFT   0.1820
dtype: float64
\end{sphinxVerbatim}

\end{sphinxuseclass}\end{sphinxVerbatimOutput}

\end{sphinxuseclass}
\sphinxAtStartPar
We multiply by \$\textbackslash{}sqrt\{252\}\$ to annualize the standard deviation of daily returns because variances grow linearly with time, there are (about) 252 trading days per year, and the standard deviation is the square root of the variance.

\begin{sphinxuseclass}{cell}\begin{sphinxVerbatimInput}

\begin{sphinxuseclass}{cell_input}
\begin{sphinxVerbatim}[commandchars=\\\{\}]
\PYG{n}{returns}\PYG{o}{.}\PYG{n}{std}\PYG{p}{(}\PYG{p}{)}\PYG{o}{.}\PYG{n}{mul}\PYG{p}{(}\PYG{n}{np}\PYG{o}{.}\PYG{n}{sqrt}\PYG{p}{(}\PYG{l+m+mi}{252}\PYG{p}{)}\PYG{p}{)}
\end{sphinxVerbatim}

\end{sphinxuseclass}\end{sphinxVerbatimInput}
\begin{sphinxVerbatimOutput}

\begin{sphinxuseclass}{cell_output}
\begin{sphinxVerbatim}[commandchars=\\\{\}]
AAPL   0.3332
GOOG   0.3077
IBM    0.2294
MSFT   0.2738
dtype: float64
\end{sphinxVerbatim}

\end{sphinxuseclass}\end{sphinxVerbatimOutput}

\end{sphinxuseclass}
\sphinxAtStartPar
\sphinxstyleemphasis{\sphinxstylestrong{The best explanation I have found on why stock return volatility (the standard deviation of stocks returns) grows with the square root of time is at the bottom of page 7 of \sphinxhref{https://book.ivo-welch.info/read/source5.mba/08-invchoice.pdf}{chapter 8 of Ivo Welch’s free corporate finance textbook}.}}

\sphinxAtStartPar
We can calculate pairwise correlations.

\begin{sphinxuseclass}{cell}\begin{sphinxVerbatimInput}

\begin{sphinxuseclass}{cell_input}
\begin{sphinxVerbatim}[commandchars=\\\{\}]
\PYG{n}{returns}\PYG{p}{[}\PYG{l+s+s1}{\PYGZsq{}}\PYG{l+s+s1}{MSFT}\PYG{l+s+s1}{\PYGZsq{}}\PYG{p}{]}\PYG{o}{.}\PYG{n}{corr}\PYG{p}{(}\PYG{n}{returns}\PYG{p}{[}\PYG{l+s+s1}{\PYGZsq{}}\PYG{l+s+s1}{IBM}\PYG{l+s+s1}{\PYGZsq{}}\PYG{p}{]}\PYG{p}{)}
\end{sphinxVerbatim}

\end{sphinxuseclass}\end{sphinxVerbatimInput}
\begin{sphinxVerbatimOutput}

\begin{sphinxuseclass}{cell_output}
\begin{sphinxVerbatim}[commandchars=\\\{\}]
0.5076061163270416
\end{sphinxVerbatim}

\end{sphinxuseclass}\end{sphinxVerbatimOutput}

\end{sphinxuseclass}
\sphinxAtStartPar
We can also calculate correlation matrices.

\begin{sphinxuseclass}{cell}\begin{sphinxVerbatimInput}

\begin{sphinxuseclass}{cell_input}
\begin{sphinxVerbatim}[commandchars=\\\{\}]
\PYG{n}{returns}\PYG{o}{.}\PYG{n}{corr}\PYG{p}{(}\PYG{p}{)}
\end{sphinxVerbatim}

\end{sphinxuseclass}\end{sphinxVerbatimInput}
\begin{sphinxVerbatimOutput}

\begin{sphinxuseclass}{cell_output}
\begin{sphinxVerbatim}[commandchars=\\\{\}]
       AAPL   GOOG    IBM   MSFT
AAPL 1.0000 0.5204 0.4345 0.5237
GOOG 0.5204 1.0000 0.4068 0.5650
IBM  0.4345 0.4068 1.0000 0.5076
MSFT 0.5237 0.5650 0.5076 1.0000
\end{sphinxVerbatim}

\end{sphinxuseclass}\end{sphinxVerbatimOutput}

\end{sphinxuseclass}
\begin{sphinxuseclass}{cell}\begin{sphinxVerbatimInput}

\begin{sphinxuseclass}{cell_input}
\begin{sphinxVerbatim}[commandchars=\\\{\}]
\PYG{n}{returns}\PYG{o}{.}\PYG{n}{corr}\PYG{p}{(}\PYG{p}{)}\PYG{o}{.}\PYG{n}{loc}\PYG{p}{[}\PYG{l+s+s1}{\PYGZsq{}}\PYG{l+s+s1}{MSFT}\PYG{l+s+s1}{\PYGZsq{}}\PYG{p}{,} \PYG{l+s+s1}{\PYGZsq{}}\PYG{l+s+s1}{IBM}\PYG{l+s+s1}{\PYGZsq{}}\PYG{p}{]}
\end{sphinxVerbatim}

\end{sphinxuseclass}\end{sphinxVerbatimInput}
\begin{sphinxVerbatimOutput}

\begin{sphinxuseclass}{cell_output}
\begin{sphinxVerbatim}[commandchars=\\\{\}]
0.5076061163270453
\end{sphinxVerbatim}

\end{sphinxuseclass}\end{sphinxVerbatimOutput}

\end{sphinxuseclass}
\sphinxstepscope


\section{McKinney Chapter 5 \sphinxhyphen{} Practice (Blank)}
\label{\detokenize{mckinney_05_practice:mckinney-chapter-5-practice-blank}}\label{\detokenize{mckinney_05_practice::doc}}
\begin{sphinxuseclass}{cell}\begin{sphinxVerbatimInput}

\begin{sphinxuseclass}{cell_input}
\begin{sphinxVerbatim}[commandchars=\\\{\}]
\PYG{k+kn}{import} \PYG{n+nn}{matplotlib}\PYG{n+nn}{.}\PYG{n+nn}{pyplot} \PYG{k}{as} \PYG{n+nn}{plt}
\PYG{k+kn}{import} \PYG{n+nn}{numpy} \PYG{k}{as} \PYG{n+nn}{np}
\PYG{k+kn}{import} \PYG{n+nn}{pandas} \PYG{k}{as} \PYG{n+nn}{pd}

\PYG{k+kn}{import} \PYG{n+nn}{yfinance} \PYG{k}{as} \PYG{n+nn}{yf}
\PYG{k+kn}{import} \PYG{n+nn}{requests\PYGZus{}cache}
\end{sphinxVerbatim}

\end{sphinxuseclass}\end{sphinxVerbatimInput}

\end{sphinxuseclass}
\begin{sphinxuseclass}{cell}\begin{sphinxVerbatimInput}

\begin{sphinxuseclass}{cell_input}
\begin{sphinxVerbatim}[commandchars=\\\{\}]
\PYG{o}{\PYGZpc{}}\PYG{k}{config} InlineBackend.figure\PYGZus{}format = \PYGZsq{}retina\PYGZsq{}
\PYG{o}{\PYGZpc{}}\PYG{k}{precision} 4
\PYG{n}{pd}\PYG{o}{.}\PYG{n}{options}\PYG{o}{.}\PYG{n}{display}\PYG{o}{.}\PYG{n}{float\PYGZus{}format} \PYG{o}{=} \PYG{l+s+s1}{\PYGZsq{}}\PYG{l+s+si}{\PYGZob{}:.4f\PYGZcb{}}\PYG{l+s+s1}{\PYGZsq{}}\PYG{o}{.}\PYG{n}{format}
\PYG{n}{session} \PYG{o}{=} \PYG{n}{requests\PYGZus{}cache}\PYG{o}{.}\PYG{n}{CachedSession}\PYG{p}{(}\PYG{p}{)}
\end{sphinxVerbatim}

\end{sphinxuseclass}\end{sphinxVerbatimInput}

\end{sphinxuseclass}
\begin{sphinxuseclass}{cell}\begin{sphinxVerbatimInput}

\begin{sphinxuseclass}{cell_input}
\begin{sphinxVerbatim}[commandchars=\\\{\}]
\PYG{n}{tickers} \PYG{o}{=} \PYG{n}{yf}\PYG{o}{.}\PYG{n}{Tickers}\PYG{p}{(}\PYG{l+s+s1}{\PYGZsq{}}\PYG{l+s+s1}{AAPL IBM MSFT GOOG}\PYG{l+s+s1}{\PYGZsq{}}\PYG{p}{,} \PYG{n}{session}\PYG{o}{=}\PYG{n}{session}\PYG{p}{)}
\PYG{n}{prices} \PYG{o}{=} \PYG{n}{tickers}\PYG{o}{.}\PYG{n}{history}\PYG{p}{(}\PYG{n}{period}\PYG{o}{=}\PYG{l+s+s1}{\PYGZsq{}}\PYG{l+s+s1}{max}\PYG{l+s+s1}{\PYGZsq{}}\PYG{p}{,} \PYG{n}{auto\PYGZus{}adjust}\PYG{o}{=}\PYG{k+kc}{False}\PYG{p}{,} \PYG{n}{progress}\PYG{o}{=}\PYG{k+kc}{False}\PYG{p}{)}
\PYG{n}{prices}\PYG{o}{.}\PYG{n}{index} \PYG{o}{=} \PYG{n}{prices}\PYG{o}{.}\PYG{n}{index}\PYG{o}{.}\PYG{n}{tz\PYGZus{}localize}\PYG{p}{(}\PYG{k+kc}{None}\PYG{p}{)}
\PYG{n}{returns} \PYG{o}{=} \PYG{n}{prices}\PYG{p}{[}\PYG{l+s+s1}{\PYGZsq{}}\PYG{l+s+s1}{Adj Close}\PYG{l+s+s1}{\PYGZsq{}}\PYG{p}{]}\PYG{o}{.}\PYG{n}{pct\PYGZus{}change}\PYG{p}{(}\PYG{p}{)}\PYG{o}{.}\PYG{n}{dropna}\PYG{p}{(}\PYG{p}{)}
\PYG{n}{returns}
\end{sphinxVerbatim}

\end{sphinxuseclass}\end{sphinxVerbatimInput}
\begin{sphinxVerbatimOutput}

\begin{sphinxuseclass}{cell_output}
\begin{sphinxVerbatim}[commandchars=\\\{\}]
[*********************100\PYGZpc{}***********************]  4 of 4 completed
\end{sphinxVerbatim}

\begin{sphinxVerbatim}[commandchars=\\\{\}]
              AAPL    GOOG     IBM    MSFT
Date                                      
2004\PYGZhy{}08\PYGZhy{}20  0.0029  0.0794  0.0042  0.0030
2004\PYGZhy{}08\PYGZhy{}23  0.0091  0.0101 \PYGZhy{}0.0070  0.0044
2004\PYGZhy{}08\PYGZhy{}24  0.0280 \PYGZhy{}0.0414  0.0007  0.0000
2004\PYGZhy{}08\PYGZhy{}25  0.0344  0.0108  0.0042  0.0114
2004\PYGZhy{}08\PYGZhy{}26  0.0487  0.0180 \PYGZhy{}0.0045 \PYGZhy{}0.0040
...            ...     ...     ...     ...
2023\PYGZhy{}01\PYGZhy{}18 \PYGZhy{}0.0054 \PYGZhy{}0.0041 \PYGZhy{}0.0329 \PYGZhy{}0.0189
2023\PYGZhy{}01\PYGZhy{}19  0.0004  0.0232  0.0015 \PYGZhy{}0.0165
2023\PYGZhy{}01\PYGZhy{}20  0.0192  0.0572  0.0041  0.0357
2023\PYGZhy{}01\PYGZhy{}23  0.0235  0.0194  0.0047  0.0098
2023\PYGZhy{}01\PYGZhy{}24  0.0109 \PYGZhy{}0.0186 \PYGZhy{}0.0029 \PYGZhy{}0.0008

[4639 rows x 4 columns]
\end{sphinxVerbatim}

\end{sphinxuseclass}\end{sphinxVerbatimOutput}

\end{sphinxuseclass}

\subsection{Practice}
\label{\detokenize{mckinney_05_practice:practice}}

\subsubsection{What are the mean daily returns for these four stocks?}
\label{\detokenize{mckinney_05_practice:what-are-the-mean-daily-returns-for-these-four-stocks}}

\subsubsection{What are the standard deviations of daily returns for these four stocks?}
\label{\detokenize{mckinney_05_practice:what-are-the-standard-deviations-of-daily-returns-for-these-four-stocks}}

\subsubsection{What are the \sphinxstyleemphasis{annualized} means and standard deviations of daily returns for these four stocks?}
\label{\detokenize{mckinney_05_practice:what-are-the-annualized-means-and-standard-deviations-of-daily-returns-for-these-four-stocks}}

\subsubsection{Plot \sphinxstyleemphasis{annualized} means versus standard deviations of daily returns for these four stocks}
\label{\detokenize{mckinney_05_practice:plot-annualized-means-versus-standard-deviations-of-daily-returns-for-these-four-stocks}}
\sphinxAtStartPar
Use \sphinxcode{\sphinxupquote{plt.scatter()}}, which expects arguments as \sphinxcode{\sphinxupquote{x}} (standard deviations) then \sphinxcode{\sphinxupquote{y}} (means).


\subsubsection{Repeat the previous calculations and plot for the stocks in the Dow\sphinxhyphen{}Jones Industrial Index (DJIA)}
\label{\detokenize{mckinney_05_practice:repeat-the-previous-calculations-and-plot-for-the-stocks-in-the-dow-jones-industrial-index-djia}}
\sphinxAtStartPar
We can find the current DJIA stocks on \sphinxhref{https://en.wikipedia.org/wiki/Dow\_Jones\_Industrial\_Average}{Wikipedia}.
We will need to download new data, into \sphinxcode{\sphinxupquote{tickers2}}, \sphinxcode{\sphinxupquote{prices2}}, and \sphinxcode{\sphinxupquote{returns2}}.


\subsubsection{Calculate total returns for the stocks in the DJIA}
\label{\detokenize{mckinney_05_practice:calculate-total-returns-for-the-stocks-in-the-djia}}
\sphinxAtStartPar
We can use the \sphinxcode{\sphinxupquote{.prod()}} method to compound returns as \$1 + R\_T = \textbackslash{}prod\_\{t=1\}\textasciicircum{}T (1 + R\_t)\$.
Technically, we should write \$R\_T\$ as \$R\_\{0,T\}\$, but we typically omit the subscript \$0\$.


\subsubsection{Plot the distribution of total returns for the stocks in the DJIA}
\label{\detokenize{mckinney_05_practice:plot-the-distribution-of-total-returns-for-the-stocks-in-the-djia}}
\sphinxAtStartPar
We can plot a histogram, using either the \sphinxcode{\sphinxupquote{plt.hist()}} function or the \sphinxcode{\sphinxupquote{.plot(kind='hist')}} method.


\subsubsection{Which stocks have the minimum and maximum total returns?}
\label{\detokenize{mckinney_05_practice:which-stocks-have-the-minimum-and-maximum-total-returns}}

\subsubsection{Plot the cumulative returns for the stocks in the DJIA}
\label{\detokenize{mckinney_05_practice:plot-the-cumulative-returns-for-the-stocks-in-the-djia}}
\sphinxAtStartPar
We can use the cumulative product method \sphinxcode{\sphinxupquote{.cumprod()}} to calculate the right hand side of the formula above.


\subsubsection{Repeat the plot above with only the minimum and maximum total returns}
\label{\detokenize{mckinney_05_practice:repeat-the-plot-above-with-only-the-minimum-and-maximum-total-returns}}
\sphinxstepscope


\section{McKinney Chapter 5 \sphinxhyphen{} Practice (Section 3, Monday 2:45 PM)}
\label{\detokenize{mckinney_05_practice_03:mckinney-chapter-5-practice-section-3-monday-2-45-pm}}\label{\detokenize{mckinney_05_practice_03::doc}}

\subsection{Announcements}
\label{\detokenize{mckinney_05_practice_03:announcements}}\begin{itemize}
\item {} 
\sphinxAtStartPar
Due soon
\begin{itemize}
\item {} 
\sphinxAtStartPar
11:59 PM on Friday, 2/3: Quiz 2

\item {} 
\sphinxAtStartPar
11:59 PM on Friday, 2/3: 10,000 XP on DataCamp

\item {} 
\sphinxAtStartPar
Project groups
\begin{itemize}
\item {} 
\sphinxAtStartPar
9 AM on Monday, 2/6: I will open groups on Canvas

\item {} 
\sphinxAtStartPar
9 AM on Monday, 2/13: I will close groups and post project 1

\end{itemize}

\end{itemize}

\end{itemize}


\subsection{Practice}
\label{\detokenize{mckinney_05_practice_03:practice}}
\begin{sphinxuseclass}{cell}\begin{sphinxVerbatimInput}

\begin{sphinxuseclass}{cell_input}
\begin{sphinxVerbatim}[commandchars=\\\{\}]
\PYG{k+kn}{import} \PYG{n+nn}{matplotlib}\PYG{n+nn}{.}\PYG{n+nn}{pyplot} \PYG{k}{as} \PYG{n+nn}{plt}
\PYG{k+kn}{import} \PYG{n+nn}{numpy} \PYG{k}{as} \PYG{n+nn}{np}
\PYG{k+kn}{import} \PYG{n+nn}{pandas} \PYG{k}{as} \PYG{n+nn}{pd}

\PYG{k+kn}{import} \PYG{n+nn}{yfinance} \PYG{k}{as} \PYG{n+nn}{yf}
\PYG{k+kn}{import} \PYG{n+nn}{requests\PYGZus{}cache}
\end{sphinxVerbatim}

\end{sphinxuseclass}\end{sphinxVerbatimInput}

\end{sphinxuseclass}
\begin{sphinxuseclass}{cell}\begin{sphinxVerbatimInput}

\begin{sphinxuseclass}{cell_input}
\begin{sphinxVerbatim}[commandchars=\\\{\}]
\PYG{o}{\PYGZpc{}}\PYG{k}{config} InlineBackend.figure\PYGZus{}format = \PYGZsq{}retina\PYGZsq{}
\PYG{o}{\PYGZpc{}}\PYG{k}{precision} 4
\PYG{n}{pd}\PYG{o}{.}\PYG{n}{options}\PYG{o}{.}\PYG{n}{display}\PYG{o}{.}\PYG{n}{float\PYGZus{}format} \PYG{o}{=} \PYG{l+s+s1}{\PYGZsq{}}\PYG{l+s+si}{\PYGZob{}:.4f\PYGZcb{}}\PYG{l+s+s1}{\PYGZsq{}}\PYG{o}{.}\PYG{n}{format}
\PYG{n}{session} \PYG{o}{=} \PYG{n}{requests\PYGZus{}cache}\PYG{o}{.}\PYG{n}{CachedSession}\PYG{p}{(}\PYG{p}{)}
\end{sphinxVerbatim}

\end{sphinxuseclass}\end{sphinxVerbatimInput}

\end{sphinxuseclass}
\begin{sphinxuseclass}{cell}\begin{sphinxVerbatimInput}

\begin{sphinxuseclass}{cell_input}
\begin{sphinxVerbatim}[commandchars=\\\{\}]
\PYG{n}{tickers} \PYG{o}{=} \PYG{n}{yf}\PYG{o}{.}\PYG{n}{Tickers}\PYG{p}{(}\PYG{l+s+s1}{\PYGZsq{}}\PYG{l+s+s1}{AAPL IBM MSFT GOOG}\PYG{l+s+s1}{\PYGZsq{}}\PYG{p}{,} \PYG{n}{session}\PYG{o}{=}\PYG{n}{session}\PYG{p}{)} \PYG{c+c1}{\PYGZsh{} initiates a tickers object}
\PYG{n}{prices} \PYG{o}{=} \PYG{n}{tickers}\PYG{o}{.}\PYG{n}{history}\PYG{p}{(}\PYG{n}{period}\PYG{o}{=}\PYG{l+s+s1}{\PYGZsq{}}\PYG{l+s+s1}{max}\PYG{l+s+s1}{\PYGZsq{}}\PYG{p}{,} \PYG{n}{auto\PYGZus{}adjust}\PYG{o}{=}\PYG{k+kc}{False}\PYG{p}{,} \PYG{n}{progress}\PYG{o}{=}\PYG{k+kc}{False}\PYG{p}{)} \PYG{c+c1}{\PYGZsh{} downloads history data}
\PYG{n}{prices}\PYG{o}{.}\PYG{n}{index} \PYG{o}{=} \PYG{n}{prices}\PYG{o}{.}\PYG{n}{index}\PYG{o}{.}\PYG{n}{tz\PYGZus{}localize}\PYG{p}{(}\PYG{k+kc}{None}\PYG{p}{)} \PYG{c+c1}{\PYGZsh{} removes time zone data from index}
\PYG{n}{returns} \PYG{o}{=} \PYG{n}{prices}\PYG{p}{[}\PYG{l+s+s1}{\PYGZsq{}}\PYG{l+s+s1}{Adj Close}\PYG{l+s+s1}{\PYGZsq{}}\PYG{p}{]}\PYG{o}{.}\PYG{n}{pct\PYGZus{}change}\PYG{p}{(}\PYG{p}{)}\PYG{o}{.}\PYG{n}{dropna}\PYG{p}{(}\PYG{p}{)} \PYG{c+c1}{\PYGZsh{} calculates returns and drops dates with missing returns}
\PYG{n}{returns}
\end{sphinxVerbatim}

\end{sphinxuseclass}\end{sphinxVerbatimInput}
\begin{sphinxVerbatimOutput}

\begin{sphinxuseclass}{cell_output}
\begin{sphinxVerbatim}[commandchars=\\\{\}]
[*********************100\PYGZpc{}***********************]  4 of 4 completed
\end{sphinxVerbatim}

\begin{sphinxVerbatim}[commandchars=\\\{\}]
              AAPL    GOOG     IBM    MSFT
Date                                      
2004\PYGZhy{}08\PYGZhy{}20  0.0029  0.0794  0.0042  0.0029
2004\PYGZhy{}08\PYGZhy{}23  0.0091  0.0101 \PYGZhy{}0.0070  0.0044
2004\PYGZhy{}08\PYGZhy{}24  0.0280 \PYGZhy{}0.0414  0.0007  0.0000
2004\PYGZhy{}08\PYGZhy{}25  0.0344  0.0108  0.0042  0.0114
2004\PYGZhy{}08\PYGZhy{}26  0.0487  0.0180 \PYGZhy{}0.0045 \PYGZhy{}0.0040
...            ...     ...     ...     ...
2023\PYGZhy{}01\PYGZhy{}24  0.0101 \PYGZhy{}0.0198 \PYGZhy{}0.0026 \PYGZhy{}0.0022
2023\PYGZhy{}01\PYGZhy{}25 \PYGZhy{}0.0047 \PYGZhy{}0.0250 \PYGZhy{}0.0052 \PYGZhy{}0.0059
2023\PYGZhy{}01\PYGZhy{}26  0.0148  0.0251 \PYGZhy{}0.0448  0.0307
2023\PYGZhy{}01\PYGZhy{}27  0.0137  0.0156 \PYGZhy{}0.0004  0.0006
2023\PYGZhy{}01\PYGZhy{}30 \PYGZhy{}0.0201 \PYGZhy{}0.0274  0.0068 \PYGZhy{}0.0220

[4643 rows x 4 columns]
\end{sphinxVerbatim}

\end{sphinxuseclass}\end{sphinxVerbatimOutput}

\end{sphinxuseclass}

\subsubsection{What are the mean daily returns for these four stocks?}
\label{\detokenize{mckinney_05_practice_03:what-are-the-mean-daily-returns-for-these-four-stocks}}
\sphinxAtStartPar
By default, the \sphinxcode{\sphinxupquote{.mean()}} method calculates means up\sphinxhyphen{}and\sphinxhyphen{}down columns (i.e., \sphinxcode{\sphinxupquote{axis=0}}).

\begin{sphinxuseclass}{cell}\begin{sphinxVerbatimInput}

\begin{sphinxuseclass}{cell_input}
\begin{sphinxVerbatim}[commandchars=\\\{\}]
\PYG{n}{returns}\PYG{o}{.}\PYG{n}{mean}\PYG{p}{(}\PYG{p}{)}
\end{sphinxVerbatim}

\end{sphinxuseclass}\end{sphinxVerbatimInput}
\begin{sphinxVerbatimOutput}

\begin{sphinxuseclass}{cell_output}
\begin{sphinxVerbatim}[commandchars=\\\{\}]
AAPL   0.0015
GOOG   0.0010
IBM    0.0003
MSFT   0.0007
dtype: float64
\end{sphinxVerbatim}

\end{sphinxuseclass}\end{sphinxVerbatimOutput}

\end{sphinxuseclass}
\sphinxAtStartPar
We set \sphinxcode{\sphinxupquote{axis=1}} if we want to calculate means left\sphinxhyphen{}and\sphinxhyphen{}right across rows.

\begin{sphinxuseclass}{cell}\begin{sphinxVerbatimInput}

\begin{sphinxuseclass}{cell_input}
\begin{sphinxVerbatim}[commandchars=\\\{\}]
\PYG{n}{returns}\PYG{o}{.}\PYG{n}{mean}\PYG{p}{(}\PYG{n}{axis}\PYG{o}{=}\PYG{l+m+mi}{1}\PYG{p}{)}
\end{sphinxVerbatim}

\end{sphinxuseclass}\end{sphinxVerbatimInput}
\begin{sphinxVerbatimOutput}

\begin{sphinxuseclass}{cell_output}
\begin{sphinxVerbatim}[commandchars=\\\{\}]
Date
2004\PYGZhy{}08\PYGZhy{}20    0.0224
2004\PYGZhy{}08\PYGZhy{}23    0.0041
2004\PYGZhy{}08\PYGZhy{}24   \PYGZhy{}0.0032
2004\PYGZhy{}08\PYGZhy{}25    0.0152
2004\PYGZhy{}08\PYGZhy{}26    0.0146
               ...  
2023\PYGZhy{}01\PYGZhy{}24   \PYGZhy{}0.0036
2023\PYGZhy{}01\PYGZhy{}25   \PYGZhy{}0.0102
2023\PYGZhy{}01\PYGZhy{}26    0.0065
2023\PYGZhy{}01\PYGZhy{}27    0.0074
2023\PYGZhy{}01\PYGZhy{}30   \PYGZhy{}0.0157
Length: 4643, dtype: float64
\end{sphinxVerbatim}

\end{sphinxuseclass}\end{sphinxVerbatimOutput}

\end{sphinxuseclass}

\subsubsection{What are the standard deviations of daily returns for these four stocks?}
\label{\detokenize{mckinney_05_practice_03:what-are-the-standard-deviations-of-daily-returns-for-these-four-stocks}}
\begin{sphinxuseclass}{cell}\begin{sphinxVerbatimInput}

\begin{sphinxuseclass}{cell_input}
\begin{sphinxVerbatim}[commandchars=\\\{\}]
\PYG{n}{returns}\PYG{o}{.}\PYG{n}{std}\PYG{p}{(}\PYG{p}{)}
\end{sphinxVerbatim}

\end{sphinxuseclass}\end{sphinxVerbatimInput}
\begin{sphinxVerbatimOutput}

\begin{sphinxuseclass}{cell_output}
\begin{sphinxVerbatim}[commandchars=\\\{\}]
AAPL   0.0210
GOOG   0.0194
IBM    0.0145
MSFT   0.0173
dtype: float64
\end{sphinxVerbatim}

\end{sphinxuseclass}\end{sphinxVerbatimOutput}

\end{sphinxuseclass}
\sphinxAtStartPar
And the \sphinxcode{\sphinxupquote{.describe()}} is a great way to get common summary statistics.

\begin{sphinxuseclass}{cell}\begin{sphinxVerbatimInput}

\begin{sphinxuseclass}{cell_input}
\begin{sphinxVerbatim}[commandchars=\\\{\}]
\PYG{n}{returns}\PYG{o}{.}\PYG{n}{describe}\PYG{p}{(}\PYG{p}{)}
\end{sphinxVerbatim}

\end{sphinxuseclass}\end{sphinxVerbatimInput}
\begin{sphinxVerbatimOutput}

\begin{sphinxuseclass}{cell_output}
\begin{sphinxVerbatim}[commandchars=\\\{\}]
           AAPL      GOOG       IBM      MSFT
count 4643.0000 4643.0000 4643.0000 4643.0000
mean     0.0015    0.0010    0.0003    0.0007
std      0.0210    0.0194    0.0145    0.0173
min     \PYGZhy{}0.1792   \PYGZhy{}0.1161   \PYGZhy{}0.1285   \PYGZhy{}0.1474
25\PYGZpc{}     \PYGZhy{}0.0085   \PYGZhy{}0.0077   \PYGZhy{}0.0063   \PYGZhy{}0.0071
50\PYGZpc{}      0.0011    0.0007    0.0004    0.0004
75\PYGZpc{}      0.0121    0.0100    0.0074    0.0087
max      0.1390    0.1999    0.1151    0.1860
\end{sphinxVerbatim}

\end{sphinxuseclass}\end{sphinxVerbatimOutput}

\end{sphinxuseclass}

\subsubsection{What are the \sphinxstyleemphasis{annualized} means and standard deviations of daily returns for these four stocks?}
\label{\detokenize{mckinney_05_practice_03:what-are-the-annualized-means-and-standard-deviations-of-daily-returns-for-these-four-stocks}}
\begin{sphinxuseclass}{cell}\begin{sphinxVerbatimInput}

\begin{sphinxuseclass}{cell_input}
\begin{sphinxVerbatim}[commandchars=\\\{\}]
\PYG{n}{returns}\PYG{o}{.}\PYG{n}{mean}\PYG{p}{(}\PYG{p}{)}\PYG{o}{.}\PYG{n}{mul}\PYG{p}{(}\PYG{l+m+mi}{252}\PYG{p}{)} \PYG{c+c1}{\PYGZsh{} multiply by the number of periods per year to annualize a mean}
\end{sphinxVerbatim}

\end{sphinxuseclass}\end{sphinxVerbatimInput}
\begin{sphinxVerbatimOutput}

\begin{sphinxuseclass}{cell_output}
\begin{sphinxVerbatim}[commandchars=\\\{\}]
AAPL   0.3663
GOOG   0.2462
IBM    0.0823
MSFT   0.1814
dtype: float64
\end{sphinxVerbatim}

\end{sphinxuseclass}\end{sphinxVerbatimOutput}

\end{sphinxuseclass}
\begin{sphinxuseclass}{cell}\begin{sphinxVerbatimInput}

\begin{sphinxuseclass}{cell_input}
\begin{sphinxVerbatim}[commandchars=\\\{\}]
\PYG{n}{returns}\PYG{o}{.}\PYG{n}{std}\PYG{p}{(}\PYG{p}{)}\PYG{o}{.}\PYG{n}{mul}\PYG{p}{(}\PYG{n}{np}\PYG{o}{.}\PYG{n}{sqrt}\PYG{p}{(}\PYG{l+m+mi}{252}\PYG{p}{)}\PYG{p}{)} \PYG{c+c1}{\PYGZsh{} multiply by the SQUARE ROOT of the number of periods per year to annualize a std. dev.}
\end{sphinxVerbatim}

\end{sphinxuseclass}\end{sphinxVerbatimInput}
\begin{sphinxVerbatimOutput}

\begin{sphinxuseclass}{cell_output}
\begin{sphinxVerbatim}[commandchars=\\\{\}]
AAPL   0.3331
GOOG   0.3077
IBM    0.2294
MSFT   0.2739
dtype: float64
\end{sphinxVerbatim}

\end{sphinxuseclass}\end{sphinxVerbatimOutput}

\end{sphinxuseclass}
\sphinxAtStartPar
\sphinxstyleemphasis{\sphinxstylestrong{The best explanation I have found on why stock return volatility (the standard deviation of stocks returns) grows with the square root of time is at the bottom of page 7 of \sphinxhref{https://book.ivo-welch.info/read/source5.mba/08-invchoice.pdf}{chapter 8 of Ivo Welch’s free corporate finance textbook}.}}


\subsubsection{Plot \sphinxstyleemphasis{annualized} means versus standard deviations of daily returns for these four stocks}
\label{\detokenize{mckinney_05_practice_03:plot-annualized-means-versus-standard-deviations-of-daily-returns-for-these-four-stocks}}
\sphinxAtStartPar
Use \sphinxcode{\sphinxupquote{plt.scatter()}}, which expects arguments as \sphinxcode{\sphinxupquote{x}} (standard deviations) then \sphinxcode{\sphinxupquote{y}} (means).

\begin{sphinxuseclass}{cell}\begin{sphinxVerbatimInput}

\begin{sphinxuseclass}{cell_input}
\begin{sphinxVerbatim}[commandchars=\\\{\}]
\PYG{n}{plt}\PYG{o}{.}\PYG{n}{scatter}\PYG{p}{(}
    \PYG{n}{x}\PYG{o}{=}\PYG{n}{returns}\PYG{o}{.}\PYG{n}{std}\PYG{p}{(}\PYG{p}{)}\PYG{o}{.}\PYG{n}{mul}\PYG{p}{(}\PYG{n}{np}\PYG{o}{.}\PYG{n}{sqrt}\PYG{p}{(}\PYG{l+m+mi}{252}\PYG{p}{)} \PYG{o}{*} \PYG{l+m+mi}{100}\PYG{p}{)}\PYG{p}{,}
    \PYG{n}{y}\PYG{o}{=}\PYG{n}{returns}\PYG{o}{.}\PYG{n}{mean}\PYG{p}{(}\PYG{p}{)}\PYG{o}{.}\PYG{n}{mul}\PYG{p}{(}\PYG{l+m+mi}{252} \PYG{o}{*} \PYG{l+m+mi}{100}\PYG{p}{)}
\PYG{p}{)}
\PYG{n}{plt}\PYG{o}{.}\PYG{n}{xlabel}\PYG{p}{(}\PYG{l+s+s1}{\PYGZsq{}}\PYG{l+s+s1}{Annualized Std. Dev. (}\PYG{l+s+s1}{\PYGZpc{}}\PYG{l+s+s1}{)}\PYG{l+s+s1}{\PYGZsq{}}\PYG{p}{)}
\PYG{n}{plt}\PYG{o}{.}\PYG{n}{ylabel}\PYG{p}{(}\PYG{l+s+s1}{\PYGZsq{}}\PYG{l+s+s1}{Annualized Mean Return (}\PYG{l+s+s1}{\PYGZpc{}}\PYG{l+s+s1}{)}\PYG{l+s+s1}{\PYGZsq{}}\PYG{p}{)}
\PYG{n}{plt}\PYG{o}{.}\PYG{n}{title}\PYG{p}{(}\PYG{l+s+sa}{f}\PYG{l+s+s1}{\PYGZsq{}}\PYG{l+s+s1}{Return Versus Risk}\PYG{l+s+se}{\PYGZbs{}n}\PYG{l+s+s1}{ from }\PYG{l+s+si}{\PYGZob{}}\PYG{n}{returns}\PYG{o}{.}\PYG{n}{index}\PYG{p}{[}\PYG{l+m+mi}{0}\PYG{p}{]}\PYG{l+s+si}{:}\PYG{l+s+s1}{\PYGZpc{}B \PYGZpc{}Y}\PYG{l+s+si}{\PYGZcb{}}\PYG{l+s+s1}{ to }\PYG{l+s+si}{\PYGZob{}}\PYG{n}{returns}\PYG{o}{.}\PYG{n}{index}\PYG{p}{[}\PYG{o}{\PYGZhy{}}\PYG{l+m+mi}{1}\PYG{p}{]}\PYG{l+s+si}{:}\PYG{l+s+s1}{\PYGZpc{}B \PYGZpc{}Y}\PYG{l+s+si}{\PYGZcb{}}\PYG{l+s+s1}{\PYGZsq{}}\PYG{p}{)}

\PYG{k}{for} \PYG{n}{i} \PYG{o+ow}{in} \PYG{n}{returns}\PYG{p}{:}
    \PYG{n}{plt}\PYG{o}{.}\PYG{n}{text}\PYG{p}{(}
        \PYG{n}{x}\PYG{o}{=}\PYG{n}{returns}\PYG{p}{[}\PYG{n}{i}\PYG{p}{]}\PYG{o}{.}\PYG{n}{std}\PYG{p}{(}\PYG{p}{)} \PYG{o}{*} \PYG{n}{np}\PYG{o}{.}\PYG{n}{sqrt}\PYG{p}{(}\PYG{l+m+mi}{252}\PYG{p}{)} \PYG{o}{*} \PYG{l+m+mi}{100}\PYG{p}{,} 
        \PYG{n}{y}\PYG{o}{=}\PYG{n}{returns}\PYG{p}{[}\PYG{n}{i}\PYG{p}{]}\PYG{o}{.}\PYG{n}{mean}\PYG{p}{(}\PYG{p}{)} \PYG{o}{*} \PYG{l+m+mi}{252} \PYG{o}{*} \PYG{l+m+mi}{100}\PYG{p}{,} 
        \PYG{n}{s}\PYG{o}{=}\PYG{n}{i}
    \PYG{p}{)}

\PYG{n}{plt}\PYG{o}{.}\PYG{n}{show}\PYG{p}{(}\PYG{p}{)}
\end{sphinxVerbatim}

\end{sphinxuseclass}\end{sphinxVerbatimInput}
\begin{sphinxVerbatimOutput}

\begin{sphinxuseclass}{cell_output}
\noindent\sphinxincludegraphics{{2875f2309782841f48dc14a62c3a598a133e7e1912c3e75e2a0e3aa2c9985442}.png}

\end{sphinxuseclass}\end{sphinxVerbatimOutput}

\end{sphinxuseclass}

\subsubsection{Repeat the previous calculations and plot for the stocks in the Dow\sphinxhyphen{}Jones Industrial Index (DJIA)}
\label{\detokenize{mckinney_05_practice_03:repeat-the-previous-calculations-and-plot-for-the-stocks-in-the-dow-jones-industrial-index-djia}}
\sphinxAtStartPar
We can find the current DJIA stocks on \sphinxhref{https://en.wikipedia.org/wiki/Dow\_Jones\_Industrial\_Average}{Wikipedia}.
We will need to download new data, into \sphinxcode{\sphinxupquote{tickers2}}, \sphinxcode{\sphinxupquote{prices2}}, and \sphinxcode{\sphinxupquote{returns2}}.

\begin{sphinxuseclass}{cell}\begin{sphinxVerbatimInput}

\begin{sphinxuseclass}{cell_input}
\begin{sphinxVerbatim}[commandchars=\\\{\}]
\PYG{n}{symbols} \PYG{o}{=} \PYG{n}{pd}\PYG{o}{.}\PYG{n}{read\PYGZus{}html}\PYG{p}{(}\PYG{l+s+s1}{\PYGZsq{}}\PYG{l+s+s1}{https://en.wikipedia.org/wiki/Dow\PYGZus{}Jones\PYGZus{}Industrial\PYGZus{}Average}\PYG{l+s+s1}{\PYGZsq{}}\PYG{p}{)}\PYG{p}{[}\PYG{l+m+mi}{1}\PYG{p}{]}\PYG{p}{[}\PYG{l+s+s1}{\PYGZsq{}}\PYG{l+s+s1}{Symbol}\PYG{l+s+s1}{\PYGZsq{}}\PYG{p}{]}
\end{sphinxVerbatim}

\end{sphinxuseclass}\end{sphinxVerbatimInput}

\end{sphinxuseclass}
\begin{sphinxuseclass}{cell}\begin{sphinxVerbatimInput}

\begin{sphinxuseclass}{cell_input}
\begin{sphinxVerbatim}[commandchars=\\\{\}]
\PYG{n}{tickers2} \PYG{o}{=} \PYG{n}{yf}\PYG{o}{.}\PYG{n}{Tickers}\PYG{p}{(}\PYG{n}{tickers}\PYG{o}{=}\PYG{n}{symbols}\PYG{o}{.}\PYG{n}{to\PYGZus{}list}\PYG{p}{(}\PYG{p}{)}\PYG{p}{,} \PYG{n}{session}\PYG{o}{=}\PYG{n}{session}\PYG{p}{)}
\PYG{n}{prices2} \PYG{o}{=} \PYG{n}{tickers2}\PYG{o}{.}\PYG{n}{history}\PYG{p}{(}\PYG{n}{period}\PYG{o}{=}\PYG{l+s+s1}{\PYGZsq{}}\PYG{l+s+s1}{max}\PYG{l+s+s1}{\PYGZsq{}}\PYG{p}{,} \PYG{n}{auto\PYGZus{}adjust}\PYG{o}{=}\PYG{k+kc}{False}\PYG{p}{,} \PYG{n}{progress}\PYG{o}{=}\PYG{k+kc}{False}\PYG{p}{)}
\PYG{n}{prices2}\PYG{o}{.}\PYG{n}{index} \PYG{o}{=} \PYG{n}{prices2}\PYG{o}{.}\PYG{n}{index}\PYG{o}{.}\PYG{n}{tz\PYGZus{}localize}\PYG{p}{(}\PYG{k+kc}{None}\PYG{p}{)}
\PYG{n}{returns2} \PYG{o}{=} \PYG{n}{prices2}\PYG{p}{[}\PYG{l+s+s1}{\PYGZsq{}}\PYG{l+s+s1}{Adj Close}\PYG{l+s+s1}{\PYGZsq{}}\PYG{p}{]}\PYG{o}{.}\PYG{n}{pct\PYGZus{}change}\PYG{p}{(}\PYG{p}{)}\PYG{o}{.}\PYG{n}{dropna}\PYG{p}{(}\PYG{p}{)}
\PYG{c+c1}{\PYGZsh{} returns2}
\end{sphinxVerbatim}

\end{sphinxuseclass}\end{sphinxVerbatimInput}
\begin{sphinxVerbatimOutput}

\begin{sphinxuseclass}{cell_output}
\begin{sphinxVerbatim}[commandchars=\\\{\}]
[*********************100\PYGZpc{}***********************]  30 of 30 completed
\end{sphinxVerbatim}

\end{sphinxuseclass}\end{sphinxVerbatimOutput}

\end{sphinxuseclass}
\begin{sphinxuseclass}{cell}\begin{sphinxVerbatimInput}

\begin{sphinxuseclass}{cell_input}
\begin{sphinxVerbatim}[commandchars=\\\{\}]
\PYG{n}{prices2}\PYG{p}{[}\PYG{l+s+s1}{\PYGZsq{}}\PYG{l+s+s1}{Adj Close}\PYG{l+s+s1}{\PYGZsq{}}\PYG{p}{]}\PYG{o}{.}\PYG{n}{count}\PYG{p}{(}\PYG{p}{)}\PYG{o}{.}\PYG{n}{nsmallest}\PYG{p}{(}\PYG{l+m+mi}{5}\PYG{p}{)}\PYG{o}{.}\PYG{n}{div}\PYG{p}{(}\PYG{l+m+mi}{252}\PYG{p}{)}
\end{sphinxVerbatim}

\end{sphinxuseclass}\end{sphinxVerbatimInput}
\begin{sphinxVerbatimOutput}

\begin{sphinxuseclass}{cell_output}
\begin{sphinxVerbatim}[commandchars=\\\{\}]
DOW     3.8651
V      14.8532
CRM    18.5873
GS     23.7103
CSCO   32.9405
dtype: float64
\end{sphinxVerbatim}

\end{sphinxuseclass}\end{sphinxVerbatimOutput}

\end{sphinxuseclass}
\begin{sphinxuseclass}{cell}\begin{sphinxVerbatimInput}

\begin{sphinxuseclass}{cell_input}
\begin{sphinxVerbatim}[commandchars=\\\{\}]
\PYG{n}{plt}\PYG{o}{.}\PYG{n}{scatter}\PYG{p}{(}
    \PYG{n}{x}\PYG{o}{=}\PYG{n}{returns2}\PYG{o}{.}\PYG{n}{std}\PYG{p}{(}\PYG{p}{)}\PYG{o}{.}\PYG{n}{mul}\PYG{p}{(}\PYG{n}{np}\PYG{o}{.}\PYG{n}{sqrt}\PYG{p}{(}\PYG{l+m+mi}{252}\PYG{p}{)} \PYG{o}{*} \PYG{l+m+mi}{100}\PYG{p}{)}\PYG{p}{,}
    \PYG{n}{y}\PYG{o}{=}\PYG{n}{returns2}\PYG{o}{.}\PYG{n}{mean}\PYG{p}{(}\PYG{p}{)}\PYG{o}{.}\PYG{n}{mul}\PYG{p}{(}\PYG{l+m+mi}{252} \PYG{o}{*} \PYG{l+m+mi}{100}\PYG{p}{)}
\PYG{p}{)}
\PYG{n}{plt}\PYG{o}{.}\PYG{n}{xlabel}\PYG{p}{(}\PYG{l+s+s1}{\PYGZsq{}}\PYG{l+s+s1}{Annualized Std. Dev. (}\PYG{l+s+s1}{\PYGZpc{}}\PYG{l+s+s1}{)}\PYG{l+s+s1}{\PYGZsq{}}\PYG{p}{)}
\PYG{n}{plt}\PYG{o}{.}\PYG{n}{ylabel}\PYG{p}{(}\PYG{l+s+s1}{\PYGZsq{}}\PYG{l+s+s1}{Annualized Mean Return (}\PYG{l+s+s1}{\PYGZpc{}}\PYG{l+s+s1}{)}\PYG{l+s+s1}{\PYGZsq{}}\PYG{p}{)}
\PYG{n}{plt}\PYG{o}{.}\PYG{n}{title}\PYG{p}{(}\PYG{l+s+sa}{f}\PYG{l+s+s1}{\PYGZsq{}}\PYG{l+s+s1}{Return Versus Risk for DJIA Stocks}\PYG{l+s+se}{\PYGZbs{}n}\PYG{l+s+s1}{ from }\PYG{l+s+si}{\PYGZob{}}\PYG{n}{returns}\PYG{o}{.}\PYG{n}{index}\PYG{p}{[}\PYG{l+m+mi}{0}\PYG{p}{]}\PYG{l+s+si}{:}\PYG{l+s+s1}{\PYGZpc{}B \PYGZpc{}Y}\PYG{l+s+si}{\PYGZcb{}}\PYG{l+s+s1}{ to }\PYG{l+s+si}{\PYGZob{}}\PYG{n}{returns}\PYG{o}{.}\PYG{n}{index}\PYG{p}{[}\PYG{o}{\PYGZhy{}}\PYG{l+m+mi}{1}\PYG{p}{]}\PYG{l+s+si}{:}\PYG{l+s+s1}{\PYGZpc{}B \PYGZpc{}Y}\PYG{l+s+si}{\PYGZcb{}}\PYG{l+s+s1}{\PYGZsq{}}\PYG{p}{)}


\PYG{k}{for} \PYG{n}{i} \PYG{o+ow}{in} \PYG{n}{returns2}\PYG{p}{:}
    \PYG{n}{plt}\PYG{o}{.}\PYG{n}{text}\PYG{p}{(}
        \PYG{n}{x}\PYG{o}{=}\PYG{n}{returns2}\PYG{p}{[}\PYG{n}{i}\PYG{p}{]}\PYG{o}{.}\PYG{n}{std}\PYG{p}{(}\PYG{p}{)} \PYG{o}{*} \PYG{n}{np}\PYG{o}{.}\PYG{n}{sqrt}\PYG{p}{(}\PYG{l+m+mi}{252}\PYG{p}{)} \PYG{o}{*} \PYG{l+m+mi}{100}\PYG{p}{,} 
        \PYG{n}{y}\PYG{o}{=}\PYG{n}{returns2}\PYG{p}{[}\PYG{n}{i}\PYG{p}{]}\PYG{o}{.}\PYG{n}{mean}\PYG{p}{(}\PYG{p}{)} \PYG{o}{*} \PYG{l+m+mi}{252} \PYG{o}{*} \PYG{l+m+mi}{100}\PYG{p}{,} 
        \PYG{n}{s}\PYG{o}{=}\PYG{n}{i}
    \PYG{p}{)}

\PYG{n}{plt}\PYG{o}{.}\PYG{n}{show}\PYG{p}{(}\PYG{p}{)}
\end{sphinxVerbatim}

\end{sphinxuseclass}\end{sphinxVerbatimInput}
\begin{sphinxVerbatimOutput}

\begin{sphinxuseclass}{cell_output}
\noindent\sphinxincludegraphics{{095ff443524cd98cda6c87e0c77715dc3ab47cc214a7122bfa318e6d2e4bd9f4}.png}

\end{sphinxuseclass}\end{sphinxVerbatimOutput}

\end{sphinxuseclass}

\subsubsection{Calculate total returns for the stocks in the DJIA}
\label{\detokenize{mckinney_05_practice_03:calculate-total-returns-for-the-stocks-in-the-djia}}
\sphinxAtStartPar
We can use the \sphinxcode{\sphinxupquote{.prod()}} method to compound returns as \$1 + R\_T = \textbackslash{}prod\_\{t=1\}\textasciicircum{}T (1 + R\_t)\$.
Technically, we should write \$R\_T\$ as \$R\_\{0,T\}\$, but we typically omit the subscript \$0\$.

\begin{sphinxuseclass}{cell}\begin{sphinxVerbatimInput}

\begin{sphinxuseclass}{cell_input}
\begin{sphinxVerbatim}[commandchars=\\\{\}]
\PYG{n}{returns2\PYGZus{}total} \PYG{o}{=} \PYG{n}{returns2}\PYG{o}{.}\PYG{n}{add}\PYG{p}{(}\PYG{l+m+mi}{1}\PYG{p}{)}\PYG{o}{.}\PYG{n}{prod}\PYG{p}{(}\PYG{p}{)}\PYG{o}{.}\PYG{n}{sub}\PYG{p}{(}\PYG{l+m+mi}{1}\PYG{p}{)}
\end{sphinxVerbatim}

\end{sphinxuseclass}\end{sphinxVerbatimInput}

\end{sphinxuseclass}
\begin{sphinxuseclass}{cell}\begin{sphinxVerbatimInput}

\begin{sphinxuseclass}{cell_input}
\begin{sphinxVerbatim}[commandchars=\\\{\}]
\PYG{n}{returns2\PYGZus{}total\PYGZus{}alt} \PYG{o}{=} \PYG{p}{(}\PYG{n}{returns2} \PYG{o}{+} \PYG{l+m+mi}{1}\PYG{p}{)}\PYG{o}{.}\PYG{n}{prod}\PYG{p}{(}\PYG{p}{)} \PYG{o}{\PYGZhy{}} \PYG{l+m+mi}{1}
\end{sphinxVerbatim}

\end{sphinxuseclass}\end{sphinxVerbatimInput}

\end{sphinxuseclass}
\sphinxAtStartPar
The two approaches above are correct and calculate the same values, as shown by \sphinxcode{\sphinxupquote{np.allclose()}} below.
Still, I prefer the first approach because chained code is typically easier to write, read, and troubleshoot.

\begin{sphinxuseclass}{cell}\begin{sphinxVerbatimInput}

\begin{sphinxuseclass}{cell_input}
\begin{sphinxVerbatim}[commandchars=\\\{\}]
\PYG{n}{np}\PYG{o}{.}\PYG{n}{allclose}\PYG{p}{(}\PYG{n}{returns2\PYGZus{}total}\PYG{p}{,} \PYG{n}{returns2\PYGZus{}total\PYGZus{}alt}\PYG{p}{)}
\end{sphinxVerbatim}

\end{sphinxuseclass}\end{sphinxVerbatimInput}
\begin{sphinxVerbatimOutput}

\begin{sphinxuseclass}{cell_output}
\begin{sphinxVerbatim}[commandchars=\\\{\}]
True
\end{sphinxVerbatim}

\end{sphinxuseclass}\end{sphinxVerbatimOutput}

\end{sphinxuseclass}

\subsubsection{Plot the distribution of total returns for the stocks in the DJIA}
\label{\detokenize{mckinney_05_practice_03:plot-the-distribution-of-total-returns-for-the-stocks-in-the-djia}}
\sphinxAtStartPar
We can plot a histogram, using either the \sphinxcode{\sphinxupquote{plt.hist()}} function or the \sphinxcode{\sphinxupquote{.plot(kind='hist')}} method.

\begin{sphinxuseclass}{cell}\begin{sphinxVerbatimInput}

\begin{sphinxuseclass}{cell_input}
\begin{sphinxVerbatim}[commandchars=\\\{\}]
\PYG{n}{returns2\PYGZus{}total}\PYG{o}{.}\PYG{n}{mul}\PYG{p}{(}\PYG{l+m+mi}{100}\PYG{p}{)}\PYG{o}{.}\PYG{n}{plot}\PYG{p}{(}\PYG{n}{kind}\PYG{o}{=}\PYG{l+s+s1}{\PYGZsq{}}\PYG{l+s+s1}{hist}\PYG{l+s+s1}{\PYGZsq{}}\PYG{p}{,} \PYG{n}{bins}\PYG{o}{=}\PYG{l+m+mi}{25}\PYG{p}{)}
\PYG{n}{plt}\PYG{o}{.}\PYG{n}{xlabel}\PYG{p}{(}\PYG{l+s+s1}{\PYGZsq{}}\PYG{l+s+s1}{Total Return (}\PYG{l+s+s1}{\PYGZpc{}}\PYG{l+s+s1}{)}\PYG{l+s+s1}{\PYGZsq{}}\PYG{p}{)}
\PYG{n}{plt}\PYG{o}{.}\PYG{n}{title}\PYG{p}{(}\PYG{l+s+sa}{f}\PYG{l+s+s1}{\PYGZsq{}}\PYG{l+s+s1}{Distribution of Total Returns for DJIA Stocks}\PYG{l+s+se}{\PYGZbs{}n}\PYG{l+s+s1}{ from }\PYG{l+s+si}{\PYGZob{}}\PYG{n}{returns2}\PYG{o}{.}\PYG{n}{index}\PYG{p}{[}\PYG{l+m+mi}{0}\PYG{p}{]}\PYG{l+s+si}{:}\PYG{l+s+s1}{\PYGZpc{}B \PYGZpc{}Y}\PYG{l+s+si}{\PYGZcb{}}\PYG{l+s+s1}{ to  }\PYG{l+s+si}{\PYGZob{}}\PYG{n}{returns2}\PYG{o}{.}\PYG{n}{index}\PYG{p}{[}\PYG{o}{\PYGZhy{}}\PYG{l+m+mi}{1}\PYG{p}{]}\PYG{l+s+si}{:}\PYG{l+s+s1}{\PYGZpc{}B \PYGZpc{}Y}\PYG{l+s+si}{\PYGZcb{}}\PYG{l+s+s1}{\PYGZsq{}}\PYG{p}{)}
\PYG{n}{plt}\PYG{o}{.}\PYG{n}{show}\PYG{p}{(}\PYG{p}{)}
\end{sphinxVerbatim}

\end{sphinxuseclass}\end{sphinxVerbatimInput}
\begin{sphinxVerbatimOutput}

\begin{sphinxuseclass}{cell_output}
\noindent\sphinxincludegraphics{{5ba2e63e7ebf5d5c1209cc0142d9a10e93fc841dc191f5e6dca377db1857db7a}.png}

\end{sphinxuseclass}\end{sphinxVerbatimOutput}

\end{sphinxuseclass}

\subsubsection{Which stocks have the minimum and maximum total returns?}
\label{\detokenize{mckinney_05_practice_03:which-stocks-have-the-minimum-and-maximum-total-returns}}
\begin{sphinxuseclass}{cell}\begin{sphinxVerbatimInput}

\begin{sphinxuseclass}{cell_input}
\begin{sphinxVerbatim}[commandchars=\\\{\}]
\PYG{n}{returns2\PYGZus{}total}\PYG{o}{.}\PYG{n}{idxmin}\PYG{p}{(}\PYG{p}{)}
\end{sphinxVerbatim}

\end{sphinxuseclass}\end{sphinxVerbatimInput}
\begin{sphinxVerbatimOutput}

\begin{sphinxuseclass}{cell_output}
\begin{sphinxVerbatim}[commandchars=\\\{\}]
\PYGZsq{}BA\PYGZsq{}
\end{sphinxVerbatim}

\end{sphinxuseclass}\end{sphinxVerbatimOutput}

\end{sphinxuseclass}
\begin{sphinxuseclass}{cell}\begin{sphinxVerbatimInput}

\begin{sphinxuseclass}{cell_input}
\begin{sphinxVerbatim}[commandchars=\\\{\}]
\PYG{n}{returns2\PYGZus{}total}\PYG{o}{.}\PYG{n}{idxmax}\PYG{p}{(}\PYG{p}{)}
\end{sphinxVerbatim}

\end{sphinxuseclass}\end{sphinxVerbatimInput}
\begin{sphinxVerbatimOutput}

\begin{sphinxuseclass}{cell_output}
\begin{sphinxVerbatim}[commandchars=\\\{\}]
\PYGZsq{}AAPL\PYGZsq{}
\end{sphinxVerbatim}

\end{sphinxuseclass}\end{sphinxVerbatimOutput}

\end{sphinxuseclass}
\begin{sphinxuseclass}{cell}\begin{sphinxVerbatimInput}

\begin{sphinxuseclass}{cell_input}
\begin{sphinxVerbatim}[commandchars=\\\{\}]
\PYG{n}{returns2\PYGZus{}total}\PYG{o}{.}\PYG{n}{sort\PYGZus{}values}\PYG{p}{(}\PYG{p}{)}\PYG{p}{[}\PYG{p}{[}\PYG{l+m+mi}{0}\PYG{p}{,} \PYG{o}{\PYGZhy{}}\PYG{l+m+mi}{1}\PYG{p}{]}\PYG{p}{]}
\end{sphinxVerbatim}

\end{sphinxuseclass}\end{sphinxVerbatimInput}
\begin{sphinxVerbatimOutput}

\begin{sphinxuseclass}{cell_output}
\begin{sphinxVerbatim}[commandchars=\\\{\}]
BA     \PYGZhy{}0.4292
AAPL    2.1362
dtype: float64
\end{sphinxVerbatim}

\end{sphinxuseclass}\end{sphinxVerbatimOutput}

\end{sphinxuseclass}

\subsubsection{Plot the cumulative returns for the stocks in the DJIA}
\label{\detokenize{mckinney_05_practice_03:plot-the-cumulative-returns-for-the-stocks-in-the-djia}}
\sphinxAtStartPar
We can use the cumulative product method \sphinxcode{\sphinxupquote{.cumprod()}} to calculate the right hand side of the formula above.

\sphinxAtStartPar
The following code works, but the cumulative returns of 30 stocks is typically too much to digest.
Plus, the legend runs off the screen!

\begin{sphinxuseclass}{cell}\begin{sphinxVerbatimInput}

\begin{sphinxuseclass}{cell_input}
\begin{sphinxVerbatim}[commandchars=\\\{\}]
\PYG{n}{returns2}\PYG{o}{.}\PYG{n}{add}\PYG{p}{(}\PYG{l+m+mi}{1}\PYG{p}{)}\PYG{o}{.}\PYG{n}{cumprod}\PYG{p}{(}\PYG{p}{)}\PYG{o}{.}\PYG{n}{sub}\PYG{p}{(}\PYG{l+m+mi}{1}\PYG{p}{)}\PYG{o}{.}\PYG{n}{mul}\PYG{p}{(}\PYG{l+m+mi}{100}\PYG{p}{)}\PYG{o}{.}\PYG{n}{plot}\PYG{p}{(}\PYG{p}{)}
\PYG{n}{plt}\PYG{o}{.}\PYG{n}{ylabel}\PYG{p}{(}\PYG{l+s+s1}{\PYGZsq{}}\PYG{l+s+s1}{Cumulative Return (}\PYG{l+s+s1}{\PYGZpc{}}\PYG{l+s+s1}{)}\PYG{l+s+s1}{\PYGZsq{}}\PYG{p}{)}
\PYG{n}{plt}\PYG{o}{.}\PYG{n}{title}\PYG{p}{(}\PYG{l+s+sa}{f}\PYG{l+s+s1}{\PYGZsq{}}\PYG{l+s+s1}{Cumulative Returns for DJIA Stocks}\PYG{l+s+se}{\PYGZbs{}n}\PYG{l+s+s1}{ from }\PYG{l+s+si}{\PYGZob{}}\PYG{n}{returns2}\PYG{o}{.}\PYG{n}{index}\PYG{p}{[}\PYG{l+m+mi}{0}\PYG{p}{]}\PYG{l+s+si}{:}\PYG{l+s+s1}{\PYGZpc{}B \PYGZpc{}Y}\PYG{l+s+si}{\PYGZcb{}}\PYG{l+s+s1}{ to  }\PYG{l+s+si}{\PYGZob{}}\PYG{n}{returns2}\PYG{o}{.}\PYG{n}{index}\PYG{p}{[}\PYG{o}{\PYGZhy{}}\PYG{l+m+mi}{1}\PYG{p}{]}\PYG{l+s+si}{:}\PYG{l+s+s1}{\PYGZpc{}B \PYGZpc{}Y}\PYG{l+s+si}{\PYGZcb{}}\PYG{l+s+s1}{\PYGZsq{}}\PYG{p}{)}
\PYG{n}{plt}\PYG{o}{.}\PYG{n}{show}\PYG{p}{(}\PYG{p}{)}
\end{sphinxVerbatim}

\end{sphinxuseclass}\end{sphinxVerbatimInput}
\begin{sphinxVerbatimOutput}

\begin{sphinxuseclass}{cell_output}
\noindent\sphinxincludegraphics{{932637f2fa6b21aaca01438b7c5d1be51342964c320dcbe1edde20e556c7aa2f}.png}

\end{sphinxuseclass}\end{sphinxVerbatimOutput}

\end{sphinxuseclass}

\subsubsection{Repeat the plot above with only the minimum and maximum total returns}
\label{\detokenize{mckinney_05_practice_03:repeat-the-plot-above-with-only-the-minimum-and-maximum-total-returns}}
\begin{sphinxuseclass}{cell}\begin{sphinxVerbatimInput}

\begin{sphinxuseclass}{cell_input}
\begin{sphinxVerbatim}[commandchars=\\\{\}]
\PYG{n}{returns2\PYGZus{}total}\PYG{o}{.}\PYG{n}{sort\PYGZus{}values}\PYG{p}{(}\PYG{p}{)}\PYG{p}{[}\PYG{p}{[}\PYG{l+m+mi}{0}\PYG{p}{,} \PYG{o}{\PYGZhy{}}\PYG{l+m+mi}{1}\PYG{p}{]}\PYG{p}{]}\PYG{o}{.}\PYG{n}{index}
\end{sphinxVerbatim}

\end{sphinxuseclass}\end{sphinxVerbatimInput}
\begin{sphinxVerbatimOutput}

\begin{sphinxuseclass}{cell_output}
\begin{sphinxVerbatim}[commandchars=\\\{\}]
Index([\PYGZsq{}BA\PYGZsq{}, \PYGZsq{}AAPL\PYGZsq{}], dtype=\PYGZsq{}object\PYGZsq{})
\end{sphinxVerbatim}

\end{sphinxuseclass}\end{sphinxVerbatimOutput}

\end{sphinxuseclass}
\begin{sphinxuseclass}{cell}\begin{sphinxVerbatimInput}

\begin{sphinxuseclass}{cell_input}
\begin{sphinxVerbatim}[commandchars=\\\{\}]
\PYG{p}{(}
    \PYG{n}{returns2}\PYG{p}{[} \PYG{c+c1}{\PYGZsh{} daily returns for DJIA stocks}
        \PYG{n}{returns2\PYGZus{}total}\PYG{o}{.}\PYG{n}{sort\PYGZus{}values}\PYG{p}{(}\PYG{p}{)}\PYG{p}{[}\PYG{p}{[}\PYG{l+m+mi}{0}\PYG{p}{,} \PYG{o}{\PYGZhy{}}\PYG{l+m+mi}{1}\PYG{p}{]}\PYG{p}{]}\PYG{o}{.}\PYG{n}{index} \PYG{c+c1}{\PYGZsh{} tickers of stocks with min and max total returns}
        \PYG{c+c1}{\PYGZsh{} returns2.add(1).prod().sort\PYGZus{}values()[[0, \PYGZhy{}1]].index \PYGZsh{} }
    \PYG{p}{]}
    \PYG{o}{.}\PYG{n}{add}\PYG{p}{(}\PYG{l+m+mi}{1}\PYG{p}{)} \PYG{c+c1}{\PYGZsh{} adds 1 to every daily return }
    \PYG{o}{.}\PYG{n}{cumprod}\PYG{p}{(}\PYG{p}{)} \PYG{c+c1}{\PYGZsh{} calculate cumulative returns for each stock}
    \PYG{o}{.}\PYG{n}{sub}\PYG{p}{(}\PYG{l+m+mi}{1}\PYG{p}{)} \PYG{c+c1}{\PYGZsh{} substracts 1 from every cumulative return}
    \PYG{o}{.}\PYG{n}{mul}\PYG{p}{(}\PYG{l+m+mi}{100}\PYG{p}{)} \PYG{c+c1}{\PYGZsh{} converts decimal to percent returns}
    \PYG{o}{.}\PYG{n}{plot}\PYG{p}{(}\PYG{p}{)} \PYG{c+c1}{\PYGZsh{} plots}
\PYG{p}{)}
\PYG{n}{plt}\PYG{o}{.}\PYG{n}{ylabel}\PYG{p}{(}\PYG{l+s+s1}{\PYGZsq{}}\PYG{l+s+s1}{Cumulative Return (}\PYG{l+s+s1}{\PYGZpc{}}\PYG{l+s+s1}{)}\PYG{l+s+s1}{\PYGZsq{}}\PYG{p}{)}
\PYG{n}{plt}\PYG{o}{.}\PYG{n}{title}\PYG{p}{(}\PYG{l+s+sa}{f}\PYG{l+s+s1}{\PYGZsq{}}\PYG{l+s+s1}{Minimum and Maximum Cumulative Returns DJIA Stocks}\PYG{l+s+se}{\PYGZbs{}n}\PYG{l+s+s1}{ from }\PYG{l+s+si}{\PYGZob{}}\PYG{n}{returns2}\PYG{o}{.}\PYG{n}{index}\PYG{p}{[}\PYG{l+m+mi}{0}\PYG{p}{]}\PYG{l+s+si}{:}\PYG{l+s+s1}{\PYGZpc{}B \PYGZpc{}Y}\PYG{l+s+si}{\PYGZcb{}}\PYG{l+s+s1}{ to  }\PYG{l+s+si}{\PYGZob{}}\PYG{n}{returns2}\PYG{o}{.}\PYG{n}{index}\PYG{p}{[}\PYG{o}{\PYGZhy{}}\PYG{l+m+mi}{1}\PYG{p}{]}\PYG{l+s+si}{:}\PYG{l+s+s1}{\PYGZpc{}B \PYGZpc{}Y}\PYG{l+s+si}{\PYGZcb{}}\PYG{l+s+s1}{\PYGZsq{}}\PYG{p}{)}
\PYG{n}{plt}\PYG{o}{.}\PYG{n}{show}\PYG{p}{(}\PYG{p}{)}
\end{sphinxVerbatim}

\end{sphinxuseclass}\end{sphinxVerbatimInput}
\begin{sphinxVerbatimOutput}

\begin{sphinxuseclass}{cell_output}
\noindent\sphinxincludegraphics{{6d23d7e0b18057e2d9ffda8c7688dddaf35437be7f6112e9722f310dfaae07e0}.png}

\end{sphinxuseclass}\end{sphinxVerbatimOutput}

\end{sphinxuseclass}
\sphinxstepscope


\section{McKinney Chapter 5 \sphinxhyphen{} Practice (Section 4, Wednesday 11:45 AM)}
\label{\detokenize{mckinney_05_practice_04:mckinney-chapter-5-practice-section-4-wednesday-11-45-am}}\label{\detokenize{mckinney_05_practice_04::doc}}

\subsection{Announcements}
\label{\detokenize{mckinney_05_practice_04:announcements}}\begin{itemize}
\item {} 
\sphinxAtStartPar
Due on Friday 2/3 at 11:59 PM
\begin{itemize}
\item {} 
\sphinxAtStartPar
10,000 Total XP, but \sphinxstyleemphasis{\sphinxstylestrong{submit a PDF of your home page with total XP to Gradescope}}

\item {} 
\sphinxAtStartPar
quiz 2

\end{itemize}

\item {} 
\sphinxAtStartPar
I will open project groups at 9 AM on Monday 2/6

\item {} 
\sphinxAtStartPar
I will close project groups at 9 AM on Monday 2/13 and randomly assign unassigned students

\end{itemize}


\subsection{Practice}
\label{\detokenize{mckinney_05_practice_04:practice}}
\begin{sphinxuseclass}{cell}\begin{sphinxVerbatimInput}

\begin{sphinxuseclass}{cell_input}
\begin{sphinxVerbatim}[commandchars=\\\{\}]
\PYG{k+kn}{import} \PYG{n+nn}{matplotlib}\PYG{n+nn}{.}\PYG{n+nn}{pyplot} \PYG{k}{as} \PYG{n+nn}{plt}
\PYG{k+kn}{import} \PYG{n+nn}{numpy} \PYG{k}{as} \PYG{n+nn}{np}
\PYG{k+kn}{import} \PYG{n+nn}{pandas} \PYG{k}{as} \PYG{n+nn}{pd}

\PYG{k+kn}{import} \PYG{n+nn}{yfinance} \PYG{k}{as} \PYG{n+nn}{yf}
\PYG{k+kn}{import} \PYG{n+nn}{requests\PYGZus{}cache}
\end{sphinxVerbatim}

\end{sphinxuseclass}\end{sphinxVerbatimInput}

\end{sphinxuseclass}
\begin{sphinxuseclass}{cell}\begin{sphinxVerbatimInput}

\begin{sphinxuseclass}{cell_input}
\begin{sphinxVerbatim}[commandchars=\\\{\}]
\PYG{o}{\PYGZpc{}}\PYG{k}{config} InlineBackend.figure\PYGZus{}format = \PYGZsq{}retina\PYGZsq{}
\PYG{o}{\PYGZpc{}}\PYG{k}{precision} 4
\PYG{n}{pd}\PYG{o}{.}\PYG{n}{options}\PYG{o}{.}\PYG{n}{display}\PYG{o}{.}\PYG{n}{float\PYGZus{}format} \PYG{o}{=} \PYG{l+s+s1}{\PYGZsq{}}\PYG{l+s+si}{\PYGZob{}:.4f\PYGZcb{}}\PYG{l+s+s1}{\PYGZsq{}}\PYG{o}{.}\PYG{n}{format}
\PYG{n}{session} \PYG{o}{=} \PYG{n}{requests\PYGZus{}cache}\PYG{o}{.}\PYG{n}{CachedSession}\PYG{p}{(}\PYG{p}{)}
\end{sphinxVerbatim}

\end{sphinxuseclass}\end{sphinxVerbatimInput}

\end{sphinxuseclass}
\begin{sphinxuseclass}{cell}\begin{sphinxVerbatimInput}

\begin{sphinxuseclass}{cell_input}
\begin{sphinxVerbatim}[commandchars=\\\{\}]
\PYG{n}{tickers} \PYG{o}{=} \PYG{n}{yf}\PYG{o}{.}\PYG{n}{Tickers}\PYG{p}{(}\PYG{l+s+s1}{\PYGZsq{}}\PYG{l+s+s1}{AAPL IBM MSFT GOOG}\PYG{l+s+s1}{\PYGZsq{}}\PYG{p}{,} \PYG{n}{session}\PYG{o}{=}\PYG{n}{session}\PYG{p}{)}
\PYG{n}{prices} \PYG{o}{=} \PYG{n}{tickers}\PYG{o}{.}\PYG{n}{history}\PYG{p}{(}\PYG{n}{period}\PYG{o}{=}\PYG{l+s+s1}{\PYGZsq{}}\PYG{l+s+s1}{max}\PYG{l+s+s1}{\PYGZsq{}}\PYG{p}{,} \PYG{n}{auto\PYGZus{}adjust}\PYG{o}{=}\PYG{k+kc}{False}\PYG{p}{,} \PYG{n}{progress}\PYG{o}{=}\PYG{k+kc}{False}\PYG{p}{)}
\PYG{n}{prices}\PYG{o}{.}\PYG{n}{index} \PYG{o}{=} \PYG{n}{prices}\PYG{o}{.}\PYG{n}{index}\PYG{o}{.}\PYG{n}{tz\PYGZus{}localize}\PYG{p}{(}\PYG{k+kc}{None}\PYG{p}{)}
\PYG{n}{returns} \PYG{o}{=} \PYG{n}{prices}\PYG{p}{[}\PYG{l+s+s1}{\PYGZsq{}}\PYG{l+s+s1}{Adj Close}\PYG{l+s+s1}{\PYGZsq{}}\PYG{p}{]}\PYG{o}{.}\PYG{n}{pct\PYGZus{}change}\PYG{p}{(}\PYG{p}{)}\PYG{o}{.}\PYG{n}{dropna}\PYG{p}{(}\PYG{p}{)}
\PYG{n}{returns}
\end{sphinxVerbatim}

\end{sphinxuseclass}\end{sphinxVerbatimInput}
\begin{sphinxVerbatimOutput}

\begin{sphinxuseclass}{cell_output}
\begin{sphinxVerbatim}[commandchars=\\\{\}]
[*********************100\PYGZpc{}***********************]  4 of 4 completed
\end{sphinxVerbatim}

\begin{sphinxVerbatim}[commandchars=\\\{\}]
              AAPL    GOOG     IBM    MSFT
Date                                      
2004\PYGZhy{}08\PYGZhy{}20  0.0029  0.0794  0.0042  0.0029
2004\PYGZhy{}08\PYGZhy{}23  0.0091  0.0101 \PYGZhy{}0.0070  0.0044
2004\PYGZhy{}08\PYGZhy{}24  0.0280 \PYGZhy{}0.0414  0.0007  0.0000
2004\PYGZhy{}08\PYGZhy{}25  0.0344  0.0108  0.0042  0.0114
2004\PYGZhy{}08\PYGZhy{}26  0.0487  0.0180 \PYGZhy{}0.0045 \PYGZhy{}0.0040
...            ...     ...     ...     ...
2023\PYGZhy{}01\PYGZhy{}26  0.0148  0.0251 \PYGZhy{}0.0448  0.0307
2023\PYGZhy{}01\PYGZhy{}27  0.0137  0.0156 \PYGZhy{}0.0004  0.0006
2023\PYGZhy{}01\PYGZhy{}30 \PYGZhy{}0.0201 \PYGZhy{}0.0274  0.0068 \PYGZhy{}0.0220
2023\PYGZhy{}01\PYGZhy{}31  0.0090  0.0196 \PYGZhy{}0.0042  0.0210
2023\PYGZhy{}02\PYGZhy{}01  0.0079  0.0156  0.0027  0.0199

[4645 rows x 4 columns]
\end{sphinxVerbatim}

\end{sphinxuseclass}\end{sphinxVerbatimOutput}

\end{sphinxuseclass}

\subsubsection{What are the mean daily returns for these four stocks?}
\label{\detokenize{mckinney_05_practice_04:what-are-the-mean-daily-returns-for-these-four-stocks}}
\begin{sphinxuseclass}{cell}\begin{sphinxVerbatimInput}

\begin{sphinxuseclass}{cell_input}
\begin{sphinxVerbatim}[commandchars=\\\{\}]
\PYG{n}{returns}\PYG{o}{.}\PYG{n}{mean}\PYG{p}{(}\PYG{p}{)}
\end{sphinxVerbatim}

\end{sphinxuseclass}\end{sphinxVerbatimInput}
\begin{sphinxVerbatimOutput}

\begin{sphinxuseclass}{cell_output}
\begin{sphinxVerbatim}[commandchars=\\\{\}]
AAPL   0.0015
GOOG   0.0010
IBM    0.0003
MSFT   0.0007
dtype: float64
\end{sphinxVerbatim}

\end{sphinxuseclass}\end{sphinxVerbatimOutput}

\end{sphinxuseclass}

\subsubsection{What are the standard deviations of daily returns for these four stocks?}
\label{\detokenize{mckinney_05_practice_04:what-are-the-standard-deviations-of-daily-returns-for-these-four-stocks}}
\begin{sphinxuseclass}{cell}\begin{sphinxVerbatimInput}

\begin{sphinxuseclass}{cell_input}
\begin{sphinxVerbatim}[commandchars=\\\{\}]
\PYG{n}{returns}\PYG{o}{.}\PYG{n}{std}\PYG{p}{(}\PYG{p}{)}
\end{sphinxVerbatim}

\end{sphinxuseclass}\end{sphinxVerbatimInput}
\begin{sphinxVerbatimOutput}

\begin{sphinxuseclass}{cell_output}
\begin{sphinxVerbatim}[commandchars=\\\{\}]
AAPL   0.0210
GOOG   0.0194
IBM    0.0144
MSFT   0.0173
dtype: float64
\end{sphinxVerbatim}

\end{sphinxuseclass}\end{sphinxVerbatimOutput}

\end{sphinxuseclass}

\subsubsection{What are the \sphinxstyleemphasis{annualized} means and standard deviations of daily returns for these four stocks?}
\label{\detokenize{mckinney_05_practice_04:what-are-the-annualized-means-and-standard-deviations-of-daily-returns-for-these-four-stocks}}
\begin{sphinxuseclass}{cell}\begin{sphinxVerbatimInput}

\begin{sphinxuseclass}{cell_input}
\begin{sphinxVerbatim}[commandchars=\\\{\}]
\PYG{n}{returns}\PYG{o}{.}\PYG{n}{mean}\PYG{p}{(}\PYG{p}{)}\PYG{o}{.}\PYG{n}{mul}\PYG{p}{(}\PYG{l+m+mi}{252}\PYG{p}{)}
\end{sphinxVerbatim}

\end{sphinxuseclass}\end{sphinxVerbatimInput}
\begin{sphinxVerbatimOutput}

\begin{sphinxuseclass}{cell_output}
\begin{sphinxVerbatim}[commandchars=\\\{\}]
AAPL   0.3670
GOOG   0.2480
IBM    0.0822
MSFT   0.1836
dtype: float64
\end{sphinxVerbatim}

\end{sphinxuseclass}\end{sphinxVerbatimOutput}

\end{sphinxuseclass}
\begin{sphinxuseclass}{cell}\begin{sphinxVerbatimInput}

\begin{sphinxuseclass}{cell_input}
\begin{sphinxVerbatim}[commandchars=\\\{\}]
\PYG{n}{returns}\PYG{o}{.}\PYG{n}{std}\PYG{p}{(}\PYG{p}{)}\PYG{o}{.}\PYG{n}{mul}\PYG{p}{(}\PYG{n}{np}\PYG{o}{.}\PYG{n}{sqrt}\PYG{p}{(}\PYG{l+m+mi}{252}\PYG{p}{)}\PYG{p}{)}
\end{sphinxVerbatim}

\end{sphinxuseclass}\end{sphinxVerbatimInput}
\begin{sphinxVerbatimOutput}

\begin{sphinxuseclass}{cell_output}
\begin{sphinxVerbatim}[commandchars=\\\{\}]
AAPL   0.3331
GOOG   0.3077
IBM    0.2294
MSFT   0.2739
dtype: float64
\end{sphinxVerbatim}

\end{sphinxuseclass}\end{sphinxVerbatimOutput}

\end{sphinxuseclass}
\sphinxAtStartPar
\sphinxstyleemphasis{\sphinxstylestrong{The best explanation I have found on why stock return volatility (the standard deviation of stocks returns) grows with the square root of time is at the bottom of page 7 of \sphinxhref{https://book.ivo-welch.info/read/source5.mba/08-invchoice.pdf}{chapter 8 of Ivo Welch’s free corporate finance textbook}.}}


\subsubsection{Plot \sphinxstyleemphasis{annualized} means versus standard deviations of daily returns for these four stocks}
\label{\detokenize{mckinney_05_practice_04:plot-annualized-means-versus-standard-deviations-of-daily-returns-for-these-four-stocks}}
\sphinxAtStartPar
Use \sphinxcode{\sphinxupquote{plt.scatter()}}, which expects arguments as \sphinxcode{\sphinxupquote{x}} (standard deviations) then \sphinxcode{\sphinxupquote{y}} (means).

\begin{sphinxuseclass}{cell}\begin{sphinxVerbatimInput}

\begin{sphinxuseclass}{cell_input}
\begin{sphinxVerbatim}[commandchars=\\\{\}]
\PYG{n}{plt}\PYG{o}{.}\PYG{n}{scatter}\PYG{p}{(}
    \PYG{n}{returns}\PYG{o}{.}\PYG{n}{std}\PYG{p}{(}\PYG{p}{)}\PYG{o}{.}\PYG{n}{mul}\PYG{p}{(}\PYG{n}{np}\PYG{o}{.}\PYG{n}{sqrt}\PYG{p}{(}\PYG{l+m+mi}{252}\PYG{p}{)}\PYG{p}{)}\PYG{o}{.}\PYG{n}{mul}\PYG{p}{(}\PYG{l+m+mi}{100}\PYG{p}{)}\PYG{p}{,}
    \PYG{n}{returns}\PYG{o}{.}\PYG{n}{mean}\PYG{p}{(}\PYG{p}{)}\PYG{o}{.}\PYG{n}{mul}\PYG{p}{(}\PYG{l+m+mi}{252}\PYG{p}{)}\PYG{o}{.}\PYG{n}{mul}\PYG{p}{(}\PYG{l+m+mi}{100}\PYG{p}{)}
\PYG{p}{)}
\PYG{n}{plt}\PYG{o}{.}\PYG{n}{xlabel}\PYG{p}{(}\PYG{l+s+s1}{\PYGZsq{}}\PYG{l+s+s1}{Annualized Standard Deviation of Returns (}\PYG{l+s+s1}{\PYGZpc{}}\PYG{l+s+s1}{)}\PYG{l+s+s1}{\PYGZsq{}}\PYG{p}{)}
\PYG{n}{plt}\PYG{o}{.}\PYG{n}{ylabel}\PYG{p}{(}\PYG{l+s+s1}{\PYGZsq{}}\PYG{l+s+s1}{Annualized Mean of Returns (}\PYG{l+s+s1}{\PYGZpc{}}\PYG{l+s+s1}{)}\PYG{l+s+s1}{\PYGZsq{}}\PYG{p}{)}
\PYG{n}{plt}\PYG{o}{.}\PYG{n}{title}\PYG{p}{(}\PYG{l+s+sa}{f}\PYG{l+s+s1}{\PYGZsq{}}\PYG{l+s+s1}{Return Versus Risk}\PYG{l+s+se}{\PYGZbs{}n}\PYG{l+s+s1}{ from }\PYG{l+s+si}{\PYGZob{}}\PYG{n}{returns}\PYG{o}{.}\PYG{n}{index}\PYG{p}{[}\PYG{l+m+mi}{0}\PYG{p}{]}\PYG{l+s+si}{:}\PYG{l+s+s1}{\PYGZpc{}B \PYGZpc{}Y}\PYG{l+s+si}{\PYGZcb{}}\PYG{l+s+s1}{ to }\PYG{l+s+si}{\PYGZob{}}\PYG{n}{returns}\PYG{o}{.}\PYG{n}{index}\PYG{p}{[}\PYG{o}{\PYGZhy{}}\PYG{l+m+mi}{1}\PYG{p}{]}\PYG{l+s+si}{:}\PYG{l+s+s1}{\PYGZpc{}B \PYGZpc{}Y}\PYG{l+s+si}{\PYGZcb{}}\PYG{l+s+s1}{\PYGZsq{}}\PYG{p}{)}
\PYG{n}{plt}\PYG{o}{.}\PYG{n}{grid}\PYG{p}{(}\PYG{p}{)}

\PYG{k}{for} \PYG{n}{i} \PYG{o+ow}{in} \PYG{n}{returns}\PYG{p}{:}
    \PYG{n}{plt}\PYG{o}{.}\PYG{n}{text}\PYG{p}{(}
        \PYG{n}{x}\PYG{o}{=}\PYG{n}{returns}\PYG{p}{[}\PYG{n}{i}\PYG{p}{]}\PYG{o}{.}\PYG{n}{std}\PYG{p}{(}\PYG{p}{)} \PYG{o}{*} \PYG{n}{np}\PYG{o}{.}\PYG{n}{sqrt}\PYG{p}{(}\PYG{l+m+mi}{252}\PYG{p}{)} \PYG{o}{*} \PYG{l+m+mi}{100}\PYG{p}{,}
        \PYG{n}{y}\PYG{o}{=}\PYG{n}{returns}\PYG{p}{[}\PYG{n}{i}\PYG{p}{]}\PYG{o}{.}\PYG{n}{mean}\PYG{p}{(}\PYG{p}{)} \PYG{o}{*} \PYG{l+m+mi}{252} \PYG{o}{*} \PYG{l+m+mi}{100}\PYG{p}{,}
        \PYG{n}{s}\PYG{o}{=}\PYG{n}{i}
    \PYG{p}{)}

\PYG{n}{plt}\PYG{o}{.}\PYG{n}{show}\PYG{p}{(}\PYG{p}{)}
\end{sphinxVerbatim}

\end{sphinxuseclass}\end{sphinxVerbatimInput}
\begin{sphinxVerbatimOutput}

\begin{sphinxuseclass}{cell_output}
\noindent\sphinxincludegraphics{{81068732707ee4fdae68ad79b1adb968a3deee5f7f285f4735c67dd5c51e8836}.png}

\end{sphinxuseclass}\end{sphinxVerbatimOutput}

\end{sphinxuseclass}

\subsubsection{Repeat the previous calculations and plot for the stocks in the Dow\sphinxhyphen{}Jones Industrial Index (DJIA)}
\label{\detokenize{mckinney_05_practice_04:repeat-the-previous-calculations-and-plot-for-the-stocks-in-the-dow-jones-industrial-index-djia}}
\sphinxAtStartPar
We can find the current DJIA stocks on \sphinxhref{https://en.wikipedia.org/wiki/Dow\_Jones\_Industrial\_Average}{Wikipedia}.
We will need to download new data, into \sphinxcode{\sphinxupquote{tickers2}}, \sphinxcode{\sphinxupquote{prices2}}, and \sphinxcode{\sphinxupquote{returns2}}.

\begin{sphinxuseclass}{cell}\begin{sphinxVerbatimInput}

\begin{sphinxuseclass}{cell_input}
\begin{sphinxVerbatim}[commandchars=\\\{\}]
\PYG{n}{wiki} \PYG{o}{=} \PYG{n}{pd}\PYG{o}{.}\PYG{n}{read\PYGZus{}html}\PYG{p}{(}\PYG{l+s+s1}{\PYGZsq{}}\PYG{l+s+s1}{https://en.wikipedia.org/wiki/Dow\PYGZus{}Jones\PYGZus{}Industrial\PYGZus{}Average}\PYG{l+s+s1}{\PYGZsq{}}\PYG{p}{)}
\end{sphinxVerbatim}

\end{sphinxuseclass}\end{sphinxVerbatimInput}

\end{sphinxuseclass}
\begin{sphinxuseclass}{cell}\begin{sphinxVerbatimInput}

\begin{sphinxuseclass}{cell_input}
\begin{sphinxVerbatim}[commandchars=\\\{\}]
\PYG{n}{symbols} \PYG{o}{=} \PYG{n}{wiki}\PYG{p}{[}\PYG{l+m+mi}{1}\PYG{p}{]}\PYG{p}{[}\PYG{l+s+s1}{\PYGZsq{}}\PYG{l+s+s1}{Symbol}\PYG{l+s+s1}{\PYGZsq{}}\PYG{p}{]}
\end{sphinxVerbatim}

\end{sphinxuseclass}\end{sphinxVerbatimInput}

\end{sphinxuseclass}
\begin{sphinxuseclass}{cell}\begin{sphinxVerbatimInput}

\begin{sphinxuseclass}{cell_input}
\begin{sphinxVerbatim}[commandchars=\\\{\}]
\PYG{n}{tickers\PYGZus{}2} \PYG{o}{=} \PYG{n}{yf}\PYG{o}{.}\PYG{n}{Tickers}\PYG{p}{(}\PYG{n}{tickers}\PYG{o}{=}\PYG{n}{symbols}\PYG{o}{.}\PYG{n}{to\PYGZus{}list}\PYG{p}{(}\PYG{p}{)}\PYG{p}{,} \PYG{n}{session}\PYG{o}{=}\PYG{n}{session}\PYG{p}{)}
\PYG{n}{prices\PYGZus{}2} \PYG{o}{=} \PYG{n}{tickers\PYGZus{}2}\PYG{o}{.}\PYG{n}{history}\PYG{p}{(}\PYG{n}{period}\PYG{o}{=}\PYG{l+s+s1}{\PYGZsq{}}\PYG{l+s+s1}{max}\PYG{l+s+s1}{\PYGZsq{}}\PYG{p}{,} \PYG{n}{auto\PYGZus{}adjust}\PYG{o}{=}\PYG{k+kc}{False}\PYG{p}{,} \PYG{n}{progress}\PYG{o}{=}\PYG{k+kc}{False}\PYG{p}{)}
\PYG{n}{prices\PYGZus{}2}\PYG{o}{.}\PYG{n}{index} \PYG{o}{=} \PYG{n}{prices\PYGZus{}2}\PYG{o}{.}\PYG{n}{index}\PYG{o}{.}\PYG{n}{tz\PYGZus{}localize}\PYG{p}{(}\PYG{k+kc}{None}\PYG{p}{)}
\PYG{n}{returns\PYGZus{}2} \PYG{o}{=} \PYG{n}{prices\PYGZus{}2}\PYG{p}{[}\PYG{l+s+s1}{\PYGZsq{}}\PYG{l+s+s1}{Adj Close}\PYG{l+s+s1}{\PYGZsq{}}\PYG{p}{]}\PYG{o}{.}\PYG{n}{pct\PYGZus{}change}\PYG{p}{(}\PYG{p}{)}\PYG{o}{.}\PYG{n}{dropna}\PYG{p}{(}\PYG{p}{)}
\PYG{n}{returns\PYGZus{}2}\PYG{o}{.}\PYG{n}{columns}\PYG{o}{.}\PYG{n}{name} \PYG{o}{=} \PYG{l+s+s1}{\PYGZsq{}}\PYG{l+s+s1}{Ticker}\PYG{l+s+s1}{\PYGZsq{}}
\end{sphinxVerbatim}

\end{sphinxuseclass}\end{sphinxVerbatimInput}
\begin{sphinxVerbatimOutput}

\begin{sphinxuseclass}{cell_output}
\begin{sphinxVerbatim}[commandchars=\\\{\}]
[*********************100\PYGZpc{}***********************]  30 of 30 completed
\end{sphinxVerbatim}

\end{sphinxuseclass}\end{sphinxVerbatimOutput}

\end{sphinxuseclass}
\begin{sphinxuseclass}{cell}\begin{sphinxVerbatimInput}

\begin{sphinxuseclass}{cell_input}
\begin{sphinxVerbatim}[commandchars=\\\{\}]
\PYG{p}{(}
    \PYG{n}{prices\PYGZus{}2}\PYG{p}{[}\PYG{l+s+s1}{\PYGZsq{}}\PYG{l+s+s1}{Adj Close}\PYG{l+s+s1}{\PYGZsq{}}\PYG{p}{]}
    \PYG{o}{.}\PYG{n}{count}\PYG{p}{(}\PYG{p}{)}
    \PYG{o}{.}\PYG{n}{div}\PYG{p}{(}\PYG{l+m+mi}{252}\PYG{p}{)}
    \PYG{o}{.}\PYG{n}{sort\PYGZus{}values}\PYG{p}{(}\PYG{p}{)}
    \PYG{o}{.}\PYG{n}{plot}\PYG{p}{(}\PYG{n}{kind}\PYG{o}{=}\PYG{l+s+s1}{\PYGZsq{}}\PYG{l+s+s1}{barh}\PYG{l+s+s1}{\PYGZsq{}}\PYG{p}{)}
\PYG{p}{)}
\PYG{n}{plt}\PYG{o}{.}\PYG{n}{xlabel}\PYG{p}{(}\PYG{l+s+s1}{\PYGZsq{}}\PYG{l+s+s1}{Year of Complete Data}\PYG{l+s+s1}{\PYGZsq{}}\PYG{p}{)}
\PYG{n}{plt}\PYG{o}{.}\PYG{n}{title}\PYG{p}{(}\PYG{l+s+s1}{\PYGZsq{}}\PYG{l+s+s1}{Data Availability for DJIA Stocks}\PYG{l+s+s1}{\PYGZsq{}}\PYG{p}{)}
\PYG{n}{plt}\PYG{o}{.}\PYG{n}{show}\PYG{p}{(}\PYG{p}{)}
\end{sphinxVerbatim}

\end{sphinxuseclass}\end{sphinxVerbatimInput}
\begin{sphinxVerbatimOutput}

\begin{sphinxuseclass}{cell_output}
\noindent\sphinxincludegraphics{{48fb2a22953716a02201a516e7759521befddae5e9ed74bb3e1c29d8d84dd7be}.png}

\end{sphinxuseclass}\end{sphinxVerbatimOutput}

\end{sphinxuseclass}
\begin{sphinxuseclass}{cell}\begin{sphinxVerbatimInput}

\begin{sphinxuseclass}{cell_input}
\begin{sphinxVerbatim}[commandchars=\\\{\}]
\PYG{n}{plt}\PYG{o}{.}\PYG{n}{scatter}\PYG{p}{(}
    \PYG{n}{returns\PYGZus{}2}\PYG{o}{.}\PYG{n}{std}\PYG{p}{(}\PYG{p}{)}\PYG{o}{.}\PYG{n}{mul}\PYG{p}{(}\PYG{n}{np}\PYG{o}{.}\PYG{n}{sqrt}\PYG{p}{(}\PYG{l+m+mi}{252}\PYG{p}{)}\PYG{p}{)}\PYG{o}{.}\PYG{n}{mul}\PYG{p}{(}\PYG{l+m+mi}{100}\PYG{p}{)}\PYG{p}{,}
    \PYG{n}{returns\PYGZus{}2}\PYG{o}{.}\PYG{n}{mean}\PYG{p}{(}\PYG{p}{)}\PYG{o}{.}\PYG{n}{mul}\PYG{p}{(}\PYG{l+m+mi}{252}\PYG{p}{)}\PYG{o}{.}\PYG{n}{mul}\PYG{p}{(}\PYG{l+m+mi}{100}\PYG{p}{)}
\PYG{p}{)}
\PYG{n}{plt}\PYG{o}{.}\PYG{n}{xlabel}\PYG{p}{(}\PYG{l+s+s1}{\PYGZsq{}}\PYG{l+s+s1}{Annualized Standard Deviation of Returns (}\PYG{l+s+s1}{\PYGZpc{}}\PYG{l+s+s1}{)}\PYG{l+s+s1}{\PYGZsq{}}\PYG{p}{)}
\PYG{n}{plt}\PYG{o}{.}\PYG{n}{ylabel}\PYG{p}{(}\PYG{l+s+s1}{\PYGZsq{}}\PYG{l+s+s1}{Annualized Mean of Returns (}\PYG{l+s+s1}{\PYGZpc{}}\PYG{l+s+s1}{)}\PYG{l+s+s1}{\PYGZsq{}}\PYG{p}{)}
\PYG{n}{plt}\PYG{o}{.}\PYG{n}{title}\PYG{p}{(}\PYG{l+s+sa}{f}\PYG{l+s+s1}{\PYGZsq{}}\PYG{l+s+s1}{Return Versus Risk for DJIA Stocks}\PYG{l+s+se}{\PYGZbs{}n}\PYG{l+s+s1}{ from }\PYG{l+s+si}{\PYGZob{}}\PYG{n}{returns\PYGZus{}2}\PYG{o}{.}\PYG{n}{index}\PYG{p}{[}\PYG{l+m+mi}{0}\PYG{p}{]}\PYG{l+s+si}{:}\PYG{l+s+s1}{\PYGZpc{}B \PYGZpc{}Y}\PYG{l+s+si}{\PYGZcb{}}\PYG{l+s+s1}{ to }\PYG{l+s+si}{\PYGZob{}}\PYG{n}{returns\PYGZus{}2}\PYG{o}{.}\PYG{n}{index}\PYG{p}{[}\PYG{o}{\PYGZhy{}}\PYG{l+m+mi}{1}\PYG{p}{]}\PYG{l+s+si}{:}\PYG{l+s+s1}{\PYGZpc{}B \PYGZpc{}Y}\PYG{l+s+si}{\PYGZcb{}}\PYG{l+s+s1}{\PYGZsq{}}\PYG{p}{)}
\PYG{n}{plt}\PYG{o}{.}\PYG{n}{grid}\PYG{p}{(}\PYG{p}{)}

\PYG{k}{for} \PYG{n}{i} \PYG{o+ow}{in} \PYG{n}{returns\PYGZus{}2}\PYG{p}{:}
    \PYG{n}{plt}\PYG{o}{.}\PYG{n}{text}\PYG{p}{(}
        \PYG{n}{x}\PYG{o}{=}\PYG{n}{returns\PYGZus{}2}\PYG{p}{[}\PYG{n}{i}\PYG{p}{]}\PYG{o}{.}\PYG{n}{std}\PYG{p}{(}\PYG{p}{)} \PYG{o}{*} \PYG{n}{np}\PYG{o}{.}\PYG{n}{sqrt}\PYG{p}{(}\PYG{l+m+mi}{252}\PYG{p}{)} \PYG{o}{*} \PYG{l+m+mi}{100}\PYG{p}{,}
        \PYG{n}{y}\PYG{o}{=}\PYG{n}{returns\PYGZus{}2}\PYG{p}{[}\PYG{n}{i}\PYG{p}{]}\PYG{o}{.}\PYG{n}{mean}\PYG{p}{(}\PYG{p}{)} \PYG{o}{*} \PYG{l+m+mi}{252} \PYG{o}{*} \PYG{l+m+mi}{100}\PYG{p}{,}
        \PYG{n}{s}\PYG{o}{=}\PYG{n}{i}
    \PYG{p}{)}
\end{sphinxVerbatim}

\end{sphinxuseclass}\end{sphinxVerbatimInput}
\begin{sphinxVerbatimOutput}

\begin{sphinxuseclass}{cell_output}
\noindent\sphinxincludegraphics{{6fd5067bd58dfed5a5ffaafeb478cc7aeb32902edc501dcb17af22a18ca700d0}.png}

\end{sphinxuseclass}\end{sphinxVerbatimOutput}

\end{sphinxuseclass}
\sphinxAtStartPar
\sphinxstyleemphasis{\sphinxstylestrong{THERE IS NO RELATION BETWEEN RETURNS AND RISK FOR SINGLE STOCKS!}}

\begin{sphinxuseclass}{cell}\begin{sphinxVerbatimInput}

\begin{sphinxuseclass}{cell_input}
\begin{sphinxVerbatim}[commandchars=\\\{\}]
\PYG{n}{np}\PYG{o}{.}\PYG{n}{corrcoef}\PYG{p}{(}\PYG{n}{returns\PYGZus{}2}\PYG{o}{.}\PYG{n}{mean}\PYG{p}{(}\PYG{p}{)}\PYG{p}{,} \PYG{n}{returns\PYGZus{}2}\PYG{o}{.}\PYG{n}{std}\PYG{p}{(}\PYG{p}{)}\PYG{p}{)}
\end{sphinxVerbatim}

\end{sphinxuseclass}\end{sphinxVerbatimInput}
\begin{sphinxVerbatimOutput}

\begin{sphinxuseclass}{cell_output}
\begin{sphinxVerbatim}[commandchars=\\\{\}]
array([[1.    , 0.0186],
       [0.0186, 1.    ]])
\end{sphinxVerbatim}

\end{sphinxuseclass}\end{sphinxVerbatimOutput}

\end{sphinxuseclass}

\subsubsection{Calculate total returns for the stocks in the DJIA}
\label{\detokenize{mckinney_05_practice_04:calculate-total-returns-for-the-stocks-in-the-djia}}
\sphinxAtStartPar
We can use the \sphinxcode{\sphinxupquote{.prod()}} method to compound returns as \$1 + R\_T = \textbackslash{}prod\_\{t=1\}\textasciicircum{}T (1 + R\_t)\$.
Technically, we should write \$R\_T\$ as \$R\_\{0,T\}\$, but we typically omit the subscript \$0\$.

\begin{sphinxuseclass}{cell}\begin{sphinxVerbatimInput}

\begin{sphinxuseclass}{cell_input}
\begin{sphinxVerbatim}[commandchars=\\\{\}]
\PYG{p}{(}
    \PYG{n}{returns\PYGZus{}2}
    \PYG{o}{.}\PYG{n}{add}\PYG{p}{(}\PYG{l+m+mi}{1}\PYG{p}{)}
    \PYG{o}{.}\PYG{n}{prod}\PYG{p}{(}\PYG{p}{)}
    \PYG{o}{.}\PYG{n}{sub}\PYG{p}{(}\PYG{l+m+mi}{1}\PYG{p}{)}
    \PYG{o}{.}\PYG{n}{mul}\PYG{p}{(}\PYG{l+m+mi}{100}\PYG{p}{)}
    \PYG{o}{.}\PYG{n}{sort\PYGZus{}values}\PYG{p}{(}\PYG{p}{)}
    \PYG{o}{.}\PYG{n}{plot}\PYG{p}{(}\PYG{n}{kind}\PYG{o}{=}\PYG{l+s+s1}{\PYGZsq{}}\PYG{l+s+s1}{barh}\PYG{l+s+s1}{\PYGZsq{}}\PYG{p}{)}
\PYG{p}{)}
\PYG{n}{plt}\PYG{o}{.}\PYG{n}{xlabel}\PYG{p}{(}\PYG{l+s+s1}{\PYGZsq{}}\PYG{l+s+s1}{Total Return (}\PYG{l+s+s1}{\PYGZpc{}}\PYG{l+s+s1}{)}\PYG{l+s+s1}{\PYGZsq{}}\PYG{p}{)}
\PYG{n}{plt}\PYG{o}{.}\PYG{n}{title}\PYG{p}{(}\PYG{l+s+sa}{f}\PYG{l+s+s1}{\PYGZsq{}}\PYG{l+s+s1}{Total Returns for DJIA Stocks }\PYG{l+s+se}{\PYGZbs{}n}\PYG{l+s+s1}{ from }\PYG{l+s+si}{\PYGZob{}}\PYG{n}{returns}\PYG{o}{.}\PYG{n}{index}\PYG{p}{[}\PYG{l+m+mi}{0}\PYG{p}{]}\PYG{l+s+si}{:}\PYG{l+s+s1}{\PYGZpc{}B \PYGZpc{}Y}\PYG{l+s+si}{\PYGZcb{}}\PYG{l+s+s1}{ to }\PYG{l+s+si}{\PYGZob{}}\PYG{n}{returns}\PYG{o}{.}\PYG{n}{index}\PYG{p}{[}\PYG{o}{\PYGZhy{}}\PYG{l+m+mi}{1}\PYG{p}{]}\PYG{l+s+si}{:}\PYG{l+s+s1}{\PYGZpc{}B \PYGZpc{}Y}\PYG{l+s+si}{\PYGZcb{}}\PYG{l+s+s1}{\PYGZsq{}}\PYG{p}{)}
\PYG{n}{plt}\PYG{o}{.}\PYG{n}{show}\PYG{p}{(}\PYG{p}{)}
\end{sphinxVerbatim}

\end{sphinxuseclass}\end{sphinxVerbatimInput}
\begin{sphinxVerbatimOutput}

\begin{sphinxuseclass}{cell_output}
\noindent\sphinxincludegraphics{{8f75b4eb4a045ae4ad1adc453b75fd89f2400632de02b33ed1ed271f166bd17b}.png}

\end{sphinxuseclass}\end{sphinxVerbatimOutput}

\end{sphinxuseclass}

\subsubsection{Plot the distribution of total returns for the stocks in the DJIA}
\label{\detokenize{mckinney_05_practice_04:plot-the-distribution-of-total-returns-for-the-stocks-in-the-djia}}
\sphinxAtStartPar
We can plot a histogram, using either the \sphinxcode{\sphinxupquote{plt.hist()}} function or the \sphinxcode{\sphinxupquote{.plot(kind='hist')}} method.

\begin{sphinxuseclass}{cell}\begin{sphinxVerbatimInput}

\begin{sphinxuseclass}{cell_input}
\begin{sphinxVerbatim}[commandchars=\\\{\}]
\PYG{p}{(}
    \PYG{n}{returns\PYGZus{}2}
    \PYG{o}{.}\PYG{n}{add}\PYG{p}{(}\PYG{l+m+mi}{1}\PYG{p}{)}
    \PYG{o}{.}\PYG{n}{prod}\PYG{p}{(}\PYG{p}{)}
    \PYG{o}{.}\PYG{n}{sub}\PYG{p}{(}\PYG{l+m+mi}{1}\PYG{p}{)}
    \PYG{o}{.}\PYG{n}{mul}\PYG{p}{(}\PYG{l+m+mi}{100}\PYG{p}{)}
    \PYG{o}{.}\PYG{n}{plot}\PYG{p}{(}\PYG{n}{kind}\PYG{o}{=}\PYG{l+s+s1}{\PYGZsq{}}\PYG{l+s+s1}{hist}\PYG{l+s+s1}{\PYGZsq{}}\PYG{p}{,} \PYG{n}{bins}\PYG{o}{=}\PYG{l+m+mi}{20}\PYG{p}{)}
\PYG{p}{)}
\PYG{n}{plt}\PYG{o}{.}\PYG{n}{xlabel}\PYG{p}{(}\PYG{l+s+s1}{\PYGZsq{}}\PYG{l+s+s1}{Total Returns (}\PYG{l+s+s1}{\PYGZpc{}}\PYG{l+s+s1}{)}\PYG{l+s+s1}{\PYGZsq{}}\PYG{p}{)}
\PYG{n}{plt}\PYG{o}{.}\PYG{n}{title}\PYG{p}{(}\PYG{l+s+sa}{f}\PYG{l+s+s1}{\PYGZsq{}}\PYG{l+s+s1}{Total Returns for DJIA Stocks }\PYG{l+s+se}{\PYGZbs{}n}\PYG{l+s+s1}{ from }\PYG{l+s+si}{\PYGZob{}}\PYG{n}{returns}\PYG{o}{.}\PYG{n}{index}\PYG{p}{[}\PYG{l+m+mi}{0}\PYG{p}{]}\PYG{l+s+si}{:}\PYG{l+s+s1}{\PYGZpc{}B \PYGZpc{}Y}\PYG{l+s+si}{\PYGZcb{}}\PYG{l+s+s1}{ to }\PYG{l+s+si}{\PYGZob{}}\PYG{n}{returns}\PYG{o}{.}\PYG{n}{index}\PYG{p}{[}\PYG{o}{\PYGZhy{}}\PYG{l+m+mi}{1}\PYG{p}{]}\PYG{l+s+si}{:}\PYG{l+s+s1}{\PYGZpc{}B \PYGZpc{}Y}\PYG{l+s+si}{\PYGZcb{}}\PYG{l+s+s1}{\PYGZsq{}}\PYG{p}{)}
\PYG{n}{plt}\PYG{o}{.}\PYG{n}{show}\PYG{p}{(}\PYG{p}{)}
\end{sphinxVerbatim}

\end{sphinxuseclass}\end{sphinxVerbatimInput}
\begin{sphinxVerbatimOutput}

\begin{sphinxuseclass}{cell_output}
\noindent\sphinxincludegraphics{{4d1080320a7838611340f073bba83ed55b27094ed4276f5cb3c2e5fd52438d19}.png}

\end{sphinxuseclass}\end{sphinxVerbatimOutput}

\end{sphinxuseclass}

\subsubsection{Which stocks have the minimum and maximum total returns?}
\label{\detokenize{mckinney_05_practice_04:which-stocks-have-the-minimum-and-maximum-total-returns}}
\begin{sphinxuseclass}{cell}\begin{sphinxVerbatimInput}

\begin{sphinxuseclass}{cell_input}
\begin{sphinxVerbatim}[commandchars=\\\{\}]
\PYG{n}{returns\PYGZus{}2\PYGZus{}total} \PYG{o}{=} \PYG{n}{returns\PYGZus{}2}\PYG{o}{.}\PYG{n}{add}\PYG{p}{(}\PYG{l+m+mi}{1}\PYG{p}{)}\PYG{o}{.}\PYG{n}{prod}\PYG{p}{(}\PYG{p}{)}\PYG{o}{.}\PYG{n}{sub}\PYG{p}{(}\PYG{l+m+mi}{1}\PYG{p}{)}
\end{sphinxVerbatim}

\end{sphinxuseclass}\end{sphinxVerbatimInput}

\end{sphinxuseclass}
\begin{sphinxuseclass}{cell}\begin{sphinxVerbatimInput}

\begin{sphinxuseclass}{cell_input}
\begin{sphinxVerbatim}[commandchars=\\\{\}]
\PYG{n}{returns\PYGZus{}2\PYGZus{}total}\PYG{o}{.}\PYG{n}{idxmin}\PYG{p}{(}\PYG{p}{)}
\end{sphinxVerbatim}

\end{sphinxuseclass}\end{sphinxVerbatimInput}
\begin{sphinxVerbatimOutput}

\begin{sphinxuseclass}{cell_output}
\begin{sphinxVerbatim}[commandchars=\\\{\}]
\PYGZsq{}BA\PYGZsq{}
\end{sphinxVerbatim}

\end{sphinxuseclass}\end{sphinxVerbatimOutput}

\end{sphinxuseclass}
\begin{sphinxuseclass}{cell}\begin{sphinxVerbatimInput}

\begin{sphinxuseclass}{cell_input}
\begin{sphinxVerbatim}[commandchars=\\\{\}]
\PYG{n}{returns\PYGZus{}2\PYGZus{}total}\PYG{o}{.}\PYG{n}{idxmax}\PYG{p}{(}\PYG{p}{)}
\end{sphinxVerbatim}

\end{sphinxuseclass}\end{sphinxVerbatimInput}
\begin{sphinxVerbatimOutput}

\begin{sphinxuseclass}{cell_output}
\begin{sphinxVerbatim}[commandchars=\\\{\}]
\PYGZsq{}AAPL\PYGZsq{}
\end{sphinxVerbatim}

\end{sphinxuseclass}\end{sphinxVerbatimOutput}

\end{sphinxuseclass}
\begin{sphinxuseclass}{cell}\begin{sphinxVerbatimInput}

\begin{sphinxuseclass}{cell_input}
\begin{sphinxVerbatim}[commandchars=\\\{\}]
\PYG{n}{returns\PYGZus{}2\PYGZus{}total}\PYG{o}{.}\PYG{n}{nsmallest}\PYG{p}{(}\PYG{p}{)}
\end{sphinxVerbatim}

\end{sphinxuseclass}\end{sphinxVerbatimInput}
\begin{sphinxVerbatimOutput}

\begin{sphinxuseclass}{cell_output}
\begin{sphinxVerbatim}[commandchars=\\\{\}]
Ticker
BA     \PYGZhy{}0.4154
INTC   \PYGZhy{}0.3986
MMM    \PYGZhy{}0.3630
WBA    \PYGZhy{}0.3114
VZ     \PYGZhy{}0.1284
dtype: float64
\end{sphinxVerbatim}

\end{sphinxuseclass}\end{sphinxVerbatimOutput}

\end{sphinxuseclass}
\begin{sphinxuseclass}{cell}\begin{sphinxVerbatimInput}

\begin{sphinxuseclass}{cell_input}
\begin{sphinxVerbatim}[commandchars=\\\{\}]
\PYG{n}{returns\PYGZus{}2\PYGZus{}total}\PYG{o}{.}\PYG{n}{nlargest}\PYG{p}{(}\PYG{p}{)}
\end{sphinxVerbatim}

\end{sphinxuseclass}\end{sphinxVerbatimInput}
\begin{sphinxVerbatimOutput}

\begin{sphinxuseclass}{cell_output}
\begin{sphinxVerbatim}[commandchars=\\\{\}]
Ticker
AAPL   2.1895
MSFT   1.2354
UNH    1.0982
CAT    1.0732
GS     1.0461
dtype: float64
\end{sphinxVerbatim}

\end{sphinxuseclass}\end{sphinxVerbatimOutput}

\end{sphinxuseclass}
\begin{sphinxuseclass}{cell}\begin{sphinxVerbatimInput}

\begin{sphinxuseclass}{cell_input}
\begin{sphinxVerbatim}[commandchars=\\\{\}]
\PYG{n}{returns\PYGZus{}2\PYGZus{}total}\PYG{o}{.}\PYG{n}{sort\PYGZus{}values}\PYG{p}{(}\PYG{p}{)}\PYG{o}{.}\PYG{n}{iloc}\PYG{p}{[}\PYG{p}{[}\PYG{l+m+mi}{0}\PYG{p}{,} \PYG{o}{\PYGZhy{}}\PYG{l+m+mi}{1}\PYG{p}{]}\PYG{p}{]}
\end{sphinxVerbatim}

\end{sphinxuseclass}\end{sphinxVerbatimInput}
\begin{sphinxVerbatimOutput}

\begin{sphinxuseclass}{cell_output}
\begin{sphinxVerbatim}[commandchars=\\\{\}]
Ticker
BA     \PYGZhy{}0.4154
AAPL    2.1895
dtype: float64
\end{sphinxVerbatim}

\end{sphinxuseclass}\end{sphinxVerbatimOutput}

\end{sphinxuseclass}

\subsubsection{Plot the cumulative returns for the stocks in the DJIA}
\label{\detokenize{mckinney_05_practice_04:plot-the-cumulative-returns-for-the-stocks-in-the-djia}}
\sphinxAtStartPar
We can use the cumulative product method \sphinxcode{\sphinxupquote{.cumprod()}} to calculate the right hand side of the formula above.

\begin{sphinxuseclass}{cell}\begin{sphinxVerbatimInput}

\begin{sphinxuseclass}{cell_input}
\begin{sphinxVerbatim}[commandchars=\\\{\}]
\PYG{p}{(}
    \PYG{n}{returns\PYGZus{}2}
    \PYG{o}{.}\PYG{n}{add}\PYG{p}{(}\PYG{l+m+mi}{1}\PYG{p}{)}
    \PYG{o}{.}\PYG{n}{cumprod}\PYG{p}{(}\PYG{p}{)}
    \PYG{o}{.}\PYG{n}{sub}\PYG{p}{(}\PYG{l+m+mi}{1}\PYG{p}{)}
    \PYG{o}{.}\PYG{n}{mul}\PYG{p}{(}\PYG{l+m+mi}{100}\PYG{p}{)}
    \PYG{o}{.}\PYG{n}{plot}\PYG{p}{(}\PYG{p}{)}
\PYG{p}{)}
\PYG{n}{plt}\PYG{o}{.}\PYG{n}{ylabel}\PYG{p}{(}\PYG{l+s+s1}{\PYGZsq{}}\PYG{l+s+s1}{Cumulative Return (}\PYG{l+s+s1}{\PYGZpc{}}\PYG{l+s+s1}{)}\PYG{l+s+s1}{\PYGZsq{}}\PYG{p}{)}
\PYG{n}{plt}\PYG{o}{.}\PYG{n}{title}\PYG{p}{(}\PYG{l+s+sa}{f}\PYG{l+s+s1}{\PYGZsq{}}\PYG{l+s+s1}{Cumulative Returns for DJIA Stocks }\PYG{l+s+se}{\PYGZbs{}n}\PYG{l+s+s1}{ from }\PYG{l+s+si}{\PYGZob{}}\PYG{n}{returns}\PYG{o}{.}\PYG{n}{index}\PYG{p}{[}\PYG{l+m+mi}{0}\PYG{p}{]}\PYG{l+s+si}{:}\PYG{l+s+s1}{\PYGZpc{}B \PYGZpc{}Y}\PYG{l+s+si}{\PYGZcb{}}\PYG{l+s+s1}{ to }\PYG{l+s+si}{\PYGZob{}}\PYG{n}{returns}\PYG{o}{.}\PYG{n}{index}\PYG{p}{[}\PYG{o}{\PYGZhy{}}\PYG{l+m+mi}{1}\PYG{p}{]}\PYG{l+s+si}{:}\PYG{l+s+s1}{\PYGZpc{}B \PYGZpc{}Y}\PYG{l+s+si}{\PYGZcb{}}\PYG{l+s+s1}{\PYGZsq{}}\PYG{p}{)}
\PYG{n}{plt}\PYG{o}{.}\PYG{n}{show}\PYG{p}{(}\PYG{p}{)}
\end{sphinxVerbatim}

\end{sphinxuseclass}\end{sphinxVerbatimInput}
\begin{sphinxVerbatimOutput}

\begin{sphinxuseclass}{cell_output}
\noindent\sphinxincludegraphics{{d020e2d07cb7d55860f0ae7dae944fd83b1b294ac778a3f118ffb032323ae46d}.png}

\end{sphinxuseclass}\end{sphinxVerbatimOutput}

\end{sphinxuseclass}

\subsubsection{Repeat the plot above with only the minimum and maximum total returns}
\label{\detokenize{mckinney_05_practice_04:repeat-the-plot-above-with-only-the-minimum-and-maximum-total-returns}}
\begin{sphinxuseclass}{cell}\begin{sphinxVerbatimInput}

\begin{sphinxuseclass}{cell_input}
\begin{sphinxVerbatim}[commandchars=\\\{\}]
\PYG{p}{(}
    \PYG{n}{returns\PYGZus{}2}\PYG{p}{[} \PYG{c+c1}{\PYGZsh{} DJIA daily returns}
        \PYG{n}{returns\PYGZus{}2\PYGZus{}total}\PYG{o}{.}\PYG{n}{sort\PYGZus{}values}\PYG{p}{(}\PYG{p}{)}\PYG{p}{[}\PYG{p}{[}\PYG{l+m+mi}{0}\PYG{p}{,} \PYG{o}{\PYGZhy{}}\PYG{l+m+mi}{1}\PYG{p}{]}\PYG{p}{]}\PYG{o}{.}\PYG{n}{index} \PYG{c+c1}{\PYGZsh{} slice min and mix total returns}
    \PYG{p}{]}
    \PYG{o}{.}\PYG{n}{add}\PYG{p}{(}\PYG{l+m+mi}{1}\PYG{p}{)} \PYG{c+c1}{\PYGZsh{} add 1}
    \PYG{o}{.}\PYG{n}{cumprod}\PYG{p}{(}\PYG{p}{)} \PYG{c+c1}{\PYGZsh{} calculate cumulative returns}
    \PYG{o}{.}\PYG{n}{sub}\PYG{p}{(}\PYG{l+m+mi}{1}\PYG{p}{)} \PYG{c+c1}{\PYGZsh{} sutract 1}
    \PYG{o}{.}\PYG{n}{mul}\PYG{p}{(}\PYG{l+m+mi}{100}\PYG{p}{)} \PYG{c+c1}{\PYGZsh{} convert to percent}
    \PYG{o}{.}\PYG{n}{plot}\PYG{p}{(}\PYG{p}{)} \PYG{c+c1}{\PYGZsh{} plot the time series}
\PYG{p}{)}
\PYG{n}{plt}\PYG{o}{.}\PYG{n}{ylabel}\PYG{p}{(}\PYG{l+s+s1}{\PYGZsq{}}\PYG{l+s+s1}{Cumulative Return (}\PYG{l+s+s1}{\PYGZpc{}}\PYG{l+s+s1}{)}\PYG{l+s+s1}{\PYGZsq{}}\PYG{p}{)}
\PYG{n}{plt}\PYG{o}{.}\PYG{n}{title}\PYG{p}{(}\PYG{l+s+sa}{f}\PYG{l+s+s1}{\PYGZsq{}}\PYG{l+s+s1}{Cumulative Returns for Best and Worst DJIA Stocks }\PYG{l+s+se}{\PYGZbs{}n}\PYG{l+s+s1}{ from }\PYG{l+s+si}{\PYGZob{}}\PYG{n}{returns}\PYG{o}{.}\PYG{n}{index}\PYG{p}{[}\PYG{l+m+mi}{0}\PYG{p}{]}\PYG{l+s+si}{:}\PYG{l+s+s1}{\PYGZpc{}B \PYGZpc{}Y}\PYG{l+s+si}{\PYGZcb{}}\PYG{l+s+s1}{ to }\PYG{l+s+si}{\PYGZob{}}\PYG{n}{returns}\PYG{o}{.}\PYG{n}{index}\PYG{p}{[}\PYG{o}{\PYGZhy{}}\PYG{l+m+mi}{1}\PYG{p}{]}\PYG{l+s+si}{:}\PYG{l+s+s1}{\PYGZpc{}B \PYGZpc{}Y}\PYG{l+s+si}{\PYGZcb{}}\PYG{l+s+s1}{\PYGZsq{}}\PYG{p}{)}
\PYG{n}{plt}\PYG{o}{.}\PYG{n}{show}\PYG{p}{(}\PYG{p}{)}
\end{sphinxVerbatim}

\end{sphinxuseclass}\end{sphinxVerbatimInput}
\begin{sphinxVerbatimOutput}

\begin{sphinxuseclass}{cell_output}
\noindent\sphinxincludegraphics{{be3c0f9b2b801373f7fbac2dd6270c6aadd9beea7f3f5624459bd113e33a2128}.png}

\end{sphinxuseclass}\end{sphinxVerbatimOutput}

\end{sphinxuseclass}
\sphinxstepscope


\section{McKinney Chapter 5 \sphinxhyphen{} Practice (Section 2, Wednesday 2:45 PM)}
\label{\detokenize{mckinney_05_practice_02:mckinney-chapter-5-practice-section-2-wednesday-2-45-pm}}\label{\detokenize{mckinney_05_practice_02::doc}}

\subsection{Announcements}
\label{\detokenize{mckinney_05_practice_02:announcements}}\begin{itemize}
\item {} 
\sphinxAtStartPar
Due on Friday, 2/3, at 11:59 PM
\begin{itemize}
\item {} 
\sphinxAtStartPar
10,000 Total XP \sphinxstyleemphasis{\sphinxstylestrong{submited to Gradescope as a PDF}}

\item {} 
\sphinxAtStartPar
Quiz 2

\end{itemize}

\item {} 
\sphinxAtStartPar
On Monday, 2/6, at 9 AM I will open groups on Canvas

\item {} 
\sphinxAtStartPar
I will close the groups on Monday, 2/13, and randomly assign anyone left over

\end{itemize}


\subsection{Practice}
\label{\detokenize{mckinney_05_practice_02:practice}}
\begin{sphinxuseclass}{cell}\begin{sphinxVerbatimInput}

\begin{sphinxuseclass}{cell_input}
\begin{sphinxVerbatim}[commandchars=\\\{\}]
\PYG{k+kn}{import} \PYG{n+nn}{matplotlib}\PYG{n+nn}{.}\PYG{n+nn}{pyplot} \PYG{k}{as} \PYG{n+nn}{plt}
\PYG{k+kn}{import} \PYG{n+nn}{numpy} \PYG{k}{as} \PYG{n+nn}{np}
\PYG{k+kn}{import} \PYG{n+nn}{pandas} \PYG{k}{as} \PYG{n+nn}{pd}

\PYG{k+kn}{import} \PYG{n+nn}{yfinance} \PYG{k}{as} \PYG{n+nn}{yf}
\PYG{k+kn}{import} \PYG{n+nn}{requests\PYGZus{}cache}
\end{sphinxVerbatim}

\end{sphinxuseclass}\end{sphinxVerbatimInput}

\end{sphinxuseclass}
\begin{sphinxuseclass}{cell}\begin{sphinxVerbatimInput}

\begin{sphinxuseclass}{cell_input}
\begin{sphinxVerbatim}[commandchars=\\\{\}]
\PYG{o}{\PYGZpc{}}\PYG{k}{config} InlineBackend.figure\PYGZus{}format = \PYGZsq{}retina\PYGZsq{}
\PYG{o}{\PYGZpc{}}\PYG{k}{precision} 4
\PYG{n}{pd}\PYG{o}{.}\PYG{n}{options}\PYG{o}{.}\PYG{n}{display}\PYG{o}{.}\PYG{n}{float\PYGZus{}format} \PYG{o}{=} \PYG{l+s+s1}{\PYGZsq{}}\PYG{l+s+si}{\PYGZob{}:.4f\PYGZcb{}}\PYG{l+s+s1}{\PYGZsq{}}\PYG{o}{.}\PYG{n}{format}
\PYG{n}{session} \PYG{o}{=} \PYG{n}{requests\PYGZus{}cache}\PYG{o}{.}\PYG{n}{CachedSession}\PYG{p}{(}\PYG{p}{)}
\end{sphinxVerbatim}

\end{sphinxuseclass}\end{sphinxVerbatimInput}

\end{sphinxuseclass}
\begin{sphinxuseclass}{cell}\begin{sphinxVerbatimInput}

\begin{sphinxuseclass}{cell_input}
\begin{sphinxVerbatim}[commandchars=\\\{\}]
\PYG{n}{tickers} \PYG{o}{=} \PYG{n}{yf}\PYG{o}{.}\PYG{n}{Tickers}\PYG{p}{(}\PYG{l+s+s1}{\PYGZsq{}}\PYG{l+s+s1}{AAPL IBM MSFT GOOG}\PYG{l+s+s1}{\PYGZsq{}}\PYG{p}{,} \PYG{n}{session}\PYG{o}{=}\PYG{n}{session}\PYG{p}{)}
\PYG{n}{prices} \PYG{o}{=} \PYG{n}{tickers}\PYG{o}{.}\PYG{n}{history}\PYG{p}{(}\PYG{n}{period}\PYG{o}{=}\PYG{l+s+s1}{\PYGZsq{}}\PYG{l+s+s1}{max}\PYG{l+s+s1}{\PYGZsq{}}\PYG{p}{,} \PYG{n}{auto\PYGZus{}adjust}\PYG{o}{=}\PYG{k+kc}{False}\PYG{p}{,} \PYG{n}{progress}\PYG{o}{=}\PYG{k+kc}{False}\PYG{p}{)}
\PYG{n}{prices}\PYG{o}{.}\PYG{n}{index} \PYG{o}{=} \PYG{n}{prices}\PYG{o}{.}\PYG{n}{index}\PYG{o}{.}\PYG{n}{tz\PYGZus{}localize}\PYG{p}{(}\PYG{k+kc}{None}\PYG{p}{)}
\PYG{n}{returns} \PYG{o}{=} \PYG{n}{prices}\PYG{p}{[}\PYG{l+s+s1}{\PYGZsq{}}\PYG{l+s+s1}{Adj Close}\PYG{l+s+s1}{\PYGZsq{}}\PYG{p}{]}\PYG{o}{.}\PYG{n}{pct\PYGZus{}change}\PYG{p}{(}\PYG{p}{)}\PYG{o}{.}\PYG{n}{dropna}\PYG{p}{(}\PYG{p}{)}
\PYG{n}{returns}\PYG{o}{.}\PYG{n}{columns}\PYG{o}{.}\PYG{n}{name} \PYG{o}{=} \PYG{l+s+s1}{\PYGZsq{}}\PYG{l+s+s1}{Ticker}\PYG{l+s+s1}{\PYGZsq{}}
\end{sphinxVerbatim}

\end{sphinxuseclass}\end{sphinxVerbatimInput}
\begin{sphinxVerbatimOutput}

\begin{sphinxuseclass}{cell_output}
\begin{sphinxVerbatim}[commandchars=\\\{\}]
[*********************100\PYGZpc{}***********************]  4 of 4 completed
\end{sphinxVerbatim}

\end{sphinxuseclass}\end{sphinxVerbatimOutput}

\end{sphinxuseclass}
\begin{sphinxuseclass}{cell}\begin{sphinxVerbatimInput}

\begin{sphinxuseclass}{cell_input}
\begin{sphinxVerbatim}[commandchars=\\\{\}]
\PYG{n}{returns}
\end{sphinxVerbatim}

\end{sphinxuseclass}\end{sphinxVerbatimInput}
\begin{sphinxVerbatimOutput}

\begin{sphinxuseclass}{cell_output}
\begin{sphinxVerbatim}[commandchars=\\\{\}]
Ticker        AAPL    GOOG     IBM    MSFT
Date                                      
2004\PYGZhy{}08\PYGZhy{}20  0.0029  0.0794  0.0042  0.0029
2004\PYGZhy{}08\PYGZhy{}23  0.0091  0.0101 \PYGZhy{}0.0070  0.0044
2004\PYGZhy{}08\PYGZhy{}24  0.0280 \PYGZhy{}0.0414  0.0007  0.0000
2004\PYGZhy{}08\PYGZhy{}25  0.0344  0.0108  0.0042  0.0114
2004\PYGZhy{}08\PYGZhy{}26  0.0487  0.0180 \PYGZhy{}0.0045 \PYGZhy{}0.0040
...            ...     ...     ...     ...
2023\PYGZhy{}01\PYGZhy{}26  0.0148  0.0251 \PYGZhy{}0.0448  0.0307
2023\PYGZhy{}01\PYGZhy{}27  0.0137  0.0156 \PYGZhy{}0.0004  0.0006
2023\PYGZhy{}01\PYGZhy{}30 \PYGZhy{}0.0201 \PYGZhy{}0.0274  0.0068 \PYGZhy{}0.0220
2023\PYGZhy{}01\PYGZhy{}31  0.0090  0.0196 \PYGZhy{}0.0042  0.0210
2023\PYGZhy{}02\PYGZhy{}01  0.0079  0.0156  0.0027  0.0199

[4645 rows x 4 columns]
\end{sphinxVerbatim}

\end{sphinxuseclass}\end{sphinxVerbatimOutput}

\end{sphinxuseclass}

\subsubsection{What are the mean daily returns for these four stocks?}
\label{\detokenize{mckinney_05_practice_02:what-are-the-mean-daily-returns-for-these-four-stocks}}
\begin{sphinxuseclass}{cell}\begin{sphinxVerbatimInput}

\begin{sphinxuseclass}{cell_input}
\begin{sphinxVerbatim}[commandchars=\\\{\}]
\PYG{n}{returns}\PYG{o}{.}\PYG{n}{mean}\PYG{p}{(}\PYG{p}{)}
\end{sphinxVerbatim}

\end{sphinxuseclass}\end{sphinxVerbatimInput}
\begin{sphinxVerbatimOutput}

\begin{sphinxuseclass}{cell_output}
\begin{sphinxVerbatim}[commandchars=\\\{\}]
Ticker
AAPL   0.0015
GOOG   0.0010
IBM    0.0003
MSFT   0.0007
dtype: float64
\end{sphinxVerbatim}

\end{sphinxuseclass}\end{sphinxVerbatimOutput}

\end{sphinxuseclass}

\subsubsection{What are the standard deviations of daily returns for these four stocks?}
\label{\detokenize{mckinney_05_practice_02:what-are-the-standard-deviations-of-daily-returns-for-these-four-stocks}}
\begin{sphinxuseclass}{cell}\begin{sphinxVerbatimInput}

\begin{sphinxuseclass}{cell_input}
\begin{sphinxVerbatim}[commandchars=\\\{\}]
\PYG{n}{returns}\PYG{o}{.}\PYG{n}{std}\PYG{p}{(}\PYG{p}{)}
\end{sphinxVerbatim}

\end{sphinxuseclass}\end{sphinxVerbatimInput}
\begin{sphinxVerbatimOutput}

\begin{sphinxuseclass}{cell_output}
\begin{sphinxVerbatim}[commandchars=\\\{\}]
Ticker
AAPL   0.0210
GOOG   0.0194
IBM    0.0144
MSFT   0.0173
dtype: float64
\end{sphinxVerbatim}

\end{sphinxuseclass}\end{sphinxVerbatimOutput}

\end{sphinxuseclass}

\subsubsection{What are the \sphinxstyleemphasis{annualized} means and standard deviations of daily returns for these four stocks?}
\label{\detokenize{mckinney_05_practice_02:what-are-the-annualized-means-and-standard-deviations-of-daily-returns-for-these-four-stocks}}
\begin{sphinxuseclass}{cell}\begin{sphinxVerbatimInput}

\begin{sphinxuseclass}{cell_input}
\begin{sphinxVerbatim}[commandchars=\\\{\}]
\PYG{n}{returns}\PYG{o}{.}\PYG{n}{mean}\PYG{p}{(}\PYG{p}{)}\PYG{o}{.}\PYG{n}{mul}\PYG{p}{(}\PYG{l+m+mi}{252}\PYG{p}{)}
\end{sphinxVerbatim}

\end{sphinxuseclass}\end{sphinxVerbatimInput}
\begin{sphinxVerbatimOutput}

\begin{sphinxuseclass}{cell_output}
\begin{sphinxVerbatim}[commandchars=\\\{\}]
Ticker
AAPL   0.3670
GOOG   0.2480
IBM    0.0822
MSFT   0.1836
dtype: float64
\end{sphinxVerbatim}

\end{sphinxuseclass}\end{sphinxVerbatimOutput}

\end{sphinxuseclass}
\begin{sphinxuseclass}{cell}\begin{sphinxVerbatimInput}

\begin{sphinxuseclass}{cell_input}
\begin{sphinxVerbatim}[commandchars=\\\{\}]
\PYG{n}{returns}\PYG{o}{.}\PYG{n}{std}\PYG{p}{(}\PYG{p}{)}\PYG{o}{.}\PYG{n}{mul}\PYG{p}{(}\PYG{n}{np}\PYG{o}{.}\PYG{n}{sqrt}\PYG{p}{(}\PYG{l+m+mi}{252}\PYG{p}{)}\PYG{p}{)}
\end{sphinxVerbatim}

\end{sphinxuseclass}\end{sphinxVerbatimInput}
\begin{sphinxVerbatimOutput}

\begin{sphinxuseclass}{cell_output}
\begin{sphinxVerbatim}[commandchars=\\\{\}]
Ticker
AAPL   0.3331
GOOG   0.3077
IBM    0.2294
MSFT   0.2739
dtype: float64
\end{sphinxVerbatim}

\end{sphinxuseclass}\end{sphinxVerbatimOutput}

\end{sphinxuseclass}
\sphinxAtStartPar
\sphinxstyleemphasis{\sphinxstylestrong{The best explanation I have found on why stock return volatility (the standard deviation of stocks returns) grows with the square root of time is at the bottom of page 7 of \sphinxhref{https://book.ivo-welch.info/read/source5.mba/08-invchoice.pdf}{chapter 8 of Ivo Welch’s free corporate finance textbook}.}}


\subsubsection{Plot \sphinxstyleemphasis{annualized} means versus standard deviations of daily returns for these four stocks}
\label{\detokenize{mckinney_05_practice_02:plot-annualized-means-versus-standard-deviations-of-daily-returns-for-these-four-stocks}}
\sphinxAtStartPar
Use \sphinxcode{\sphinxupquote{plt.scatter()}}, which expects arguments as \sphinxcode{\sphinxupquote{x}} (standard deviations) then \sphinxcode{\sphinxupquote{y}} (means).

\begin{sphinxuseclass}{cell}\begin{sphinxVerbatimInput}

\begin{sphinxuseclass}{cell_input}
\begin{sphinxVerbatim}[commandchars=\\\{\}]
\PYG{n}{plt}\PYG{o}{.}\PYG{n}{scatter}\PYG{p}{(}
    \PYG{n}{x}\PYG{o}{=}\PYG{n}{returns}\PYG{o}{.}\PYG{n}{std}\PYG{p}{(}\PYG{p}{)} \PYG{o}{*} \PYG{n}{np}\PYG{o}{.}\PYG{n}{sqrt}\PYG{p}{(}\PYG{l+m+mi}{252}\PYG{p}{)} \PYG{o}{*} \PYG{l+m+mi}{100}\PYG{p}{,}
    \PYG{n}{y}\PYG{o}{=}\PYG{n}{returns}\PYG{o}{.}\PYG{n}{mean}\PYG{p}{(}\PYG{p}{)} \PYG{o}{*} \PYG{l+m+mi}{252} \PYG{o}{*} \PYG{l+m+mi}{100}
\PYG{p}{)}
\PYG{n}{plt}\PYG{o}{.}\PYG{n}{xlabel}\PYG{p}{(}\PYG{l+s+s1}{\PYGZsq{}}\PYG{l+s+s1}{Annualized Standard Deviation of Daily Returns (}\PYG{l+s+s1}{\PYGZpc{}}\PYG{l+s+s1}{)}\PYG{l+s+s1}{\PYGZsq{}}\PYG{p}{)}
\PYG{n}{plt}\PYG{o}{.}\PYG{n}{ylabel}\PYG{p}{(}\PYG{l+s+s1}{\PYGZsq{}}\PYG{l+s+s1}{Annualized Mean of Daily Returns (}\PYG{l+s+s1}{\PYGZpc{}}\PYG{l+s+s1}{)}\PYG{l+s+s1}{\PYGZsq{}}\PYG{p}{)}
\PYG{n}{plt}\PYG{o}{.}\PYG{n}{title}\PYG{p}{(}\PYG{l+s+sa}{f}\PYG{l+s+s1}{\PYGZsq{}}\PYG{l+s+s1}{Returns Versus Risk}\PYG{l+s+se}{\PYGZbs{}n}\PYG{l+s+s1}{ from }\PYG{l+s+si}{\PYGZob{}}\PYG{n}{returns}\PYG{o}{.}\PYG{n}{index}\PYG{p}{[}\PYG{l+m+mi}{0}\PYG{p}{]}\PYG{l+s+si}{:}\PYG{l+s+s1}{\PYGZpc{}B \PYGZpc{}Y}\PYG{l+s+si}{\PYGZcb{}}\PYG{l+s+s1}{ to }\PYG{l+s+si}{\PYGZob{}}\PYG{n}{returns}\PYG{o}{.}\PYG{n}{index}\PYG{p}{[}\PYG{o}{\PYGZhy{}}\PYG{l+m+mi}{1}\PYG{p}{]}\PYG{l+s+si}{:}\PYG{l+s+s1}{\PYGZpc{}B \PYGZpc{}Y}\PYG{l+s+si}{\PYGZcb{}}\PYG{l+s+s1}{\PYGZsq{}}\PYG{p}{)}

\PYG{k}{for} \PYG{n}{i} \PYG{o+ow}{in} \PYG{n}{returns}\PYG{p}{:}
    \PYG{n}{plt}\PYG{o}{.}\PYG{n}{text}\PYG{p}{(}
        \PYG{n}{x}\PYG{o}{=}\PYG{n}{returns}\PYG{p}{[}\PYG{n}{i}\PYG{p}{]}\PYG{o}{.}\PYG{n}{std}\PYG{p}{(}\PYG{p}{)} \PYG{o}{*} \PYG{n}{np}\PYG{o}{.}\PYG{n}{sqrt}\PYG{p}{(}\PYG{l+m+mi}{252}\PYG{p}{)} \PYG{o}{*} \PYG{l+m+mi}{100}\PYG{p}{,}
        \PYG{n}{y}\PYG{o}{=}\PYG{n}{returns}\PYG{p}{[}\PYG{n}{i}\PYG{p}{]}\PYG{o}{.}\PYG{n}{mean}\PYG{p}{(}\PYG{p}{)} \PYG{o}{*} \PYG{l+m+mi}{252} \PYG{o}{*} \PYG{l+m+mi}{100}\PYG{p}{,}
        \PYG{n}{s}\PYG{o}{=}\PYG{n}{i}
    \PYG{p}{)}
    
\PYG{n}{plt}\PYG{o}{.}\PYG{n}{show}\PYG{p}{(}\PYG{p}{)}
\end{sphinxVerbatim}

\end{sphinxuseclass}\end{sphinxVerbatimInput}
\begin{sphinxVerbatimOutput}

\begin{sphinxuseclass}{cell_output}
\noindent\sphinxincludegraphics{{30b2057f4f3cf63067542362c499ec4756baf24fdae3566bbb139e16665497b5}.png}

\end{sphinxuseclass}\end{sphinxVerbatimOutput}

\end{sphinxuseclass}

\subsubsection{Repeat the previous calculations and plot for the stocks in the Dow\sphinxhyphen{}Jones Industrial Average (DJIA)}
\label{\detokenize{mckinney_05_practice_02:repeat-the-previous-calculations-and-plot-for-the-stocks-in-the-dow-jones-industrial-average-djia}}
\sphinxAtStartPar
We can find the current DJIA stocks on \sphinxhref{https://en.wikipedia.org/wiki/Dow\_Jones\_Industrial\_Average}{Wikipedia}.
We will need to download new data, into \sphinxcode{\sphinxupquote{tickers2}}, \sphinxcode{\sphinxupquote{prices2}}, and \sphinxcode{\sphinxupquote{returns2}}.

\begin{sphinxuseclass}{cell}\begin{sphinxVerbatimInput}

\begin{sphinxuseclass}{cell_input}
\begin{sphinxVerbatim}[commandchars=\\\{\}]
\PYG{k}{def} \PYG{n+nf}{returns\PYGZus{}many}\PYG{p}{(}\PYG{n}{x}\PYG{p}{)}\PYG{p}{:}
    \PYG{n}{tickers} \PYG{o}{=} \PYG{n}{yf}\PYG{o}{.}\PYG{n}{Tickers}\PYG{p}{(}\PYG{n}{tickers}\PYG{o}{=}\PYG{n}{x}\PYG{p}{,} \PYG{n}{session}\PYG{o}{=}\PYG{n}{session}\PYG{p}{)}
    \PYG{n}{prices} \PYG{o}{=} \PYG{n}{tickers}\PYG{o}{.}\PYG{n}{history}\PYG{p}{(}\PYG{n}{period}\PYG{o}{=}\PYG{l+s+s1}{\PYGZsq{}}\PYG{l+s+s1}{max}\PYG{l+s+s1}{\PYGZsq{}}\PYG{p}{,} \PYG{n}{auto\PYGZus{}adjust}\PYG{o}{=}\PYG{k+kc}{False}\PYG{p}{,} \PYG{n}{progress}\PYG{o}{=}\PYG{k+kc}{False}\PYG{p}{)}
    \PYG{n}{prices}\PYG{o}{.}\PYG{n}{index} \PYG{o}{=} \PYG{n}{prices}\PYG{o}{.}\PYG{n}{index}\PYG{o}{.}\PYG{n}{tz\PYGZus{}localize}\PYG{p}{(}\PYG{k+kc}{None}\PYG{p}{)}
    \PYG{n}{returns} \PYG{o}{=} \PYG{n}{prices}\PYG{p}{[}\PYG{l+s+s1}{\PYGZsq{}}\PYG{l+s+s1}{Adj Close}\PYG{l+s+s1}{\PYGZsq{}}\PYG{p}{]}\PYG{o}{.}\PYG{n}{pct\PYGZus{}change}\PYG{p}{(}\PYG{p}{)}\PYG{o}{.}\PYG{n}{dropna}\PYG{p}{(}\PYG{p}{)}
    \PYG{n}{returns}\PYG{o}{.}\PYG{n}{columns}\PYG{o}{.}\PYG{n}{name} \PYG{o}{=} \PYG{l+s+s1}{\PYGZsq{}}\PYG{l+s+s1}{Ticker}\PYG{l+s+s1}{\PYGZsq{}}
    \PYG{k}{return} \PYG{n}{returns}
\end{sphinxVerbatim}

\end{sphinxuseclass}\end{sphinxVerbatimInput}

\end{sphinxuseclass}
\begin{sphinxuseclass}{cell}\begin{sphinxVerbatimInput}

\begin{sphinxuseclass}{cell_input}
\begin{sphinxVerbatim}[commandchars=\\\{\}]
\PYG{k}{def} \PYG{n+nf}{plot\PYGZus{}many}\PYG{p}{(}\PYG{n}{x}\PYG{p}{)}\PYG{p}{:}
    \PYG{n}{plt}\PYG{o}{.}\PYG{n}{scatter}\PYG{p}{(}
        \PYG{n}{x}\PYG{o}{=}\PYG{n}{x}\PYG{o}{.}\PYG{n}{std}\PYG{p}{(}\PYG{p}{)} \PYG{o}{*} \PYG{n}{np}\PYG{o}{.}\PYG{n}{sqrt}\PYG{p}{(}\PYG{l+m+mi}{252}\PYG{p}{)} \PYG{o}{*} \PYG{l+m+mi}{100}\PYG{p}{,}
        \PYG{n}{y}\PYG{o}{=}\PYG{n}{x}\PYG{o}{.}\PYG{n}{mean}\PYG{p}{(}\PYG{p}{)} \PYG{o}{*} \PYG{l+m+mi}{252} \PYG{o}{*} \PYG{l+m+mi}{100}
    \PYG{p}{)}
    \PYG{n}{plt}\PYG{o}{.}\PYG{n}{xlabel}\PYG{p}{(}\PYG{l+s+s1}{\PYGZsq{}}\PYG{l+s+s1}{Annualized Standard Deviation of Daily Returns (}\PYG{l+s+s1}{\PYGZpc{}}\PYG{l+s+s1}{)}\PYG{l+s+s1}{\PYGZsq{}}\PYG{p}{)}
    \PYG{n}{plt}\PYG{o}{.}\PYG{n}{ylabel}\PYG{p}{(}\PYG{l+s+s1}{\PYGZsq{}}\PYG{l+s+s1}{Annualized Mean of Daily Returns (}\PYG{l+s+s1}{\PYGZpc{}}\PYG{l+s+s1}{)}\PYG{l+s+s1}{\PYGZsq{}}\PYG{p}{)}
    \PYG{n}{plt}\PYG{o}{.}\PYG{n}{title}\PYG{p}{(}\PYG{l+s+sa}{f}\PYG{l+s+s1}{\PYGZsq{}}\PYG{l+s+s1}{Returns Versus Risk}\PYG{l+s+se}{\PYGZbs{}n}\PYG{l+s+s1}{ from }\PYG{l+s+si}{\PYGZob{}}\PYG{n}{x}\PYG{o}{.}\PYG{n}{index}\PYG{p}{[}\PYG{l+m+mi}{0}\PYG{p}{]}\PYG{l+s+si}{:}\PYG{l+s+s1}{\PYGZpc{}B \PYGZpc{}Y}\PYG{l+s+si}{\PYGZcb{}}\PYG{l+s+s1}{ to }\PYG{l+s+si}{\PYGZob{}}\PYG{n}{x}\PYG{o}{.}\PYG{n}{index}\PYG{p}{[}\PYG{o}{\PYGZhy{}}\PYG{l+m+mi}{1}\PYG{p}{]}\PYG{l+s+si}{:}\PYG{l+s+s1}{\PYGZpc{}B \PYGZpc{}Y}\PYG{l+s+si}{\PYGZcb{}}\PYG{l+s+s1}{\PYGZsq{}}\PYG{p}{)}

    \PYG{k}{for} \PYG{n}{i} \PYG{o+ow}{in} \PYG{n}{x}\PYG{p}{:}
        \PYG{n}{plt}\PYG{o}{.}\PYG{n}{text}\PYG{p}{(}
            \PYG{n}{x}\PYG{o}{=}\PYG{n}{x}\PYG{p}{[}\PYG{n}{i}\PYG{p}{]}\PYG{o}{.}\PYG{n}{std}\PYG{p}{(}\PYG{p}{)} \PYG{o}{*} \PYG{n}{np}\PYG{o}{.}\PYG{n}{sqrt}\PYG{p}{(}\PYG{l+m+mi}{252}\PYG{p}{)} \PYG{o}{*} \PYG{l+m+mi}{100}\PYG{p}{,}
            \PYG{n}{y}\PYG{o}{=}\PYG{n}{x}\PYG{p}{[}\PYG{n}{i}\PYG{p}{]}\PYG{o}{.}\PYG{n}{mean}\PYG{p}{(}\PYG{p}{)} \PYG{o}{*} \PYG{l+m+mi}{252} \PYG{o}{*} \PYG{l+m+mi}{100}\PYG{p}{,}
            \PYG{n}{s}\PYG{o}{=}\PYG{n}{i}
        \PYG{p}{)}

    \PYG{n}{plt}\PYG{o}{.}\PYG{n}{show}\PYG{p}{(}\PYG{p}{)}
\end{sphinxVerbatim}

\end{sphinxuseclass}\end{sphinxVerbatimInput}

\end{sphinxuseclass}
\begin{sphinxuseclass}{cell}\begin{sphinxVerbatimInput}

\begin{sphinxuseclass}{cell_input}
\begin{sphinxVerbatim}[commandchars=\\\{\}]
\PYG{n}{wiki} \PYG{o}{=} \PYG{n}{pd}\PYG{o}{.}\PYG{n}{read\PYGZus{}html}\PYG{p}{(}\PYG{l+s+s1}{\PYGZsq{}}\PYG{l+s+s1}{https://en.wikipedia.org/wiki/Dow\PYGZus{}Jones\PYGZus{}Industrial\PYGZus{}Average}\PYG{l+s+s1}{\PYGZsq{}}\PYG{p}{)}
\end{sphinxVerbatim}

\end{sphinxuseclass}\end{sphinxVerbatimInput}

\end{sphinxuseclass}
\begin{sphinxuseclass}{cell}\begin{sphinxVerbatimInput}

\begin{sphinxuseclass}{cell_input}
\begin{sphinxVerbatim}[commandchars=\\\{\}]
\PYG{n}{returns\PYGZus{}2} \PYG{o}{=} \PYG{n}{returns\PYGZus{}many}\PYG{p}{(}\PYG{n}{wiki}\PYG{p}{[}\PYG{l+m+mi}{1}\PYG{p}{]}\PYG{p}{[}\PYG{l+s+s1}{\PYGZsq{}}\PYG{l+s+s1}{Symbol}\PYG{l+s+s1}{\PYGZsq{}}\PYG{p}{]}\PYG{o}{.}\PYG{n}{to\PYGZus{}list}\PYG{p}{(}\PYG{p}{)}\PYG{p}{)}
\end{sphinxVerbatim}

\end{sphinxuseclass}\end{sphinxVerbatimInput}
\begin{sphinxVerbatimOutput}

\begin{sphinxuseclass}{cell_output}
\begin{sphinxVerbatim}[commandchars=\\\{\}]
[*********************100\PYGZpc{}***********************]  30 of 30 completed
\end{sphinxVerbatim}

\end{sphinxuseclass}\end{sphinxVerbatimOutput}

\end{sphinxuseclass}
\begin{sphinxuseclass}{cell}\begin{sphinxVerbatimInput}

\begin{sphinxuseclass}{cell_input}
\begin{sphinxVerbatim}[commandchars=\\\{\}]
\PYG{n}{plot\PYGZus{}many}\PYG{p}{(}\PYG{n}{returns\PYGZus{}2}\PYG{p}{)}
\end{sphinxVerbatim}

\end{sphinxuseclass}\end{sphinxVerbatimInput}
\begin{sphinxVerbatimOutput}

\begin{sphinxuseclass}{cell_output}
\noindent\sphinxincludegraphics{{a0507d06845140f9e47a29056e6e1f9de8003cc4edbe1c180658ca8e60f00eb2}.png}

\end{sphinxuseclass}\end{sphinxVerbatimOutput}

\end{sphinxuseclass}
\sphinxAtStartPar
\sphinxstyleemphasis{\sphinxstylestrong{THERE IS NO RELATION BETWEEN RETURNS AND RISK FOR SINGLE STOCKS!}}

\begin{sphinxuseclass}{cell}\begin{sphinxVerbatimInput}

\begin{sphinxuseclass}{cell_input}
\begin{sphinxVerbatim}[commandchars=\\\{\}]
\PYG{n}{np}\PYG{o}{.}\PYG{n}{corrcoef}\PYG{p}{(}\PYG{n}{returns\PYGZus{}2}\PYG{o}{.}\PYG{n}{mean}\PYG{p}{(}\PYG{p}{)}\PYG{p}{,} \PYG{n}{returns\PYGZus{}2}\PYG{o}{.}\PYG{n}{std}\PYG{p}{(}\PYG{p}{)}\PYG{p}{)}
\end{sphinxVerbatim}

\end{sphinxuseclass}\end{sphinxVerbatimInput}
\begin{sphinxVerbatimOutput}

\begin{sphinxuseclass}{cell_output}
\begin{sphinxVerbatim}[commandchars=\\\{\}]
array([[1.    , 0.0186],
       [0.0186, 1.    ]])
\end{sphinxVerbatim}

\end{sphinxuseclass}\end{sphinxVerbatimOutput}

\end{sphinxuseclass}
\begin{sphinxuseclass}{cell}\begin{sphinxVerbatimInput}

\begin{sphinxuseclass}{cell_input}
\begin{sphinxVerbatim}[commandchars=\\\{\}]
\PYG{n}{returns\PYGZus{}2}\PYG{o}{.}\PYG{n}{pipe}\PYG{p}{(}\PYG{n}{plot\PYGZus{}many}\PYG{p}{)}
\end{sphinxVerbatim}

\end{sphinxuseclass}\end{sphinxVerbatimInput}
\begin{sphinxVerbatimOutput}

\begin{sphinxuseclass}{cell_output}
\noindent\sphinxincludegraphics{{a0507d06845140f9e47a29056e6e1f9de8003cc4edbe1c180658ca8e60f00eb2}.png}

\end{sphinxuseclass}\end{sphinxVerbatimOutput}

\end{sphinxuseclass}

\subsubsection{Calculate total returns for the stocks in the DJIA}
\label{\detokenize{mckinney_05_practice_02:calculate-total-returns-for-the-stocks-in-the-djia}}
\sphinxAtStartPar
We can use the \sphinxcode{\sphinxupquote{.prod()}} method to compound returns as \$1 + R\_T = \textbackslash{}prod\_\{t=1\}\textasciicircum{}T (1 + R\_t)\$.
Technically, we should write \$R\_T\$ as \$R\_\{0,T\}\$, but we typically omit the subscript \$0\$.

\begin{sphinxuseclass}{cell}\begin{sphinxVerbatimInput}

\begin{sphinxuseclass}{cell_input}
\begin{sphinxVerbatim}[commandchars=\\\{\}]
\PYG{p}{(}
    \PYG{n}{returns\PYGZus{}2}
    \PYG{o}{.}\PYG{n}{add}\PYG{p}{(}\PYG{l+m+mi}{1}\PYG{p}{)}
    \PYG{o}{.}\PYG{n}{prod}\PYG{p}{(}\PYG{p}{)}
    \PYG{o}{.}\PYG{n}{sub}\PYG{p}{(}\PYG{l+m+mi}{1}\PYG{p}{)}
    \PYG{o}{.}\PYG{n}{mul}\PYG{p}{(}\PYG{l+m+mi}{100}\PYG{p}{)}
    \PYG{o}{.}\PYG{n}{sort\PYGZus{}values}\PYG{p}{(}\PYG{p}{)}
    \PYG{o}{.}\PYG{n}{plot}\PYG{p}{(}\PYG{n}{kind}\PYG{o}{=}\PYG{l+s+s1}{\PYGZsq{}}\PYG{l+s+s1}{barh}\PYG{l+s+s1}{\PYGZsq{}}\PYG{p}{)}
\PYG{p}{)}
\PYG{n}{plt}\PYG{o}{.}\PYG{n}{xlabel}\PYG{p}{(}\PYG{l+s+s1}{\PYGZsq{}}\PYG{l+s+s1}{Total Return (}\PYG{l+s+s1}{\PYGZpc{}}\PYG{l+s+s1}{)}\PYG{l+s+s1}{\PYGZsq{}}\PYG{p}{)}
\PYG{n}{plt}\PYG{o}{.}\PYG{n}{title}\PYG{p}{(}\PYG{l+s+sa}{f}\PYG{l+s+s1}{\PYGZsq{}}\PYG{l+s+s1}{Total Returns for DJIA Stocks }\PYG{l+s+se}{\PYGZbs{}n}\PYG{l+s+s1}{ from }\PYG{l+s+si}{\PYGZob{}}\PYG{n}{returns}\PYG{o}{.}\PYG{n}{index}\PYG{p}{[}\PYG{l+m+mi}{0}\PYG{p}{]}\PYG{l+s+si}{:}\PYG{l+s+s1}{\PYGZpc{}B \PYGZpc{}Y}\PYG{l+s+si}{\PYGZcb{}}\PYG{l+s+s1}{ to }\PYG{l+s+si}{\PYGZob{}}\PYG{n}{returns}\PYG{o}{.}\PYG{n}{index}\PYG{p}{[}\PYG{o}{\PYGZhy{}}\PYG{l+m+mi}{1}\PYG{p}{]}\PYG{l+s+si}{:}\PYG{l+s+s1}{\PYGZpc{}B \PYGZpc{}Y}\PYG{l+s+si}{\PYGZcb{}}\PYG{l+s+s1}{\PYGZsq{}}\PYG{p}{)}
\PYG{n}{plt}\PYG{o}{.}\PYG{n}{show}\PYG{p}{(}\PYG{p}{)}
\end{sphinxVerbatim}

\end{sphinxuseclass}\end{sphinxVerbatimInput}
\begin{sphinxVerbatimOutput}

\begin{sphinxuseclass}{cell_output}
\noindent\sphinxincludegraphics{{8f75b4eb4a045ae4ad1adc453b75fd89f2400632de02b33ed1ed271f166bd17b}.png}

\end{sphinxuseclass}\end{sphinxVerbatimOutput}

\end{sphinxuseclass}

\subsubsection{Plot the distribution of total returns for the stocks in the DJIA}
\label{\detokenize{mckinney_05_practice_02:plot-the-distribution-of-total-returns-for-the-stocks-in-the-djia}}
\sphinxAtStartPar
We can plot a histogram, using either the \sphinxcode{\sphinxupquote{plt.hist()}} function or the \sphinxcode{\sphinxupquote{.plot(kind='hist')}} method.

\begin{sphinxuseclass}{cell}\begin{sphinxVerbatimInput}

\begin{sphinxuseclass}{cell_input}
\begin{sphinxVerbatim}[commandchars=\\\{\}]
\PYG{p}{(}
    \PYG{n}{returns\PYGZus{}2}
    \PYG{o}{.}\PYG{n}{add}\PYG{p}{(}\PYG{l+m+mi}{1}\PYG{p}{)}
    \PYG{o}{.}\PYG{n}{prod}\PYG{p}{(}\PYG{p}{)}
    \PYG{o}{.}\PYG{n}{sub}\PYG{p}{(}\PYG{l+m+mi}{1}\PYG{p}{)}
    \PYG{o}{.}\PYG{n}{mul}\PYG{p}{(}\PYG{l+m+mi}{100}\PYG{p}{)}
    \PYG{o}{.}\PYG{n}{plot}\PYG{p}{(}\PYG{n}{kind}\PYG{o}{=}\PYG{l+s+s1}{\PYGZsq{}}\PYG{l+s+s1}{hist}\PYG{l+s+s1}{\PYGZsq{}}\PYG{p}{,} \PYG{n}{bins}\PYG{o}{=}\PYG{l+m+mi}{20}\PYG{p}{)}
\PYG{p}{)}
\PYG{n}{plt}\PYG{o}{.}\PYG{n}{xlabel}\PYG{p}{(}\PYG{l+s+s1}{\PYGZsq{}}\PYG{l+s+s1}{Total Returns (}\PYG{l+s+s1}{\PYGZpc{}}\PYG{l+s+s1}{)}\PYG{l+s+s1}{\PYGZsq{}}\PYG{p}{)}
\PYG{n}{plt}\PYG{o}{.}\PYG{n}{title}\PYG{p}{(}\PYG{l+s+sa}{f}\PYG{l+s+s1}{\PYGZsq{}}\PYG{l+s+s1}{Total Returns for DJIA Stocks }\PYG{l+s+se}{\PYGZbs{}n}\PYG{l+s+s1}{ from }\PYG{l+s+si}{\PYGZob{}}\PYG{n}{returns}\PYG{o}{.}\PYG{n}{index}\PYG{p}{[}\PYG{l+m+mi}{0}\PYG{p}{]}\PYG{l+s+si}{:}\PYG{l+s+s1}{\PYGZpc{}B \PYGZpc{}Y}\PYG{l+s+si}{\PYGZcb{}}\PYG{l+s+s1}{ to }\PYG{l+s+si}{\PYGZob{}}\PYG{n}{returns}\PYG{o}{.}\PYG{n}{index}\PYG{p}{[}\PYG{o}{\PYGZhy{}}\PYG{l+m+mi}{1}\PYG{p}{]}\PYG{l+s+si}{:}\PYG{l+s+s1}{\PYGZpc{}B \PYGZpc{}Y}\PYG{l+s+si}{\PYGZcb{}}\PYG{l+s+s1}{\PYGZsq{}}\PYG{p}{)}
\PYG{n}{plt}\PYG{o}{.}\PYG{n}{show}\PYG{p}{(}\PYG{p}{)}
\end{sphinxVerbatim}

\end{sphinxuseclass}\end{sphinxVerbatimInput}
\begin{sphinxVerbatimOutput}

\begin{sphinxuseclass}{cell_output}
\noindent\sphinxincludegraphics{{4d1080320a7838611340f073bba83ed55b27094ed4276f5cb3c2e5fd52438d19}.png}

\end{sphinxuseclass}\end{sphinxVerbatimOutput}

\end{sphinxuseclass}

\subsubsection{Which stocks have the minimum and maximum total returns?}
\label{\detokenize{mckinney_05_practice_02:which-stocks-have-the-minimum-and-maximum-total-returns}}
\begin{sphinxuseclass}{cell}\begin{sphinxVerbatimInput}

\begin{sphinxuseclass}{cell_input}
\begin{sphinxVerbatim}[commandchars=\\\{\}]
\PYG{n}{returns\PYGZus{}2\PYGZus{}total} \PYG{o}{=} \PYG{n}{returns\PYGZus{}2}\PYG{o}{.}\PYG{n}{add}\PYG{p}{(}\PYG{l+m+mi}{1}\PYG{p}{)}\PYG{o}{.}\PYG{n}{prod}\PYG{p}{(}\PYG{p}{)}\PYG{o}{.}\PYG{n}{sub}\PYG{p}{(}\PYG{l+m+mi}{1}\PYG{p}{)}
\end{sphinxVerbatim}

\end{sphinxuseclass}\end{sphinxVerbatimInput}

\end{sphinxuseclass}
\begin{sphinxuseclass}{cell}\begin{sphinxVerbatimInput}

\begin{sphinxuseclass}{cell_input}
\begin{sphinxVerbatim}[commandchars=\\\{\}]
\PYG{n}{returns\PYGZus{}2\PYGZus{}total}\PYG{o}{.}\PYG{n}{min}\PYG{p}{(}\PYG{p}{)}
\end{sphinxVerbatim}

\end{sphinxuseclass}\end{sphinxVerbatimInput}
\begin{sphinxVerbatimOutput}

\begin{sphinxuseclass}{cell_output}
\begin{sphinxVerbatim}[commandchars=\\\{\}]
\PYGZhy{}0.4154
\end{sphinxVerbatim}

\end{sphinxuseclass}\end{sphinxVerbatimOutput}

\end{sphinxuseclass}
\begin{sphinxuseclass}{cell}\begin{sphinxVerbatimInput}

\begin{sphinxuseclass}{cell_input}
\begin{sphinxVerbatim}[commandchars=\\\{\}]
\PYG{n}{returns\PYGZus{}2\PYGZus{}total}\PYG{o}{.}\PYG{n}{idxmin}\PYG{p}{(}\PYG{p}{)}
\end{sphinxVerbatim}

\end{sphinxuseclass}\end{sphinxVerbatimInput}
\begin{sphinxVerbatimOutput}

\begin{sphinxuseclass}{cell_output}
\begin{sphinxVerbatim}[commandchars=\\\{\}]
\PYGZsq{}BA\PYGZsq{}
\end{sphinxVerbatim}

\end{sphinxuseclass}\end{sphinxVerbatimOutput}

\end{sphinxuseclass}
\begin{sphinxuseclass}{cell}\begin{sphinxVerbatimInput}

\begin{sphinxuseclass}{cell_input}
\begin{sphinxVerbatim}[commandchars=\\\{\}]
\PYG{n}{returns\PYGZus{}2\PYGZus{}total}\PYG{o}{.}\PYG{n}{max}\PYG{p}{(}\PYG{p}{)}
\end{sphinxVerbatim}

\end{sphinxuseclass}\end{sphinxVerbatimInput}
\begin{sphinxVerbatimOutput}

\begin{sphinxuseclass}{cell_output}
\begin{sphinxVerbatim}[commandchars=\\\{\}]
2.1895
\end{sphinxVerbatim}

\end{sphinxuseclass}\end{sphinxVerbatimOutput}

\end{sphinxuseclass}
\begin{sphinxuseclass}{cell}\begin{sphinxVerbatimInput}

\begin{sphinxuseclass}{cell_input}
\begin{sphinxVerbatim}[commandchars=\\\{\}]
\PYG{n}{returns\PYGZus{}2\PYGZus{}total}\PYG{o}{.}\PYG{n}{idxmax}\PYG{p}{(}\PYG{p}{)}
\end{sphinxVerbatim}

\end{sphinxuseclass}\end{sphinxVerbatimInput}
\begin{sphinxVerbatimOutput}

\begin{sphinxuseclass}{cell_output}
\begin{sphinxVerbatim}[commandchars=\\\{\}]
\PYGZsq{}AAPL\PYGZsq{}
\end{sphinxVerbatim}

\end{sphinxuseclass}\end{sphinxVerbatimOutput}

\end{sphinxuseclass}
\begin{sphinxuseclass}{cell}\begin{sphinxVerbatimInput}

\begin{sphinxuseclass}{cell_input}
\begin{sphinxVerbatim}[commandchars=\\\{\}]
\PYG{n}{returns\PYGZus{}2\PYGZus{}total}\PYG{o}{.}\PYG{n}{sort\PYGZus{}values}\PYG{p}{(}\PYG{p}{)}\PYG{o}{.}\PYG{n}{iloc}\PYG{p}{[}\PYG{p}{[}\PYG{l+m+mi}{0}\PYG{p}{,} \PYG{o}{\PYGZhy{}}\PYG{l+m+mi}{1}\PYG{p}{]}\PYG{p}{]}
\end{sphinxVerbatim}

\end{sphinxuseclass}\end{sphinxVerbatimInput}
\begin{sphinxVerbatimOutput}

\begin{sphinxuseclass}{cell_output}
\begin{sphinxVerbatim}[commandchars=\\\{\}]
Ticker
BA     \PYGZhy{}0.4154
AAPL    2.1895
dtype: float64
\end{sphinxVerbatim}

\end{sphinxuseclass}\end{sphinxVerbatimOutput}

\end{sphinxuseclass}
\begin{sphinxuseclass}{cell}\begin{sphinxVerbatimInput}

\begin{sphinxuseclass}{cell_input}
\begin{sphinxVerbatim}[commandchars=\\\{\}]
\PYG{n}{returns\PYGZus{}2\PYGZus{}total}\PYG{o}{.}\PYG{n}{nsmallest}\PYG{p}{(}\PYG{p}{)}
\end{sphinxVerbatim}

\end{sphinxuseclass}\end{sphinxVerbatimInput}
\begin{sphinxVerbatimOutput}

\begin{sphinxuseclass}{cell_output}
\begin{sphinxVerbatim}[commandchars=\\\{\}]
Ticker
BA     \PYGZhy{}0.4154
INTC   \PYGZhy{}0.3986
MMM    \PYGZhy{}0.3630
WBA    \PYGZhy{}0.3114
VZ     \PYGZhy{}0.1284
dtype: float64
\end{sphinxVerbatim}

\end{sphinxuseclass}\end{sphinxVerbatimOutput}

\end{sphinxuseclass}
\begin{sphinxuseclass}{cell}\begin{sphinxVerbatimInput}

\begin{sphinxuseclass}{cell_input}
\begin{sphinxVerbatim}[commandchars=\\\{\}]
\PYG{n}{returns\PYGZus{}2\PYGZus{}total}\PYG{o}{.}\PYG{n}{nlargest}\PYG{p}{(}\PYG{p}{)}
\end{sphinxVerbatim}

\end{sphinxuseclass}\end{sphinxVerbatimInput}
\begin{sphinxVerbatimOutput}

\begin{sphinxuseclass}{cell_output}
\begin{sphinxVerbatim}[commandchars=\\\{\}]
Ticker
AAPL   2.1895
MSFT   1.2354
UNH    1.0982
CAT    1.0732
GS     1.0461
dtype: float64
\end{sphinxVerbatim}

\end{sphinxuseclass}\end{sphinxVerbatimOutput}

\end{sphinxuseclass}

\subsubsection{Plot the cumulative returns for the stocks in the DJIA}
\label{\detokenize{mckinney_05_practice_02:plot-the-cumulative-returns-for-the-stocks-in-the-djia}}
\sphinxAtStartPar
We can use the cumulative product method \sphinxcode{\sphinxupquote{.cumprod()}} to calculate the right hand side of the formula above.

\begin{sphinxuseclass}{cell}\begin{sphinxVerbatimInput}

\begin{sphinxuseclass}{cell_input}
\begin{sphinxVerbatim}[commandchars=\\\{\}]
\PYG{p}{(}
    \PYG{n}{returns\PYGZus{}2}
    \PYG{o}{.}\PYG{n}{add}\PYG{p}{(}\PYG{l+m+mi}{1}\PYG{p}{)}
    \PYG{o}{.}\PYG{n}{cumprod}\PYG{p}{(}\PYG{p}{)}
    \PYG{o}{.}\PYG{n}{sub}\PYG{p}{(}\PYG{l+m+mi}{1}\PYG{p}{)}
    \PYG{o}{.}\PYG{n}{mul}\PYG{p}{(}\PYG{l+m+mi}{100}\PYG{p}{)}
    \PYG{o}{.}\PYG{n}{plot}\PYG{p}{(}\PYG{p}{)}
\PYG{p}{)}
\PYG{n}{plt}\PYG{o}{.}\PYG{n}{ylabel}\PYG{p}{(}\PYG{l+s+s1}{\PYGZsq{}}\PYG{l+s+s1}{Cumulative Return (}\PYG{l+s+s1}{\PYGZpc{}}\PYG{l+s+s1}{)}\PYG{l+s+s1}{\PYGZsq{}}\PYG{p}{)}
\PYG{n}{plt}\PYG{o}{.}\PYG{n}{title}\PYG{p}{(}\PYG{l+s+sa}{f}\PYG{l+s+s1}{\PYGZsq{}}\PYG{l+s+s1}{Cumulative Returns for DJIA Stocks }\PYG{l+s+se}{\PYGZbs{}n}\PYG{l+s+s1}{ from }\PYG{l+s+si}{\PYGZob{}}\PYG{n}{returns}\PYG{o}{.}\PYG{n}{index}\PYG{p}{[}\PYG{l+m+mi}{0}\PYG{p}{]}\PYG{l+s+si}{:}\PYG{l+s+s1}{\PYGZpc{}B \PYGZpc{}Y}\PYG{l+s+si}{\PYGZcb{}}\PYG{l+s+s1}{ to }\PYG{l+s+si}{\PYGZob{}}\PYG{n}{returns}\PYG{o}{.}\PYG{n}{index}\PYG{p}{[}\PYG{o}{\PYGZhy{}}\PYG{l+m+mi}{1}\PYG{p}{]}\PYG{l+s+si}{:}\PYG{l+s+s1}{\PYGZpc{}B \PYGZpc{}Y}\PYG{l+s+si}{\PYGZcb{}}\PYG{l+s+s1}{\PYGZsq{}}\PYG{p}{)}
\PYG{n}{plt}\PYG{o}{.}\PYG{n}{show}\PYG{p}{(}\PYG{p}{)}
\end{sphinxVerbatim}

\end{sphinxuseclass}\end{sphinxVerbatimInput}
\begin{sphinxVerbatimOutput}

\begin{sphinxuseclass}{cell_output}
\noindent\sphinxincludegraphics{{6076f301717c19a5969428bf65e6914c3e281e67c849ffba29596ab9e7107dfd}.png}

\end{sphinxuseclass}\end{sphinxVerbatimOutput}

\end{sphinxuseclass}

\subsubsection{Repeat the plot above with only the minimum and maximum total returns}
\label{\detokenize{mckinney_05_practice_02:repeat-the-plot-above-with-only-the-minimum-and-maximum-total-returns}}
\begin{sphinxuseclass}{cell}\begin{sphinxVerbatimInput}

\begin{sphinxuseclass}{cell_input}
\begin{sphinxVerbatim}[commandchars=\\\{\}]
\PYG{p}{(}
    \PYG{n}{returns\PYGZus{}2}\PYG{p}{[}\PYG{p}{[}\PYG{n}{returns\PYGZus{}2\PYGZus{}total}\PYG{o}{.}\PYG{n}{idxmax}\PYG{p}{(}\PYG{p}{)}\PYG{p}{,} \PYG{n}{returns\PYGZus{}2\PYGZus{}total}\PYG{o}{.}\PYG{n}{idxmin}\PYG{p}{(}\PYG{p}{)}\PYG{p}{]}\PYG{p}{]}
    \PYG{o}{.}\PYG{n}{add}\PYG{p}{(}\PYG{l+m+mi}{1}\PYG{p}{)}
    \PYG{o}{.}\PYG{n}{cumprod}\PYG{p}{(}\PYG{p}{)}
    \PYG{o}{.}\PYG{n}{sub}\PYG{p}{(}\PYG{l+m+mi}{1}\PYG{p}{)}
    \PYG{o}{.}\PYG{n}{mul}\PYG{p}{(}\PYG{l+m+mi}{100}\PYG{p}{)}
    \PYG{o}{.}\PYG{n}{plot}\PYG{p}{(}\PYG{p}{)}
\PYG{p}{)}
\PYG{n}{plt}\PYG{o}{.}\PYG{n}{ylabel}\PYG{p}{(}\PYG{l+s+s1}{\PYGZsq{}}\PYG{l+s+s1}{Cumulative Return (}\PYG{l+s+s1}{\PYGZpc{}}\PYG{l+s+s1}{)}\PYG{l+s+s1}{\PYGZsq{}}\PYG{p}{)}
\PYG{n}{plt}\PYG{o}{.}\PYG{n}{title}\PYG{p}{(}\PYG{l+s+sa}{f}\PYG{l+s+s1}{\PYGZsq{}}\PYG{l+s+s1}{Cumulative Returns for Best and Worst DJIA Stocks }\PYG{l+s+se}{\PYGZbs{}n}\PYG{l+s+s1}{ from }\PYG{l+s+si}{\PYGZob{}}\PYG{n}{returns}\PYG{o}{.}\PYG{n}{index}\PYG{p}{[}\PYG{l+m+mi}{0}\PYG{p}{]}\PYG{l+s+si}{:}\PYG{l+s+s1}{\PYGZpc{}B \PYGZpc{}Y}\PYG{l+s+si}{\PYGZcb{}}\PYG{l+s+s1}{ to }\PYG{l+s+si}{\PYGZob{}}\PYG{n}{returns}\PYG{o}{.}\PYG{n}{index}\PYG{p}{[}\PYG{o}{\PYGZhy{}}\PYG{l+m+mi}{1}\PYG{p}{]}\PYG{l+s+si}{:}\PYG{l+s+s1}{\PYGZpc{}B \PYGZpc{}Y}\PYG{l+s+si}{\PYGZcb{}}\PYG{l+s+s1}{\PYGZsq{}}\PYG{p}{)}
\PYG{n}{plt}\PYG{o}{.}\PYG{n}{show}\PYG{p}{(}\PYG{p}{)}
\end{sphinxVerbatim}

\end{sphinxuseclass}\end{sphinxVerbatimInput}
\begin{sphinxVerbatimOutput}

\begin{sphinxuseclass}{cell_output}
\noindent\sphinxincludegraphics{{3644cd3e0275e36ad55f6b74192242d07d74755ef718887c3641e324abe4866c}.png}

\end{sphinxuseclass}\end{sphinxVerbatimOutput}

\end{sphinxuseclass}
\sphinxstepscope


\chapter{Herron Topic 1 \sphinxhyphen{} Web Data, Log and Simple Returns, and Portfolio Math}
\label{\detokenize{herron_01_lecture:herron-topic-1-web-data-log-and-simple-returns-and-portfolio-math}}\label{\detokenize{herron_01_lecture::doc}}
\sphinxAtStartPar
This notebook covers three topics:
\begin{enumerate}
\sphinxsetlistlabels{\arabic}{enumi}{enumii}{}{.}%
\item {} 
\sphinxAtStartPar
How to download web data with the yfinance, pandas\sphinxhyphen{}datareader, and requests\sphinxhyphen{}cache packages

\item {} 
\sphinxAtStartPar
How to calculate log and simple returns

\item {} 
\sphinxAtStartPar
How to calculate portfolio returns

\end{enumerate}

\begin{sphinxuseclass}{cell}\begin{sphinxVerbatimInput}

\begin{sphinxuseclass}{cell_input}
\begin{sphinxVerbatim}[commandchars=\\\{\}]
\PYG{k+kn}{import} \PYG{n+nn}{matplotlib}\PYG{n+nn}{.}\PYG{n+nn}{pyplot} \PYG{k}{as} \PYG{n+nn}{plt}
\PYG{k+kn}{import} \PYG{n+nn}{numpy} \PYG{k}{as} \PYG{n+nn}{np}
\PYG{k+kn}{import} \PYG{n+nn}{pandas} \PYG{k}{as} \PYG{n+nn}{pd}
\end{sphinxVerbatim}

\end{sphinxuseclass}\end{sphinxVerbatimInput}

\end{sphinxuseclass}
\begin{sphinxuseclass}{cell}\begin{sphinxVerbatimInput}

\begin{sphinxuseclass}{cell_input}
\begin{sphinxVerbatim}[commandchars=\\\{\}]
\PYG{o}{\PYGZpc{}}\PYG{k}{config} InlineBackend.figure\PYGZus{}format = \PYGZsq{}retina\PYGZsq{}
\PYG{o}{\PYGZpc{}}\PYG{k}{precision} 4
\PYG{n}{pd}\PYG{o}{.}\PYG{n}{options}\PYG{o}{.}\PYG{n}{display}\PYG{o}{.}\PYG{n}{float\PYGZus{}format} \PYG{o}{=} \PYG{l+s+s1}{\PYGZsq{}}\PYG{l+s+si}{\PYGZob{}:.4f\PYGZcb{}}\PYG{l+s+s1}{\PYGZsq{}}\PYG{o}{.}\PYG{n}{format}
\end{sphinxVerbatim}

\end{sphinxuseclass}\end{sphinxVerbatimInput}

\end{sphinxuseclass}

\section{Web Data}
\label{\detokenize{herron_01_lecture:web-data}}
\sphinxAtStartPar
We will typically use the yfinance and pandas\sphinxhyphen{}datarader packages (combined with the requests\sphinxhyphen{}cache package) to download data from the web.
\begin{itemize}
\item {} 
\sphinxAtStartPar
If you followed my instructions to install Anaconda on your computer, you have already installed these packages

\item {} 
\sphinxAtStartPar
If you use DataCamp Workspace or Binder, I have already installed these packages

\item {} 
\sphinxAtStartPar
If you use Notheastern’s Open OnDemand, they are working on my request, and you will have to install these packages every login by running the following in a code cell: \sphinxcode{\sphinxupquote{\%pip install yfinance pandas\sphinxhyphen{}datareader requests\sphinxhyphen{}cache}}

\end{itemize}


\subsection{The yfinance Package}
\label{\detokenize{herron_01_lecture:the-yfinance-package}}
\sphinxAtStartPar
The \sphinxhref{https://github.com/ranaroussi/yfinance}{yfinance package} provides “a reliable, threaded, and Pythonic way to download historical market data from Yahoo! finance.”
Other packages provide similar functionality, but yfinance is best.
We will use the \sphinxhref{https://github.com/requests-cache/requests-cache}{requests\sphinxhyphen{}cache package} to cache our data downloads locally.
This local cache lets us reduce the number of times we ask the Yahoo! Finance application programming interface (API).

\begin{sphinxuseclass}{cell}\begin{sphinxVerbatimInput}

\begin{sphinxuseclass}{cell_input}
\begin{sphinxVerbatim}[commandchars=\\\{\}]
\PYG{k+kn}{import} \PYG{n+nn}{yfinance} \PYG{k}{as} \PYG{n+nn}{yf}
\PYG{k+kn}{import} \PYG{n+nn}{requests\PYGZus{}cache}
\PYG{n}{session} \PYG{o}{=} \PYG{n}{requests\PYGZus{}cache}\PYG{o}{.}\PYG{n}{CachedSession}\PYG{p}{(}\PYG{n}{expire\PYGZus{}after}\PYG{o}{=}\PYG{l+m+mi}{1}\PYG{p}{)}
\end{sphinxVerbatim}

\end{sphinxuseclass}\end{sphinxVerbatimInput}

\end{sphinxuseclass}
\sphinxAtStartPar
We can download data for the MATANA stocks (Microsoft, Alphabet, Tesla, Amazon, Nvidia, and Apple).
We can pass tickers as either a space\sphinxhyphen{}delimited string or a list of strings.

\begin{sphinxuseclass}{cell}\begin{sphinxVerbatimInput}

\begin{sphinxuseclass}{cell_input}
\begin{sphinxVerbatim}[commandchars=\\\{\}]
\PYG{n}{tickers} \PYG{o}{=} \PYG{n}{yf}\PYG{o}{.}\PYG{n}{Tickers}\PYG{p}{(}\PYG{n}{tickers}\PYG{o}{=}\PYG{l+s+s1}{\PYGZsq{}}\PYG{l+s+s1}{MSFT GOOG TSLA AMZN NVDA AAPL}\PYG{l+s+s1}{\PYGZsq{}}\PYG{p}{,} \PYG{n}{session}\PYG{o}{=}\PYG{n}{session}\PYG{p}{)}
\PYG{n}{histories} \PYG{o}{=} \PYG{n}{tickers}\PYG{o}{.}\PYG{n}{history}\PYG{p}{(}\PYG{n}{period}\PYG{o}{=}\PYG{l+s+s1}{\PYGZsq{}}\PYG{l+s+s1}{max}\PYG{l+s+s1}{\PYGZsq{}}\PYG{p}{,} \PYG{n}{auto\PYGZus{}adjust}\PYG{o}{=}\PYG{k+kc}{False}\PYG{p}{,} \PYG{n}{progress}\PYG{o}{=}\PYG{k+kc}{False}\PYG{p}{)}
\PYG{n}{histories}\PYG{o}{.}\PYG{n}{index} \PYG{o}{=} \PYG{n}{histories}\PYG{o}{.}\PYG{n}{index}\PYG{o}{.}\PYG{n}{tz\PYGZus{}localize}\PYG{p}{(}\PYG{k+kc}{None}\PYG{p}{)}
\PYG{n}{histories}
\end{sphinxVerbatim}

\end{sphinxuseclass}\end{sphinxVerbatimInput}
\begin{sphinxVerbatimOutput}

\begin{sphinxuseclass}{cell_output}
\begin{sphinxVerbatim}[commandchars=\\\{\}]
           Adj Close                                                 Close  \PYGZbs{}
                AAPL     AMZN     GOOG     MSFT     NVDA     TSLA     AAPL   
Date                                                                         
1980\PYGZhy{}12\PYGZhy{}12    0.0999      NaN      NaN      NaN      NaN      NaN   0.1283   
1980\PYGZhy{}12\PYGZhy{}15    0.0947      NaN      NaN      NaN      NaN      NaN   0.1217   
1980\PYGZhy{}12\PYGZhy{}16    0.0877      NaN      NaN      NaN      NaN      NaN   0.1127   
1980\PYGZhy{}12\PYGZhy{}17    0.0899      NaN      NaN      NaN      NaN      NaN   0.1155   
1980\PYGZhy{}12\PYGZhy{}18    0.0925      NaN      NaN      NaN      NaN      NaN   0.1189   
...              ...      ...      ...      ...      ...      ...      ...   
2023\PYGZhy{}01\PYGZhy{}30  143.0000 100.5500  97.9500 242.7100 191.6200 166.6600 143.0000   
2023\PYGZhy{}01\PYGZhy{}31  144.2900 103.1300  99.8700 247.8100 195.3700 173.2200 144.2900   
2023\PYGZhy{}02\PYGZhy{}01  145.4300 105.1500 101.4300 252.7500 209.4300 181.4100 145.4300   
2023\PYGZhy{}02\PYGZhy{}02  150.8200 112.9100 108.8000 264.6000 217.0900 188.2700 150.8200   
2023\PYGZhy{}02\PYGZhy{}03  154.8900 103.2600 105.0700 258.3100 210.5624      NaN 154.8900   

                                       ... Stock Splits                       \PYGZbs{}
               AMZN     GOOG     MSFT  ...         GOOG   MSFT   NVDA   TSLA   
Date                                   ...                                     
1980\PYGZhy{}12\PYGZhy{}12      NaN      NaN      NaN  ...          NaN    NaN    NaN    NaN   
1980\PYGZhy{}12\PYGZhy{}15      NaN      NaN      NaN  ...          NaN    NaN    NaN    NaN   
1980\PYGZhy{}12\PYGZhy{}16      NaN      NaN      NaN  ...          NaN    NaN    NaN    NaN   
1980\PYGZhy{}12\PYGZhy{}17      NaN      NaN      NaN  ...          NaN    NaN    NaN    NaN   
1980\PYGZhy{}12\PYGZhy{}18      NaN      NaN      NaN  ...          NaN    NaN    NaN    NaN   
...             ...      ...      ...  ...          ...    ...    ...    ...   
2023\PYGZhy{}01\PYGZhy{}30 100.5500  97.9500 242.7100  ...       0.0000 0.0000 0.0000 0.0000   
2023\PYGZhy{}01\PYGZhy{}31 103.1300  99.8700 247.8100  ...       0.0000 0.0000 0.0000 0.0000   
2023\PYGZhy{}02\PYGZhy{}01 105.1500 101.4300 252.7500  ...       0.0000 0.0000 0.0000 0.0000   
2023\PYGZhy{}02\PYGZhy{}02 112.9100 108.8000 264.6000  ...       0.0000 0.0000 0.0000 0.0000   
2023\PYGZhy{}02\PYGZhy{}03 103.2600 105.0700 258.3100  ...       0.0000 0.0000 0.0000    NaN   

               Volume                                             \PYGZbs{}
                 AAPL           AMZN          GOOG          MSFT   
Date                                                               
1980\PYGZhy{}12\PYGZhy{}12  469033600            NaN           NaN           NaN   
1980\PYGZhy{}12\PYGZhy{}15  175884800            NaN           NaN           NaN   
1980\PYGZhy{}12\PYGZhy{}16  105728000            NaN           NaN           NaN   
1980\PYGZhy{}12\PYGZhy{}17   86441600            NaN           NaN           NaN   
1980\PYGZhy{}12\PYGZhy{}18   73449600            NaN           NaN           NaN   
...               ...            ...           ...           ...   
2023\PYGZhy{}01\PYGZhy{}30   64015300  70691900.0000 24365100.0000 25867400.0000   
2023\PYGZhy{}01\PYGZhy{}31   65874500  66527300.0000 22306800.0000 26541100.0000   
2023\PYGZhy{}02\PYGZhy{}01   77663600  80450100.0000 26392600.0000 31259900.0000   
2023\PYGZhy{}02\PYGZhy{}02  118339000 158154200.0000 46622600.0000 39940400.0000   
2023\PYGZhy{}02\PYGZhy{}03  130069253 118655627.0000 29959465.0000 20269785.0000   

                                         
                    NVDA           TSLA  
Date                                     
1980\PYGZhy{}12\PYGZhy{}12           NaN            NaN  
1980\PYGZhy{}12\PYGZhy{}15           NaN            NaN  
1980\PYGZhy{}12\PYGZhy{}16           NaN            NaN  
1980\PYGZhy{}12\PYGZhy{}17           NaN            NaN  
1980\PYGZhy{}12\PYGZhy{}18           NaN            NaN  
...                  ...            ...  
2023\PYGZhy{}01\PYGZhy{}30 48861100.0000 230878800.0000  
2023\PYGZhy{}01\PYGZhy{}31 49801700.0000 196813500.0000  
2023\PYGZhy{}02\PYGZhy{}01 66047700.0000 213806300.0000  
2023\PYGZhy{}02\PYGZhy{}02 56427600.0000 217448300.0000  
2023\PYGZhy{}02\PYGZhy{}03 36649525.0000            NaN  

[10626 rows x 48 columns]
\end{sphinxVerbatim}

\end{sphinxuseclass}\end{sphinxVerbatimOutput}

\end{sphinxuseclass}
\begin{sphinxuseclass}{cell}\begin{sphinxVerbatimInput}

\begin{sphinxuseclass}{cell_input}
\begin{sphinxVerbatim}[commandchars=\\\{\}]
\PYG{p}{(} \PYG{c+c1}{\PYGZsh{} Python ignores line breaks and white space inside ()}
    \PYG{n}{histories} \PYG{c+c1}{\PYGZsh{} start with MATANA data frame}
    \PYG{p}{[}\PYG{l+s+s1}{\PYGZsq{}}\PYG{l+s+s1}{Adj Close}\PYG{l+s+s1}{\PYGZsq{}}\PYG{p}{]} \PYG{c+c1}{\PYGZsh{} slice adjusted close columns}
    \PYG{o}{.}\PYG{n}{pct\PYGZus{}change}\PYG{p}{(}\PYG{p}{)} \PYG{c+c1}{\PYGZsh{} calculate simple returns}
    \PYG{o}{.}\PYG{n}{loc}\PYG{p}{[}\PYG{l+s+s1}{\PYGZsq{}}\PYG{l+s+s1}{2022}\PYG{l+s+s1}{\PYGZsq{}}\PYG{p}{]} \PYG{c+c1}{\PYGZsh{} select 2022 returns}
    \PYG{o}{.}\PYG{n}{add}\PYG{p}{(}\PYG{l+m+mi}{1}\PYG{p}{)} \PYG{c+c1}{\PYGZsh{} add 1}
    \PYG{o}{.}\PYG{n}{cumprod}\PYG{p}{(}\PYG{p}{)} \PYG{c+c1}{\PYGZsh{} compound cumulative returns}
    \PYG{o}{.}\PYG{n}{sub}\PYG{p}{(}\PYG{l+m+mi}{1}\PYG{p}{)} \PYG{c+c1}{\PYGZsh{} subtract 1}
    \PYG{o}{.}\PYG{n}{mul}\PYG{p}{(}\PYG{l+m+mi}{100}\PYG{p}{)} \PYG{c+c1}{\PYGZsh{} convert decimals to percent}
    \PYG{o}{.}\PYG{n}{plot}\PYG{p}{(}\PYG{p}{)} \PYG{c+c1}{\PYGZsh{} plot}
\PYG{p}{)}
\PYG{n}{plt}\PYG{o}{.}\PYG{n}{ylabel}\PYG{p}{(}\PYG{l+s+s1}{\PYGZsq{}}\PYG{l+s+s1}{Year\PYGZhy{}to\PYGZhy{}Date Return (}\PYG{l+s+s1}{\PYGZpc{}}\PYG{l+s+s1}{)}\PYG{l+s+s1}{\PYGZsq{}}\PYG{p}{)}
\PYG{n}{plt}\PYG{o}{.}\PYG{n}{title}\PYG{p}{(}\PYG{l+s+s1}{\PYGZsq{}}\PYG{l+s+s1}{Year\PYGZhy{}to\PYGZhy{}Date Returns for MATANA Stocks}\PYG{l+s+s1}{\PYGZsq{}}\PYG{p}{)}
\PYG{n}{plt}\PYG{o}{.}\PYG{n}{show}\PYG{p}{(}\PYG{p}{)}
\end{sphinxVerbatim}

\end{sphinxuseclass}\end{sphinxVerbatimInput}
\begin{sphinxVerbatimOutput}

\begin{sphinxuseclass}{cell_output}
\noindent\sphinxincludegraphics{{61ce6f23b9e0a0d877336fe33293b48ace81e194642b157c516922e1ef9fe741}.png}

\end{sphinxuseclass}\end{sphinxVerbatimOutput}

\end{sphinxuseclass}

\subsection{The pandas\sphinxhyphen{}datareader package}
\label{\detokenize{herron_01_lecture:the-pandas-datareader-package}}
\sphinxAtStartPar
The \sphinxhref{https://pandas-datareader.readthedocs.io/en/latest/index.html}{pandas\sphinxhyphen{}datareader} package provides easy access to various data sources, including \sphinxhref{https://mba.tuck.dartmouth.edu/pages/faculty/ken.french/data\_library.html}{the Kenneth French Data Library} and \sphinxhref{https://fred.stlouisfed.org/}{the Federal Reserve Economic Data (FRED)}.
The pandas\sphinxhyphen{}datareader package also downloads Yahoo! Finance data, but the yfinance package has better documentation.
We will use \sphinxcode{\sphinxupquote{pdr}} as the abbreviated prefix for pandas\sphinxhyphen{}datareader.

\begin{sphinxuseclass}{cell}\begin{sphinxVerbatimInput}

\begin{sphinxuseclass}{cell_input}
\begin{sphinxVerbatim}[commandchars=\\\{\}]
\PYG{k+kn}{import} \PYG{n+nn}{pandas\PYGZus{}datareader} \PYG{k}{as} \PYG{n+nn}{pdr}
\end{sphinxVerbatim}

\end{sphinxuseclass}\end{sphinxVerbatimInput}

\end{sphinxuseclass}
\sphinxAtStartPar
Here we download the daily benchmark factors from Ken French’s Data Library.

\begin{sphinxuseclass}{cell}\begin{sphinxVerbatimInput}

\begin{sphinxuseclass}{cell_input}
\begin{sphinxVerbatim}[commandchars=\\\{\}]
\PYG{n}{pdr}\PYG{o}{.}\PYG{n}{famafrench}\PYG{o}{.}\PYG{n}{get\PYGZus{}available\PYGZus{}datasets}\PYG{p}{(}\PYG{n}{session}\PYG{o}{=}\PYG{n}{session}\PYG{p}{)}\PYG{p}{[}\PYG{p}{:}\PYG{l+m+mi}{5}\PYG{p}{]}
\end{sphinxVerbatim}

\end{sphinxuseclass}\end{sphinxVerbatimInput}
\begin{sphinxVerbatimOutput}

\begin{sphinxuseclass}{cell_output}
\begin{sphinxVerbatim}[commandchars=\\\{\}]
[\PYGZsq{}F\PYGZhy{}F\PYGZus{}Research\PYGZus{}Data\PYGZus{}Factors\PYGZsq{},
 \PYGZsq{}F\PYGZhy{}F\PYGZus{}Research\PYGZus{}Data\PYGZus{}Factors\PYGZus{}weekly\PYGZsq{},
 \PYGZsq{}F\PYGZhy{}F\PYGZus{}Research\PYGZus{}Data\PYGZus{}Factors\PYGZus{}daily\PYGZsq{},
 \PYGZsq{}F\PYGZhy{}F\PYGZus{}Research\PYGZus{}Data\PYGZus{}5\PYGZus{}Factors\PYGZus{}2x3\PYGZsq{},
 \PYGZsq{}F\PYGZhy{}F\PYGZus{}Research\PYGZus{}Data\PYGZus{}5\PYGZus{}Factors\PYGZus{}2x3\PYGZus{}daily\PYGZsq{}]
\end{sphinxVerbatim}

\end{sphinxuseclass}\end{sphinxVerbatimOutput}

\end{sphinxuseclass}
\sphinxAtStartPar
For Fama and French data, pandas\sphinxhyphen{}datareader returns the most recent five years of data unless we specify a \sphinxcode{\sphinxupquote{start}} date.
French typically provides data back through the second half of 1926.
pandas\sphinxhyphen{}datareader returns dictionaries of data frames, and the \sphinxcode{\sphinxupquote{'DESCR'}} value describes these data frames.

\begin{sphinxuseclass}{cell}\begin{sphinxVerbatimInput}

\begin{sphinxuseclass}{cell_input}
\begin{sphinxVerbatim}[commandchars=\\\{\}]
\PYG{n}{ff\PYGZus{}all} \PYG{o}{=} \PYG{n}{pdr}\PYG{o}{.}\PYG{n}{DataReader}\PYG{p}{(}
    \PYG{n}{name}\PYG{o}{=}\PYG{l+s+s1}{\PYGZsq{}}\PYG{l+s+s1}{F\PYGZhy{}F\PYGZus{}Research\PYGZus{}Data\PYGZus{}Factors\PYGZus{}daily}\PYG{l+s+s1}{\PYGZsq{}}\PYG{p}{,}
    \PYG{n}{data\PYGZus{}source}\PYG{o}{=}\PYG{l+s+s1}{\PYGZsq{}}\PYG{l+s+s1}{famafrench}\PYG{l+s+s1}{\PYGZsq{}}\PYG{p}{,}
    \PYG{n}{start}\PYG{o}{=}\PYG{l+s+s1}{\PYGZsq{}}\PYG{l+s+s1}{1900}\PYG{l+s+s1}{\PYGZsq{}}\PYG{p}{,} 
    \PYG{n}{session}\PYG{o}{=}\PYG{n}{session}
\PYG{p}{)}
\end{sphinxVerbatim}

\end{sphinxuseclass}\end{sphinxVerbatimInput}

\end{sphinxuseclass}
\begin{sphinxuseclass}{cell}\begin{sphinxVerbatimInput}

\begin{sphinxuseclass}{cell_input}
\begin{sphinxVerbatim}[commandchars=\\\{\}]
\PYG{n+nb}{print}\PYG{p}{(}\PYG{n}{ff\PYGZus{}all}\PYG{p}{[}\PYG{l+s+s1}{\PYGZsq{}}\PYG{l+s+s1}{DESCR}\PYG{l+s+s1}{\PYGZsq{}}\PYG{p}{]}\PYG{p}{)}
\end{sphinxVerbatim}

\end{sphinxuseclass}\end{sphinxVerbatimInput}
\begin{sphinxVerbatimOutput}

\begin{sphinxuseclass}{cell_output}
\begin{sphinxVerbatim}[commandchars=\\\{\}]
F\PYGZhy{}F Research Data Factors daily
\PYGZhy{}\PYGZhy{}\PYGZhy{}\PYGZhy{}\PYGZhy{}\PYGZhy{}\PYGZhy{}\PYGZhy{}\PYGZhy{}\PYGZhy{}\PYGZhy{}\PYGZhy{}\PYGZhy{}\PYGZhy{}\PYGZhy{}\PYGZhy{}\PYGZhy{}\PYGZhy{}\PYGZhy{}\PYGZhy{}\PYGZhy{}\PYGZhy{}\PYGZhy{}\PYGZhy{}\PYGZhy{}\PYGZhy{}\PYGZhy{}\PYGZhy{}\PYGZhy{}\PYGZhy{}\PYGZhy{}

This file was created by CMPT\PYGZus{}ME\PYGZus{}BEME\PYGZus{}RETS\PYGZus{}DAILY using the 202212 CRSP database. The Tbill return is the simple daily rate that, over the number of trading days in the month, compounds to 1\PYGZhy{}month TBill rate from Ibbotson and Associates Inc. Copyright 2022 Kenneth R. French

  0 : (25399 rows x 4 cols)
\end{sphinxVerbatim}

\end{sphinxuseclass}\end{sphinxVerbatimOutput}

\end{sphinxuseclass}
\begin{sphinxuseclass}{cell}\begin{sphinxVerbatimInput}

\begin{sphinxuseclass}{cell_input}
\begin{sphinxVerbatim}[commandchars=\\\{\}]
\PYG{p}{(}
    \PYG{n}{ff\PYGZus{}all}\PYG{p}{[}\PYG{l+m+mi}{0}\PYG{p}{]}
    \PYG{o}{.}\PYG{n}{div}\PYG{p}{(}\PYG{l+m+mi}{100}\PYG{p}{)}
    \PYG{o}{.}\PYG{n}{add}\PYG{p}{(}\PYG{l+m+mi}{1}\PYG{p}{)}
    \PYG{o}{.}\PYG{n}{cumprod}\PYG{p}{(}\PYG{p}{)}
    \PYG{o}{.}\PYG{n}{sub}\PYG{p}{(}\PYG{l+m+mi}{1}\PYG{p}{)}
    \PYG{o}{.}\PYG{n}{mul}\PYG{p}{(}\PYG{l+m+mi}{100}\PYG{p}{)}
    \PYG{o}{.}\PYG{n}{plot}\PYG{p}{(}\PYG{p}{)}
\PYG{p}{)}
\PYG{n}{plt}\PYG{o}{.}\PYG{n}{ylabel}\PYG{p}{(}\PYG{l+s+s1}{\PYGZsq{}}\PYG{l+s+s1}{Cumulative Return (}\PYG{l+s+s1}{\PYGZpc{}}\PYG{l+s+s1}{)}\PYG{l+s+s1}{\PYGZsq{}}\PYG{p}{)}
\PYG{n}{plt}\PYG{o}{.}\PYG{n}{title}\PYG{p}{(}\PYG{l+s+s1}{\PYGZsq{}}\PYG{l+s+s1}{Cumulative Returns for the Daily Benchmark Factors (}\PYG{l+s+s1}{\PYGZpc{}}\PYG{l+s+s1}{)}\PYG{l+s+s1}{\PYGZsq{}}\PYG{p}{)}
\PYG{n}{plt}\PYG{o}{.}\PYG{n}{gca}\PYG{p}{(}\PYG{p}{)}\PYG{o}{.}\PYG{n}{yaxis}\PYG{o}{.}\PYG{n}{set\PYGZus{}major\PYGZus{}formatter}\PYG{p}{(}\PYG{n}{plt}\PYG{o}{.}\PYG{n}{matplotlib}\PYG{o}{.}\PYG{n}{ticker}\PYG{o}{.}\PYG{n}{StrMethodFormatter}\PYG{p}{(}\PYG{l+s+s1}{\PYGZsq{}}\PYG{l+s+si}{\PYGZob{}x:,.0f\PYGZcb{}}\PYG{l+s+s1}{\PYGZsq{}}\PYG{p}{)}\PYG{p}{)}
\PYG{n}{plt}\PYG{o}{.}\PYG{n}{show}\PYG{p}{(}\PYG{p}{)}
\end{sphinxVerbatim}

\end{sphinxuseclass}\end{sphinxVerbatimInput}
\begin{sphinxVerbatimOutput}

\begin{sphinxuseclass}{cell_output}
\noindent\sphinxincludegraphics{{75ed8cf829f66de137a937463bb33050701a9e7f3fec815e3ffd36c7f1c1f458}.png}

\end{sphinxuseclass}\end{sphinxVerbatimOutput}

\end{sphinxuseclass}

\section{Log and Simple Returns}
\label{\detokenize{herron_01_lecture:log-and-simple-returns}}
\sphinxAtStartPar
We will typically use \sphinxstyleemphasis{simple} returns, calculated as \$R\_\{simple,t\} = \textbackslash{}frac\{P\_t + D\_t \sphinxhyphen{} P\_\{t\sphinxhyphen{}1\}\}\{P\_\{t\sphinxhyphen{}1\}\} = \textbackslash{}frac\{P\_t + D\_t\}\{P\_\{t\sphinxhyphen{}1\}\} \sphinxhyphen{} 1\$.
The simple return is the return that investors receive on invested dollars.
We can calculate simple returns from Yahoo Finance data with the \sphinxcode{\sphinxupquote{.pct\_change()}} method on the adjusted close column (i.e., \sphinxcode{\sphinxupquote{Adj Close}}), which adjusts for dividends and splits.
The adjusted close column is a reverse\sphinxhyphen{}engineered close price (i.e., end\sphinxhyphen{}of\sphinxhyphen{}trading\sphinxhyphen{}day price) that incorporates dividends and splits, making simple return calculations easy.

\sphinxAtStartPar
However, we may see \sphinxstyleemphasis{log} returns elsewhere, which are the (natural) log of one plus simple returns:

\sphinxAtStartPar
\$R\_\{log,t\} = \textbackslash{}log(1 + R\_\{simple,t\}) = \textbackslash{}log\textbackslash{}left(1 +  \textbackslash{}frac\{P\_t + D\_t\}\{P\_\{t\sphinxhyphen{}1\}\} \sphinxhyphen{} 1 \textbackslash{}right) = \textbackslash{}log\textbackslash{}left(\textbackslash{}frac\{P\_t + D\_t\}\{P\_\{t\sphinxhyphen{}1\}\} \textbackslash{}right) = \textbackslash{}log(P\_t + D\_t) \sphinxhyphen{} \textbackslash{}log(P\_\{t\sphinxhyphen{}1\})\$

\sphinxAtStartPar
Therefore, we calculate log returns as either the log of one plus simple returns or the difference of the logs of the adjusted close column.
Log returns are also known as \sphinxstyleemphasis{continuously\sphinxhyphen{}compounded} returns.

\sphinxAtStartPar
We will typically use \sphinxstyleemphasis{simple} returns instead of \sphinxstyleemphasis{log} returns.
However, this section explains the differences between simple and log returns and where each is appropriate.


\subsection{Simple and Log Returns are Similar for Small Returns}
\label{\detokenize{herron_01_lecture:simple-and-log-returns-are-similar-for-small-returns}}
\sphinxAtStartPar
\$\textbackslash{}log(1 + x) \textbackslash{}approx x\$ for small values of \$x\$, so simple returns and log returns are similar for small returns.
Returns are typically small at daily and monthly horizons, so the difference between simple and log returns is small at these horizons.
The following figure shows \$R\_\{simple,t\} \textbackslash{}approx R\_\{log,t\}\$ for small \$R\$s.

\begin{sphinxuseclass}{cell}\begin{sphinxVerbatimInput}

\begin{sphinxuseclass}{cell_input}
\begin{sphinxVerbatim}[commandchars=\\\{\}]
\PYG{n}{R} \PYG{o}{=} \PYG{n}{np}\PYG{o}{.}\PYG{n}{linspace}\PYG{p}{(}\PYG{o}{\PYGZhy{}}\PYG{l+m+mf}{0.75}\PYG{p}{,} \PYG{l+m+mf}{0.75}\PYG{p}{,} \PYG{l+m+mi}{100}\PYG{p}{)}
\PYG{n}{logR} \PYG{o}{=} \PYG{n}{np}\PYG{o}{.}\PYG{n}{log}\PYG{p}{(}\PYG{l+m+mi}{1} \PYG{o}{+} \PYG{n}{R}\PYG{p}{)}
\end{sphinxVerbatim}

\end{sphinxuseclass}\end{sphinxVerbatimInput}

\end{sphinxuseclass}
\begin{sphinxuseclass}{cell}\begin{sphinxVerbatimInput}

\begin{sphinxuseclass}{cell_input}
\begin{sphinxVerbatim}[commandchars=\\\{\}]
\PYG{n}{plt}\PYG{o}{.}\PYG{n}{plot}\PYG{p}{(}\PYG{n}{R}\PYG{p}{,} \PYG{n}{logR}\PYG{p}{)}
\PYG{n}{plt}\PYG{o}{.}\PYG{n}{plot}\PYG{p}{(}\PYG{p}{[}\PYG{o}{\PYGZhy{}}\PYG{l+m+mi}{1}\PYG{p}{,} \PYG{l+m+mi}{1}\PYG{p}{]}\PYG{p}{,} \PYG{p}{[}\PYG{o}{\PYGZhy{}}\PYG{l+m+mi}{1}\PYG{p}{,} \PYG{l+m+mi}{1}\PYG{p}{]}\PYG{p}{)}
\PYG{n}{plt}\PYG{o}{.}\PYG{n}{xlabel}\PYG{p}{(}\PYG{l+s+s1}{\PYGZsq{}}\PYG{l+s+s1}{Simple Return}\PYG{l+s+s1}{\PYGZsq{}}\PYG{p}{)}
\PYG{n}{plt}\PYG{o}{.}\PYG{n}{ylabel}\PYG{p}{(}\PYG{l+s+s1}{\PYGZsq{}}\PYG{l+s+s1}{Log Return}\PYG{l+s+s1}{\PYGZsq{}}\PYG{p}{)}
\PYG{n}{plt}\PYG{o}{.}\PYG{n}{title}\PYG{p}{(}\PYG{l+s+s1}{\PYGZsq{}}\PYG{l+s+s1}{Log Versus Simple Returns}\PYG{l+s+s1}{\PYGZsq{}}\PYG{p}{)}
\PYG{n}{plt}\PYG{o}{.}\PYG{n}{legend}\PYG{p}{(}\PYG{p}{[}\PYG{l+s+s1}{\PYGZsq{}}\PYG{l+s+s1}{Actual}\PYG{l+s+s1}{\PYGZsq{}}\PYG{p}{,} \PYG{l+s+s1}{\PYGZsq{}}\PYG{l+s+s1}{If Log = Simple}\PYG{l+s+s1}{\PYGZsq{}}\PYG{p}{]}\PYG{p}{)}
\PYG{n}{plt}\PYG{o}{.}\PYG{n}{show}\PYG{p}{(}\PYG{p}{)}
\end{sphinxVerbatim}

\end{sphinxuseclass}\end{sphinxVerbatimInput}
\begin{sphinxVerbatimOutput}

\begin{sphinxuseclass}{cell_output}
\noindent\sphinxincludegraphics{{be30c7d407a1dffdd992d58d2b7f0cc7045264ebe1c4c89f5e2750194df26c27}.png}

\end{sphinxuseclass}\end{sphinxVerbatimOutput}

\end{sphinxuseclass}

\subsection{Simple Return Advantage: Portfolio Calculations}
\label{\detokenize{herron_01_lecture:simple-return-advantage-portfolio-calculations}}
\sphinxAtStartPar
We can only perform portfolio calculations with simple returns.
For a portfolio of \$N\$ assets with portfolio weights \$w\_i\$, the portfolio return \$R\_\{p\}\$ is the weighted average of the returns of its assets, \$R\_\{p\} = \textbackslash{}sum\_\{i=1\}\textasciicircum{}N w\_i R\_\{i\}\$.
For two stocks with portfolio weights of 50\%, our portfolio return is \$R\_\{portfolio\} = 0.5 R\_1 + 0.5 R\_2 = \textbackslash{}frac\{R\_1 + R\_2\}\{2\}\$.
However, we cannot calculate portfolio returns with log returns because the sum of logs is the log of products.

\sphinxAtStartPar
\sphinxstyleemphasis{\sphinxstylestrong{We cannot calculate portfolio returns as the weighted average of log returns.}}


\subsection{Log Return Advantage: Log Returns are Additive}
\label{\detokenize{herron_01_lecture:log-return-advantage-log-returns-are-additive}}
\sphinxAtStartPar
The advantage of log returns is that we can compound log returns with addition.
The additive property of log returns makes code simple, computations fast, and proofs easy when we compound returns over multiple periods.

\sphinxAtStartPar
We compound returns from \$t=0\$ to \$t=T\$ as follows:

\sphinxAtStartPar
\$1 + R\_\{0, T\} = (1 + R\_1) \textbackslash{}times (1 + R\_2) \textbackslash{}times \textbackslash{}dots \textbackslash{}times (1 + R\_T)\$

\sphinxAtStartPar
Next, we take the log of both sides of the previous equation and use the property that the log of products is the sum of logs:

\sphinxAtStartPar
\$\textbackslash{}log(1 + R\_\{0, T\}) = \textbackslash{}log((1 + R\_1) \textbackslash{}times (1 + R\_2) \textbackslash{}times \textbackslash{}dots \textbackslash{}times (1 + R\_T)) = \textbackslash{}log(1 + R\_1) + \textbackslash{}log(1 + R\_2) + \textbackslash{}dots + \textbackslash{}log(1 + R\_T) = \textbackslash{}sum\_\{t=1\}\textasciicircum{}T \textbackslash{}log(1 + R\_t)\$

\sphinxAtStartPar
Next, we exponentiate both sides of the previous equation:

\sphinxAtStartPar
\$e\textasciicircum{}\{\textbackslash{}log(1 + R\_\{0, T\})\} = e\textasciicircum{}\{\textbackslash{}sum\_\{t=0\}\textasciicircum{}T \textbackslash{}log(1 + R\_t)\}\$

\sphinxAtStartPar
Next, we use the property that \$e\textasciicircum{}\{\textbackslash{}log(x)\} = x\$ to simplify the previous equation:

\sphinxAtStartPar
\$1 + R\_\{0,T\} = e\textasciicircum{}\{\textbackslash{}sum\_\{t=0\}\textasciicircum{}T \textbackslash{}log(1 + R\_t)\}\$

\sphinxAtStartPar
Finally, we subtract 1 from both sides:

\sphinxAtStartPar
\$R\_\{0 ,T\} = e\textasciicircum{}\{\textbackslash{}sum\_\{t=0\}\textasciicircum{}T \textbackslash{}log(1 + R\_t)\} \sphinxhyphen{} 1\$

\sphinxAtStartPar
So, the return \$R\_\{0,T\}\$ from \$t=0\$ to \$t=T\$ is the exponentiated sum of log returns.
The pandas developers assume users understand the math above and focus on optimizing sums.

\sphinxAtStartPar
The following code generates 10,000 random log returns.
The \sphinxcode{\sphinxupquote{np.random.randn()}} call generates normally distributed random numbers.
To generate equivalent simple returns, we exponentiate these log returns, then subtract one.

\begin{sphinxuseclass}{cell}\begin{sphinxVerbatimInput}

\begin{sphinxuseclass}{cell_input}
\begin{sphinxVerbatim}[commandchars=\\\{\}]
\PYG{n}{np}\PYG{o}{.}\PYG{n}{random}\PYG{o}{.}\PYG{n}{seed}\PYG{p}{(}\PYG{l+m+mi}{42}\PYG{p}{)}
\PYG{n}{df} \PYG{o}{=} \PYG{n}{pd}\PYG{o}{.}\PYG{n}{DataFrame}\PYG{p}{(}\PYG{n}{data}\PYG{o}{=}\PYG{p}{\PYGZob{}}\PYG{l+s+s1}{\PYGZsq{}}\PYG{l+s+s1}{R}\PYG{l+s+s1}{\PYGZsq{}}\PYG{p}{:} \PYG{n}{np}\PYG{o}{.}\PYG{n}{exp}\PYG{p}{(}\PYG{n}{np}\PYG{o}{.}\PYG{n}{random}\PYG{o}{.}\PYG{n}{randn}\PYG{p}{(}\PYG{l+m+mi}{10000}\PYG{p}{)}\PYG{p}{)} \PYG{o}{\PYGZhy{}} \PYG{l+m+mi}{1}\PYG{p}{\PYGZcb{}}\PYG{p}{)}
\end{sphinxVerbatim}

\end{sphinxuseclass}\end{sphinxVerbatimInput}

\end{sphinxuseclass}
\begin{sphinxuseclass}{cell}\begin{sphinxVerbatimInput}

\begin{sphinxuseclass}{cell_input}
\begin{sphinxVerbatim}[commandchars=\\\{\}]
\PYG{n}{df}\PYG{o}{.}\PYG{n}{describe}\PYG{p}{(}\PYG{p}{)}
\end{sphinxVerbatim}

\end{sphinxuseclass}\end{sphinxVerbatimInput}
\begin{sphinxVerbatimOutput}

\begin{sphinxuseclass}{cell_output}
\begin{sphinxVerbatim}[commandchars=\\\{\}]
               R
count 10000.0000
mean      0.6529
std       2.1918
min      \PYGZhy{}0.9802
25\PYGZpc{}      \PYGZhy{}0.4896
50\PYGZpc{}      \PYGZhy{}0.0026
75\PYGZpc{}       0.9564
max      49.7158
\end{sphinxVerbatim}

\end{sphinxuseclass}\end{sphinxVerbatimOutput}

\end{sphinxuseclass}
\sphinxAtStartPar
We can time the calculation of 12\sphinxhyphen{}observation rolling returns.
We use \sphinxcode{\sphinxupquote{.apply()}} for the simple return version because \sphinxcode{\sphinxupquote{.rolling()}} does not have a product method.
We find that \sphinxcode{\sphinxupquote{.rolling()}} is slower with \sphinxcode{\sphinxupquote{.apply()}} than with \sphinxcode{\sphinxupquote{.sum()}} by a factor of 2,000.
\sphinxstyleemphasis{\sphinxstylestrong{We will learn about \sphinxcode{\sphinxupquote{.rolling()}} and \sphinxcode{\sphinxupquote{.apply()}} in a few weeks, but they provide the best example of when to use log returns.}}

\begin{sphinxuseclass}{cell}\begin{sphinxVerbatimInput}

\begin{sphinxuseclass}{cell_input}
\begin{sphinxVerbatim}[commandchars=\\\{\}]
\PYG{o}{\PYGZpc{}\PYGZpc{}time}it
\PYG{n}{df}\PYG{p}{[}\PYG{l+s+s1}{\PYGZsq{}}\PYG{l+s+s1}{R12\PYGZus{}via\PYGZus{}simple}\PYG{l+s+s1}{\PYGZsq{}}\PYG{p}{]} \PYG{o}{=} \PYG{p}{(}
    \PYG{n}{df}\PYG{p}{[}\PYG{l+s+s1}{\PYGZsq{}}\PYG{l+s+s1}{R}\PYG{l+s+s1}{\PYGZsq{}}\PYG{p}{]}
    \PYG{o}{.}\PYG{n}{add}\PYG{p}{(}\PYG{l+m+mi}{1}\PYG{p}{)}
    \PYG{o}{.}\PYG{n}{rolling}\PYG{p}{(}\PYG{l+m+mi}{12}\PYG{p}{)}
    \PYG{o}{.}\PYG{n}{apply}\PYG{p}{(}\PYG{k}{lambda} \PYG{n}{x}\PYG{p}{:} \PYG{n}{x}\PYG{o}{.}\PYG{n}{prod}\PYG{p}{(}\PYG{p}{)}\PYG{p}{)}
    \PYG{o}{.}\PYG{n}{sub}\PYG{p}{(}\PYG{l+m+mi}{1}\PYG{p}{)}
\PYG{p}{)}
\end{sphinxVerbatim}

\end{sphinxuseclass}\end{sphinxVerbatimInput}
\begin{sphinxVerbatimOutput}

\begin{sphinxuseclass}{cell_output}
\begin{sphinxVerbatim}[commandchars=\\\{\}]
644 ms ± 8.23 ms per loop (mean ± std. dev. of 7 runs, 1 loop each)
\end{sphinxVerbatim}

\end{sphinxuseclass}\end{sphinxVerbatimOutput}

\end{sphinxuseclass}
\begin{sphinxuseclass}{cell}\begin{sphinxVerbatimInput}

\begin{sphinxuseclass}{cell_input}
\begin{sphinxVerbatim}[commandchars=\\\{\}]
\PYG{o}{\PYGZpc{}\PYGZpc{}time}it
\PYG{n}{df}\PYG{p}{[}\PYG{l+s+s1}{\PYGZsq{}}\PYG{l+s+s1}{R12\PYGZus{}via\PYGZus{}log}\PYG{l+s+s1}{\PYGZsq{}}\PYG{p}{]} \PYG{o}{=} \PYG{p}{(}
    \PYG{n}{df}\PYG{p}{[}\PYG{l+s+s1}{\PYGZsq{}}\PYG{l+s+s1}{R}\PYG{l+s+s1}{\PYGZsq{}}\PYG{p}{]}
    \PYG{o}{.}\PYG{n}{add}\PYG{p}{(}\PYG{l+m+mi}{1}\PYG{p}{)}
    \PYG{o}{.}\PYG{n}{pipe}\PYG{p}{(}\PYG{n}{np}\PYG{o}{.}\PYG{n}{log}\PYG{p}{)}
    \PYG{o}{.}\PYG{n}{rolling}\PYG{p}{(}\PYG{l+m+mi}{12}\PYG{p}{)}
    \PYG{o}{.}\PYG{n}{sum}\PYG{p}{(}\PYG{p}{)}
    \PYG{o}{.}\PYG{n}{pipe}\PYG{p}{(}\PYG{n}{np}\PYG{o}{.}\PYG{n}{exp}\PYG{p}{)}
    \PYG{o}{.}\PYG{n}{sub}\PYG{p}{(}\PYG{l+m+mi}{1}\PYG{p}{)}
\PYG{p}{)}
\end{sphinxVerbatim}

\end{sphinxuseclass}\end{sphinxVerbatimInput}
\begin{sphinxVerbatimOutput}

\begin{sphinxuseclass}{cell_output}
\begin{sphinxVerbatim}[commandchars=\\\{\}]
878 µs ± 16.8 µs per loop (mean ± std. dev. of 7 runs, 1000 loops each)
\end{sphinxVerbatim}

\end{sphinxuseclass}\end{sphinxVerbatimOutput}

\end{sphinxuseclass}
\begin{sphinxuseclass}{cell}\begin{sphinxVerbatimInput}

\begin{sphinxuseclass}{cell_input}
\begin{sphinxVerbatim}[commandchars=\\\{\}]
\PYG{n}{np}\PYG{o}{.}\PYG{n}{allclose}\PYG{p}{(}\PYG{n}{df}\PYG{p}{[}\PYG{l+s+s1}{\PYGZsq{}}\PYG{l+s+s1}{R12\PYGZus{}via\PYGZus{}simple}\PYG{l+s+s1}{\PYGZsq{}}\PYG{p}{]}\PYG{p}{,} \PYG{n}{df}\PYG{p}{[}\PYG{l+s+s1}{\PYGZsq{}}\PYG{l+s+s1}{R12\PYGZus{}via\PYGZus{}log}\PYG{l+s+s1}{\PYGZsq{}}\PYG{p}{]}\PYG{p}{,} \PYG{n}{equal\PYGZus{}nan}\PYG{o}{=}\PYG{k+kc}{True}\PYG{p}{)}
\end{sphinxVerbatim}

\end{sphinxuseclass}\end{sphinxVerbatimInput}
\begin{sphinxVerbatimOutput}

\begin{sphinxuseclass}{cell_output}
\begin{sphinxVerbatim}[commandchars=\\\{\}]
True
\end{sphinxVerbatim}

\end{sphinxuseclass}\end{sphinxVerbatimOutput}

\end{sphinxuseclass}
\sphinxAtStartPar
These two approaches calculate the same return, but the simple\sphinxhyphen{}return approach is 1,000 times slower than the log\sphinxhyphen{}return approach!

\sphinxAtStartPar
\sphinxstyleemphasis{\sphinxstylestrong{We can use log returns to calculate total or holding period returns very quickly!}}


\section{Portfolio Math}
\label{\detokenize{herron_01_lecture:portfolio-math}}
\sphinxAtStartPar
Portfolio return \$R\_\{p\}\$ is the weighted average of its asset returns, so \$R\_\{p\} = \textbackslash{}sum\_\{i=1\}\textasciicircum{}N w\_i R\_\{i\}\$.
Here \$N\$ is the number of assets, and \$w\_i\$ is the weight on asset \$i\$.


\subsection{The 1/N Portfolio}
\label{\detokenize{herron_01_lecture:the-1-n-portfolio}}
\sphinxAtStartPar
The \$\textbackslash{}frac\{1\}\{N\}\$ portfolio equally weights portfolio assets, so \$w\_1 = w\_2 = \textbackslash{}dots = w\_N = \textbackslash{}frac\{1\}\{N\}\$.
We typically rebalance the \$\textbackslash{}frac\{1\}\{N\}\$ portfolio every period.
If \$w\_i = \textbackslash{}frac\{1\}\{N\}\$, then \$R\_\{p\} = \textbackslash{}sum\_\{i=1\}\textasciicircum{}N \textbackslash{}frac\{1\}\{N\} R\_\{i\} = \textbackslash{}frac\{\textbackslash{}sum\_\{i=1\}\textasciicircum{}N R\_i\}\{N\} = \textbackslash{}bar\{R\}\$.
Therefore, we can use \sphinxcode{\sphinxupquote{.mean()}} to calculate \$\textbackslash{}frac\{1\}\{N\}\$ portfolio returns.

\begin{sphinxuseclass}{cell}\begin{sphinxVerbatimInput}

\begin{sphinxuseclass}{cell_input}
\begin{sphinxVerbatim}[commandchars=\\\{\}]
\PYG{n}{returns} \PYG{o}{=} \PYG{n}{histories}\PYG{p}{[}\PYG{l+s+s1}{\PYGZsq{}}\PYG{l+s+s1}{Adj Close}\PYG{l+s+s1}{\PYGZsq{}}\PYG{p}{]}\PYG{o}{.}\PYG{n}{pct\PYGZus{}change}\PYG{p}{(}\PYG{p}{)}\PYG{o}{.}\PYG{n}{loc}\PYG{p}{[}\PYG{l+s+s1}{\PYGZsq{}}\PYG{l+s+s1}{2022}\PYG{l+s+s1}{\PYGZsq{}}\PYG{p}{]}
\end{sphinxVerbatim}

\end{sphinxuseclass}\end{sphinxVerbatimInput}

\end{sphinxuseclass}
\begin{sphinxuseclass}{cell}\begin{sphinxVerbatimInput}

\begin{sphinxuseclass}{cell_input}
\begin{sphinxVerbatim}[commandchars=\\\{\}]
\PYG{n}{returns}
\end{sphinxVerbatim}

\end{sphinxuseclass}\end{sphinxVerbatimInput}
\begin{sphinxVerbatimOutput}

\begin{sphinxuseclass}{cell_output}
\begin{sphinxVerbatim}[commandchars=\\\{\}]
              AAPL    AMZN    GOOG    MSFT    NVDA    TSLA
Date                                                      
2022\PYGZhy{}01\PYGZhy{}03  0.0250  0.0221  0.0027 \PYGZhy{}0.0047  0.0241  0.1353
2022\PYGZhy{}01\PYGZhy{}04 \PYGZhy{}0.0127 \PYGZhy{}0.0169 \PYGZhy{}0.0045 \PYGZhy{}0.0171 \PYGZhy{}0.0276 \PYGZhy{}0.0418
2022\PYGZhy{}01\PYGZhy{}05 \PYGZhy{}0.0266 \PYGZhy{}0.0189 \PYGZhy{}0.0468 \PYGZhy{}0.0384 \PYGZhy{}0.0576 \PYGZhy{}0.0535
2022\PYGZhy{}01\PYGZhy{}06 \PYGZhy{}0.0167 \PYGZhy{}0.0067 \PYGZhy{}0.0007 \PYGZhy{}0.0079  0.0208 \PYGZhy{}0.0215
2022\PYGZhy{}01\PYGZhy{}07  0.0010 \PYGZhy{}0.0043 \PYGZhy{}0.0040  0.0005 \PYGZhy{}0.0330 \PYGZhy{}0.0354
...            ...     ...     ...     ...     ...     ...
2022\PYGZhy{}12\PYGZhy{}23 \PYGZhy{}0.0028  0.0174  0.0176  0.0023 \PYGZhy{}0.0087 \PYGZhy{}0.0176
2022\PYGZhy{}12\PYGZhy{}27 \PYGZhy{}0.0139 \PYGZhy{}0.0259 \PYGZhy{}0.0209 \PYGZhy{}0.0074 \PYGZhy{}0.0714 \PYGZhy{}0.1141
2022\PYGZhy{}12\PYGZhy{}28 \PYGZhy{}0.0307 \PYGZhy{}0.0147 \PYGZhy{}0.0167 \PYGZhy{}0.0103 \PYGZhy{}0.0060  0.0331
2022\PYGZhy{}12\PYGZhy{}29  0.0283  0.0288  0.0288  0.0276  0.0404  0.0808
2022\PYGZhy{}12\PYGZhy{}30  0.0025 \PYGZhy{}0.0021 \PYGZhy{}0.0025 \PYGZhy{}0.0049  0.0008  0.0112

[251 rows x 6 columns]
\end{sphinxVerbatim}

\end{sphinxuseclass}\end{sphinxVerbatimOutput}

\end{sphinxuseclass}
\begin{sphinxuseclass}{cell}\begin{sphinxVerbatimInput}

\begin{sphinxuseclass}{cell_input}
\begin{sphinxVerbatim}[commandchars=\\\{\}]
\PYG{n}{returns}\PYG{o}{.}\PYG{n}{mean}\PYG{p}{(}\PYG{p}{)}
\end{sphinxVerbatim}

\end{sphinxuseclass}\end{sphinxVerbatimInput}
\begin{sphinxVerbatimOutput}

\begin{sphinxuseclass}{cell_output}
\begin{sphinxVerbatim}[commandchars=\\\{\}]
AAPL   \PYGZhy{}0.0010
AMZN   \PYGZhy{}0.0022
GOOG   \PYGZhy{}0.0016
MSFT   \PYGZhy{}0.0011
NVDA   \PYGZhy{}0.0020
TSLA   \PYGZhy{}0.0033
dtype: float64
\end{sphinxVerbatim}

\end{sphinxuseclass}\end{sphinxVerbatimOutput}

\end{sphinxuseclass}
\begin{sphinxuseclass}{cell}\begin{sphinxVerbatimInput}

\begin{sphinxuseclass}{cell_input}
\begin{sphinxVerbatim}[commandchars=\\\{\}]
\PYG{n}{rp\PYGZus{}1} \PYG{o}{=} \PYG{n}{returns}\PYG{o}{.}\PYG{n}{mean}\PYG{p}{(}\PYG{n}{axis}\PYG{o}{=}\PYG{l+m+mi}{1}\PYG{p}{)}
\PYG{n}{rp\PYGZus{}1}
\end{sphinxVerbatim}

\end{sphinxuseclass}\end{sphinxVerbatimInput}
\begin{sphinxVerbatimOutput}

\begin{sphinxuseclass}{cell_output}
\begin{sphinxVerbatim}[commandchars=\\\{\}]
Date
2022\PYGZhy{}01\PYGZhy{}03    0.0341
2022\PYGZhy{}01\PYGZhy{}04   \PYGZhy{}0.0201
2022\PYGZhy{}01\PYGZhy{}05   \PYGZhy{}0.0403
2022\PYGZhy{}01\PYGZhy{}06   \PYGZhy{}0.0055
2022\PYGZhy{}01\PYGZhy{}07   \PYGZhy{}0.0125
               ...  
2022\PYGZhy{}12\PYGZhy{}23    0.0014
2022\PYGZhy{}12\PYGZhy{}27   \PYGZhy{}0.0423
2022\PYGZhy{}12\PYGZhy{}28   \PYGZhy{}0.0075
2022\PYGZhy{}12\PYGZhy{}29    0.0391
2022\PYGZhy{}12\PYGZhy{}30    0.0008
Length: 251, dtype: float64
\end{sphinxVerbatim}

\end{sphinxuseclass}\end{sphinxVerbatimOutput}

\end{sphinxuseclass}
\sphinxAtStartPar
\sphinxstyleemphasis{\sphinxstylestrong{Note that when we apply the same portfolio weights every period, we rebalance at the same frequency as the returns data.}}
If we have daily data, rebalance daily.
If we have monthly data, we rebalance monthly, and so on.


\subsection{A More General Solution}
\label{\detokenize{herron_01_lecture:a-more-general-solution}}
\sphinxAtStartPar
If we combine weights into vector \$w\$ and the time series of asset returns into matrix \$\textbackslash{}bf\{R\}\$, then we can calculate the time series of portfolio returns as \$R\_p = w\textasciicircum{}T \textbackslash{}bf\{R\}\$.
The pandas version of this calculation is \sphinxcode{\sphinxupquote{R.dot(w)}}, where \sphinxcode{\sphinxupquote{R}} is a data frame of asset returns and \sphinxcode{\sphinxupquote{w}} is a series of portfolio weights.
We can use this approach to calculate \$\textbackslash{}frac\{1\}\{N\}\$ portfolio returns, too.

\begin{sphinxuseclass}{cell}\begin{sphinxVerbatimInput}

\begin{sphinxuseclass}{cell_input}
\begin{sphinxVerbatim}[commandchars=\\\{\}]
\PYG{n}{weights} \PYG{o}{=} \PYG{n}{np}\PYG{o}{.}\PYG{n}{ones}\PYG{p}{(}\PYG{n}{returns}\PYG{o}{.}\PYG{n}{shape}\PYG{p}{[}\PYG{l+m+mi}{1}\PYG{p}{]}\PYG{p}{)} \PYG{o}{/} \PYG{n}{returns}\PYG{o}{.}\PYG{n}{shape}\PYG{p}{[}\PYG{l+m+mi}{1}\PYG{p}{]}
\PYG{n}{weights}
\end{sphinxVerbatim}

\end{sphinxuseclass}\end{sphinxVerbatimInput}
\begin{sphinxVerbatimOutput}

\begin{sphinxuseclass}{cell_output}
\begin{sphinxVerbatim}[commandchars=\\\{\}]
array([0.1667, 0.1667, 0.1667, 0.1667, 0.1667, 0.1667])
\end{sphinxVerbatim}

\end{sphinxuseclass}\end{sphinxVerbatimOutput}

\end{sphinxuseclass}
\begin{sphinxuseclass}{cell}\begin{sphinxVerbatimInput}

\begin{sphinxuseclass}{cell_input}
\begin{sphinxVerbatim}[commandchars=\\\{\}]
\PYG{n}{rp\PYGZus{}2} \PYG{o}{=} \PYG{n}{returns}\PYG{o}{.}\PYG{n}{dot}\PYG{p}{(}\PYG{n}{weights}\PYG{p}{)}
\PYG{n}{rp\PYGZus{}2}
\end{sphinxVerbatim}

\end{sphinxuseclass}\end{sphinxVerbatimInput}
\begin{sphinxVerbatimOutput}

\begin{sphinxuseclass}{cell_output}
\begin{sphinxVerbatim}[commandchars=\\\{\}]
Date
2022\PYGZhy{}01\PYGZhy{}03    0.0341
2022\PYGZhy{}01\PYGZhy{}04   \PYGZhy{}0.0201
2022\PYGZhy{}01\PYGZhy{}05   \PYGZhy{}0.0403
2022\PYGZhy{}01\PYGZhy{}06   \PYGZhy{}0.0055
2022\PYGZhy{}01\PYGZhy{}07   \PYGZhy{}0.0125
               ...  
2022\PYGZhy{}12\PYGZhy{}23    0.0014
2022\PYGZhy{}12\PYGZhy{}27   \PYGZhy{}0.0423
2022\PYGZhy{}12\PYGZhy{}28   \PYGZhy{}0.0075
2022\PYGZhy{}12\PYGZhy{}29    0.0391
2022\PYGZhy{}12\PYGZhy{}30    0.0008
Length: 251, dtype: float64
\end{sphinxVerbatim}

\end{sphinxuseclass}\end{sphinxVerbatimOutput}

\end{sphinxuseclass}
\sphinxAtStartPar
Both approaches give the same answer!

\begin{sphinxuseclass}{cell}\begin{sphinxVerbatimInput}

\begin{sphinxuseclass}{cell_input}
\begin{sphinxVerbatim}[commandchars=\\\{\}]
\PYG{n}{np}\PYG{o}{.}\PYG{n}{allclose}\PYG{p}{(}\PYG{n}{rp\PYGZus{}1}\PYG{p}{,} \PYG{n}{rp\PYGZus{}2}\PYG{p}{,} \PYG{n}{equal\PYGZus{}nan}\PYG{o}{=}\PYG{k+kc}{True}\PYG{p}{)}
\end{sphinxVerbatim}

\end{sphinxuseclass}\end{sphinxVerbatimInput}
\begin{sphinxVerbatimOutput}

\begin{sphinxuseclass}{cell_output}
\begin{sphinxVerbatim}[commandchars=\\\{\}]
True
\end{sphinxVerbatim}

\end{sphinxuseclass}\end{sphinxVerbatimOutput}

\end{sphinxuseclass}
\sphinxstepscope


\section{Herron Topic 1 \sphinxhyphen{} Practice (Blank)}
\label{\detokenize{herron_01_practice:herron-topic-1-practice-blank}}\label{\detokenize{herron_01_practice::doc}}

\subsection{Announcements}
\label{\detokenize{herron_01_practice:announcements}}

\subsection{Practice}
\label{\detokenize{herron_01_practice:practice}}
\begin{sphinxuseclass}{cell}\begin{sphinxVerbatimInput}

\begin{sphinxuseclass}{cell_input}
\begin{sphinxVerbatim}[commandchars=\\\{\}]
\PYG{k+kn}{import} \PYG{n+nn}{matplotlib}\PYG{n+nn}{.}\PYG{n+nn}{pyplot} \PYG{k}{as} \PYG{n+nn}{plt}
\PYG{k+kn}{import} \PYG{n+nn}{numpy} \PYG{k}{as} \PYG{n+nn}{np}
\PYG{k+kn}{import} \PYG{n+nn}{pandas} \PYG{k}{as} \PYG{n+nn}{pd}

\PYG{k+kn}{import} \PYG{n+nn}{yfinance} \PYG{k}{as} \PYG{n+nn}{yf}
\PYG{k+kn}{import} \PYG{n+nn}{pandas\PYGZus{}datareader} \PYG{k}{as} \PYG{n+nn}{pdr}
\PYG{k+kn}{import} \PYG{n+nn}{requests\PYGZus{}cache}
\end{sphinxVerbatim}

\end{sphinxuseclass}\end{sphinxVerbatimInput}

\end{sphinxuseclass}
\begin{sphinxuseclass}{cell}\begin{sphinxVerbatimInput}

\begin{sphinxuseclass}{cell_input}
\begin{sphinxVerbatim}[commandchars=\\\{\}]
\PYG{o}{\PYGZpc{}}\PYG{k}{config} InlineBackend.figure\PYGZus{}format = \PYGZsq{}retina\PYGZsq{}
\PYG{o}{\PYGZpc{}}\PYG{k}{precision} 4
\PYG{n}{pd}\PYG{o}{.}\PYG{n}{options}\PYG{o}{.}\PYG{n}{display}\PYG{o}{.}\PYG{n}{float\PYGZus{}format} \PYG{o}{=} \PYG{l+s+s1}{\PYGZsq{}}\PYG{l+s+si}{\PYGZob{}:.4f\PYGZcb{}}\PYG{l+s+s1}{\PYGZsq{}}\PYG{o}{.}\PYG{n}{format}
\PYG{n}{session} \PYG{o}{=} \PYG{n}{requests\PYGZus{}cache}\PYG{o}{.}\PYG{n}{CachedSession}\PYG{p}{(}\PYG{n}{expire\PYGZus{}after}\PYG{o}{=}\PYG{l+m+mi}{1}\PYG{p}{)}
\end{sphinxVerbatim}

\end{sphinxuseclass}\end{sphinxVerbatimInput}

\end{sphinxuseclass}

\subsubsection{Download all available daily price data for tickers TSLA, F, AAPL, AMZN, and META to data frame \sphinxstyleliteralintitle{\sphinxupquote{histories}}}
\label{\detokenize{herron_01_practice:download-all-available-daily-price-data-for-tickers-tsla-f-aapl-amzn-and-meta-to-data-frame-histories}}
\sphinxAtStartPar
Remove time zone information from the index and use \sphinxcode{\sphinxupquote{histories.columns.names}} to label the variables and tickers as \sphinxcode{\sphinxupquote{Variable}} and \sphinxcode{\sphinxupquote{Ticker}}.


\subsubsection{Calculate all available daily returns and save to data frame \sphinxstyleliteralintitle{\sphinxupquote{returns}}}
\label{\detokenize{herron_01_practice:calculate-all-available-daily-returns-and-save-to-data-frame-returns}}

\subsubsection{Slices returns for the 2020s and assign to \sphinxstyleliteralintitle{\sphinxupquote{returns\_2020s}}}
\label{\detokenize{herron_01_practice:slices-returns-for-the-2020s-and-assign-to-returns-2020s}}

\subsubsection{Download all available data for the Fama and French daily benchmark factors to dictionary \sphinxstyleliteralintitle{\sphinxupquote{ff\_all}}}
\label{\detokenize{herron_01_practice:download-all-available-data-for-the-fama-and-french-daily-benchmark-factors-to-dictionary-ff-all}}
\sphinxAtStartPar
I often use the following code snippet to find the exact name for the the daily benchmark factors file.

\begin{sphinxuseclass}{cell}\begin{sphinxVerbatimInput}

\begin{sphinxuseclass}{cell_input}
\begin{sphinxVerbatim}[commandchars=\\\{\}]
\PYG{n}{pdr}\PYG{o}{.}\PYG{n}{famafrench}\PYG{o}{.}\PYG{n}{get\PYGZus{}available\PYGZus{}datasets}\PYG{p}{(}\PYG{p}{)}\PYG{p}{[}\PYG{p}{:}\PYG{l+m+mi}{5}\PYG{p}{]}
\end{sphinxVerbatim}

\end{sphinxuseclass}\end{sphinxVerbatimInput}
\begin{sphinxVerbatimOutput}

\begin{sphinxuseclass}{cell_output}
\begin{sphinxVerbatim}[commandchars=\\\{\}]
[\PYGZsq{}F\PYGZhy{}F\PYGZus{}Research\PYGZus{}Data\PYGZus{}Factors\PYGZsq{},
 \PYGZsq{}F\PYGZhy{}F\PYGZus{}Research\PYGZus{}Data\PYGZus{}Factors\PYGZus{}weekly\PYGZsq{},
 \PYGZsq{}F\PYGZhy{}F\PYGZus{}Research\PYGZus{}Data\PYGZus{}Factors\PYGZus{}daily\PYGZsq{},
 \PYGZsq{}F\PYGZhy{}F\PYGZus{}Research\PYGZus{}Data\PYGZus{}5\PYGZus{}Factors\PYGZus{}2x3\PYGZsq{},
 \PYGZsq{}F\PYGZhy{}F\PYGZus{}Research\PYGZus{}Data\PYGZus{}5\PYGZus{}Factors\PYGZus{}2x3\PYGZus{}daily\PYGZsq{}]
\end{sphinxVerbatim}

\end{sphinxuseclass}\end{sphinxVerbatimOutput}

\end{sphinxuseclass}

\subsubsection{Slice the daily benchmark factors, convert them to decimal returns, and assign to \sphinxstyleliteralintitle{\sphinxupquote{ff}}}
\label{\detokenize{herron_01_practice:slice-the-daily-benchmark-factors-convert-them-to-decimal-returns-and-assign-to-ff}}

\subsubsection{Use the \sphinxstyleliteralintitle{\sphinxupquote{.cumprod()}} method to plot cumulative returns for these stocks in the 2020s}
\label{\detokenize{herron_01_practice:use-the-cumprod-method-to-plot-cumulative-returns-for-these-stocks-in-the-2020s}}

\subsubsection{Use the \sphinxstyleliteralintitle{\sphinxupquote{.cumsum()}} method with log returns to plot cumulative returns for these stocks in the 2020s}
\label{\detokenize{herron_01_practice:use-the-cumsum-method-with-log-returns-to-plot-cumulative-returns-for-these-stocks-in-the-2020s}}

\subsubsection{Use price data only to plot cumulative returns for these stocks in the 2020s}
\label{\detokenize{herron_01_practice:use-price-data-only-to-plot-cumulative-returns-for-these-stocks-in-the-2020s}}

\subsubsection{Calculate the Sharpe Ratio for TSLA}
\label{\detokenize{herron_01_practice:calculate-the-sharpe-ratio-for-tsla}}
\sphinxAtStartPar
Calculate the Sharpe Ratio with all available returns and 2020s returns.
Recall the Sharpe Ratio is \$\textbackslash{}frac\{\textbackslash{}overline\{R\_i \sphinxhyphen{} R\_f\}\}\{\textbackslash{}sigma\_i\}\$, where \$\textbackslash{}sigma\_i\$ is the volatility of \sphinxstyleemphasis{excess} returns.

\sphinxAtStartPar
\sphinxstyleemphasis{\sphinxstylestrong{I suggest you write a function named \sphinxcode{\sphinxupquote{sharpe()}} to use for the rest of this notebook.}}


\subsubsection{Calculate the market beta for TSLA}
\label{\detokenize{herron_01_practice:calculate-the-market-beta-for-tsla}}
\sphinxAtStartPar
Calculate the market beta with all available returns and 2020s returns.
Recall we estimate market beta with the ordinary least squares (OLS) regression \$R\_i\sphinxhyphen{}R\_f = \textbackslash{}alpha + \textbackslash{}beta (R\_m\sphinxhyphen{}R\_f) + \textbackslash{}epsilon\$.
We can estimate market beta with the covariance formula above for a univariate regression if we do not need goodness of fit statistics.

\sphinxAtStartPar
\sphinxstyleemphasis{\sphinxstylestrong{I suggest you write a function named \sphinxcode{\sphinxupquote{beta()}} to use for the rest of this notebook.}}


\subsubsection{Guess the Sharpe Ratios for these stocks in the 2020s}
\label{\detokenize{herron_01_practice:guess-the-sharpe-ratios-for-these-stocks-in-the-2020s}}

\subsubsection{Guess the market betas for these stocks in the 2020s}
\label{\detokenize{herron_01_practice:guess-the-market-betas-for-these-stocks-in-the-2020s}}

\subsubsection{Calculate the Sharpe Ratios for these stocks in the 2020s}
\label{\detokenize{herron_01_practice:calculate-the-sharpe-ratios-for-these-stocks-in-the-2020s}}
\sphinxAtStartPar
How good were your guesses?


\subsubsection{Calculate the market betas for these stocks in the 2020s}
\label{\detokenize{herron_01_practice:calculate-the-market-betas-for-these-stocks-in-the-2020s}}
\sphinxAtStartPar
How good were your guesses?


\subsubsection{Calculate the Sharpe Ratio for an \sphinxstyleemphasis{equally weighted} portfolio of these stocks in the 2020s}
\label{\detokenize{herron_01_practice:calculate-the-sharpe-ratio-for-an-equally-weighted-portfolio-of-these-stocks-in-the-2020s}}
\sphinxAtStartPar
What do you notice?


\subsubsection{Calculate the market beta for an \sphinxstyleemphasis{equally weighted} portfolio of these stocks in the 2020s}
\label{\detokenize{herron_01_practice:calculate-the-market-beta-for-an-equally-weighted-portfolio-of-these-stocks-in-the-2020s}}
\sphinxAtStartPar
What do you notice?


\subsubsection{Calculate the market betas for these stocks every calendar year for every possible year}
\label{\detokenize{herron_01_practice:calculate-the-market-betas-for-these-stocks-every-calendar-year-for-every-possible-year}}
\sphinxAtStartPar
Save these market betas to data frame \sphinxcode{\sphinxupquote{betas}}.
Our current Python knowledge limits us to a for\sphinxhyphen{}loop, but we will learn easier and faster approaches soon!


\subsubsection{Plot the time series of market betas}
\label{\detokenize{herron_01_practice:plot-the-time-series-of-market-betas}}
\sphinxstepscope


\section{Herron Topic 1 \sphinxhyphen{} Practice (Section 3, Monday 2:45 PM)}
\label{\detokenize{herron_01_practice_03:herron-topic-1-practice-section-3-monday-2-45-pm}}\label{\detokenize{herron_01_practice_03::doc}}

\subsection{Announcements}
\label{\detokenize{herron_01_practice_03:announcements}}\begin{itemize}
\item {} 
\sphinxAtStartPar
Quiz 2 \sphinxhyphen{} mean was \$\textbackslash{}approx 90\%\$

\item {} 
\sphinxAtStartPar
Quiz 3 \sphinxhyphen{} due by 11:59 on Friday, 2/10

\item {} 
\sphinxAtStartPar
Project groups open on Canvas under People \sphinxhyphen{} please sign up!

\item {} 
\sphinxAtStartPar
Optional, anonymous survey on Canvas under “Quizzes” \sphinxhyphen{} I value your feedback

\end{itemize}


\subsection{Practice}
\label{\detokenize{herron_01_practice_03:practice}}
\sphinxAtStartPar
On Discovery, we need to install the following pacakges every time we log in:

\begin{sphinxuseclass}{cell}\begin{sphinxVerbatimInput}

\begin{sphinxuseclass}{cell_input}
\begin{sphinxVerbatim}[commandchars=\\\{\}]
\PYG{c+c1}{\PYGZsh{} \PYGZpc{}pip install yfinance pandas\PYGZhy{}datareader requests\PYGZhy{}cache}
\end{sphinxVerbatim}

\end{sphinxuseclass}\end{sphinxVerbatimInput}

\end{sphinxuseclass}
\begin{sphinxuseclass}{cell}\begin{sphinxVerbatimInput}

\begin{sphinxuseclass}{cell_input}
\begin{sphinxVerbatim}[commandchars=\\\{\}]
\PYG{k+kn}{import} \PYG{n+nn}{matplotlib}\PYG{n+nn}{.}\PYG{n+nn}{pyplot} \PYG{k}{as} \PYG{n+nn}{plt}
\PYG{k+kn}{import} \PYG{n+nn}{numpy} \PYG{k}{as} \PYG{n+nn}{np}
\PYG{k+kn}{import} \PYG{n+nn}{pandas} \PYG{k}{as} \PYG{n+nn}{pd}

\PYG{k+kn}{import} \PYG{n+nn}{yfinance} \PYG{k}{as} \PYG{n+nn}{yf}
\PYG{k+kn}{import} \PYG{n+nn}{pandas\PYGZus{}datareader} \PYG{k}{as} \PYG{n+nn}{pdr}
\PYG{k+kn}{import} \PYG{n+nn}{requests\PYGZus{}cache}
\end{sphinxVerbatim}

\end{sphinxuseclass}\end{sphinxVerbatimInput}

\end{sphinxuseclass}
\begin{sphinxuseclass}{cell}\begin{sphinxVerbatimInput}

\begin{sphinxuseclass}{cell_input}
\begin{sphinxVerbatim}[commandchars=\\\{\}]
\PYG{o}{\PYGZpc{}}\PYG{k}{config} InlineBackend.figure\PYGZus{}format = \PYGZsq{}retina\PYGZsq{}
\PYG{o}{\PYGZpc{}}\PYG{k}{precision} 4
\PYG{n}{pd}\PYG{o}{.}\PYG{n}{options}\PYG{o}{.}\PYG{n}{display}\PYG{o}{.}\PYG{n}{float\PYGZus{}format} \PYG{o}{=} \PYG{l+s+s1}{\PYGZsq{}}\PYG{l+s+si}{\PYGZob{}:.4f\PYGZcb{}}\PYG{l+s+s1}{\PYGZsq{}}\PYG{o}{.}\PYG{n}{format}
\PYG{n}{session} \PYG{o}{=} \PYG{n}{requests\PYGZus{}cache}\PYG{o}{.}\PYG{n}{CachedSession}\PYG{p}{(}\PYG{n}{expire\PYGZus{}after}\PYG{o}{=}\PYG{l+m+mi}{1}\PYG{p}{)}
\end{sphinxVerbatim}

\end{sphinxuseclass}\end{sphinxVerbatimInput}

\end{sphinxuseclass}

\subsubsection{Download all available daily price data for tickers TSLA, F, AAPL, AMZN, and META to data frame \sphinxstyleliteralintitle{\sphinxupquote{histories}}}
\label{\detokenize{herron_01_practice_03:download-all-available-daily-price-data-for-tickers-tsla-f-aapl-amzn-and-meta-to-data-frame-histories}}
\sphinxAtStartPar
Remove time zone information from the index and use \sphinxcode{\sphinxupquote{histories.columns.names}} to label the variables and tickers as \sphinxcode{\sphinxupquote{Variable}} and \sphinxcode{\sphinxupquote{Ticker}}.

\begin{sphinxuseclass}{cell}\begin{sphinxVerbatimInput}

\begin{sphinxuseclass}{cell_input}
\begin{sphinxVerbatim}[commandchars=\\\{\}]
\PYG{n}{tickers} \PYG{o}{=} \PYG{n}{yf}\PYG{o}{.}\PYG{n}{Tickers}\PYG{p}{(}\PYG{l+s+s1}{\PYGZsq{}}\PYG{l+s+s1}{TSLA F AAPL AMZN META}\PYG{l+s+s1}{\PYGZsq{}}\PYG{p}{,} \PYG{n}{session}\PYG{o}{=}\PYG{n}{session}\PYG{p}{)}
\PYG{n}{histories} \PYG{o}{=} \PYG{n}{tickers}\PYG{o}{.}\PYG{n}{history}\PYG{p}{(}\PYG{n}{period}\PYG{o}{=}\PYG{l+s+s1}{\PYGZsq{}}\PYG{l+s+s1}{max}\PYG{l+s+s1}{\PYGZsq{}}\PYG{p}{,} \PYG{n}{auto\PYGZus{}adjust}\PYG{o}{=}\PYG{k+kc}{False}\PYG{p}{,} \PYG{n}{progress}\PYG{o}{=}\PYG{k+kc}{False}\PYG{p}{)}
\PYG{n}{histories}\PYG{o}{.}\PYG{n}{columns}\PYG{o}{.}\PYG{n}{names} \PYG{o}{=} \PYG{p}{[}\PYG{l+s+s1}{\PYGZsq{}}\PYG{l+s+s1}{Variable}\PYG{l+s+s1}{\PYGZsq{}}\PYG{p}{,} \PYG{l+s+s1}{\PYGZsq{}}\PYG{l+s+s1}{Ticker}\PYG{l+s+s1}{\PYGZsq{}}\PYG{p}{]}
\PYG{n}{histories}\PYG{o}{.}\PYG{n}{index} \PYG{o}{=} \PYG{n}{histories}\PYG{o}{.}\PYG{n}{index}\PYG{o}{.}\PYG{n}{tz\PYGZus{}localize}\PYG{p}{(}\PYG{k+kc}{None}\PYG{p}{)}
\PYG{n}{histories}\PYG{o}{.}\PYG{n}{head}\PYG{p}{(}\PYG{p}{)}
\end{sphinxVerbatim}

\end{sphinxuseclass}\end{sphinxVerbatimInput}
\begin{sphinxVerbatimOutput}

\begin{sphinxuseclass}{cell_output}
\begin{sphinxVerbatim}[commandchars=\\\{\}]
[*********************100\PYGZpc{}***********************]  5 of 5 completed
\end{sphinxVerbatim}

\begin{sphinxVerbatim}[commandchars=\\\{\}]
Variable   Adj Close                       Close                        ...  \PYGZbs{}
Ticker          AAPL AMZN      F META TSLA  AAPL AMZN      F META TSLA  ...   
Date                                                                    ...   
1972\PYGZhy{}06\PYGZhy{}01       NaN  NaN 0.2673  NaN  NaN   NaN  NaN 2.1532  NaN  NaN  ...   
1972\PYGZhy{}06\PYGZhy{}02       NaN  NaN 0.2668  NaN  NaN   NaN  NaN 2.1492  NaN  NaN  ...   
1972\PYGZhy{}06\PYGZhy{}05       NaN  NaN 0.2668  NaN  NaN   NaN  NaN 2.1492  NaN  NaN  ...   
1972\PYGZhy{}06\PYGZhy{}06       NaN  NaN 0.2638  NaN  NaN   NaN  NaN 2.1248  NaN  NaN  ...   
1972\PYGZhy{}06\PYGZhy{}07       NaN  NaN 0.2623  NaN  NaN   NaN  NaN 2.1127  NaN  NaN  ...   

Variable   Stock Splits                       Volume                          
Ticker             AAPL AMZN      F META TSLA   AAPL AMZN        F META TSLA  
Date                                                                          
1972\PYGZhy{}06\PYGZhy{}01          NaN  NaN 0.0000  NaN  NaN    NaN  NaN  1091238  NaN  NaN  
1972\PYGZhy{}06\PYGZhy{}02          NaN  NaN 0.0000  NaN  NaN    NaN  NaN  1174468  NaN  NaN  
1972\PYGZhy{}06\PYGZhy{}05          NaN  NaN 0.0000  NaN  NaN    NaN  NaN  5209582  NaN  NaN  
1972\PYGZhy{}06\PYGZhy{}06          NaN  NaN 0.0000  NaN  NaN    NaN  NaN  1424158  NaN  NaN  
1972\PYGZhy{}06\PYGZhy{}07          NaN  NaN 0.0000  NaN  NaN    NaN  NaN   675088  NaN  NaN  

[5 rows x 40 columns]
\end{sphinxVerbatim}

\end{sphinxuseclass}\end{sphinxVerbatimOutput}

\end{sphinxuseclass}

\subsubsection{Calculate all available daily returns and save to data frame \sphinxstyleliteralintitle{\sphinxupquote{returns}}}
\label{\detokenize{herron_01_practice_03:calculate-all-available-daily-returns-and-save-to-data-frame-returns}}
\sphinxAtStartPar
\sphinxstyleemphasis{\sphinxstylestrong{The following code assumes data are chronologically ordered!}}
The yfinance package returns sorted data, and we can use \sphinxcode{\sphinxupquote{.sort\_index()}} to sort our data, if necessary.

\begin{sphinxuseclass}{cell}\begin{sphinxVerbatimInput}

\begin{sphinxuseclass}{cell_input}
\begin{sphinxVerbatim}[commandchars=\\\{\}]
\PYG{n}{returns} \PYG{o}{=} \PYG{n}{histories}\PYG{p}{[}\PYG{l+s+s1}{\PYGZsq{}}\PYG{l+s+s1}{Adj Close}\PYG{l+s+s1}{\PYGZsq{}}\PYG{p}{]}\PYG{o}{.}\PYG{n}{pct\PYGZus{}change}\PYG{p}{(}\PYG{p}{)}
\PYG{n}{returns}\PYG{o}{.}\PYG{n}{head}\PYG{p}{(}\PYG{p}{)}
\end{sphinxVerbatim}

\end{sphinxuseclass}\end{sphinxVerbatimInput}
\begin{sphinxVerbatimOutput}

\begin{sphinxuseclass}{cell_output}
\begin{sphinxVerbatim}[commandchars=\\\{\}]
Ticker      AAPL  AMZN       F  META  TSLA
Date                                      
1972\PYGZhy{}06\PYGZhy{}01   NaN   NaN     NaN   NaN   NaN
1972\PYGZhy{}06\PYGZhy{}02   NaN   NaN \PYGZhy{}0.0019   NaN   NaN
1972\PYGZhy{}06\PYGZhy{}05   NaN   NaN  0.0000   NaN   NaN
1972\PYGZhy{}06\PYGZhy{}06   NaN   NaN \PYGZhy{}0.0113   NaN   NaN
1972\PYGZhy{}06\PYGZhy{}07   NaN   NaN \PYGZhy{}0.0057   NaN   NaN
\end{sphinxVerbatim}

\end{sphinxuseclass}\end{sphinxVerbatimOutput}

\end{sphinxuseclass}

\subsubsection{Slices returns for the 2020s and assign to \sphinxstyleliteralintitle{\sphinxupquote{returns\_2020s}}}
\label{\detokenize{herron_01_practice_03:slices-returns-for-the-2020s-and-assign-to-returns-2020s}}
\begin{sphinxuseclass}{cell}\begin{sphinxVerbatimInput}

\begin{sphinxuseclass}{cell_input}
\begin{sphinxVerbatim}[commandchars=\\\{\}]
\PYG{n}{returns\PYGZus{}2020s} \PYG{o}{=} \PYG{n}{returns}\PYG{o}{.}\PYG{n}{loc}\PYG{p}{[}\PYG{l+s+s1}{\PYGZsq{}}\PYG{l+s+s1}{2020}\PYG{l+s+s1}{\PYGZsq{}}\PYG{p}{:}\PYG{p}{]}
\PYG{n}{returns\PYGZus{}2020s}\PYG{o}{.}\PYG{n}{head}\PYG{p}{(}\PYG{p}{)}
\end{sphinxVerbatim}

\end{sphinxuseclass}\end{sphinxVerbatimInput}
\begin{sphinxVerbatimOutput}

\begin{sphinxuseclass}{cell_output}
\begin{sphinxVerbatim}[commandchars=\\\{\}]
Ticker        AAPL    AMZN       F    META   TSLA
Date                                             
2020\PYGZhy{}01\PYGZhy{}02  0.0228  0.0272  0.0129  0.0221 0.0285
2020\PYGZhy{}01\PYGZhy{}03 \PYGZhy{}0.0097 \PYGZhy{}0.0121 \PYGZhy{}0.0223 \PYGZhy{}0.0053 0.0296
2020\PYGZhy{}01\PYGZhy{}06  0.0080  0.0149 \PYGZhy{}0.0054  0.0188 0.0193
2020\PYGZhy{}01\PYGZhy{}07 \PYGZhy{}0.0047  0.0021  0.0098  0.0022 0.0388
2020\PYGZhy{}01\PYGZhy{}08  0.0161 \PYGZhy{}0.0078  0.0000  0.0101 0.0492
\end{sphinxVerbatim}

\end{sphinxuseclass}\end{sphinxVerbatimOutput}

\end{sphinxuseclass}

\subsubsection{Download all available data for the Fama and French daily benchmark factors to dictionary \sphinxstyleliteralintitle{\sphinxupquote{ff\_all}}}
\label{\detokenize{herron_01_practice_03:download-all-available-data-for-the-fama-and-french-daily-benchmark-factors-to-dictionary-ff-all}}
\sphinxAtStartPar
I often use the following code snippet to find the exact name for the the daily benchmark factors file.

\begin{sphinxuseclass}{cell}\begin{sphinxVerbatimInput}

\begin{sphinxuseclass}{cell_input}
\begin{sphinxVerbatim}[commandchars=\\\{\}]
\PYG{n}{pdr}\PYG{o}{.}\PYG{n}{famafrench}\PYG{o}{.}\PYG{n}{get\PYGZus{}available\PYGZus{}datasets}\PYG{p}{(}\PYG{p}{)}\PYG{p}{[}\PYG{p}{:}\PYG{l+m+mi}{5}\PYG{p}{]}
\end{sphinxVerbatim}

\end{sphinxuseclass}\end{sphinxVerbatimInput}
\begin{sphinxVerbatimOutput}

\begin{sphinxuseclass}{cell_output}
\begin{sphinxVerbatim}[commandchars=\\\{\}]
[\PYGZsq{}F\PYGZhy{}F\PYGZus{}Research\PYGZus{}Data\PYGZus{}Factors\PYGZsq{},
 \PYGZsq{}F\PYGZhy{}F\PYGZus{}Research\PYGZus{}Data\PYGZus{}Factors\PYGZus{}weekly\PYGZsq{},
 \PYGZsq{}F\PYGZhy{}F\PYGZus{}Research\PYGZus{}Data\PYGZus{}Factors\PYGZus{}daily\PYGZsq{},
 \PYGZsq{}F\PYGZhy{}F\PYGZus{}Research\PYGZus{}Data\PYGZus{}5\PYGZus{}Factors\PYGZus{}2x3\PYGZsq{},
 \PYGZsq{}F\PYGZhy{}F\PYGZus{}Research\PYGZus{}Data\PYGZus{}5\PYGZus{}Factors\PYGZus{}2x3\PYGZus{}daily\PYGZsq{}]
\end{sphinxVerbatim}

\end{sphinxuseclass}\end{sphinxVerbatimOutput}

\end{sphinxuseclass}
\sphinxAtStartPar
Then I copy\sphinxhyphen{}and\sphinxhyphen{}paste that file name into \sphinxcode{\sphinxupquote{pdr.DataReader()}}.

\begin{sphinxuseclass}{cell}\begin{sphinxVerbatimInput}

\begin{sphinxuseclass}{cell_input}
\begin{sphinxVerbatim}[commandchars=\\\{\}]
\PYG{n}{ff\PYGZus{}all} \PYG{o}{=} \PYG{n}{pdr}\PYG{o}{.}\PYG{n}{DataReader}\PYG{p}{(}
    \PYG{n}{name}\PYG{o}{=}\PYG{l+s+s1}{\PYGZsq{}}\PYG{l+s+s1}{F\PYGZhy{}F\PYGZus{}Research\PYGZus{}Data\PYGZus{}Factors\PYGZus{}daily}\PYG{l+s+s1}{\PYGZsq{}}\PYG{p}{,}
    \PYG{n}{data\PYGZus{}source}\PYG{o}{=}\PYG{l+s+s1}{\PYGZsq{}}\PYG{l+s+s1}{famafrench}\PYG{l+s+s1}{\PYGZsq{}}\PYG{p}{,}
    \PYG{n}{start}\PYG{o}{=}\PYG{l+s+s1}{\PYGZsq{}}\PYG{l+s+s1}{1900}\PYG{l+s+s1}{\PYGZsq{}}\PYG{p}{,}
    \PYG{n}{session}\PYG{o}{=}\PYG{n}{session}
\PYG{p}{)}
\end{sphinxVerbatim}

\end{sphinxuseclass}\end{sphinxVerbatimInput}

\end{sphinxuseclass}

\subsubsection{Slice the daily benchmark factors, convert them to decimal returns, and assign to \sphinxstyleliteralintitle{\sphinxupquote{ff}}}
\label{\detokenize{herron_01_practice_03:slice-the-daily-benchmark-factors-convert-them-to-decimal-returns-and-assign-to-ff}}
\begin{sphinxuseclass}{cell}\begin{sphinxVerbatimInput}

\begin{sphinxuseclass}{cell_input}
\begin{sphinxVerbatim}[commandchars=\\\{\}]
\PYG{n}{ff} \PYG{o}{=} \PYG{n}{ff\PYGZus{}all}\PYG{p}{[}\PYG{l+m+mi}{0}\PYG{p}{]} \PYG{o}{/} \PYG{l+m+mi}{100}
\PYG{n}{ff}\PYG{o}{.}\PYG{n}{head}\PYG{p}{(}\PYG{p}{)}
\end{sphinxVerbatim}

\end{sphinxuseclass}\end{sphinxVerbatimInput}
\begin{sphinxVerbatimOutput}

\begin{sphinxuseclass}{cell_output}
\begin{sphinxVerbatim}[commandchars=\\\{\}]
            Mkt\PYGZhy{}RF     SMB     HML     RF
Date                                     
1926\PYGZhy{}07\PYGZhy{}01  0.0010 \PYGZhy{}0.0025 \PYGZhy{}0.0027 0.0001
1926\PYGZhy{}07\PYGZhy{}02  0.0045 \PYGZhy{}0.0033 \PYGZhy{}0.0006 0.0001
1926\PYGZhy{}07\PYGZhy{}06  0.0017  0.0030 \PYGZhy{}0.0039 0.0001
1926\PYGZhy{}07\PYGZhy{}07  0.0009 \PYGZhy{}0.0058  0.0002 0.0001
1926\PYGZhy{}07\PYGZhy{}08  0.0021 \PYGZhy{}0.0038  0.0019 0.0001
\end{sphinxVerbatim}

\end{sphinxuseclass}\end{sphinxVerbatimOutput}

\end{sphinxuseclass}

\subsubsection{Use the \sphinxstyleliteralintitle{\sphinxupquote{.cumprod()}} method to plot cumulative returns for these stocks in the 2020s}
\label{\detokenize{herron_01_practice_03:use-the-cumprod-method-to-plot-cumulative-returns-for-these-stocks-in-the-2020s}}
\begin{sphinxuseclass}{cell}\begin{sphinxVerbatimInput}

\begin{sphinxuseclass}{cell_input}
\begin{sphinxVerbatim}[commandchars=\\\{\}]
\PYG{n}{cumprod\PYGZus{}solution} \PYG{o}{=} \PYG{n}{returns\PYGZus{}2020s}\PYG{o}{.}\PYG{n}{add}\PYG{p}{(}\PYG{l+m+mi}{1}\PYG{p}{)}\PYG{o}{.}\PYG{n}{cumprod}\PYG{p}{(}\PYG{p}{)}\PYG{o}{.}\PYG{n}{sub}\PYG{p}{(}\PYG{l+m+mi}{1}\PYG{p}{)}\PYG{o}{.}\PYG{n}{mul}\PYG{p}{(}\PYG{l+m+mi}{100}\PYG{p}{)}
\PYG{n}{cumprod\PYGZus{}solution}\PYG{o}{.}\PYG{n}{plot}\PYG{p}{(}\PYG{p}{)}
\PYG{n}{plt}\PYG{o}{.}\PYG{n}{ylabel}\PYG{p}{(}\PYG{l+s+s1}{\PYGZsq{}}\PYG{l+s+s1}{Cumulative Return (}\PYG{l+s+s1}{\PYGZpc{}}\PYG{l+s+s1}{)}\PYG{l+s+s1}{\PYGZsq{}}\PYG{p}{)}
\PYG{n}{plt}\PYG{o}{.}\PYG{n}{title}\PYG{p}{(}\PYG{l+s+sa}{f}\PYG{l+s+s1}{\PYGZsq{}}\PYG{l+s+s1}{Cumulative Returns from }\PYG{l+s+si}{\PYGZob{}}\PYG{n}{cumprod\PYGZus{}solution}\PYG{o}{.}\PYG{n}{index}\PYG{p}{[}\PYG{l+m+mi}{0}\PYG{p}{]}\PYG{l+s+si}{:}\PYG{l+s+s1}{\PYGZpc{}Y}\PYG{l+s+si}{\PYGZcb{}}\PYG{l+s+s1}{ to }\PYG{l+s+si}{\PYGZob{}}\PYG{n}{cumprod\PYGZus{}solution}\PYG{o}{.}\PYG{n}{index}\PYG{p}{[}\PYG{o}{\PYGZhy{}}\PYG{l+m+mi}{1}\PYG{p}{]}\PYG{l+s+si}{:}\PYG{l+s+s1}{\PYGZpc{}Y}\PYG{l+s+si}{\PYGZcb{}}\PYG{l+s+s1}{\PYGZsq{}}\PYG{p}{)}
\PYG{n}{plt}\PYG{o}{.}\PYG{n}{show}\PYG{p}{(}\PYG{p}{)}
\end{sphinxVerbatim}

\end{sphinxuseclass}\end{sphinxVerbatimInput}
\begin{sphinxVerbatimOutput}

\begin{sphinxuseclass}{cell_output}
\noindent\sphinxincludegraphics{{8e5367d3e3e786a97bee08ca5e1fe6a6e208ae2f5e6e0813e814f2c2cb9f0bab}.png}

\end{sphinxuseclass}\end{sphinxVerbatimOutput}

\end{sphinxuseclass}

\subsubsection{Use the \sphinxstyleliteralintitle{\sphinxupquote{.cumsum()}} method with log returns to plot cumulative returns for these stocks in the 2020s}
\label{\detokenize{herron_01_practice_03:use-the-cumsum-method-with-log-returns-to-plot-cumulative-returns-for-these-stocks-in-the-2020s}}
\begin{sphinxuseclass}{cell}\begin{sphinxVerbatimInput}

\begin{sphinxuseclass}{cell_input}
\begin{sphinxVerbatim}[commandchars=\\\{\}]
\PYG{n}{cumsum\PYGZus{}solution} \PYG{o}{=} \PYG{n}{returns\PYGZus{}2020s}\PYG{o}{.}\PYG{n}{add}\PYG{p}{(}\PYG{l+m+mi}{1}\PYG{p}{)}\PYG{o}{.}\PYG{n}{pipe}\PYG{p}{(}\PYG{n}{np}\PYG{o}{.}\PYG{n}{log}\PYG{p}{)}\PYG{o}{.}\PYG{n}{cumsum}\PYG{p}{(}\PYG{p}{)}\PYG{o}{.}\PYG{n}{pipe}\PYG{p}{(}\PYG{n}{np}\PYG{o}{.}\PYG{n}{exp}\PYG{p}{)}\PYG{o}{.}\PYG{n}{sub}\PYG{p}{(}\PYG{l+m+mi}{1}\PYG{p}{)}\PYG{o}{.}\PYG{n}{mul}\PYG{p}{(}\PYG{l+m+mi}{100}\PYG{p}{)}
\PYG{n}{cumsum\PYGZus{}solution}\PYG{o}{.}\PYG{n}{plot}\PYG{p}{(}\PYG{p}{)}
\PYG{n}{plt}\PYG{o}{.}\PYG{n}{ylabel}\PYG{p}{(}\PYG{l+s+s1}{\PYGZsq{}}\PYG{l+s+s1}{Cumulative Return (}\PYG{l+s+s1}{\PYGZpc{}}\PYG{l+s+s1}{)}\PYG{l+s+s1}{\PYGZsq{}}\PYG{p}{)}
\PYG{n}{plt}\PYG{o}{.}\PYG{n}{title}\PYG{p}{(}\PYG{l+s+sa}{f}\PYG{l+s+s1}{\PYGZsq{}}\PYG{l+s+s1}{Cumulative Returns from }\PYG{l+s+si}{\PYGZob{}}\PYG{n}{cumsum\PYGZus{}solution}\PYG{o}{.}\PYG{n}{index}\PYG{p}{[}\PYG{l+m+mi}{0}\PYG{p}{]}\PYG{l+s+si}{:}\PYG{l+s+s1}{\PYGZpc{}Y}\PYG{l+s+si}{\PYGZcb{}}\PYG{l+s+s1}{ to }\PYG{l+s+si}{\PYGZob{}}\PYG{n}{cumsum\PYGZus{}solution}\PYG{o}{.}\PYG{n}{index}\PYG{p}{[}\PYG{o}{\PYGZhy{}}\PYG{l+m+mi}{1}\PYG{p}{]}\PYG{l+s+si}{:}\PYG{l+s+s1}{\PYGZpc{}Y}\PYG{l+s+si}{\PYGZcb{}}\PYG{l+s+s1}{\PYGZsq{}}\PYG{p}{)}
\PYG{n}{plt}\PYG{o}{.}\PYG{n}{show}\PYG{p}{(}\PYG{p}{)}
\end{sphinxVerbatim}

\end{sphinxuseclass}\end{sphinxVerbatimInput}
\begin{sphinxVerbatimOutput}

\begin{sphinxuseclass}{cell_output}
\noindent\sphinxincludegraphics{{8e5367d3e3e786a97bee08ca5e1fe6a6e208ae2f5e6e0813e814f2c2cb9f0bab}.png}

\end{sphinxuseclass}\end{sphinxVerbatimOutput}

\end{sphinxuseclass}
\sphinxAtStartPar
The \sphinxcode{\sphinxupquote{.cumprod()}} and  \sphinxcode{\sphinxupquote{.cumsum()}} solutions are the same!

\begin{sphinxuseclass}{cell}\begin{sphinxVerbatimInput}

\begin{sphinxuseclass}{cell_input}
\begin{sphinxVerbatim}[commandchars=\\\{\}]
\PYG{n}{np}\PYG{o}{.}\PYG{n}{allclose}\PYG{p}{(}\PYG{n}{cumprod\PYGZus{}solution}\PYG{p}{,} \PYG{n}{cumsum\PYGZus{}solution}\PYG{p}{)}
\end{sphinxVerbatim}

\end{sphinxuseclass}\end{sphinxVerbatimInput}
\begin{sphinxVerbatimOutput}

\begin{sphinxuseclass}{cell_output}
\begin{sphinxVerbatim}[commandchars=\\\{\}]
True
\end{sphinxVerbatim}

\end{sphinxuseclass}\end{sphinxVerbatimOutput}

\end{sphinxuseclass}
\sphinxAtStartPar
In this case, the \sphinxcode{\sphinxupquote{.cumprod()}} solution is faster than the \sphinxcode{\sphinxupquote{.cumsum()}} solution, but summing log returns is typically faster than compounding simple returns.

\begin{sphinxuseclass}{cell}\begin{sphinxVerbatimInput}

\begin{sphinxuseclass}{cell_input}
\begin{sphinxVerbatim}[commandchars=\\\{\}]
\PYG{o}{\PYGZpc{}}\PYG{k}{timeit} returns\PYGZus{}2020s.add(1).cumprod().sub(1)
\end{sphinxVerbatim}

\end{sphinxuseclass}\end{sphinxVerbatimInput}
\begin{sphinxVerbatimOutput}

\begin{sphinxuseclass}{cell_output}
\begin{sphinxVerbatim}[commandchars=\\\{\}]
96.3 µs ± 719 ns per loop (mean ± std. dev. of 7 runs, 10000 loops each)
\end{sphinxVerbatim}

\end{sphinxuseclass}\end{sphinxVerbatimOutput}

\end{sphinxuseclass}
\begin{sphinxuseclass}{cell}\begin{sphinxVerbatimInput}

\begin{sphinxuseclass}{cell_input}
\begin{sphinxVerbatim}[commandchars=\\\{\}]
\PYG{o}{\PYGZpc{}}\PYG{k}{timeit} returns\PYGZus{}2020s.add(1).pipe(np.log).cumsum().pipe(np.exp).sub(1)
\end{sphinxVerbatim}

\end{sphinxuseclass}\end{sphinxVerbatimInput}
\begin{sphinxVerbatimOutput}

\begin{sphinxuseclass}{cell_output}
\begin{sphinxVerbatim}[commandchars=\\\{\}]
168 µs ± 3.64 µs per loop (mean ± std. dev. of 7 runs, 10000 loops each)
\end{sphinxVerbatim}

\end{sphinxuseclass}\end{sphinxVerbatimOutput}

\end{sphinxuseclass}

\subsubsection{Use price data only to plot cumulative returns for these stocks in the 2020s}
\label{\detokenize{herron_01_practice_03:use-price-data-only-to-plot-cumulative-returns-for-these-stocks-in-the-2020s}}
\sphinxAtStartPar
We can also calculate cumulative returns as the ratio of adjusted closed.
That is \$R\_\{0,T\} = \textbackslash{}frac\{AC\_T\}\{AC\_0\} \sphinxhyphen{} 1\$.

\begin{sphinxuseclass}{cell}\begin{sphinxVerbatimInput}

\begin{sphinxuseclass}{cell_input}
\begin{sphinxVerbatim}[commandchars=\\\{\}]
\PYG{n}{adjclose\PYGZus{}2019\PYGZus{}last} \PYG{o}{=} \PYG{n}{histories}\PYG{p}{[}\PYG{l+s+s1}{\PYGZsq{}}\PYG{l+s+s1}{Adj Close}\PYG{l+s+s1}{\PYGZsq{}}\PYG{p}{]}\PYG{o}{.}\PYG{n}{loc}\PYG{p}{[}\PYG{l+s+s1}{\PYGZsq{}}\PYG{l+s+s1}{2019}\PYG{l+s+s1}{\PYGZsq{}}\PYG{p}{]}\PYG{o}{.}\PYG{n}{iloc}\PYG{p}{[}\PYG{o}{\PYGZhy{}}\PYG{l+m+mi}{1}\PYG{p}{]}
\PYG{n}{adjclose\PYGZus{}2019\PYGZus{}last}
\end{sphinxVerbatim}

\end{sphinxuseclass}\end{sphinxVerbatimInput}
\begin{sphinxVerbatimOutput}

\begin{sphinxuseclass}{cell_output}
\begin{sphinxVerbatim}[commandchars=\\\{\}]
Ticker
AAPL    71.9206
AMZN    92.3920
F        8.8114
META   205.2500
TSLA    27.8887
Name: 2019\PYGZhy{}12\PYGZhy{}31 00:00:00, dtype: float64
\end{sphinxVerbatim}

\end{sphinxuseclass}\end{sphinxVerbatimOutput}

\end{sphinxuseclass}
\begin{sphinxuseclass}{cell}\begin{sphinxVerbatimInput}

\begin{sphinxuseclass}{cell_input}
\begin{sphinxVerbatim}[commandchars=\\\{\}]
\PYG{n}{adjclose\PYGZus{}2020s} \PYG{o}{=} \PYG{n}{histories}\PYG{p}{[}\PYG{l+s+s1}{\PYGZsq{}}\PYG{l+s+s1}{Adj Close}\PYG{l+s+s1}{\PYGZsq{}}\PYG{p}{]}\PYG{o}{.}\PYG{n}{loc}\PYG{p}{[}\PYG{l+s+s1}{\PYGZsq{}}\PYG{l+s+s1}{2020}\PYG{l+s+s1}{\PYGZsq{}}\PYG{p}{:}\PYG{p}{]}
\PYG{n}{adjclose\PYGZus{}2020s}\PYG{o}{.}\PYG{n}{head}\PYG{p}{(}\PYG{p}{)}
\end{sphinxVerbatim}

\end{sphinxuseclass}\end{sphinxVerbatimInput}
\begin{sphinxVerbatimOutput}

\begin{sphinxuseclass}{cell_output}
\begin{sphinxVerbatim}[commandchars=\\\{\}]
Ticker        AAPL    AMZN      F     META    TSLA
Date                                              
2020\PYGZhy{}01\PYGZhy{}02 73.5615 94.9005 8.9251 209.7800 28.6840
2020\PYGZhy{}01\PYGZhy{}03 72.8464 93.7485 8.7261 208.6700 29.5340
2020\PYGZhy{}01\PYGZhy{}06 73.4268 95.1440 8.6788 212.6000 30.1027
2020\PYGZhy{}01\PYGZhy{}07 73.0815 95.3430 8.7640 213.0600 31.2707
2020\PYGZhy{}01\PYGZhy{}08 74.2571 94.5985 8.7640 215.2200 32.8093
\end{sphinxVerbatim}

\end{sphinxuseclass}\end{sphinxVerbatimOutput}

\end{sphinxuseclass}
\begin{sphinxuseclass}{cell}\begin{sphinxVerbatimInput}

\begin{sphinxuseclass}{cell_input}
\begin{sphinxVerbatim}[commandchars=\\\{\}]
\PYG{n}{adjclose\PYGZus{}solution} \PYG{o}{=} \PYG{n}{adjclose\PYGZus{}2020s}\PYG{o}{.}\PYG{n}{div}\PYG{p}{(}\PYG{n}{adjclose\PYGZus{}2019\PYGZus{}last}\PYG{p}{)}\PYG{o}{.}\PYG{n}{sub}\PYG{p}{(}\PYG{l+m+mi}{1}\PYG{p}{)}\PYG{o}{.}\PYG{n}{mul}\PYG{p}{(}\PYG{l+m+mi}{100}\PYG{p}{)}
\PYG{n}{adjclose\PYGZus{}solution}\PYG{o}{.}\PYG{n}{plot}\PYG{p}{(}\PYG{p}{)}
\PYG{n}{plt}\PYG{o}{.}\PYG{n}{ylabel}\PYG{p}{(}\PYG{l+s+s1}{\PYGZsq{}}\PYG{l+s+s1}{Cumulative Return (}\PYG{l+s+s1}{\PYGZpc{}}\PYG{l+s+s1}{)}\PYG{l+s+s1}{\PYGZsq{}}\PYG{p}{)}
\PYG{n}{plt}\PYG{o}{.}\PYG{n}{title}\PYG{p}{(}\PYG{l+s+sa}{f}\PYG{l+s+s1}{\PYGZsq{}}\PYG{l+s+s1}{Cumulative Returns from }\PYG{l+s+si}{\PYGZob{}}\PYG{n}{cumsum\PYGZus{}solution}\PYG{o}{.}\PYG{n}{index}\PYG{p}{[}\PYG{l+m+mi}{0}\PYG{p}{]}\PYG{l+s+si}{:}\PYG{l+s+s1}{\PYGZpc{}Y}\PYG{l+s+si}{\PYGZcb{}}\PYG{l+s+s1}{ to }\PYG{l+s+si}{\PYGZob{}}\PYG{n}{cumsum\PYGZus{}solution}\PYG{o}{.}\PYG{n}{index}\PYG{p}{[}\PYG{o}{\PYGZhy{}}\PYG{l+m+mi}{1}\PYG{p}{]}\PYG{l+s+si}{:}\PYG{l+s+s1}{\PYGZpc{}Y}\PYG{l+s+si}{\PYGZcb{}}\PYG{l+s+s1}{\PYGZsq{}}\PYG{p}{)}
\PYG{n}{plt}\PYG{o}{.}\PYG{n}{show}\PYG{p}{(}\PYG{p}{)}
\end{sphinxVerbatim}

\end{sphinxuseclass}\end{sphinxVerbatimInput}
\begin{sphinxVerbatimOutput}

\begin{sphinxuseclass}{cell_output}
\noindent\sphinxincludegraphics{{8e5367d3e3e786a97bee08ca5e1fe6a6e208ae2f5e6e0813e814f2c2cb9f0bab}.png}

\end{sphinxuseclass}\end{sphinxVerbatimOutput}

\end{sphinxuseclass}
\sphinxAtStartPar
This solution is the same as the previous two!

\begin{sphinxuseclass}{cell}\begin{sphinxVerbatimInput}

\begin{sphinxuseclass}{cell_input}
\begin{sphinxVerbatim}[commandchars=\\\{\}]
\PYG{n}{np}\PYG{o}{.}\PYG{n}{allclose}\PYG{p}{(}\PYG{n}{cumprod\PYGZus{}solution}\PYG{p}{,} \PYG{n}{adjclose\PYGZus{}solution}\PYG{p}{)}
\end{sphinxVerbatim}

\end{sphinxuseclass}\end{sphinxVerbatimInput}
\begin{sphinxVerbatimOutput}

\begin{sphinxuseclass}{cell_output}
\begin{sphinxVerbatim}[commandchars=\\\{\}]
True
\end{sphinxVerbatim}

\end{sphinxuseclass}\end{sphinxVerbatimOutput}

\end{sphinxuseclass}
\sphinxAtStartPar
What should we make of these three options?
\begin{enumerate}
\sphinxsetlistlabels{\arabic}{enumi}{enumii}{}{.}%
\item {} 
\sphinxAtStartPar
The \sphinxcode{\sphinxupquote{.cumprod()}} solution is the most intutive, and I use it most often

\item {} 
\sphinxAtStartPar
The \sphinxcode{\sphinxupquote{.cumsum()}} solution is typically faster than the \sphinxcode{\sphinxupquote{.cumprod()}} solution, and I use it when I need to calcualate millions of returns

\item {} 
\sphinxAtStartPar
The ratio of adjusted closes helps build understanding

\end{enumerate}


\subsubsection{Calculate the Sharpe Ratio for TSLA}
\label{\detokenize{herron_01_practice_03:calculate-the-sharpe-ratio-for-tsla}}
\sphinxAtStartPar
Calculate the Sharpe Ratio with all available returns and 2020s returns.
Recall the Sharpe Ratio is \$\textbackslash{}frac\{\textbackslash{}overline\{R\_i \sphinxhyphen{} R\_f\}\}\{\textbackslash{}sigma\_i\}\$, where \$\textbackslash{}sigma\_i\$ is the volatility of \sphinxstyleemphasis{excess} returns.

\sphinxAtStartPar
\sphinxstyleemphasis{\sphinxstylestrong{I suggest you write a function named \sphinxcode{\sphinxupquote{sharpe()}} to use for the rest of this notebook.}}

\begin{sphinxuseclass}{cell}\begin{sphinxVerbatimInput}

\begin{sphinxuseclass}{cell_input}
\begin{sphinxVerbatim}[commandchars=\\\{\}]
\PYG{k}{def} \PYG{n+nf}{sharpe}\PYG{p}{(}\PYG{n}{ri}\PYG{p}{,} \PYG{n}{rf}\PYG{o}{=}\PYG{n}{ff}\PYG{p}{[}\PYG{l+s+s1}{\PYGZsq{}}\PYG{l+s+s1}{RF}\PYG{l+s+s1}{\PYGZsq{}}\PYG{p}{]}\PYG{p}{)}\PYG{p}{:}
    \PYG{n}{ri\PYGZus{}rf} \PYG{o}{=} \PYG{n}{ri}\PYG{o}{.}\PYG{n}{sub}\PYG{p}{(}\PYG{n}{rf}\PYG{p}{)}\PYG{o}{.}\PYG{n}{dropna}\PYG{p}{(}\PYG{p}{)}
    \PYG{k}{return} \PYG{n}{np}\PYG{o}{.}\PYG{n}{sqrt}\PYG{p}{(}\PYG{l+m+mi}{252}\PYG{p}{)} \PYG{o}{*} \PYG{n}{ri\PYGZus{}rf}\PYG{o}{.}\PYG{n}{mean}\PYG{p}{(}\PYG{p}{)} \PYG{o}{/} \PYG{n}{ri\PYGZus{}rf}\PYG{o}{.}\PYG{n}{std}\PYG{p}{(}\PYG{p}{)}
\end{sphinxVerbatim}

\end{sphinxuseclass}\end{sphinxVerbatimInput}

\end{sphinxuseclass}
\begin{sphinxuseclass}{cell}\begin{sphinxVerbatimInput}

\begin{sphinxuseclass}{cell_input}
\begin{sphinxVerbatim}[commandchars=\\\{\}]
\PYG{n}{sharpe}\PYG{p}{(}\PYG{n}{returns\PYGZus{}2020s}\PYG{p}{[}\PYG{l+s+s1}{\PYGZsq{}}\PYG{l+s+s1}{TSLA}\PYG{l+s+s1}{\PYGZsq{}}\PYG{p}{]}\PYG{p}{)}
\end{sphinxVerbatim}

\end{sphinxuseclass}\end{sphinxVerbatimInput}
\begin{sphinxVerbatimOutput}

\begin{sphinxuseclass}{cell_output}
\begin{sphinxVerbatim}[commandchars=\\\{\}]
1.0386986513828833
\end{sphinxVerbatim}

\end{sphinxuseclass}\end{sphinxVerbatimOutput}

\end{sphinxuseclass}
\sphinxAtStartPar
We can use the \sphinxcode{\sphinxupquote{.pipe()}} method to chain the previous calculation.

\begin{sphinxuseclass}{cell}\begin{sphinxVerbatimInput}

\begin{sphinxuseclass}{cell_input}
\begin{sphinxVerbatim}[commandchars=\\\{\}]
\PYG{n}{returns\PYGZus{}2020s}\PYG{p}{[}\PYG{l+s+s1}{\PYGZsq{}}\PYG{l+s+s1}{TSLA}\PYG{l+s+s1}{\PYGZsq{}}\PYG{p}{]}\PYG{o}{.}\PYG{n}{pipe}\PYG{p}{(}\PYG{n}{sharpe}\PYG{p}{)}
\end{sphinxVerbatim}

\end{sphinxuseclass}\end{sphinxVerbatimInput}
\begin{sphinxVerbatimOutput}

\begin{sphinxuseclass}{cell_output}
\begin{sphinxVerbatim}[commandchars=\\\{\}]
1.0386986513828833
\end{sphinxVerbatim}

\end{sphinxuseclass}\end{sphinxVerbatimOutput}

\end{sphinxuseclass}

\subsubsection{Calculate the market beta for TSLA}
\label{\detokenize{herron_01_practice_03:calculate-the-market-beta-for-tsla}}
\sphinxAtStartPar
Calculate the market beta with all available returns and 2020s returns.
Recall we estimate market beta with the ordinary least squares (OLS) regression \$R\_i\sphinxhyphen{}R\_f = \textbackslash{}alpha + \textbackslash{}beta (R\_m\sphinxhyphen{}R\_f) + \textbackslash{}epsilon\$.
We can estimate market beta with the covariance formula (i.e., \$\textbackslash{}beta\_i = \textbackslash{}frac\{Cov(R\_i \sphinxhyphen{} R\_f, R\_m \sphinxhyphen{} R\_f)\}\{Var(R\_m\sphinxhyphen{}R\_f)\}\$) for a univariate regression if we do not need goodness of fit statistics.

\sphinxAtStartPar
\sphinxstyleemphasis{\sphinxstylestrong{I suggest you write a function named \sphinxcode{\sphinxupquote{beta()}} to use for the rest of this notebook.}}

\begin{sphinxuseclass}{cell}\begin{sphinxVerbatimInput}

\begin{sphinxuseclass}{cell_input}
\begin{sphinxVerbatim}[commandchars=\\\{\}]
\PYG{k}{def} \PYG{n+nf}{beta}\PYG{p}{(}\PYG{n}{ri}\PYG{p}{,} \PYG{n}{rf}\PYG{o}{=}\PYG{n}{ff}\PYG{p}{[}\PYG{l+s+s1}{\PYGZsq{}}\PYG{l+s+s1}{RF}\PYG{l+s+s1}{\PYGZsq{}}\PYG{p}{]}\PYG{p}{,} \PYG{n}{rm\PYGZus{}rf}\PYG{o}{=}\PYG{n}{ff}\PYG{p}{[}\PYG{l+s+s1}{\PYGZsq{}}\PYG{l+s+s1}{Mkt\PYGZhy{}RF}\PYG{l+s+s1}{\PYGZsq{}}\PYG{p}{]}\PYG{p}{)}\PYG{p}{:}
    \PYG{n}{ri\PYGZus{}rf} \PYG{o}{=} \PYG{n}{ri}\PYG{o}{.}\PYG{n}{sub}\PYG{p}{(}\PYG{n}{rf}\PYG{p}{)}\PYG{o}{.}\PYG{n}{dropna}\PYG{p}{(}\PYG{p}{)}
    \PYG{n}{rm\PYGZus{}rf} \PYG{o}{=} \PYG{n}{rm\PYGZus{}rf}\PYG{o}{.}\PYG{n}{loc}\PYG{p}{[}\PYG{n}{ri\PYGZus{}rf}\PYG{o}{.}\PYG{n}{index}\PYG{p}{]} \PYG{c+c1}{\PYGZsh{} use same peiod for stock and market}
    \PYG{k}{return} \PYG{n}{ri\PYGZus{}rf}\PYG{o}{.}\PYG{n}{cov}\PYG{p}{(}\PYG{n}{rm\PYGZus{}rf}\PYG{p}{)} \PYG{o}{/} \PYG{n}{rm\PYGZus{}rf}\PYG{o}{.}\PYG{n}{var}\PYG{p}{(}\PYG{p}{)}
\end{sphinxVerbatim}

\end{sphinxuseclass}\end{sphinxVerbatimInput}

\end{sphinxuseclass}
\begin{sphinxuseclass}{cell}\begin{sphinxVerbatimInput}

\begin{sphinxuseclass}{cell_input}
\begin{sphinxVerbatim}[commandchars=\\\{\}]
\PYG{n}{beta}\PYG{p}{(}\PYG{n}{returns\PYGZus{}2020s}\PYG{p}{[}\PYG{l+s+s1}{\PYGZsq{}}\PYG{l+s+s1}{TSLA}\PYG{l+s+s1}{\PYGZsq{}}\PYG{p}{]}\PYG{p}{)}
\end{sphinxVerbatim}

\end{sphinxuseclass}\end{sphinxVerbatimInput}
\begin{sphinxVerbatimOutput}

\begin{sphinxuseclass}{cell_output}
\begin{sphinxVerbatim}[commandchars=\\\{\}]
1.519561127968873
\end{sphinxVerbatim}

\end{sphinxuseclass}\end{sphinxVerbatimOutput}

\end{sphinxuseclass}
\sphinxAtStartPar
Again, we can \sphinxcode{\sphinxupquote{.pipe()}} this calculation.

\begin{sphinxuseclass}{cell}\begin{sphinxVerbatimInput}

\begin{sphinxuseclass}{cell_input}
\begin{sphinxVerbatim}[commandchars=\\\{\}]
\PYG{n}{returns\PYGZus{}2020s}\PYG{p}{[}\PYG{l+s+s1}{\PYGZsq{}}\PYG{l+s+s1}{TSLA}\PYG{l+s+s1}{\PYGZsq{}}\PYG{p}{]}\PYG{o}{.}\PYG{n}{pipe}\PYG{p}{(}\PYG{n}{beta}\PYG{p}{)}
\end{sphinxVerbatim}

\end{sphinxuseclass}\end{sphinxVerbatimInput}
\begin{sphinxVerbatimOutput}

\begin{sphinxuseclass}{cell_output}
\begin{sphinxVerbatim}[commandchars=\\\{\}]
1.519561127968873
\end{sphinxVerbatim}

\end{sphinxuseclass}\end{sphinxVerbatimOutput}

\end{sphinxuseclass}

\subsubsection{Guess the Sharpe Ratios for these stocks in the 2020s}
\label{\detokenize{herron_01_practice_03:guess-the-sharpe-ratios-for-these-stocks-in-the-2020s}}

\subsubsection{Guess the market betas for these stocks in the 2020s}
\label{\detokenize{herron_01_practice_03:guess-the-market-betas-for-these-stocks-in-the-2020s}}

\subsubsection{Calculate the Sharpe Ratios for these stocks in the 2020s}
\label{\detokenize{herron_01_practice_03:calculate-the-sharpe-ratios-for-these-stocks-in-the-2020s}}
\begin{sphinxuseclass}{cell}\begin{sphinxVerbatimInput}

\begin{sphinxuseclass}{cell_input}
\begin{sphinxVerbatim}[commandchars=\\\{\}]
\PYG{k}{for} \PYG{n}{i} \PYG{o+ow}{in} \PYG{n}{returns\PYGZus{}2020s}\PYG{p}{:}
    \PYG{n}{sharpe\PYGZus{}i} \PYG{o}{=} \PYG{n}{sharpe}\PYG{p}{(}\PYG{n}{returns\PYGZus{}2020s}\PYG{p}{[}\PYG{n}{i}\PYG{p}{]}\PYG{p}{)}
    \PYG{n+nb}{print}\PYG{p}{(}\PYG{l+s+sa}{f}\PYG{l+s+s1}{\PYGZsq{}}\PYG{l+s+s1}{Sharpe Ratio for }\PYG{l+s+si}{\PYGZob{}}\PYG{n}{i}\PYG{l+s+si}{\PYGZcb{}}\PYG{l+s+s1}{:}\PYG{l+s+se}{\PYGZbs{}t}\PYG{l+s+s1}{ }\PYG{l+s+si}{\PYGZob{}}\PYG{n}{sharpe\PYGZus{}i}\PYG{l+s+si}{:}\PYG{l+s+s1}{0.2f}\PYG{l+s+si}{\PYGZcb{}}\PYG{l+s+s1}{\PYGZsq{}}\PYG{p}{)}
\end{sphinxVerbatim}

\end{sphinxuseclass}\end{sphinxVerbatimInput}
\begin{sphinxVerbatimOutput}

\begin{sphinxuseclass}{cell_output}
\begin{sphinxVerbatim}[commandchars=\\\{\}]
Sharpe Ratio for AAPL:	 0.70
Sharpe Ratio for AMZN:	 0.10
Sharpe Ratio for F:	 0.42
Sharpe Ratio for META:	 \PYGZhy{}0.13
Sharpe Ratio for TSLA:	 1.04
\end{sphinxVerbatim}

\end{sphinxuseclass}\end{sphinxVerbatimOutput}

\end{sphinxuseclass}
\sphinxAtStartPar
We can also use pandas notation to vectorize this calculation.
First calculate \sphinxstyleemphasis{excess} returns as \$R\_i \sphinxhyphen{} R\_f\$.

\begin{sphinxuseclass}{cell}\begin{sphinxVerbatimInput}

\begin{sphinxuseclass}{cell_input}
\begin{sphinxVerbatim}[commandchars=\\\{\}]
\PYG{n}{returns\PYGZus{}2020s\PYGZus{}excess} \PYG{o}{=} \PYG{n}{returns\PYGZus{}2020s}\PYG{o}{.}\PYG{n}{sub}\PYG{p}{(}\PYG{n}{ff}\PYG{p}{[}\PYG{l+s+s1}{\PYGZsq{}}\PYG{l+s+s1}{RF}\PYG{l+s+s1}{\PYGZsq{}}\PYG{p}{]}\PYG{p}{,} \PYG{n}{axis}\PYG{o}{=}\PYG{l+m+mi}{0}\PYG{p}{)}\PYG{o}{.}\PYG{n}{dropna}\PYG{p}{(}\PYG{p}{)}
\PYG{n}{returns\PYGZus{}2020s\PYGZus{}excess}\PYG{o}{.}\PYG{n}{head}\PYG{p}{(}\PYG{p}{)}
\end{sphinxVerbatim}

\end{sphinxuseclass}\end{sphinxVerbatimInput}
\begin{sphinxVerbatimOutput}

\begin{sphinxuseclass}{cell_output}
\begin{sphinxVerbatim}[commandchars=\\\{\}]
Ticker        AAPL    AMZN       F    META   TSLA
Date                                             
2020\PYGZhy{}01\PYGZhy{}02  0.0228  0.0271  0.0128  0.0220 0.0285
2020\PYGZhy{}01\PYGZhy{}03 \PYGZhy{}0.0098 \PYGZhy{}0.0122 \PYGZhy{}0.0224 \PYGZhy{}0.0054 0.0296
2020\PYGZhy{}01\PYGZhy{}06  0.0079  0.0148 \PYGZhy{}0.0055  0.0188 0.0192
2020\PYGZhy{}01\PYGZhy{}07 \PYGZhy{}0.0048  0.0020  0.0098  0.0021 0.0387
2020\PYGZhy{}01\PYGZhy{}08  0.0160 \PYGZhy{}0.0079 \PYGZhy{}0.0001  0.0101 0.0491
\end{sphinxVerbatim}

\end{sphinxuseclass}\end{sphinxVerbatimOutput}

\end{sphinxuseclass}
\sphinxAtStartPar
Then use pandas notation to calculate means, standard deviations, and annualize.

\begin{sphinxuseclass}{cell}\begin{sphinxVerbatimInput}

\begin{sphinxuseclass}{cell_input}
\begin{sphinxVerbatim}[commandchars=\\\{\}]
\PYG{p}{(}
    \PYG{n}{returns\PYGZus{}2020s\PYGZus{}excess}\PYG{o}{.}\PYG{n}{mean}\PYG{p}{(}\PYG{p}{)}
    \PYG{o}{.}\PYG{n}{div}\PYG{p}{(}\PYG{n}{returns\PYGZus{}2020s\PYGZus{}excess}\PYG{o}{.}\PYG{n}{std}\PYG{p}{(}\PYG{p}{)}\PYG{p}{)}
    \PYG{o}{.}\PYG{n}{mul}\PYG{p}{(}\PYG{n}{np}\PYG{o}{.}\PYG{n}{sqrt}\PYG{p}{(}\PYG{l+m+mi}{252}\PYG{p}{)}\PYG{p}{)}
\PYG{p}{)}
\end{sphinxVerbatim}

\end{sphinxuseclass}\end{sphinxVerbatimInput}
\begin{sphinxVerbatimOutput}

\begin{sphinxuseclass}{cell_output}
\begin{sphinxVerbatim}[commandchars=\\\{\}]
Ticker
AAPL    0.7019
AMZN    0.0985
F       0.4199
META   \PYGZhy{}0.1257
TSLA    1.0387
dtype: float64
\end{sphinxVerbatim}

\end{sphinxuseclass}\end{sphinxVerbatimOutput}

\end{sphinxuseclass}
\sphinxAtStartPar
\sphinxstyleemphasis{\sphinxstylestrong{Note:}}
In a few weeks we will learn the \sphinxcode{\sphinxupquote{.apply()}} method, which avoids the loop syntax.

\begin{sphinxuseclass}{cell}\begin{sphinxVerbatimInput}

\begin{sphinxuseclass}{cell_input}
\begin{sphinxVerbatim}[commandchars=\\\{\}]
\PYG{n}{returns\PYGZus{}2020s}\PYG{o}{.}\PYG{n}{apply}\PYG{p}{(}\PYG{n}{sharpe}\PYG{p}{)}
\end{sphinxVerbatim}

\end{sphinxuseclass}\end{sphinxVerbatimInput}
\begin{sphinxVerbatimOutput}

\begin{sphinxuseclass}{cell_output}
\begin{sphinxVerbatim}[commandchars=\\\{\}]
Ticker
AAPL    0.7019
AMZN    0.0985
F       0.4199
META   \PYGZhy{}0.1257
TSLA    1.0387
dtype: float64
\end{sphinxVerbatim}

\end{sphinxuseclass}\end{sphinxVerbatimOutput}

\end{sphinxuseclass}

\subsubsection{Calculate the market betas for these stocks in the 2020s}
\label{\detokenize{herron_01_practice_03:calculate-the-market-betas-for-these-stocks-in-the-2020s}}
\sphinxAtStartPar
We can loop over \sphinxcode{\sphinxupquote{returns\_2020s}}, but a loop solution is tedious.

\begin{sphinxuseclass}{cell}\begin{sphinxVerbatimInput}

\begin{sphinxuseclass}{cell_input}
\begin{sphinxVerbatim}[commandchars=\\\{\}]
\PYG{k}{for} \PYG{n}{i} \PYG{o+ow}{in} \PYG{n}{returns\PYGZus{}2020s}\PYG{p}{:}
    \PYG{n}{beta\PYGZus{}i} \PYG{o}{=} \PYG{n}{beta}\PYG{p}{(}\PYG{n}{returns\PYGZus{}2020s}\PYG{p}{[}\PYG{n}{i}\PYG{p}{]}\PYG{p}{)}
    \PYG{n+nb}{print}\PYG{p}{(}\PYG{l+s+sa}{f}\PYG{l+s+s1}{\PYGZsq{}}\PYG{l+s+s1}{Beta for }\PYG{l+s+si}{\PYGZob{}}\PYG{n}{i}\PYG{l+s+si}{\PYGZcb{}}\PYG{l+s+s1}{:}\PYG{l+s+se}{\PYGZbs{}t}\PYG{l+s+s1}{ }\PYG{l+s+si}{\PYGZob{}}\PYG{n}{beta\PYGZus{}i}\PYG{l+s+si}{:}\PYG{l+s+s1}{0.2f}\PYG{l+s+si}{\PYGZcb{}}\PYG{l+s+s1}{\PYGZsq{}}\PYG{p}{)}
\end{sphinxVerbatim}

\end{sphinxuseclass}\end{sphinxVerbatimInput}
\begin{sphinxVerbatimOutput}

\begin{sphinxuseclass}{cell_output}
\begin{sphinxVerbatim}[commandchars=\\\{\}]
Beta for AAPL:	 1.16
Beta for AMZN:	 1.00
Beta for F:	 1.21
Beta for META:	 1.23
Beta for TSLA:	 1.52
\end{sphinxVerbatim}

\end{sphinxuseclass}\end{sphinxVerbatimOutput}

\end{sphinxuseclass}
\sphinxAtStartPar
Or we can follow out approach above to vectorize this calculation.
First, we need to add a market excess return column to \sphinxcode{\sphinxupquote{returns\_2020s\_excess}}.

\begin{sphinxuseclass}{cell}\begin{sphinxVerbatimInput}

\begin{sphinxuseclass}{cell_input}
\begin{sphinxVerbatim}[commandchars=\\\{\}]
\PYG{n}{returns\PYGZus{}2020s\PYGZus{}excess}\PYG{p}{[}\PYG{l+s+s1}{\PYGZsq{}}\PYG{l+s+s1}{Mkt\PYGZhy{}RF}\PYG{l+s+s1}{\PYGZsq{}}\PYG{p}{]} \PYG{o}{=} \PYG{n}{ff}\PYG{p}{[}\PYG{l+s+s1}{\PYGZsq{}}\PYG{l+s+s1}{Mkt\PYGZhy{}RF}\PYG{l+s+s1}{\PYGZsq{}}\PYG{p}{]}
\PYG{n}{returns\PYGZus{}2020s\PYGZus{}excess}\PYG{o}{.}\PYG{n}{head}\PYG{p}{(}\PYG{p}{)}
\end{sphinxVerbatim}

\end{sphinxuseclass}\end{sphinxVerbatimInput}
\begin{sphinxVerbatimOutput}

\begin{sphinxuseclass}{cell_output}
\begin{sphinxVerbatim}[commandchars=\\\{\}]
Ticker        AAPL    AMZN       F    META   TSLA  Mkt\PYGZhy{}RF
Date                                                     
2020\PYGZhy{}01\PYGZhy{}02  0.0228  0.0271  0.0128  0.0220 0.0285  0.0086
2020\PYGZhy{}01\PYGZhy{}03 \PYGZhy{}0.0098 \PYGZhy{}0.0122 \PYGZhy{}0.0224 \PYGZhy{}0.0054 0.0296 \PYGZhy{}0.0067
2020\PYGZhy{}01\PYGZhy{}06  0.0079  0.0148 \PYGZhy{}0.0055  0.0188 0.0192  0.0036
2020\PYGZhy{}01\PYGZhy{}07 \PYGZhy{}0.0048  0.0020  0.0098  0.0021 0.0387 \PYGZhy{}0.0019
2020\PYGZhy{}01\PYGZhy{}08  0.0160 \PYGZhy{}0.0079 \PYGZhy{}0.0001  0.0101 0.0491  0.0047
\end{sphinxVerbatim}

\end{sphinxuseclass}\end{sphinxVerbatimOutput}

\end{sphinxuseclass}
\begin{sphinxuseclass}{cell}\begin{sphinxVerbatimInput}

\begin{sphinxuseclass}{cell_input}
\begin{sphinxVerbatim}[commandchars=\\\{\}]
\PYG{n}{vcv} \PYG{o}{=} \PYG{n}{returns\PYGZus{}2020s\PYGZus{}excess}\PYG{o}{.}\PYG{n}{cov}\PYG{p}{(}\PYG{p}{)}
\PYG{n}{vcv}\PYG{o}{.}\PYG{n}{head}\PYG{p}{(}\PYG{p}{)}
\end{sphinxVerbatim}

\end{sphinxuseclass}\end{sphinxVerbatimInput}
\begin{sphinxVerbatimOutput}

\begin{sphinxuseclass}{cell_output}
\begin{sphinxVerbatim}[commandchars=\\\{\}]
Ticker   AAPL   AMZN      F   META   TSLA  Mkt\PYGZhy{}RF
Ticker                                           
AAPL   0.0005 0.0004 0.0003 0.0004 0.0005  0.0003
AMZN   0.0004 0.0006 0.0002 0.0005 0.0005  0.0003
F      0.0003 0.0002 0.0010 0.0003 0.0005  0.0003
META   0.0004 0.0005 0.0003 0.0009 0.0005  0.0003
TSLA   0.0005 0.0005 0.0005 0.0005 0.0021  0.0004
\end{sphinxVerbatim}

\end{sphinxuseclass}\end{sphinxVerbatimOutput}

\end{sphinxuseclass}
\begin{sphinxuseclass}{cell}\begin{sphinxVerbatimInput}

\begin{sphinxuseclass}{cell_input}
\begin{sphinxVerbatim}[commandchars=\\\{\}]
\PYG{n}{vcv}\PYG{p}{[}\PYG{l+s+s1}{\PYGZsq{}}\PYG{l+s+s1}{Mkt\PYGZhy{}RF}\PYG{l+s+s1}{\PYGZsq{}}\PYG{p}{]}\PYG{o}{.}\PYG{n}{div}\PYG{p}{(}\PYG{n}{vcv}\PYG{o}{.}\PYG{n}{loc}\PYG{p}{[}\PYG{l+s+s1}{\PYGZsq{}}\PYG{l+s+s1}{Mkt\PYGZhy{}RF}\PYG{l+s+s1}{\PYGZsq{}}\PYG{p}{,} \PYG{l+s+s1}{\PYGZsq{}}\PYG{l+s+s1}{Mkt\PYGZhy{}RF}\PYG{l+s+s1}{\PYGZsq{}}\PYG{p}{]}\PYG{p}{)}\PYG{o}{.}\PYG{n}{plot}\PYG{p}{(}\PYG{n}{kind}\PYG{o}{=}\PYG{l+s+s1}{\PYGZsq{}}\PYG{l+s+s1}{barh}\PYG{l+s+s1}{\PYGZsq{}}\PYG{p}{)}
\PYG{n}{plt}\PYG{o}{.}\PYG{n}{xlabel}\PYG{p}{(}\PYG{l+s+s1}{\PYGZsq{}}\PYG{l+s+s1}{CAPM Beta}\PYG{l+s+s1}{\PYGZsq{}}\PYG{p}{)}
\PYG{n}{plt}\PYG{o}{.}\PYG{n}{title}\PYG{p}{(}\PYG{l+s+s1}{\PYGZsq{}}\PYG{l+s+s1}{CAPM Betas}\PYG{l+s+s1}{\PYGZsq{}}\PYG{p}{)}
\PYG{n}{plt}\PYG{o}{.}\PYG{n}{show}\PYG{p}{(}\PYG{p}{)}
\end{sphinxVerbatim}

\end{sphinxuseclass}\end{sphinxVerbatimInput}
\begin{sphinxVerbatimOutput}

\begin{sphinxuseclass}{cell_output}
\noindent\sphinxincludegraphics{{d524bdf4c2e8921865b8c16e8a2eeb26f16879dafc5aed2f7dd843d939fe75b6}.png}

\end{sphinxuseclass}\end{sphinxVerbatimOutput}

\end{sphinxuseclass}
\sphinxAtStartPar
\sphinxstyleemphasis{\sphinxstylestrong{Note:}}
In a few weeks we will learn the \sphinxcode{\sphinxupquote{.apply()}} method, which avoids the loop syntax.

\begin{sphinxuseclass}{cell}\begin{sphinxVerbatimInput}

\begin{sphinxuseclass}{cell_input}
\begin{sphinxVerbatim}[commandchars=\\\{\}]
\PYG{n}{returns\PYGZus{}2020s}\PYG{o}{.}\PYG{n}{apply}\PYG{p}{(}\PYG{n}{beta}\PYG{p}{)}
\end{sphinxVerbatim}

\end{sphinxuseclass}\end{sphinxVerbatimInput}
\begin{sphinxVerbatimOutput}

\begin{sphinxuseclass}{cell_output}
\begin{sphinxVerbatim}[commandchars=\\\{\}]
Ticker
AAPL   1.1649
AMZN   1.0038
F      1.2078
META   1.2337
TSLA   1.5196
dtype: float64
\end{sphinxVerbatim}

\end{sphinxuseclass}\end{sphinxVerbatimOutput}

\end{sphinxuseclass}

\subsubsection{Calculate the Sharpe Ratio for an \sphinxstyleemphasis{equally weighted} portfolio of these stocks in the 2020s}
\label{\detokenize{herron_01_practice_03:calculate-the-sharpe-ratio-for-an-equally-weighted-portfolio-of-these-stocks-in-the-2020s}}
\begin{sphinxuseclass}{cell}\begin{sphinxVerbatimInput}

\begin{sphinxuseclass}{cell_input}
\begin{sphinxVerbatim}[commandchars=\\\{\}]
\PYG{n}{returns\PYGZus{}2020s}\PYG{o}{.}\PYG{n}{mean}\PYG{p}{(}\PYG{n}{axis}\PYG{o}{=}\PYG{l+m+mi}{1}\PYG{p}{)}\PYG{o}{.}\PYG{n}{pipe}\PYG{p}{(}\PYG{n}{sharpe}\PYG{p}{)}
\end{sphinxVerbatim}

\end{sphinxuseclass}\end{sphinxVerbatimInput}
\begin{sphinxVerbatimOutput}

\begin{sphinxuseclass}{cell_output}
\begin{sphinxVerbatim}[commandchars=\\\{\}]
0.6439557860929555
\end{sphinxVerbatim}

\end{sphinxuseclass}\end{sphinxVerbatimOutput}

\end{sphinxuseclass}
\sphinxAtStartPar
The Sharpe Ratio of the portfolio increases because diversification decreases the denominator (risk) more than the numerator (return)!

\begin{sphinxuseclass}{cell}\begin{sphinxVerbatimInput}

\begin{sphinxuseclass}{cell_input}
\begin{sphinxVerbatim}[commandchars=\\\{\}]
\PYG{n}{returns\PYGZus{}2020s}\PYG{o}{.}\PYG{n}{apply}\PYG{p}{(}\PYG{n}{sharpe}\PYG{p}{)}\PYG{o}{.}\PYG{n}{mean}\PYG{p}{(}\PYG{p}{)}
\end{sphinxVerbatim}

\end{sphinxuseclass}\end{sphinxVerbatimInput}
\begin{sphinxVerbatimOutput}

\begin{sphinxuseclass}{cell_output}
\begin{sphinxVerbatim}[commandchars=\\\{\}]
0.4267
\end{sphinxVerbatim}

\end{sphinxuseclass}\end{sphinxVerbatimOutput}

\end{sphinxuseclass}

\bigskip\hrule\bigskip


\sphinxAtStartPar
During another class someone asked about the portfolio variance notation from investments class (i.e., \$w\textasciicircum{}T \textbackslash{}Sigma w\$).
We typically will not use this formula because we can calculate the portfolio return series with \sphinxcode{\sphinxupquote{returns.dot(weights)}}, then calculate the variance with \sphinxcode{\sphinxupquote{.var()}}.
Here is a comparison.

\begin{sphinxuseclass}{cell}\begin{sphinxVerbatimInput}

\begin{sphinxuseclass}{cell_input}
\begin{sphinxVerbatim}[commandchars=\\\{\}]
\PYG{n}{\PYGZus{}} \PYG{o}{=} \PYG{n}{returns\PYGZus{}2020s}\PYG{o}{.}\PYG{n}{shape}\PYG{p}{[}\PYG{l+m+mi}{1}\PYG{p}{]}
\PYG{n}{weights} \PYG{o}{=} \PYG{n}{np}\PYG{o}{.}\PYG{n}{ones}\PYG{p}{(}\PYG{n}{\PYGZus{}}\PYG{p}{)} \PYG{o}{/} \PYG{n}{\PYGZus{}}
\end{sphinxVerbatim}

\end{sphinxuseclass}\end{sphinxVerbatimInput}

\end{sphinxuseclass}
\begin{sphinxuseclass}{cell}\begin{sphinxVerbatimInput}

\begin{sphinxuseclass}{cell_input}
\begin{sphinxVerbatim}[commandchars=\\\{\}]
\PYG{n}{np}\PYG{o}{.}\PYG{n}{allclose}\PYG{p}{(}
    \PYG{n}{returns\PYGZus{}2020s}\PYG{o}{.}\PYG{n}{cov}\PYG{p}{(}\PYG{p}{)}\PYG{o}{.}\PYG{n}{dot}\PYG{p}{(}\PYG{n}{weights}\PYG{p}{)}\PYG{o}{.}\PYG{n}{dot}\PYG{p}{(}\PYG{n}{weights}\PYG{p}{)}\PYG{p}{,} \PYG{c+c1}{\PYGZsh{} from investments class}
    \PYG{n}{returns\PYGZus{}2020s}\PYG{o}{.}\PYG{n}{mean}\PYG{p}{(}\PYG{n}{axis}\PYG{o}{=}\PYG{l+m+mi}{1}\PYG{p}{)}\PYG{o}{.}\PYG{n}{var}\PYG{p}{(}\PYG{p}{)} \PYG{c+c1}{\PYGZsh{} from this class}
\PYG{p}{)}
\end{sphinxVerbatim}

\end{sphinxuseclass}\end{sphinxVerbatimInput}
\begin{sphinxVerbatimOutput}

\begin{sphinxuseclass}{cell_output}
\begin{sphinxVerbatim}[commandchars=\\\{\}]
True
\end{sphinxVerbatim}

\end{sphinxuseclass}\end{sphinxVerbatimOutput}

\end{sphinxuseclass}

\bigskip\hrule\bigskip



\subsubsection{Calculate the market beta for an \sphinxstyleemphasis{equally weighted} portfolio of these stocks in the 2020s}
\label{\detokenize{herron_01_practice_03:calculate-the-market-beta-for-an-equally-weighted-portfolio-of-these-stocks-in-the-2020s}}
\begin{sphinxuseclass}{cell}\begin{sphinxVerbatimInput}

\begin{sphinxuseclass}{cell_input}
\begin{sphinxVerbatim}[commandchars=\\\{\}]
\PYG{n}{returns\PYGZus{}2020s}\PYG{o}{.}\PYG{n}{mean}\PYG{p}{(}\PYG{n}{axis}\PYG{o}{=}\PYG{l+m+mi}{1}\PYG{p}{)}\PYG{o}{.}\PYG{n}{pipe}\PYG{p}{(}\PYG{n}{beta}\PYG{p}{)}
\end{sphinxVerbatim}

\end{sphinxuseclass}\end{sphinxVerbatimInput}
\begin{sphinxVerbatimOutput}

\begin{sphinxuseclass}{cell_output}
\begin{sphinxVerbatim}[commandchars=\\\{\}]
1.2259363845829498
\end{sphinxVerbatim}

\end{sphinxuseclass}\end{sphinxVerbatimOutput}

\end{sphinxuseclass}
\sphinxAtStartPar
The portfolio beta is the mean of the portfolio stock betas!

\begin{sphinxuseclass}{cell}\begin{sphinxVerbatimInput}

\begin{sphinxuseclass}{cell_input}
\begin{sphinxVerbatim}[commandchars=\\\{\}]
\PYG{n}{returns\PYGZus{}2020s}\PYG{o}{.}\PYG{n}{apply}\PYG{p}{(}\PYG{n}{beta}\PYG{p}{)}\PYG{o}{.}\PYG{n}{mean}\PYG{p}{(}\PYG{p}{)}
\end{sphinxVerbatim}

\end{sphinxuseclass}\end{sphinxVerbatimInput}
\begin{sphinxVerbatimOutput}

\begin{sphinxuseclass}{cell_output}
\begin{sphinxVerbatim}[commandchars=\\\{\}]
1.2259
\end{sphinxVerbatim}

\end{sphinxuseclass}\end{sphinxVerbatimOutput}

\end{sphinxuseclass}

\subsubsection{Calculate the market betas for these stocks every calendar year for every possible year}
\label{\detokenize{herron_01_practice_03:calculate-the-market-betas-for-these-stocks-every-calendar-year-for-every-possible-year}}
\sphinxAtStartPar
Save these market betas to data frame \sphinxcode{\sphinxupquote{betas}}.
Our current Python knowledge limits us to a for\sphinxhyphen{}loop, but we will learn easier and faster approaches soon!

\begin{sphinxuseclass}{cell}\begin{sphinxVerbatimInput}

\begin{sphinxuseclass}{cell_input}
\begin{sphinxVerbatim}[commandchars=\\\{\}]
\PYG{n}{betas} \PYG{o}{=} \PYG{p}{[}\PYG{p}{]}
\PYG{n}{years} \PYG{o}{=} \PYG{n+nb}{list}\PYG{p}{(}\PYG{n+nb}{range}\PYG{p}{(}\PYG{l+m+mi}{1973}\PYG{p}{,} \PYG{l+m+mi}{2023}\PYG{p}{)}\PYG{p}{)}
\PYG{k}{for} \PYG{n}{year} \PYG{o+ow}{in} \PYG{n}{years}\PYG{p}{:}
    \PYG{n}{betas}\PYG{o}{.}\PYG{n}{append}\PYG{p}{(}\PYG{n}{returns}\PYG{o}{.}\PYG{n}{loc}\PYG{p}{[}\PYG{n+nb}{str}\PYG{p}{(}\PYG{n}{year}\PYG{p}{)}\PYG{p}{]}\PYG{o}{.}\PYG{n}{apply}\PYG{p}{(}\PYG{n}{beta}\PYG{p}{)}\PYG{p}{)}
    
\PYG{n}{betas} \PYG{o}{=} \PYG{n}{pd}\PYG{o}{.}\PYG{n}{DataFrame}\PYG{p}{(}\PYG{n}{betas}\PYG{p}{,} \PYG{n}{index}\PYG{o}{=}\PYG{n}{years}\PYG{p}{)}
\PYG{n}{betas}\PYG{o}{.}\PYG{n}{columns}\PYG{o}{.}\PYG{n}{name} \PYG{o}{=} \PYG{l+s+s1}{\PYGZsq{}}\PYG{l+s+s1}{Ticker}\PYG{l+s+s1}{\PYGZsq{}}
\PYG{n}{betas}\PYG{o}{.}\PYG{n}{index}\PYG{o}{.}\PYG{n}{name} \PYG{o}{=} \PYG{l+s+s1}{\PYGZsq{}}\PYG{l+s+s1}{Year}\PYG{l+s+s1}{\PYGZsq{}}
\PYG{n}{betas}\PYG{o}{.}\PYG{n}{head}\PYG{p}{(}\PYG{p}{)}
\end{sphinxVerbatim}

\end{sphinxuseclass}\end{sphinxVerbatimInput}
\begin{sphinxVerbatimOutput}

\begin{sphinxuseclass}{cell_output}
\begin{sphinxVerbatim}[commandchars=\\\{\}]
Ticker  AAPL  AMZN      F  META  TSLA
Year                                 
1973     NaN   NaN 1.2621   NaN   NaN
1974     NaN   NaN 0.9625   NaN   NaN
1975     NaN   NaN 1.0576   NaN   NaN
1976     NaN   NaN 1.3623   NaN   NaN
1977     NaN   NaN 1.2652   NaN   NaN
\end{sphinxVerbatim}

\end{sphinxuseclass}\end{sphinxVerbatimOutput}

\end{sphinxuseclass}

\subsubsection{Plot the time series of market betas}
\label{\detokenize{herron_01_practice_03:plot-the-time-series-of-market-betas}}
\begin{sphinxuseclass}{cell}\begin{sphinxVerbatimInput}

\begin{sphinxuseclass}{cell_input}
\begin{sphinxVerbatim}[commandchars=\\\{\}]
\PYG{n}{betas}\PYG{o}{.}\PYG{n}{plot}\PYG{p}{(}\PYG{p}{)}
\PYG{n}{plt}\PYG{o}{.}\PYG{n}{ylabel}\PYG{p}{(}\PYG{l+s+s1}{\PYGZsq{}}\PYG{l+s+s1}{CAPM Beta}\PYG{l+s+s1}{\PYGZsq{}}\PYG{p}{)}
\PYG{n}{plt}\PYG{o}{.}\PYG{n}{title}\PYG{p}{(}\PYG{l+s+s1}{\PYGZsq{}}\PYG{l+s+s1}{CAPM Betas}\PYG{l+s+s1}{\PYGZsq{}}\PYG{p}{)}
\PYG{n}{plt}\PYG{o}{.}\PYG{n}{show}\PYG{p}{(}\PYG{p}{)}
\end{sphinxVerbatim}

\end{sphinxuseclass}\end{sphinxVerbatimInput}
\begin{sphinxVerbatimOutput}

\begin{sphinxuseclass}{cell_output}
\noindent\sphinxincludegraphics{{f10a88212df6afe65bd859d405c330119b61cd550bc531334c3e22c41cd0b2bb}.png}

\end{sphinxuseclass}\end{sphinxVerbatimOutput}

\end{sphinxuseclass}
\sphinxstepscope


\section{Herron Topic 1 \sphinxhyphen{} Practice (Section 4, Wednesday 11:45 AM)}
\label{\detokenize{herron_01_practice_04:herron-topic-1-practice-section-4-wednesday-11-45-am}}\label{\detokenize{herron_01_practice_04::doc}}

\subsection{Announcements}
\label{\detokenize{herron_01_practice_04:announcements}}\begin{itemize}
\item {} 
\sphinxAtStartPar
Quiz 2 \sphinxhyphen{} mean was \$\textbackslash{}approx 90\%\$

\item {} 
\sphinxAtStartPar
Quiz 3 \sphinxhyphen{} due by 11:59 on Friday, 2/10

\item {} 
\sphinxAtStartPar
Project groups open on Canvas under People \sphinxhyphen{} please sign up!

\item {} 
\sphinxAtStartPar
Optional, anonymous survey on Canvas under “Quizzes” \sphinxhyphen{} I value your feedback

\end{itemize}


\subsection{Practice}
\label{\detokenize{herron_01_practice_04:practice}}
\sphinxAtStartPar
On Discovery, we need to install the following pacakges every time we log in:

\begin{sphinxuseclass}{cell}\begin{sphinxVerbatimInput}

\begin{sphinxuseclass}{cell_input}
\begin{sphinxVerbatim}[commandchars=\\\{\}]
\PYG{c+c1}{\PYGZsh{} \PYGZpc{}pip install yfinance pandas\PYGZhy{}datareader requests\PYGZhy{}cache}
\end{sphinxVerbatim}

\end{sphinxuseclass}\end{sphinxVerbatimInput}

\end{sphinxuseclass}
\begin{sphinxuseclass}{cell}\begin{sphinxVerbatimInput}

\begin{sphinxuseclass}{cell_input}
\begin{sphinxVerbatim}[commandchars=\\\{\}]
\PYG{k+kn}{import} \PYG{n+nn}{matplotlib}\PYG{n+nn}{.}\PYG{n+nn}{pyplot} \PYG{k}{as} \PYG{n+nn}{plt}
\PYG{k+kn}{import} \PYG{n+nn}{numpy} \PYG{k}{as} \PYG{n+nn}{np}
\PYG{k+kn}{import} \PYG{n+nn}{pandas} \PYG{k}{as} \PYG{n+nn}{pd}

\PYG{k+kn}{import} \PYG{n+nn}{yfinance} \PYG{k}{as} \PYG{n+nn}{yf}
\PYG{k+kn}{import} \PYG{n+nn}{pandas\PYGZus{}datareader} \PYG{k}{as} \PYG{n+nn}{pdr}
\PYG{k+kn}{import} \PYG{n+nn}{requests\PYGZus{}cache}
\end{sphinxVerbatim}

\end{sphinxuseclass}\end{sphinxVerbatimInput}

\end{sphinxuseclass}
\begin{sphinxuseclass}{cell}\begin{sphinxVerbatimInput}

\begin{sphinxuseclass}{cell_input}
\begin{sphinxVerbatim}[commandchars=\\\{\}]
\PYG{o}{\PYGZpc{}}\PYG{k}{config} InlineBackend.figure\PYGZus{}format = \PYGZsq{}retina\PYGZsq{}
\PYG{o}{\PYGZpc{}}\PYG{k}{precision} 4
\PYG{n}{pd}\PYG{o}{.}\PYG{n}{options}\PYG{o}{.}\PYG{n}{display}\PYG{o}{.}\PYG{n}{float\PYGZus{}format} \PYG{o}{=} \PYG{l+s+s1}{\PYGZsq{}}\PYG{l+s+si}{\PYGZob{}:.4f\PYGZcb{}}\PYG{l+s+s1}{\PYGZsq{}}\PYG{o}{.}\PYG{n}{format}
\PYG{n}{session} \PYG{o}{=} \PYG{n}{requests\PYGZus{}cache}\PYG{o}{.}\PYG{n}{CachedSession}\PYG{p}{(}\PYG{n}{expire\PYGZus{}after}\PYG{o}{=}\PYG{l+m+mi}{1}\PYG{p}{)}
\end{sphinxVerbatim}

\end{sphinxuseclass}\end{sphinxVerbatimInput}

\end{sphinxuseclass}

\subsubsection{Download all available daily price data for tickers TSLA, F, AAPL, AMZN, and META to data frame \sphinxstyleliteralintitle{\sphinxupquote{histories}}}
\label{\detokenize{herron_01_practice_04:download-all-available-daily-price-data-for-tickers-tsla-f-aapl-amzn-and-meta-to-data-frame-histories}}
\sphinxAtStartPar
Remove time zone information from the index and use \sphinxcode{\sphinxupquote{histories.columns.names}} to label the variables and tickers as \sphinxcode{\sphinxupquote{Variable}} and \sphinxcode{\sphinxupquote{Ticker}}.

\begin{sphinxuseclass}{cell}\begin{sphinxVerbatimInput}

\begin{sphinxuseclass}{cell_input}
\begin{sphinxVerbatim}[commandchars=\\\{\}]
\PYG{n}{tickers} \PYG{o}{=} \PYG{n}{yf}\PYG{o}{.}\PYG{n}{Tickers}\PYG{p}{(}\PYG{l+s+s1}{\PYGZsq{}}\PYG{l+s+s1}{TSLA F AAPL AMZN META}\PYG{l+s+s1}{\PYGZsq{}}\PYG{p}{,} \PYG{n}{session}\PYG{o}{=}\PYG{n}{session}\PYG{p}{)}
\PYG{n}{histories} \PYG{o}{=} \PYG{n}{tickers}\PYG{o}{.}\PYG{n}{history}\PYG{p}{(}\PYG{n}{period}\PYG{o}{=}\PYG{l+s+s1}{\PYGZsq{}}\PYG{l+s+s1}{max}\PYG{l+s+s1}{\PYGZsq{}}\PYG{p}{,} \PYG{n}{auto\PYGZus{}adjust}\PYG{o}{=}\PYG{k+kc}{False}\PYG{p}{,} \PYG{n}{progress}\PYG{o}{=}\PYG{k+kc}{False}\PYG{p}{)}
\PYG{n}{histories}\PYG{o}{.}\PYG{n}{index} \PYG{o}{=} \PYG{n}{histories}\PYG{o}{.}\PYG{n}{index}\PYG{o}{.}\PYG{n}{tz\PYGZus{}localize}\PYG{p}{(}\PYG{k+kc}{None}\PYG{p}{)}
\PYG{n}{histories}\PYG{o}{.}\PYG{n}{columns}\PYG{o}{.}\PYG{n}{names} \PYG{o}{=} \PYG{p}{[}\PYG{l+s+s1}{\PYGZsq{}}\PYG{l+s+s1}{Variable}\PYG{l+s+s1}{\PYGZsq{}}\PYG{p}{,} \PYG{l+s+s1}{\PYGZsq{}}\PYG{l+s+s1}{Ticker}\PYG{l+s+s1}{\PYGZsq{}}\PYG{p}{]}
\PYG{n}{histories}\PYG{o}{.}\PYG{n}{head}\PYG{p}{(}\PYG{p}{)}
\end{sphinxVerbatim}

\end{sphinxuseclass}\end{sphinxVerbatimInput}
\begin{sphinxVerbatimOutput}

\begin{sphinxuseclass}{cell_output}
\begin{sphinxVerbatim}[commandchars=\\\{\}]
[*********************100\PYGZpc{}***********************]  5 of 5 completed
\end{sphinxVerbatim}

\begin{sphinxVerbatim}[commandchars=\\\{\}]
Variable   Adj Close                       Close                        ...  \PYGZbs{}
Ticker          AAPL AMZN      F META TSLA  AAPL AMZN      F META TSLA  ...   
Date                                                                    ...   
1972\PYGZhy{}06\PYGZhy{}01       NaN  NaN 0.2673  NaN  NaN   NaN  NaN 2.1532  NaN  NaN  ...   
1972\PYGZhy{}06\PYGZhy{}02       NaN  NaN 0.2668  NaN  NaN   NaN  NaN 2.1492  NaN  NaN  ...   
1972\PYGZhy{}06\PYGZhy{}05       NaN  NaN 0.2668  NaN  NaN   NaN  NaN 2.1492  NaN  NaN  ...   
1972\PYGZhy{}06\PYGZhy{}06       NaN  NaN 0.2638  NaN  NaN   NaN  NaN 2.1248  NaN  NaN  ...   
1972\PYGZhy{}06\PYGZhy{}07       NaN  NaN 0.2623  NaN  NaN   NaN  NaN 2.1127  NaN  NaN  ...   

Variable   Stock Splits                       Volume                          
Ticker             AAPL AMZN      F META TSLA   AAPL AMZN        F META TSLA  
Date                                                                          
1972\PYGZhy{}06\PYGZhy{}01          NaN  NaN 0.0000  NaN  NaN    NaN  NaN  1091238  NaN  NaN  
1972\PYGZhy{}06\PYGZhy{}02          NaN  NaN 0.0000  NaN  NaN    NaN  NaN  1174468  NaN  NaN  
1972\PYGZhy{}06\PYGZhy{}05          NaN  NaN 0.0000  NaN  NaN    NaN  NaN  5209582  NaN  NaN  
1972\PYGZhy{}06\PYGZhy{}06          NaN  NaN 0.0000  NaN  NaN    NaN  NaN  1424158  NaN  NaN  
1972\PYGZhy{}06\PYGZhy{}07          NaN  NaN 0.0000  NaN  NaN    NaN  NaN   675088  NaN  NaN  

[5 rows x 40 columns]
\end{sphinxVerbatim}

\end{sphinxuseclass}\end{sphinxVerbatimOutput}

\end{sphinxuseclass}
\sphinxAtStartPar
Here is another way to get the same data.
However, it less flexible because it does create a \sphinxcode{\sphinxupquote{Tickers}} object that we can use to get other data (e.g., earnings per share).
I use \sphinxcode{\sphinxupquote{\_}} as a temporary vaiable for data that I will not need elsewhere in the notebook.

\begin{sphinxuseclass}{cell}\begin{sphinxVerbatimInput}

\begin{sphinxuseclass}{cell_input}
\begin{sphinxVerbatim}[commandchars=\\\{\}]
\PYG{n}{\PYGZus{}} \PYG{o}{=} \PYG{n}{yf}\PYG{o}{.}\PYG{n}{download}\PYG{p}{(}\PYG{l+s+s1}{\PYGZsq{}}\PYG{l+s+s1}{TSLA F AAPL AMZN META}\PYG{l+s+s1}{\PYGZsq{}}\PYG{p}{,} \PYG{n}{progress}\PYG{o}{=}\PYG{k+kc}{False}\PYG{p}{)}
\PYG{n}{\PYGZus{}}\PYG{o}{.}\PYG{n}{head}\PYG{p}{(}\PYG{p}{)}
\end{sphinxVerbatim}

\end{sphinxuseclass}\end{sphinxVerbatimInput}
\begin{sphinxVerbatimOutput}

\begin{sphinxuseclass}{cell_output}
\begin{sphinxVerbatim}[commandchars=\\\{\}]
[*********************100\PYGZpc{}***********************]  5 of 5 completed
\end{sphinxVerbatim}

\begin{sphinxVerbatim}[commandchars=\\\{\}]
                          Adj Close                       Close              \PYGZbs{}
                               AAPL AMZN      F META TSLA  AAPL AMZN      F   
Date                                                                          
1972\PYGZhy{}06\PYGZhy{}01 00:00:00\PYGZhy{}04:00       NaN  NaN 0.2673  NaN  NaN   NaN  NaN 2.1532   
1972\PYGZhy{}06\PYGZhy{}02 00:00:00\PYGZhy{}04:00       NaN  NaN 0.2668  NaN  NaN   NaN  NaN 2.1492   
1972\PYGZhy{}06\PYGZhy{}05 00:00:00\PYGZhy{}04:00       NaN  NaN 0.2668  NaN  NaN   NaN  NaN 2.1492   
1972\PYGZhy{}06\PYGZhy{}06 00:00:00\PYGZhy{}04:00       NaN  NaN 0.2638  NaN  NaN   NaN  NaN 2.1248   
1972\PYGZhy{}06\PYGZhy{}07 00:00:00\PYGZhy{}04:00       NaN  NaN 0.2623  NaN  NaN   NaN  NaN 2.1127   

                                     ... Open                       Volume  \PYGZbs{}
                          META TSLA  ... AAPL AMZN      F META TSLA   AAPL   
Date                                 ...                                     
1972\PYGZhy{}06\PYGZhy{}01 00:00:00\PYGZhy{}04:00  NaN  NaN  ...  NaN  NaN 0.0000  NaN  NaN    NaN   
1972\PYGZhy{}06\PYGZhy{}02 00:00:00\PYGZhy{}04:00  NaN  NaN  ...  NaN  NaN 2.1532  NaN  NaN    NaN   
1972\PYGZhy{}06\PYGZhy{}05 00:00:00\PYGZhy{}04:00  NaN  NaN  ...  NaN  NaN 2.1492  NaN  NaN    NaN   
1972\PYGZhy{}06\PYGZhy{}06 00:00:00\PYGZhy{}04:00  NaN  NaN  ...  NaN  NaN 2.1492  NaN  NaN    NaN   
1972\PYGZhy{}06\PYGZhy{}07 00:00:00\PYGZhy{}04:00  NaN  NaN  ...  NaN  NaN 2.1248  NaN  NaN    NaN   

                                                   
                          AMZN        F META TSLA  
Date                                               
1972\PYGZhy{}06\PYGZhy{}01 00:00:00\PYGZhy{}04:00  NaN  1091238  NaN  NaN  
1972\PYGZhy{}06\PYGZhy{}02 00:00:00\PYGZhy{}04:00  NaN  1174468  NaN  NaN  
1972\PYGZhy{}06\PYGZhy{}05 00:00:00\PYGZhy{}04:00  NaN  5209582  NaN  NaN  
1972\PYGZhy{}06\PYGZhy{}06 00:00:00\PYGZhy{}04:00  NaN  1424158  NaN  NaN  
1972\PYGZhy{}06\PYGZhy{}07 00:00:00\PYGZhy{}04:00  NaN   675088  NaN  NaN  

[5 rows x 30 columns]
\end{sphinxVerbatim}

\end{sphinxuseclass}\end{sphinxVerbatimOutput}

\end{sphinxuseclass}

\subsubsection{Calculate all available daily returns and save to data frame \sphinxstyleliteralintitle{\sphinxupquote{returns}}}
\label{\detokenize{herron_01_practice_04:calculate-all-available-daily-returns-and-save-to-data-frame-returns}}
\sphinxAtStartPar
\sphinxstyleemphasis{\sphinxstylestrong{The following code assumes data are chronologically ordered!}}
The yfinance package returns sorted data, and we can use \sphinxcode{\sphinxupquote{.sort\_index()}} to sort our data, if necessary.

\begin{sphinxuseclass}{cell}\begin{sphinxVerbatimInput}

\begin{sphinxuseclass}{cell_input}
\begin{sphinxVerbatim}[commandchars=\\\{\}]
\PYG{n}{returns} \PYG{o}{=} \PYG{n}{histories}\PYG{p}{[}\PYG{l+s+s1}{\PYGZsq{}}\PYG{l+s+s1}{Adj Close}\PYG{l+s+s1}{\PYGZsq{}}\PYG{p}{]}\PYG{o}{.}\PYG{n}{pct\PYGZus{}change}\PYG{p}{(}\PYG{p}{)}
\PYG{n}{returns}\PYG{o}{.}\PYG{n}{head}\PYG{p}{(}\PYG{p}{)}
\end{sphinxVerbatim}

\end{sphinxuseclass}\end{sphinxVerbatimInput}
\begin{sphinxVerbatimOutput}

\begin{sphinxuseclass}{cell_output}
\begin{sphinxVerbatim}[commandchars=\\\{\}]
Ticker      AAPL  AMZN       F  META  TSLA
Date                                      
1972\PYGZhy{}06\PYGZhy{}01   NaN   NaN     NaN   NaN   NaN
1972\PYGZhy{}06\PYGZhy{}02   NaN   NaN \PYGZhy{}0.0019   NaN   NaN
1972\PYGZhy{}06\PYGZhy{}05   NaN   NaN  0.0000   NaN   NaN
1972\PYGZhy{}06\PYGZhy{}06   NaN   NaN \PYGZhy{}0.0113   NaN   NaN
1972\PYGZhy{}06\PYGZhy{}07   NaN   NaN \PYGZhy{}0.0057   NaN   NaN
\end{sphinxVerbatim}

\end{sphinxuseclass}\end{sphinxVerbatimOutput}

\end{sphinxuseclass}

\subsubsection{Slices returns for the 2020s and assign to \sphinxstyleliteralintitle{\sphinxupquote{returns\_2020s}}}
\label{\detokenize{herron_01_practice_04:slices-returns-for-the-2020s-and-assign-to-returns-2020s}}
\begin{sphinxuseclass}{cell}\begin{sphinxVerbatimInput}

\begin{sphinxuseclass}{cell_input}
\begin{sphinxVerbatim}[commandchars=\\\{\}]
\PYG{n}{returns\PYGZus{}2020s} \PYG{o}{=} \PYG{n}{returns}\PYG{o}{.}\PYG{n}{loc}\PYG{p}{[}\PYG{l+s+s1}{\PYGZsq{}}\PYG{l+s+s1}{2020}\PYG{l+s+s1}{\PYGZsq{}}\PYG{p}{:}\PYG{p}{]}
\PYG{n}{returns\PYGZus{}2020s}\PYG{o}{.}\PYG{n}{head}\PYG{p}{(}\PYG{p}{)}
\end{sphinxVerbatim}

\end{sphinxuseclass}\end{sphinxVerbatimInput}
\begin{sphinxVerbatimOutput}

\begin{sphinxuseclass}{cell_output}
\begin{sphinxVerbatim}[commandchars=\\\{\}]
Ticker        AAPL    AMZN       F    META   TSLA
Date                                             
2020\PYGZhy{}01\PYGZhy{}02  0.0228  0.0272  0.0129  0.0221 0.0285
2020\PYGZhy{}01\PYGZhy{}03 \PYGZhy{}0.0097 \PYGZhy{}0.0121 \PYGZhy{}0.0223 \PYGZhy{}0.0053 0.0296
2020\PYGZhy{}01\PYGZhy{}06  0.0080  0.0149 \PYGZhy{}0.0054  0.0188 0.0193
2020\PYGZhy{}01\PYGZhy{}07 \PYGZhy{}0.0047  0.0021  0.0098  0.0022 0.0388
2020\PYGZhy{}01\PYGZhy{}08  0.0161 \PYGZhy{}0.0078  0.0000  0.0101 0.0492
\end{sphinxVerbatim}

\end{sphinxuseclass}\end{sphinxVerbatimOutput}

\end{sphinxuseclass}

\subsubsection{Download all available data for the Fama and French daily benchmark factors to dictionary \sphinxstyleliteralintitle{\sphinxupquote{ff\_all}}}
\label{\detokenize{herron_01_practice_04:download-all-available-data-for-the-fama-and-french-daily-benchmark-factors-to-dictionary-ff-all}}
\sphinxAtStartPar
I often use the following code snippet to find the exact name for the the daily benchmark factors file.

\begin{sphinxuseclass}{cell}\begin{sphinxVerbatimInput}

\begin{sphinxuseclass}{cell_input}
\begin{sphinxVerbatim}[commandchars=\\\{\}]
\PYG{n}{pdr}\PYG{o}{.}\PYG{n}{famafrench}\PYG{o}{.}\PYG{n}{get\PYGZus{}available\PYGZus{}datasets}\PYG{p}{(}\PYG{p}{)}\PYG{p}{[}\PYG{p}{:}\PYG{l+m+mi}{5}\PYG{p}{]}
\end{sphinxVerbatim}

\end{sphinxuseclass}\end{sphinxVerbatimInput}
\begin{sphinxVerbatimOutput}

\begin{sphinxuseclass}{cell_output}
\begin{sphinxVerbatim}[commandchars=\\\{\}]
[\PYGZsq{}F\PYGZhy{}F\PYGZus{}Research\PYGZus{}Data\PYGZus{}Factors\PYGZsq{},
 \PYGZsq{}F\PYGZhy{}F\PYGZus{}Research\PYGZus{}Data\PYGZus{}Factors\PYGZus{}weekly\PYGZsq{},
 \PYGZsq{}F\PYGZhy{}F\PYGZus{}Research\PYGZus{}Data\PYGZus{}Factors\PYGZus{}daily\PYGZsq{},
 \PYGZsq{}F\PYGZhy{}F\PYGZus{}Research\PYGZus{}Data\PYGZus{}5\PYGZus{}Factors\PYGZus{}2x3\PYGZsq{},
 \PYGZsq{}F\PYGZhy{}F\PYGZus{}Research\PYGZus{}Data\PYGZus{}5\PYGZus{}Factors\PYGZus{}2x3\PYGZus{}daily\PYGZsq{}]
\end{sphinxVerbatim}

\end{sphinxuseclass}\end{sphinxVerbatimOutput}

\end{sphinxuseclass}
\sphinxAtStartPar
Then I copy\sphinxhyphen{}and\sphinxhyphen{}paste that file name into \sphinxcode{\sphinxupquote{pdr.DataReader()}}.

\begin{sphinxuseclass}{cell}\begin{sphinxVerbatimInput}

\begin{sphinxuseclass}{cell_input}
\begin{sphinxVerbatim}[commandchars=\\\{\}]
\PYG{n}{ff\PYGZus{}all} \PYG{o}{=} \PYG{n}{pdr}\PYG{o}{.}\PYG{n}{DataReader}\PYG{p}{(}
    \PYG{n}{name}\PYG{o}{=}\PYG{l+s+s1}{\PYGZsq{}}\PYG{l+s+s1}{F\PYGZhy{}F\PYGZus{}Research\PYGZus{}Data\PYGZus{}Factors\PYGZus{}daily}\PYG{l+s+s1}{\PYGZsq{}}\PYG{p}{,}
    \PYG{n}{data\PYGZus{}source}\PYG{o}{=}\PYG{l+s+s1}{\PYGZsq{}}\PYG{l+s+s1}{famafrench}\PYG{l+s+s1}{\PYGZsq{}}\PYG{p}{,}
    \PYG{n}{start}\PYG{o}{=}\PYG{l+s+s1}{\PYGZsq{}}\PYG{l+s+s1}{1900}\PYG{l+s+s1}{\PYGZsq{}}\PYG{p}{,}
    \PYG{n}{session}\PYG{o}{=}\PYG{n}{session}
\PYG{p}{)}
\end{sphinxVerbatim}

\end{sphinxuseclass}\end{sphinxVerbatimInput}

\end{sphinxuseclass}

\subsubsection{Slice the daily benchmark factors, convert them to decimal returns, and assign to \sphinxstyleliteralintitle{\sphinxupquote{ff}}}
\label{\detokenize{herron_01_practice_04:slice-the-daily-benchmark-factors-convert-them-to-decimal-returns-and-assign-to-ff}}
\begin{sphinxuseclass}{cell}\begin{sphinxVerbatimInput}

\begin{sphinxuseclass}{cell_input}
\begin{sphinxVerbatim}[commandchars=\\\{\}]
\PYG{n}{ff} \PYG{o}{=} \PYG{n}{ff\PYGZus{}all}\PYG{p}{[}\PYG{l+m+mi}{0}\PYG{p}{]}\PYG{o}{.}\PYG{n}{div}\PYG{p}{(}\PYG{l+m+mi}{100}\PYG{p}{)}
\PYG{n}{ff}\PYG{o}{.}\PYG{n}{head}\PYG{p}{(}\PYG{p}{)}
\end{sphinxVerbatim}

\end{sphinxuseclass}\end{sphinxVerbatimInput}
\begin{sphinxVerbatimOutput}

\begin{sphinxuseclass}{cell_output}
\begin{sphinxVerbatim}[commandchars=\\\{\}]
            Mkt\PYGZhy{}RF     SMB     HML     RF
Date                                     
1926\PYGZhy{}07\PYGZhy{}01  0.0010 \PYGZhy{}0.0025 \PYGZhy{}0.0027 0.0001
1926\PYGZhy{}07\PYGZhy{}02  0.0045 \PYGZhy{}0.0033 \PYGZhy{}0.0006 0.0001
1926\PYGZhy{}07\PYGZhy{}06  0.0017  0.0030 \PYGZhy{}0.0039 0.0001
1926\PYGZhy{}07\PYGZhy{}07  0.0009 \PYGZhy{}0.0058  0.0002 0.0001
1926\PYGZhy{}07\PYGZhy{}08  0.0021 \PYGZhy{}0.0038  0.0019 0.0001
\end{sphinxVerbatim}

\end{sphinxuseclass}\end{sphinxVerbatimOutput}

\end{sphinxuseclass}

\subsubsection{Use the \sphinxstyleliteralintitle{\sphinxupquote{.cumprod()}} method to plot cumulative returns for these stocks in the 2020s}
\label{\detokenize{herron_01_practice_04:use-the-cumprod-method-to-plot-cumulative-returns-for-these-stocks-in-the-2020s}}
\begin{sphinxuseclass}{cell}\begin{sphinxVerbatimInput}

\begin{sphinxuseclass}{cell_input}
\begin{sphinxVerbatim}[commandchars=\\\{\}]
\PYG{n}{cumret\PYGZus{}cumprod} \PYG{o}{=} \PYG{n}{returns\PYGZus{}2020s}\PYG{o}{.}\PYG{n}{add}\PYG{p}{(}\PYG{l+m+mi}{1}\PYG{p}{)}\PYG{o}{.}\PYG{n}{cumprod}\PYG{p}{(}\PYG{p}{)}\PYG{o}{.}\PYG{n}{sub}\PYG{p}{(}\PYG{l+m+mi}{1}\PYG{p}{)}\PYG{o}{.}\PYG{n}{mul}\PYG{p}{(}\PYG{l+m+mi}{100}\PYG{p}{)}
\PYG{n}{cumret\PYGZus{}cumprod}\PYG{o}{.}\PYG{n}{plot}\PYG{p}{(}\PYG{p}{)}
\PYG{n}{plt}\PYG{o}{.}\PYG{n}{ylabel}\PYG{p}{(}\PYG{l+s+s1}{\PYGZsq{}}\PYG{l+s+s1}{Cumulative Return (}\PYG{l+s+s1}{\PYGZpc{}}\PYG{l+s+s1}{)}\PYG{l+s+s1}{\PYGZsq{}}\PYG{p}{)}
\PYG{n}{plt}\PYG{o}{.}\PYG{n}{title}\PYG{p}{(}\PYG{l+s+s1}{\PYGZsq{}}\PYG{l+s+s1}{Cumulative Returns}\PYG{l+s+s1}{\PYGZsq{}}\PYG{p}{)}
\PYG{n}{plt}\PYG{o}{.}\PYG{n}{show}\PYG{p}{(}\PYG{p}{)}
\end{sphinxVerbatim}

\end{sphinxuseclass}\end{sphinxVerbatimInput}
\begin{sphinxVerbatimOutput}

\begin{sphinxuseclass}{cell_output}
\noindent\sphinxincludegraphics{{c9c037e87e5c050ed4c2d1704286c907447145d5369901eab03a2dcd840021e6}.png}

\end{sphinxuseclass}\end{sphinxVerbatimOutput}

\end{sphinxuseclass}

\subsubsection{Use the \sphinxstyleliteralintitle{\sphinxupquote{.cumsum()}} method with log returns to plot cumulative returns for these stocks in the 2020s}
\label{\detokenize{herron_01_practice_04:use-the-cumsum-method-with-log-returns-to-plot-cumulative-returns-for-these-stocks-in-the-2020s}}
\begin{sphinxuseclass}{cell}\begin{sphinxVerbatimInput}

\begin{sphinxuseclass}{cell_input}
\begin{sphinxVerbatim}[commandchars=\\\{\}]
\PYG{n}{cumret\PYGZus{}cumsum} \PYG{o}{=} \PYG{n}{returns\PYGZus{}2020s}\PYG{o}{.}\PYG{n}{add}\PYG{p}{(}\PYG{l+m+mi}{1}\PYG{p}{)}\PYG{o}{.}\PYG{n}{pipe}\PYG{p}{(}\PYG{n}{np}\PYG{o}{.}\PYG{n}{log}\PYG{p}{)}\PYG{o}{.}\PYG{n}{cumsum}\PYG{p}{(}\PYG{p}{)}\PYG{o}{.}\PYG{n}{pipe}\PYG{p}{(}\PYG{n}{np}\PYG{o}{.}\PYG{n}{exp}\PYG{p}{)}\PYG{o}{.}\PYG{n}{sub}\PYG{p}{(}\PYG{l+m+mi}{1}\PYG{p}{)}\PYG{o}{.}\PYG{n}{mul}\PYG{p}{(}\PYG{l+m+mi}{100}\PYG{p}{)}
\PYG{n}{cumret\PYGZus{}cumsum}\PYG{o}{.}\PYG{n}{plot}\PYG{p}{(}\PYG{p}{)}
\PYG{n}{plt}\PYG{o}{.}\PYG{n}{ylabel}\PYG{p}{(}\PYG{l+s+s1}{\PYGZsq{}}\PYG{l+s+s1}{Cumulative Return (}\PYG{l+s+s1}{\PYGZpc{}}\PYG{l+s+s1}{)}\PYG{l+s+s1}{\PYGZsq{}}\PYG{p}{)}
\PYG{n}{plt}\PYG{o}{.}\PYG{n}{title}\PYG{p}{(}\PYG{l+s+s1}{\PYGZsq{}}\PYG{l+s+s1}{Cumulative Returns}\PYG{l+s+s1}{\PYGZsq{}}\PYG{p}{)}
\PYG{n}{plt}\PYG{o}{.}\PYG{n}{show}\PYG{p}{(}\PYG{p}{)}
\end{sphinxVerbatim}

\end{sphinxuseclass}\end{sphinxVerbatimInput}
\begin{sphinxVerbatimOutput}

\begin{sphinxuseclass}{cell_output}
\noindent\sphinxincludegraphics{{c9c037e87e5c050ed4c2d1704286c907447145d5369901eab03a2dcd840021e6}.png}

\end{sphinxuseclass}\end{sphinxVerbatimOutput}

\end{sphinxuseclass}
\sphinxAtStartPar
The \sphinxcode{\sphinxupquote{.cumprod()}} and  \sphinxcode{\sphinxupquote{.cumsum()}} solutions are the same!

\begin{sphinxuseclass}{cell}\begin{sphinxVerbatimInput}

\begin{sphinxuseclass}{cell_input}
\begin{sphinxVerbatim}[commandchars=\\\{\}]
\PYG{n}{np}\PYG{o}{.}\PYG{n}{allclose}\PYG{p}{(}\PYG{n}{cumret\PYGZus{}cumprod}\PYG{p}{,} \PYG{n}{cumret\PYGZus{}cumsum}\PYG{p}{)}
\end{sphinxVerbatim}

\end{sphinxuseclass}\end{sphinxVerbatimInput}
\begin{sphinxVerbatimOutput}

\begin{sphinxuseclass}{cell_output}
\begin{sphinxVerbatim}[commandchars=\\\{\}]
True
\end{sphinxVerbatim}

\end{sphinxuseclass}\end{sphinxVerbatimOutput}

\end{sphinxuseclass}

\subsubsection{Use price data only to plot cumulative returns for these stocks in the 2020s}
\label{\detokenize{herron_01_practice_04:use-price-data-only-to-plot-cumulative-returns-for-these-stocks-in-the-2020s}}
\sphinxAtStartPar
We can also calculate cumulative returns as the ratio of adjusted closed.
That is \$R\_\{0,T\} = \textbackslash{}frac\{AC\_T\}\{AC\_0\} \sphinxhyphen{} 1\$.

\begin{sphinxuseclass}{cell}\begin{sphinxVerbatimInput}

\begin{sphinxuseclass}{cell_input}
\begin{sphinxVerbatim}[commandchars=\\\{\}]
\PYG{n}{histories}\PYG{o}{.}\PYG{n}{loc}\PYG{p}{[}\PYG{l+s+s1}{\PYGZsq{}}\PYG{l+s+s1}{2019}\PYG{l+s+s1}{\PYGZsq{}}\PYG{p}{,} \PYG{l+s+s1}{\PYGZsq{}}\PYG{l+s+s1}{Adj Close}\PYG{l+s+s1}{\PYGZsq{}}\PYG{p}{]}\PYG{o}{.}\PYG{n}{iloc}\PYG{p}{[}\PYG{o}{\PYGZhy{}}\PYG{l+m+mi}{1}\PYG{p}{]}
\end{sphinxVerbatim}

\end{sphinxuseclass}\end{sphinxVerbatimInput}
\begin{sphinxVerbatimOutput}

\begin{sphinxuseclass}{cell_output}
\begin{sphinxVerbatim}[commandchars=\\\{\}]
Ticker
AAPL    71.9206
AMZN    92.3920
F        8.8114
META   205.2500
TSLA    27.8887
Name: 2019\PYGZhy{}12\PYGZhy{}31 00:00:00, dtype: float64
\end{sphinxVerbatim}

\end{sphinxuseclass}\end{sphinxVerbatimOutput}

\end{sphinxuseclass}
\begin{sphinxuseclass}{cell}\begin{sphinxVerbatimInput}

\begin{sphinxuseclass}{cell_input}
\begin{sphinxVerbatim}[commandchars=\\\{\}]
\PYG{n}{cumret\PYGZus{}prices} \PYG{o}{=} \PYG{p}{(}
    \PYG{n}{histories}\PYG{p}{[}\PYG{l+s+s1}{\PYGZsq{}}\PYG{l+s+s1}{Adj Close}\PYG{l+s+s1}{\PYGZsq{}}\PYG{p}{]}
    \PYG{o}{.}\PYG{n}{loc}\PYG{p}{[}\PYG{l+s+s1}{\PYGZsq{}}\PYG{l+s+s1}{2020}\PYG{l+s+s1}{\PYGZsq{}}\PYG{p}{:}\PYG{p}{]}
    \PYG{o}{.}\PYG{n}{div}\PYG{p}{(}
        \PYG{n}{histories}
        \PYG{o}{.}\PYG{n}{loc}\PYG{p}{[}\PYG{l+s+s1}{\PYGZsq{}}\PYG{l+s+s1}{2019}\PYG{l+s+s1}{\PYGZsq{}}\PYG{p}{,} \PYG{l+s+s1}{\PYGZsq{}}\PYG{l+s+s1}{Adj Close}\PYG{l+s+s1}{\PYGZsq{}}\PYG{p}{]}
        \PYG{o}{.}\PYG{n}{iloc}\PYG{p}{[}\PYG{o}{\PYGZhy{}}\PYG{l+m+mi}{1}\PYG{p}{]}
    \PYG{p}{)}
    \PYG{o}{.}\PYG{n}{sub}\PYG{p}{(}\PYG{l+m+mi}{1}\PYG{p}{)}
    \PYG{o}{.}\PYG{n}{mul}\PYG{p}{(}\PYG{l+m+mi}{100}\PYG{p}{)}
\PYG{p}{)}
\end{sphinxVerbatim}

\end{sphinxuseclass}\end{sphinxVerbatimInput}

\end{sphinxuseclass}
\begin{sphinxuseclass}{cell}\begin{sphinxVerbatimInput}

\begin{sphinxuseclass}{cell_input}
\begin{sphinxVerbatim}[commandchars=\\\{\}]
\PYG{n}{cumret\PYGZus{}prices}\PYG{o}{.}\PYG{n}{plot}\PYG{p}{(}\PYG{p}{)}
\PYG{n}{plt}\PYG{o}{.}\PYG{n}{ylabel}\PYG{p}{(}\PYG{l+s+s1}{\PYGZsq{}}\PYG{l+s+s1}{Cumulative Return (}\PYG{l+s+s1}{\PYGZpc{}}\PYG{l+s+s1}{)}\PYG{l+s+s1}{\PYGZsq{}}\PYG{p}{)}
\PYG{n}{plt}\PYG{o}{.}\PYG{n}{title}\PYG{p}{(}\PYG{l+s+s1}{\PYGZsq{}}\PYG{l+s+s1}{Cumulative Returns}\PYG{l+s+s1}{\PYGZsq{}}\PYG{p}{)}
\PYG{n}{plt}\PYG{o}{.}\PYG{n}{show}\PYG{p}{(}\PYG{p}{)}
\end{sphinxVerbatim}

\end{sphinxuseclass}\end{sphinxVerbatimInput}
\begin{sphinxVerbatimOutput}

\begin{sphinxuseclass}{cell_output}
\noindent\sphinxincludegraphics{{c9c037e87e5c050ed4c2d1704286c907447145d5369901eab03a2dcd840021e6}.png}

\end{sphinxuseclass}\end{sphinxVerbatimOutput}

\end{sphinxuseclass}
\begin{sphinxuseclass}{cell}\begin{sphinxVerbatimInput}

\begin{sphinxuseclass}{cell_input}
\begin{sphinxVerbatim}[commandchars=\\\{\}]
\PYG{n}{np}\PYG{o}{.}\PYG{n}{allclose}\PYG{p}{(}\PYG{n}{cumret\PYGZus{}prices}\PYG{p}{,} \PYG{n}{cumret\PYGZus{}cumprod}\PYG{p}{)}
\end{sphinxVerbatim}

\end{sphinxuseclass}\end{sphinxVerbatimInput}
\begin{sphinxVerbatimOutput}

\begin{sphinxuseclass}{cell_output}
\begin{sphinxVerbatim}[commandchars=\\\{\}]
True
\end{sphinxVerbatim}

\end{sphinxuseclass}\end{sphinxVerbatimOutput}

\end{sphinxuseclass}

\subsubsection{Calculate the Sharpe Ratio for TSLA}
\label{\detokenize{herron_01_practice_04:calculate-the-sharpe-ratio-for-tsla}}
\sphinxAtStartPar
Calculate the Sharpe Ratio with all available returns and 2020s returns.
Recall the Sharpe Ratio is \$\textbackslash{}frac\{\textbackslash{}overline\{R\_i \sphinxhyphen{} R\_f\}\}\{\textbackslash{}sigma\_i\}\$, where \$\textbackslash{}sigma\_i\$ is the volatility of \sphinxstyleemphasis{excess} returns.

\sphinxAtStartPar
\sphinxstyleemphasis{\sphinxstylestrong{I suggest you write a function named \sphinxcode{\sphinxupquote{sharpe()}} to use for the rest of this notebook.}}

\begin{sphinxuseclass}{cell}\begin{sphinxVerbatimInput}

\begin{sphinxuseclass}{cell_input}
\begin{sphinxVerbatim}[commandchars=\\\{\}]
\PYG{k}{def} \PYG{n+nf}{sharpe}\PYG{p}{(}\PYG{n}{ri}\PYG{p}{,} \PYG{n}{rf}\PYG{o}{=}\PYG{n}{ff}\PYG{p}{[}\PYG{l+s+s1}{\PYGZsq{}}\PYG{l+s+s1}{RF}\PYG{l+s+s1}{\PYGZsq{}}\PYG{p}{]}\PYG{p}{,} \PYG{n}{ann\PYGZus{}fac}\PYG{o}{=}\PYG{n}{np}\PYG{o}{.}\PYG{n}{sqrt}\PYG{p}{(}\PYG{l+m+mi}{252}\PYG{p}{)}\PYG{p}{)}\PYG{p}{:}
    \PYG{n}{ri\PYGZus{}rf} \PYG{o}{=} \PYG{n}{ri}\PYG{o}{.}\PYG{n}{sub}\PYG{p}{(}\PYG{n}{rf}\PYG{p}{)}\PYG{o}{.}\PYG{n}{dropna}\PYG{p}{(}\PYG{p}{)}
    \PYG{k}{return} \PYG{n}{ann\PYGZus{}fac} \PYG{o}{*} \PYG{n}{ri\PYGZus{}rf}\PYG{o}{.}\PYG{n}{mean}\PYG{p}{(}\PYG{p}{)} \PYG{o}{/} \PYG{n}{ri\PYGZus{}rf}\PYG{o}{.}\PYG{n}{std}\PYG{p}{(}\PYG{p}{)}
\end{sphinxVerbatim}

\end{sphinxuseclass}\end{sphinxVerbatimInput}

\end{sphinxuseclass}
\begin{sphinxuseclass}{cell}\begin{sphinxVerbatimInput}

\begin{sphinxuseclass}{cell_input}
\begin{sphinxVerbatim}[commandchars=\\\{\}]
\PYG{n}{sharpe}\PYG{p}{(}\PYG{n}{returns\PYGZus{}2020s}\PYG{p}{[}\PYG{l+s+s1}{\PYGZsq{}}\PYG{l+s+s1}{TSLA}\PYG{l+s+s1}{\PYGZsq{}}\PYG{p}{]}\PYG{p}{)}
\end{sphinxVerbatim}

\end{sphinxuseclass}\end{sphinxVerbatimInput}
\begin{sphinxVerbatimOutput}

\begin{sphinxuseclass}{cell_output}
\begin{sphinxVerbatim}[commandchars=\\\{\}]
1.0386986513828833
\end{sphinxVerbatim}

\end{sphinxuseclass}\end{sphinxVerbatimOutput}

\end{sphinxuseclass}
\sphinxAtStartPar
We can use the \sphinxcode{\sphinxupquote{.pipe()}} method to chain the previous calculation.

\begin{sphinxuseclass}{cell}\begin{sphinxVerbatimInput}

\begin{sphinxuseclass}{cell_input}
\begin{sphinxVerbatim}[commandchars=\\\{\}]
\PYG{n}{returns\PYGZus{}2020s}\PYG{p}{[}\PYG{l+s+s1}{\PYGZsq{}}\PYG{l+s+s1}{TSLA}\PYG{l+s+s1}{\PYGZsq{}}\PYG{p}{]}\PYG{o}{.}\PYG{n}{pipe}\PYG{p}{(}\PYG{n}{sharpe}\PYG{p}{)}
\end{sphinxVerbatim}

\end{sphinxuseclass}\end{sphinxVerbatimInput}
\begin{sphinxVerbatimOutput}

\begin{sphinxuseclass}{cell_output}
\begin{sphinxVerbatim}[commandchars=\\\{\}]
1.0386986513828833
\end{sphinxVerbatim}

\end{sphinxuseclass}\end{sphinxVerbatimOutput}

\end{sphinxuseclass}

\subsubsection{Calculate the market beta for TSLA}
\label{\detokenize{herron_01_practice_04:calculate-the-market-beta-for-tsla}}
\sphinxAtStartPar
Calculate the market beta with all available returns and 2020s returns.
Recall we estimate market beta with the ordinary least squares (OLS) regression \$R\_i\sphinxhyphen{}R\_f = \textbackslash{}alpha + \textbackslash{}beta (R\_m\sphinxhyphen{}R\_f) + \textbackslash{}epsilon\$.
We can estimate market beta with the covariance formula (i.e., \$\textbackslash{}beta\_i = \textbackslash{}frac\{Cov(R\_i \sphinxhyphen{} R\_f, R\_m \sphinxhyphen{} R\_f)\}\{Var(R\_m\sphinxhyphen{}R\_f)\}\$) above for a univariate regression if we do not need goodness of fit statistics.

\sphinxAtStartPar
\sphinxstyleemphasis{\sphinxstylestrong{I suggest you write a function named \sphinxcode{\sphinxupquote{beta()}} to use for the rest of this notebook.}}

\begin{sphinxuseclass}{cell}\begin{sphinxVerbatimInput}

\begin{sphinxuseclass}{cell_input}
\begin{sphinxVerbatim}[commandchars=\\\{\}]
\PYG{k}{def} \PYG{n+nf}{beta}\PYG{p}{(}\PYG{n}{ri}\PYG{p}{,} \PYG{n}{rf}\PYG{o}{=}\PYG{n}{ff}\PYG{p}{[}\PYG{l+s+s1}{\PYGZsq{}}\PYG{l+s+s1}{RF}\PYG{l+s+s1}{\PYGZsq{}}\PYG{p}{]}\PYG{p}{,} \PYG{n}{rm\PYGZus{}rf}\PYG{o}{=}\PYG{n}{ff}\PYG{p}{[}\PYG{l+s+s1}{\PYGZsq{}}\PYG{l+s+s1}{Mkt\PYGZhy{}RF}\PYG{l+s+s1}{\PYGZsq{}}\PYG{p}{]}\PYG{p}{)}\PYG{p}{:}
    \PYG{n}{ri\PYGZus{}rf} \PYG{o}{=} \PYG{n}{ri}\PYG{o}{.}\PYG{n}{sub}\PYG{p}{(}\PYG{n}{rf}\PYG{p}{)}\PYG{o}{.}\PYG{n}{dropna}\PYG{p}{(}\PYG{p}{)}
    \PYG{k}{return} \PYG{n}{ri\PYGZus{}rf}\PYG{o}{.}\PYG{n}{cov}\PYG{p}{(}\PYG{n}{rm\PYGZus{}rf}\PYG{p}{)} \PYG{o}{/} \PYG{n}{rm\PYGZus{}rf}\PYG{o}{.}\PYG{n}{loc}\PYG{p}{[}\PYG{n}{ri\PYGZus{}rf}\PYG{o}{.}\PYG{n}{index}\PYG{p}{]}\PYG{o}{.}\PYG{n}{var}\PYG{p}{(}\PYG{p}{)}
\end{sphinxVerbatim}

\end{sphinxuseclass}\end{sphinxVerbatimInput}

\end{sphinxuseclass}
\begin{sphinxuseclass}{cell}\begin{sphinxVerbatimInput}

\begin{sphinxuseclass}{cell_input}
\begin{sphinxVerbatim}[commandchars=\\\{\}]
\PYG{n}{beta}\PYG{p}{(}\PYG{n}{returns\PYGZus{}2020s}\PYG{p}{[}\PYG{l+s+s1}{\PYGZsq{}}\PYG{l+s+s1}{TSLA}\PYG{l+s+s1}{\PYGZsq{}}\PYG{p}{]}\PYG{p}{)}
\end{sphinxVerbatim}

\end{sphinxuseclass}\end{sphinxVerbatimInput}
\begin{sphinxVerbatimOutput}

\begin{sphinxuseclass}{cell_output}
\begin{sphinxVerbatim}[commandchars=\\\{\}]
1.519561127968873
\end{sphinxVerbatim}

\end{sphinxuseclass}\end{sphinxVerbatimOutput}

\end{sphinxuseclass}
\sphinxAtStartPar
Again, we can \sphinxcode{\sphinxupquote{.pipe()}} this calculation.

\begin{sphinxuseclass}{cell}\begin{sphinxVerbatimInput}

\begin{sphinxuseclass}{cell_input}
\begin{sphinxVerbatim}[commandchars=\\\{\}]
\PYG{n}{returns\PYGZus{}2020s}\PYG{p}{[}\PYG{l+s+s1}{\PYGZsq{}}\PYG{l+s+s1}{TSLA}\PYG{l+s+s1}{\PYGZsq{}}\PYG{p}{]}\PYG{o}{.}\PYG{n}{pipe}\PYG{p}{(}\PYG{n}{beta}\PYG{p}{)}
\end{sphinxVerbatim}

\end{sphinxuseclass}\end{sphinxVerbatimInput}
\begin{sphinxVerbatimOutput}

\begin{sphinxuseclass}{cell_output}
\begin{sphinxVerbatim}[commandchars=\\\{\}]
1.519561127968873
\end{sphinxVerbatim}

\end{sphinxuseclass}\end{sphinxVerbatimOutput}

\end{sphinxuseclass}

\subsubsection{Guess the Sharpe Ratios for these stocks in the 2020s}
\label{\detokenize{herron_01_practice_04:guess-the-sharpe-ratios-for-these-stocks-in-the-2020s}}

\subsubsection{Guess the market betas for these stocks in the 2020s}
\label{\detokenize{herron_01_practice_04:guess-the-market-betas-for-these-stocks-in-the-2020s}}

\subsubsection{Calculate the Sharpe Ratios for these stocks in the 2020s}
\label{\detokenize{herron_01_practice_04:calculate-the-sharpe-ratios-for-these-stocks-in-the-2020s}}
\begin{sphinxuseclass}{cell}\begin{sphinxVerbatimInput}

\begin{sphinxuseclass}{cell_input}
\begin{sphinxVerbatim}[commandchars=\\\{\}]
\PYG{k}{for} \PYG{n}{i} \PYG{o+ow}{in} \PYG{n}{returns\PYGZus{}2020s}\PYG{p}{:}
    \PYG{n}{sharpe\PYGZus{}i} \PYG{o}{=} \PYG{n}{sharpe}\PYG{p}{(}\PYG{n}{returns\PYGZus{}2020s}\PYG{p}{[}\PYG{n}{i}\PYG{p}{]}\PYG{p}{)}
    \PYG{n+nb}{print}\PYG{p}{(}\PYG{l+s+sa}{f}\PYG{l+s+s1}{\PYGZsq{}}\PYG{l+s+s1}{Sharpe Ratio for }\PYG{l+s+si}{\PYGZob{}}\PYG{n}{i}\PYG{l+s+si}{\PYGZcb{}}\PYG{l+s+s1}{:}\PYG{l+s+se}{\PYGZbs{}t}\PYG{l+s+s1}{ }\PYG{l+s+si}{\PYGZob{}}\PYG{n}{sharpe\PYGZus{}i}\PYG{l+s+si}{:}\PYG{l+s+s1}{0.2f}\PYG{l+s+si}{\PYGZcb{}}\PYG{l+s+s1}{\PYGZsq{}}\PYG{p}{)}
\end{sphinxVerbatim}

\end{sphinxuseclass}\end{sphinxVerbatimInput}
\begin{sphinxVerbatimOutput}

\begin{sphinxuseclass}{cell_output}
\begin{sphinxVerbatim}[commandchars=\\\{\}]
Sharpe Ratio for AAPL:	 0.70
Sharpe Ratio for AMZN:	 0.10
Sharpe Ratio for F:	 0.42
Sharpe Ratio for META:	 \PYGZhy{}0.13
Sharpe Ratio for TSLA:	 1.04
\end{sphinxVerbatim}

\end{sphinxuseclass}\end{sphinxVerbatimOutput}

\end{sphinxuseclass}
\sphinxAtStartPar
We can also use pandas notation to vectorize this calculation.
First calculate \sphinxstyleemphasis{excess} returns as \$R\_i \sphinxhyphen{} R\_f\$.

\begin{sphinxuseclass}{cell}\begin{sphinxVerbatimInput}

\begin{sphinxuseclass}{cell_input}
\begin{sphinxVerbatim}[commandchars=\\\{\}]
\PYG{n}{returns\PYGZus{}2020s\PYGZus{}excess} \PYG{o}{=} \PYG{n}{returns\PYGZus{}2020s}\PYG{o}{.}\PYG{n}{sub}\PYG{p}{(}\PYG{n}{ff}\PYG{p}{[}\PYG{l+s+s1}{\PYGZsq{}}\PYG{l+s+s1}{RF}\PYG{l+s+s1}{\PYGZsq{}}\PYG{p}{]}\PYG{p}{,} \PYG{n}{axis}\PYG{o}{=}\PYG{l+m+mi}{0}\PYG{p}{)}\PYG{o}{.}\PYG{n}{dropna}\PYG{p}{(}\PYG{p}{)}
\PYG{n}{returns\PYGZus{}2020s\PYGZus{}excess}\PYG{o}{.}\PYG{n}{head}\PYG{p}{(}\PYG{p}{)}
\end{sphinxVerbatim}

\end{sphinxuseclass}\end{sphinxVerbatimInput}
\begin{sphinxVerbatimOutput}

\begin{sphinxuseclass}{cell_output}
\begin{sphinxVerbatim}[commandchars=\\\{\}]
Ticker        AAPL    AMZN       F    META   TSLA
Date                                             
2020\PYGZhy{}01\PYGZhy{}02  0.0228  0.0271  0.0128  0.0220 0.0285
2020\PYGZhy{}01\PYGZhy{}03 \PYGZhy{}0.0098 \PYGZhy{}0.0122 \PYGZhy{}0.0224 \PYGZhy{}0.0054 0.0296
2020\PYGZhy{}01\PYGZhy{}06  0.0079  0.0148 \PYGZhy{}0.0055  0.0188 0.0192
2020\PYGZhy{}01\PYGZhy{}07 \PYGZhy{}0.0048  0.0020  0.0098  0.0021 0.0387
2020\PYGZhy{}01\PYGZhy{}08  0.0160 \PYGZhy{}0.0079 \PYGZhy{}0.0001  0.0101 0.0491
\end{sphinxVerbatim}

\end{sphinxuseclass}\end{sphinxVerbatimOutput}

\end{sphinxuseclass}
\sphinxAtStartPar
Then use pandas notation to calculate means, standard deviations, and annualize.

\begin{sphinxuseclass}{cell}\begin{sphinxVerbatimInput}

\begin{sphinxuseclass}{cell_input}
\begin{sphinxVerbatim}[commandchars=\\\{\}]
\PYG{p}{(}
    \PYG{n}{returns\PYGZus{}2020s\PYGZus{}excess}\PYG{o}{.}\PYG{n}{mean}\PYG{p}{(}\PYG{p}{)}
    \PYG{o}{.}\PYG{n}{div}\PYG{p}{(}\PYG{n}{returns\PYGZus{}2020s\PYGZus{}excess}\PYG{o}{.}\PYG{n}{std}\PYG{p}{(}\PYG{p}{)}\PYG{p}{)}
    \PYG{o}{.}\PYG{n}{mul}\PYG{p}{(}\PYG{n}{np}\PYG{o}{.}\PYG{n}{sqrt}\PYG{p}{(}\PYG{l+m+mi}{252}\PYG{p}{)}\PYG{p}{)}
\PYG{p}{)}
\end{sphinxVerbatim}

\end{sphinxuseclass}\end{sphinxVerbatimInput}
\begin{sphinxVerbatimOutput}

\begin{sphinxuseclass}{cell_output}
\begin{sphinxVerbatim}[commandchars=\\\{\}]
Ticker
AAPL    0.7019
AMZN    0.0985
F       0.4199
META   \PYGZhy{}0.1257
TSLA    1.0387
dtype: float64
\end{sphinxVerbatim}

\end{sphinxuseclass}\end{sphinxVerbatimOutput}

\end{sphinxuseclass}
\sphinxAtStartPar
\sphinxstyleemphasis{\sphinxstylestrong{Note:}}
In a few weeks we will learn the \sphinxcode{\sphinxupquote{.apply()}} method, which avoids the loop syntax.

\begin{sphinxuseclass}{cell}\begin{sphinxVerbatimInput}

\begin{sphinxuseclass}{cell_input}
\begin{sphinxVerbatim}[commandchars=\\\{\}]
\PYG{n}{returns\PYGZus{}2020s}\PYG{o}{.}\PYG{n}{apply}\PYG{p}{(}\PYG{n}{sharpe}\PYG{p}{)}
\end{sphinxVerbatim}

\end{sphinxuseclass}\end{sphinxVerbatimInput}
\begin{sphinxVerbatimOutput}

\begin{sphinxuseclass}{cell_output}
\begin{sphinxVerbatim}[commandchars=\\\{\}]
Ticker
AAPL    0.7019
AMZN    0.0985
F       0.4199
META   \PYGZhy{}0.1257
TSLA    1.0387
dtype: float64
\end{sphinxVerbatim}

\end{sphinxuseclass}\end{sphinxVerbatimOutput}

\end{sphinxuseclass}

\subsubsection{Calculate the market betas for these stocks in the 2020s}
\label{\detokenize{herron_01_practice_04:calculate-the-market-betas-for-these-stocks-in-the-2020s}}
\sphinxAtStartPar
We can loop over \sphinxcode{\sphinxupquote{returns\_2020s}}, but a loop solution is tedious.

\begin{sphinxuseclass}{cell}\begin{sphinxVerbatimInput}

\begin{sphinxuseclass}{cell_input}
\begin{sphinxVerbatim}[commandchars=\\\{\}]
\PYG{k}{for} \PYG{n}{i} \PYG{o+ow}{in} \PYG{n}{returns\PYGZus{}2020s}\PYG{p}{:}
    \PYG{n}{beta\PYGZus{}i} \PYG{o}{=} \PYG{n}{beta}\PYG{p}{(}\PYG{n}{returns\PYGZus{}2020s}\PYG{p}{[}\PYG{n}{i}\PYG{p}{]}\PYG{p}{)}
    \PYG{n+nb}{print}\PYG{p}{(}\PYG{l+s+sa}{f}\PYG{l+s+s1}{\PYGZsq{}}\PYG{l+s+s1}{Beta for }\PYG{l+s+si}{\PYGZob{}}\PYG{n}{i}\PYG{l+s+si}{\PYGZcb{}}\PYG{l+s+s1}{:}\PYG{l+s+se}{\PYGZbs{}t}\PYG{l+s+s1}{ }\PYG{l+s+si}{\PYGZob{}}\PYG{n}{beta\PYGZus{}i}\PYG{l+s+si}{:}\PYG{l+s+s1}{0.2f}\PYG{l+s+si}{\PYGZcb{}}\PYG{l+s+s1}{\PYGZsq{}}\PYG{p}{)}
\end{sphinxVerbatim}

\end{sphinxuseclass}\end{sphinxVerbatimInput}
\begin{sphinxVerbatimOutput}

\begin{sphinxuseclass}{cell_output}
\begin{sphinxVerbatim}[commandchars=\\\{\}]
Beta for AAPL:	 1.16
Beta for AMZN:	 1.00
Beta for F:	 1.21
Beta for META:	 1.23
Beta for TSLA:	 1.52
\end{sphinxVerbatim}

\end{sphinxuseclass}\end{sphinxVerbatimOutput}

\end{sphinxuseclass}
\sphinxAtStartPar
Or we can follow out approach above to vectorize this calculation.
First, we need to add a market excess return column to \sphinxcode{\sphinxupquote{returns\_2020s\_excess}}.

\begin{sphinxuseclass}{cell}\begin{sphinxVerbatimInput}

\begin{sphinxuseclass}{cell_input}
\begin{sphinxVerbatim}[commandchars=\\\{\}]
\PYG{n}{returns\PYGZus{}2020s\PYGZus{}excess}\PYG{p}{[}\PYG{l+s+s1}{\PYGZsq{}}\PYG{l+s+s1}{Mkt\PYGZhy{}RF}\PYG{l+s+s1}{\PYGZsq{}}\PYG{p}{]} \PYG{o}{=} \PYG{n}{ff}\PYG{p}{[}\PYG{l+s+s1}{\PYGZsq{}}\PYG{l+s+s1}{Mkt\PYGZhy{}RF}\PYG{l+s+s1}{\PYGZsq{}}\PYG{p}{]}
\PYG{n}{returns\PYGZus{}2020s\PYGZus{}excess}\PYG{o}{.}\PYG{n}{head}\PYG{p}{(}\PYG{p}{)}
\end{sphinxVerbatim}

\end{sphinxuseclass}\end{sphinxVerbatimInput}
\begin{sphinxVerbatimOutput}

\begin{sphinxuseclass}{cell_output}
\begin{sphinxVerbatim}[commandchars=\\\{\}]
Ticker        AAPL    AMZN       F    META   TSLA  Mkt\PYGZhy{}RF
Date                                                     
2020\PYGZhy{}01\PYGZhy{}02  0.0228  0.0271  0.0128  0.0220 0.0285  0.0086
2020\PYGZhy{}01\PYGZhy{}03 \PYGZhy{}0.0098 \PYGZhy{}0.0122 \PYGZhy{}0.0224 \PYGZhy{}0.0054 0.0296 \PYGZhy{}0.0067
2020\PYGZhy{}01\PYGZhy{}06  0.0079  0.0148 \PYGZhy{}0.0055  0.0188 0.0192  0.0036
2020\PYGZhy{}01\PYGZhy{}07 \PYGZhy{}0.0048  0.0020  0.0098  0.0021 0.0387 \PYGZhy{}0.0019
2020\PYGZhy{}01\PYGZhy{}08  0.0160 \PYGZhy{}0.0079 \PYGZhy{}0.0001  0.0101 0.0491  0.0047
\end{sphinxVerbatim}

\end{sphinxuseclass}\end{sphinxVerbatimOutput}

\end{sphinxuseclass}
\begin{sphinxuseclass}{cell}\begin{sphinxVerbatimInput}

\begin{sphinxuseclass}{cell_input}
\begin{sphinxVerbatim}[commandchars=\\\{\}]
\PYG{n}{vcv} \PYG{o}{=} \PYG{n}{returns\PYGZus{}2020s\PYGZus{}excess}\PYG{o}{.}\PYG{n}{cov}\PYG{p}{(}\PYG{p}{)}
\PYG{n}{vcv}\PYG{o}{.}\PYG{n}{head}\PYG{p}{(}\PYG{p}{)}
\end{sphinxVerbatim}

\end{sphinxuseclass}\end{sphinxVerbatimInput}
\begin{sphinxVerbatimOutput}

\begin{sphinxuseclass}{cell_output}
\begin{sphinxVerbatim}[commandchars=\\\{\}]
Ticker   AAPL   AMZN      F   META   TSLA  Mkt\PYGZhy{}RF
Ticker                                           
AAPL   0.0005 0.0004 0.0003 0.0004 0.0005  0.0003
AMZN   0.0004 0.0006 0.0002 0.0005 0.0005  0.0003
F      0.0003 0.0002 0.0010 0.0003 0.0005  0.0003
META   0.0004 0.0005 0.0003 0.0009 0.0005  0.0003
TSLA   0.0005 0.0005 0.0005 0.0005 0.0021  0.0004
\end{sphinxVerbatim}

\end{sphinxuseclass}\end{sphinxVerbatimOutput}

\end{sphinxuseclass}
\begin{sphinxuseclass}{cell}\begin{sphinxVerbatimInput}

\begin{sphinxuseclass}{cell_input}
\begin{sphinxVerbatim}[commandchars=\\\{\}]
\PYG{n}{vcv}\PYG{p}{[}\PYG{l+s+s1}{\PYGZsq{}}\PYG{l+s+s1}{Mkt\PYGZhy{}RF}\PYG{l+s+s1}{\PYGZsq{}}\PYG{p}{]}\PYG{o}{.}\PYG{n}{div}\PYG{p}{(}\PYG{n}{vcv}\PYG{o}{.}\PYG{n}{loc}\PYG{p}{[}\PYG{l+s+s1}{\PYGZsq{}}\PYG{l+s+s1}{Mkt\PYGZhy{}RF}\PYG{l+s+s1}{\PYGZsq{}}\PYG{p}{,} \PYG{l+s+s1}{\PYGZsq{}}\PYG{l+s+s1}{Mkt\PYGZhy{}RF}\PYG{l+s+s1}{\PYGZsq{}}\PYG{p}{]}\PYG{p}{)}\PYG{o}{.}\PYG{n}{plot}\PYG{p}{(}\PYG{n}{kind}\PYG{o}{=}\PYG{l+s+s1}{\PYGZsq{}}\PYG{l+s+s1}{barh}\PYG{l+s+s1}{\PYGZsq{}}\PYG{p}{)}
\PYG{n}{plt}\PYG{o}{.}\PYG{n}{xlabel}\PYG{p}{(}\PYG{l+s+s1}{\PYGZsq{}}\PYG{l+s+s1}{CAPM Beta}\PYG{l+s+s1}{\PYGZsq{}}\PYG{p}{)}
\PYG{n}{plt}\PYG{o}{.}\PYG{n}{title}\PYG{p}{(}\PYG{l+s+s1}{\PYGZsq{}}\PYG{l+s+s1}{CAPM Betas}\PYG{l+s+s1}{\PYGZsq{}}\PYG{p}{)}
\PYG{n}{plt}\PYG{o}{.}\PYG{n}{show}\PYG{p}{(}\PYG{p}{)}
\end{sphinxVerbatim}

\end{sphinxuseclass}\end{sphinxVerbatimInput}
\begin{sphinxVerbatimOutput}

\begin{sphinxuseclass}{cell_output}
\noindent\sphinxincludegraphics{{d524bdf4c2e8921865b8c16e8a2eeb26f16879dafc5aed2f7dd843d939fe75b6}.png}

\end{sphinxuseclass}\end{sphinxVerbatimOutput}

\end{sphinxuseclass}
\sphinxAtStartPar
\sphinxstyleemphasis{\sphinxstylestrong{Note:}}
In a few weeks we will learn the \sphinxcode{\sphinxupquote{.apply()}} method, which avoids the loop syntax.

\begin{sphinxuseclass}{cell}\begin{sphinxVerbatimInput}

\begin{sphinxuseclass}{cell_input}
\begin{sphinxVerbatim}[commandchars=\\\{\}]
\PYG{n}{returns\PYGZus{}2020s}\PYG{o}{.}\PYG{n}{apply}\PYG{p}{(}\PYG{n}{beta}\PYG{p}{)}
\end{sphinxVerbatim}

\end{sphinxuseclass}\end{sphinxVerbatimInput}
\begin{sphinxVerbatimOutput}

\begin{sphinxuseclass}{cell_output}
\begin{sphinxVerbatim}[commandchars=\\\{\}]
Ticker
AAPL   1.1649
AMZN   1.0038
F      1.2078
META   1.2337
TSLA   1.5196
dtype: float64
\end{sphinxVerbatim}

\end{sphinxuseclass}\end{sphinxVerbatimOutput}

\end{sphinxuseclass}

\subsubsection{Calculate the Sharpe Ratio for an \sphinxstyleemphasis{equally weighted} portfolio of these stocks in the 2020s}
\label{\detokenize{herron_01_practice_04:calculate-the-sharpe-ratio-for-an-equally-weighted-portfolio-of-these-stocks-in-the-2020s}}
\begin{sphinxuseclass}{cell}\begin{sphinxVerbatimInput}

\begin{sphinxuseclass}{cell_input}
\begin{sphinxVerbatim}[commandchars=\\\{\}]
\PYG{n}{returns\PYGZus{}2020s}\PYG{o}{.}\PYG{n}{mean}\PYG{p}{(}\PYG{n}{axis}\PYG{o}{=}\PYG{l+m+mi}{1}\PYG{p}{)}\PYG{o}{.}\PYG{n}{pipe}\PYG{p}{(}\PYG{n}{sharpe}\PYG{p}{)}
\end{sphinxVerbatim}

\end{sphinxuseclass}\end{sphinxVerbatimInput}
\begin{sphinxVerbatimOutput}

\begin{sphinxuseclass}{cell_output}
\begin{sphinxVerbatim}[commandchars=\\\{\}]
0.6439557637538056
\end{sphinxVerbatim}

\end{sphinxuseclass}\end{sphinxVerbatimOutput}

\end{sphinxuseclass}
\sphinxAtStartPar
Later, we will use the notation from investments class (i.e., \$R\_P = w\textasciicircum{}T \textbackslash{}bf\{R\}\$), but \sphinxcode{\sphinxupquote{.mean(axis=1)}} is simpler for equally weighted returns.

\begin{sphinxuseclass}{cell}\begin{sphinxVerbatimInput}

\begin{sphinxuseclass}{cell_input}
\begin{sphinxVerbatim}[commandchars=\\\{\}]
\PYG{n}{\PYGZus{}} \PYG{o}{=} \PYG{n}{returns\PYGZus{}2020s}\PYG{o}{.}\PYG{n}{shape}\PYG{p}{[}\PYG{l+m+mi}{1}\PYG{p}{]}
\PYG{n}{weights} \PYG{o}{=} \PYG{n}{np}\PYG{o}{.}\PYG{n}{ones}\PYG{p}{(}\PYG{n}{\PYGZus{}}\PYG{p}{)} \PYG{o}{/} \PYG{n}{\PYGZus{}}
\PYG{n}{weights}
\end{sphinxVerbatim}

\end{sphinxuseclass}\end{sphinxVerbatimInput}
\begin{sphinxVerbatimOutput}

\begin{sphinxuseclass}{cell_output}
\begin{sphinxVerbatim}[commandchars=\\\{\}]
array([0.2, 0.2, 0.2, 0.2, 0.2])
\end{sphinxVerbatim}

\end{sphinxuseclass}\end{sphinxVerbatimOutput}

\end{sphinxuseclass}
\begin{sphinxuseclass}{cell}\begin{sphinxVerbatimInput}

\begin{sphinxuseclass}{cell_input}
\begin{sphinxVerbatim}[commandchars=\\\{\}]
\PYG{n}{np}\PYG{o}{.}\PYG{n}{allclose}\PYG{p}{(}\PYG{n}{returns\PYGZus{}2020s}\PYG{o}{.}\PYG{n}{dot}\PYG{p}{(}\PYG{n}{weights}\PYG{p}{)}\PYG{p}{,} \PYG{n}{returns\PYGZus{}2020s}\PYG{o}{.}\PYG{n}{mean}\PYG{p}{(}\PYG{n}{axis}\PYG{o}{=}\PYG{l+m+mi}{1}\PYG{p}{)}\PYG{p}{)}
\end{sphinxVerbatim}

\end{sphinxuseclass}\end{sphinxVerbatimInput}
\begin{sphinxVerbatimOutput}

\begin{sphinxuseclass}{cell_output}
\begin{sphinxVerbatim}[commandchars=\\\{\}]
True
\end{sphinxVerbatim}

\end{sphinxuseclass}\end{sphinxVerbatimOutput}

\end{sphinxuseclass}

\bigskip\hrule\bigskip


\sphinxAtStartPar
During class someone asked about the portfolio variance notation from investments class (i.e., \$w\textasciicircum{}T \textbackslash{}Sigma w\$).
We typically will not use this formula because we can calculate the portfolio return series with \sphinxcode{\sphinxupquote{returns.dot(weights)}}, then calculate the variance with \sphinxcode{\sphinxupquote{.var()}}.
Here is a comparison.

\begin{sphinxuseclass}{cell}\begin{sphinxVerbatimInput}

\begin{sphinxuseclass}{cell_input}
\begin{sphinxVerbatim}[commandchars=\\\{\}]
\PYG{n}{np}\PYG{o}{.}\PYG{n}{allclose}\PYG{p}{(}
    \PYG{n}{returns\PYGZus{}2020s}\PYG{o}{.}\PYG{n}{cov}\PYG{p}{(}\PYG{p}{)}\PYG{o}{.}\PYG{n}{dot}\PYG{p}{(}\PYG{n}{weights}\PYG{p}{)}\PYG{o}{.}\PYG{n}{dot}\PYG{p}{(}\PYG{n}{weights}\PYG{p}{)}\PYG{p}{,} \PYG{c+c1}{\PYGZsh{} from investments class}
    \PYG{n}{returns\PYGZus{}2020s}\PYG{o}{.}\PYG{n}{mean}\PYG{p}{(}\PYG{n}{axis}\PYG{o}{=}\PYG{l+m+mi}{1}\PYG{p}{)}\PYG{o}{.}\PYG{n}{var}\PYG{p}{(}\PYG{p}{)} \PYG{c+c1}{\PYGZsh{} from this class}
\PYG{p}{)}
\end{sphinxVerbatim}

\end{sphinxuseclass}\end{sphinxVerbatimInput}
\begin{sphinxVerbatimOutput}

\begin{sphinxuseclass}{cell_output}
\begin{sphinxVerbatim}[commandchars=\\\{\}]
True
\end{sphinxVerbatim}

\end{sphinxuseclass}\end{sphinxVerbatimOutput}

\end{sphinxuseclass}

\bigskip\hrule\bigskip


\sphinxAtStartPar
Because diversification reduces portfolio standard deviation less than the sum of its parts, the Sharpe Ratio of the equally weighted portfolio is less than the equally weighted mean of the single\sphinxhyphen{}stock Sharpe Ratios.

\begin{sphinxuseclass}{cell}\begin{sphinxVerbatimInput}

\begin{sphinxuseclass}{cell_input}
\begin{sphinxVerbatim}[commandchars=\\\{\}]
\PYG{n}{returns\PYGZus{}2020s}\PYG{o}{.}\PYG{n}{apply}\PYG{p}{(}\PYG{n}{sharpe}\PYG{p}{)}\PYG{o}{.}\PYG{n}{mean}\PYG{p}{(}\PYG{p}{)}
\end{sphinxVerbatim}

\end{sphinxuseclass}\end{sphinxVerbatimInput}
\begin{sphinxVerbatimOutput}

\begin{sphinxuseclass}{cell_output}
\begin{sphinxVerbatim}[commandchars=\\\{\}]
0.4267
\end{sphinxVerbatim}

\end{sphinxuseclass}\end{sphinxVerbatimOutput}

\end{sphinxuseclass}

\subsubsection{Calculate the market beta for an \sphinxstyleemphasis{equally weighted} portfolio of these stocks in the 2020s}
\label{\detokenize{herron_01_practice_04:calculate-the-market-beta-for-an-equally-weighted-portfolio-of-these-stocks-in-the-2020s}}
\sphinxAtStartPar
Beta measures \sphinxstyleemphasis{non}diversifiable risk, so \$\textbackslash{}beta\_P = \textbackslash{}sum w\_i \textbackslash{}beta\_i\$!

\begin{sphinxuseclass}{cell}\begin{sphinxVerbatimInput}

\begin{sphinxuseclass}{cell_input}
\begin{sphinxVerbatim}[commandchars=\\\{\}]
\PYG{n}{returns\PYGZus{}2020s}\PYG{o}{.}\PYG{n}{mean}\PYG{p}{(}\PYG{n}{axis}\PYG{o}{=}\PYG{l+m+mi}{1}\PYG{p}{)}\PYG{o}{.}\PYG{n}{pipe}\PYG{p}{(}\PYG{n}{beta}\PYG{p}{)}
\end{sphinxVerbatim}

\end{sphinxuseclass}\end{sphinxVerbatimInput}
\begin{sphinxVerbatimOutput}

\begin{sphinxuseclass}{cell_output}
\begin{sphinxVerbatim}[commandchars=\\\{\}]
1.2259364290960117
\end{sphinxVerbatim}

\end{sphinxuseclass}\end{sphinxVerbatimOutput}

\end{sphinxuseclass}
\begin{sphinxuseclass}{cell}\begin{sphinxVerbatimInput}

\begin{sphinxuseclass}{cell_input}
\begin{sphinxVerbatim}[commandchars=\\\{\}]
\PYG{n}{returns\PYGZus{}2020s}\PYG{o}{.}\PYG{n}{apply}\PYG{p}{(}\PYG{n}{beta}\PYG{p}{)}\PYG{o}{.}\PYG{n}{mean}\PYG{p}{(}\PYG{p}{)}
\end{sphinxVerbatim}

\end{sphinxuseclass}\end{sphinxVerbatimInput}
\begin{sphinxVerbatimOutput}

\begin{sphinxuseclass}{cell_output}
\begin{sphinxVerbatim}[commandchars=\\\{\}]
1.2259
\end{sphinxVerbatim}

\end{sphinxuseclass}\end{sphinxVerbatimOutput}

\end{sphinxuseclass}

\subsubsection{Calculate the market betas for these stocks every calendar year for every possible year}
\label{\detokenize{herron_01_practice_04:calculate-the-market-betas-for-these-stocks-every-calendar-year-for-every-possible-year}}
\sphinxAtStartPar
Save these market betas to data frame \sphinxcode{\sphinxupquote{betas}}.
Our current Python knowledge limits us to a for\sphinxhyphen{}loop, but we will learn easier and faster approaches soon!

\begin{sphinxuseclass}{cell}\begin{sphinxVerbatimInput}

\begin{sphinxuseclass}{cell_input}
\begin{sphinxVerbatim}[commandchars=\\\{\}]
\PYG{n}{betas} \PYG{o}{=} \PYG{p}{[}\PYG{p}{]}
\PYG{n}{years} \PYG{o}{=} \PYG{n+nb}{list}\PYG{p}{(}\PYG{n+nb}{range}\PYG{p}{(}\PYG{l+m+mi}{1973}\PYG{p}{,} \PYG{l+m+mi}{2023}\PYG{p}{)}\PYG{p}{)}
\PYG{k}{for} \PYG{n}{year} \PYG{o+ow}{in} \PYG{n}{years}\PYG{p}{:}
    \PYG{n}{betas}\PYG{o}{.}\PYG{n}{append}\PYG{p}{(}\PYG{n}{returns}\PYG{o}{.}\PYG{n}{loc}\PYG{p}{[}\PYG{n+nb}{str}\PYG{p}{(}\PYG{n}{year}\PYG{p}{)}\PYG{p}{]}\PYG{o}{.}\PYG{n}{apply}\PYG{p}{(}\PYG{n}{beta}\PYG{p}{)}\PYG{p}{)}
    
\PYG{n}{betas} \PYG{o}{=} \PYG{n}{pd}\PYG{o}{.}\PYG{n}{DataFrame}\PYG{p}{(}\PYG{n}{betas}\PYG{p}{,} \PYG{n}{index}\PYG{o}{=}\PYG{n}{years}\PYG{p}{)}
\PYG{n}{betas}\PYG{o}{.}\PYG{n}{columns}\PYG{o}{.}\PYG{n}{name} \PYG{o}{=} \PYG{l+s+s1}{\PYGZsq{}}\PYG{l+s+s1}{Ticker}\PYG{l+s+s1}{\PYGZsq{}}
\PYG{n}{betas}\PYG{o}{.}\PYG{n}{index}\PYG{o}{.}\PYG{n}{name} \PYG{o}{=} \PYG{l+s+s1}{\PYGZsq{}}\PYG{l+s+s1}{Year}\PYG{l+s+s1}{\PYGZsq{}}
\PYG{n}{betas}\PYG{o}{.}\PYG{n}{head}\PYG{p}{(}\PYG{p}{)}
\end{sphinxVerbatim}

\end{sphinxuseclass}\end{sphinxVerbatimInput}
\begin{sphinxVerbatimOutput}

\begin{sphinxuseclass}{cell_output}
\begin{sphinxVerbatim}[commandchars=\\\{\}]
Ticker  AAPL  AMZN      F  META  TSLA
Year                                 
1973     NaN   NaN 1.2621   NaN   NaN
1974     NaN   NaN 0.9625   NaN   NaN
1975     NaN   NaN 1.0576   NaN   NaN
1976     NaN   NaN 1.3623   NaN   NaN
1977     NaN   NaN 1.2652   NaN   NaN
\end{sphinxVerbatim}

\end{sphinxuseclass}\end{sphinxVerbatimOutput}

\end{sphinxuseclass}

\subsubsection{Plot the time series of market betas}
\label{\detokenize{herron_01_practice_04:plot-the-time-series-of-market-betas}}
\begin{sphinxuseclass}{cell}\begin{sphinxVerbatimInput}

\begin{sphinxuseclass}{cell_input}
\begin{sphinxVerbatim}[commandchars=\\\{\}]
\PYG{n}{betas}\PYG{o}{.}\PYG{n}{plot}\PYG{p}{(}\PYG{p}{)}
\PYG{n}{plt}\PYG{o}{.}\PYG{n}{ylabel}\PYG{p}{(}\PYG{l+s+s1}{\PYGZsq{}}\PYG{l+s+s1}{CAPM Beta}\PYG{l+s+s1}{\PYGZsq{}}\PYG{p}{)}
\PYG{n}{plt}\PYG{o}{.}\PYG{n}{title}\PYG{p}{(}\PYG{l+s+s1}{\PYGZsq{}}\PYG{l+s+s1}{CAPM Betas}\PYG{l+s+s1}{\PYGZsq{}}\PYG{p}{)}
\PYG{n}{plt}\PYG{o}{.}\PYG{n}{show}\PYG{p}{(}\PYG{p}{)}
\end{sphinxVerbatim}

\end{sphinxuseclass}\end{sphinxVerbatimInput}
\begin{sphinxVerbatimOutput}

\begin{sphinxuseclass}{cell_output}
\noindent\sphinxincludegraphics{{d94114b910b94e99ef2d660eac2fad2e98f74d666877f39a8d7ab07192471265}.png}

\end{sphinxuseclass}\end{sphinxVerbatimOutput}

\end{sphinxuseclass}
\sphinxstepscope

\begin{sphinxuseclass}{cell}\begin{sphinxVerbatimInput}

\begin{sphinxuseclass}{cell_input}
\begin{sphinxVerbatim}[commandchars=\\\{\}]
\PYG{c+c1}{\PYGZsh{} \PYGZpc{}pip install yfinance pandas\PYGZhy{}datareader requests\PYGZhy{}cache}
\end{sphinxVerbatim}

\end{sphinxuseclass}\end{sphinxVerbatimInput}

\end{sphinxuseclass}

\section{Herron Topic 1 \sphinxhyphen{} Practice (Section 2, Wednesday 2:45 PM)}
\label{\detokenize{herron_01_practice_02:herron-topic-1-practice-section-2-wednesday-2-45-pm}}\label{\detokenize{herron_01_practice_02::doc}}

\subsection{Announcements}
\label{\detokenize{herron_01_practice_02:announcements}}\begin{itemize}
\item {} 
\sphinxAtStartPar
Quiz 2 \sphinxhyphen{} mean was \$\textbackslash{}approx 90\%\$

\item {} 
\sphinxAtStartPar
Quiz 3 \sphinxhyphen{} due by 11:59 on Friday, 2/10

\item {} 
\sphinxAtStartPar
Project groups open on Canvas under People \sphinxhyphen{} please sign up!

\item {} 
\sphinxAtStartPar
Optional, anonymous survey on Canvas under “Quizzes” \sphinxhyphen{} I value your feedback

\end{itemize}


\subsection{Practice}
\label{\detokenize{herron_01_practice_02:practice}}
\sphinxAtStartPar
On Discovery, we need to install the following pacakges every time we log in:

\begin{sphinxuseclass}{cell}\begin{sphinxVerbatimInput}

\begin{sphinxuseclass}{cell_input}
\begin{sphinxVerbatim}[commandchars=\\\{\}]
\PYG{c+c1}{\PYGZsh{} \PYGZpc{}pip install yfinance pandas\PYGZhy{}datareader requests\PYGZhy{}cache}
\end{sphinxVerbatim}

\end{sphinxuseclass}\end{sphinxVerbatimInput}

\end{sphinxuseclass}
\begin{sphinxuseclass}{cell}\begin{sphinxVerbatimInput}

\begin{sphinxuseclass}{cell_input}
\begin{sphinxVerbatim}[commandchars=\\\{\}]
\PYG{k+kn}{import} \PYG{n+nn}{matplotlib}\PYG{n+nn}{.}\PYG{n+nn}{pyplot} \PYG{k}{as} \PYG{n+nn}{plt}
\PYG{k+kn}{import} \PYG{n+nn}{numpy} \PYG{k}{as} \PYG{n+nn}{np}
\PYG{k+kn}{import} \PYG{n+nn}{pandas} \PYG{k}{as} \PYG{n+nn}{pd}

\PYG{k+kn}{import} \PYG{n+nn}{yfinance} \PYG{k}{as} \PYG{n+nn}{yf}
\PYG{k+kn}{import} \PYG{n+nn}{pandas\PYGZus{}datareader} \PYG{k}{as} \PYG{n+nn}{pdr}
\PYG{k+kn}{import} \PYG{n+nn}{requests\PYGZus{}cache}
\end{sphinxVerbatim}

\end{sphinxuseclass}\end{sphinxVerbatimInput}

\end{sphinxuseclass}
\begin{sphinxuseclass}{cell}\begin{sphinxVerbatimInput}

\begin{sphinxuseclass}{cell_input}
\begin{sphinxVerbatim}[commandchars=\\\{\}]
\PYG{o}{\PYGZpc{}}\PYG{k}{config} InlineBackend.figure\PYGZus{}format = \PYGZsq{}retina\PYGZsq{}
\PYG{o}{\PYGZpc{}}\PYG{k}{precision} 4
\PYG{n}{pd}\PYG{o}{.}\PYG{n}{options}\PYG{o}{.}\PYG{n}{display}\PYG{o}{.}\PYG{n}{float\PYGZus{}format} \PYG{o}{=} \PYG{l+s+s1}{\PYGZsq{}}\PYG{l+s+si}{\PYGZob{}:.4f\PYGZcb{}}\PYG{l+s+s1}{\PYGZsq{}}\PYG{o}{.}\PYG{n}{format}
\PYG{n}{session} \PYG{o}{=} \PYG{n}{requests\PYGZus{}cache}\PYG{o}{.}\PYG{n}{CachedSession}\PYG{p}{(}\PYG{n}{expire\PYGZus{}after}\PYG{o}{=}\PYG{l+m+mi}{1}\PYG{p}{)}
\end{sphinxVerbatim}

\end{sphinxuseclass}\end{sphinxVerbatimInput}

\end{sphinxuseclass}

\subsubsection{Download all available daily price data for tickers TSLA, F, AAPL, AMZN, and META to data frame \sphinxstyleliteralintitle{\sphinxupquote{histories}}}
\label{\detokenize{herron_01_practice_02:download-all-available-daily-price-data-for-tickers-tsla-f-aapl-amzn-and-meta-to-data-frame-histories}}
\sphinxAtStartPar
Remove time zone information from the index and use \sphinxcode{\sphinxupquote{histories.columns.names}} to label the variables and tickers as \sphinxcode{\sphinxupquote{Variable}} and \sphinxcode{\sphinxupquote{Ticker}}.

\begin{sphinxuseclass}{cell}\begin{sphinxVerbatimInput}

\begin{sphinxuseclass}{cell_input}
\begin{sphinxVerbatim}[commandchars=\\\{\}]
\PYG{n}{tickers} \PYG{o}{=} \PYG{n}{yf}\PYG{o}{.}\PYG{n}{Tickers}\PYG{p}{(}\PYG{n}{tickers}\PYG{o}{=}\PYG{l+s+s1}{\PYGZsq{}}\PYG{l+s+s1}{TSLA F AAPL AMZN META}\PYG{l+s+s1}{\PYGZsq{}}\PYG{p}{,} \PYG{n}{session}\PYG{o}{=}\PYG{n}{session}\PYG{p}{)}
\PYG{n}{histories} \PYG{o}{=} \PYG{n}{tickers}\PYG{o}{.}\PYG{n}{history}\PYG{p}{(}\PYG{n}{period}\PYG{o}{=}\PYG{l+s+s1}{\PYGZsq{}}\PYG{l+s+s1}{max}\PYG{l+s+s1}{\PYGZsq{}}\PYG{p}{,} \PYG{n}{auto\PYGZus{}adjust}\PYG{o}{=}\PYG{k+kc}{False}\PYG{p}{,} \PYG{n}{progress}\PYG{o}{=}\PYG{k+kc}{False}\PYG{p}{)}
\PYG{n}{histories}\PYG{o}{.}\PYG{n}{index} \PYG{o}{=} \PYG{n}{histories}\PYG{o}{.}\PYG{n}{index}\PYG{o}{.}\PYG{n}{tz\PYGZus{}localize}\PYG{p}{(}\PYG{k+kc}{None}\PYG{p}{)}
\PYG{n}{histories}\PYG{o}{.}\PYG{n}{columns}\PYG{o}{.}\PYG{n}{names} \PYG{o}{=} \PYG{p}{[}\PYG{l+s+s1}{\PYGZsq{}}\PYG{l+s+s1}{Variable}\PYG{l+s+s1}{\PYGZsq{}}\PYG{p}{,} \PYG{l+s+s1}{\PYGZsq{}}\PYG{l+s+s1}{Ticker}\PYG{l+s+s1}{\PYGZsq{}}\PYG{p}{]}
\PYG{n}{histories}\PYG{o}{.}\PYG{n}{head}\PYG{p}{(}\PYG{p}{)}
\end{sphinxVerbatim}

\end{sphinxuseclass}\end{sphinxVerbatimInput}
\begin{sphinxVerbatimOutput}

\begin{sphinxuseclass}{cell_output}
\begin{sphinxVerbatim}[commandchars=\\\{\}]
[*********************100\PYGZpc{}***********************]  5 of 5 completed
\end{sphinxVerbatim}

\begin{sphinxVerbatim}[commandchars=\\\{\}]
Variable   Adj Close                       Close                        ...  \PYGZbs{}
Ticker          AAPL AMZN      F META TSLA  AAPL AMZN      F META TSLA  ...   
Date                                                                    ...   
1972\PYGZhy{}06\PYGZhy{}01       NaN  NaN 0.2673  NaN  NaN   NaN  NaN 2.1532  NaN  NaN  ...   
1972\PYGZhy{}06\PYGZhy{}02       NaN  NaN 0.2668  NaN  NaN   NaN  NaN 2.1492  NaN  NaN  ...   
1972\PYGZhy{}06\PYGZhy{}05       NaN  NaN 0.2668  NaN  NaN   NaN  NaN 2.1492  NaN  NaN  ...   
1972\PYGZhy{}06\PYGZhy{}06       NaN  NaN 0.2638  NaN  NaN   NaN  NaN 2.1248  NaN  NaN  ...   
1972\PYGZhy{}06\PYGZhy{}07       NaN  NaN 0.2623  NaN  NaN   NaN  NaN 2.1127  NaN  NaN  ...   

Variable   Stock Splits                       Volume                          
Ticker             AAPL AMZN      F META TSLA   AAPL AMZN        F META TSLA  
Date                                                                          
1972\PYGZhy{}06\PYGZhy{}01          NaN  NaN 0.0000  NaN  NaN    NaN  NaN  1091238  NaN  NaN  
1972\PYGZhy{}06\PYGZhy{}02          NaN  NaN 0.0000  NaN  NaN    NaN  NaN  1174468  NaN  NaN  
1972\PYGZhy{}06\PYGZhy{}05          NaN  NaN 0.0000  NaN  NaN    NaN  NaN  5209582  NaN  NaN  
1972\PYGZhy{}06\PYGZhy{}06          NaN  NaN 0.0000  NaN  NaN    NaN  NaN  1424158  NaN  NaN  
1972\PYGZhy{}06\PYGZhy{}07          NaN  NaN 0.0000  NaN  NaN    NaN  NaN   675088  NaN  NaN  

[5 rows x 40 columns]
\end{sphinxVerbatim}

\end{sphinxuseclass}\end{sphinxVerbatimOutput}

\end{sphinxuseclass}

\subsubsection{Calculate all available daily returns and save to data frame \sphinxstyleliteralintitle{\sphinxupquote{returns}}}
\label{\detokenize{herron_01_practice_02:calculate-all-available-daily-returns-and-save-to-data-frame-returns}}
\begin{sphinxuseclass}{cell}\begin{sphinxVerbatimInput}

\begin{sphinxuseclass}{cell_input}
\begin{sphinxVerbatim}[commandchars=\\\{\}]
\PYG{n}{returns} \PYG{o}{=} \PYG{n}{histories}\PYG{p}{[}\PYG{l+s+s1}{\PYGZsq{}}\PYG{l+s+s1}{Adj Close}\PYG{l+s+s1}{\PYGZsq{}}\PYG{p}{]}\PYG{o}{.}\PYG{n}{pct\PYGZus{}change}\PYG{p}{(}\PYG{p}{)}
\PYG{n}{returns}\PYG{o}{.}\PYG{n}{head}\PYG{p}{(}\PYG{p}{)}
\end{sphinxVerbatim}

\end{sphinxuseclass}\end{sphinxVerbatimInput}
\begin{sphinxVerbatimOutput}

\begin{sphinxuseclass}{cell_output}
\begin{sphinxVerbatim}[commandchars=\\\{\}]
Ticker      AAPL  AMZN       F  META  TSLA
Date                                      
1972\PYGZhy{}06\PYGZhy{}01   NaN   NaN     NaN   NaN   NaN
1972\PYGZhy{}06\PYGZhy{}02   NaN   NaN \PYGZhy{}0.0019   NaN   NaN
1972\PYGZhy{}06\PYGZhy{}05   NaN   NaN  0.0000   NaN   NaN
1972\PYGZhy{}06\PYGZhy{}06   NaN   NaN \PYGZhy{}0.0113   NaN   NaN
1972\PYGZhy{}06\PYGZhy{}07   NaN   NaN \PYGZhy{}0.0057   NaN   NaN
\end{sphinxVerbatim}

\end{sphinxuseclass}\end{sphinxVerbatimOutput}

\end{sphinxuseclass}

\subsubsection{Slices returns for the 2020s and assign to \sphinxstyleliteralintitle{\sphinxupquote{returns\_2020s}}}
\label{\detokenize{herron_01_practice_02:slices-returns-for-the-2020s-and-assign-to-returns-2020s}}
\begin{sphinxuseclass}{cell}\begin{sphinxVerbatimInput}

\begin{sphinxuseclass}{cell_input}
\begin{sphinxVerbatim}[commandchars=\\\{\}]
\PYG{n}{returns\PYGZus{}2020s} \PYG{o}{=} \PYG{n}{returns}\PYG{o}{.}\PYG{n}{loc}\PYG{p}{[}\PYG{l+s+s1}{\PYGZsq{}}\PYG{l+s+s1}{2020}\PYG{l+s+s1}{\PYGZsq{}}\PYG{p}{:}\PYG{p}{]}
\PYG{n}{returns\PYGZus{}2020s}\PYG{o}{.}\PYG{n}{head}\PYG{p}{(}\PYG{p}{)}
\end{sphinxVerbatim}

\end{sphinxuseclass}\end{sphinxVerbatimInput}
\begin{sphinxVerbatimOutput}

\begin{sphinxuseclass}{cell_output}
\begin{sphinxVerbatim}[commandchars=\\\{\}]
Ticker        AAPL    AMZN       F    META   TSLA
Date                                             
2020\PYGZhy{}01\PYGZhy{}02  0.0228  0.0272  0.0129  0.0221 0.0285
2020\PYGZhy{}01\PYGZhy{}03 \PYGZhy{}0.0097 \PYGZhy{}0.0121 \PYGZhy{}0.0223 \PYGZhy{}0.0053 0.0296
2020\PYGZhy{}01\PYGZhy{}06  0.0080  0.0149 \PYGZhy{}0.0054  0.0188 0.0193
2020\PYGZhy{}01\PYGZhy{}07 \PYGZhy{}0.0047  0.0021  0.0098  0.0022 0.0388
2020\PYGZhy{}01\PYGZhy{}08  0.0161 \PYGZhy{}0.0078  0.0000  0.0101 0.0492
\end{sphinxVerbatim}

\end{sphinxuseclass}\end{sphinxVerbatimOutput}

\end{sphinxuseclass}

\subsubsection{Download all available data for the Fama and French daily benchmark factors to dictionary \sphinxstyleliteralintitle{\sphinxupquote{ff\_all}}}
\label{\detokenize{herron_01_practice_02:download-all-available-data-for-the-fama-and-french-daily-benchmark-factors-to-dictionary-ff-all}}
\sphinxAtStartPar
I often use the following code snippet to find the exact name for the the daily benchmark factors file.

\begin{sphinxuseclass}{cell}\begin{sphinxVerbatimInput}

\begin{sphinxuseclass}{cell_input}
\begin{sphinxVerbatim}[commandchars=\\\{\}]
\PYG{n}{pdr}\PYG{o}{.}\PYG{n}{famafrench}\PYG{o}{.}\PYG{n}{get\PYGZus{}available\PYGZus{}datasets}\PYG{p}{(}\PYG{p}{)}\PYG{p}{[}\PYG{p}{:}\PYG{l+m+mi}{5}\PYG{p}{]}
\end{sphinxVerbatim}

\end{sphinxuseclass}\end{sphinxVerbatimInput}
\begin{sphinxVerbatimOutput}

\begin{sphinxuseclass}{cell_output}
\begin{sphinxVerbatim}[commandchars=\\\{\}]
[\PYGZsq{}F\PYGZhy{}F\PYGZus{}Research\PYGZus{}Data\PYGZus{}Factors\PYGZsq{},
 \PYGZsq{}F\PYGZhy{}F\PYGZus{}Research\PYGZus{}Data\PYGZus{}Factors\PYGZus{}weekly\PYGZsq{},
 \PYGZsq{}F\PYGZhy{}F\PYGZus{}Research\PYGZus{}Data\PYGZus{}Factors\PYGZus{}daily\PYGZsq{},
 \PYGZsq{}F\PYGZhy{}F\PYGZus{}Research\PYGZus{}Data\PYGZus{}5\PYGZus{}Factors\PYGZus{}2x3\PYGZsq{},
 \PYGZsq{}F\PYGZhy{}F\PYGZus{}Research\PYGZus{}Data\PYGZus{}5\PYGZus{}Factors\PYGZus{}2x3\PYGZus{}daily\PYGZsq{}]
\end{sphinxVerbatim}

\end{sphinxuseclass}\end{sphinxVerbatimOutput}

\end{sphinxuseclass}
\sphinxAtStartPar
Then I copy\sphinxhyphen{}and\sphinxhyphen{}paste that file name into \sphinxcode{\sphinxupquote{pdr.DataReader()}}.

\begin{sphinxuseclass}{cell}\begin{sphinxVerbatimInput}

\begin{sphinxuseclass}{cell_input}
\begin{sphinxVerbatim}[commandchars=\\\{\}]
\PYG{n}{ff\PYGZus{}all} \PYG{o}{=} \PYG{n}{pdr}\PYG{o}{.}\PYG{n}{DataReader}\PYG{p}{(}
    \PYG{n}{name}\PYG{o}{=}\PYG{l+s+s1}{\PYGZsq{}}\PYG{l+s+s1}{F\PYGZhy{}F\PYGZus{}Research\PYGZus{}Data\PYGZus{}Factors\PYGZus{}daily}\PYG{l+s+s1}{\PYGZsq{}}\PYG{p}{,}
    \PYG{n}{data\PYGZus{}source}\PYG{o}{=}\PYG{l+s+s1}{\PYGZsq{}}\PYG{l+s+s1}{famafrench}\PYG{l+s+s1}{\PYGZsq{}}\PYG{p}{,}
    \PYG{n}{start}\PYG{o}{=}\PYG{l+s+s1}{\PYGZsq{}}\PYG{l+s+s1}{1900}\PYG{l+s+s1}{\PYGZsq{}}\PYG{p}{,}
    \PYG{n}{session}\PYG{o}{=}\PYG{n}{session}
\PYG{p}{)}
\end{sphinxVerbatim}

\end{sphinxuseclass}\end{sphinxVerbatimInput}

\end{sphinxuseclass}

\subsubsection{Slice the daily benchmark factors, convert them to decimal returns, and assign to \sphinxstyleliteralintitle{\sphinxupquote{ff}}}
\label{\detokenize{herron_01_practice_02:slice-the-daily-benchmark-factors-convert-them-to-decimal-returns-and-assign-to-ff}}
\begin{sphinxuseclass}{cell}\begin{sphinxVerbatimInput}

\begin{sphinxuseclass}{cell_input}
\begin{sphinxVerbatim}[commandchars=\\\{\}]
\PYG{n}{ff} \PYG{o}{=} \PYG{n}{ff\PYGZus{}all}\PYG{p}{[}\PYG{l+m+mi}{0}\PYG{p}{]}\PYG{o}{.}\PYG{n}{div}\PYG{p}{(}\PYG{l+m+mi}{100}\PYG{p}{)}
\PYG{n}{ff}\PYG{o}{.}\PYG{n}{head}\PYG{p}{(}\PYG{p}{)}
\end{sphinxVerbatim}

\end{sphinxuseclass}\end{sphinxVerbatimInput}
\begin{sphinxVerbatimOutput}

\begin{sphinxuseclass}{cell_output}
\begin{sphinxVerbatim}[commandchars=\\\{\}]
            Mkt\PYGZhy{}RF     SMB     HML     RF
Date                                     
1926\PYGZhy{}07\PYGZhy{}01  0.0010 \PYGZhy{}0.0025 \PYGZhy{}0.0027 0.0001
1926\PYGZhy{}07\PYGZhy{}02  0.0045 \PYGZhy{}0.0033 \PYGZhy{}0.0006 0.0001
1926\PYGZhy{}07\PYGZhy{}06  0.0017  0.0030 \PYGZhy{}0.0039 0.0001
1926\PYGZhy{}07\PYGZhy{}07  0.0009 \PYGZhy{}0.0058  0.0002 0.0001
1926\PYGZhy{}07\PYGZhy{}08  0.0021 \PYGZhy{}0.0038  0.0019 0.0001
\end{sphinxVerbatim}

\end{sphinxuseclass}\end{sphinxVerbatimOutput}

\end{sphinxuseclass}

\subsubsection{Use the \sphinxstyleliteralintitle{\sphinxupquote{.cumprod()}} method to plot cumulative returns for these stocks in the 2020s}
\label{\detokenize{herron_01_practice_02:use-the-cumprod-method-to-plot-cumulative-returns-for-these-stocks-in-the-2020s}}
\begin{sphinxuseclass}{cell}\begin{sphinxVerbatimInput}

\begin{sphinxuseclass}{cell_input}
\begin{sphinxVerbatim}[commandchars=\\\{\}]
\PYG{n}{cumret\PYGZus{}simple} \PYG{o}{=} \PYG{n}{returns\PYGZus{}2020s}\PYG{o}{.}\PYG{n}{add}\PYG{p}{(}\PYG{l+m+mi}{1}\PYG{p}{)}\PYG{o}{.}\PYG{n}{cumprod}\PYG{p}{(}\PYG{p}{)}\PYG{o}{.}\PYG{n}{sub}\PYG{p}{(}\PYG{l+m+mi}{1}\PYG{p}{)}\PYG{o}{.}\PYG{n}{mul}\PYG{p}{(}\PYG{l+m+mi}{100}\PYG{p}{)}
\end{sphinxVerbatim}

\end{sphinxuseclass}\end{sphinxVerbatimInput}

\end{sphinxuseclass}
\begin{sphinxuseclass}{cell}\begin{sphinxVerbatimInput}

\begin{sphinxuseclass}{cell_input}
\begin{sphinxVerbatim}[commandchars=\\\{\}]
\PYG{n}{cumret\PYGZus{}simple}\PYG{o}{.}\PYG{n}{plot}\PYG{p}{(}\PYG{p}{)}
\PYG{n}{plt}\PYG{o}{.}\PYG{n}{ylabel}\PYG{p}{(}\PYG{l+s+s1}{\PYGZsq{}}\PYG{l+s+s1}{Cumulative Return (}\PYG{l+s+s1}{\PYGZpc{}}\PYG{l+s+s1}{)}\PYG{l+s+s1}{\PYGZsq{}}\PYG{p}{)}
\PYG{n}{plt}\PYG{o}{.}\PYG{n}{title}\PYG{p}{(}\PYG{l+s+s1}{\PYGZsq{}}\PYG{l+s+s1}{Cumulative Returns}\PYG{l+s+s1}{\PYGZsq{}}\PYG{p}{)}
\PYG{n}{plt}\PYG{o}{.}\PYG{n}{show}\PYG{p}{(}\PYG{p}{)}
\end{sphinxVerbatim}

\end{sphinxuseclass}\end{sphinxVerbatimInput}
\begin{sphinxVerbatimOutput}

\begin{sphinxuseclass}{cell_output}
\noindent\sphinxincludegraphics{{a31f25ccc622788178cc4171eb517ac321ffbf3fb37cd0eaafdb85d13d771518}.png}

\end{sphinxuseclass}\end{sphinxVerbatimOutput}

\end{sphinxuseclass}

\subsubsection{Use the \sphinxstyleliteralintitle{\sphinxupquote{.cumsum()}} method with log returns to plot cumulative returns for these stocks in the 2020s}
\label{\detokenize{herron_01_practice_02:use-the-cumsum-method-with-log-returns-to-plot-cumulative-returns-for-these-stocks-in-the-2020s}}
\begin{sphinxuseclass}{cell}\begin{sphinxVerbatimInput}

\begin{sphinxuseclass}{cell_input}
\begin{sphinxVerbatim}[commandchars=\\\{\}]
\PYG{n}{cumret\PYGZus{}log} \PYG{o}{=} \PYG{p}{(}
    \PYG{n}{returns\PYGZus{}2020s}
    \PYG{o}{.}\PYG{n}{add}\PYG{p}{(}\PYG{l+m+mi}{1}\PYG{p}{)}
    \PYG{o}{.}\PYG{n}{pipe}\PYG{p}{(}\PYG{n}{np}\PYG{o}{.}\PYG{n}{log}\PYG{p}{)} \PYG{c+c1}{\PYGZsh{} creates log returns}
    \PYG{o}{.}\PYG{n}{cumsum}\PYG{p}{(}\PYG{p}{)} \PYG{c+c1}{\PYGZsh{} compounds log returns by summing}
    \PYG{o}{.}\PYG{n}{pipe}\PYG{p}{(}\PYG{n}{np}\PYG{o}{.}\PYG{n}{exp}\PYG{p}{)} 
    \PYG{o}{.}\PYG{n}{sub}\PYG{p}{(}\PYG{l+m+mi}{1}\PYG{p}{)} \PYG{c+c1}{\PYGZsh{} converts back to simple returns}
    \PYG{o}{.}\PYG{n}{mul}\PYG{p}{(}\PYG{l+m+mi}{100}\PYG{p}{)} \PYG{c+c1}{\PYGZsh{} converts from decimal to percent}
\PYG{p}{)}
\end{sphinxVerbatim}

\end{sphinxuseclass}\end{sphinxVerbatimInput}

\end{sphinxuseclass}
\begin{sphinxuseclass}{cell}\begin{sphinxVerbatimInput}

\begin{sphinxuseclass}{cell_input}
\begin{sphinxVerbatim}[commandchars=\\\{\}]
\PYG{n}{cumret\PYGZus{}log}\PYG{o}{.}\PYG{n}{plot}\PYG{p}{(}\PYG{p}{)}
\PYG{n}{plt}\PYG{o}{.}\PYG{n}{ylabel}\PYG{p}{(}\PYG{l+s+s1}{\PYGZsq{}}\PYG{l+s+s1}{Cumulative Return (}\PYG{l+s+s1}{\PYGZpc{}}\PYG{l+s+s1}{)}\PYG{l+s+s1}{\PYGZsq{}}\PYG{p}{)}
\PYG{n}{plt}\PYG{o}{.}\PYG{n}{title}\PYG{p}{(}\PYG{l+s+s1}{\PYGZsq{}}\PYG{l+s+s1}{Cumulative Returns}\PYG{l+s+s1}{\PYGZsq{}}\PYG{p}{)}
\PYG{n}{plt}\PYG{o}{.}\PYG{n}{show}\PYG{p}{(}\PYG{p}{)}
\end{sphinxVerbatim}

\end{sphinxuseclass}\end{sphinxVerbatimInput}
\begin{sphinxVerbatimOutput}

\begin{sphinxuseclass}{cell_output}
\noindent\sphinxincludegraphics{{a31f25ccc622788178cc4171eb517ac321ffbf3fb37cd0eaafdb85d13d771518}.png}

\end{sphinxuseclass}\end{sphinxVerbatimOutput}

\end{sphinxuseclass}
\sphinxAtStartPar
The \sphinxcode{\sphinxupquote{.cumprod()}} and  \sphinxcode{\sphinxupquote{.cumsum()}} solutions are the same!

\begin{sphinxuseclass}{cell}\begin{sphinxVerbatimInput}

\begin{sphinxuseclass}{cell_input}
\begin{sphinxVerbatim}[commandchars=\\\{\}]
\PYG{n}{np}\PYG{o}{.}\PYG{n}{allclose}\PYG{p}{(}\PYG{n}{cumret\PYGZus{}log}\PYG{p}{,} \PYG{n}{cumret\PYGZus{}simple}\PYG{p}{)}
\end{sphinxVerbatim}

\end{sphinxuseclass}\end{sphinxVerbatimInput}
\begin{sphinxVerbatimOutput}

\begin{sphinxuseclass}{cell_output}
\begin{sphinxVerbatim}[commandchars=\\\{\}]
True
\end{sphinxVerbatim}

\end{sphinxuseclass}\end{sphinxVerbatimOutput}

\end{sphinxuseclass}

\subsubsection{Use price data only to plot cumulative returns for these stocks in the 2020s}
\label{\detokenize{herron_01_practice_02:use-price-data-only-to-plot-cumulative-returns-for-these-stocks-in-the-2020s}}
\sphinxAtStartPar
We can also calculate cumulative returns as the ratio of adjusted closed.
That is \$R\_\{0,T\} = \textbackslash{}frac\{AC\_T\}\{AC\_0\} \sphinxhyphen{} 1\$.

\begin{sphinxuseclass}{cell}\begin{sphinxVerbatimInput}

\begin{sphinxuseclass}{cell_input}
\begin{sphinxVerbatim}[commandchars=\\\{\}]
\PYG{n}{cumret\PYGZus{}prices} \PYG{o}{=} \PYG{p}{(}
    \PYG{n}{histories}
    \PYG{o}{.}\PYG{n}{loc}\PYG{p}{[}\PYG{l+s+s1}{\PYGZsq{}}\PYG{l+s+s1}{2020}\PYG{l+s+s1}{\PYGZsq{}}\PYG{p}{:}\PYG{p}{,} \PYG{l+s+s1}{\PYGZsq{}}\PYG{l+s+s1}{Adj Close}\PYG{l+s+s1}{\PYGZsq{}}\PYG{p}{]}
    \PYG{o}{.}\PYG{n}{div}\PYG{p}{(}
        \PYG{n}{histories}
        \PYG{o}{.}\PYG{n}{loc}\PYG{p}{[}\PYG{l+s+s1}{\PYGZsq{}}\PYG{l+s+s1}{2019}\PYG{l+s+s1}{\PYGZsq{}}\PYG{p}{,} \PYG{l+s+s1}{\PYGZsq{}}\PYG{l+s+s1}{Adj Close}\PYG{l+s+s1}{\PYGZsq{}}\PYG{p}{]}
        \PYG{o}{.}\PYG{n}{iloc}\PYG{p}{[}\PYG{o}{\PYGZhy{}}\PYG{l+m+mi}{1}\PYG{p}{]}
    \PYG{p}{)}
    \PYG{o}{.}\PYG{n}{sub}\PYG{p}{(}\PYG{l+m+mi}{1}\PYG{p}{)}
    \PYG{o}{.}\PYG{n}{mul}\PYG{p}{(}\PYG{l+m+mi}{100}\PYG{p}{)}
\PYG{p}{)}
\end{sphinxVerbatim}

\end{sphinxuseclass}\end{sphinxVerbatimInput}

\end{sphinxuseclass}
\begin{sphinxuseclass}{cell}\begin{sphinxVerbatimInput}

\begin{sphinxuseclass}{cell_input}
\begin{sphinxVerbatim}[commandchars=\\\{\}]
\PYG{n}{cumret\PYGZus{}prices}\PYG{o}{.}\PYG{n}{plot}\PYG{p}{(}\PYG{p}{)}
\PYG{n}{plt}\PYG{o}{.}\PYG{n}{ylabel}\PYG{p}{(}\PYG{l+s+s1}{\PYGZsq{}}\PYG{l+s+s1}{Cumulative Return (}\PYG{l+s+s1}{\PYGZpc{}}\PYG{l+s+s1}{)}\PYG{l+s+s1}{\PYGZsq{}}\PYG{p}{)}
\PYG{n}{plt}\PYG{o}{.}\PYG{n}{title}\PYG{p}{(}\PYG{l+s+s1}{\PYGZsq{}}\PYG{l+s+s1}{Cumulative Returns}\PYG{l+s+s1}{\PYGZsq{}}\PYG{p}{)}
\PYG{n}{plt}\PYG{o}{.}\PYG{n}{show}\PYG{p}{(}\PYG{p}{)}
\end{sphinxVerbatim}

\end{sphinxuseclass}\end{sphinxVerbatimInput}
\begin{sphinxVerbatimOutput}

\begin{sphinxuseclass}{cell_output}
\noindent\sphinxincludegraphics{{a31f25ccc622788178cc4171eb517ac321ffbf3fb37cd0eaafdb85d13d771518}.png}

\end{sphinxuseclass}\end{sphinxVerbatimOutput}

\end{sphinxuseclass}
\begin{sphinxuseclass}{cell}\begin{sphinxVerbatimInput}

\begin{sphinxuseclass}{cell_input}
\begin{sphinxVerbatim}[commandchars=\\\{\}]
\PYG{n}{np}\PYG{o}{.}\PYG{n}{allclose}\PYG{p}{(}\PYG{n}{cumret\PYGZus{}prices}\PYG{p}{,} \PYG{n}{cumret\PYGZus{}simple}\PYG{p}{)}
\end{sphinxVerbatim}

\end{sphinxuseclass}\end{sphinxVerbatimInput}
\begin{sphinxVerbatimOutput}

\begin{sphinxuseclass}{cell_output}
\begin{sphinxVerbatim}[commandchars=\\\{\}]
True
\end{sphinxVerbatim}

\end{sphinxuseclass}\end{sphinxVerbatimOutput}

\end{sphinxuseclass}

\subsubsection{Calculate the Sharpe Ratio for TSLA}
\label{\detokenize{herron_01_practice_02:calculate-the-sharpe-ratio-for-tsla}}
\sphinxAtStartPar
Calculate the Sharpe Ratio with all available returns and 2020s returns.
Recall the Sharpe Ratio is \$\textbackslash{}frac\{\textbackslash{}overline\{R\_i \sphinxhyphen{} R\_f\}\}\{\textbackslash{}sigma\_i\}\$, where \$\textbackslash{}sigma\_i\$ is the volatility of \sphinxstyleemphasis{excess} returns.

\sphinxAtStartPar
\sphinxstyleemphasis{\sphinxstylestrong{I suggest you write a function named \sphinxcode{\sphinxupquote{sharpe()}} to use for the rest of this notebook.}}

\begin{sphinxuseclass}{cell}\begin{sphinxVerbatimInput}

\begin{sphinxuseclass}{cell_input}
\begin{sphinxVerbatim}[commandchars=\\\{\}]
\PYG{k}{def} \PYG{n+nf}{sharpe}\PYG{p}{(}\PYG{n}{ri}\PYG{p}{,} \PYG{n}{rf}\PYG{o}{=}\PYG{n}{ff}\PYG{p}{[}\PYG{l+s+s1}{\PYGZsq{}}\PYG{l+s+s1}{RF}\PYG{l+s+s1}{\PYGZsq{}}\PYG{p}{]}\PYG{p}{,} \PYG{n}{ann\PYGZus{}fac}\PYG{o}{=}\PYG{n}{np}\PYG{o}{.}\PYG{n}{sqrt}\PYG{p}{(}\PYG{l+m+mi}{252}\PYG{p}{)}\PYG{p}{)}\PYG{p}{:}
    \PYG{n}{ri\PYGZus{}rf} \PYG{o}{=} \PYG{n}{ri}\PYG{o}{.}\PYG{n}{sub}\PYG{p}{(}\PYG{n}{rf}\PYG{p}{)}\PYG{o}{.}\PYG{n}{dropna}\PYG{p}{(}\PYG{p}{)}
    \PYG{k}{return} \PYG{n}{ann\PYGZus{}fac} \PYG{o}{*} \PYG{n}{ri\PYGZus{}rf}\PYG{o}{.}\PYG{n}{mean}\PYG{p}{(}\PYG{p}{)} \PYG{o}{/} \PYG{n}{ri\PYGZus{}rf}\PYG{o}{.}\PYG{n}{std}\PYG{p}{(}\PYG{p}{)}
\end{sphinxVerbatim}

\end{sphinxuseclass}\end{sphinxVerbatimInput}

\end{sphinxuseclass}
\begin{sphinxuseclass}{cell}\begin{sphinxVerbatimInput}

\begin{sphinxuseclass}{cell_input}
\begin{sphinxVerbatim}[commandchars=\\\{\}]
\PYG{n}{sharpe}\PYG{p}{(}\PYG{n}{returns\PYGZus{}2020s}\PYG{p}{[}\PYG{l+s+s1}{\PYGZsq{}}\PYG{l+s+s1}{TSLA}\PYG{l+s+s1}{\PYGZsq{}}\PYG{p}{]}\PYG{p}{)}
\end{sphinxVerbatim}

\end{sphinxuseclass}\end{sphinxVerbatimInput}
\begin{sphinxVerbatimOutput}

\begin{sphinxuseclass}{cell_output}
\begin{sphinxVerbatim}[commandchars=\\\{\}]
1.0386986513828833
\end{sphinxVerbatim}

\end{sphinxuseclass}\end{sphinxVerbatimOutput}

\end{sphinxuseclass}
\sphinxAtStartPar
We can use the \sphinxcode{\sphinxupquote{.pipe()}} method to chain the previous calculation.

\begin{sphinxuseclass}{cell}\begin{sphinxVerbatimInput}

\begin{sphinxuseclass}{cell_input}
\begin{sphinxVerbatim}[commandchars=\\\{\}]
\PYG{n}{returns\PYGZus{}2020s}\PYG{p}{[}\PYG{l+s+s1}{\PYGZsq{}}\PYG{l+s+s1}{TSLA}\PYG{l+s+s1}{\PYGZsq{}}\PYG{p}{]}\PYG{o}{.}\PYG{n}{pipe}\PYG{p}{(}\PYG{n}{sharpe}\PYG{p}{)}
\end{sphinxVerbatim}

\end{sphinxuseclass}\end{sphinxVerbatimInput}
\begin{sphinxVerbatimOutput}

\begin{sphinxuseclass}{cell_output}
\begin{sphinxVerbatim}[commandchars=\\\{\}]
1.0386986513828833
\end{sphinxVerbatim}

\end{sphinxuseclass}\end{sphinxVerbatimOutput}

\end{sphinxuseclass}

\subsubsection{Calculate the market beta for TSLA}
\label{\detokenize{herron_01_practice_02:calculate-the-market-beta-for-tsla}}
\sphinxAtStartPar
Calculate the market beta with all available returns and 2020s returns.
Recall we estimate market beta with the ordinary least squares (OLS) regression \$R\_i\sphinxhyphen{}R\_f = \textbackslash{}alpha + \textbackslash{}beta (R\_m\sphinxhyphen{}R\_f) + \textbackslash{}epsilon\$.
We can estimate market beta with the covariance formula (i.e., \$\textbackslash{}beta\_i = \textbackslash{}frac\{Cov(R\_i\sphinxhyphen{}R\_f, R\_m\sphinxhyphen{}R\_f)\}\{Var(R\_m\sphinxhyphen{}R\_f)\}\$) for a univariate regression if we do not need goodness of fit statistics.

\sphinxAtStartPar
\sphinxstyleemphasis{\sphinxstylestrong{I suggest you write a function named \sphinxcode{\sphinxupquote{beta()}} to use for the rest of this notebook.}}

\begin{sphinxuseclass}{cell}\begin{sphinxVerbatimInput}

\begin{sphinxuseclass}{cell_input}
\begin{sphinxVerbatim}[commandchars=\\\{\}]
\PYG{k}{def} \PYG{n+nf}{beta}\PYG{p}{(}\PYG{n}{ri}\PYG{p}{,} \PYG{n}{rf}\PYG{o}{=}\PYG{n}{ff}\PYG{p}{[}\PYG{l+s+s1}{\PYGZsq{}}\PYG{l+s+s1}{RF}\PYG{l+s+s1}{\PYGZsq{}}\PYG{p}{]}\PYG{p}{,} \PYG{n}{rm\PYGZus{}rf}\PYG{o}{=}\PYG{n}{ff}\PYG{p}{[}\PYG{l+s+s1}{\PYGZsq{}}\PYG{l+s+s1}{Mkt\PYGZhy{}RF}\PYG{l+s+s1}{\PYGZsq{}}\PYG{p}{]}\PYG{p}{)}\PYG{p}{:}
    \PYG{n}{ri\PYGZus{}rf} \PYG{o}{=} \PYG{n}{ri}\PYG{o}{.}\PYG{n}{sub}\PYG{p}{(}\PYG{n}{rf}\PYG{p}{)}\PYG{o}{.}\PYG{n}{dropna}\PYG{p}{(}\PYG{p}{)}
    \PYG{k}{return} \PYG{n}{ri\PYGZus{}rf}\PYG{o}{.}\PYG{n}{cov}\PYG{p}{(}\PYG{n}{rm\PYGZus{}rf}\PYG{p}{)} \PYG{o}{/} \PYG{n}{rm\PYGZus{}rf}\PYG{o}{.}\PYG{n}{loc}\PYG{p}{[}\PYG{n}{ri\PYGZus{}rf}\PYG{o}{.}\PYG{n}{index}\PYG{p}{]}\PYG{o}{.}\PYG{n}{var}\PYG{p}{(}\PYG{p}{)}
\end{sphinxVerbatim}

\end{sphinxuseclass}\end{sphinxVerbatimInput}

\end{sphinxuseclass}
\begin{sphinxuseclass}{cell}\begin{sphinxVerbatimInput}

\begin{sphinxuseclass}{cell_input}
\begin{sphinxVerbatim}[commandchars=\\\{\}]
\PYG{n}{beta}\PYG{p}{(}\PYG{n}{returns\PYGZus{}2020s}\PYG{p}{[}\PYG{l+s+s1}{\PYGZsq{}}\PYG{l+s+s1}{TSLA}\PYG{l+s+s1}{\PYGZsq{}}\PYG{p}{]}\PYG{p}{)}
\end{sphinxVerbatim}

\end{sphinxuseclass}\end{sphinxVerbatimInput}
\begin{sphinxVerbatimOutput}

\begin{sphinxuseclass}{cell_output}
\begin{sphinxVerbatim}[commandchars=\\\{\}]
1.519561127968873
\end{sphinxVerbatim}

\end{sphinxuseclass}\end{sphinxVerbatimOutput}

\end{sphinxuseclass}
\sphinxAtStartPar
Again, we can \sphinxcode{\sphinxupquote{.pipe()}} this calculation.

\begin{sphinxuseclass}{cell}\begin{sphinxVerbatimInput}

\begin{sphinxuseclass}{cell_input}
\begin{sphinxVerbatim}[commandchars=\\\{\}]
\PYG{n}{returns\PYGZus{}2020s}\PYG{p}{[}\PYG{l+s+s1}{\PYGZsq{}}\PYG{l+s+s1}{TSLA}\PYG{l+s+s1}{\PYGZsq{}}\PYG{p}{]}\PYG{o}{.}\PYG{n}{pipe}\PYG{p}{(}\PYG{n}{beta}\PYG{p}{)}
\end{sphinxVerbatim}

\end{sphinxuseclass}\end{sphinxVerbatimInput}
\begin{sphinxVerbatimOutput}

\begin{sphinxuseclass}{cell_output}
\begin{sphinxVerbatim}[commandchars=\\\{\}]
1.519561127968873
\end{sphinxVerbatim}

\end{sphinxuseclass}\end{sphinxVerbatimOutput}

\end{sphinxuseclass}

\subsubsection{Guess the Sharpe Ratios for these stocks in the 2020s}
\label{\detokenize{herron_01_practice_02:guess-the-sharpe-ratios-for-these-stocks-in-the-2020s}}

\subsubsection{Guess the market betas for these stocks in the 2020s}
\label{\detokenize{herron_01_practice_02:guess-the-market-betas-for-these-stocks-in-the-2020s}}

\subsubsection{Calculate the Sharpe Ratios for these stocks in the 2020s}
\label{\detokenize{herron_01_practice_02:calculate-the-sharpe-ratios-for-these-stocks-in-the-2020s}}
\begin{sphinxuseclass}{cell}\begin{sphinxVerbatimInput}

\begin{sphinxuseclass}{cell_input}
\begin{sphinxVerbatim}[commandchars=\\\{\}]
\PYG{k}{for} \PYG{n}{i} \PYG{o+ow}{in} \PYG{n}{returns\PYGZus{}2020s}\PYG{p}{:}
    \PYG{n+nb}{print}\PYG{p}{(}\PYG{l+s+sa}{f}\PYG{l+s+s1}{\PYGZsq{}}\PYG{l+s+s1}{Sharpe Ratio for }\PYG{l+s+si}{\PYGZob{}}\PYG{n}{i}\PYG{l+s+si}{\PYGZcb{}}\PYG{l+s+s1}{: }\PYG{l+s+se}{\PYGZbs{}t}\PYG{l+s+si}{\PYGZob{}}\PYG{n}{returns\PYGZus{}2020s}\PYG{p}{[}\PYG{n}{i}\PYG{p}{]}\PYG{o}{.}\PYG{n}{pipe}\PYG{p}{(}\PYG{n}{sharpe}\PYG{p}{)}\PYG{l+s+si}{:}\PYG{l+s+s1}{0.4f}\PYG{l+s+si}{\PYGZcb{}}\PYG{l+s+s1}{\PYGZsq{}}\PYG{p}{)}
\end{sphinxVerbatim}

\end{sphinxuseclass}\end{sphinxVerbatimInput}
\begin{sphinxVerbatimOutput}

\begin{sphinxuseclass}{cell_output}
\begin{sphinxVerbatim}[commandchars=\\\{\}]
Sharpe Ratio for AAPL: 	0.7019
Sharpe Ratio for AMZN: 	0.0985
Sharpe Ratio for F: 	0.4199
Sharpe Ratio for META: 	\PYGZhy{}0.1257
Sharpe Ratio for TSLA: 	1.0387
\end{sphinxVerbatim}

\end{sphinxuseclass}\end{sphinxVerbatimOutput}

\end{sphinxuseclass}
\begin{sphinxuseclass}{cell}\begin{sphinxVerbatimInput}

\begin{sphinxuseclass}{cell_input}
\begin{sphinxVerbatim}[commandchars=\\\{\}]
\PYG{n}{returns\PYGZus{}2020s}\PYG{o}{.}\PYG{n}{apply}\PYG{p}{(}\PYG{n}{sharpe}\PYG{p}{)}
\end{sphinxVerbatim}

\end{sphinxuseclass}\end{sphinxVerbatimInput}
\begin{sphinxVerbatimOutput}

\begin{sphinxuseclass}{cell_output}
\begin{sphinxVerbatim}[commandchars=\\\{\}]
Ticker
AAPL    0.7019
AMZN    0.0985
F       0.4199
META   \PYGZhy{}0.1257
TSLA    1.0387
dtype: float64
\end{sphinxVerbatim}

\end{sphinxuseclass}\end{sphinxVerbatimOutput}

\end{sphinxuseclass}
\sphinxAtStartPar
We can also use pandas notation to vectorize this calculation.
First calculate \sphinxstyleemphasis{excess} returns as \$R\_i \sphinxhyphen{} R\_f\$.

\begin{sphinxuseclass}{cell}\begin{sphinxVerbatimInput}

\begin{sphinxuseclass}{cell_input}
\begin{sphinxVerbatim}[commandchars=\\\{\}]
\PYG{n}{returns\PYGZus{}2020s\PYGZus{}excess} \PYG{o}{=} \PYG{n}{returns\PYGZus{}2020s}\PYG{o}{.}\PYG{n}{sub}\PYG{p}{(}\PYG{n}{ff}\PYG{p}{[}\PYG{l+s+s1}{\PYGZsq{}}\PYG{l+s+s1}{RF}\PYG{l+s+s1}{\PYGZsq{}}\PYG{p}{]}\PYG{p}{,} \PYG{n}{axis}\PYG{o}{=}\PYG{l+m+mi}{0}\PYG{p}{)}\PYG{o}{.}\PYG{n}{dropna}\PYG{p}{(}\PYG{p}{)}
\PYG{n}{returns\PYGZus{}2020s\PYGZus{}excess}\PYG{o}{.}\PYG{n}{head}\PYG{p}{(}\PYG{p}{)}
\end{sphinxVerbatim}

\end{sphinxuseclass}\end{sphinxVerbatimInput}
\begin{sphinxVerbatimOutput}

\begin{sphinxuseclass}{cell_output}
\begin{sphinxVerbatim}[commandchars=\\\{\}]
Ticker        AAPL    AMZN       F    META   TSLA
Date                                             
2020\PYGZhy{}01\PYGZhy{}02  0.0228  0.0271  0.0128  0.0220 0.0285
2020\PYGZhy{}01\PYGZhy{}03 \PYGZhy{}0.0098 \PYGZhy{}0.0122 \PYGZhy{}0.0224 \PYGZhy{}0.0054 0.0296
2020\PYGZhy{}01\PYGZhy{}06  0.0079  0.0148 \PYGZhy{}0.0055  0.0188 0.0192
2020\PYGZhy{}01\PYGZhy{}07 \PYGZhy{}0.0048  0.0020  0.0098  0.0021 0.0387
2020\PYGZhy{}01\PYGZhy{}08  0.0160 \PYGZhy{}0.0079 \PYGZhy{}0.0001  0.0101 0.0491
\end{sphinxVerbatim}

\end{sphinxuseclass}\end{sphinxVerbatimOutput}

\end{sphinxuseclass}
\sphinxAtStartPar
Then use pandas notation to calculate means, standard deviations, and annualize.

\begin{sphinxuseclass}{cell}\begin{sphinxVerbatimInput}

\begin{sphinxuseclass}{cell_input}
\begin{sphinxVerbatim}[commandchars=\\\{\}]
\PYG{p}{(}
    \PYG{n}{returns\PYGZus{}2020s\PYGZus{}excess}\PYG{o}{.}\PYG{n}{mean}\PYG{p}{(}\PYG{p}{)}
    \PYG{o}{.}\PYG{n}{div}\PYG{p}{(}\PYG{n}{returns\PYGZus{}2020s\PYGZus{}excess}\PYG{o}{.}\PYG{n}{std}\PYG{p}{(}\PYG{p}{)}\PYG{p}{)}
    \PYG{o}{.}\PYG{n}{mul}\PYG{p}{(}\PYG{n}{np}\PYG{o}{.}\PYG{n}{sqrt}\PYG{p}{(}\PYG{l+m+mi}{252}\PYG{p}{)}\PYG{p}{)}
\PYG{p}{)}
\end{sphinxVerbatim}

\end{sphinxuseclass}\end{sphinxVerbatimInput}
\begin{sphinxVerbatimOutput}

\begin{sphinxuseclass}{cell_output}
\begin{sphinxVerbatim}[commandchars=\\\{\}]
Ticker
AAPL    0.7019
AMZN    0.0985
F       0.4199
META   \PYGZhy{}0.1257
TSLA    1.0387
dtype: float64
\end{sphinxVerbatim}

\end{sphinxuseclass}\end{sphinxVerbatimOutput}

\end{sphinxuseclass}

\subsubsection{Calculate the market betas for these stocks in the 2020s}
\label{\detokenize{herron_01_practice_02:calculate-the-market-betas-for-these-stocks-in-the-2020s}}
\sphinxAtStartPar
We can loop over \sphinxcode{\sphinxupquote{returns\_2020s}}, but a loop solution is tedious.

\begin{sphinxuseclass}{cell}\begin{sphinxVerbatimInput}

\begin{sphinxuseclass}{cell_input}
\begin{sphinxVerbatim}[commandchars=\\\{\}]
\PYG{k}{for} \PYG{n}{i} \PYG{o+ow}{in} \PYG{n}{returns\PYGZus{}2020s}\PYG{p}{:}
    \PYG{n+nb}{print}\PYG{p}{(}\PYG{l+s+sa}{f}\PYG{l+s+s1}{\PYGZsq{}}\PYG{l+s+s1}{CAPM Beta for }\PYG{l+s+si}{\PYGZob{}}\PYG{n}{i}\PYG{l+s+si}{\PYGZcb{}}\PYG{l+s+s1}{: }\PYG{l+s+se}{\PYGZbs{}t}\PYG{l+s+si}{\PYGZob{}}\PYG{n}{returns\PYGZus{}2020s}\PYG{p}{[}\PYG{n}{i}\PYG{p}{]}\PYG{o}{.}\PYG{n}{pipe}\PYG{p}{(}\PYG{n}{beta}\PYG{p}{)}\PYG{l+s+si}{:}\PYG{l+s+s1}{0.4f}\PYG{l+s+si}{\PYGZcb{}}\PYG{l+s+s1}{\PYGZsq{}}\PYG{p}{)}
\end{sphinxVerbatim}

\end{sphinxuseclass}\end{sphinxVerbatimInput}
\begin{sphinxVerbatimOutput}

\begin{sphinxuseclass}{cell_output}
\begin{sphinxVerbatim}[commandchars=\\\{\}]
CAPM Beta for AAPL: 	1.1649
CAPM Beta for AMZN: 	1.0038
CAPM Beta for F: 	1.2078
CAPM Beta for META: 	1.2337
CAPM Beta for TSLA: 	1.5196
\end{sphinxVerbatim}

\end{sphinxuseclass}\end{sphinxVerbatimOutput}

\end{sphinxuseclass}
\begin{sphinxuseclass}{cell}\begin{sphinxVerbatimInput}

\begin{sphinxuseclass}{cell_input}
\begin{sphinxVerbatim}[commandchars=\\\{\}]
\PYG{n}{returns\PYGZus{}2020s}\PYG{o}{.}\PYG{n}{apply}\PYG{p}{(}\PYG{n}{beta}\PYG{p}{)}
\end{sphinxVerbatim}

\end{sphinxuseclass}\end{sphinxVerbatimInput}
\begin{sphinxVerbatimOutput}

\begin{sphinxuseclass}{cell_output}
\begin{sphinxVerbatim}[commandchars=\\\{\}]
Ticker
AAPL   1.1649
AMZN   1.0038
F      1.2078
META   1.2337
TSLA   1.5196
dtype: float64
\end{sphinxVerbatim}

\end{sphinxuseclass}\end{sphinxVerbatimOutput}

\end{sphinxuseclass}
\sphinxAtStartPar
We can even calculate Sharpe Ratios and betas as the same time with the \sphinxcode{\sphinxupquote{.agg()}} method.
More later in the course!

\begin{sphinxuseclass}{cell}\begin{sphinxVerbatimInput}

\begin{sphinxuseclass}{cell_input}
\begin{sphinxVerbatim}[commandchars=\\\{\}]
\PYG{n}{returns\PYGZus{}2020s}\PYG{o}{.}\PYG{n}{agg}\PYG{p}{(}\PYG{p}{[}\PYG{n}{sharpe}\PYG{p}{,} \PYG{n}{beta}\PYG{p}{]}\PYG{p}{)}
\end{sphinxVerbatim}

\end{sphinxuseclass}\end{sphinxVerbatimInput}
\begin{sphinxVerbatimOutput}

\begin{sphinxuseclass}{cell_output}
\begin{sphinxVerbatim}[commandchars=\\\{\}]
Ticker   AAPL   AMZN      F    META   TSLA
sharpe 0.7019 0.0985 0.4199 \PYGZhy{}0.1257 1.0387
beta   1.1649 1.0038 1.2078  1.2337 1.5196
\end{sphinxVerbatim}

\end{sphinxuseclass}\end{sphinxVerbatimOutput}

\end{sphinxuseclass}
\sphinxAtStartPar
Or we can follow out approach above to vectorize this calculation.
First, we need to add a market excess return column to \sphinxcode{\sphinxupquote{returns\_2020s\_excess}}.

\begin{sphinxuseclass}{cell}\begin{sphinxVerbatimInput}

\begin{sphinxuseclass}{cell_input}
\begin{sphinxVerbatim}[commandchars=\\\{\}]
\PYG{n}{returns\PYGZus{}2020s\PYGZus{}excess}\PYG{p}{[}\PYG{l+s+s1}{\PYGZsq{}}\PYG{l+s+s1}{Mkt\PYGZhy{}RF}\PYG{l+s+s1}{\PYGZsq{}}\PYG{p}{]} \PYG{o}{=} \PYG{n}{ff}\PYG{p}{[}\PYG{l+s+s1}{\PYGZsq{}}\PYG{l+s+s1}{Mkt\PYGZhy{}RF}\PYG{l+s+s1}{\PYGZsq{}}\PYG{p}{]}
\PYG{n}{returns\PYGZus{}2020s\PYGZus{}excess}\PYG{o}{.}\PYG{n}{head}\PYG{p}{(}\PYG{p}{)}
\end{sphinxVerbatim}

\end{sphinxuseclass}\end{sphinxVerbatimInput}
\begin{sphinxVerbatimOutput}

\begin{sphinxuseclass}{cell_output}
\begin{sphinxVerbatim}[commandchars=\\\{\}]
Ticker        AAPL    AMZN       F    META   TSLA  Mkt\PYGZhy{}RF
Date                                                     
2020\PYGZhy{}01\PYGZhy{}02  0.0228  0.0271  0.0128  0.0220 0.0285  0.0086
2020\PYGZhy{}01\PYGZhy{}03 \PYGZhy{}0.0098 \PYGZhy{}0.0122 \PYGZhy{}0.0224 \PYGZhy{}0.0054 0.0296 \PYGZhy{}0.0067
2020\PYGZhy{}01\PYGZhy{}06  0.0079  0.0148 \PYGZhy{}0.0055  0.0188 0.0192  0.0036
2020\PYGZhy{}01\PYGZhy{}07 \PYGZhy{}0.0048  0.0020  0.0098  0.0021 0.0387 \PYGZhy{}0.0019
2020\PYGZhy{}01\PYGZhy{}08  0.0160 \PYGZhy{}0.0079 \PYGZhy{}0.0001  0.0101 0.0491  0.0047
\end{sphinxVerbatim}

\end{sphinxuseclass}\end{sphinxVerbatimOutput}

\end{sphinxuseclass}
\begin{sphinxuseclass}{cell}\begin{sphinxVerbatimInput}

\begin{sphinxuseclass}{cell_input}
\begin{sphinxVerbatim}[commandchars=\\\{\}]
\PYG{n}{vcv} \PYG{o}{=} \PYG{n}{returns\PYGZus{}2020s\PYGZus{}excess}\PYG{o}{.}\PYG{n}{cov}\PYG{p}{(}\PYG{p}{)}
\PYG{n}{vcv}\PYG{o}{.}\PYG{n}{head}\PYG{p}{(}\PYG{p}{)}
\end{sphinxVerbatim}

\end{sphinxuseclass}\end{sphinxVerbatimInput}
\begin{sphinxVerbatimOutput}

\begin{sphinxuseclass}{cell_output}
\begin{sphinxVerbatim}[commandchars=\\\{\}]
Ticker   AAPL   AMZN      F   META   TSLA  Mkt\PYGZhy{}RF
Ticker                                           
AAPL   0.0005 0.0004 0.0003 0.0004 0.0005  0.0003
AMZN   0.0004 0.0006 0.0002 0.0005 0.0005  0.0003
F      0.0003 0.0002 0.0010 0.0003 0.0005  0.0003
META   0.0004 0.0005 0.0003 0.0009 0.0005  0.0003
TSLA   0.0005 0.0005 0.0005 0.0005 0.0021  0.0004
\end{sphinxVerbatim}

\end{sphinxuseclass}\end{sphinxVerbatimOutput}

\end{sphinxuseclass}
\begin{sphinxuseclass}{cell}\begin{sphinxVerbatimInput}

\begin{sphinxuseclass}{cell_input}
\begin{sphinxVerbatim}[commandchars=\\\{\}]
\PYG{n}{vcv}\PYG{p}{[}\PYG{l+s+s1}{\PYGZsq{}}\PYG{l+s+s1}{Mkt\PYGZhy{}RF}\PYG{l+s+s1}{\PYGZsq{}}\PYG{p}{]}\PYG{o}{.}\PYG{n}{div}\PYG{p}{(}\PYG{n}{vcv}\PYG{o}{.}\PYG{n}{loc}\PYG{p}{[}\PYG{l+s+s1}{\PYGZsq{}}\PYG{l+s+s1}{Mkt\PYGZhy{}RF}\PYG{l+s+s1}{\PYGZsq{}}\PYG{p}{,} \PYG{l+s+s1}{\PYGZsq{}}\PYG{l+s+s1}{Mkt\PYGZhy{}RF}\PYG{l+s+s1}{\PYGZsq{}}\PYG{p}{]}\PYG{p}{)}\PYG{o}{.}\PYG{n}{plot}\PYG{p}{(}\PYG{n}{kind}\PYG{o}{=}\PYG{l+s+s1}{\PYGZsq{}}\PYG{l+s+s1}{barh}\PYG{l+s+s1}{\PYGZsq{}}\PYG{p}{)}
\PYG{n}{plt}\PYG{o}{.}\PYG{n}{xlabel}\PYG{p}{(}\PYG{l+s+s1}{\PYGZsq{}}\PYG{l+s+s1}{CAPM Beta}\PYG{l+s+s1}{\PYGZsq{}}\PYG{p}{)}
\PYG{n}{plt}\PYG{o}{.}\PYG{n}{title}\PYG{p}{(}\PYG{l+s+s1}{\PYGZsq{}}\PYG{l+s+s1}{CAPM Betas}\PYG{l+s+s1}{\PYGZsq{}}\PYG{p}{)}
\PYG{n}{plt}\PYG{o}{.}\PYG{n}{show}\PYG{p}{(}\PYG{p}{)}
\end{sphinxVerbatim}

\end{sphinxuseclass}\end{sphinxVerbatimInput}
\begin{sphinxVerbatimOutput}

\begin{sphinxuseclass}{cell_output}
\noindent\sphinxincludegraphics{{d524bdf4c2e8921865b8c16e8a2eeb26f16879dafc5aed2f7dd843d939fe75b6}.png}

\end{sphinxuseclass}\end{sphinxVerbatimOutput}

\end{sphinxuseclass}

\subsubsection{Calculate the Sharpe Ratio for an \sphinxstyleemphasis{equally weighted} portfolio of these stocks in the 2020s}
\label{\detokenize{herron_01_practice_02:calculate-the-sharpe-ratio-for-an-equally-weighted-portfolio-of-these-stocks-in-the-2020s}}
\begin{sphinxuseclass}{cell}\begin{sphinxVerbatimInput}

\begin{sphinxuseclass}{cell_input}
\begin{sphinxVerbatim}[commandchars=\\\{\}]
\PYG{n}{returns\PYGZus{}2020s}\PYG{o}{.}\PYG{n}{mean}\PYG{p}{(}\PYG{n}{axis}\PYG{o}{=}\PYG{l+m+mi}{1}\PYG{p}{)}\PYG{o}{.}\PYG{n}{pipe}\PYG{p}{(}\PYG{n}{sharpe}\PYG{p}{)}
\end{sphinxVerbatim}

\end{sphinxuseclass}\end{sphinxVerbatimInput}
\begin{sphinxVerbatimOutput}

\begin{sphinxuseclass}{cell_output}
\begin{sphinxVerbatim}[commandchars=\\\{\}]
0.643955749303917
\end{sphinxVerbatim}

\end{sphinxuseclass}\end{sphinxVerbatimOutput}

\end{sphinxuseclass}
\sphinxAtStartPar
The Sharpe Ratio of the portfolio increases because diversification decreases the denominator (risk) more than the numerator (return)!

\begin{sphinxuseclass}{cell}\begin{sphinxVerbatimInput}

\begin{sphinxuseclass}{cell_input}
\begin{sphinxVerbatim}[commandchars=\\\{\}]
\PYG{n}{returns\PYGZus{}2020s}\PYG{o}{.}\PYG{n}{apply}\PYG{p}{(}\PYG{n}{sharpe}\PYG{p}{)}\PYG{o}{.}\PYG{n}{mean}\PYG{p}{(}\PYG{p}{)}
\end{sphinxVerbatim}

\end{sphinxuseclass}\end{sphinxVerbatimInput}
\begin{sphinxVerbatimOutput}

\begin{sphinxuseclass}{cell_output}
\begin{sphinxVerbatim}[commandchars=\\\{\}]
0.4267
\end{sphinxVerbatim}

\end{sphinxuseclass}\end{sphinxVerbatimOutput}

\end{sphinxuseclass}

\bigskip\hrule\bigskip


\sphinxAtStartPar
During another class someone asked about the portfolio variance notation from investments class (i.e., \$w\textasciicircum{}T \textbackslash{}Sigma w\$).
We typically will not use this formula because we can calculate the portfolio return series with \sphinxcode{\sphinxupquote{returns.dot(weights)}}, then calculate the variance with \sphinxcode{\sphinxupquote{.var()}}.
Here is a comparison.

\begin{sphinxuseclass}{cell}\begin{sphinxVerbatimInput}

\begin{sphinxuseclass}{cell_input}
\begin{sphinxVerbatim}[commandchars=\\\{\}]
\PYG{n}{\PYGZus{}} \PYG{o}{=} \PYG{n}{returns\PYGZus{}2020s}\PYG{o}{.}\PYG{n}{shape}\PYG{p}{[}\PYG{l+m+mi}{1}\PYG{p}{]}
\PYG{n}{weights} \PYG{o}{=} \PYG{n}{np}\PYG{o}{.}\PYG{n}{ones}\PYG{p}{(}\PYG{n}{\PYGZus{}}\PYG{p}{)} \PYG{o}{/} \PYG{n}{\PYGZus{}}
\end{sphinxVerbatim}

\end{sphinxuseclass}\end{sphinxVerbatimInput}

\end{sphinxuseclass}
\begin{sphinxuseclass}{cell}\begin{sphinxVerbatimInput}

\begin{sphinxuseclass}{cell_input}
\begin{sphinxVerbatim}[commandchars=\\\{\}]
\PYG{n}{np}\PYG{o}{.}\PYG{n}{allclose}\PYG{p}{(}
    \PYG{n}{returns\PYGZus{}2020s}\PYG{o}{.}\PYG{n}{cov}\PYG{p}{(}\PYG{p}{)}\PYG{o}{.}\PYG{n}{dot}\PYG{p}{(}\PYG{n}{weights}\PYG{p}{)}\PYG{o}{.}\PYG{n}{dot}\PYG{p}{(}\PYG{n}{weights}\PYG{p}{)}\PYG{p}{,} \PYG{c+c1}{\PYGZsh{} from investments class}
    \PYG{n}{returns\PYGZus{}2020s}\PYG{o}{.}\PYG{n}{mean}\PYG{p}{(}\PYG{n}{axis}\PYG{o}{=}\PYG{l+m+mi}{1}\PYG{p}{)}\PYG{o}{.}\PYG{n}{var}\PYG{p}{(}\PYG{p}{)} \PYG{c+c1}{\PYGZsh{} from this class}
\PYG{p}{)}
\end{sphinxVerbatim}

\end{sphinxuseclass}\end{sphinxVerbatimInput}
\begin{sphinxVerbatimOutput}

\begin{sphinxuseclass}{cell_output}
\begin{sphinxVerbatim}[commandchars=\\\{\}]
True
\end{sphinxVerbatim}

\end{sphinxuseclass}\end{sphinxVerbatimOutput}

\end{sphinxuseclass}

\bigskip\hrule\bigskip



\subsubsection{Calculate the market beta for an \sphinxstyleemphasis{equally weighted} portfolio of these stocks in the 2020s}
\label{\detokenize{herron_01_practice_02:calculate-the-market-beta-for-an-equally-weighted-portfolio-of-these-stocks-in-the-2020s}}
\sphinxAtStartPar
What do you notice?

\begin{sphinxuseclass}{cell}\begin{sphinxVerbatimInput}

\begin{sphinxuseclass}{cell_input}
\begin{sphinxVerbatim}[commandchars=\\\{\}]
\PYG{n}{returns\PYGZus{}2020s}\PYG{o}{.}\PYG{n}{mean}\PYG{p}{(}\PYG{n}{axis}\PYG{o}{=}\PYG{l+m+mi}{1}\PYG{p}{)}\PYG{o}{.}\PYG{n}{pipe}\PYG{p}{(}\PYG{n}{beta}\PYG{p}{)}
\end{sphinxVerbatim}

\end{sphinxuseclass}\end{sphinxVerbatimInput}
\begin{sphinxVerbatimOutput}

\begin{sphinxuseclass}{cell_output}
\begin{sphinxVerbatim}[commandchars=\\\{\}]
1.2259364793423735
\end{sphinxVerbatim}

\end{sphinxuseclass}\end{sphinxVerbatimOutput}

\end{sphinxuseclass}
\sphinxAtStartPar
The portfolio beta is the mean of the portfolio stock betas!

\begin{sphinxuseclass}{cell}\begin{sphinxVerbatimInput}

\begin{sphinxuseclass}{cell_input}
\begin{sphinxVerbatim}[commandchars=\\\{\}]
\PYG{n}{returns\PYGZus{}2020s}\PYG{o}{.}\PYG{n}{apply}\PYG{p}{(}\PYG{n}{beta}\PYG{p}{)}\PYG{o}{.}\PYG{n}{mean}\PYG{p}{(}\PYG{p}{)}
\end{sphinxVerbatim}

\end{sphinxuseclass}\end{sphinxVerbatimInput}
\begin{sphinxVerbatimOutput}

\begin{sphinxuseclass}{cell_output}
\begin{sphinxVerbatim}[commandchars=\\\{\}]
1.2259
\end{sphinxVerbatim}

\end{sphinxuseclass}\end{sphinxVerbatimOutput}

\end{sphinxuseclass}

\subsubsection{Calculate the market betas for these stocks every calendar year for every possible year}
\label{\detokenize{herron_01_practice_02:calculate-the-market-betas-for-these-stocks-every-calendar-year-for-every-possible-year}}
\sphinxAtStartPar
Save these market betas to data frame \sphinxcode{\sphinxupquote{betas}}.
Our current Python knowledge limits us to a for\sphinxhyphen{}loop, but we will learn easier and faster approaches soon!

\begin{sphinxuseclass}{cell}\begin{sphinxVerbatimInput}

\begin{sphinxuseclass}{cell_input}
\begin{sphinxVerbatim}[commandchars=\\\{\}]
\PYG{n}{betas} \PYG{o}{=} \PYG{p}{[}\PYG{p}{]}
\PYG{n}{years} \PYG{o}{=} \PYG{n+nb}{list}\PYG{p}{(}\PYG{n+nb}{range}\PYG{p}{(}\PYG{l+m+mi}{1973}\PYG{p}{,} \PYG{l+m+mi}{2023}\PYG{p}{)}\PYG{p}{)}
\PYG{k}{for} \PYG{n}{year} \PYG{o+ow}{in} \PYG{n}{years}\PYG{p}{:}
    \PYG{n}{betas}\PYG{o}{.}\PYG{n}{append}\PYG{p}{(}\PYG{n}{returns}\PYG{o}{.}\PYG{n}{loc}\PYG{p}{[}\PYG{n+nb}{str}\PYG{p}{(}\PYG{n}{year}\PYG{p}{)}\PYG{p}{]}\PYG{o}{.}\PYG{n}{apply}\PYG{p}{(}\PYG{n}{beta}\PYG{p}{)}\PYG{p}{)}
    
\PYG{n}{betas} \PYG{o}{=} \PYG{n}{pd}\PYG{o}{.}\PYG{n}{DataFrame}\PYG{p}{(}\PYG{n}{betas}\PYG{p}{,} \PYG{n}{index}\PYG{o}{=}\PYG{n}{years}\PYG{p}{)}
\PYG{n}{betas}\PYG{o}{.}\PYG{n}{columns}\PYG{o}{.}\PYG{n}{name} \PYG{o}{=} \PYG{l+s+s1}{\PYGZsq{}}\PYG{l+s+s1}{Ticker}\PYG{l+s+s1}{\PYGZsq{}}
\PYG{n}{betas}\PYG{o}{.}\PYG{n}{index}\PYG{o}{.}\PYG{n}{name} \PYG{o}{=} \PYG{l+s+s1}{\PYGZsq{}}\PYG{l+s+s1}{Year}\PYG{l+s+s1}{\PYGZsq{}}
\PYG{n}{betas}\PYG{o}{.}\PYG{n}{head}\PYG{p}{(}\PYG{p}{)}
\end{sphinxVerbatim}

\end{sphinxuseclass}\end{sphinxVerbatimInput}
\begin{sphinxVerbatimOutput}

\begin{sphinxuseclass}{cell_output}
\begin{sphinxVerbatim}[commandchars=\\\{\}]
Ticker  AAPL  AMZN      F  META  TSLA
Year                                 
1973     NaN   NaN 1.2621   NaN   NaN
1974     NaN   NaN 0.9625   NaN   NaN
1975     NaN   NaN 1.0576   NaN   NaN
1976     NaN   NaN 1.3623   NaN   NaN
1977     NaN   NaN 1.2652   NaN   NaN
\end{sphinxVerbatim}

\end{sphinxuseclass}\end{sphinxVerbatimOutput}

\end{sphinxuseclass}

\subsubsection{Plot the time series of market betas}
\label{\detokenize{herron_01_practice_02:plot-the-time-series-of-market-betas}}
\begin{sphinxuseclass}{cell}\begin{sphinxVerbatimInput}

\begin{sphinxuseclass}{cell_input}
\begin{sphinxVerbatim}[commandchars=\\\{\}]
\PYG{n}{betas}\PYG{o}{.}\PYG{n}{plot}\PYG{p}{(}\PYG{p}{)}
\PYG{n}{plt}\PYG{o}{.}\PYG{n}{ylabel}\PYG{p}{(}\PYG{l+s+s1}{\PYGZsq{}}\PYG{l+s+s1}{CAPM Beta}\PYG{l+s+s1}{\PYGZsq{}}\PYG{p}{)}
\PYG{n}{plt}\PYG{o}{.}\PYG{n}{title}\PYG{p}{(}\PYG{l+s+s1}{\PYGZsq{}}\PYG{l+s+s1}{CAPM Betas}\PYG{l+s+s1}{\PYGZsq{}}\PYG{p}{)}
\PYG{n}{plt}\PYG{o}{.}\PYG{n}{show}\PYG{p}{(}\PYG{p}{)}
\end{sphinxVerbatim}

\end{sphinxuseclass}\end{sphinxVerbatimInput}
\begin{sphinxVerbatimOutput}

\begin{sphinxuseclass}{cell_output}
\noindent\sphinxincludegraphics{{3b6edf1a2ac8f6a3754b09b614ccf28e78b921343dcac06b0f2dab9539896161}.png}

\end{sphinxuseclass}\end{sphinxVerbatimOutput}

\end{sphinxuseclass}
\sphinxstepscope


\chapter{McKinney Chapter 8 \sphinxhyphen{} Data Wrangling: Join, Combine, and Reshape}
\label{\detokenize{mckinney_08_lecture:mckinney-chapter-8-data-wrangling-join-combine-and-reshape}}\label{\detokenize{mckinney_08_lecture::doc}}

\section{Introduction}
\label{\detokenize{mckinney_08_lecture:introduction}}
\sphinxAtStartPar
Chapter 8 of Wes McKinney’s \sphinxhref{https://wesmckinney.com/book/}{\sphinxstyleemphasis{Python for Data Analysis}} introduces a few important pandas concepts:
\begin{enumerate}
\sphinxsetlistlabels{\arabic}{enumi}{enumii}{}{.}%
\item {} 
\sphinxAtStartPar
Joining or merging is combining 2+ data frames on 1+ indexes or columns into 1 data frame

\item {} 
\sphinxAtStartPar
Reshaping is rearranging data frames so it has fewer columns and more rows (wide to long) or more columns and fewer rows (long to wide); we can also reshape a series to a data frame and vice versa

\end{enumerate}

\sphinxAtStartPar
\sphinxstyleemphasis{\sphinxstylestrong{Note:}}
Indented block quotes are from McKinney unless otherwise indicated.
The section numbers here differ from McKinney because we will only discuss some topics.

\begin{sphinxuseclass}{cell}\begin{sphinxVerbatimInput}

\begin{sphinxuseclass}{cell_input}
\begin{sphinxVerbatim}[commandchars=\\\{\}]
\PYG{k+kn}{import} \PYG{n+nn}{matplotlib}\PYG{n+nn}{.}\PYG{n+nn}{pyplot} \PYG{k}{as} \PYG{n+nn}{plt}
\PYG{k+kn}{import} \PYG{n+nn}{numpy} \PYG{k}{as} \PYG{n+nn}{np}
\PYG{k+kn}{import} \PYG{n+nn}{pandas} \PYG{k}{as} \PYG{n+nn}{pd}
\end{sphinxVerbatim}

\end{sphinxuseclass}\end{sphinxVerbatimInput}

\end{sphinxuseclass}
\begin{sphinxuseclass}{cell}\begin{sphinxVerbatimInput}

\begin{sphinxuseclass}{cell_input}
\begin{sphinxVerbatim}[commandchars=\\\{\}]
\PYG{o}{\PYGZpc{}}\PYG{k}{config} InlineBackend.figure\PYGZus{}format = \PYGZsq{}retina\PYGZsq{}
\PYG{o}{\PYGZpc{}}\PYG{k}{precision} 4
\PYG{n}{pd}\PYG{o}{.}\PYG{n}{options}\PYG{o}{.}\PYG{n}{display}\PYG{o}{.}\PYG{n}{float\PYGZus{}format} \PYG{o}{=} \PYG{l+s+s1}{\PYGZsq{}}\PYG{l+s+si}{\PYGZob{}:.4f\PYGZcb{}}\PYG{l+s+s1}{\PYGZsq{}}\PYG{o}{.}\PYG{n}{format}
\end{sphinxVerbatim}

\end{sphinxuseclass}\end{sphinxVerbatimInput}

\end{sphinxuseclass}
\begin{sphinxuseclass}{cell}\begin{sphinxVerbatimInput}

\begin{sphinxuseclass}{cell_input}
\begin{sphinxVerbatim}[commandchars=\\\{\}]
\PYG{k+kn}{import} \PYG{n+nn}{yfinance} \PYG{k}{as} \PYG{n+nn}{yf}
\PYG{k+kn}{import} \PYG{n+nn}{pandas\PYGZus{}datareader} \PYG{k}{as} \PYG{n+nn}{pdr}
\PYG{k+kn}{import} \PYG{n+nn}{requests\PYGZus{}cache}
\PYG{n}{session} \PYG{o}{=} \PYG{n}{requests\PYGZus{}cache}\PYG{o}{.}\PYG{n}{CachedSession}\PYG{p}{(}\PYG{p}{)}
\end{sphinxVerbatim}

\end{sphinxuseclass}\end{sphinxVerbatimInput}

\end{sphinxuseclass}

\section{Hierarchical Indexing}
\label{\detokenize{mckinney_08_lecture:hierarchical-indexing}}
\sphinxAtStartPar
We need to learn about hierarchical indexing before we learn about combining and reshaping data.
A hierarchical index gives two or more index levels to an axis.
For example, we could index rows by ticker and date.
Or we could index columns by variable and ticker.
Hierarchical indexing helps us work with high\sphinxhyphen{}dimensional data in a low\sphinxhyphen{}dimensional form.

\begin{sphinxuseclass}{cell}\begin{sphinxVerbatimInput}

\begin{sphinxuseclass}{cell_input}
\begin{sphinxVerbatim}[commandchars=\\\{\}]
\PYG{n}{np}\PYG{o}{.}\PYG{n}{random}\PYG{o}{.}\PYG{n}{seed}\PYG{p}{(}\PYG{l+m+mi}{42}\PYG{p}{)}
\PYG{n}{data} \PYG{o}{=} \PYG{n}{pd}\PYG{o}{.}\PYG{n}{Series}\PYG{p}{(}
    \PYG{n}{data}\PYG{o}{=}\PYG{n}{np}\PYG{o}{.}\PYG{n}{random}\PYG{o}{.}\PYG{n}{randn}\PYG{p}{(}\PYG{l+m+mi}{9}\PYG{p}{)}\PYG{p}{,}
    \PYG{n}{index}\PYG{o}{=}\PYG{p}{[}
        \PYG{p}{[}\PYG{l+s+s1}{\PYGZsq{}}\PYG{l+s+s1}{a}\PYG{l+s+s1}{\PYGZsq{}}\PYG{p}{,} \PYG{l+s+s1}{\PYGZsq{}}\PYG{l+s+s1}{a}\PYG{l+s+s1}{\PYGZsq{}}\PYG{p}{,} \PYG{l+s+s1}{\PYGZsq{}}\PYG{l+s+s1}{a}\PYG{l+s+s1}{\PYGZsq{}}\PYG{p}{,} \PYG{l+s+s1}{\PYGZsq{}}\PYG{l+s+s1}{b}\PYG{l+s+s1}{\PYGZsq{}}\PYG{p}{,} \PYG{l+s+s1}{\PYGZsq{}}\PYG{l+s+s1}{b}\PYG{l+s+s1}{\PYGZsq{}}\PYG{p}{,} \PYG{l+s+s1}{\PYGZsq{}}\PYG{l+s+s1}{c}\PYG{l+s+s1}{\PYGZsq{}}\PYG{p}{,} \PYG{l+s+s1}{\PYGZsq{}}\PYG{l+s+s1}{c}\PYG{l+s+s1}{\PYGZsq{}}\PYG{p}{,} \PYG{l+s+s1}{\PYGZsq{}}\PYG{l+s+s1}{d}\PYG{l+s+s1}{\PYGZsq{}}\PYG{p}{,} \PYG{l+s+s1}{\PYGZsq{}}\PYG{l+s+s1}{d}\PYG{l+s+s1}{\PYGZsq{}}\PYG{p}{]}\PYG{p}{,}
        \PYG{p}{[}\PYG{l+m+mi}{1}\PYG{p}{,} \PYG{l+m+mi}{2}\PYG{p}{,} \PYG{l+m+mi}{3}\PYG{p}{,} \PYG{l+m+mi}{1}\PYG{p}{,} \PYG{l+m+mi}{3}\PYG{p}{,} \PYG{l+m+mi}{1}\PYG{p}{,} \PYG{l+m+mi}{2}\PYG{p}{,} \PYG{l+m+mi}{2}\PYG{p}{,} \PYG{l+m+mi}{3}\PYG{p}{]}
    \PYG{p}{]}
\PYG{p}{)}

\PYG{n}{data}
\end{sphinxVerbatim}

\end{sphinxuseclass}\end{sphinxVerbatimInput}
\begin{sphinxVerbatimOutput}

\begin{sphinxuseclass}{cell_output}
\begin{sphinxVerbatim}[commandchars=\\\{\}]
a  1    0.4967
   2   \PYGZhy{}0.1383
   3    0.6477
b  1    1.5230
   3   \PYGZhy{}0.2342
c  1   \PYGZhy{}0.2341
   2    1.5792
d  2    0.7674
   3   \PYGZhy{}0.4695
dtype: float64
\end{sphinxVerbatim}

\end{sphinxuseclass}\end{sphinxVerbatimOutput}

\end{sphinxuseclass}
\sphinxAtStartPar
We can partially index this series to concisely subset data.

\begin{sphinxuseclass}{cell}\begin{sphinxVerbatimInput}

\begin{sphinxuseclass}{cell_input}
\begin{sphinxVerbatim}[commandchars=\\\{\}]
\PYG{n}{data}\PYG{p}{[}\PYG{l+s+s1}{\PYGZsq{}}\PYG{l+s+s1}{b}\PYG{l+s+s1}{\PYGZsq{}}\PYG{p}{]}
\end{sphinxVerbatim}

\end{sphinxuseclass}\end{sphinxVerbatimInput}
\begin{sphinxVerbatimOutput}

\begin{sphinxuseclass}{cell_output}
\begin{sphinxVerbatim}[commandchars=\\\{\}]
1    1.5230
3   \PYGZhy{}0.2342
dtype: float64
\end{sphinxVerbatim}

\end{sphinxuseclass}\end{sphinxVerbatimOutput}

\end{sphinxuseclass}
\begin{sphinxuseclass}{cell}\begin{sphinxVerbatimInput}

\begin{sphinxuseclass}{cell_input}
\begin{sphinxVerbatim}[commandchars=\\\{\}]
\PYG{n}{data}\PYG{p}{[}\PYG{l+s+s1}{\PYGZsq{}}\PYG{l+s+s1}{b}\PYG{l+s+s1}{\PYGZsq{}}\PYG{p}{:}\PYG{l+s+s1}{\PYGZsq{}}\PYG{l+s+s1}{c}\PYG{l+s+s1}{\PYGZsq{}}\PYG{p}{]}
\end{sphinxVerbatim}

\end{sphinxuseclass}\end{sphinxVerbatimInput}
\begin{sphinxVerbatimOutput}

\begin{sphinxuseclass}{cell_output}
\begin{sphinxVerbatim}[commandchars=\\\{\}]
b  1    1.5230
   3   \PYGZhy{}0.2342
c  1   \PYGZhy{}0.2341
   2    1.5792
dtype: float64
\end{sphinxVerbatim}

\end{sphinxuseclass}\end{sphinxVerbatimOutput}

\end{sphinxuseclass}
\begin{sphinxuseclass}{cell}\begin{sphinxVerbatimInput}

\begin{sphinxuseclass}{cell_input}
\begin{sphinxVerbatim}[commandchars=\\\{\}]
\PYG{n}{data}\PYG{o}{.}\PYG{n}{loc}\PYG{p}{[}\PYG{p}{[}\PYG{l+s+s1}{\PYGZsq{}}\PYG{l+s+s1}{b}\PYG{l+s+s1}{\PYGZsq{}}\PYG{p}{,} \PYG{l+s+s1}{\PYGZsq{}}\PYG{l+s+s1}{d}\PYG{l+s+s1}{\PYGZsq{}}\PYG{p}{]}\PYG{p}{]}
\end{sphinxVerbatim}

\end{sphinxuseclass}\end{sphinxVerbatimInput}
\begin{sphinxVerbatimOutput}

\begin{sphinxuseclass}{cell_output}
\begin{sphinxVerbatim}[commandchars=\\\{\}]
b  1    1.5230
   3   \PYGZhy{}0.2342
d  2    0.7674
   3   \PYGZhy{}0.4695
dtype: float64
\end{sphinxVerbatim}

\end{sphinxuseclass}\end{sphinxVerbatimOutput}

\end{sphinxuseclass}
\sphinxAtStartPar
We can subset on the index inner level, too.
Here the first \sphinxcode{\sphinxupquote{:}} slices all values in the outer index.

\begin{sphinxuseclass}{cell}\begin{sphinxVerbatimInput}

\begin{sphinxuseclass}{cell_input}
\begin{sphinxVerbatim}[commandchars=\\\{\}]
\PYG{n}{data}\PYG{o}{.}\PYG{n}{loc}\PYG{p}{[}\PYG{p}{:}\PYG{p}{,} \PYG{l+m+mi}{2}\PYG{p}{]}
\end{sphinxVerbatim}

\end{sphinxuseclass}\end{sphinxVerbatimInput}
\begin{sphinxVerbatimOutput}

\begin{sphinxuseclass}{cell_output}
\begin{sphinxVerbatim}[commandchars=\\\{\}]
a   \PYGZhy{}0.1383
c    1.5792
d    0.7674
dtype: float64
\end{sphinxVerbatim}

\end{sphinxuseclass}\end{sphinxVerbatimOutput}

\end{sphinxuseclass}
\sphinxAtStartPar
Here \sphinxcode{\sphinxupquote{data}} has a stacked format.
For each outer index level (the letters), we have multiple observations based on the inner index level (the numbers).
We can un\sphinxhyphen{}stack \sphinxcode{\sphinxupquote{data}} to convert the inner index level to columns.

\begin{sphinxuseclass}{cell}\begin{sphinxVerbatimInput}

\begin{sphinxuseclass}{cell_input}
\begin{sphinxVerbatim}[commandchars=\\\{\}]
\PYG{n}{data}\PYG{o}{.}\PYG{n}{unstack}\PYG{p}{(}\PYG{p}{)}
\end{sphinxVerbatim}

\end{sphinxuseclass}\end{sphinxVerbatimInput}
\begin{sphinxVerbatimOutput}

\begin{sphinxuseclass}{cell_output}
\begin{sphinxVerbatim}[commandchars=\\\{\}]
        1       2       3
a  0.4967 \PYGZhy{}0.1383  0.6477
b  1.5230     NaN \PYGZhy{}0.2342
c \PYGZhy{}0.2341  1.5792     NaN
d     NaN  0.7674 \PYGZhy{}0.4695
\end{sphinxVerbatim}

\end{sphinxuseclass}\end{sphinxVerbatimOutput}

\end{sphinxuseclass}
\begin{sphinxuseclass}{cell}\begin{sphinxVerbatimInput}

\begin{sphinxuseclass}{cell_input}
\begin{sphinxVerbatim}[commandchars=\\\{\}]
\PYG{n}{data}\PYG{o}{.}\PYG{n}{unstack}\PYG{p}{(}\PYG{p}{)}\PYG{o}{.}\PYG{n}{stack}\PYG{p}{(}\PYG{p}{)}
\end{sphinxVerbatim}

\end{sphinxuseclass}\end{sphinxVerbatimInput}
\begin{sphinxVerbatimOutput}

\begin{sphinxuseclass}{cell_output}
\begin{sphinxVerbatim}[commandchars=\\\{\}]
a  1    0.4967
   2   \PYGZhy{}0.1383
   3    0.6477
b  1    1.5230
   3   \PYGZhy{}0.2342
c  1   \PYGZhy{}0.2341
   2    1.5792
d  2    0.7674
   3   \PYGZhy{}0.4695
dtype: float64
\end{sphinxVerbatim}

\end{sphinxuseclass}\end{sphinxVerbatimOutput}

\end{sphinxuseclass}
\sphinxAtStartPar
We can create a data frame with hieracrhical indexes or multi\sphinxhyphen{}indexes on rows \sphinxstyleemphasis{and} columns.

\begin{sphinxuseclass}{cell}\begin{sphinxVerbatimInput}

\begin{sphinxuseclass}{cell_input}
\begin{sphinxVerbatim}[commandchars=\\\{\}]
\PYG{n}{frame} \PYG{o}{=} \PYG{n}{pd}\PYG{o}{.}\PYG{n}{DataFrame}\PYG{p}{(}
    \PYG{n}{data}\PYG{o}{=}\PYG{n}{np}\PYG{o}{.}\PYG{n}{arange}\PYG{p}{(}\PYG{l+m+mi}{12}\PYG{p}{)}\PYG{o}{.}\PYG{n}{reshape}\PYG{p}{(}\PYG{p}{(}\PYG{l+m+mi}{4}\PYG{p}{,} \PYG{l+m+mi}{3}\PYG{p}{)}\PYG{p}{)}\PYG{p}{,}
    \PYG{n}{index}\PYG{o}{=}\PYG{p}{[}\PYG{p}{[}\PYG{l+s+s1}{\PYGZsq{}}\PYG{l+s+s1}{a}\PYG{l+s+s1}{\PYGZsq{}}\PYG{p}{,} \PYG{l+s+s1}{\PYGZsq{}}\PYG{l+s+s1}{a}\PYG{l+s+s1}{\PYGZsq{}}\PYG{p}{,} \PYG{l+s+s1}{\PYGZsq{}}\PYG{l+s+s1}{b}\PYG{l+s+s1}{\PYGZsq{}}\PYG{p}{,} \PYG{l+s+s1}{\PYGZsq{}}\PYG{l+s+s1}{b}\PYG{l+s+s1}{\PYGZsq{}}\PYG{p}{]}\PYG{p}{,} \PYG{p}{[}\PYG{l+m+mi}{1}\PYG{p}{,} \PYG{l+m+mi}{2}\PYG{p}{,} \PYG{l+m+mi}{1}\PYG{p}{,} \PYG{l+m+mi}{2}\PYG{p}{]}\PYG{p}{]}\PYG{p}{,}
    \PYG{n}{columns}\PYG{o}{=}\PYG{p}{[}\PYG{p}{[}\PYG{l+s+s1}{\PYGZsq{}}\PYG{l+s+s1}{Ohio}\PYG{l+s+s1}{\PYGZsq{}}\PYG{p}{,} \PYG{l+s+s1}{\PYGZsq{}}\PYG{l+s+s1}{Ohio}\PYG{l+s+s1}{\PYGZsq{}}\PYG{p}{,} \PYG{l+s+s1}{\PYGZsq{}}\PYG{l+s+s1}{Colorado}\PYG{l+s+s1}{\PYGZsq{}}\PYG{p}{]}\PYG{p}{,} \PYG{p}{[}\PYG{l+s+s1}{\PYGZsq{}}\PYG{l+s+s1}{Green}\PYG{l+s+s1}{\PYGZsq{}}\PYG{p}{,} \PYG{l+s+s1}{\PYGZsq{}}\PYG{l+s+s1}{Red}\PYG{l+s+s1}{\PYGZsq{}}\PYG{p}{,} \PYG{l+s+s1}{\PYGZsq{}}\PYG{l+s+s1}{Green}\PYG{l+s+s1}{\PYGZsq{}}\PYG{p}{]}\PYG{p}{]}
\PYG{p}{)}
\PYG{n}{frame}
\end{sphinxVerbatim}

\end{sphinxuseclass}\end{sphinxVerbatimInput}
\begin{sphinxVerbatimOutput}

\begin{sphinxuseclass}{cell_output}
\begin{sphinxVerbatim}[commandchars=\\\{\}]
     Ohio     Colorado
    Green Red    Green
a 1     0   1        2
  2     3   4        5
b 1     6   7        8
  2     9  10       11
\end{sphinxVerbatim}

\end{sphinxuseclass}\end{sphinxVerbatimOutput}

\end{sphinxuseclass}
\sphinxAtStartPar
We can name these multi\sphinxhyphen{}indexes but names are not required.

\begin{sphinxuseclass}{cell}\begin{sphinxVerbatimInput}

\begin{sphinxuseclass}{cell_input}
\begin{sphinxVerbatim}[commandchars=\\\{\}]
\PYG{n}{frame}\PYG{o}{.}\PYG{n}{index}\PYG{o}{.}\PYG{n}{names} \PYG{o}{=} \PYG{p}{[}\PYG{l+s+s1}{\PYGZsq{}}\PYG{l+s+s1}{key1}\PYG{l+s+s1}{\PYGZsq{}}\PYG{p}{,} \PYG{l+s+s1}{\PYGZsq{}}\PYG{l+s+s1}{key2}\PYG{l+s+s1}{\PYGZsq{}}\PYG{p}{]}
\PYG{n}{frame}\PYG{o}{.}\PYG{n}{columns}\PYG{o}{.}\PYG{n}{names} \PYG{o}{=} \PYG{p}{[}\PYG{l+s+s1}{\PYGZsq{}}\PYG{l+s+s1}{state}\PYG{l+s+s1}{\PYGZsq{}}\PYG{p}{,} \PYG{l+s+s1}{\PYGZsq{}}\PYG{l+s+s1}{color}\PYG{l+s+s1}{\PYGZsq{}}\PYG{p}{]}
\PYG{n}{frame}
\end{sphinxVerbatim}

\end{sphinxuseclass}\end{sphinxVerbatimInput}
\begin{sphinxVerbatimOutput}

\begin{sphinxuseclass}{cell_output}
\begin{sphinxVerbatim}[commandchars=\\\{\}]
state      Ohio     Colorado
color     Green Red    Green
key1 key2                   
a    1        0   1        2
     2        3   4        5
b    1        6   7        8
     2        9  10       11
\end{sphinxVerbatim}

\end{sphinxuseclass}\end{sphinxVerbatimOutput}

\end{sphinxuseclass}
\sphinxAtStartPar
Recall that \sphinxcode{\sphinxupquote{df{[}val{]}}} selects the \sphinxcode{\sphinxupquote{val}} column.
Here \sphinxcode{\sphinxupquote{frame}} has a multi\sphinxhyphen{}index for the columns, so \sphinxcode{\sphinxupquote{frame{[}'Ohio'{]}}} selects all columns with Ohio as the outer index level.

\begin{sphinxuseclass}{cell}\begin{sphinxVerbatimInput}

\begin{sphinxuseclass}{cell_input}
\begin{sphinxVerbatim}[commandchars=\\\{\}]
\PYG{n}{frame}\PYG{p}{[}\PYG{l+s+s1}{\PYGZsq{}}\PYG{l+s+s1}{Ohio}\PYG{l+s+s1}{\PYGZsq{}}\PYG{p}{]}
\end{sphinxVerbatim}

\end{sphinxuseclass}\end{sphinxVerbatimInput}
\begin{sphinxVerbatimOutput}

\begin{sphinxuseclass}{cell_output}
\begin{sphinxVerbatim}[commandchars=\\\{\}]
color      Green  Red
key1 key2            
a    1         0    1
     2         3    4
b    1         6    7
     2         9   10
\end{sphinxVerbatim}

\end{sphinxuseclass}\end{sphinxVerbatimOutput}

\end{sphinxuseclass}
\sphinxAtStartPar
We can pass a tuple if we only want one column.

\begin{sphinxuseclass}{cell}\begin{sphinxVerbatimInput}

\begin{sphinxuseclass}{cell_input}
\begin{sphinxVerbatim}[commandchars=\\\{\}]
\PYG{n}{frame}\PYG{p}{[}\PYG{p}{[}\PYG{p}{(}\PYG{l+s+s1}{\PYGZsq{}}\PYG{l+s+s1}{Ohio}\PYG{l+s+s1}{\PYGZsq{}}\PYG{p}{,} \PYG{l+s+s1}{\PYGZsq{}}\PYG{l+s+s1}{Green}\PYG{l+s+s1}{\PYGZsq{}}\PYG{p}{)}\PYG{p}{]}\PYG{p}{]}
\end{sphinxVerbatim}

\end{sphinxuseclass}\end{sphinxVerbatimInput}
\begin{sphinxVerbatimOutput}

\begin{sphinxuseclass}{cell_output}
\begin{sphinxVerbatim}[commandchars=\\\{\}]
state      Ohio
color     Green
key1 key2      
a    1        0
     2        3
b    1        6
     2        9
\end{sphinxVerbatim}

\end{sphinxuseclass}\end{sphinxVerbatimOutput}

\end{sphinxuseclass}
\sphinxAtStartPar
We have to do a more work to slice the inner level of the column index.

\begin{sphinxuseclass}{cell}\begin{sphinxVerbatimInput}

\begin{sphinxuseclass}{cell_input}
\begin{sphinxVerbatim}[commandchars=\\\{\}]
\PYG{n}{frame}\PYG{o}{.}\PYG{n}{loc}\PYG{p}{[}\PYG{p}{:}\PYG{p}{,} \PYG{p}{(}\PYG{n+nb}{slice}\PYG{p}{(}\PYG{k+kc}{None}\PYG{p}{)}\PYG{p}{,} \PYG{l+s+s1}{\PYGZsq{}}\PYG{l+s+s1}{Green}\PYG{l+s+s1}{\PYGZsq{}}\PYG{p}{)}\PYG{p}{]}
\end{sphinxVerbatim}

\end{sphinxuseclass}\end{sphinxVerbatimInput}
\begin{sphinxVerbatimOutput}

\begin{sphinxuseclass}{cell_output}
\begin{sphinxVerbatim}[commandchars=\\\{\}]
state      Ohio Colorado
color     Green    Green
key1 key2               
a    1        0        2
     2        3        5
b    1        6        8
     2        9       11
\end{sphinxVerbatim}

\end{sphinxuseclass}\end{sphinxVerbatimOutput}

\end{sphinxuseclass}
\sphinxAtStartPar
We can use \sphinxcode{\sphinxupquote{pd.IndexSlice{[}:, 'Green'{]}}} an alternative to \sphinxcode{\sphinxupquote{(slice(None), 'Green')}}.

\begin{sphinxuseclass}{cell}\begin{sphinxVerbatimInput}

\begin{sphinxuseclass}{cell_input}
\begin{sphinxVerbatim}[commandchars=\\\{\}]
\PYG{n}{frame}\PYG{o}{.}\PYG{n}{loc}\PYG{p}{[}\PYG{p}{:}\PYG{p}{,} \PYG{n}{pd}\PYG{o}{.}\PYG{n}{IndexSlice}\PYG{p}{[}\PYG{p}{:}\PYG{p}{,} \PYG{l+s+s1}{\PYGZsq{}}\PYG{l+s+s1}{Green}\PYG{l+s+s1}{\PYGZsq{}}\PYG{p}{]}\PYG{p}{]}
\end{sphinxVerbatim}

\end{sphinxuseclass}\end{sphinxVerbatimInput}
\begin{sphinxVerbatimOutput}

\begin{sphinxuseclass}{cell_output}
\begin{sphinxVerbatim}[commandchars=\\\{\}]
state      Ohio Colorado
color     Green    Green
key1 key2               
a    1        0        2
     2        3        5
b    1        6        8
     2        9       11
\end{sphinxVerbatim}

\end{sphinxuseclass}\end{sphinxVerbatimOutput}

\end{sphinxuseclass}

\subsection{Reordering and Sorting Levels}
\label{\detokenize{mckinney_08_lecture:reordering-and-sorting-levels}}
\sphinxAtStartPar
We can swap index levels with the \sphinxcode{\sphinxupquote{.swaplevel()}} method.
The default arguments are \sphinxcode{\sphinxupquote{i=\sphinxhyphen{}2}} and \sphinxcode{\sphinxupquote{j=\sphinxhyphen{}1}}, which swap the two innermost index levels.

\begin{sphinxuseclass}{cell}\begin{sphinxVerbatimInput}

\begin{sphinxuseclass}{cell_input}
\begin{sphinxVerbatim}[commandchars=\\\{\}]
\PYG{n}{frame}\PYG{o}{.}\PYG{n}{swaplevel}\PYG{p}{(}\PYG{p}{)}
\end{sphinxVerbatim}

\end{sphinxuseclass}\end{sphinxVerbatimInput}
\begin{sphinxVerbatimOutput}

\begin{sphinxuseclass}{cell_output}
\begin{sphinxVerbatim}[commandchars=\\\{\}]
state      Ohio     Colorado
color     Green Red    Green
key2 key1                   
1    a        0   1        2
2    a        3   4        5
1    b        6   7        8
2    b        9  10       11
\end{sphinxVerbatim}

\end{sphinxuseclass}\end{sphinxVerbatimOutput}

\end{sphinxuseclass}
\sphinxAtStartPar
We can use index \sphinxstyleemphasis{names}, too.

\begin{sphinxuseclass}{cell}\begin{sphinxVerbatimInput}

\begin{sphinxuseclass}{cell_input}
\begin{sphinxVerbatim}[commandchars=\\\{\}]
\PYG{n}{frame}\PYG{o}{.}\PYG{n}{swaplevel}\PYG{p}{(}\PYG{l+s+s1}{\PYGZsq{}}\PYG{l+s+s1}{key1}\PYG{l+s+s1}{\PYGZsq{}}\PYG{p}{,} \PYG{l+s+s1}{\PYGZsq{}}\PYG{l+s+s1}{key2}\PYG{l+s+s1}{\PYGZsq{}}\PYG{p}{)}
\end{sphinxVerbatim}

\end{sphinxuseclass}\end{sphinxVerbatimInput}
\begin{sphinxVerbatimOutput}

\begin{sphinxuseclass}{cell_output}
\begin{sphinxVerbatim}[commandchars=\\\{\}]
state      Ohio     Colorado
color     Green Red    Green
key2 key1                   
1    a        0   1        2
2    a        3   4        5
1    b        6   7        8
2    b        9  10       11
\end{sphinxVerbatim}

\end{sphinxuseclass}\end{sphinxVerbatimOutput}

\end{sphinxuseclass}
\sphinxAtStartPar
We can also sort on an index (or list of indexes).
After we swap levels, we may want to sort our data.

\begin{sphinxuseclass}{cell}\begin{sphinxVerbatimInput}

\begin{sphinxuseclass}{cell_input}
\begin{sphinxVerbatim}[commandchars=\\\{\}]
\PYG{n}{frame}
\end{sphinxVerbatim}

\end{sphinxuseclass}\end{sphinxVerbatimInput}
\begin{sphinxVerbatimOutput}

\begin{sphinxuseclass}{cell_output}
\begin{sphinxVerbatim}[commandchars=\\\{\}]
state      Ohio     Colorado
color     Green Red    Green
key1 key2                   
a    1        0   1        2
     2        3   4        5
b    1        6   7        8
     2        9  10       11
\end{sphinxVerbatim}

\end{sphinxuseclass}\end{sphinxVerbatimOutput}

\end{sphinxuseclass}
\begin{sphinxuseclass}{cell}\begin{sphinxVerbatimInput}

\begin{sphinxuseclass}{cell_input}
\begin{sphinxVerbatim}[commandchars=\\\{\}]
\PYG{n}{frame}\PYG{o}{.}\PYG{n}{sort\PYGZus{}index}\PYG{p}{(}\PYG{n}{level}\PYG{o}{=}\PYG{l+m+mi}{1}\PYG{p}{)}
\end{sphinxVerbatim}

\end{sphinxuseclass}\end{sphinxVerbatimInput}
\begin{sphinxVerbatimOutput}

\begin{sphinxuseclass}{cell_output}
\begin{sphinxVerbatim}[commandchars=\\\{\}]
state      Ohio     Colorado
color     Green Red    Green
key1 key2                   
a    1        0   1        2
b    1        6   7        8
a    2        3   4        5
b    2        9  10       11
\end{sphinxVerbatim}

\end{sphinxuseclass}\end{sphinxVerbatimOutput}

\end{sphinxuseclass}
\sphinxAtStartPar
Again, we can give index \sphinxstyleemphasis{names}, too.

\begin{sphinxuseclass}{cell}\begin{sphinxVerbatimInput}

\begin{sphinxuseclass}{cell_input}
\begin{sphinxVerbatim}[commandchars=\\\{\}]
\PYG{n}{frame}\PYG{o}{.}\PYG{n}{sort\PYGZus{}index}\PYG{p}{(}\PYG{n}{level}\PYG{o}{=}\PYG{l+s+s1}{\PYGZsq{}}\PYG{l+s+s1}{key2}\PYG{l+s+s1}{\PYGZsq{}}\PYG{p}{)}
\end{sphinxVerbatim}

\end{sphinxuseclass}\end{sphinxVerbatimInput}
\begin{sphinxVerbatimOutput}

\begin{sphinxuseclass}{cell_output}
\begin{sphinxVerbatim}[commandchars=\\\{\}]
state      Ohio     Colorado
color     Green Red    Green
key1 key2                   
a    1        0   1        2
b    1        6   7        8
a    2        3   4        5
b    2        9  10       11
\end{sphinxVerbatim}

\end{sphinxuseclass}\end{sphinxVerbatimOutput}

\end{sphinxuseclass}
\sphinxAtStartPar
We can sort by two or more index levels by passing a list of index levels or names.

\begin{sphinxuseclass}{cell}\begin{sphinxVerbatimInput}

\begin{sphinxuseclass}{cell_input}
\begin{sphinxVerbatim}[commandchars=\\\{\}]
\PYG{n}{frame}\PYG{o}{.}\PYG{n}{sort\PYGZus{}index}\PYG{p}{(}\PYG{n}{level}\PYG{o}{=}\PYG{p}{[}\PYG{l+m+mi}{0}\PYG{p}{,} \PYG{l+m+mi}{1}\PYG{p}{]}\PYG{p}{)}
\end{sphinxVerbatim}

\end{sphinxuseclass}\end{sphinxVerbatimInput}
\begin{sphinxVerbatimOutput}

\begin{sphinxuseclass}{cell_output}
\begin{sphinxVerbatim}[commandchars=\\\{\}]
state      Ohio     Colorado
color     Green Red    Green
key1 key2                   
a    1        0   1        2
     2        3   4        5
b    1        6   7        8
     2        9  10       11
\end{sphinxVerbatim}

\end{sphinxuseclass}\end{sphinxVerbatimOutput}

\end{sphinxuseclass}
\sphinxAtStartPar
We can chain these methods, too.

\begin{sphinxuseclass}{cell}\begin{sphinxVerbatimInput}

\begin{sphinxuseclass}{cell_input}
\begin{sphinxVerbatim}[commandchars=\\\{\}]
\PYG{n}{frame}\PYG{o}{.}\PYG{n}{swaplevel}\PYG{p}{(}\PYG{l+m+mi}{0}\PYG{p}{,} \PYG{l+m+mi}{1}\PYG{p}{)}\PYG{o}{.}\PYG{n}{sort\PYGZus{}index}\PYG{p}{(}\PYG{n}{level}\PYG{o}{=}\PYG{l+m+mi}{0}\PYG{p}{)}
\end{sphinxVerbatim}

\end{sphinxuseclass}\end{sphinxVerbatimInput}
\begin{sphinxVerbatimOutput}

\begin{sphinxuseclass}{cell_output}
\begin{sphinxVerbatim}[commandchars=\\\{\}]
state      Ohio     Colorado
color     Green Red    Green
key2 key1                   
1    a        0   1        2
     b        6   7        8
2    a        3   4        5
     b        9  10       11
\end{sphinxVerbatim}

\end{sphinxuseclass}\end{sphinxVerbatimOutput}

\end{sphinxuseclass}

\subsection{Indexing with a DataFrame’s columns}
\label{\detokenize{mckinney_08_lecture:indexing-with-a-dataframe-s-columns}}
\sphinxAtStartPar
We can convert a column into an index and an index into a column with the \sphinxcode{\sphinxupquote{.set\_index()}} and \sphinxcode{\sphinxupquote{.reset\_index()}} methods.

\begin{sphinxuseclass}{cell}\begin{sphinxVerbatimInput}

\begin{sphinxuseclass}{cell_input}
\begin{sphinxVerbatim}[commandchars=\\\{\}]
\PYG{n}{frame} \PYG{o}{=} \PYG{n}{pd}\PYG{o}{.}\PYG{n}{DataFrame}\PYG{p}{(}\PYG{p}{\PYGZob{}}
    \PYG{l+s+s1}{\PYGZsq{}}\PYG{l+s+s1}{a}\PYG{l+s+s1}{\PYGZsq{}}\PYG{p}{:} \PYG{n+nb}{range}\PYG{p}{(}\PYG{l+m+mi}{7}\PYG{p}{)}\PYG{p}{,} 
    \PYG{l+s+s1}{\PYGZsq{}}\PYG{l+s+s1}{b}\PYG{l+s+s1}{\PYGZsq{}}\PYG{p}{:} \PYG{n+nb}{range}\PYG{p}{(}\PYG{l+m+mi}{7}\PYG{p}{,} \PYG{l+m+mi}{0}\PYG{p}{,} \PYG{o}{\PYGZhy{}}\PYG{l+m+mi}{1}\PYG{p}{)}\PYG{p}{,}
    \PYG{l+s+s1}{\PYGZsq{}}\PYG{l+s+s1}{c}\PYG{l+s+s1}{\PYGZsq{}}\PYG{p}{:} \PYG{p}{[}\PYG{l+s+s1}{\PYGZsq{}}\PYG{l+s+s1}{one}\PYG{l+s+s1}{\PYGZsq{}}\PYG{p}{,} \PYG{l+s+s1}{\PYGZsq{}}\PYG{l+s+s1}{one}\PYG{l+s+s1}{\PYGZsq{}}\PYG{p}{,} \PYG{l+s+s1}{\PYGZsq{}}\PYG{l+s+s1}{one}\PYG{l+s+s1}{\PYGZsq{}}\PYG{p}{,} \PYG{l+s+s1}{\PYGZsq{}}\PYG{l+s+s1}{two}\PYG{l+s+s1}{\PYGZsq{}}\PYG{p}{,} \PYG{l+s+s1}{\PYGZsq{}}\PYG{l+s+s1}{two}\PYG{l+s+s1}{\PYGZsq{}}\PYG{p}{,}\PYG{l+s+s1}{\PYGZsq{}}\PYG{l+s+s1}{two}\PYG{l+s+s1}{\PYGZsq{}}\PYG{p}{,} \PYG{l+s+s1}{\PYGZsq{}}\PYG{l+s+s1}{two}\PYG{l+s+s1}{\PYGZsq{}}\PYG{p}{]}\PYG{p}{,}
    \PYG{l+s+s1}{\PYGZsq{}}\PYG{l+s+s1}{d}\PYG{l+s+s1}{\PYGZsq{}}\PYG{p}{:} \PYG{p}{[}\PYG{l+m+mi}{0}\PYG{p}{,} \PYG{l+m+mi}{1}\PYG{p}{,} \PYG{l+m+mi}{2}\PYG{p}{,} \PYG{l+m+mi}{0}\PYG{p}{,} \PYG{l+m+mi}{1}\PYG{p}{,} \PYG{l+m+mi}{2}\PYG{p}{,} \PYG{l+m+mi}{3}\PYG{p}{]}
\PYG{p}{\PYGZcb{}}\PYG{p}{)}
\PYG{n}{frame}
\end{sphinxVerbatim}

\end{sphinxuseclass}\end{sphinxVerbatimInput}
\begin{sphinxVerbatimOutput}

\begin{sphinxuseclass}{cell_output}
\begin{sphinxVerbatim}[commandchars=\\\{\}]
   a  b    c  d
0  0  7  one  0
1  1  6  one  1
2  2  5  one  2
3  3  4  two  0
4  4  3  two  1
5  5  2  two  2
6  6  1  two  3
\end{sphinxVerbatim}

\end{sphinxuseclass}\end{sphinxVerbatimOutput}

\end{sphinxuseclass}
\sphinxAtStartPar
The \sphinxcode{\sphinxupquote{.set\_index()}} method converts columns to indexes, and removes the columns from the data frame by default.

\begin{sphinxuseclass}{cell}\begin{sphinxVerbatimInput}

\begin{sphinxuseclass}{cell_input}
\begin{sphinxVerbatim}[commandchars=\\\{\}]
\PYG{n}{frame2} \PYG{o}{=} \PYG{n}{frame}\PYG{o}{.}\PYG{n}{set\PYGZus{}index}\PYG{p}{(}\PYG{p}{[}\PYG{l+s+s1}{\PYGZsq{}}\PYG{l+s+s1}{c}\PYG{l+s+s1}{\PYGZsq{}}\PYG{p}{,} \PYG{l+s+s1}{\PYGZsq{}}\PYG{l+s+s1}{d}\PYG{l+s+s1}{\PYGZsq{}}\PYG{p}{]}\PYG{p}{)}
\PYG{n}{frame2}
\end{sphinxVerbatim}

\end{sphinxuseclass}\end{sphinxVerbatimInput}
\begin{sphinxVerbatimOutput}

\begin{sphinxuseclass}{cell_output}
\begin{sphinxVerbatim}[commandchars=\\\{\}]
       a  b
c   d      
one 0  0  7
    1  1  6
    2  2  5
two 0  3  4
    1  4  3
    2  5  2
    3  6  1
\end{sphinxVerbatim}

\end{sphinxuseclass}\end{sphinxVerbatimOutput}

\end{sphinxuseclass}
\sphinxAtStartPar
The \sphinxcode{\sphinxupquote{.reset\_index()}} method removes the indexes, adds them as columns, and sets in integer index.

\begin{sphinxuseclass}{cell}\begin{sphinxVerbatimInput}

\begin{sphinxuseclass}{cell_input}
\begin{sphinxVerbatim}[commandchars=\\\{\}]
\PYG{n}{frame2}\PYG{o}{.}\PYG{n}{reset\PYGZus{}index}\PYG{p}{(}\PYG{p}{)}
\end{sphinxVerbatim}

\end{sphinxuseclass}\end{sphinxVerbatimInput}
\begin{sphinxVerbatimOutput}

\begin{sphinxuseclass}{cell_output}
\begin{sphinxVerbatim}[commandchars=\\\{\}]
     c  d  a  b
0  one  0  0  7
1  one  1  1  6
2  one  2  2  5
3  two  0  3  4
4  two  1  4  3
5  two  2  5  2
6  two  3  6  1
\end{sphinxVerbatim}

\end{sphinxuseclass}\end{sphinxVerbatimOutput}

\end{sphinxuseclass}

\section{Combining and Merging Datasets}
\label{\detokenize{mckinney_08_lecture:combining-and-merging-datasets}}
\sphinxAtStartPar
pandas provides several methods and functions to combine and merge data.
We can typically create the same output with any of these methods or functions, but one may be more efficient than the others.
If I want to combine data frames with similar indexes, I try the \sphinxcode{\sphinxupquote{.join()}} method first.
The \sphinxcode{\sphinxupquote{.join()}} also lets use can combine more than two data frames at once.
Otherwise, I try the \sphinxcode{\sphinxupquote{.merge()}} method, which has a function \sphinxcode{\sphinxupquote{pd.merge()}}, too.
The \sphinxcode{\sphinxupquote{pd.merge()}} function is more general than the \sphinxcode{\sphinxupquote{.join()}} method, so we will start with \sphinxcode{\sphinxupquote{pd.merge()}}.

\sphinxAtStartPar
The \sphinxhref{https://pandas.pydata.org/pandas-docs/stable/user\_guide/merging.html\#}{pandas website} provides helpful visualizations.


\subsection{Database\sphinxhyphen{}Style DataFrame Joins}
\label{\detokenize{mckinney_08_lecture:database-style-dataframe-joins}}\begin{quote}

\sphinxAtStartPar
Merge or join operations combine datasets by linking rows using one or more keys. These operations are central to relational databases (e.g., SQL\sphinxhyphen{}based). The merge function in pandas is the main entry point for using these algorithms on your data.
\end{quote}

\sphinxAtStartPar
We will start with the \sphinxcode{\sphinxupquote{pd.merge()}} syntax, but pandas also has \sphinxcode{\sphinxupquote{.merge()}} and \sphinxcode{\sphinxupquote{.join()}} methods.
Learning these other syntaxes is easy once we understand the \sphinxcode{\sphinxupquote{pd.merge()}} syntax.

\begin{sphinxuseclass}{cell}\begin{sphinxVerbatimInput}

\begin{sphinxuseclass}{cell_input}
\begin{sphinxVerbatim}[commandchars=\\\{\}]
\PYG{n}{df1} \PYG{o}{=} \PYG{n}{pd}\PYG{o}{.}\PYG{n}{DataFrame}\PYG{p}{(}\PYG{p}{\PYGZob{}}\PYG{l+s+s1}{\PYGZsq{}}\PYG{l+s+s1}{key}\PYG{l+s+s1}{\PYGZsq{}}\PYG{p}{:} \PYG{p}{[}\PYG{l+s+s1}{\PYGZsq{}}\PYG{l+s+s1}{b}\PYG{l+s+s1}{\PYGZsq{}}\PYG{p}{,} \PYG{l+s+s1}{\PYGZsq{}}\PYG{l+s+s1}{b}\PYG{l+s+s1}{\PYGZsq{}}\PYG{p}{,} \PYG{l+s+s1}{\PYGZsq{}}\PYG{l+s+s1}{a}\PYG{l+s+s1}{\PYGZsq{}}\PYG{p}{,} \PYG{l+s+s1}{\PYGZsq{}}\PYG{l+s+s1}{c}\PYG{l+s+s1}{\PYGZsq{}}\PYG{p}{,} \PYG{l+s+s1}{\PYGZsq{}}\PYG{l+s+s1}{a}\PYG{l+s+s1}{\PYGZsq{}}\PYG{p}{,} \PYG{l+s+s1}{\PYGZsq{}}\PYG{l+s+s1}{a}\PYG{l+s+s1}{\PYGZsq{}}\PYG{p}{,} \PYG{l+s+s1}{\PYGZsq{}}\PYG{l+s+s1}{b}\PYG{l+s+s1}{\PYGZsq{}}\PYG{p}{]}\PYG{p}{,} \PYG{l+s+s1}{\PYGZsq{}}\PYG{l+s+s1}{data1}\PYG{l+s+s1}{\PYGZsq{}}\PYG{p}{:} \PYG{n+nb}{range}\PYG{p}{(}\PYG{l+m+mi}{7}\PYG{p}{)}\PYG{p}{\PYGZcb{}}\PYG{p}{)}
\PYG{n}{df2} \PYG{o}{=} \PYG{n}{pd}\PYG{o}{.}\PYG{n}{DataFrame}\PYG{p}{(}\PYG{p}{\PYGZob{}}\PYG{l+s+s1}{\PYGZsq{}}\PYG{l+s+s1}{key}\PYG{l+s+s1}{\PYGZsq{}}\PYG{p}{:} \PYG{p}{[}\PYG{l+s+s1}{\PYGZsq{}}\PYG{l+s+s1}{a}\PYG{l+s+s1}{\PYGZsq{}}\PYG{p}{,} \PYG{l+s+s1}{\PYGZsq{}}\PYG{l+s+s1}{b}\PYG{l+s+s1}{\PYGZsq{}}\PYG{p}{,} \PYG{l+s+s1}{\PYGZsq{}}\PYG{l+s+s1}{d}\PYG{l+s+s1}{\PYGZsq{}}\PYG{p}{]}\PYG{p}{,} \PYG{l+s+s1}{\PYGZsq{}}\PYG{l+s+s1}{data2}\PYG{l+s+s1}{\PYGZsq{}}\PYG{p}{:} \PYG{n+nb}{range}\PYG{p}{(}\PYG{l+m+mi}{3}\PYG{p}{)}\PYG{p}{\PYGZcb{}}\PYG{p}{)}
\end{sphinxVerbatim}

\end{sphinxuseclass}\end{sphinxVerbatimInput}

\end{sphinxuseclass}
\begin{sphinxuseclass}{cell}\begin{sphinxVerbatimInput}

\begin{sphinxuseclass}{cell_input}
\begin{sphinxVerbatim}[commandchars=\\\{\}]
\PYG{n}{df1}
\end{sphinxVerbatim}

\end{sphinxuseclass}\end{sphinxVerbatimInput}
\begin{sphinxVerbatimOutput}

\begin{sphinxuseclass}{cell_output}
\begin{sphinxVerbatim}[commandchars=\\\{\}]
  key  data1
0   b      0
1   b      1
2   a      2
3   c      3
4   a      4
5   a      5
6   b      6
\end{sphinxVerbatim}

\end{sphinxuseclass}\end{sphinxVerbatimOutput}

\end{sphinxuseclass}
\begin{sphinxuseclass}{cell}\begin{sphinxVerbatimInput}

\begin{sphinxuseclass}{cell_input}
\begin{sphinxVerbatim}[commandchars=\\\{\}]
\PYG{n}{df2}
\end{sphinxVerbatim}

\end{sphinxuseclass}\end{sphinxVerbatimInput}
\begin{sphinxVerbatimOutput}

\begin{sphinxuseclass}{cell_output}
\begin{sphinxVerbatim}[commandchars=\\\{\}]
  key  data2
0   a      0
1   b      1
2   d      2
\end{sphinxVerbatim}

\end{sphinxuseclass}\end{sphinxVerbatimOutput}

\end{sphinxuseclass}
\begin{sphinxuseclass}{cell}\begin{sphinxVerbatimInput}

\begin{sphinxuseclass}{cell_input}
\begin{sphinxVerbatim}[commandchars=\\\{\}]
\PYG{n}{pd}\PYG{o}{.}\PYG{n}{merge}\PYG{p}{(}\PYG{n}{df1}\PYG{p}{,} \PYG{n}{df2}\PYG{p}{)}
\end{sphinxVerbatim}

\end{sphinxuseclass}\end{sphinxVerbatimInput}
\begin{sphinxVerbatimOutput}

\begin{sphinxuseclass}{cell_output}
\begin{sphinxVerbatim}[commandchars=\\\{\}]
  key  data1  data2
0   b      0      1
1   b      1      1
2   b      6      1
3   a      2      0
4   a      4      0
5   a      5      0
\end{sphinxVerbatim}

\end{sphinxuseclass}\end{sphinxVerbatimOutput}

\end{sphinxuseclass}
\sphinxAtStartPar
The default \sphinxcode{\sphinxupquote{how}} is \sphinxcode{\sphinxupquote{how='inner'}}, so \sphinxcode{\sphinxupquote{pd.merge()}} inner joins left and right data frames by default, keeping only rows that appear in both.
We can specify \sphinxcode{\sphinxupquote{how='outer'}}, so \sphinxcode{\sphinxupquote{pd.merge()}} outer joins left and right data frames, keeping all rows that appear in either.

\begin{sphinxuseclass}{cell}\begin{sphinxVerbatimInput}

\begin{sphinxuseclass}{cell_input}
\begin{sphinxVerbatim}[commandchars=\\\{\}]
\PYG{n}{pd}\PYG{o}{.}\PYG{n}{merge}\PYG{p}{(}\PYG{n}{df1}\PYG{p}{,} \PYG{n}{df2}\PYG{p}{,} \PYG{n}{how}\PYG{o}{=}\PYG{l+s+s1}{\PYGZsq{}}\PYG{l+s+s1}{outer}\PYG{l+s+s1}{\PYGZsq{}}\PYG{p}{)}
\end{sphinxVerbatim}

\end{sphinxuseclass}\end{sphinxVerbatimInput}
\begin{sphinxVerbatimOutput}

\begin{sphinxuseclass}{cell_output}
\begin{sphinxVerbatim}[commandchars=\\\{\}]
  key  data1  data2
0   b 0.0000 1.0000
1   b 1.0000 1.0000
2   b 6.0000 1.0000
3   a 2.0000 0.0000
4   a 4.0000 0.0000
5   a 5.0000 0.0000
6   c 3.0000    NaN
7   d    NaN 2.0000
\end{sphinxVerbatim}

\end{sphinxuseclass}\end{sphinxVerbatimOutput}

\end{sphinxuseclass}
\sphinxAtStartPar
A left merge keeps only rows that appear in the left data frame.

\begin{sphinxuseclass}{cell}\begin{sphinxVerbatimInput}

\begin{sphinxuseclass}{cell_input}
\begin{sphinxVerbatim}[commandchars=\\\{\}]
\PYG{n}{pd}\PYG{o}{.}\PYG{n}{merge}\PYG{p}{(}\PYG{n}{df1}\PYG{p}{,} \PYG{n}{df2}\PYG{p}{,} \PYG{n}{how}\PYG{o}{=}\PYG{l+s+s1}{\PYGZsq{}}\PYG{l+s+s1}{left}\PYG{l+s+s1}{\PYGZsq{}}\PYG{p}{)}
\end{sphinxVerbatim}

\end{sphinxuseclass}\end{sphinxVerbatimInput}
\begin{sphinxVerbatimOutput}

\begin{sphinxuseclass}{cell_output}
\begin{sphinxVerbatim}[commandchars=\\\{\}]
  key  data1  data2
0   b      0 1.0000
1   b      1 1.0000
2   a      2 0.0000
3   c      3    NaN
4   a      4 0.0000
5   a      5 0.0000
6   b      6 1.0000
\end{sphinxVerbatim}

\end{sphinxuseclass}\end{sphinxVerbatimOutput}

\end{sphinxuseclass}
\sphinxAtStartPar
A rights merge keeps only rows that appear in the right data frame.

\begin{sphinxuseclass}{cell}\begin{sphinxVerbatimInput}

\begin{sphinxuseclass}{cell_input}
\begin{sphinxVerbatim}[commandchars=\\\{\}]
\PYG{n}{pd}\PYG{o}{.}\PYG{n}{merge}\PYG{p}{(}\PYG{n}{df1}\PYG{p}{,} \PYG{n}{df2}\PYG{p}{,} \PYG{n}{how}\PYG{o}{=}\PYG{l+s+s1}{\PYGZsq{}}\PYG{l+s+s1}{right}\PYG{l+s+s1}{\PYGZsq{}}\PYG{p}{)}
\end{sphinxVerbatim}

\end{sphinxuseclass}\end{sphinxVerbatimInput}
\begin{sphinxVerbatimOutput}

\begin{sphinxuseclass}{cell_output}
\begin{sphinxVerbatim}[commandchars=\\\{\}]
  key  data1  data2
0   a 2.0000      0
1   a 4.0000      0
2   a 5.0000      0
3   b 0.0000      1
4   b 1.0000      1
5   b 6.0000      1
6   d    NaN      2
\end{sphinxVerbatim}

\end{sphinxuseclass}\end{sphinxVerbatimOutput}

\end{sphinxuseclass}
\sphinxAtStartPar
By default, \sphinxcode{\sphinxupquote{pd.merge()}} merges on all columns that appear in both data frames.
\begin{quote}

\sphinxAtStartPar
\sphinxcode{\sphinxupquote{on}} : label or list
Column or index level names to join on. These must be found in both
DataFrames. If \sphinxcode{\sphinxupquote{on}} is None and not merging on indexes then this defaults
to the intersection of the columns in both DataFrames.
\end{quote}

\sphinxAtStartPar
Here \sphinxcode{\sphinxupquote{key}} is the only common column between \sphinxcode{\sphinxupquote{df1}} and \sphinxcode{\sphinxupquote{df2}}.
We \sphinxstyleemphasis{should} specify \sphinxcode{\sphinxupquote{on}} to avoid unexpected results.

\begin{sphinxuseclass}{cell}\begin{sphinxVerbatimInput}

\begin{sphinxuseclass}{cell_input}
\begin{sphinxVerbatim}[commandchars=\\\{\}]
\PYG{n}{pd}\PYG{o}{.}\PYG{n}{merge}\PYG{p}{(}\PYG{n}{df1}\PYG{p}{,} \PYG{n}{df2}\PYG{p}{,} \PYG{n}{on}\PYG{o}{=}\PYG{l+s+s1}{\PYGZsq{}}\PYG{l+s+s1}{key}\PYG{l+s+s1}{\PYGZsq{}}\PYG{p}{)}
\end{sphinxVerbatim}

\end{sphinxuseclass}\end{sphinxVerbatimInput}
\begin{sphinxVerbatimOutput}

\begin{sphinxuseclass}{cell_output}
\begin{sphinxVerbatim}[commandchars=\\\{\}]
  key  data1  data2
0   b      0      1
1   b      1      1
2   b      6      1
3   a      2      0
4   a      4      0
5   a      5      0
\end{sphinxVerbatim}

\end{sphinxuseclass}\end{sphinxVerbatimOutput}

\end{sphinxuseclass}
\sphinxAtStartPar
We \sphinxstyleemphasis{must} specify \sphinxcode{\sphinxupquote{left\_on}} and \sphinxcode{\sphinxupquote{right\_on}} if our left and right data frames do not have a common column.

\begin{sphinxuseclass}{cell}\begin{sphinxVerbatimInput}

\begin{sphinxuseclass}{cell_input}
\begin{sphinxVerbatim}[commandchars=\\\{\}]
\PYG{n}{df3} \PYG{o}{=} \PYG{n}{pd}\PYG{o}{.}\PYG{n}{DataFrame}\PYG{p}{(}\PYG{p}{\PYGZob{}}\PYG{l+s+s1}{\PYGZsq{}}\PYG{l+s+s1}{lkey}\PYG{l+s+s1}{\PYGZsq{}}\PYG{p}{:} \PYG{p}{[}\PYG{l+s+s1}{\PYGZsq{}}\PYG{l+s+s1}{b}\PYG{l+s+s1}{\PYGZsq{}}\PYG{p}{,} \PYG{l+s+s1}{\PYGZsq{}}\PYG{l+s+s1}{b}\PYG{l+s+s1}{\PYGZsq{}}\PYG{p}{,} \PYG{l+s+s1}{\PYGZsq{}}\PYG{l+s+s1}{a}\PYG{l+s+s1}{\PYGZsq{}}\PYG{p}{,} \PYG{l+s+s1}{\PYGZsq{}}\PYG{l+s+s1}{c}\PYG{l+s+s1}{\PYGZsq{}}\PYG{p}{,} \PYG{l+s+s1}{\PYGZsq{}}\PYG{l+s+s1}{a}\PYG{l+s+s1}{\PYGZsq{}}\PYG{p}{,} \PYG{l+s+s1}{\PYGZsq{}}\PYG{l+s+s1}{a}\PYG{l+s+s1}{\PYGZsq{}}\PYG{p}{,} \PYG{l+s+s1}{\PYGZsq{}}\PYG{l+s+s1}{b}\PYG{l+s+s1}{\PYGZsq{}}\PYG{p}{]}\PYG{p}{,} \PYG{l+s+s1}{\PYGZsq{}}\PYG{l+s+s1}{data1}\PYG{l+s+s1}{\PYGZsq{}}\PYG{p}{:} \PYG{n+nb}{range}\PYG{p}{(}\PYG{l+m+mi}{7}\PYG{p}{)}\PYG{p}{\PYGZcb{}}\PYG{p}{)}
\PYG{n}{df4} \PYG{o}{=} \PYG{n}{pd}\PYG{o}{.}\PYG{n}{DataFrame}\PYG{p}{(}\PYG{p}{\PYGZob{}}\PYG{l+s+s1}{\PYGZsq{}}\PYG{l+s+s1}{rkey}\PYG{l+s+s1}{\PYGZsq{}}\PYG{p}{:} \PYG{p}{[}\PYG{l+s+s1}{\PYGZsq{}}\PYG{l+s+s1}{a}\PYG{l+s+s1}{\PYGZsq{}}\PYG{p}{,} \PYG{l+s+s1}{\PYGZsq{}}\PYG{l+s+s1}{b}\PYG{l+s+s1}{\PYGZsq{}}\PYG{p}{,} \PYG{l+s+s1}{\PYGZsq{}}\PYG{l+s+s1}{d}\PYG{l+s+s1}{\PYGZsq{}}\PYG{p}{]}\PYG{p}{,} \PYG{l+s+s1}{\PYGZsq{}}\PYG{l+s+s1}{data2}\PYG{l+s+s1}{\PYGZsq{}}\PYG{p}{:} \PYG{n+nb}{range}\PYG{p}{(}\PYG{l+m+mi}{3}\PYG{p}{)}\PYG{p}{\PYGZcb{}}\PYG{p}{)}
\end{sphinxVerbatim}

\end{sphinxuseclass}\end{sphinxVerbatimInput}

\end{sphinxuseclass}
\begin{sphinxuseclass}{cell}\begin{sphinxVerbatimInput}

\begin{sphinxuseclass}{cell_input}
\begin{sphinxVerbatim}[commandchars=\\\{\}]
\PYG{n}{df3}
\end{sphinxVerbatim}

\end{sphinxuseclass}\end{sphinxVerbatimInput}
\begin{sphinxVerbatimOutput}

\begin{sphinxuseclass}{cell_output}
\begin{sphinxVerbatim}[commandchars=\\\{\}]
  lkey  data1
0    b      0
1    b      1
2    a      2
3    c      3
4    a      4
5    a      5
6    b      6
\end{sphinxVerbatim}

\end{sphinxuseclass}\end{sphinxVerbatimOutput}

\end{sphinxuseclass}
\begin{sphinxuseclass}{cell}\begin{sphinxVerbatimInput}

\begin{sphinxuseclass}{cell_input}
\begin{sphinxVerbatim}[commandchars=\\\{\}]
\PYG{n}{df4}
\end{sphinxVerbatim}

\end{sphinxuseclass}\end{sphinxVerbatimInput}
\begin{sphinxVerbatimOutput}

\begin{sphinxuseclass}{cell_output}
\begin{sphinxVerbatim}[commandchars=\\\{\}]
  rkey  data2
0    a      0
1    b      1
2    d      2
\end{sphinxVerbatim}

\end{sphinxuseclass}\end{sphinxVerbatimOutput}

\end{sphinxuseclass}
\begin{sphinxuseclass}{cell}\begin{sphinxVerbatimInput}

\begin{sphinxuseclass}{cell_input}
\begin{sphinxVerbatim}[commandchars=\\\{\}]
\PYG{c+c1}{\PYGZsh{} pd.merge(df3, df4) \PYGZsh{} this code fails/errors because there are not common columns}
\end{sphinxVerbatim}

\end{sphinxuseclass}\end{sphinxVerbatimInput}

\end{sphinxuseclass}
\begin{sphinxuseclass}{cell}\begin{sphinxVerbatimInput}

\begin{sphinxuseclass}{cell_input}
\begin{sphinxVerbatim}[commandchars=\\\{\}]
\PYG{n}{pd}\PYG{o}{.}\PYG{n}{merge}\PYG{p}{(}\PYG{n}{df3}\PYG{p}{,} \PYG{n}{df4}\PYG{p}{,} \PYG{n}{left\PYGZus{}on}\PYG{o}{=}\PYG{l+s+s1}{\PYGZsq{}}\PYG{l+s+s1}{lkey}\PYG{l+s+s1}{\PYGZsq{}}\PYG{p}{,} \PYG{n}{right\PYGZus{}on}\PYG{o}{=}\PYG{l+s+s1}{\PYGZsq{}}\PYG{l+s+s1}{rkey}\PYG{l+s+s1}{\PYGZsq{}}\PYG{p}{)}
\end{sphinxVerbatim}

\end{sphinxuseclass}\end{sphinxVerbatimInput}
\begin{sphinxVerbatimOutput}

\begin{sphinxuseclass}{cell_output}
\begin{sphinxVerbatim}[commandchars=\\\{\}]
  lkey  data1 rkey  data2
0    b      0    b      1
1    b      1    b      1
2    b      6    b      1
3    a      2    a      0
4    a      4    a      0
5    a      5    a      0
\end{sphinxVerbatim}

\end{sphinxuseclass}\end{sphinxVerbatimOutput}

\end{sphinxuseclass}
\sphinxAtStartPar
Here \sphinxcode{\sphinxupquote{pd.merge()}} dropped row \sphinxcode{\sphinxupquote{c}} from \sphinxcode{\sphinxupquote{df3}} and row \sphinxcode{\sphinxupquote{d}} from \sphinxcode{\sphinxupquote{df4}}.
Rows \sphinxcode{\sphinxupquote{c}} and \sphinxcode{\sphinxupquote{d}} dropped because \sphinxcode{\sphinxupquote{pd.merge()}} \sphinxstyleemphasis{inner} joins be default.
An inner join keeps the intersection of the left and right data frame keys.
Further, rows \sphinxcode{\sphinxupquote{a}} and \sphinxcode{\sphinxupquote{b}} from \sphinxcode{\sphinxupquote{df4}} appear three times to match \sphinxcode{\sphinxupquote{df3}}.
If we want to keep rows \sphinxcode{\sphinxupquote{c}} and \sphinxcode{\sphinxupquote{d}}, we can \sphinxstyleemphasis{outer} join \sphinxcode{\sphinxupquote{df3}} and \sphinxcode{\sphinxupquote{df4}} with \sphinxcode{\sphinxupquote{how='outer'}}.

\begin{sphinxuseclass}{cell}\begin{sphinxVerbatimInput}

\begin{sphinxuseclass}{cell_input}
\begin{sphinxVerbatim}[commandchars=\\\{\}]
\PYG{n}{pd}\PYG{o}{.}\PYG{n}{merge}\PYG{p}{(}\PYG{n}{df1}\PYG{p}{,} \PYG{n}{df2}\PYG{p}{,} \PYG{n}{how}\PYG{o}{=}\PYG{l+s+s1}{\PYGZsq{}}\PYG{l+s+s1}{outer}\PYG{l+s+s1}{\PYGZsq{}}\PYG{p}{)}
\end{sphinxVerbatim}

\end{sphinxuseclass}\end{sphinxVerbatimInput}
\begin{sphinxVerbatimOutput}

\begin{sphinxuseclass}{cell_output}
\begin{sphinxVerbatim}[commandchars=\\\{\}]
  key  data1  data2
0   b 0.0000 1.0000
1   b 1.0000 1.0000
2   b 6.0000 1.0000
3   a 2.0000 0.0000
4   a 4.0000 0.0000
5   a 5.0000 0.0000
6   c 3.0000    NaN
7   d    NaN 2.0000
\end{sphinxVerbatim}

\end{sphinxuseclass}\end{sphinxVerbatimOutput}

\end{sphinxuseclass}\begin{quote}

\sphinxAtStartPar
Many\sphinxhyphen{}to\sphinxhyphen{}many merges have well\sphinxhyphen{}defined, though not necessarily intuitive, behavior.
\end{quote}

\begin{sphinxuseclass}{cell}\begin{sphinxVerbatimInput}

\begin{sphinxuseclass}{cell_input}
\begin{sphinxVerbatim}[commandchars=\\\{\}]
\PYG{n}{df1} \PYG{o}{=} \PYG{n}{pd}\PYG{o}{.}\PYG{n}{DataFrame}\PYG{p}{(}\PYG{p}{\PYGZob{}}\PYG{l+s+s1}{\PYGZsq{}}\PYG{l+s+s1}{key}\PYG{l+s+s1}{\PYGZsq{}}\PYG{p}{:} \PYG{p}{[}\PYG{l+s+s1}{\PYGZsq{}}\PYG{l+s+s1}{b}\PYG{l+s+s1}{\PYGZsq{}}\PYG{p}{,} \PYG{l+s+s1}{\PYGZsq{}}\PYG{l+s+s1}{b}\PYG{l+s+s1}{\PYGZsq{}}\PYG{p}{,} \PYG{l+s+s1}{\PYGZsq{}}\PYG{l+s+s1}{a}\PYG{l+s+s1}{\PYGZsq{}}\PYG{p}{,} \PYG{l+s+s1}{\PYGZsq{}}\PYG{l+s+s1}{c}\PYG{l+s+s1}{\PYGZsq{}}\PYG{p}{,} \PYG{l+s+s1}{\PYGZsq{}}\PYG{l+s+s1}{a}\PYG{l+s+s1}{\PYGZsq{}}\PYG{p}{,} \PYG{l+s+s1}{\PYGZsq{}}\PYG{l+s+s1}{b}\PYG{l+s+s1}{\PYGZsq{}}\PYG{p}{]}\PYG{p}{,} \PYG{l+s+s1}{\PYGZsq{}}\PYG{l+s+s1}{data1}\PYG{l+s+s1}{\PYGZsq{}}\PYG{p}{:} \PYG{n+nb}{range}\PYG{p}{(}\PYG{l+m+mi}{6}\PYG{p}{)}\PYG{p}{\PYGZcb{}}\PYG{p}{)}
\PYG{n}{df2} \PYG{o}{=} \PYG{n}{pd}\PYG{o}{.}\PYG{n}{DataFrame}\PYG{p}{(}\PYG{p}{\PYGZob{}}\PYG{l+s+s1}{\PYGZsq{}}\PYG{l+s+s1}{key}\PYG{l+s+s1}{\PYGZsq{}}\PYG{p}{:} \PYG{p}{[}\PYG{l+s+s1}{\PYGZsq{}}\PYG{l+s+s1}{a}\PYG{l+s+s1}{\PYGZsq{}}\PYG{p}{,} \PYG{l+s+s1}{\PYGZsq{}}\PYG{l+s+s1}{b}\PYG{l+s+s1}{\PYGZsq{}}\PYG{p}{,} \PYG{l+s+s1}{\PYGZsq{}}\PYG{l+s+s1}{a}\PYG{l+s+s1}{\PYGZsq{}}\PYG{p}{,} \PYG{l+s+s1}{\PYGZsq{}}\PYG{l+s+s1}{b}\PYG{l+s+s1}{\PYGZsq{}}\PYG{p}{,} \PYG{l+s+s1}{\PYGZsq{}}\PYG{l+s+s1}{d}\PYG{l+s+s1}{\PYGZsq{}}\PYG{p}{]}\PYG{p}{,} \PYG{l+s+s1}{\PYGZsq{}}\PYG{l+s+s1}{data2}\PYG{l+s+s1}{\PYGZsq{}}\PYG{p}{:} \PYG{n+nb}{range}\PYG{p}{(}\PYG{l+m+mi}{5}\PYG{p}{)}\PYG{p}{\PYGZcb{}}\PYG{p}{)}
\end{sphinxVerbatim}

\end{sphinxuseclass}\end{sphinxVerbatimInput}

\end{sphinxuseclass}
\begin{sphinxuseclass}{cell}\begin{sphinxVerbatimInput}

\begin{sphinxuseclass}{cell_input}
\begin{sphinxVerbatim}[commandchars=\\\{\}]
\PYG{n}{df1}
\end{sphinxVerbatim}

\end{sphinxuseclass}\end{sphinxVerbatimInput}
\begin{sphinxVerbatimOutput}

\begin{sphinxuseclass}{cell_output}
\begin{sphinxVerbatim}[commandchars=\\\{\}]
  key  data1
0   b      0
1   b      1
2   a      2
3   c      3
4   a      4
5   b      5
\end{sphinxVerbatim}

\end{sphinxuseclass}\end{sphinxVerbatimOutput}

\end{sphinxuseclass}
\begin{sphinxuseclass}{cell}\begin{sphinxVerbatimInput}

\begin{sphinxuseclass}{cell_input}
\begin{sphinxVerbatim}[commandchars=\\\{\}]
\PYG{n}{df2}
\end{sphinxVerbatim}

\end{sphinxuseclass}\end{sphinxVerbatimInput}
\begin{sphinxVerbatimOutput}

\begin{sphinxuseclass}{cell_output}
\begin{sphinxVerbatim}[commandchars=\\\{\}]
  key  data2
0   a      0
1   b      1
2   a      2
3   b      3
4   d      4
\end{sphinxVerbatim}

\end{sphinxuseclass}\end{sphinxVerbatimOutput}

\end{sphinxuseclass}
\begin{sphinxuseclass}{cell}\begin{sphinxVerbatimInput}

\begin{sphinxuseclass}{cell_input}
\begin{sphinxVerbatim}[commandchars=\\\{\}]
\PYG{n}{pd}\PYG{o}{.}\PYG{n}{merge}\PYG{p}{(}\PYG{n}{df1}\PYG{p}{,} \PYG{n}{df2}\PYG{p}{,} \PYG{n}{on}\PYG{o}{=}\PYG{l+s+s1}{\PYGZsq{}}\PYG{l+s+s1}{key}\PYG{l+s+s1}{\PYGZsq{}}\PYG{p}{)}
\end{sphinxVerbatim}

\end{sphinxuseclass}\end{sphinxVerbatimInput}
\begin{sphinxVerbatimOutput}

\begin{sphinxuseclass}{cell_output}
\begin{sphinxVerbatim}[commandchars=\\\{\}]
  key  data1  data2
0   b      0      1
1   b      0      3
2   b      1      1
3   b      1      3
4   b      5      1
5   b      5      3
6   a      2      0
7   a      2      2
8   a      4      0
9   a      4      2
\end{sphinxVerbatim}

\end{sphinxuseclass}\end{sphinxVerbatimOutput}

\end{sphinxuseclass}\begin{quote}

\sphinxAtStartPar
Many\sphinxhyphen{}to\sphinxhyphen{}many joins form the Cartesian product of the rows. Since there were three \sphinxcode{\sphinxupquote{b}} rows in the left DataFrame and two in the right one, there are six \sphinxcode{\sphinxupquote{b}} rows in the result. The join method only affects the distinct key values appearing in the result.
\end{quote}

\sphinxAtStartPar
Be careful with many\sphinxhyphen{}to\sphinxhyphen{}many joins!
In finance, we do not expect many\sphinxhyphen{}to\sphinxhyphen{}many joins because we expect at least one of the data frames to have unique observations.
\sphinxstyleemphasis{\sphinxstylestrong{pandas will not warn us if we accidentally perform a many\sphinxhyphen{}to\sphinxhyphen{}many join instead of a one\sphinxhyphen{}to\sphinxhyphen{}one or many\sphinxhyphen{}to\sphinxhyphen{}one join.}}

\sphinxAtStartPar
We can merge on more than one key.
For example, we may merge two data sets on ticker\sphinxhyphen{}date pairs or industry\sphinxhyphen{}date pairs.

\begin{sphinxuseclass}{cell}\begin{sphinxVerbatimInput}

\begin{sphinxuseclass}{cell_input}
\begin{sphinxVerbatim}[commandchars=\\\{\}]
\PYG{n}{left} \PYG{o}{=} \PYG{n}{pd}\PYG{o}{.}\PYG{n}{DataFrame}\PYG{p}{(}\PYG{p}{\PYGZob{}}\PYG{l+s+s1}{\PYGZsq{}}\PYG{l+s+s1}{key1}\PYG{l+s+s1}{\PYGZsq{}}\PYG{p}{:} \PYG{p}{[}\PYG{l+s+s1}{\PYGZsq{}}\PYG{l+s+s1}{foo}\PYG{l+s+s1}{\PYGZsq{}}\PYG{p}{,} \PYG{l+s+s1}{\PYGZsq{}}\PYG{l+s+s1}{foo}\PYG{l+s+s1}{\PYGZsq{}}\PYG{p}{,} \PYG{l+s+s1}{\PYGZsq{}}\PYG{l+s+s1}{bar}\PYG{l+s+s1}{\PYGZsq{}}\PYG{p}{]}\PYG{p}{,}
                     \PYG{l+s+s1}{\PYGZsq{}}\PYG{l+s+s1}{key2}\PYG{l+s+s1}{\PYGZsq{}}\PYG{p}{:} \PYG{p}{[}\PYG{l+s+s1}{\PYGZsq{}}\PYG{l+s+s1}{one}\PYG{l+s+s1}{\PYGZsq{}}\PYG{p}{,} \PYG{l+s+s1}{\PYGZsq{}}\PYG{l+s+s1}{two}\PYG{l+s+s1}{\PYGZsq{}}\PYG{p}{,} \PYG{l+s+s1}{\PYGZsq{}}\PYG{l+s+s1}{one}\PYG{l+s+s1}{\PYGZsq{}}\PYG{p}{]}\PYG{p}{,}
                     \PYG{l+s+s1}{\PYGZsq{}}\PYG{l+s+s1}{lval}\PYG{l+s+s1}{\PYGZsq{}}\PYG{p}{:} \PYG{p}{[}\PYG{l+m+mi}{1}\PYG{p}{,} \PYG{l+m+mi}{2}\PYG{p}{,} \PYG{l+m+mi}{3}\PYG{p}{]}\PYG{p}{\PYGZcb{}}\PYG{p}{)}
\PYG{n}{right} \PYG{o}{=} \PYG{n}{pd}\PYG{o}{.}\PYG{n}{DataFrame}\PYG{p}{(}\PYG{p}{\PYGZob{}}\PYG{l+s+s1}{\PYGZsq{}}\PYG{l+s+s1}{key1}\PYG{l+s+s1}{\PYGZsq{}}\PYG{p}{:} \PYG{p}{[}\PYG{l+s+s1}{\PYGZsq{}}\PYG{l+s+s1}{foo}\PYG{l+s+s1}{\PYGZsq{}}\PYG{p}{,} \PYG{l+s+s1}{\PYGZsq{}}\PYG{l+s+s1}{foo}\PYG{l+s+s1}{\PYGZsq{}}\PYG{p}{,} \PYG{l+s+s1}{\PYGZsq{}}\PYG{l+s+s1}{bar}\PYG{l+s+s1}{\PYGZsq{}}\PYG{p}{,} \PYG{l+s+s1}{\PYGZsq{}}\PYG{l+s+s1}{bar}\PYG{l+s+s1}{\PYGZsq{}}\PYG{p}{]}\PYG{p}{,}
                      \PYG{l+s+s1}{\PYGZsq{}}\PYG{l+s+s1}{key2}\PYG{l+s+s1}{\PYGZsq{}}\PYG{p}{:} \PYG{p}{[}\PYG{l+s+s1}{\PYGZsq{}}\PYG{l+s+s1}{one}\PYG{l+s+s1}{\PYGZsq{}}\PYG{p}{,} \PYG{l+s+s1}{\PYGZsq{}}\PYG{l+s+s1}{one}\PYG{l+s+s1}{\PYGZsq{}}\PYG{p}{,} \PYG{l+s+s1}{\PYGZsq{}}\PYG{l+s+s1}{one}\PYG{l+s+s1}{\PYGZsq{}}\PYG{p}{,} \PYG{l+s+s1}{\PYGZsq{}}\PYG{l+s+s1}{two}\PYG{l+s+s1}{\PYGZsq{}}\PYG{p}{]}\PYG{p}{,}
                      \PYG{l+s+s1}{\PYGZsq{}}\PYG{l+s+s1}{rval}\PYG{l+s+s1}{\PYGZsq{}}\PYG{p}{:} \PYG{p}{[}\PYG{l+m+mi}{4}\PYG{p}{,} \PYG{l+m+mi}{5}\PYG{p}{,} \PYG{l+m+mi}{6}\PYG{p}{,} \PYG{l+m+mi}{7}\PYG{p}{]}\PYG{p}{\PYGZcb{}}\PYG{p}{)}
\end{sphinxVerbatim}

\end{sphinxuseclass}\end{sphinxVerbatimInput}

\end{sphinxuseclass}
\begin{sphinxuseclass}{cell}\begin{sphinxVerbatimInput}

\begin{sphinxuseclass}{cell_input}
\begin{sphinxVerbatim}[commandchars=\\\{\}]
\PYG{n}{left}
\end{sphinxVerbatim}

\end{sphinxuseclass}\end{sphinxVerbatimInput}
\begin{sphinxVerbatimOutput}

\begin{sphinxuseclass}{cell_output}
\begin{sphinxVerbatim}[commandchars=\\\{\}]
  key1 key2  lval
0  foo  one     1
1  foo  two     2
2  bar  one     3
\end{sphinxVerbatim}

\end{sphinxuseclass}\end{sphinxVerbatimOutput}

\end{sphinxuseclass}
\begin{sphinxuseclass}{cell}\begin{sphinxVerbatimInput}

\begin{sphinxuseclass}{cell_input}
\begin{sphinxVerbatim}[commandchars=\\\{\}]
\PYG{n}{right}
\end{sphinxVerbatim}

\end{sphinxuseclass}\end{sphinxVerbatimInput}
\begin{sphinxVerbatimOutput}

\begin{sphinxuseclass}{cell_output}
\begin{sphinxVerbatim}[commandchars=\\\{\}]
  key1 key2  rval
0  foo  one     4
1  foo  one     5
2  bar  one     6
3  bar  two     7
\end{sphinxVerbatim}

\end{sphinxuseclass}\end{sphinxVerbatimOutput}

\end{sphinxuseclass}
\begin{sphinxuseclass}{cell}\begin{sphinxVerbatimInput}

\begin{sphinxuseclass}{cell_input}
\begin{sphinxVerbatim}[commandchars=\\\{\}]
\PYG{n}{pd}\PYG{o}{.}\PYG{n}{merge}\PYG{p}{(}\PYG{n}{left}\PYG{p}{,} \PYG{n}{right}\PYG{p}{,} \PYG{n}{on}\PYG{o}{=}\PYG{p}{[}\PYG{l+s+s1}{\PYGZsq{}}\PYG{l+s+s1}{key1}\PYG{l+s+s1}{\PYGZsq{}}\PYG{p}{,} \PYG{l+s+s1}{\PYGZsq{}}\PYG{l+s+s1}{key2}\PYG{l+s+s1}{\PYGZsq{}}\PYG{p}{]}\PYG{p}{,} \PYG{n}{how}\PYG{o}{=}\PYG{l+s+s1}{\PYGZsq{}}\PYG{l+s+s1}{outer}\PYG{l+s+s1}{\PYGZsq{}}\PYG{p}{)}
\end{sphinxVerbatim}

\end{sphinxuseclass}\end{sphinxVerbatimInput}
\begin{sphinxVerbatimOutput}

\begin{sphinxuseclass}{cell_output}
\begin{sphinxVerbatim}[commandchars=\\\{\}]
  key1 key2   lval   rval
0  foo  one 1.0000 4.0000
1  foo  one 1.0000 5.0000
2  foo  two 2.0000    NaN
3  bar  one 3.0000 6.0000
4  bar  two    NaN 7.0000
\end{sphinxVerbatim}

\end{sphinxuseclass}\end{sphinxVerbatimOutput}

\end{sphinxuseclass}
\sphinxAtStartPar
When column names overlap between the left and right data frames, \sphinxcode{\sphinxupquote{pd.merge()}} appends \sphinxcode{\sphinxupquote{\_x}} and \sphinxcode{\sphinxupquote{\_y}} to the left and right versions of the overlapping column names.

\begin{sphinxuseclass}{cell}\begin{sphinxVerbatimInput}

\begin{sphinxuseclass}{cell_input}
\begin{sphinxVerbatim}[commandchars=\\\{\}]
\PYG{n}{pd}\PYG{o}{.}\PYG{n}{merge}\PYG{p}{(}\PYG{n}{left}\PYG{p}{,} \PYG{n}{right}\PYG{p}{,} \PYG{n}{on}\PYG{o}{=}\PYG{l+s+s1}{\PYGZsq{}}\PYG{l+s+s1}{key1}\PYG{l+s+s1}{\PYGZsq{}}\PYG{p}{)}
\end{sphinxVerbatim}

\end{sphinxuseclass}\end{sphinxVerbatimInput}
\begin{sphinxVerbatimOutput}

\begin{sphinxuseclass}{cell_output}
\begin{sphinxVerbatim}[commandchars=\\\{\}]
  key1 key2\PYGZus{}x  lval key2\PYGZus{}y  rval
0  foo    one     1    one     4
1  foo    one     1    one     5
2  foo    two     2    one     4
3  foo    two     2    one     5
4  bar    one     3    one     6
5  bar    one     3    two     7
\end{sphinxVerbatim}

\end{sphinxuseclass}\end{sphinxVerbatimOutput}

\end{sphinxuseclass}
\sphinxAtStartPar
I typically specify suffixes to avoid later confusion.

\begin{sphinxuseclass}{cell}\begin{sphinxVerbatimInput}

\begin{sphinxuseclass}{cell_input}
\begin{sphinxVerbatim}[commandchars=\\\{\}]
\PYG{n}{pd}\PYG{o}{.}\PYG{n}{merge}\PYG{p}{(}\PYG{n}{left}\PYG{p}{,} \PYG{n}{right}\PYG{p}{,} \PYG{n}{on}\PYG{o}{=}\PYG{l+s+s1}{\PYGZsq{}}\PYG{l+s+s1}{key1}\PYG{l+s+s1}{\PYGZsq{}}\PYG{p}{,} \PYG{n}{suffixes}\PYG{o}{=}\PYG{p}{(}\PYG{l+s+s1}{\PYGZsq{}}\PYG{l+s+s1}{\PYGZus{}left}\PYG{l+s+s1}{\PYGZsq{}}\PYG{p}{,} \PYG{l+s+s1}{\PYGZsq{}}\PYG{l+s+s1}{\PYGZus{}right}\PYG{l+s+s1}{\PYGZsq{}}\PYG{p}{)}\PYG{p}{)}
\end{sphinxVerbatim}

\end{sphinxuseclass}\end{sphinxVerbatimInput}
\begin{sphinxVerbatimOutput}

\begin{sphinxuseclass}{cell_output}
\begin{sphinxVerbatim}[commandchars=\\\{\}]
  key1 key2\PYGZus{}left  lval key2\PYGZus{}right  rval
0  foo       one     1        one     4
1  foo       one     1        one     5
2  foo       two     2        one     4
3  foo       two     2        one     5
4  bar       one     3        one     6
5  bar       one     3        two     7
\end{sphinxVerbatim}

\end{sphinxuseclass}\end{sphinxVerbatimOutput}

\end{sphinxuseclass}
\sphinxAtStartPar
I read the \sphinxcode{\sphinxupquote{pd.merge()}} docstring whenever I am in doubt.
\sphinxstyleemphasis{\sphinxstylestrong{Table 8\sphinxhyphen{}2}} lists the most commonly used arguments for \sphinxcode{\sphinxupquote{pd.merge()}}.
\begin{itemize}
\item {} 
\sphinxAtStartPar
\sphinxcode{\sphinxupquote{left}}: DataFrame to be merged on the left side.

\item {} 
\sphinxAtStartPar
\sphinxcode{\sphinxupquote{right}}: DataFrame to be merged on the right side.

\item {} 
\sphinxAtStartPar
\sphinxcode{\sphinxupquote{how}}: One of ‘inner’, ‘outer’, ‘left’, or ‘right’; defaults to ‘inner’.

\item {} 
\sphinxAtStartPar
\sphinxcode{\sphinxupquote{on}}: Column names to join on. Must be found in both DataFrame objects. If not specified and no other join keys given will use the intersection of the column names in left and right as the join keys.

\item {} 
\sphinxAtStartPar
\sphinxcode{\sphinxupquote{left\_on}}: Columns in left DataFrame to use as join keys.

\item {} 
\sphinxAtStartPar
\sphinxcode{\sphinxupquote{right\_on}}: Analogous to left\_on for left DataFrame.

\item {} 
\sphinxAtStartPar
\sphinxcode{\sphinxupquote{left\_index}}: Use row index in left as its join key (or keys, if a MultiIndex).

\item {} 
\sphinxAtStartPar
\sphinxcode{\sphinxupquote{right\_index}}: Analogous to left\_index.

\item {} 
\sphinxAtStartPar
\sphinxcode{\sphinxupquote{sort}}: Sort merged data lexicographically by join keys; True by default (disable to get better performance in some cases on large datasets).

\item {} 
\sphinxAtStartPar
\sphinxcode{\sphinxupquote{suffixes}}: Tuple of string values to append to column names in case of overlap; defaults to (‘\_x’, ‘\_y’) (e.g., if ‘data’ in both DataFrame objects, would appear as ‘data\_x’ and ‘data\_y’ in result).

\item {} 
\sphinxAtStartPar
\sphinxcode{\sphinxupquote{copy}}: If False, avoid copying data into resulting data structure in some exceptional cases; by default always copies.

\item {} 
\sphinxAtStartPar
\sphinxcode{\sphinxupquote{indicator}}: Adds a special column \_merge that indicates the source of each row; values will be ‘left\_only’, ‘right\_only’, or ‘both’ based on the origin of the joined data in each row.

\end{itemize}


\subsection{Merging on Index}
\label{\detokenize{mckinney_08_lecture:merging-on-index}}
\sphinxAtStartPar
If we want to use \sphinxcode{\sphinxupquote{pd.merge()}} to join on row indexes, we can use the \sphinxcode{\sphinxupquote{left\_index}} and \sphinxcode{\sphinxupquote{right\_index}} arguments.

\begin{sphinxuseclass}{cell}\begin{sphinxVerbatimInput}

\begin{sphinxuseclass}{cell_input}
\begin{sphinxVerbatim}[commandchars=\\\{\}]
\PYG{n}{left1} \PYG{o}{=} \PYG{n}{pd}\PYG{o}{.}\PYG{n}{DataFrame}\PYG{p}{(}\PYG{p}{\PYGZob{}}\PYG{l+s+s1}{\PYGZsq{}}\PYG{l+s+s1}{key}\PYG{l+s+s1}{\PYGZsq{}}\PYG{p}{:} \PYG{p}{[}\PYG{l+s+s1}{\PYGZsq{}}\PYG{l+s+s1}{a}\PYG{l+s+s1}{\PYGZsq{}}\PYG{p}{,} \PYG{l+s+s1}{\PYGZsq{}}\PYG{l+s+s1}{b}\PYG{l+s+s1}{\PYGZsq{}}\PYG{p}{,} \PYG{l+s+s1}{\PYGZsq{}}\PYG{l+s+s1}{a}\PYG{l+s+s1}{\PYGZsq{}}\PYG{p}{,} \PYG{l+s+s1}{\PYGZsq{}}\PYG{l+s+s1}{a}\PYG{l+s+s1}{\PYGZsq{}}\PYG{p}{,} \PYG{l+s+s1}{\PYGZsq{}}\PYG{l+s+s1}{b}\PYG{l+s+s1}{\PYGZsq{}}\PYG{p}{,} \PYG{l+s+s1}{\PYGZsq{}}\PYG{l+s+s1}{c}\PYG{l+s+s1}{\PYGZsq{}}\PYG{p}{]}\PYG{p}{,} \PYG{l+s+s1}{\PYGZsq{}}\PYG{l+s+s1}{value}\PYG{l+s+s1}{\PYGZsq{}}\PYG{p}{:} \PYG{n+nb}{range}\PYG{p}{(}\PYG{l+m+mi}{6}\PYG{p}{)}\PYG{p}{\PYGZcb{}}\PYG{p}{)}
\PYG{n}{right1} \PYG{o}{=} \PYG{n}{pd}\PYG{o}{.}\PYG{n}{DataFrame}\PYG{p}{(}\PYG{p}{\PYGZob{}}\PYG{l+s+s1}{\PYGZsq{}}\PYG{l+s+s1}{group\PYGZus{}val}\PYG{l+s+s1}{\PYGZsq{}}\PYG{p}{:} \PYG{p}{[}\PYG{l+m+mf}{3.5}\PYG{p}{,} \PYG{l+m+mi}{7}\PYG{p}{]}\PYG{p}{\PYGZcb{}}\PYG{p}{,} \PYG{n}{index}\PYG{o}{=}\PYG{p}{[}\PYG{l+s+s1}{\PYGZsq{}}\PYG{l+s+s1}{a}\PYG{l+s+s1}{\PYGZsq{}}\PYG{p}{,} \PYG{l+s+s1}{\PYGZsq{}}\PYG{l+s+s1}{b}\PYG{l+s+s1}{\PYGZsq{}}\PYG{p}{]}\PYG{p}{)}
\end{sphinxVerbatim}

\end{sphinxuseclass}\end{sphinxVerbatimInput}

\end{sphinxuseclass}
\begin{sphinxuseclass}{cell}\begin{sphinxVerbatimInput}

\begin{sphinxuseclass}{cell_input}
\begin{sphinxVerbatim}[commandchars=\\\{\}]
\PYG{n}{left1}
\end{sphinxVerbatim}

\end{sphinxuseclass}\end{sphinxVerbatimInput}
\begin{sphinxVerbatimOutput}

\begin{sphinxuseclass}{cell_output}
\begin{sphinxVerbatim}[commandchars=\\\{\}]
  key  value
0   a      0
1   b      1
2   a      2
3   a      3
4   b      4
5   c      5
\end{sphinxVerbatim}

\end{sphinxuseclass}\end{sphinxVerbatimOutput}

\end{sphinxuseclass}
\begin{sphinxuseclass}{cell}\begin{sphinxVerbatimInput}

\begin{sphinxuseclass}{cell_input}
\begin{sphinxVerbatim}[commandchars=\\\{\}]
\PYG{n}{right1}
\end{sphinxVerbatim}

\end{sphinxuseclass}\end{sphinxVerbatimInput}
\begin{sphinxVerbatimOutput}

\begin{sphinxuseclass}{cell_output}
\begin{sphinxVerbatim}[commandchars=\\\{\}]
   group\PYGZus{}val
a     3.5000
b     7.0000
\end{sphinxVerbatim}

\end{sphinxuseclass}\end{sphinxVerbatimOutput}

\end{sphinxuseclass}
\begin{sphinxuseclass}{cell}\begin{sphinxVerbatimInput}

\begin{sphinxuseclass}{cell_input}
\begin{sphinxVerbatim}[commandchars=\\\{\}]
\PYG{n}{pd}\PYG{o}{.}\PYG{n}{merge}\PYG{p}{(}\PYG{n}{left1}\PYG{p}{,} \PYG{n}{right1}\PYG{p}{,} \PYG{n}{left\PYGZus{}on}\PYG{o}{=}\PYG{l+s+s1}{\PYGZsq{}}\PYG{l+s+s1}{key}\PYG{l+s+s1}{\PYGZsq{}}\PYG{p}{,} \PYG{n}{right\PYGZus{}index}\PYG{o}{=}\PYG{k+kc}{True}\PYG{p}{,} \PYG{n}{how}\PYG{o}{=}\PYG{l+s+s1}{\PYGZsq{}}\PYG{l+s+s1}{outer}\PYG{l+s+s1}{\PYGZsq{}}\PYG{p}{)}
\end{sphinxVerbatim}

\end{sphinxuseclass}\end{sphinxVerbatimInput}
\begin{sphinxVerbatimOutput}

\begin{sphinxuseclass}{cell_output}
\begin{sphinxVerbatim}[commandchars=\\\{\}]
  key  value  group\PYGZus{}val
0   a      0     3.5000
2   a      2     3.5000
3   a      3     3.5000
1   b      1     7.0000
4   b      4     7.0000
5   c      5        NaN
\end{sphinxVerbatim}

\end{sphinxuseclass}\end{sphinxVerbatimOutput}

\end{sphinxuseclass}
\sphinxAtStartPar
The index arguments work for hierarchical indexes (multi indexes), too.

\begin{sphinxuseclass}{cell}\begin{sphinxVerbatimInput}

\begin{sphinxuseclass}{cell_input}
\begin{sphinxVerbatim}[commandchars=\\\{\}]
\PYG{n}{lefth} \PYG{o}{=} \PYG{n}{pd}\PYG{o}{.}\PYG{n}{DataFrame}\PYG{p}{(}\PYG{p}{\PYGZob{}}\PYG{l+s+s1}{\PYGZsq{}}\PYG{l+s+s1}{key1}\PYG{l+s+s1}{\PYGZsq{}}\PYG{p}{:} \PYG{p}{[}\PYG{l+s+s1}{\PYGZsq{}}\PYG{l+s+s1}{Ohio}\PYG{l+s+s1}{\PYGZsq{}}\PYG{p}{,} \PYG{l+s+s1}{\PYGZsq{}}\PYG{l+s+s1}{Ohio}\PYG{l+s+s1}{\PYGZsq{}}\PYG{p}{,} \PYG{l+s+s1}{\PYGZsq{}}\PYG{l+s+s1}{Ohio}\PYG{l+s+s1}{\PYGZsq{}}\PYG{p}{,} \PYG{l+s+s1}{\PYGZsq{}}\PYG{l+s+s1}{Nevada}\PYG{l+s+s1}{\PYGZsq{}}\PYG{p}{,} \PYG{l+s+s1}{\PYGZsq{}}\PYG{l+s+s1}{Nevada}\PYG{l+s+s1}{\PYGZsq{}}\PYG{p}{]}\PYG{p}{,}
                      \PYG{l+s+s1}{\PYGZsq{}}\PYG{l+s+s1}{key2}\PYG{l+s+s1}{\PYGZsq{}}\PYG{p}{:} \PYG{p}{[}\PYG{l+m+mi}{2000}\PYG{p}{,} \PYG{l+m+mi}{2001}\PYG{p}{,} \PYG{l+m+mi}{2002}\PYG{p}{,} \PYG{l+m+mi}{2001}\PYG{p}{,} \PYG{l+m+mi}{2002}\PYG{p}{]}\PYG{p}{,}
                      \PYG{l+s+s1}{\PYGZsq{}}\PYG{l+s+s1}{data}\PYG{l+s+s1}{\PYGZsq{}}\PYG{p}{:} \PYG{n}{np}\PYG{o}{.}\PYG{n}{arange}\PYG{p}{(}\PYG{l+m+mf}{5.}\PYG{p}{)}\PYG{p}{\PYGZcb{}}\PYG{p}{)}
\PYG{n}{righth} \PYG{o}{=} \PYG{n}{pd}\PYG{o}{.}\PYG{n}{DataFrame}\PYG{p}{(}\PYG{n}{np}\PYG{o}{.}\PYG{n}{arange}\PYG{p}{(}\PYG{l+m+mi}{12}\PYG{p}{)}\PYG{o}{.}\PYG{n}{reshape}\PYG{p}{(}\PYG{p}{(}\PYG{l+m+mi}{6}\PYG{p}{,} \PYG{l+m+mi}{2}\PYG{p}{)}\PYG{p}{)}\PYG{p}{,}
                      \PYG{n}{index}\PYG{o}{=}\PYG{p}{[}\PYG{p}{[}\PYG{l+s+s1}{\PYGZsq{}}\PYG{l+s+s1}{Nevada}\PYG{l+s+s1}{\PYGZsq{}}\PYG{p}{,} \PYG{l+s+s1}{\PYGZsq{}}\PYG{l+s+s1}{Nevada}\PYG{l+s+s1}{\PYGZsq{}}\PYG{p}{,} \PYG{l+s+s1}{\PYGZsq{}}\PYG{l+s+s1}{Ohio}\PYG{l+s+s1}{\PYGZsq{}}\PYG{p}{,} \PYG{l+s+s1}{\PYGZsq{}}\PYG{l+s+s1}{Ohio}\PYG{l+s+s1}{\PYGZsq{}}\PYG{p}{,} \PYG{l+s+s1}{\PYGZsq{}}\PYG{l+s+s1}{Ohio}\PYG{l+s+s1}{\PYGZsq{}}\PYG{p}{,} \PYG{l+s+s1}{\PYGZsq{}}\PYG{l+s+s1}{Ohio}\PYG{l+s+s1}{\PYGZsq{}}\PYG{p}{]}\PYG{p}{,}
                             \PYG{p}{[}\PYG{l+m+mi}{2001}\PYG{p}{,} \PYG{l+m+mi}{2000}\PYG{p}{,} \PYG{l+m+mi}{2000}\PYG{p}{,} \PYG{l+m+mi}{2000}\PYG{p}{,} \PYG{l+m+mi}{2001}\PYG{p}{,} \PYG{l+m+mi}{2002}\PYG{p}{]}\PYG{p}{]}\PYG{p}{,}
                      \PYG{n}{columns}\PYG{o}{=}\PYG{p}{[}\PYG{l+s+s1}{\PYGZsq{}}\PYG{l+s+s1}{event1}\PYG{l+s+s1}{\PYGZsq{}}\PYG{p}{,} \PYG{l+s+s1}{\PYGZsq{}}\PYG{l+s+s1}{event2}\PYG{l+s+s1}{\PYGZsq{}}\PYG{p}{]}\PYG{p}{)}
\end{sphinxVerbatim}

\end{sphinxuseclass}\end{sphinxVerbatimInput}

\end{sphinxuseclass}
\begin{sphinxuseclass}{cell}\begin{sphinxVerbatimInput}

\begin{sphinxuseclass}{cell_input}
\begin{sphinxVerbatim}[commandchars=\\\{\}]
\PYG{n}{pd}\PYG{o}{.}\PYG{n}{merge}\PYG{p}{(}\PYG{n}{lefth}\PYG{p}{,} \PYG{n}{righth}\PYG{p}{,} \PYG{n}{left\PYGZus{}on}\PYG{o}{=}\PYG{p}{[}\PYG{l+s+s1}{\PYGZsq{}}\PYG{l+s+s1}{key1}\PYG{l+s+s1}{\PYGZsq{}}\PYG{p}{,} \PYG{l+s+s1}{\PYGZsq{}}\PYG{l+s+s1}{key2}\PYG{l+s+s1}{\PYGZsq{}}\PYG{p}{]}\PYG{p}{,} \PYG{n}{right\PYGZus{}index}\PYG{o}{=}\PYG{k+kc}{True}\PYG{p}{,} \PYG{n}{how}\PYG{o}{=}\PYG{l+s+s1}{\PYGZsq{}}\PYG{l+s+s1}{outer}\PYG{l+s+s1}{\PYGZsq{}}\PYG{p}{)}
\end{sphinxVerbatim}

\end{sphinxuseclass}\end{sphinxVerbatimInput}
\begin{sphinxVerbatimOutput}

\begin{sphinxuseclass}{cell_output}
\begin{sphinxVerbatim}[commandchars=\\\{\}]
     key1  key2   data  event1  event2
0    Ohio  2000 0.0000  4.0000  5.0000
0    Ohio  2000 0.0000  6.0000  7.0000
1    Ohio  2001 1.0000  8.0000  9.0000
2    Ohio  2002 2.0000 10.0000 11.0000
3  Nevada  2001 3.0000  0.0000  1.0000
4  Nevada  2002 4.0000     NaN     NaN
4  Nevada  2000    NaN  2.0000  3.0000
\end{sphinxVerbatim}

\end{sphinxuseclass}\end{sphinxVerbatimOutput}

\end{sphinxuseclass}
\begin{sphinxuseclass}{cell}\begin{sphinxVerbatimInput}

\begin{sphinxuseclass}{cell_input}
\begin{sphinxVerbatim}[commandchars=\\\{\}]
\PYG{n}{left2} \PYG{o}{=} \PYG{n}{pd}\PYG{o}{.}\PYG{n}{DataFrame}\PYG{p}{(}\PYG{p}{[}\PYG{p}{[}\PYG{l+m+mf}{1.}\PYG{p}{,} \PYG{l+m+mf}{2.}\PYG{p}{]}\PYG{p}{,} \PYG{p}{[}\PYG{l+m+mf}{3.}\PYG{p}{,} \PYG{l+m+mf}{4.}\PYG{p}{]}\PYG{p}{,} \PYG{p}{[}\PYG{l+m+mf}{5.}\PYG{p}{,} \PYG{l+m+mf}{6.}\PYG{p}{]}\PYG{p}{]}\PYG{p}{,}
                     \PYG{n}{index}\PYG{o}{=}\PYG{p}{[}\PYG{l+s+s1}{\PYGZsq{}}\PYG{l+s+s1}{a}\PYG{l+s+s1}{\PYGZsq{}}\PYG{p}{,} \PYG{l+s+s1}{\PYGZsq{}}\PYG{l+s+s1}{c}\PYG{l+s+s1}{\PYGZsq{}}\PYG{p}{,} \PYG{l+s+s1}{\PYGZsq{}}\PYG{l+s+s1}{e}\PYG{l+s+s1}{\PYGZsq{}}\PYG{p}{]}\PYG{p}{,}
                     \PYG{n}{columns}\PYG{o}{=}\PYG{p}{[}\PYG{l+s+s1}{\PYGZsq{}}\PYG{l+s+s1}{Ohio}\PYG{l+s+s1}{\PYGZsq{}}\PYG{p}{,} \PYG{l+s+s1}{\PYGZsq{}}\PYG{l+s+s1}{Nevada}\PYG{l+s+s1}{\PYGZsq{}}\PYG{p}{]}\PYG{p}{)}
\PYG{n}{right2} \PYG{o}{=} \PYG{n}{pd}\PYG{o}{.}\PYG{n}{DataFrame}\PYG{p}{(}\PYG{p}{[}\PYG{p}{[}\PYG{l+m+mf}{7.}\PYG{p}{,} \PYG{l+m+mf}{8.}\PYG{p}{]}\PYG{p}{,} \PYG{p}{[}\PYG{l+m+mf}{9.}\PYG{p}{,} \PYG{l+m+mf}{10.}\PYG{p}{]}\PYG{p}{,} \PYG{p}{[}\PYG{l+m+mf}{11.}\PYG{p}{,} \PYG{l+m+mf}{12.}\PYG{p}{]}\PYG{p}{,} \PYG{p}{[}\PYG{l+m+mi}{13}\PYG{p}{,} \PYG{l+m+mi}{14}\PYG{p}{]}\PYG{p}{]}\PYG{p}{,}
                      \PYG{n}{index}\PYG{o}{=}\PYG{p}{[}\PYG{l+s+s1}{\PYGZsq{}}\PYG{l+s+s1}{b}\PYG{l+s+s1}{\PYGZsq{}}\PYG{p}{,} \PYG{l+s+s1}{\PYGZsq{}}\PYG{l+s+s1}{c}\PYG{l+s+s1}{\PYGZsq{}}\PYG{p}{,} \PYG{l+s+s1}{\PYGZsq{}}\PYG{l+s+s1}{d}\PYG{l+s+s1}{\PYGZsq{}}\PYG{p}{,} \PYG{l+s+s1}{\PYGZsq{}}\PYG{l+s+s1}{e}\PYG{l+s+s1}{\PYGZsq{}}\PYG{p}{]}\PYG{p}{,}
                      \PYG{n}{columns}\PYG{o}{=}\PYG{p}{[}\PYG{l+s+s1}{\PYGZsq{}}\PYG{l+s+s1}{Missouri}\PYG{l+s+s1}{\PYGZsq{}}\PYG{p}{,} \PYG{l+s+s1}{\PYGZsq{}}\PYG{l+s+s1}{Alabama}\PYG{l+s+s1}{\PYGZsq{}}\PYG{p}{]}\PYG{p}{)}
\end{sphinxVerbatim}

\end{sphinxuseclass}\end{sphinxVerbatimInput}

\end{sphinxuseclass}
\sphinxAtStartPar
If we use both left and right indexes, \sphinxcode{\sphinxupquote{pd.merge()}} will keep the index.

\begin{sphinxuseclass}{cell}\begin{sphinxVerbatimInput}

\begin{sphinxuseclass}{cell_input}
\begin{sphinxVerbatim}[commandchars=\\\{\}]
\PYG{n}{pd}\PYG{o}{.}\PYG{n}{merge}\PYG{p}{(}\PYG{n}{left2}\PYG{p}{,} \PYG{n}{right2}\PYG{p}{,} \PYG{n}{how}\PYG{o}{=}\PYG{l+s+s1}{\PYGZsq{}}\PYG{l+s+s1}{outer}\PYG{l+s+s1}{\PYGZsq{}}\PYG{p}{,} \PYG{n}{left\PYGZus{}index}\PYG{o}{=}\PYG{k+kc}{True}\PYG{p}{,} \PYG{n}{right\PYGZus{}index}\PYG{o}{=}\PYG{k+kc}{True}\PYG{p}{)}
\end{sphinxVerbatim}

\end{sphinxuseclass}\end{sphinxVerbatimInput}
\begin{sphinxVerbatimOutput}

\begin{sphinxuseclass}{cell_output}
\begin{sphinxVerbatim}[commandchars=\\\{\}]
    Ohio  Nevada  Missouri  Alabama
a 1.0000  2.0000       NaN      NaN
b    NaN     NaN    7.0000   8.0000
c 3.0000  4.0000    9.0000  10.0000
d    NaN     NaN   11.0000  12.0000
e 5.0000  6.0000   13.0000  14.0000
\end{sphinxVerbatim}

\end{sphinxuseclass}\end{sphinxVerbatimOutput}

\end{sphinxuseclass}\begin{quote}

\sphinxAtStartPar
DataFrame has a convenient join instance for merging by index. It can also be used to combine together many DataFrame objects having the same or similar indexes but non\sphinxhyphen{}overlapping columns.
\end{quote}

\sphinxAtStartPar
If we have matching indexes on left and right, we can use \sphinxcode{\sphinxupquote{.join()}}.

\begin{sphinxuseclass}{cell}\begin{sphinxVerbatimInput}

\begin{sphinxuseclass}{cell_input}
\begin{sphinxVerbatim}[commandchars=\\\{\}]
\PYG{n}{left2}
\end{sphinxVerbatim}

\end{sphinxuseclass}\end{sphinxVerbatimInput}
\begin{sphinxVerbatimOutput}

\begin{sphinxuseclass}{cell_output}
\begin{sphinxVerbatim}[commandchars=\\\{\}]
    Ohio  Nevada
a 1.0000  2.0000
c 3.0000  4.0000
e 5.0000  6.0000
\end{sphinxVerbatim}

\end{sphinxuseclass}\end{sphinxVerbatimOutput}

\end{sphinxuseclass}
\begin{sphinxuseclass}{cell}\begin{sphinxVerbatimInput}

\begin{sphinxuseclass}{cell_input}
\begin{sphinxVerbatim}[commandchars=\\\{\}]
\PYG{n}{right2}
\end{sphinxVerbatim}

\end{sphinxuseclass}\end{sphinxVerbatimInput}
\begin{sphinxVerbatimOutput}

\begin{sphinxuseclass}{cell_output}
\begin{sphinxVerbatim}[commandchars=\\\{\}]
   Missouri  Alabama
b    7.0000   8.0000
c    9.0000  10.0000
d   11.0000  12.0000
e   13.0000  14.0000
\end{sphinxVerbatim}

\end{sphinxuseclass}\end{sphinxVerbatimOutput}

\end{sphinxuseclass}
\begin{sphinxuseclass}{cell}\begin{sphinxVerbatimInput}

\begin{sphinxuseclass}{cell_input}
\begin{sphinxVerbatim}[commandchars=\\\{\}]
\PYG{n}{left2}\PYG{o}{.}\PYG{n}{join}\PYG{p}{(}\PYG{n}{right2}\PYG{p}{,} \PYG{n}{how}\PYG{o}{=}\PYG{l+s+s1}{\PYGZsq{}}\PYG{l+s+s1}{outer}\PYG{l+s+s1}{\PYGZsq{}}\PYG{p}{)}
\end{sphinxVerbatim}

\end{sphinxuseclass}\end{sphinxVerbatimInput}
\begin{sphinxVerbatimOutput}

\begin{sphinxuseclass}{cell_output}
\begin{sphinxVerbatim}[commandchars=\\\{\}]
    Ohio  Nevada  Missouri  Alabama
a 1.0000  2.0000       NaN      NaN
b    NaN     NaN    7.0000   8.0000
c 3.0000  4.0000    9.0000  10.0000
d    NaN     NaN   11.0000  12.0000
e 5.0000  6.0000   13.0000  14.0000
\end{sphinxVerbatim}

\end{sphinxuseclass}\end{sphinxVerbatimOutput}

\end{sphinxuseclass}
\sphinxAtStartPar
The \sphinxcode{\sphinxupquote{.join()}} method left joins by default.
The \sphinxcode{\sphinxupquote{.join()}} method uses indexes, so it requires few arguments and accepts a list of data frames.

\begin{sphinxuseclass}{cell}\begin{sphinxVerbatimInput}

\begin{sphinxuseclass}{cell_input}
\begin{sphinxVerbatim}[commandchars=\\\{\}]
\PYG{n}{another} \PYG{o}{=} \PYG{n}{pd}\PYG{o}{.}\PYG{n}{DataFrame}\PYG{p}{(}
    \PYG{n}{data}\PYG{o}{=}\PYG{p}{[}\PYG{p}{[}\PYG{l+m+mf}{7.}\PYG{p}{,} \PYG{l+m+mf}{8.}\PYG{p}{]}\PYG{p}{,} \PYG{p}{[}\PYG{l+m+mf}{9.}\PYG{p}{,} \PYG{l+m+mf}{10.}\PYG{p}{]}\PYG{p}{,} \PYG{p}{[}\PYG{l+m+mf}{11.}\PYG{p}{,} \PYG{l+m+mf}{12.}\PYG{p}{]}\PYG{p}{,} \PYG{p}{[}\PYG{l+m+mf}{16.}\PYG{p}{,} \PYG{l+m+mf}{17.}\PYG{p}{]}\PYG{p}{]}\PYG{p}{,}
    \PYG{n}{index}\PYG{o}{=}\PYG{p}{[}\PYG{l+s+s1}{\PYGZsq{}}\PYG{l+s+s1}{a}\PYG{l+s+s1}{\PYGZsq{}}\PYG{p}{,} \PYG{l+s+s1}{\PYGZsq{}}\PYG{l+s+s1}{c}\PYG{l+s+s1}{\PYGZsq{}}\PYG{p}{,} \PYG{l+s+s1}{\PYGZsq{}}\PYG{l+s+s1}{e}\PYG{l+s+s1}{\PYGZsq{}}\PYG{p}{,} \PYG{l+s+s1}{\PYGZsq{}}\PYG{l+s+s1}{f}\PYG{l+s+s1}{\PYGZsq{}}\PYG{p}{]}\PYG{p}{,}
    \PYG{n}{columns}\PYG{o}{=}\PYG{p}{[}\PYG{l+s+s1}{\PYGZsq{}}\PYG{l+s+s1}{New York}\PYG{l+s+s1}{\PYGZsq{}}\PYG{p}{,} \PYG{l+s+s1}{\PYGZsq{}}\PYG{l+s+s1}{Oregon}\PYG{l+s+s1}{\PYGZsq{}}\PYG{p}{]}
\PYG{p}{)}

\PYG{n}{another}
\end{sphinxVerbatim}

\end{sphinxuseclass}\end{sphinxVerbatimInput}
\begin{sphinxVerbatimOutput}

\begin{sphinxuseclass}{cell_output}
\begin{sphinxVerbatim}[commandchars=\\\{\}]
   New York  Oregon
a    7.0000  8.0000
c    9.0000 10.0000
e   11.0000 12.0000
f   16.0000 17.0000
\end{sphinxVerbatim}

\end{sphinxuseclass}\end{sphinxVerbatimOutput}

\end{sphinxuseclass}
\begin{sphinxuseclass}{cell}\begin{sphinxVerbatimInput}

\begin{sphinxuseclass}{cell_input}
\begin{sphinxVerbatim}[commandchars=\\\{\}]
\PYG{n}{left2}\PYG{o}{.}\PYG{n}{join}\PYG{p}{(}\PYG{p}{[}\PYG{n}{right2}\PYG{p}{,} \PYG{n}{another}\PYG{p}{]}\PYG{p}{)}
\end{sphinxVerbatim}

\end{sphinxuseclass}\end{sphinxVerbatimInput}
\begin{sphinxVerbatimOutput}

\begin{sphinxuseclass}{cell_output}
\begin{sphinxVerbatim}[commandchars=\\\{\}]
    Ohio  Nevada  Missouri  Alabama  New York  Oregon
a 1.0000  2.0000       NaN      NaN    7.0000  8.0000
c 3.0000  4.0000    9.0000  10.0000    9.0000 10.0000
e 5.0000  6.0000   13.0000  14.0000   11.0000 12.0000
\end{sphinxVerbatim}

\end{sphinxuseclass}\end{sphinxVerbatimOutput}

\end{sphinxuseclass}
\begin{sphinxuseclass}{cell}\begin{sphinxVerbatimInput}

\begin{sphinxuseclass}{cell_input}
\begin{sphinxVerbatim}[commandchars=\\\{\}]
\PYG{n}{left2}\PYG{o}{.}\PYG{n}{join}\PYG{p}{(}\PYG{p}{[}\PYG{n}{right2}\PYG{p}{,} \PYG{n}{another}\PYG{p}{]}\PYG{p}{,} \PYG{n}{how}\PYG{o}{=}\PYG{l+s+s1}{\PYGZsq{}}\PYG{l+s+s1}{outer}\PYG{l+s+s1}{\PYGZsq{}}\PYG{p}{)}
\end{sphinxVerbatim}

\end{sphinxuseclass}\end{sphinxVerbatimInput}
\begin{sphinxVerbatimOutput}

\begin{sphinxuseclass}{cell_output}
\begin{sphinxVerbatim}[commandchars=\\\{\}]
    Ohio  Nevada  Missouri  Alabama  New York  Oregon
a 1.0000  2.0000       NaN      NaN    7.0000  8.0000
c 3.0000  4.0000    9.0000  10.0000    9.0000 10.0000
e 5.0000  6.0000   13.0000  14.0000   11.0000 12.0000
b    NaN     NaN    7.0000   8.0000       NaN     NaN
d    NaN     NaN   11.0000  12.0000       NaN     NaN
f    NaN     NaN       NaN      NaN   16.0000 17.0000
\end{sphinxVerbatim}

\end{sphinxuseclass}\end{sphinxVerbatimOutput}

\end{sphinxuseclass}

\subsection{Concatenating Along an Axis}
\label{\detokenize{mckinney_08_lecture:concatenating-along-an-axis}}
\sphinxAtStartPar
The \sphinxcode{\sphinxupquote{pd.concat()}} function provides a flexible way to combine data frames and series along either axis.
I typically use \sphinxcode{\sphinxupquote{pd.concat()}} to combine:
\begin{enumerate}
\sphinxsetlistlabels{\arabic}{enumi}{enumii}{}{.}%
\item {} 
\sphinxAtStartPar
A list of data frames with similar layouts

\item {} 
\sphinxAtStartPar
A list of series because series do not have \sphinxcode{\sphinxupquote{.join()}} or \sphinxcode{\sphinxupquote{.merge()}} methods

\end{enumerate}

\sphinxAtStartPar
The first is handy if we have to read and combine a directory of .csv files.

\begin{sphinxuseclass}{cell}\begin{sphinxVerbatimInput}

\begin{sphinxuseclass}{cell_input}
\begin{sphinxVerbatim}[commandchars=\\\{\}]
\PYG{n}{s1} \PYG{o}{=} \PYG{n}{pd}\PYG{o}{.}\PYG{n}{Series}\PYG{p}{(}\PYG{p}{[}\PYG{l+m+mi}{0}\PYG{p}{,} \PYG{l+m+mi}{1}\PYG{p}{]}\PYG{p}{,} \PYG{n}{index}\PYG{o}{=}\PYG{p}{[}\PYG{l+s+s1}{\PYGZsq{}}\PYG{l+s+s1}{a}\PYG{l+s+s1}{\PYGZsq{}}\PYG{p}{,} \PYG{l+s+s1}{\PYGZsq{}}\PYG{l+s+s1}{b}\PYG{l+s+s1}{\PYGZsq{}}\PYG{p}{]}\PYG{p}{)}
\PYG{n}{s2} \PYG{o}{=} \PYG{n}{pd}\PYG{o}{.}\PYG{n}{Series}\PYG{p}{(}\PYG{p}{[}\PYG{l+m+mi}{2}\PYG{p}{,} \PYG{l+m+mi}{3}\PYG{p}{,} \PYG{l+m+mi}{4}\PYG{p}{]}\PYG{p}{,} \PYG{n}{index}\PYG{o}{=}\PYG{p}{[}\PYG{l+s+s1}{\PYGZsq{}}\PYG{l+s+s1}{c}\PYG{l+s+s1}{\PYGZsq{}}\PYG{p}{,} \PYG{l+s+s1}{\PYGZsq{}}\PYG{l+s+s1}{d}\PYG{l+s+s1}{\PYGZsq{}}\PYG{p}{,} \PYG{l+s+s1}{\PYGZsq{}}\PYG{l+s+s1}{e}\PYG{l+s+s1}{\PYGZsq{}}\PYG{p}{]}\PYG{p}{)}
\PYG{n}{s3} \PYG{o}{=} \PYG{n}{pd}\PYG{o}{.}\PYG{n}{Series}\PYG{p}{(}\PYG{p}{[}\PYG{l+m+mi}{5}\PYG{p}{,} \PYG{l+m+mi}{6}\PYG{p}{]}\PYG{p}{,} \PYG{n}{index}\PYG{o}{=}\PYG{p}{[}\PYG{l+s+s1}{\PYGZsq{}}\PYG{l+s+s1}{f}\PYG{l+s+s1}{\PYGZsq{}}\PYG{p}{,} \PYG{l+s+s1}{\PYGZsq{}}\PYG{l+s+s1}{g}\PYG{l+s+s1}{\PYGZsq{}}\PYG{p}{]}\PYG{p}{)}
\end{sphinxVerbatim}

\end{sphinxuseclass}\end{sphinxVerbatimInput}

\end{sphinxuseclass}
\begin{sphinxuseclass}{cell}\begin{sphinxVerbatimInput}

\begin{sphinxuseclass}{cell_input}
\begin{sphinxVerbatim}[commandchars=\\\{\}]
\PYG{n}{pd}\PYG{o}{.}\PYG{n}{concat}\PYG{p}{(}\PYG{p}{[}\PYG{n}{s1}\PYG{p}{,} \PYG{n}{s2}\PYG{p}{,} \PYG{n}{s3}\PYG{p}{]}\PYG{p}{)}
\end{sphinxVerbatim}

\end{sphinxuseclass}\end{sphinxVerbatimInput}
\begin{sphinxVerbatimOutput}

\begin{sphinxuseclass}{cell_output}
\begin{sphinxVerbatim}[commandchars=\\\{\}]
a    0
b    1
c    2
d    3
e    4
f    5
g    6
dtype: int64
\end{sphinxVerbatim}

\end{sphinxuseclass}\end{sphinxVerbatimOutput}

\end{sphinxuseclass}
\begin{sphinxuseclass}{cell}\begin{sphinxVerbatimInput}

\begin{sphinxuseclass}{cell_input}
\begin{sphinxVerbatim}[commandchars=\\\{\}]
\PYG{n}{pd}\PYG{o}{.}\PYG{n}{concat}\PYG{p}{(}\PYG{p}{[}\PYG{n}{s1}\PYG{p}{,} \PYG{n}{s2}\PYG{p}{,} \PYG{n}{s3}\PYG{p}{]}\PYG{p}{,} \PYG{n}{axis}\PYG{o}{=}\PYG{l+m+mi}{1}\PYG{p}{)}
\end{sphinxVerbatim}

\end{sphinxuseclass}\end{sphinxVerbatimInput}
\begin{sphinxVerbatimOutput}

\begin{sphinxuseclass}{cell_output}
\begin{sphinxVerbatim}[commandchars=\\\{\}]
       0      1      2
a 0.0000    NaN    NaN
b 1.0000    NaN    NaN
c    NaN 2.0000    NaN
d    NaN 3.0000    NaN
e    NaN 4.0000    NaN
f    NaN    NaN 5.0000
g    NaN    NaN 6.0000
\end{sphinxVerbatim}

\end{sphinxuseclass}\end{sphinxVerbatimOutput}

\end{sphinxuseclass}
\begin{sphinxuseclass}{cell}\begin{sphinxVerbatimInput}

\begin{sphinxuseclass}{cell_input}
\begin{sphinxVerbatim}[commandchars=\\\{\}]
\PYG{n}{result} \PYG{o}{=} \PYG{n}{pd}\PYG{o}{.}\PYG{n}{concat}\PYG{p}{(}\PYG{p}{[}\PYG{n}{s1}\PYG{p}{,} \PYG{n}{s2}\PYG{p}{,} \PYG{n}{s3}\PYG{p}{]}\PYG{p}{,} \PYG{n}{keys}\PYG{o}{=}\PYG{p}{[}\PYG{l+s+s1}{\PYGZsq{}}\PYG{l+s+s1}{one}\PYG{l+s+s1}{\PYGZsq{}}\PYG{p}{,} \PYG{l+s+s1}{\PYGZsq{}}\PYG{l+s+s1}{two}\PYG{l+s+s1}{\PYGZsq{}}\PYG{p}{,} \PYG{l+s+s1}{\PYGZsq{}}\PYG{l+s+s1}{three}\PYG{l+s+s1}{\PYGZsq{}}\PYG{p}{]}\PYG{p}{)}

\PYG{n}{result}
\end{sphinxVerbatim}

\end{sphinxuseclass}\end{sphinxVerbatimInput}
\begin{sphinxVerbatimOutput}

\begin{sphinxuseclass}{cell_output}
\begin{sphinxVerbatim}[commandchars=\\\{\}]
one    a    0
       b    1
two    c    2
       d    3
       e    4
three  f    5
       g    6
dtype: int64
\end{sphinxVerbatim}

\end{sphinxuseclass}\end{sphinxVerbatimOutput}

\end{sphinxuseclass}
\begin{sphinxuseclass}{cell}\begin{sphinxVerbatimInput}

\begin{sphinxuseclass}{cell_input}
\begin{sphinxVerbatim}[commandchars=\\\{\}]
\PYG{n}{result}\PYG{o}{.}\PYG{n}{unstack}\PYG{p}{(}\PYG{p}{)}
\end{sphinxVerbatim}

\end{sphinxuseclass}\end{sphinxVerbatimInput}
\begin{sphinxVerbatimOutput}

\begin{sphinxuseclass}{cell_output}
\begin{sphinxVerbatim}[commandchars=\\\{\}]
           a      b      c      d      e      f      g
one   0.0000 1.0000    NaN    NaN    NaN    NaN    NaN
two      NaN    NaN 2.0000 3.0000 4.0000    NaN    NaN
three    NaN    NaN    NaN    NaN    NaN 5.0000 6.0000
\end{sphinxVerbatim}

\end{sphinxuseclass}\end{sphinxVerbatimOutput}

\end{sphinxuseclass}
\begin{sphinxuseclass}{cell}\begin{sphinxVerbatimInput}

\begin{sphinxuseclass}{cell_input}
\begin{sphinxVerbatim}[commandchars=\\\{\}]
\PYG{n}{pd}\PYG{o}{.}\PYG{n}{concat}\PYG{p}{(}\PYG{p}{[}\PYG{n}{s1}\PYG{p}{,} \PYG{n}{s2}\PYG{p}{,} \PYG{n}{s3}\PYG{p}{]}\PYG{p}{,} \PYG{n}{axis}\PYG{o}{=}\PYG{l+m+mi}{1}\PYG{p}{,} \PYG{n}{keys}\PYG{o}{=}\PYG{p}{[}\PYG{l+s+s1}{\PYGZsq{}}\PYG{l+s+s1}{one}\PYG{l+s+s1}{\PYGZsq{}}\PYG{p}{,} \PYG{l+s+s1}{\PYGZsq{}}\PYG{l+s+s1}{two}\PYG{l+s+s1}{\PYGZsq{}}\PYG{p}{,} \PYG{l+s+s1}{\PYGZsq{}}\PYG{l+s+s1}{three}\PYG{l+s+s1}{\PYGZsq{}}\PYG{p}{]}\PYG{p}{)}
\end{sphinxVerbatim}

\end{sphinxuseclass}\end{sphinxVerbatimInput}
\begin{sphinxVerbatimOutput}

\begin{sphinxuseclass}{cell_output}
\begin{sphinxVerbatim}[commandchars=\\\{\}]
     one    two  three
a 0.0000    NaN    NaN
b 1.0000    NaN    NaN
c    NaN 2.0000    NaN
d    NaN 3.0000    NaN
e    NaN 4.0000    NaN
f    NaN    NaN 5.0000
g    NaN    NaN 6.0000
\end{sphinxVerbatim}

\end{sphinxuseclass}\end{sphinxVerbatimOutput}

\end{sphinxuseclass}
\begin{sphinxuseclass}{cell}\begin{sphinxVerbatimInput}

\begin{sphinxuseclass}{cell_input}
\begin{sphinxVerbatim}[commandchars=\\\{\}]
\PYG{n}{df1} \PYG{o}{=} \PYG{n}{pd}\PYG{o}{.}\PYG{n}{DataFrame}\PYG{p}{(}\PYG{n}{np}\PYG{o}{.}\PYG{n}{arange}\PYG{p}{(}\PYG{l+m+mi}{6}\PYG{p}{)}\PYG{o}{.}\PYG{n}{reshape}\PYG{p}{(}\PYG{l+m+mi}{3}\PYG{p}{,} \PYG{l+m+mi}{2}\PYG{p}{)}\PYG{p}{,} \PYG{n}{index}\PYG{o}{=}\PYG{p}{[}\PYG{l+s+s1}{\PYGZsq{}}\PYG{l+s+s1}{a}\PYG{l+s+s1}{\PYGZsq{}}\PYG{p}{,} \PYG{l+s+s1}{\PYGZsq{}}\PYG{l+s+s1}{b}\PYG{l+s+s1}{\PYGZsq{}}\PYG{p}{,} \PYG{l+s+s1}{\PYGZsq{}}\PYG{l+s+s1}{c}\PYG{l+s+s1}{\PYGZsq{}}\PYG{p}{]}\PYG{p}{,} \PYG{n}{columns}\PYG{o}{=}\PYG{p}{[}\PYG{l+s+s1}{\PYGZsq{}}\PYG{l+s+s1}{one}\PYG{l+s+s1}{\PYGZsq{}}\PYG{p}{,} \PYG{l+s+s1}{\PYGZsq{}}\PYG{l+s+s1}{two}\PYG{l+s+s1}{\PYGZsq{}}\PYG{p}{]}\PYG{p}{)}
\PYG{n}{df2} \PYG{o}{=} \PYG{n}{pd}\PYG{o}{.}\PYG{n}{DataFrame}\PYG{p}{(}\PYG{l+m+mi}{5} \PYG{o}{+} \PYG{n}{np}\PYG{o}{.}\PYG{n}{arange}\PYG{p}{(}\PYG{l+m+mi}{4}\PYG{p}{)}\PYG{o}{.}\PYG{n}{reshape}\PYG{p}{(}\PYG{l+m+mi}{2}\PYG{p}{,} \PYG{l+m+mi}{2}\PYG{p}{)}\PYG{p}{,} \PYG{n}{index}\PYG{o}{=}\PYG{p}{[}\PYG{l+s+s1}{\PYGZsq{}}\PYG{l+s+s1}{a}\PYG{l+s+s1}{\PYGZsq{}}\PYG{p}{,} \PYG{l+s+s1}{\PYGZsq{}}\PYG{l+s+s1}{c}\PYG{l+s+s1}{\PYGZsq{}}\PYG{p}{]}\PYG{p}{,} \PYG{n}{columns}\PYG{o}{=}\PYG{p}{[}\PYG{l+s+s1}{\PYGZsq{}}\PYG{l+s+s1}{three}\PYG{l+s+s1}{\PYGZsq{}}\PYG{p}{,} \PYG{l+s+s1}{\PYGZsq{}}\PYG{l+s+s1}{four}\PYG{l+s+s1}{\PYGZsq{}}\PYG{p}{]}\PYG{p}{)}
\end{sphinxVerbatim}

\end{sphinxuseclass}\end{sphinxVerbatimInput}

\end{sphinxuseclass}
\begin{sphinxuseclass}{cell}\begin{sphinxVerbatimInput}

\begin{sphinxuseclass}{cell_input}
\begin{sphinxVerbatim}[commandchars=\\\{\}]
\PYG{n}{pd}\PYG{o}{.}\PYG{n}{concat}\PYG{p}{(}\PYG{p}{[}\PYG{n}{df1}\PYG{p}{,} \PYG{n}{df2}\PYG{p}{]}\PYG{p}{,} \PYG{n}{axis}\PYG{o}{=}\PYG{l+m+mi}{1}\PYG{p}{,} \PYG{n}{keys}\PYG{o}{=}\PYG{p}{[}\PYG{l+s+s1}{\PYGZsq{}}\PYG{l+s+s1}{level1}\PYG{l+s+s1}{\PYGZsq{}}\PYG{p}{,} \PYG{l+s+s1}{\PYGZsq{}}\PYG{l+s+s1}{level2}\PYG{l+s+s1}{\PYGZsq{}}\PYG{p}{]}\PYG{p}{)}
\end{sphinxVerbatim}

\end{sphinxuseclass}\end{sphinxVerbatimInput}
\begin{sphinxVerbatimOutput}

\begin{sphinxuseclass}{cell_output}
\begin{sphinxVerbatim}[commandchars=\\\{\}]
  level1     level2       
     one two  three   four
a      0   1 5.0000 6.0000
b      2   3    NaN    NaN
c      4   5 7.0000 8.0000
\end{sphinxVerbatim}

\end{sphinxuseclass}\end{sphinxVerbatimOutput}

\end{sphinxuseclass}
\begin{sphinxuseclass}{cell}\begin{sphinxVerbatimInput}

\begin{sphinxuseclass}{cell_input}
\begin{sphinxVerbatim}[commandchars=\\\{\}]
\PYG{n}{pd}\PYG{o}{.}\PYG{n}{concat}\PYG{p}{(}\PYG{p}{[}\PYG{n}{df1}\PYG{p}{,} \PYG{n}{df2}\PYG{p}{]}\PYG{p}{,} \PYG{n}{axis}\PYG{o}{=}\PYG{l+m+mi}{1}\PYG{p}{,} \PYG{n}{keys}\PYG{o}{=}\PYG{p}{[}\PYG{l+s+s1}{\PYGZsq{}}\PYG{l+s+s1}{level1}\PYG{l+s+s1}{\PYGZsq{}}\PYG{p}{,} \PYG{l+s+s1}{\PYGZsq{}}\PYG{l+s+s1}{level2}\PYG{l+s+s1}{\PYGZsq{}}\PYG{p}{]}\PYG{p}{,} \PYG{n}{names}\PYG{o}{=}\PYG{p}{[}\PYG{l+s+s1}{\PYGZsq{}}\PYG{l+s+s1}{upper}\PYG{l+s+s1}{\PYGZsq{}}\PYG{p}{,} \PYG{l+s+s1}{\PYGZsq{}}\PYG{l+s+s1}{lower}\PYG{l+s+s1}{\PYGZsq{}}\PYG{p}{]}\PYG{p}{)}
\end{sphinxVerbatim}

\end{sphinxuseclass}\end{sphinxVerbatimInput}
\begin{sphinxVerbatimOutput}

\begin{sphinxuseclass}{cell_output}
\begin{sphinxVerbatim}[commandchars=\\\{\}]
upper level1     level2       
lower    one two  three   four
a          0   1 5.0000 6.0000
b          2   3    NaN    NaN
c          4   5 7.0000 8.0000
\end{sphinxVerbatim}

\end{sphinxuseclass}\end{sphinxVerbatimOutput}

\end{sphinxuseclass}

\section{Reshaping and Pivoting}
\label{\detokenize{mckinney_08_lecture:reshaping-and-pivoting}}
\sphinxAtStartPar
Above, we briefly explore reshaping data with \sphinxcode{\sphinxupquote{.stack()}} and \sphinxcode{\sphinxupquote{.unstack()}}.
Here we explore reshaping data more deeply.


\subsection{Reshaping with Hierarchical Indexing}
\label{\detokenize{mckinney_08_lecture:reshaping-with-hierarchical-indexing}}
\sphinxAtStartPar
Hierarchical indexes (multi\sphinxhyphen{}indexes) help reshape data.
\begin{quote}

\sphinxAtStartPar
There are two primary actions:
\begin{itemize}
\item {} 
\sphinxAtStartPar
stack: This “rotates” or pivots from the columns in the data to the rows

\item {} 
\sphinxAtStartPar
unstack: This pivots from the rows into the columns

\end{itemize}
\end{quote}

\begin{sphinxuseclass}{cell}\begin{sphinxVerbatimInput}

\begin{sphinxuseclass}{cell_input}
\begin{sphinxVerbatim}[commandchars=\\\{\}]
\PYG{n}{data} \PYG{o}{=} \PYG{n}{pd}\PYG{o}{.}\PYG{n}{DataFrame}\PYG{p}{(}\PYG{n}{np}\PYG{o}{.}\PYG{n}{arange}\PYG{p}{(}\PYG{l+m+mi}{6}\PYG{p}{)}\PYG{o}{.}\PYG{n}{reshape}\PYG{p}{(}\PYG{p}{(}\PYG{l+m+mi}{2}\PYG{p}{,} \PYG{l+m+mi}{3}\PYG{p}{)}\PYG{p}{)}\PYG{p}{,}
                    \PYG{n}{index}\PYG{o}{=}\PYG{n}{pd}\PYG{o}{.}\PYG{n}{Index}\PYG{p}{(}\PYG{p}{[}\PYG{l+s+s1}{\PYGZsq{}}\PYG{l+s+s1}{Ohio}\PYG{l+s+s1}{\PYGZsq{}}\PYG{p}{,} \PYG{l+s+s1}{\PYGZsq{}}\PYG{l+s+s1}{Colorado}\PYG{l+s+s1}{\PYGZsq{}}\PYG{p}{]}\PYG{p}{,} \PYG{n}{name}\PYG{o}{=}\PYG{l+s+s1}{\PYGZsq{}}\PYG{l+s+s1}{state}\PYG{l+s+s1}{\PYGZsq{}}\PYG{p}{)}\PYG{p}{,}
                    \PYG{n}{columns}\PYG{o}{=}\PYG{n}{pd}\PYG{o}{.}\PYG{n}{Index}\PYG{p}{(}\PYG{p}{[}\PYG{l+s+s1}{\PYGZsq{}}\PYG{l+s+s1}{one}\PYG{l+s+s1}{\PYGZsq{}}\PYG{p}{,} \PYG{l+s+s1}{\PYGZsq{}}\PYG{l+s+s1}{two}\PYG{l+s+s1}{\PYGZsq{}}\PYG{p}{,} \PYG{l+s+s1}{\PYGZsq{}}\PYG{l+s+s1}{three}\PYG{l+s+s1}{\PYGZsq{}}\PYG{p}{]}\PYG{p}{,}
                    \PYG{n}{name}\PYG{o}{=}\PYG{l+s+s1}{\PYGZsq{}}\PYG{l+s+s1}{number}\PYG{l+s+s1}{\PYGZsq{}}\PYG{p}{)}\PYG{p}{)}

\PYG{n}{data}
\end{sphinxVerbatim}

\end{sphinxuseclass}\end{sphinxVerbatimInput}
\begin{sphinxVerbatimOutput}

\begin{sphinxuseclass}{cell_output}
\begin{sphinxVerbatim}[commandchars=\\\{\}]
number    one  two  three
state                    
Ohio        0    1      2
Colorado    3    4      5
\end{sphinxVerbatim}

\end{sphinxuseclass}\end{sphinxVerbatimOutput}

\end{sphinxuseclass}
\begin{sphinxuseclass}{cell}\begin{sphinxVerbatimInput}

\begin{sphinxuseclass}{cell_input}
\begin{sphinxVerbatim}[commandchars=\\\{\}]
\PYG{n}{result} \PYG{o}{=} \PYG{n}{data}\PYG{o}{.}\PYG{n}{stack}\PYG{p}{(}\PYG{p}{)}

\PYG{n}{result}
\end{sphinxVerbatim}

\end{sphinxuseclass}\end{sphinxVerbatimInput}
\begin{sphinxVerbatimOutput}

\begin{sphinxuseclass}{cell_output}
\begin{sphinxVerbatim}[commandchars=\\\{\}]
state     number
Ohio      one       0
          two       1
          three     2
Colorado  one       3
          two       4
          three     5
dtype: int64
\end{sphinxVerbatim}

\end{sphinxuseclass}\end{sphinxVerbatimOutput}

\end{sphinxuseclass}
\begin{sphinxuseclass}{cell}\begin{sphinxVerbatimInput}

\begin{sphinxuseclass}{cell_input}
\begin{sphinxVerbatim}[commandchars=\\\{\}]
\PYG{n}{result}\PYG{o}{.}\PYG{n}{unstack}\PYG{p}{(}\PYG{p}{)}
\end{sphinxVerbatim}

\end{sphinxuseclass}\end{sphinxVerbatimInput}
\begin{sphinxVerbatimOutput}

\begin{sphinxuseclass}{cell_output}
\begin{sphinxVerbatim}[commandchars=\\\{\}]
number    one  two  three
state                    
Ohio        0    1      2
Colorado    3    4      5
\end{sphinxVerbatim}

\end{sphinxuseclass}\end{sphinxVerbatimOutput}

\end{sphinxuseclass}
\begin{sphinxuseclass}{cell}\begin{sphinxVerbatimInput}

\begin{sphinxuseclass}{cell_input}
\begin{sphinxVerbatim}[commandchars=\\\{\}]
\PYG{n}{s1} \PYG{o}{=} \PYG{n}{pd}\PYG{o}{.}\PYG{n}{Series}\PYG{p}{(}\PYG{p}{[}\PYG{l+m+mi}{0}\PYG{p}{,} \PYG{l+m+mi}{1}\PYG{p}{,} \PYG{l+m+mi}{2}\PYG{p}{,} \PYG{l+m+mi}{3}\PYG{p}{]}\PYG{p}{,} \PYG{n}{index}\PYG{o}{=}\PYG{p}{[}\PYG{l+s+s1}{\PYGZsq{}}\PYG{l+s+s1}{a}\PYG{l+s+s1}{\PYGZsq{}}\PYG{p}{,} \PYG{l+s+s1}{\PYGZsq{}}\PYG{l+s+s1}{b}\PYG{l+s+s1}{\PYGZsq{}}\PYG{p}{,} \PYG{l+s+s1}{\PYGZsq{}}\PYG{l+s+s1}{c}\PYG{l+s+s1}{\PYGZsq{}}\PYG{p}{,} \PYG{l+s+s1}{\PYGZsq{}}\PYG{l+s+s1}{d}\PYG{l+s+s1}{\PYGZsq{}}\PYG{p}{]}\PYG{p}{)}
\PYG{n}{s2} \PYG{o}{=} \PYG{n}{pd}\PYG{o}{.}\PYG{n}{Series}\PYG{p}{(}\PYG{p}{[}\PYG{l+m+mi}{4}\PYG{p}{,} \PYG{l+m+mi}{5}\PYG{p}{,} \PYG{l+m+mi}{6}\PYG{p}{]}\PYG{p}{,} \PYG{n}{index}\PYG{o}{=}\PYG{p}{[}\PYG{l+s+s1}{\PYGZsq{}}\PYG{l+s+s1}{c}\PYG{l+s+s1}{\PYGZsq{}}\PYG{p}{,} \PYG{l+s+s1}{\PYGZsq{}}\PYG{l+s+s1}{d}\PYG{l+s+s1}{\PYGZsq{}}\PYG{p}{,} \PYG{l+s+s1}{\PYGZsq{}}\PYG{l+s+s1}{e}\PYG{l+s+s1}{\PYGZsq{}}\PYG{p}{]}\PYG{p}{)}
\PYG{n}{data2} \PYG{o}{=} \PYG{n}{pd}\PYG{o}{.}\PYG{n}{concat}\PYG{p}{(}\PYG{p}{[}\PYG{n}{s1}\PYG{p}{,} \PYG{n}{s2}\PYG{p}{]}\PYG{p}{,} \PYG{n}{keys}\PYG{o}{=}\PYG{p}{[}\PYG{l+s+s1}{\PYGZsq{}}\PYG{l+s+s1}{one}\PYG{l+s+s1}{\PYGZsq{}}\PYG{p}{,} \PYG{l+s+s1}{\PYGZsq{}}\PYG{l+s+s1}{two}\PYG{l+s+s1}{\PYGZsq{}}\PYG{p}{]}\PYG{p}{)}

\PYG{n}{data2}
\end{sphinxVerbatim}

\end{sphinxuseclass}\end{sphinxVerbatimInput}
\begin{sphinxVerbatimOutput}

\begin{sphinxuseclass}{cell_output}
\begin{sphinxVerbatim}[commandchars=\\\{\}]
one  a    0
     b    1
     c    2
     d    3
two  c    4
     d    5
     e    6
dtype: int64
\end{sphinxVerbatim}

\end{sphinxuseclass}\end{sphinxVerbatimOutput}

\end{sphinxuseclass}
\begin{sphinxuseclass}{cell}\begin{sphinxVerbatimInput}

\begin{sphinxuseclass}{cell_input}
\begin{sphinxVerbatim}[commandchars=\\\{\}]
\PYG{n}{data2}\PYG{o}{.}\PYG{n}{unstack}\PYG{p}{(}\PYG{p}{)}
\end{sphinxVerbatim}

\end{sphinxuseclass}\end{sphinxVerbatimInput}
\begin{sphinxVerbatimOutput}

\begin{sphinxuseclass}{cell_output}
\begin{sphinxVerbatim}[commandchars=\\\{\}]
         a      b      c      d      e
one 0.0000 1.0000 2.0000 3.0000    NaN
two    NaN    NaN 4.0000 5.0000 6.0000
\end{sphinxVerbatim}

\end{sphinxuseclass}\end{sphinxVerbatimOutput}

\end{sphinxuseclass}
\sphinxAtStartPar
Un\sphinxhyphen{}stacking may introduce missing values because data frames are rectangular.

\sphinxAtStartPar
By default, stacking drops these missing values.

\begin{sphinxuseclass}{cell}\begin{sphinxVerbatimInput}

\begin{sphinxuseclass}{cell_input}
\begin{sphinxVerbatim}[commandchars=\\\{\}]
\PYG{n}{data2}\PYG{o}{.}\PYG{n}{unstack}\PYG{p}{(}\PYG{p}{)}\PYG{o}{.}\PYG{n}{stack}\PYG{p}{(}\PYG{p}{)}
\end{sphinxVerbatim}

\end{sphinxuseclass}\end{sphinxVerbatimInput}
\begin{sphinxVerbatimOutput}

\begin{sphinxuseclass}{cell_output}
\begin{sphinxVerbatim}[commandchars=\\\{\}]
one  a   0.0000
     b   1.0000
     c   2.0000
     d   3.0000
two  c   4.0000
     d   5.0000
     e   6.0000
dtype: float64
\end{sphinxVerbatim}

\end{sphinxuseclass}\end{sphinxVerbatimOutput}

\end{sphinxuseclass}
\sphinxAtStartPar
However, we can keep missing values with \sphinxcode{\sphinxupquote{dropna=False}}.

\begin{sphinxuseclass}{cell}\begin{sphinxVerbatimInput}

\begin{sphinxuseclass}{cell_input}
\begin{sphinxVerbatim}[commandchars=\\\{\}]
\PYG{n}{data2}\PYG{o}{.}\PYG{n}{unstack}\PYG{p}{(}\PYG{p}{)}\PYG{o}{.}\PYG{n}{stack}\PYG{p}{(}\PYG{n}{dropna}\PYG{o}{=}\PYG{k+kc}{False}\PYG{p}{)}
\end{sphinxVerbatim}

\end{sphinxuseclass}\end{sphinxVerbatimInput}
\begin{sphinxVerbatimOutput}

\begin{sphinxuseclass}{cell_output}
\begin{sphinxVerbatim}[commandchars=\\\{\}]
one  a   0.0000
     b   1.0000
     c   2.0000
     d   3.0000
     e      NaN
two  a      NaN
     b      NaN
     c   4.0000
     d   5.0000
     e   6.0000
dtype: float64
\end{sphinxVerbatim}

\end{sphinxuseclass}\end{sphinxVerbatimOutput}

\end{sphinxuseclass}
\begin{sphinxuseclass}{cell}\begin{sphinxVerbatimInput}

\begin{sphinxuseclass}{cell_input}
\begin{sphinxVerbatim}[commandchars=\\\{\}]
\PYG{n}{df} \PYG{o}{=} \PYG{n}{pd}\PYG{o}{.}\PYG{n}{DataFrame}\PYG{p}{(}\PYG{p}{\PYGZob{}}
    \PYG{l+s+s1}{\PYGZsq{}}\PYG{l+s+s1}{left}\PYG{l+s+s1}{\PYGZsq{}}\PYG{p}{:} \PYG{n}{result}\PYG{p}{,} 
    \PYG{l+s+s1}{\PYGZsq{}}\PYG{l+s+s1}{right}\PYG{l+s+s1}{\PYGZsq{}}\PYG{p}{:} \PYG{n}{result} \PYG{o}{+} \PYG{l+m+mi}{5}
    \PYG{p}{\PYGZcb{}}\PYG{p}{,}
    \PYG{n}{columns}\PYG{o}{=}\PYG{n}{pd}\PYG{o}{.}\PYG{n}{Index}\PYG{p}{(}\PYG{p}{[}\PYG{l+s+s1}{\PYGZsq{}}\PYG{l+s+s1}{left}\PYG{l+s+s1}{\PYGZsq{}}\PYG{p}{,} \PYG{l+s+s1}{\PYGZsq{}}\PYG{l+s+s1}{right}\PYG{l+s+s1}{\PYGZsq{}}\PYG{p}{]}\PYG{p}{,} \PYG{n}{name}\PYG{o}{=}\PYG{l+s+s1}{\PYGZsq{}}\PYG{l+s+s1}{side}\PYG{l+s+s1}{\PYGZsq{}}\PYG{p}{)}
\PYG{p}{)}

\PYG{n}{df}
\end{sphinxVerbatim}

\end{sphinxuseclass}\end{sphinxVerbatimInput}
\begin{sphinxVerbatimOutput}

\begin{sphinxuseclass}{cell_output}
\begin{sphinxVerbatim}[commandchars=\\\{\}]
side             left  right
state    number             
Ohio     one        0      5
         two        1      6
         three      2      7
Colorado one        3      8
         two        4      9
         three      5     10
\end{sphinxVerbatim}

\end{sphinxuseclass}\end{sphinxVerbatimOutput}

\end{sphinxuseclass}
\sphinxAtStartPar
If we un\sphinxhyphen{}stack a data frame, the un\sphinxhyphen{}stacked level becomes the innermost level in the resulting index.

\begin{sphinxuseclass}{cell}\begin{sphinxVerbatimInput}

\begin{sphinxuseclass}{cell_input}
\begin{sphinxVerbatim}[commandchars=\\\{\}]
\PYG{n}{df}\PYG{o}{.}\PYG{n}{unstack}\PYG{p}{(}\PYG{l+s+s1}{\PYGZsq{}}\PYG{l+s+s1}{state}\PYG{l+s+s1}{\PYGZsq{}}\PYG{p}{)}
\end{sphinxVerbatim}

\end{sphinxuseclass}\end{sphinxVerbatimInput}
\begin{sphinxVerbatimOutput}

\begin{sphinxuseclass}{cell_output}
\begin{sphinxVerbatim}[commandchars=\\\{\}]
side   left          right         
state  Ohio Colorado  Ohio Colorado
number                             
one       0        3     5        8
two       1        4     6        9
three     2        5     7       10
\end{sphinxVerbatim}

\end{sphinxuseclass}\end{sphinxVerbatimOutput}

\end{sphinxuseclass}
\sphinxAtStartPar
We can chain \sphinxcode{\sphinxupquote{.stack()}} and \sphinxcode{\sphinxupquote{.unstack()}} to rearrange our data.

\begin{sphinxuseclass}{cell}\begin{sphinxVerbatimInput}

\begin{sphinxuseclass}{cell_input}
\begin{sphinxVerbatim}[commandchars=\\\{\}]
\PYG{n}{df}\PYG{o}{.}\PYG{n}{unstack}\PYG{p}{(}\PYG{l+s+s1}{\PYGZsq{}}\PYG{l+s+s1}{state}\PYG{l+s+s1}{\PYGZsq{}}\PYG{p}{)}\PYG{o}{.}\PYG{n}{stack}\PYG{p}{(}\PYG{l+s+s1}{\PYGZsq{}}\PYG{l+s+s1}{side}\PYG{l+s+s1}{\PYGZsq{}}\PYG{p}{)}
\end{sphinxVerbatim}

\end{sphinxuseclass}\end{sphinxVerbatimInput}
\begin{sphinxVerbatimOutput}

\begin{sphinxuseclass}{cell_output}
\begin{sphinxVerbatim}[commandchars=\\\{\}]
state         Colorado  Ohio
number side                 
one    left          3     0
       right         8     5
two    left          4     1
       right         9     6
three  left          5     2
       right        10     7
\end{sphinxVerbatim}

\end{sphinxuseclass}\end{sphinxVerbatimOutput}

\end{sphinxuseclass}
\sphinxAtStartPar
McKinney provides two more subsections on reshaping data with the \sphinxcode{\sphinxupquote{.pivot()}} and \sphinxcode{\sphinxupquote{.melt()}} methods.
Unlike, the stacking methods, the pivoting methods can aggregate data and do not require an index.
We will skip these additional aggregation methods for now.

\sphinxstepscope


\section{McKinney Chapter 8 \sphinxhyphen{} Practice (Blank)}
\label{\detokenize{mckinney_08_practice:mckinney-chapter-8-practice-blank}}\label{\detokenize{mckinney_08_practice::doc}}

\subsection{Announcements}
\label{\detokenize{mckinney_08_practice:announcements}}

\subsection{Practice}
\label{\detokenize{mckinney_08_practice:practice}}

\subsubsection{Download data from Yahoo! Finance for BAC, C, GS, JPM, MS, and PNC and assign to data frame \sphinxstyleliteralintitle{\sphinxupquote{stocks}}.}
\label{\detokenize{mckinney_08_practice:download-data-from-yahoo-finance-for-bac-c-gs-jpm-ms-and-pnc-and-assign-to-data-frame-stocks}}
\sphinxAtStartPar
Use \sphinxcode{\sphinxupquote{stocks.columns.names}} to assign the names \sphinxcode{\sphinxupquote{Variable}} and \sphinxcode{\sphinxupquote{Ticker}} to the column multi index.


\subsubsection{Reshape \sphinxstyleliteralintitle{\sphinxupquote{stocks}} from wide to long with dates and tickers as row indexes and assign to data frame \sphinxstyleliteralintitle{\sphinxupquote{stocks\_long}}.}
\label{\detokenize{mckinney_08_practice:reshape-stocks-from-wide-to-long-with-dates-and-tickers-as-row-indexes-and-assign-to-data-frame-stocks-long}}

\subsubsection{Add daily returns for each stock to data frames \sphinxstyleliteralintitle{\sphinxupquote{stocks}} and \sphinxstyleliteralintitle{\sphinxupquote{stocks\_long}}.}
\label{\detokenize{mckinney_08_practice:add-daily-returns-for-each-stock-to-data-frames-stocks-and-stocks-long}}
\sphinxAtStartPar
Name the returns variable \sphinxcode{\sphinxupquote{Returns}}, and maintain all multi indexes.
\sphinxstyleemphasis{Hint:} Use \sphinxcode{\sphinxupquote{pd.MultiIndex()}} to create a multi index for the the wide data frame \sphinxcode{\sphinxupquote{stocks}}.


\subsubsection{Download the daily benchmark return factors from Ken French’s data library.}
\label{\detokenize{mckinney_08_practice:download-the-daily-benchmark-return-factors-from-ken-french-s-data-library}}

\subsubsection{Add the daily benchmark return factors to \sphinxstyleliteralintitle{\sphinxupquote{stocks}} and \sphinxstyleliteralintitle{\sphinxupquote{stocks\_long}}.}
\label{\detokenize{mckinney_08_practice:add-the-daily-benchmark-return-factors-to-stocks-and-stocks-long}}
\sphinxAtStartPar
For the wide data frame \sphinxcode{\sphinxupquote{stocks}}, use the outer index name \sphinxcode{\sphinxupquote{Factors}}.


\subsubsection{Write a function \sphinxstyleliteralintitle{\sphinxupquote{download()}} that accepts tickers and returns a wide data frame of returns with the daily benchmark return factors.}
\label{\detokenize{mckinney_08_practice:write-a-function-download-that-accepts-tickers-and-returns-a-wide-data-frame-of-returns-with-the-daily-benchmark-return-factors}}

\subsubsection{Download earnings per share for the stocks in \sphinxstyleliteralintitle{\sphinxupquote{stocks}} and combine to a long data frame \sphinxstyleliteralintitle{\sphinxupquote{earnings}}.}
\label{\detokenize{mckinney_08_practice:download-earnings-per-share-for-the-stocks-in-stocks-and-combine-to-a-long-data-frame-earnings}}
\sphinxAtStartPar
Use the \sphinxcode{\sphinxupquote{.earnings\_dates}} method described \sphinxhref{https://pypi.org/project/yfinance/}{here}.
Use \sphinxcode{\sphinxupquote{pd.concat()}} to combine the result of each the \sphinxcode{\sphinxupquote{.earnings\_date}} data frames and assign them to a new data frame \sphinxcode{\sphinxupquote{earnings}}.
Name the row indexes \sphinxcode{\sphinxupquote{Ticker}} and \sphinxcode{\sphinxupquote{Date}} and swap to match the order of the row index in \sphinxcode{\sphinxupquote{stocks\_long}}.

\begin{sphinxuseclass}{cell}\begin{sphinxVerbatimInput}

\begin{sphinxuseclass}{cell_input}
\begin{sphinxVerbatim}[commandchars=\\\{\}]
\PYG{c+c1}{\PYGZsh{} some students had to update yfinance to use the .earnings\PYGZus{}dates atrtibute}
\PYG{c+c1}{\PYGZsh{} \PYGZpc{}pip install \PYGZhy{}U yfinance}
\end{sphinxVerbatim}

\end{sphinxuseclass}\end{sphinxVerbatimInput}

\end{sphinxuseclass}

\subsubsection{Combine \sphinxstyleliteralintitle{\sphinxupquote{earnings}} with the returns from \sphinxstyleliteralintitle{\sphinxupquote{stocks\_long}}.}
\label{\detokenize{mckinney_08_practice:combine-earnings-with-the-returns-from-stocks-long}}
\sphinxAtStartPar
\sphinxstyleemphasis{\sphinxstylestrong{It is easier to leave \sphinxcode{\sphinxupquote{stocks}} and \sphinxcode{\sphinxupquote{stocks\_long}} as\sphinxhyphen{}is and work with slices \sphinxcode{\sphinxupquote{returns}} and \sphinxcode{\sphinxupquote{returns\_long}}.}}
Use the \sphinxcode{\sphinxupquote{tz\_localize('America/New\_York')}} method add time zone information back to \sphinxcode{\sphinxupquote{returns.index}} and use \sphinxcode{\sphinxupquote{pd.to\_timedelta(16, unit='h')}} to set time to the market close in New York City.
Use \sphinxhref{https://pandas.pydata.org/pandas-docs/stable/reference/api/pandas.merge\_asof.html}{\sphinxcode{\sphinxupquote{pd.merge\_asof()}}} to match earnings announcement dates and times to appropriate return periods.
For example, if a firm announces earnings after the close at 5 PM on February 7, we want to match the return period from 4 PM on February 7 to 4 PM on February 8.


\subsubsection{Plot the relation between daily returns and earnings surprises}
\label{\detokenize{mckinney_08_practice:plot-the-relation-between-daily-returns-and-earnings-surprises}}
\sphinxAtStartPar
Three options in increasing difficulty:
\begin{enumerate}
\sphinxsetlistlabels{\arabic}{enumi}{enumii}{}{.}%
\item {} 
\sphinxAtStartPar
Scatter plot

\item {} 
\sphinxAtStartPar
Scatter plot with a best\sphinxhyphen{}fit line using \sphinxcode{\sphinxupquote{regplot()}} from the seaborn package

\item {} 
\sphinxAtStartPar
Bar plot using \sphinxcode{\sphinxupquote{barplot()}} from the seaborn package after using \sphinxcode{\sphinxupquote{pd.qcut()}} to form five groups on earnings surprises

\end{enumerate}


\subsubsection{Repeat the earnings exercise with the S\&P 100 stocks}
\label{\detokenize{mckinney_08_practice:repeat-the-earnings-exercise-with-the-s-p-100-stocks}}

\subsubsection{Repeat the earnings exercise with \sphinxstyleemphasis{excess returns} of the S\&P 100 Stocks}
\label{\detokenize{mckinney_08_practice:repeat-the-earnings-exercise-with-excess-returns-of-the-s-p-100-stocks}}
\sphinxAtStartPar
Excess returns are returns minus market returns.
We need to add a timezone and the closing time to the market return from Fama and French.


\subsubsection{Improve your \sphinxstyleliteralintitle{\sphinxupquote{download()}} function from above}
\label{\detokenize{mckinney_08_practice:improve-your-download-function-from-above}}
\sphinxAtStartPar
Modify \sphinxcode{\sphinxupquote{download()}} to accept one or more than one ticker.
Since we will not use the advanced functionality of the tickers object that \sphinxcode{\sphinxupquote{yf.Tickers()}} creates, we will use \sphinxcode{\sphinxupquote{yf.download()}}.
The current version of \sphinxcode{\sphinxupquote{yf.download()}} does not accept a \sphinxcode{\sphinxupquote{session=}} argument.

\sphinxstepscope


\section{McKinney Chapter 8 \sphinxhyphen{} Practice (Monday 2:45 PM, Section 3)}
\label{\detokenize{mckinney_08_practice_03:mckinney-chapter-8-practice-monday-2-45-pm-section-3}}\label{\detokenize{mckinney_08_practice_03::doc}}

\subsection{Announcements}
\label{\detokenize{mckinney_08_practice_03:announcements}}\begin{enumerate}
\sphinxsetlistlabels{\arabic}{enumi}{enumii}{}{.}%
\item {} 
\sphinxAtStartPar
Quiz 3 mean was \$\textbackslash{}approx 80\%\$

\item {} 
\sphinxAtStartPar
I posted project 1 to Canvas

\end{enumerate}


\subsection{Practice}
\label{\detokenize{mckinney_08_practice_03:practice}}

\subsubsection{Download data from Yahoo! Finance for BAC, C, GS, JPM, MS, and PNC and assign to data frame \sphinxstyleliteralintitle{\sphinxupquote{stocks}}.}
\label{\detokenize{mckinney_08_practice_03:download-data-from-yahoo-finance-for-bac-c-gs-jpm-ms-and-pnc-and-assign-to-data-frame-stocks}}
\sphinxAtStartPar
Use \sphinxcode{\sphinxupquote{stocks.columns.names}} to assign the names \sphinxcode{\sphinxupquote{Variable}} and \sphinxcode{\sphinxupquote{Ticker}} to the column multi index.

\begin{sphinxuseclass}{cell}\begin{sphinxVerbatimInput}

\begin{sphinxuseclass}{cell_input}
\begin{sphinxVerbatim}[commandchars=\\\{\}]
\PYG{k+kn}{import} \PYG{n+nn}{matplotlib}\PYG{n+nn}{.}\PYG{n+nn}{pyplot} \PYG{k}{as} \PYG{n+nn}{plt}
\PYG{k+kn}{import} \PYG{n+nn}{numpy} \PYG{k}{as} \PYG{n+nn}{np}
\PYG{k+kn}{import} \PYG{n+nn}{pandas} \PYG{k}{as} \PYG{n+nn}{pd}
\end{sphinxVerbatim}

\end{sphinxuseclass}\end{sphinxVerbatimInput}

\end{sphinxuseclass}
\begin{sphinxuseclass}{cell}\begin{sphinxVerbatimInput}

\begin{sphinxuseclass}{cell_input}
\begin{sphinxVerbatim}[commandchars=\\\{\}]
\PYG{o}{\PYGZpc{}}\PYG{k}{config} InlineBackend.figure\PYGZus{}format = \PYGZsq{}retina\PYGZsq{}
\PYG{o}{\PYGZpc{}}\PYG{k}{precision} 4
\PYG{n}{pd}\PYG{o}{.}\PYG{n}{options}\PYG{o}{.}\PYG{n}{display}\PYG{o}{.}\PYG{n}{float\PYGZus{}format} \PYG{o}{=} \PYG{l+s+s1}{\PYGZsq{}}\PYG{l+s+si}{\PYGZob{}:.4f\PYGZcb{}}\PYG{l+s+s1}{\PYGZsq{}}\PYG{o}{.}\PYG{n}{format}
\end{sphinxVerbatim}

\end{sphinxuseclass}\end{sphinxVerbatimInput}

\end{sphinxuseclass}
\begin{sphinxuseclass}{cell}\begin{sphinxVerbatimInput}

\begin{sphinxuseclass}{cell_input}
\begin{sphinxVerbatim}[commandchars=\\\{\}]
\PYG{k+kn}{import} \PYG{n+nn}{yfinance} \PYG{k}{as} \PYG{n+nn}{yf}
\PYG{k+kn}{import} \PYG{n+nn}{pandas\PYGZus{}datareader} \PYG{k}{as} \PYG{n+nn}{pdr}
\PYG{k+kn}{import} \PYG{n+nn}{requests\PYGZus{}cache}
\PYG{n}{session} \PYG{o}{=} \PYG{n}{requests\PYGZus{}cache}\PYG{o}{.}\PYG{n}{CachedSession}\PYG{p}{(}\PYG{p}{)}
\end{sphinxVerbatim}

\end{sphinxuseclass}\end{sphinxVerbatimInput}

\end{sphinxuseclass}
\begin{sphinxuseclass}{cell}\begin{sphinxVerbatimInput}

\begin{sphinxuseclass}{cell_input}
\begin{sphinxVerbatim}[commandchars=\\\{\}]
\PYG{n}{tickers} \PYG{o}{=} \PYG{n}{yf}\PYG{o}{.}\PYG{n}{Tickers}\PYG{p}{(}\PYG{l+s+s1}{\PYGZsq{}}\PYG{l+s+s1}{BAC C GS JPM MS PNC}\PYG{l+s+s1}{\PYGZsq{}}\PYG{p}{,} \PYG{n}{session}\PYG{o}{=}\PYG{n}{session}\PYG{p}{)}
\PYG{n}{stocks} \PYG{o}{=} \PYG{n}{tickers}\PYG{o}{.}\PYG{n}{history}\PYG{p}{(}\PYG{n}{period}\PYG{o}{=}\PYG{l+s+s1}{\PYGZsq{}}\PYG{l+s+s1}{max}\PYG{l+s+s1}{\PYGZsq{}}\PYG{p}{,} \PYG{n}{auto\PYGZus{}adjust}\PYG{o}{=}\PYG{k+kc}{False}\PYG{p}{,} \PYG{n}{progress}\PYG{o}{=}\PYG{k+kc}{False}\PYG{p}{)}
\PYG{n}{stocks}\PYG{o}{.}\PYG{n}{index} \PYG{o}{=} \PYG{n}{stocks}\PYG{o}{.}\PYG{n}{index}\PYG{o}{.}\PYG{n}{tz\PYGZus{}localize}\PYG{p}{(}\PYG{k+kc}{None}\PYG{p}{)}
\PYG{n}{stocks}\PYG{o}{.}\PYG{n}{columns}\PYG{o}{.}\PYG{n}{names} \PYG{o}{=} \PYG{p}{[}\PYG{l+s+s1}{\PYGZsq{}}\PYG{l+s+s1}{Variable}\PYG{l+s+s1}{\PYGZsq{}}\PYG{p}{,} \PYG{l+s+s1}{\PYGZsq{}}\PYG{l+s+s1}{Ticker}\PYG{l+s+s1}{\PYGZsq{}}\PYG{p}{]}
\PYG{n}{stocks}\PYG{o}{.}\PYG{n}{head}\PYG{p}{(}\PYG{p}{)}
\end{sphinxVerbatim}

\end{sphinxuseclass}\end{sphinxVerbatimInput}
\begin{sphinxVerbatimOutput}

\begin{sphinxuseclass}{cell_output}
\begin{sphinxVerbatim}[commandchars=\\\{\}]
Variable   Adj Close                      Close              ... Stock Splits  \PYGZbs{}
Ticker           BAC   C  GS JPM  MS PNC    BAC   C  GS JPM  ...           GS   
Date                                                         ...                
1973\PYGZhy{}02\PYGZhy{}21    1.6311 NaN NaN NaN NaN NaN 4.6250 NaN NaN NaN  ...          NaN   
1973\PYGZhy{}02\PYGZhy{}22    1.6366 NaN NaN NaN NaN NaN 4.6406 NaN NaN NaN  ...          NaN   
1973\PYGZhy{}02\PYGZhy{}23    1.6311 NaN NaN NaN NaN NaN 4.6250 NaN NaN NaN  ...          NaN   
1973\PYGZhy{}02\PYGZhy{}26    1.6311 NaN NaN NaN NaN NaN 4.6250 NaN NaN NaN  ...          NaN   
1973\PYGZhy{}02\PYGZhy{}27    1.6311 NaN NaN NaN NaN NaN 4.6250 NaN NaN NaN  ...          NaN   

Variable                Volume                      
Ticker     JPM  MS PNC     BAC   C  GS JPM  MS PNC  
Date                                                
1973\PYGZhy{}02\PYGZhy{}21 NaN NaN NaN   99200 NaN NaN NaN NaN NaN  
1973\PYGZhy{}02\PYGZhy{}22 NaN NaN NaN   47200 NaN NaN NaN NaN NaN  
1973\PYGZhy{}02\PYGZhy{}23 NaN NaN NaN  133600 NaN NaN NaN NaN NaN  
1973\PYGZhy{}02\PYGZhy{}26 NaN NaN NaN   24000 NaN NaN NaN NaN NaN  
1973\PYGZhy{}02\PYGZhy{}27 NaN NaN NaN   41600 NaN NaN NaN NaN NaN  

[5 rows x 48 columns]
\end{sphinxVerbatim}

\end{sphinxuseclass}\end{sphinxVerbatimOutput}

\end{sphinxuseclass}

\subsubsection{Reshape \sphinxstyleliteralintitle{\sphinxupquote{stocks}} from wide to long with dates and tickers as row indexes and assign to data frame \sphinxstyleliteralintitle{\sphinxupquote{stocks\_long}}.}
\label{\detokenize{mckinney_08_practice_03:reshape-stocks-from-wide-to-long-with-dates-and-tickers-as-row-indexes-and-assign-to-data-frame-stocks-long}}
\begin{sphinxuseclass}{cell}\begin{sphinxVerbatimInput}

\begin{sphinxuseclass}{cell_input}
\begin{sphinxVerbatim}[commandchars=\\\{\}]
\PYG{n}{stocks\PYGZus{}long} \PYG{o}{=} \PYG{n}{stocks}\PYG{o}{.}\PYG{n}{stack}\PYG{p}{(}\PYG{p}{)}
\PYG{n}{stocks\PYGZus{}long}\PYG{o}{.}\PYG{n}{head}\PYG{p}{(}\PYG{p}{)}
\end{sphinxVerbatim}

\end{sphinxuseclass}\end{sphinxVerbatimInput}
\begin{sphinxVerbatimOutput}

\begin{sphinxuseclass}{cell_output}
\begin{sphinxVerbatim}[commandchars=\\\{\}]
Variable           Adj Close  Close  Dividends   High    Low   Open  \PYGZbs{}
Date       Ticker                                                     
1973\PYGZhy{}02\PYGZhy{}21 BAC        1.6311 4.6250     0.0000 4.6250 4.6250 4.6250   
1973\PYGZhy{}02\PYGZhy{}22 BAC        1.6366 4.6406     0.0000 4.6406 4.6406 4.6406   
1973\PYGZhy{}02\PYGZhy{}23 BAC        1.6311 4.6250     0.0000 4.6250 4.6250 4.6250   
1973\PYGZhy{}02\PYGZhy{}26 BAC        1.6311 4.6250     0.0000 4.6250 4.6250 4.6250   
1973\PYGZhy{}02\PYGZhy{}27 BAC        1.6311 4.6250     0.0000 4.6250 4.6250 4.6250   

Variable           Stock Splits      Volume  
Date       Ticker                            
1973\PYGZhy{}02\PYGZhy{}21 BAC           0.0000  99200.0000  
1973\PYGZhy{}02\PYGZhy{}22 BAC           0.0000  47200.0000  
1973\PYGZhy{}02\PYGZhy{}23 BAC           0.0000 133600.0000  
1973\PYGZhy{}02\PYGZhy{}26 BAC           0.0000  24000.0000  
1973\PYGZhy{}02\PYGZhy{}27 BAC           0.0000  41600.0000  
\end{sphinxVerbatim}

\end{sphinxuseclass}\end{sphinxVerbatimOutput}

\end{sphinxuseclass}

\subsubsection{Add daily returns for each stock to data frames \sphinxstyleliteralintitle{\sphinxupquote{stocks}} and \sphinxstyleliteralintitle{\sphinxupquote{stocks\_long}}.}
\label{\detokenize{mckinney_08_practice_03:add-daily-returns-for-each-stock-to-data-frames-stocks-and-stocks-long}}
\sphinxAtStartPar
Name the returns variable \sphinxcode{\sphinxupquote{Returns}}, and maintain all multi indexes.
\sphinxstyleemphasis{Hint:} Use \sphinxcode{\sphinxupquote{pd.MultiIndex()}} to create a multi index for the the wide data frame \sphinxcode{\sphinxupquote{stocks}}.

\begin{sphinxuseclass}{cell}\begin{sphinxVerbatimInput}

\begin{sphinxuseclass}{cell_input}
\begin{sphinxVerbatim}[commandchars=\\\{\}]
\PYG{n}{\PYGZus{}} \PYG{o}{=} \PYG{n}{pd}\PYG{o}{.}\PYG{n}{MultiIndex}\PYG{o}{.}\PYG{n}{from\PYGZus{}product}\PYG{p}{(}\PYG{p}{[}\PYG{p}{[}\PYG{l+s+s1}{\PYGZsq{}}\PYG{l+s+s1}{Returns}\PYG{l+s+s1}{\PYGZsq{}}\PYG{p}{]}\PYG{p}{,} \PYG{n}{stocks}\PYG{p}{[}\PYG{l+s+s1}{\PYGZsq{}}\PYG{l+s+s1}{Adj Close}\PYG{l+s+s1}{\PYGZsq{}}\PYG{p}{]}\PYG{o}{.}\PYG{n}{columns}\PYG{p}{]}\PYG{p}{)}
\PYG{n}{stocks}\PYG{p}{[}\PYG{n}{\PYGZus{}}\PYG{p}{]} \PYG{o}{=} \PYG{n}{stocks}\PYG{p}{[}\PYG{l+s+s1}{\PYGZsq{}}\PYG{l+s+s1}{Adj Close}\PYG{l+s+s1}{\PYGZsq{}}\PYG{p}{]}\PYG{o}{.}\PYG{n}{pct\PYGZus{}change}\PYG{p}{(}\PYG{p}{)}
\PYG{n}{stocks}\PYG{o}{.}\PYG{n}{head}\PYG{p}{(}\PYG{p}{)}
\end{sphinxVerbatim}

\end{sphinxuseclass}\end{sphinxVerbatimInput}
\begin{sphinxVerbatimOutput}

\begin{sphinxuseclass}{cell_output}
\begin{sphinxVerbatim}[commandchars=\\\{\}]
Variable   Adj Close                      Close              ... Volume      \PYGZbs{}
Ticker           BAC   C  GS JPM  MS PNC    BAC   C  GS JPM  ...     GS JPM   
Date                                                         ...              
1973\PYGZhy{}02\PYGZhy{}21    1.6311 NaN NaN NaN NaN NaN 4.6250 NaN NaN NaN  ...    NaN NaN   
1973\PYGZhy{}02\PYGZhy{}22    1.6366 NaN NaN NaN NaN NaN 4.6406 NaN NaN NaN  ...    NaN NaN   
1973\PYGZhy{}02\PYGZhy{}23    1.6311 NaN NaN NaN NaN NaN 4.6250 NaN NaN NaN  ...    NaN NaN   
1973\PYGZhy{}02\PYGZhy{}26    1.6311 NaN NaN NaN NaN NaN 4.6250 NaN NaN NaN  ...    NaN NaN   
1973\PYGZhy{}02\PYGZhy{}27    1.6311 NaN NaN NaN NaN NaN 4.6250 NaN NaN NaN  ...    NaN NaN   

Variable           Returns                      
Ticker      MS PNC     BAC   C  GS JPM  MS PNC  
Date                                            
1973\PYGZhy{}02\PYGZhy{}21 NaN NaN     NaN NaN NaN NaN NaN NaN  
1973\PYGZhy{}02\PYGZhy{}22 NaN NaN  0.0034 NaN NaN NaN NaN NaN  
1973\PYGZhy{}02\PYGZhy{}23 NaN NaN \PYGZhy{}0.0034 NaN NaN NaN NaN NaN  
1973\PYGZhy{}02\PYGZhy{}26 NaN NaN  0.0000 NaN NaN NaN NaN NaN  
1973\PYGZhy{}02\PYGZhy{}27 NaN NaN  0.0000 NaN NaN NaN NaN NaN  

[5 rows x 54 columns]
\end{sphinxVerbatim}

\end{sphinxuseclass}\end{sphinxVerbatimOutput}

\end{sphinxuseclass}
\sphinxAtStartPar
The easiest way to add returns to long data frame \sphinxcode{\sphinxupquote{stocks\_long}} is to \sphinxcode{\sphinxupquote{.stack()}} wide data frame \sphinxcode{\sphinxupquote{stocks}}!
We could sort \sphinxcode{\sphinxupquote{stocks\_long}} by ticker and date (to sort chronologically within each ticker), then use \sphinxcode{\sphinxupquote{.pct\_change()}}.
However, this approach miscalculates the first return for every ticker except for the first ticker.
The easiest and safest solution is to \sphinxcode{\sphinxupquote{.stack()}} the wide data frame \sphinxcode{\sphinxupquote{stocks}}!

\begin{sphinxuseclass}{cell}\begin{sphinxVerbatimInput}

\begin{sphinxuseclass}{cell_input}
\begin{sphinxVerbatim}[commandchars=\\\{\}]
\PYG{c+c1}{\PYGZsh{} see that the first return for C is wrong}
\PYG{c+c1}{\PYGZsh{} stocks\PYGZus{}long[\PYGZsq{}Adj Close\PYGZsq{}].sort\PYGZus{}index(level=[\PYGZsq{}Ticker\PYGZsq{}, \PYGZsq{}Date\PYGZsq{}]).pct\PYGZus{}change().loc[(slice(None), \PYGZsq{}C\PYGZsq{})]}
\end{sphinxVerbatim}

\end{sphinxuseclass}\end{sphinxVerbatimInput}

\end{sphinxuseclass}
\begin{sphinxuseclass}{cell}\begin{sphinxVerbatimInput}

\begin{sphinxuseclass}{cell_input}
\begin{sphinxVerbatim}[commandchars=\\\{\}]
\PYG{n}{stocks\PYGZus{}long} \PYG{o}{=} \PYG{n}{stocks}\PYG{o}{.}\PYG{n}{stack}\PYG{p}{(}\PYG{p}{)}
\end{sphinxVerbatim}

\end{sphinxuseclass}\end{sphinxVerbatimInput}

\end{sphinxuseclass}

\subsubsection{Download the daily benchmark return factors from Ken French’s data library.}
\label{\detokenize{mckinney_08_practice_03:download-the-daily-benchmark-return-factors-from-ken-french-s-data-library}}
\begin{sphinxuseclass}{cell}\begin{sphinxVerbatimInput}

\begin{sphinxuseclass}{cell_input}
\begin{sphinxVerbatim}[commandchars=\\\{\}]
\PYG{n}{pdr}\PYG{o}{.}\PYG{n}{famafrench}\PYG{o}{.}\PYG{n}{get\PYGZus{}available\PYGZus{}datasets}\PYG{p}{(}\PYG{p}{)}\PYG{p}{[}\PYG{p}{:}\PYG{l+m+mi}{5}\PYG{p}{]}
\end{sphinxVerbatim}

\end{sphinxuseclass}\end{sphinxVerbatimInput}
\begin{sphinxVerbatimOutput}

\begin{sphinxuseclass}{cell_output}
\begin{sphinxVerbatim}[commandchars=\\\{\}]
[\PYGZsq{}F\PYGZhy{}F\PYGZus{}Research\PYGZus{}Data\PYGZus{}Factors\PYGZsq{},
 \PYGZsq{}F\PYGZhy{}F\PYGZus{}Research\PYGZus{}Data\PYGZus{}Factors\PYGZus{}weekly\PYGZsq{},
 \PYGZsq{}F\PYGZhy{}F\PYGZus{}Research\PYGZus{}Data\PYGZus{}Factors\PYGZus{}daily\PYGZsq{},
 \PYGZsq{}F\PYGZhy{}F\PYGZus{}Research\PYGZus{}Data\PYGZus{}5\PYGZus{}Factors\PYGZus{}2x3\PYGZsq{},
 \PYGZsq{}F\PYGZhy{}F\PYGZus{}Research\PYGZus{}Data\PYGZus{}5\PYGZus{}Factors\PYGZus{}2x3\PYGZus{}daily\PYGZsq{}]
\end{sphinxVerbatim}

\end{sphinxuseclass}\end{sphinxVerbatimOutput}

\end{sphinxuseclass}
\begin{sphinxuseclass}{cell}\begin{sphinxVerbatimInput}

\begin{sphinxuseclass}{cell_input}
\begin{sphinxVerbatim}[commandchars=\\\{\}]
\PYG{n}{ff} \PYG{o}{=} \PYG{p}{(}
    \PYG{n}{pdr}\PYG{o}{.}\PYG{n}{DataReader}\PYG{p}{(}
        \PYG{n}{name}\PYG{o}{=}\PYG{l+s+s1}{\PYGZsq{}}\PYG{l+s+s1}{F\PYGZhy{}F\PYGZus{}Research\PYGZus{}Data\PYGZus{}Factors\PYGZus{}daily}\PYG{l+s+s1}{\PYGZsq{}}\PYG{p}{,}
        \PYG{n}{data\PYGZus{}source}\PYG{o}{=}\PYG{l+s+s1}{\PYGZsq{}}\PYG{l+s+s1}{famafrench}\PYG{l+s+s1}{\PYGZsq{}}\PYG{p}{,}
        \PYG{n}{start}\PYG{o}{=}\PYG{l+s+s1}{\PYGZsq{}}\PYG{l+s+s1}{1900}\PYG{l+s+s1}{\PYGZsq{}}\PYG{p}{,}
        \PYG{n}{session}\PYG{o}{=}\PYG{n}{session}
    \PYG{p}{)}
    \PYG{p}{[}\PYG{l+m+mi}{0}\PYG{p}{]}
    \PYG{o}{.}\PYG{n}{div}\PYG{p}{(}\PYG{l+m+mi}{100}\PYG{p}{)}
\PYG{p}{)}
\end{sphinxVerbatim}

\end{sphinxuseclass}\end{sphinxVerbatimInput}

\end{sphinxuseclass}

\subsubsection{Add the daily benchmark return factors to \sphinxstyleliteralintitle{\sphinxupquote{stocks}} and \sphinxstyleliteralintitle{\sphinxupquote{stocks\_long}}.}
\label{\detokenize{mckinney_08_practice_03:add-the-daily-benchmark-return-factors-to-stocks-and-stocks-long}}
\sphinxAtStartPar
For the wide data frame \sphinxcode{\sphinxupquote{stocks}}, use the outer index name \sphinxcode{\sphinxupquote{Factors}}.

\begin{sphinxuseclass}{cell}\begin{sphinxVerbatimInput}

\begin{sphinxuseclass}{cell_input}
\begin{sphinxVerbatim}[commandchars=\\\{\}]
\PYG{n}{\PYGZus{}} \PYG{o}{=} \PYG{n}{pd}\PYG{o}{.}\PYG{n}{MultiIndex}\PYG{o}{.}\PYG{n}{from\PYGZus{}product}\PYG{p}{(}\PYG{p}{[}\PYG{p}{[}\PYG{l+s+s1}{\PYGZsq{}}\PYG{l+s+s1}{Factors}\PYG{l+s+s1}{\PYGZsq{}}\PYG{p}{]}\PYG{p}{,} \PYG{n}{ff}\PYG{o}{.}\PYG{n}{columns}\PYG{p}{]}\PYG{p}{)}
\PYG{n}{stocks}\PYG{p}{[}\PYG{n}{\PYGZus{}}\PYG{p}{]} \PYG{o}{=} \PYG{n}{ff}
\PYG{n}{stocks}\PYG{o}{.}\PYG{n}{head}\PYG{p}{(}\PYG{p}{)}
\end{sphinxVerbatim}

\end{sphinxuseclass}\end{sphinxVerbatimInput}
\begin{sphinxVerbatimOutput}

\begin{sphinxuseclass}{cell_output}
\begin{sphinxVerbatim}[commandchars=\\\{\}]
Variable   Adj Close                      Close              ... Returns      \PYGZbs{}
Ticker           BAC   C  GS JPM  MS PNC    BAC   C  GS JPM  ...     BAC   C   
Date                                                         ...               
1973\PYGZhy{}02\PYGZhy{}21    1.6311 NaN NaN NaN NaN NaN 4.6250 NaN NaN NaN  ...     NaN NaN   
1973\PYGZhy{}02\PYGZhy{}22    1.6366 NaN NaN NaN NaN NaN 4.6406 NaN NaN NaN  ...  0.0034 NaN   
1973\PYGZhy{}02\PYGZhy{}23    1.6311 NaN NaN NaN NaN NaN 4.6250 NaN NaN NaN  ... \PYGZhy{}0.0034 NaN   
1973\PYGZhy{}02\PYGZhy{}26    1.6311 NaN NaN NaN NaN NaN 4.6250 NaN NaN NaN  ...  0.0000 NaN   
1973\PYGZhy{}02\PYGZhy{}27    1.6311 NaN NaN NaN NaN NaN 4.6250 NaN NaN NaN  ...  0.0000 NaN   

Variable                   Factors                        
Ticker      GS JPM  MS PNC  Mkt\PYGZhy{}RF     SMB    HML     RF  
Date                                                      
1973\PYGZhy{}02\PYGZhy{}21 NaN NaN NaN NaN \PYGZhy{}0.0074 \PYGZhy{}0.0039 0.0054 0.0002  
1973\PYGZhy{}02\PYGZhy{}22 NaN NaN NaN NaN \PYGZhy{}0.0030 \PYGZhy{}0.0037 0.0022 0.0002  
1973\PYGZhy{}02\PYGZhy{}23 NaN NaN NaN NaN \PYGZhy{}0.0108 \PYGZhy{}0.0019 0.0054 0.0002  
1973\PYGZhy{}02\PYGZhy{}26 NaN NaN NaN NaN \PYGZhy{}0.0088 \PYGZhy{}0.0050 0.0054 0.0002  
1973\PYGZhy{}02\PYGZhy{}27 NaN NaN NaN NaN \PYGZhy{}0.0115 \PYGZhy{}0.0018 0.0064 0.0002  

[5 rows x 58 columns]
\end{sphinxVerbatim}

\end{sphinxuseclass}\end{sphinxVerbatimOutput}

\end{sphinxuseclass}
\sphinxAtStartPar
We can use \sphinxcode{\sphinxupquote{.join()}} even though \sphinxcode{\sphinxupquote{stocks\_long}} has a multi index.
\sphinxstyleemphasis{\sphinxstylestrong{Note that re\sphinxhyphen{}running a “self join” can create duplicate columns.}}
We should be careful to run self joins only once!

\begin{sphinxuseclass}{cell}\begin{sphinxVerbatimInput}

\begin{sphinxuseclass}{cell_input}
\begin{sphinxVerbatim}[commandchars=\\\{\}]
\PYG{n}{stocks\PYGZus{}long} \PYG{o}{=} \PYG{n}{stocks\PYGZus{}long}\PYG{o}{.}\PYG{n}{join}\PYG{p}{(}\PYG{n}{ff}\PYG{p}{)}
\PYG{n}{stocks\PYGZus{}long}\PYG{o}{.}\PYG{n}{head}\PYG{p}{(}\PYG{p}{)}
\end{sphinxVerbatim}

\end{sphinxuseclass}\end{sphinxVerbatimInput}
\begin{sphinxVerbatimOutput}

\begin{sphinxuseclass}{cell_output}
\begin{sphinxVerbatim}[commandchars=\\\{\}]
                   Adj Close  Close  Dividends   High    Low   Open  \PYGZbs{}
Date       Ticker                                                     
1973\PYGZhy{}02\PYGZhy{}21 BAC        1.6311 4.6250     0.0000 4.6250 4.6250 4.6250   
1973\PYGZhy{}02\PYGZhy{}22 BAC        1.6366 4.6406     0.0000 4.6406 4.6406 4.6406   
1973\PYGZhy{}02\PYGZhy{}23 BAC        1.6311 4.6250     0.0000 4.6250 4.6250 4.6250   
1973\PYGZhy{}02\PYGZhy{}26 BAC        1.6311 4.6250     0.0000 4.6250 4.6250 4.6250   
1973\PYGZhy{}02\PYGZhy{}27 BAC        1.6311 4.6250     0.0000 4.6250 4.6250 4.6250   

                   Stock Splits      Volume  Returns  Mkt\PYGZhy{}RF     SMB    HML  \PYGZbs{}
Date       Ticker                                                             
1973\PYGZhy{}02\PYGZhy{}21 BAC           0.0000  99200.0000      NaN \PYGZhy{}0.0074 \PYGZhy{}0.0039 0.0054   
1973\PYGZhy{}02\PYGZhy{}22 BAC           0.0000  47200.0000   0.0034 \PYGZhy{}0.0030 \PYGZhy{}0.0037 0.0022   
1973\PYGZhy{}02\PYGZhy{}23 BAC           0.0000 133600.0000  \PYGZhy{}0.0034 \PYGZhy{}0.0108 \PYGZhy{}0.0019 0.0054   
1973\PYGZhy{}02\PYGZhy{}26 BAC           0.0000  24000.0000   0.0000 \PYGZhy{}0.0088 \PYGZhy{}0.0050 0.0054   
1973\PYGZhy{}02\PYGZhy{}27 BAC           0.0000  41600.0000   0.0000 \PYGZhy{}0.0115 \PYGZhy{}0.0018 0.0064   

                      RF  
Date       Ticker         
1973\PYGZhy{}02\PYGZhy{}21 BAC    0.0002  
1973\PYGZhy{}02\PYGZhy{}22 BAC    0.0002  
1973\PYGZhy{}02\PYGZhy{}23 BAC    0.0002  
1973\PYGZhy{}02\PYGZhy{}26 BAC    0.0002  
1973\PYGZhy{}02\PYGZhy{}27 BAC    0.0002  
\end{sphinxVerbatim}

\end{sphinxuseclass}\end{sphinxVerbatimOutput}

\end{sphinxuseclass}

\subsubsection{Write a function \sphinxstyleliteralintitle{\sphinxupquote{download()}} that accepts tickers and returns a wide data frame of returns with the daily benchmark return factors.}
\label{\detokenize{mckinney_08_practice_03:write-a-function-download-that-accepts-tickers-and-returns-a-wide-data-frame-of-returns-with-the-daily-benchmark-return-factors}}
\begin{sphinxuseclass}{cell}\begin{sphinxVerbatimInput}

\begin{sphinxuseclass}{cell_input}
\begin{sphinxVerbatim}[commandchars=\\\{\}]
\PYG{k}{def} \PYG{n+nf}{download}\PYG{p}{(}\PYG{o}{*}\PYG{o}{*}\PYG{n}{kwargs}\PYG{p}{)}\PYG{p}{:}
    \PYG{c+c1}{\PYGZsh{} get stocks data}
    \PYG{n}{tickers} \PYG{o}{=} \PYG{n}{yf}\PYG{o}{.}\PYG{n}{Tickers}\PYG{p}{(}\PYG{o}{*}\PYG{o}{*}\PYG{n}{kwargs}\PYG{p}{)}
    \PYG{n}{stocks} \PYG{o}{=} \PYG{n}{tickers}\PYG{o}{.}\PYG{n}{history}\PYG{p}{(}\PYG{n}{period}\PYG{o}{=}\PYG{l+s+s1}{\PYGZsq{}}\PYG{l+s+s1}{max}\PYG{l+s+s1}{\PYGZsq{}}\PYG{p}{,} \PYG{n}{auto\PYGZus{}adjust}\PYG{o}{=}\PYG{k+kc}{False}\PYG{p}{,} \PYG{n}{progress}\PYG{o}{=}\PYG{k+kc}{False}\PYG{p}{)}
    \PYG{n}{stocks}\PYG{o}{.}\PYG{n}{index} \PYG{o}{=} \PYG{n}{stocks}\PYG{o}{.}\PYG{n}{index}\PYG{o}{.}\PYG{n}{tz\PYGZus{}localize}\PYG{p}{(}\PYG{k+kc}{None}\PYG{p}{)}
    \PYG{n}{stocks}\PYG{o}{.}\PYG{n}{columns}\PYG{o}{.}\PYG{n}{names} \PYG{o}{=} \PYG{p}{[}\PYG{l+s+s1}{\PYGZsq{}}\PYG{l+s+s1}{Variable}\PYG{l+s+s1}{\PYGZsq{}}\PYG{p}{,} \PYG{l+s+s1}{\PYGZsq{}}\PYG{l+s+s1}{Ticker}\PYG{l+s+s1}{\PYGZsq{}}\PYG{p}{]}

    \PYG{n}{\PYGZus{}} \PYG{o}{=} \PYG{n}{pd}\PYG{o}{.}\PYG{n}{MultiIndex}\PYG{o}{.}\PYG{n}{from\PYGZus{}product}\PYG{p}{(}\PYG{p}{[}\PYG{p}{[}\PYG{l+s+s1}{\PYGZsq{}}\PYG{l+s+s1}{Returns}\PYG{l+s+s1}{\PYGZsq{}}\PYG{p}{]}\PYG{p}{,} \PYG{n}{stocks}\PYG{p}{[}\PYG{l+s+s1}{\PYGZsq{}}\PYG{l+s+s1}{Adj Close}\PYG{l+s+s1}{\PYGZsq{}}\PYG{p}{]}\PYG{o}{.}\PYG{n}{columns}\PYG{p}{]}\PYG{p}{)}
    \PYG{n}{stocks}\PYG{p}{[}\PYG{n}{\PYGZus{}}\PYG{p}{]} \PYG{o}{=} \PYG{n}{stocks}\PYG{p}{[}\PYG{l+s+s1}{\PYGZsq{}}\PYG{l+s+s1}{Adj Close}\PYG{l+s+s1}{\PYGZsq{}}\PYG{p}{]}\PYG{o}{.}\PYG{n}{pct\PYGZus{}change}\PYG{p}{(}\PYG{p}{)}

    \PYG{c+c1}{\PYGZsh{} get factor data}
    \PYG{n}{ff} \PYG{o}{=} \PYG{p}{(}
        \PYG{n}{pdr}\PYG{o}{.}\PYG{n}{DataReader}\PYG{p}{(}
            \PYG{n}{name}\PYG{o}{=}\PYG{l+s+s1}{\PYGZsq{}}\PYG{l+s+s1}{F\PYGZhy{}F\PYGZus{}Research\PYGZus{}Data\PYGZus{}Factors\PYGZus{}daily}\PYG{l+s+s1}{\PYGZsq{}}\PYG{p}{,}
            \PYG{n}{data\PYGZus{}source}\PYG{o}{=}\PYG{l+s+s1}{\PYGZsq{}}\PYG{l+s+s1}{famafrench}\PYG{l+s+s1}{\PYGZsq{}}\PYG{p}{,}
            \PYG{n}{start}\PYG{o}{=}\PYG{l+s+s1}{\PYGZsq{}}\PYG{l+s+s1}{1900}\PYG{l+s+s1}{\PYGZsq{}}\PYG{p}{,}
            \PYG{n}{session}\PYG{o}{=}\PYG{n}{session}
        \PYG{p}{)}
        \PYG{p}{[}\PYG{l+m+mi}{0}\PYG{p}{]}
        \PYG{o}{.}\PYG{n}{div}\PYG{p}{(}\PYG{l+m+mi}{100}\PYG{p}{)}
    \PYG{p}{)}

    \PYG{c+c1}{\PYGZsh{} combine}
    \PYG{n}{\PYGZus{}} \PYG{o}{=} \PYG{n}{pd}\PYG{o}{.}\PYG{n}{MultiIndex}\PYG{o}{.}\PYG{n}{from\PYGZus{}product}\PYG{p}{(}\PYG{p}{[}\PYG{p}{[}\PYG{l+s+s1}{\PYGZsq{}}\PYG{l+s+s1}{Factors}\PYG{l+s+s1}{\PYGZsq{}}\PYG{p}{]}\PYG{p}{,} \PYG{n}{ff}\PYG{o}{.}\PYG{n}{columns}\PYG{p}{]}\PYG{p}{)}
    \PYG{n}{stocks}\PYG{p}{[}\PYG{n}{\PYGZus{}}\PYG{p}{]} \PYG{o}{=} \PYG{n}{ff}

    \PYG{k}{return} \PYG{n}{stocks}
\end{sphinxVerbatim}

\end{sphinxuseclass}\end{sphinxVerbatimInput}

\end{sphinxuseclass}
\sphinxAtStartPar
Below I will provide a more compact and flexible version of \sphinxcode{\sphinxupquote{download()}}.


\subsubsection{Download earnings per share for the stocks in \sphinxstyleliteralintitle{\sphinxupquote{stocks}} and combine to a long data frame \sphinxstyleliteralintitle{\sphinxupquote{earnings}}.}
\label{\detokenize{mckinney_08_practice_03:download-earnings-per-share-for-the-stocks-in-stocks-and-combine-to-a-long-data-frame-earnings}}
\sphinxAtStartPar
Use the \sphinxcode{\sphinxupquote{.earnings\_dates}} method described \sphinxhref{https://pypi.org/project/yfinance/}{here}.
Use \sphinxcode{\sphinxupquote{pd.concat()}} to combine the result of each the \sphinxcode{\sphinxupquote{.earnings\_date}} data frames and assign them to a new data frame \sphinxcode{\sphinxupquote{earnings}}.
Name the row indexes \sphinxcode{\sphinxupquote{Ticker}} and \sphinxcode{\sphinxupquote{Date}} and swap to match the order of the row index in \sphinxcode{\sphinxupquote{stocks\_long}}.

\begin{sphinxuseclass}{cell}\begin{sphinxVerbatimInput}

\begin{sphinxuseclass}{cell_input}
\begin{sphinxVerbatim}[commandchars=\\\{\}]
\PYG{c+c1}{\PYGZsh{} some students had to update yfinance to use the .earnings\PYGZus{}dates atrtibute}
\PYG{c+c1}{\PYGZsh{} \PYGZpc{}pip install \PYGZhy{}U yfinance}
\end{sphinxVerbatim}

\end{sphinxuseclass}\end{sphinxVerbatimInput}

\end{sphinxuseclass}
\begin{sphinxuseclass}{cell}\begin{sphinxVerbatimInput}

\begin{sphinxuseclass}{cell_input}
\begin{sphinxVerbatim}[commandchars=\\\{\}]
\PYG{n}{tickers}\PYG{o}{.}\PYG{n}{tickers}\PYG{p}{[}\PYG{l+s+s1}{\PYGZsq{}}\PYG{l+s+s1}{BAC}\PYG{l+s+s1}{\PYGZsq{}}\PYG{p}{]}\PYG{o}{.}\PYG{n}{earnings\PYGZus{}dates}\PYG{o}{.}\PYG{n}{head}\PYG{p}{(}\PYG{l+m+mi}{2}\PYG{p}{)}
\end{sphinxVerbatim}

\end{sphinxuseclass}\end{sphinxVerbatimInput}
\begin{sphinxVerbatimOutput}

\begin{sphinxuseclass}{cell_output}
\begin{sphinxVerbatim}[commandchars=\\\{\}]
                           EPS Estimate  Reported EPS  Surprise(\PYGZpc{})
Earnings Date                                                     
2024\PYGZhy{}01\PYGZhy{}11 08:00:00\PYGZhy{}05:00           NaN           NaN          NaN
2023\PYGZhy{}10\PYGZhy{}17 08:00:00\PYGZhy{}04:00           NaN           NaN          NaN
\end{sphinxVerbatim}

\end{sphinxuseclass}\end{sphinxVerbatimOutput}

\end{sphinxuseclass}
\begin{sphinxuseclass}{cell}\begin{sphinxVerbatimInput}

\begin{sphinxuseclass}{cell_input}
\begin{sphinxVerbatim}[commandchars=\\\{\}]
\PYG{n}{earnings} \PYG{o}{=} \PYG{p}{(}
    \PYG{n}{pd}\PYG{o}{.}\PYG{n}{concat}\PYG{p}{(}
        \PYG{n}{objs}\PYG{o}{=}\PYG{p}{[}\PYG{n}{tickers}\PYG{o}{.}\PYG{n}{tickers}\PYG{p}{[}\PYG{n}{t}\PYG{p}{]}\PYG{o}{.}\PYG{n}{earnings\PYGZus{}dates} \PYG{k}{for} \PYG{n}{t} \PYG{o+ow}{in} \PYG{n}{tickers}\PYG{o}{.}\PYG{n}{tickers}\PYG{p}{]}\PYG{p}{,}
        \PYG{n}{keys}\PYG{o}{=}\PYG{n}{tickers}\PYG{o}{.}\PYG{n}{tickers}\PYG{p}{,}
        \PYG{n}{names}\PYG{o}{=}\PYG{p}{[}\PYG{l+s+s1}{\PYGZsq{}}\PYG{l+s+s1}{Ticker}\PYG{l+s+s1}{\PYGZsq{}}\PYG{p}{,} \PYG{l+s+s1}{\PYGZsq{}}\PYG{l+s+s1}{Date}\PYG{l+s+s1}{\PYGZsq{}}\PYG{p}{]}
    \PYG{p}{)}
    \PYG{o}{.}\PYG{n}{swaplevel}\PYG{p}{(}\PYG{p}{)}
    \PYG{o}{.}\PYG{n}{rename\PYGZus{}axis}\PYG{p}{(}\PYG{n}{columns}\PYG{o}{=}\PYG{l+s+s1}{\PYGZsq{}}\PYG{l+s+s1}{Variable}\PYG{l+s+s1}{\PYGZsq{}}\PYG{p}{)}
\PYG{p}{)}

\PYG{n}{earnings}\PYG{o}{.}\PYG{n}{head}\PYG{p}{(}\PYG{l+m+mi}{2}\PYG{p}{)}
\end{sphinxVerbatim}

\end{sphinxuseclass}\end{sphinxVerbatimInput}
\begin{sphinxVerbatimOutput}

\begin{sphinxuseclass}{cell_output}
\begin{sphinxVerbatim}[commandchars=\\\{\}]
Variable                          EPS Estimate  Reported EPS  Surprise(\PYGZpc{})
Date                      Ticker                                         
2024\PYGZhy{}01\PYGZhy{}11 08:00:00\PYGZhy{}05:00 BAC              NaN           NaN          NaN
2023\PYGZhy{}10\PYGZhy{}17 08:00:00\PYGZhy{}04:00 BAC              NaN           NaN          NaN
\end{sphinxVerbatim}

\end{sphinxuseclass}\end{sphinxVerbatimOutput}

\end{sphinxuseclass}

\subsubsection{Combine \sphinxstyleliteralintitle{\sphinxupquote{earnings}} with the returns from \sphinxstyleliteralintitle{\sphinxupquote{stocks\_long}}.}
\label{\detokenize{mckinney_08_practice_03:combine-earnings-with-the-returns-from-stocks-long}}
\sphinxAtStartPar
\sphinxstyleemphasis{\sphinxstylestrong{It is easier to leave \sphinxcode{\sphinxupquote{stocks}} and \sphinxcode{\sphinxupquote{stocks\_long}} as\sphinxhyphen{}is and work with slices \sphinxcode{\sphinxupquote{returns}} and \sphinxcode{\sphinxupquote{returns\_long}}.}}
Use the \sphinxcode{\sphinxupquote{tz\_localize('America/New\_York')}} method add time zone information back to \sphinxcode{\sphinxupquote{returns.index}} and use \sphinxcode{\sphinxupquote{pd.to\_timedelta(16, unit='h')}} to set time to the market close in New York City.
Use \sphinxhref{https://pandas.pydata.org/pandas-docs/stable/reference/api/pandas.merge\_asof.html}{\sphinxcode{\sphinxupquote{pd.merge\_asof()}}} to match earnings announcement dates and times to appropriate return periods.
For example, if a firm announces earnings after the close at 5 PM on February 7, we want to match the return period from 4 PM on February 7 to 4 PM on February 8.

\begin{sphinxuseclass}{cell}\begin{sphinxVerbatimInput}

\begin{sphinxuseclass}{cell_input}
\begin{sphinxVerbatim}[commandchars=\\\{\}]
\PYG{n}{returns} \PYG{o}{=} \PYG{n}{stocks}\PYG{p}{[}\PYG{l+s+s1}{\PYGZsq{}}\PYG{l+s+s1}{Returns}\PYG{l+s+s1}{\PYGZsq{}}\PYG{p}{]}
\PYG{n}{returns}\PYG{o}{.}\PYG{n}{index} \PYG{o}{=} \PYG{n}{returns}\PYG{o}{.}\PYG{n}{index}\PYG{o}{.}\PYG{n}{tz\PYGZus{}localize}\PYG{p}{(}\PYG{l+s+s1}{\PYGZsq{}}\PYG{l+s+s1}{America/New\PYGZus{}York}\PYG{l+s+s1}{\PYGZsq{}}\PYG{p}{)} \PYG{o}{+} \PYG{n}{pd}\PYG{o}{.}\PYG{n}{to\PYGZus{}timedelta}\PYG{p}{(}\PYG{l+m+mi}{16}\PYG{p}{,} \PYG{n}{unit}\PYG{o}{=}\PYG{l+s+s1}{\PYGZsq{}}\PYG{l+s+s1}{h}\PYG{l+s+s1}{\PYGZsq{}}\PYG{p}{)}
\PYG{n}{returns\PYGZus{}long} \PYG{o}{=} \PYG{n}{returns}\PYG{o}{.}\PYG{n}{stack}\PYG{p}{(}\PYG{p}{)}\PYG{o}{.}\PYG{n}{to\PYGZus{}frame}\PYG{p}{(}\PYG{l+s+s1}{\PYGZsq{}}\PYG{l+s+s1}{Returns}\PYG{l+s+s1}{\PYGZsq{}}\PYG{p}{)}
\PYG{n}{returns\PYGZus{}long}\PYG{o}{.}\PYG{n}{columns}\PYG{o}{.}\PYG{n}{name} \PYG{o}{=} \PYG{l+s+s1}{\PYGZsq{}}\PYG{l+s+s1}{Variable}\PYG{l+s+s1}{\PYGZsq{}}
\PYG{n}{returns\PYGZus{}long}\PYG{o}{.}\PYG{n}{head}\PYG{p}{(}\PYG{p}{)}
\end{sphinxVerbatim}

\end{sphinxuseclass}\end{sphinxVerbatimInput}
\begin{sphinxVerbatimOutput}

\begin{sphinxuseclass}{cell_output}
\begin{sphinxVerbatim}[commandchars=\\\{\}]
Variable                          Returns
Date                      Ticker         
1973\PYGZhy{}02\PYGZhy{}22 16:00:00\PYGZhy{}05:00 BAC      0.0034
1973\PYGZhy{}02\PYGZhy{}23 16:00:00\PYGZhy{}05:00 BAC     \PYGZhy{}0.0034
1973\PYGZhy{}02\PYGZhy{}26 16:00:00\PYGZhy{}05:00 BAC      0.0000
1973\PYGZhy{}02\PYGZhy{}27 16:00:00\PYGZhy{}05:00 BAC      0.0000
1973\PYGZhy{}02\PYGZhy{}28 16:00:00\PYGZhy{}05:00 BAC      0.0000
\end{sphinxVerbatim}

\end{sphinxuseclass}\end{sphinxVerbatimOutput}

\end{sphinxuseclass}
\begin{sphinxuseclass}{cell}\begin{sphinxVerbatimInput}

\begin{sphinxuseclass}{cell_input}
\begin{sphinxVerbatim}[commandchars=\\\{\}]
\PYG{n}{surprises} \PYG{o}{=} \PYG{p}{(}
    \PYG{n}{pd}\PYG{o}{.}\PYG{n}{merge\PYGZus{}asof}\PYG{p}{(}
        \PYG{n}{left}\PYG{o}{=}\PYG{n}{earnings}\PYG{o}{.}\PYG{n}{sort\PYGZus{}index}\PYG{p}{(}\PYG{n}{level}\PYG{o}{=}\PYG{p}{[}\PYG{l+s+s1}{\PYGZsq{}}\PYG{l+s+s1}{Date}\PYG{l+s+s1}{\PYGZsq{}}\PYG{p}{,} \PYG{l+s+s1}{\PYGZsq{}}\PYG{l+s+s1}{Ticker}\PYG{l+s+s1}{\PYGZsq{}}\PYG{p}{]}\PYG{p}{)}\PYG{p}{,}
        \PYG{n}{right}\PYG{o}{=}\PYG{n}{returns\PYGZus{}long}\PYG{o}{.}\PYG{n}{sort\PYGZus{}index}\PYG{p}{(}\PYG{n}{level}\PYG{o}{=}\PYG{p}{[}\PYG{l+s+s1}{\PYGZsq{}}\PYG{l+s+s1}{Date}\PYG{l+s+s1}{\PYGZsq{}}\PYG{p}{,} \PYG{l+s+s1}{\PYGZsq{}}\PYG{l+s+s1}{Ticker}\PYG{l+s+s1}{\PYGZsq{}}\PYG{p}{]}\PYG{p}{)}\PYG{p}{,}
        \PYG{n}{on}\PYG{o}{=}\PYG{l+s+s1}{\PYGZsq{}}\PYG{l+s+s1}{Date}\PYG{l+s+s1}{\PYGZsq{}}\PYG{p}{,}
        \PYG{n}{by}\PYG{o}{=}\PYG{l+s+s1}{\PYGZsq{}}\PYG{l+s+s1}{Ticker}\PYG{l+s+s1}{\PYGZsq{}}\PYG{p}{,}
        \PYG{n}{direction}\PYG{o}{=}\PYG{l+s+s1}{\PYGZsq{}}\PYG{l+s+s1}{forward}\PYG{l+s+s1}{\PYGZsq{}}\PYG{p}{,}
        \PYG{n}{allow\PYGZus{}exact\PYGZus{}matches}\PYG{o}{=}\PYG{k+kc}{False}
    \PYG{p}{)}
    \PYG{o}{.}\PYG{n}{set\PYGZus{}index}\PYG{p}{(}\PYG{p}{[}\PYG{l+s+s1}{\PYGZsq{}}\PYG{l+s+s1}{Date}\PYG{l+s+s1}{\PYGZsq{}}\PYG{p}{,} \PYG{l+s+s1}{\PYGZsq{}}\PYG{l+s+s1}{Ticker}\PYG{l+s+s1}{\PYGZsq{}}\PYG{p}{]}\PYG{p}{)}
\PYG{p}{)}

\PYG{n}{surprises}\PYG{o}{.}\PYG{n}{head}\PYG{p}{(}\PYG{p}{)}
\end{sphinxVerbatim}

\end{sphinxuseclass}\end{sphinxVerbatimInput}
\begin{sphinxVerbatimOutput}

\begin{sphinxuseclass}{cell_output}
\begin{sphinxVerbatim}[commandchars=\\\{\}]
Variable                          EPS Estimate  Reported EPS  Surprise(\PYGZpc{})  \PYGZbs{}
Date                      Ticker                                            
2021\PYGZhy{}04\PYGZhy{}14 03:00:00\PYGZhy{}04:00 GS           10.2200       18.6000       0.8194   
2021\PYGZhy{}04\PYGZhy{}16 03:00:00\PYGZhy{}04:00 MS            1.7000        2.1900       0.2890   
2021\PYGZhy{}07\PYGZhy{}13 03:00:00\PYGZhy{}04:00 GS           10.2400       15.0200       0.4668   
2021\PYGZhy{}07\PYGZhy{}15 03:00:00\PYGZhy{}04:00 MS            1.6500        1.8500       0.1185   
2021\PYGZhy{}10\PYGZhy{}14 03:00:00\PYGZhy{}04:00 MS            1.6900        1.9800       0.1751   

Variable                          Returns  
Date                      Ticker           
2021\PYGZhy{}04\PYGZhy{}14 03:00:00\PYGZhy{}04:00 GS       0.0234  
2021\PYGZhy{}04\PYGZhy{}16 03:00:00\PYGZhy{}04:00 MS      \PYGZhy{}0.0276  
2021\PYGZhy{}07\PYGZhy{}13 03:00:00\PYGZhy{}04:00 GS      \PYGZhy{}0.0119  
2021\PYGZhy{}07\PYGZhy{}15 03:00:00\PYGZhy{}04:00 MS       0.0018  
2021\PYGZhy{}10\PYGZhy{}14 03:00:00\PYGZhy{}04:00 MS       0.0248  
\end{sphinxVerbatim}

\end{sphinxuseclass}\end{sphinxVerbatimOutput}

\end{sphinxuseclass}
\begin{sphinxuseclass}{cell}\begin{sphinxVerbatimInput}

\begin{sphinxuseclass}{cell_input}
\begin{sphinxVerbatim}[commandchars=\\\{\}]
\PYG{n}{surprises}\PYG{o}{.}\PYG{n}{corr}\PYG{p}{(}\PYG{p}{)}
\end{sphinxVerbatim}

\end{sphinxuseclass}\end{sphinxVerbatimInput}
\begin{sphinxVerbatimOutput}

\begin{sphinxuseclass}{cell_output}
\begin{sphinxVerbatim}[commandchars=\\\{\}]
Variable      EPS Estimate  Reported EPS  Surprise(\PYGZpc{})  Returns
Variable                                                      
EPS Estimate        1.0000        0.9444       0.3620  \PYGZhy{}0.2239
Reported EPS        0.9444        1.0000       0.6171  \PYGZhy{}0.0615
Surprise(\PYGZpc{})         0.3620        0.6171       1.0000   0.4397
Returns            \PYGZhy{}0.2239       \PYGZhy{}0.0615       0.4397   1.0000
\end{sphinxVerbatim}

\end{sphinxuseclass}\end{sphinxVerbatimOutput}

\end{sphinxuseclass}

\subsubsection{Plot the relation between daily returns and earnings surprises}
\label{\detokenize{mckinney_08_practice_03:plot-the-relation-between-daily-returns-and-earnings-surprises}}
\sphinxAtStartPar
Three options in increasing difficulty:
\begin{enumerate}
\sphinxsetlistlabels{\arabic}{enumi}{enumii}{}{.}%
\item {} 
\sphinxAtStartPar
Scatter plot

\item {} 
\sphinxAtStartPar
Scatter plot with a best\sphinxhyphen{}fit line using \sphinxcode{\sphinxupquote{regplot()}} from the seaborn package

\item {} 
\sphinxAtStartPar
Bar plot using \sphinxcode{\sphinxupquote{barplot()}} from the seaborn package after using \sphinxcode{\sphinxupquote{pd.qcut()}} to form five groups on earnings surprises

\end{enumerate}

\begin{sphinxuseclass}{cell}\begin{sphinxVerbatimInput}

\begin{sphinxuseclass}{cell_input}
\begin{sphinxVerbatim}[commandchars=\\\{\}]
\PYG{p}{(}
    \PYG{n}{surprises}
    \PYG{p}{[}\PYG{p}{[}\PYG{l+s+s1}{\PYGZsq{}}\PYG{l+s+s1}{Surprise(}\PYG{l+s+s1}{\PYGZpc{}}\PYG{l+s+s1}{)}\PYG{l+s+s1}{\PYGZsq{}}\PYG{p}{,} \PYG{l+s+s1}{\PYGZsq{}}\PYG{l+s+s1}{Returns}\PYG{l+s+s1}{\PYGZsq{}}\PYG{p}{]}\PYG{p}{]}
    \PYG{o}{.}\PYG{n}{mul}\PYG{p}{(}\PYG{l+m+mi}{100}\PYG{p}{)}
    \PYG{o}{.}\PYG{n}{plot}\PYG{p}{(}\PYG{n}{x}\PYG{o}{=}\PYG{l+s+s1}{\PYGZsq{}}\PYG{l+s+s1}{Surprise(}\PYG{l+s+s1}{\PYGZpc{}}\PYG{l+s+s1}{)}\PYG{l+s+s1}{\PYGZsq{}}\PYG{p}{,} \PYG{n}{y}\PYG{o}{=}\PYG{l+s+s1}{\PYGZsq{}}\PYG{l+s+s1}{Returns}\PYG{l+s+s1}{\PYGZsq{}}\PYG{p}{,} \PYG{n}{kind}\PYG{o}{=}\PYG{l+s+s1}{\PYGZsq{}}\PYG{l+s+s1}{scatter}\PYG{l+s+s1}{\PYGZsq{}}\PYG{p}{)}
\PYG{p}{)}
\PYG{n}{plt}\PYG{o}{.}\PYG{n}{xlabel}\PYG{p}{(}\PYG{l+s+s1}{\PYGZsq{}}\PYG{l+s+s1}{Earnings Suprise (}\PYG{l+s+s1}{\PYGZpc{}}\PYG{l+s+s1}{)}\PYG{l+s+s1}{\PYGZsq{}}\PYG{p}{)}
\PYG{n}{plt}\PYG{o}{.}\PYG{n}{ylabel}\PYG{p}{(}\PYG{l+s+s1}{\PYGZsq{}}\PYG{l+s+s1}{Announcement Return (}\PYG{l+s+s1}{\PYGZpc{}}\PYG{l+s+s1}{)}\PYG{l+s+s1}{\PYGZsq{}}\PYG{p}{)}

\PYG{n}{\PYGZus{}} \PYG{o}{=} \PYG{l+s+s1}{\PYGZsq{}}\PYG{l+s+s1}{ }\PYG{l+s+s1}{\PYGZsq{}}\PYG{o}{.}\PYG{n}{join}\PYG{p}{(}\PYG{n}{surprises}\PYG{o}{.}\PYG{n}{index}\PYG{o}{.}\PYG{n}{get\PYGZus{}level\PYGZus{}values}\PYG{p}{(}\PYG{l+s+s1}{\PYGZsq{}}\PYG{l+s+s1}{Ticker}\PYG{l+s+s1}{\PYGZsq{}}\PYG{p}{)}\PYG{o}{.}\PYG{n}{unique}\PYG{p}{(}\PYG{p}{)}\PYG{p}{)}
\PYG{n}{\PYGZus{}\PYGZus{}} \PYG{o}{=} \PYG{n}{surprises}\PYG{o}{.}\PYG{n}{index}\PYG{o}{.}\PYG{n}{get\PYGZus{}level\PYGZus{}values}\PYG{p}{(}\PYG{l+s+s1}{\PYGZsq{}}\PYG{l+s+s1}{Date}\PYG{l+s+s1}{\PYGZsq{}}\PYG{p}{)}
\PYG{n}{plt}\PYG{o}{.}\PYG{n}{title}\PYG{p}{(}\PYG{l+s+sa}{f}\PYG{l+s+s1}{\PYGZsq{}}\PYG{l+s+s1}{Earnings Announcements}\PYG{l+s+se}{\PYGZbs{}n}\PYG{l+s+s1}{ for }\PYG{l+s+si}{\PYGZob{}}\PYG{n}{\PYGZus{}}\PYG{l+s+si}{\PYGZcb{}}\PYG{l+s+se}{\PYGZbs{}n}\PYG{l+s+s1}{ from }\PYG{l+s+si}{\PYGZob{}}\PYG{n}{\PYGZus{}\PYGZus{}}\PYG{o}{.}\PYG{n}{min}\PYG{p}{(}\PYG{p}{)}\PYG{l+s+si}{:}\PYG{l+s+s1}{\PYGZpc{}B \PYGZpc{}Y}\PYG{l+s+si}{\PYGZcb{}}\PYG{l+s+s1}{ to }\PYG{l+s+si}{\PYGZob{}}\PYG{n}{\PYGZus{}\PYGZus{}}\PYG{o}{.}\PYG{n}{max}\PYG{p}{(}\PYG{p}{)}\PYG{l+s+si}{:}\PYG{l+s+s1}{\PYGZpc{}B \PYGZpc{}Y}\PYG{l+s+si}{\PYGZcb{}}\PYG{l+s+s1}{\PYGZsq{}}\PYG{p}{)}
\PYG{n}{plt}\PYG{o}{.}\PYG{n}{show}\PYG{p}{(}\PYG{p}{)}
\end{sphinxVerbatim}

\end{sphinxuseclass}\end{sphinxVerbatimInput}
\begin{sphinxVerbatimOutput}

\begin{sphinxuseclass}{cell_output}
\noindent\sphinxincludegraphics{{ddb8d2140998e68e00f41f701aac3893910abc69a528374a6462b885622c8173}.png}

\end{sphinxuseclass}\end{sphinxVerbatimOutput}

\end{sphinxuseclass}
\begin{sphinxuseclass}{cell}\begin{sphinxVerbatimInput}

\begin{sphinxuseclass}{cell_input}
\begin{sphinxVerbatim}[commandchars=\\\{\}]
\PYG{k+kn}{import} \PYG{n+nn}{seaborn} \PYG{k}{as} \PYG{n+nn}{sns}
\end{sphinxVerbatim}

\end{sphinxuseclass}\end{sphinxVerbatimInput}

\end{sphinxuseclass}
\begin{sphinxuseclass}{cell}\begin{sphinxVerbatimInput}

\begin{sphinxuseclass}{cell_input}
\begin{sphinxVerbatim}[commandchars=\\\{\}]
\PYG{n}{sns}\PYG{o}{.}\PYG{n}{regplot}\PYG{p}{(}
    \PYG{n}{x}\PYG{o}{=}\PYG{l+s+s1}{\PYGZsq{}}\PYG{l+s+s1}{Surprise(}\PYG{l+s+s1}{\PYGZpc{}}\PYG{l+s+s1}{)}\PYG{l+s+s1}{\PYGZsq{}}\PYG{p}{,}
    \PYG{n}{y}\PYG{o}{=} \PYG{l+s+s1}{\PYGZsq{}}\PYG{l+s+s1}{Returns}\PYG{l+s+s1}{\PYGZsq{}}\PYG{p}{,}
    \PYG{n}{data}\PYG{o}{=}\PYG{n}{surprises}\PYG{p}{[}\PYG{p}{[}\PYG{l+s+s1}{\PYGZsq{}}\PYG{l+s+s1}{Surprise(}\PYG{l+s+s1}{\PYGZpc{}}\PYG{l+s+s1}{)}\PYG{l+s+s1}{\PYGZsq{}}\PYG{p}{,} \PYG{l+s+s1}{\PYGZsq{}}\PYG{l+s+s1}{Returns}\PYG{l+s+s1}{\PYGZsq{}}\PYG{p}{]}\PYG{p}{]}\PYG{o}{.}\PYG{n}{mul}\PYG{p}{(}\PYG{l+m+mi}{100}\PYG{p}{)}
\PYG{p}{)}

\PYG{n}{plt}\PYG{o}{.}\PYG{n}{xlabel}\PYG{p}{(}\PYG{l+s+s1}{\PYGZsq{}}\PYG{l+s+s1}{Earnings Suprise (}\PYG{l+s+s1}{\PYGZpc{}}\PYG{l+s+s1}{)}\PYG{l+s+s1}{\PYGZsq{}}\PYG{p}{)}
\PYG{n}{plt}\PYG{o}{.}\PYG{n}{ylabel}\PYG{p}{(}\PYG{l+s+s1}{\PYGZsq{}}\PYG{l+s+s1}{Announcement Return (}\PYG{l+s+s1}{\PYGZpc{}}\PYG{l+s+s1}{)}\PYG{l+s+s1}{\PYGZsq{}}\PYG{p}{)}

\PYG{n}{\PYGZus{}} \PYG{o}{=} \PYG{l+s+s1}{\PYGZsq{}}\PYG{l+s+s1}{ }\PYG{l+s+s1}{\PYGZsq{}}\PYG{o}{.}\PYG{n}{join}\PYG{p}{(}\PYG{n}{surprises}\PYG{o}{.}\PYG{n}{index}\PYG{o}{.}\PYG{n}{get\PYGZus{}level\PYGZus{}values}\PYG{p}{(}\PYG{l+s+s1}{\PYGZsq{}}\PYG{l+s+s1}{Ticker}\PYG{l+s+s1}{\PYGZsq{}}\PYG{p}{)}\PYG{o}{.}\PYG{n}{unique}\PYG{p}{(}\PYG{p}{)}\PYG{p}{)}
\PYG{n}{\PYGZus{}\PYGZus{}} \PYG{o}{=} \PYG{n}{surprises}\PYG{o}{.}\PYG{n}{index}\PYG{o}{.}\PYG{n}{get\PYGZus{}level\PYGZus{}values}\PYG{p}{(}\PYG{l+s+s1}{\PYGZsq{}}\PYG{l+s+s1}{Date}\PYG{l+s+s1}{\PYGZsq{}}\PYG{p}{)}
\PYG{n}{plt}\PYG{o}{.}\PYG{n}{title}\PYG{p}{(}\PYG{l+s+sa}{f}\PYG{l+s+s1}{\PYGZsq{}}\PYG{l+s+s1}{Earnings Announcements}\PYG{l+s+se}{\PYGZbs{}n}\PYG{l+s+s1}{ for }\PYG{l+s+si}{\PYGZob{}}\PYG{n}{\PYGZus{}}\PYG{l+s+si}{\PYGZcb{}}\PYG{l+s+se}{\PYGZbs{}n}\PYG{l+s+s1}{ from }\PYG{l+s+si}{\PYGZob{}}\PYG{n}{\PYGZus{}\PYGZus{}}\PYG{o}{.}\PYG{n}{min}\PYG{p}{(}\PYG{p}{)}\PYG{l+s+si}{:}\PYG{l+s+s1}{\PYGZpc{}B \PYGZpc{}Y}\PYG{l+s+si}{\PYGZcb{}}\PYG{l+s+s1}{ to }\PYG{l+s+si}{\PYGZob{}}\PYG{n}{\PYGZus{}\PYGZus{}}\PYG{o}{.}\PYG{n}{max}\PYG{p}{(}\PYG{p}{)}\PYG{l+s+si}{:}\PYG{l+s+s1}{\PYGZpc{}B \PYGZpc{}Y}\PYG{l+s+si}{\PYGZcb{}}\PYG{l+s+s1}{\PYGZsq{}}\PYG{p}{)}
\PYG{n}{plt}\PYG{o}{.}\PYG{n}{show}\PYG{p}{(}\PYG{p}{)}
\end{sphinxVerbatim}

\end{sphinxuseclass}\end{sphinxVerbatimInput}
\begin{sphinxVerbatimOutput}

\begin{sphinxuseclass}{cell_output}
\noindent\sphinxincludegraphics{{bf6fa0db63fc3e745ec9772cc7b5eeee3b544d3ce5dbc0eea55cbaaa8f83b705}.png}

\end{sphinxuseclass}\end{sphinxVerbatimOutput}

\end{sphinxuseclass}
\begin{sphinxuseclass}{cell}\begin{sphinxVerbatimInput}

\begin{sphinxuseclass}{cell_input}
\begin{sphinxVerbatim}[commandchars=\\\{\}]
\PYG{n}{surprises}\PYG{p}{[}\PYG{l+s+s1}{\PYGZsq{}}\PYG{l+s+s1}{ESQ}\PYG{l+s+s1}{\PYGZsq{}}\PYG{p}{]} \PYG{o}{=} \PYG{n}{pd}\PYG{o}{.}\PYG{n}{qcut}\PYG{p}{(}\PYG{n}{x}\PYG{o}{=}\PYG{n}{surprises}\PYG{p}{[}\PYG{l+s+s1}{\PYGZsq{}}\PYG{l+s+s1}{Surprise(}\PYG{l+s+s1}{\PYGZpc{}}\PYG{l+s+s1}{)}\PYG{l+s+s1}{\PYGZsq{}}\PYG{p}{]}\PYG{p}{,} \PYG{n}{q}\PYG{o}{=}\PYG{l+m+mi}{5}\PYG{p}{,} \PYG{n}{labels}\PYG{o}{=}\PYG{k+kc}{False}\PYG{p}{)}
\end{sphinxVerbatim}

\end{sphinxuseclass}\end{sphinxVerbatimInput}

\end{sphinxuseclass}
\begin{sphinxuseclass}{cell}\begin{sphinxVerbatimInput}

\begin{sphinxuseclass}{cell_input}
\begin{sphinxVerbatim}[commandchars=\\\{\}]
\PYG{n}{sns}\PYG{o}{.}\PYG{n}{barplot}\PYG{p}{(}
    \PYG{n}{x}\PYG{o}{=}\PYG{l+s+s1}{\PYGZsq{}}\PYG{l+s+s1}{ESQ}\PYG{l+s+s1}{\PYGZsq{}}\PYG{p}{,}
    \PYG{n}{y}\PYG{o}{=} \PYG{l+s+s1}{\PYGZsq{}}\PYG{l+s+s1}{Returns}\PYG{l+s+s1}{\PYGZsq{}}\PYG{p}{,}
    \PYG{n}{data}\PYG{o}{=}\PYG{p}{(}
        \PYG{n}{surprises}
        \PYG{p}{[}\PYG{p}{[}\PYG{l+s+s1}{\PYGZsq{}}\PYG{l+s+s1}{Surprise(}\PYG{l+s+s1}{\PYGZpc{}}\PYG{l+s+s1}{)}\PYG{l+s+s1}{\PYGZsq{}}\PYG{p}{,} \PYG{l+s+s1}{\PYGZsq{}}\PYG{l+s+s1}{Returns}\PYG{l+s+s1}{\PYGZsq{}}\PYG{p}{]}\PYG{p}{]}
        \PYG{o}{.}\PYG{n}{mul}\PYG{p}{(}\PYG{l+m+mi}{100}\PYG{p}{)}
        \PYG{o}{.}\PYG{n}{assign}\PYG{p}{(}\PYG{n}{ESQ} \PYG{o}{=} \PYG{k}{lambda} \PYG{n}{x}\PYG{p}{:} \PYG{n}{pd}\PYG{o}{.}\PYG{n}{qcut}\PYG{p}{(}\PYG{n}{x}\PYG{o}{=}\PYG{n}{x}\PYG{p}{[}\PYG{l+s+s1}{\PYGZsq{}}\PYG{l+s+s1}{Surprise(}\PYG{l+s+s1}{\PYGZpc{}}\PYG{l+s+s1}{)}\PYG{l+s+s1}{\PYGZsq{}}\PYG{p}{]}\PYG{p}{,} \PYG{n}{q}\PYG{o}{=}\PYG{l+m+mi}{5}\PYG{p}{,} \PYG{n}{labels}\PYG{o}{=}\PYG{k+kc}{False}\PYG{p}{)}\PYG{p}{)}
    \PYG{p}{)}
\PYG{p}{)}

\PYG{n}{plt}\PYG{o}{.}\PYG{n}{xlabel}\PYG{p}{(}\PYG{l+s+s1}{\PYGZsq{}}\PYG{l+s+s1}{Earnings Suprise Portfolio}\PYG{l+s+s1}{\PYGZsq{}}\PYG{p}{)}
\PYG{n}{plt}\PYG{o}{.}\PYG{n}{ylabel}\PYG{p}{(}\PYG{l+s+s1}{\PYGZsq{}}\PYG{l+s+s1}{Announcement Return (}\PYG{l+s+s1}{\PYGZpc{}}\PYG{l+s+s1}{)}\PYG{l+s+s1}{\PYGZsq{}}\PYG{p}{)}

\PYG{n}{\PYGZus{}} \PYG{o}{=} \PYG{l+s+s1}{\PYGZsq{}}\PYG{l+s+s1}{ }\PYG{l+s+s1}{\PYGZsq{}}\PYG{o}{.}\PYG{n}{join}\PYG{p}{(}\PYG{n}{surprises}\PYG{o}{.}\PYG{n}{index}\PYG{o}{.}\PYG{n}{get\PYGZus{}level\PYGZus{}values}\PYG{p}{(}\PYG{l+s+s1}{\PYGZsq{}}\PYG{l+s+s1}{Ticker}\PYG{l+s+s1}{\PYGZsq{}}\PYG{p}{)}\PYG{o}{.}\PYG{n}{unique}\PYG{p}{(}\PYG{p}{)}\PYG{p}{)}
\PYG{n}{\PYGZus{}\PYGZus{}} \PYG{o}{=} \PYG{n}{surprises}\PYG{o}{.}\PYG{n}{index}\PYG{o}{.}\PYG{n}{get\PYGZus{}level\PYGZus{}values}\PYG{p}{(}\PYG{l+s+s1}{\PYGZsq{}}\PYG{l+s+s1}{Date}\PYG{l+s+s1}{\PYGZsq{}}\PYG{p}{)}
\PYG{n}{plt}\PYG{o}{.}\PYG{n}{title}\PYG{p}{(}\PYG{l+s+sa}{f}\PYG{l+s+s1}{\PYGZsq{}}\PYG{l+s+s1}{Earnings Announcements}\PYG{l+s+se}{\PYGZbs{}n}\PYG{l+s+s1}{ for }\PYG{l+s+si}{\PYGZob{}}\PYG{n}{\PYGZus{}}\PYG{l+s+si}{\PYGZcb{}}\PYG{l+s+se}{\PYGZbs{}n}\PYG{l+s+s1}{ from }\PYG{l+s+si}{\PYGZob{}}\PYG{n}{\PYGZus{}\PYGZus{}}\PYG{o}{.}\PYG{n}{min}\PYG{p}{(}\PYG{p}{)}\PYG{l+s+si}{:}\PYG{l+s+s1}{\PYGZpc{}B \PYGZpc{}Y}\PYG{l+s+si}{\PYGZcb{}}\PYG{l+s+s1}{ to }\PYG{l+s+si}{\PYGZob{}}\PYG{n}{\PYGZus{}\PYGZus{}}\PYG{o}{.}\PYG{n}{max}\PYG{p}{(}\PYG{p}{)}\PYG{l+s+si}{:}\PYG{l+s+s1}{\PYGZpc{}B \PYGZpc{}Y}\PYG{l+s+si}{\PYGZcb{}}\PYG{l+s+s1}{\PYGZsq{}}\PYG{p}{)}
\PYG{n}{plt}\PYG{o}{.}\PYG{n}{show}\PYG{p}{(}\PYG{p}{)}
\end{sphinxVerbatim}

\end{sphinxuseclass}\end{sphinxVerbatimInput}
\begin{sphinxVerbatimOutput}

\begin{sphinxuseclass}{cell_output}
\noindent\sphinxincludegraphics{{bf26a3081f2f3ae9aa5ac6ad242dafad282378b82b6c0b8634107b7b5faa580a}.png}

\end{sphinxuseclass}\end{sphinxVerbatimOutput}

\end{sphinxuseclass}
\sphinxAtStartPar
\sphinxstyleemphasis{\sphinxstylestrong{There is a positive relation between announcment returns and earnings surprises!}}
Of course, to say more we need more data and to control for market movements, but this analaysis is a start!


\subsubsection{Repeat the earnings exercise with the S\&P 100 stocks}
\label{\detokenize{mckinney_08_practice_03:repeat-the-earnings-exercise-with-the-s-p-100-stocks}}
\begin{sphinxuseclass}{cell}\begin{sphinxVerbatimInput}

\begin{sphinxuseclass}{cell_input}
\begin{sphinxVerbatim}[commandchars=\\\{\}]
\PYG{n}{wiki} \PYG{o}{=} \PYG{n}{pd}\PYG{o}{.}\PYG{n}{read\PYGZus{}html}\PYG{p}{(}\PYG{l+s+s1}{\PYGZsq{}}\PYG{l+s+s1}{https://en.wikipedia.org/wiki/S}\PYG{l+s+s1}{\PYGZpc{}}\PYG{l+s+s1}{26P\PYGZus{}100}\PYG{l+s+s1}{\PYGZsq{}}\PYG{p}{)}
\end{sphinxVerbatim}

\end{sphinxuseclass}\end{sphinxVerbatimInput}

\end{sphinxuseclass}
\begin{sphinxuseclass}{cell}\begin{sphinxVerbatimInput}

\begin{sphinxuseclass}{cell_input}
\begin{sphinxVerbatim}[commandchars=\\\{\}]
\PYG{n}{symbols} \PYG{o}{=} \PYG{n}{wiki}\PYG{p}{[}\PYG{l+m+mi}{2}\PYG{p}{]}\PYG{p}{[}\PYG{l+s+s1}{\PYGZsq{}}\PYG{l+s+s1}{Symbol}\PYG{l+s+s1}{\PYGZsq{}}\PYG{p}{]}\PYG{o}{.}\PYG{n}{str}\PYG{o}{.}\PYG{n}{replace}\PYG{p}{(}\PYG{l+s+s1}{\PYGZsq{}}\PYG{l+s+s1}{.}\PYG{l+s+s1}{\PYGZsq{}}\PYG{p}{,} \PYG{l+s+s1}{\PYGZsq{}}\PYG{l+s+s1}{\PYGZhy{}}\PYG{l+s+s1}{\PYGZsq{}}\PYG{p}{,} \PYG{n}{regex}\PYG{o}{=}\PYG{k+kc}{False}\PYG{p}{)}\PYG{o}{.}\PYG{n}{to\PYGZus{}list}\PYG{p}{(}\PYG{p}{)}
\end{sphinxVerbatim}

\end{sphinxuseclass}\end{sphinxVerbatimInput}

\end{sphinxuseclass}
\begin{sphinxuseclass}{cell}\begin{sphinxVerbatimInput}

\begin{sphinxuseclass}{cell_input}
\begin{sphinxVerbatim}[commandchars=\\\{\}]
\PYG{n}{tickers\PYGZus{}2} \PYG{o}{=} \PYG{n}{yf}\PYG{o}{.}\PYG{n}{Tickers}\PYG{p}{(}\PYG{n}{tickers}\PYG{o}{=}\PYG{n}{symbols}\PYG{p}{,} \PYG{n}{session}\PYG{o}{=}\PYG{n}{session}\PYG{p}{)}
\end{sphinxVerbatim}

\end{sphinxuseclass}\end{sphinxVerbatimInput}

\end{sphinxuseclass}
\begin{sphinxuseclass}{cell}\begin{sphinxVerbatimInput}

\begin{sphinxuseclass}{cell_input}
\begin{sphinxVerbatim}[commandchars=\\\{\}]
\PYG{n}{returns\PYGZus{}2} \PYG{o}{=} \PYG{p}{(}
    \PYG{n}{tickers\PYGZus{}2}
    \PYG{o}{.}\PYG{n}{history}\PYG{p}{(}\PYG{n}{period}\PYG{o}{=}\PYG{l+s+s1}{\PYGZsq{}}\PYG{l+s+s1}{max}\PYG{l+s+s1}{\PYGZsq{}}\PYG{p}{,} \PYG{n}{auto\PYGZus{}adjust}\PYG{o}{=}\PYG{k+kc}{False}\PYG{p}{,} \PYG{n}{progress}\PYG{o}{=}\PYG{k+kc}{False}\PYG{p}{)}
    \PYG{o}{.}\PYG{n}{rename\PYGZus{}axis}\PYG{p}{(}\PYG{n}{columns}\PYG{o}{=}\PYG{p}{[}\PYG{l+s+s1}{\PYGZsq{}}\PYG{l+s+s1}{Variable}\PYG{l+s+s1}{\PYGZsq{}}\PYG{p}{,} \PYG{l+s+s1}{\PYGZsq{}}\PYG{l+s+s1}{Ticker}\PYG{l+s+s1}{\PYGZsq{}}\PYG{p}{]}\PYG{p}{)}
    \PYG{p}{[}\PYG{l+s+s1}{\PYGZsq{}}\PYG{l+s+s1}{Adj Close}\PYG{l+s+s1}{\PYGZsq{}}\PYG{p}{]}
    \PYG{o}{.}\PYG{n}{pct\PYGZus{}change}\PYG{p}{(}\PYG{p}{)}
    \PYG{o}{.}\PYG{n}{assign}\PYG{p}{(}\PYG{n}{Date}\PYG{o}{=}\PYG{k}{lambda} \PYG{n}{x}\PYG{p}{:} \PYG{n}{x}\PYG{o}{.}\PYG{n}{index}\PYG{o}{.}\PYG{n}{tz\PYGZus{}localize}\PYG{p}{(}\PYG{l+s+s1}{\PYGZsq{}}\PYG{l+s+s1}{America/New\PYGZus{}York}\PYG{l+s+s1}{\PYGZsq{}}\PYG{p}{)} \PYG{o}{+} \PYG{n}{pd}\PYG{o}{.}\PYG{n}{to\PYGZus{}timedelta}\PYG{p}{(}\PYG{l+m+mi}{16}\PYG{p}{,} \PYG{n}{unit}\PYG{o}{=}\PYG{l+s+s1}{\PYGZsq{}}\PYG{l+s+s1}{H}\PYG{l+s+s1}{\PYGZsq{}}\PYG{p}{)}\PYG{p}{)}
    \PYG{o}{.}\PYG{n}{set\PYGZus{}index}\PYG{p}{(}\PYG{l+s+s1}{\PYGZsq{}}\PYG{l+s+s1}{Date}\PYG{l+s+s1}{\PYGZsq{}}\PYG{p}{)}
\PYG{p}{)}

\PYG{n}{returns\PYGZus{}2}\PYG{o}{.}\PYG{n}{head}\PYG{p}{(}\PYG{p}{)}
\end{sphinxVerbatim}

\end{sphinxuseclass}\end{sphinxVerbatimInput}
\begin{sphinxVerbatimOutput}

\begin{sphinxuseclass}{cell_output}
\begin{sphinxVerbatim}[commandchars=\\\{\}]
Ticker                     AAPL  ABBV  ABT  ACN  ADBE  AIG  AMD  AMGN  AMT  \PYGZbs{}
Date                                                                         
1962\PYGZhy{}01\PYGZhy{}02 16:00:00\PYGZhy{}05:00   NaN   NaN  NaN  NaN   NaN  NaN  NaN   NaN  NaN   
1962\PYGZhy{}01\PYGZhy{}03 16:00:00\PYGZhy{}05:00   NaN   NaN  NaN  NaN   NaN  NaN  NaN   NaN  NaN   
1962\PYGZhy{}01\PYGZhy{}04 16:00:00\PYGZhy{}05:00   NaN   NaN  NaN  NaN   NaN  NaN  NaN   NaN  NaN   
1962\PYGZhy{}01\PYGZhy{}05 16:00:00\PYGZhy{}05:00   NaN   NaN  NaN  NaN   NaN  NaN  NaN   NaN  NaN   
1962\PYGZhy{}01\PYGZhy{}08 16:00:00\PYGZhy{}05:00   NaN   NaN  NaN  NaN   NaN  NaN  NaN   NaN  NaN   

Ticker                     AMZN  ...  UNH  UNP  UPS  USB   V  VZ  WBA  WFC  \PYGZbs{}
Date                             ...                                         
1962\PYGZhy{}01\PYGZhy{}02 16:00:00\PYGZhy{}05:00   NaN  ...  NaN  NaN  NaN  NaN NaN NaN  NaN  NaN   
1962\PYGZhy{}01\PYGZhy{}03 16:00:00\PYGZhy{}05:00   NaN  ...  NaN  NaN  NaN  NaN NaN NaN  NaN  NaN   
1962\PYGZhy{}01\PYGZhy{}04 16:00:00\PYGZhy{}05:00   NaN  ...  NaN  NaN  NaN  NaN NaN NaN  NaN  NaN   
1962\PYGZhy{}01\PYGZhy{}05 16:00:00\PYGZhy{}05:00   NaN  ...  NaN  NaN  NaN  NaN NaN NaN  NaN  NaN   
1962\PYGZhy{}01\PYGZhy{}08 16:00:00\PYGZhy{}05:00   NaN  ...  NaN  NaN  NaN  NaN NaN NaN  NaN  NaN   

Ticker                     WMT     XOM  
Date                                    
1962\PYGZhy{}01\PYGZhy{}02 16:00:00\PYGZhy{}05:00  NaN     NaN  
1962\PYGZhy{}01\PYGZhy{}03 16:00:00\PYGZhy{}05:00  NaN  0.0149  
1962\PYGZhy{}01\PYGZhy{}04 16:00:00\PYGZhy{}05:00  NaN  0.0024  
1962\PYGZhy{}01\PYGZhy{}05 16:00:00\PYGZhy{}05:00  NaN \PYGZhy{}0.0219  
1962\PYGZhy{}01\PYGZhy{}08 16:00:00\PYGZhy{}05:00  NaN \PYGZhy{}0.0025  

[5 rows x 101 columns]
\end{sphinxVerbatim}

\end{sphinxuseclass}\end{sphinxVerbatimOutput}

\end{sphinxuseclass}
\begin{sphinxuseclass}{cell}\begin{sphinxVerbatimInput}

\begin{sphinxuseclass}{cell_input}
\begin{sphinxVerbatim}[commandchars=\\\{\}]
\PYG{n}{earnings\PYGZus{}2} \PYG{o}{=} \PYG{p}{(}
    \PYG{n}{pd}\PYG{o}{.}\PYG{n}{concat}\PYG{p}{(}
        \PYG{n}{objs}\PYG{o}{=}\PYG{p}{[}\PYG{n}{tickers\PYGZus{}2}\PYG{o}{.}\PYG{n}{tickers}\PYG{p}{[}\PYG{n}{t}\PYG{p}{]}\PYG{o}{.}\PYG{n}{earnings\PYGZus{}dates} \PYG{k}{for} \PYG{n}{t} \PYG{o+ow}{in} \PYG{n}{tickers\PYGZus{}2}\PYG{o}{.}\PYG{n}{tickers}\PYG{p}{]}\PYG{p}{,}
        \PYG{n}{keys}\PYG{o}{=}\PYG{n}{tickers\PYGZus{}2}\PYG{o}{.}\PYG{n}{tickers}\PYG{p}{,}
        \PYG{n}{names}\PYG{o}{=}\PYG{p}{[}\PYG{l+s+s1}{\PYGZsq{}}\PYG{l+s+s1}{Ticker}\PYG{l+s+s1}{\PYGZsq{}}\PYG{p}{,} \PYG{l+s+s1}{\PYGZsq{}}\PYG{l+s+s1}{Date}\PYG{l+s+s1}{\PYGZsq{}}\PYG{p}{]}
    \PYG{p}{)}
    \PYG{o}{.}\PYG{n}{rename\PYGZus{}axis}\PYG{p}{(}\PYG{n}{columns}\PYG{o}{=}\PYG{l+s+s1}{\PYGZsq{}}\PYG{l+s+s1}{Variable}\PYG{l+s+s1}{\PYGZsq{}}\PYG{p}{)}
\PYG{p}{)}
\end{sphinxVerbatim}

\end{sphinxuseclass}\end{sphinxVerbatimInput}

\end{sphinxuseclass}
\begin{sphinxuseclass}{cell}\begin{sphinxVerbatimInput}

\begin{sphinxuseclass}{cell_input}
\begin{sphinxVerbatim}[commandchars=\\\{\}]
\PYG{n}{surprises\PYGZus{}2} \PYG{o}{=} \PYG{p}{(}
    \PYG{n}{pd}\PYG{o}{.}\PYG{n}{merge\PYGZus{}asof}\PYG{p}{(}
        \PYG{n}{left}\PYG{o}{=}\PYG{n}{earnings\PYGZus{}2}\PYG{o}{.}\PYG{n}{sort\PYGZus{}index}\PYG{p}{(}\PYG{n}{level}\PYG{o}{=}\PYG{p}{[}\PYG{l+s+s1}{\PYGZsq{}}\PYG{l+s+s1}{Date}\PYG{l+s+s1}{\PYGZsq{}}\PYG{p}{,} \PYG{l+s+s1}{\PYGZsq{}}\PYG{l+s+s1}{Ticker}\PYG{l+s+s1}{\PYGZsq{}}\PYG{p}{]}\PYG{p}{)}\PYG{p}{,}
        \PYG{n}{right}\PYG{o}{=}\PYG{n}{returns\PYGZus{}2}\PYG{o}{.}\PYG{n}{stack}\PYG{p}{(}\PYG{p}{)}\PYG{o}{.}\PYG{n}{to\PYGZus{}frame}\PYG{p}{(}\PYG{l+s+s1}{\PYGZsq{}}\PYG{l+s+s1}{Returns}\PYG{l+s+s1}{\PYGZsq{}}\PYG{p}{)}\PYG{o}{.}\PYG{n}{swaplevel}\PYG{p}{(}\PYG{p}{)}\PYG{o}{.}\PYG{n}{sort\PYGZus{}index}\PYG{p}{(}\PYG{n}{level}\PYG{o}{=}\PYG{p}{[}\PYG{l+s+s1}{\PYGZsq{}}\PYG{l+s+s1}{Date}\PYG{l+s+s1}{\PYGZsq{}}\PYG{p}{,} \PYG{l+s+s1}{\PYGZsq{}}\PYG{l+s+s1}{Ticker}\PYG{l+s+s1}{\PYGZsq{}}\PYG{p}{]}\PYG{p}{)}\PYG{p}{,}
        \PYG{n}{on}\PYG{o}{=}\PYG{l+s+s1}{\PYGZsq{}}\PYG{l+s+s1}{Date}\PYG{l+s+s1}{\PYGZsq{}}\PYG{p}{,}
        \PYG{n}{by}\PYG{o}{=}\PYG{l+s+s1}{\PYGZsq{}}\PYG{l+s+s1}{Ticker}\PYG{l+s+s1}{\PYGZsq{}}\PYG{p}{,}
        \PYG{n}{direction}\PYG{o}{=}\PYG{l+s+s1}{\PYGZsq{}}\PYG{l+s+s1}{forward}\PYG{l+s+s1}{\PYGZsq{}}\PYG{p}{,}
        \PYG{n}{allow\PYGZus{}exact\PYGZus{}matches}\PYG{o}{=}\PYG{k+kc}{False}
    \PYG{p}{)}
    \PYG{o}{.}\PYG{n}{dropna}\PYG{p}{(}\PYG{p}{)}
    \PYG{o}{.}\PYG{n}{set\PYGZus{}index}\PYG{p}{(}\PYG{p}{[}\PYG{l+s+s1}{\PYGZsq{}}\PYG{l+s+s1}{Date}\PYG{l+s+s1}{\PYGZsq{}}\PYG{p}{,} \PYG{l+s+s1}{\PYGZsq{}}\PYG{l+s+s1}{Ticker}\PYG{l+s+s1}{\PYGZsq{}}\PYG{p}{]}\PYG{p}{)}
\PYG{p}{)}
\end{sphinxVerbatim}

\end{sphinxuseclass}\end{sphinxVerbatimInput}

\end{sphinxuseclass}
\begin{sphinxuseclass}{cell}\begin{sphinxVerbatimInput}

\begin{sphinxuseclass}{cell_input}
\begin{sphinxVerbatim}[commandchars=\\\{\}]
\PYG{n}{sns}\PYG{o}{.}\PYG{n}{barplot}\PYG{p}{(}
    \PYG{n}{x}\PYG{o}{=}\PYG{l+s+s1}{\PYGZsq{}}\PYG{l+s+s1}{ESQ}\PYG{l+s+s1}{\PYGZsq{}}\PYG{p}{,}
    \PYG{n}{y}\PYG{o}{=} \PYG{l+s+s1}{\PYGZsq{}}\PYG{l+s+s1}{Returns}\PYG{l+s+s1}{\PYGZsq{}}\PYG{p}{,}
    \PYG{n}{data}\PYG{o}{=}\PYG{p}{(}
        \PYG{n}{surprises\PYGZus{}2}
        \PYG{p}{[}\PYG{p}{[}\PYG{l+s+s1}{\PYGZsq{}}\PYG{l+s+s1}{Surprise(}\PYG{l+s+s1}{\PYGZpc{}}\PYG{l+s+s1}{)}\PYG{l+s+s1}{\PYGZsq{}}\PYG{p}{,} \PYG{l+s+s1}{\PYGZsq{}}\PYG{l+s+s1}{Returns}\PYG{l+s+s1}{\PYGZsq{}}\PYG{p}{]}\PYG{p}{]}
        \PYG{o}{.}\PYG{n}{mul}\PYG{p}{(}\PYG{l+m+mi}{100}\PYG{p}{)}
        \PYG{o}{.}\PYG{n}{assign}\PYG{p}{(}\PYG{n}{ESQ} \PYG{o}{=} \PYG{k}{lambda} \PYG{n}{x}\PYG{p}{:} \PYG{n}{pd}\PYG{o}{.}\PYG{n}{qcut}\PYG{p}{(}\PYG{n}{x}\PYG{o}{=}\PYG{n}{x}\PYG{p}{[}\PYG{l+s+s1}{\PYGZsq{}}\PYG{l+s+s1}{Surprise(}\PYG{l+s+s1}{\PYGZpc{}}\PYG{l+s+s1}{)}\PYG{l+s+s1}{\PYGZsq{}}\PYG{p}{]}\PYG{p}{,} \PYG{n}{q}\PYG{o}{=}\PYG{l+m+mi}{5}\PYG{p}{,} \PYG{n}{labels}\PYG{o}{=}\PYG{k+kc}{False}\PYG{p}{)}\PYG{p}{)}
    \PYG{p}{)}
\PYG{p}{)}

\PYG{n}{plt}\PYG{o}{.}\PYG{n}{xlabel}\PYG{p}{(}\PYG{l+s+s1}{\PYGZsq{}}\PYG{l+s+s1}{Earnings Suprise Portfolio}\PYG{l+s+s1}{\PYGZsq{}}\PYG{p}{)}
\PYG{n}{plt}\PYG{o}{.}\PYG{n}{ylabel}\PYG{p}{(}\PYG{l+s+s1}{\PYGZsq{}}\PYG{l+s+s1}{Announcement Return (}\PYG{l+s+s1}{\PYGZpc{}}\PYG{l+s+s1}{)}\PYG{l+s+s1}{\PYGZsq{}}\PYG{p}{)}

\PYG{n}{\PYGZus{}\PYGZus{}} \PYG{o}{=} \PYG{n}{surprises\PYGZus{}2}\PYG{o}{.}\PYG{n}{index}\PYG{o}{.}\PYG{n}{get\PYGZus{}level\PYGZus{}values}\PYG{p}{(}\PYG{l+s+s1}{\PYGZsq{}}\PYG{l+s+s1}{Date}\PYG{l+s+s1}{\PYGZsq{}}\PYG{p}{)}
\PYG{n}{plt}\PYG{o}{.}\PYG{n}{title}\PYG{p}{(}\PYG{l+s+sa}{f}\PYG{l+s+s1}{\PYGZsq{}}\PYG{l+s+s1}{Earnings Announcements for S\PYGZam{}P 100 Stocks }\PYG{l+s+se}{\PYGZbs{}n}\PYG{l+s+s1}{ from }\PYG{l+s+si}{\PYGZob{}}\PYG{n}{\PYGZus{}\PYGZus{}}\PYG{o}{.}\PYG{n}{min}\PYG{p}{(}\PYG{p}{)}\PYG{l+s+si}{:}\PYG{l+s+s1}{\PYGZpc{}B \PYGZpc{}Y}\PYG{l+s+si}{\PYGZcb{}}\PYG{l+s+s1}{ to }\PYG{l+s+si}{\PYGZob{}}\PYG{n}{\PYGZus{}\PYGZus{}}\PYG{o}{.}\PYG{n}{max}\PYG{p}{(}\PYG{p}{)}\PYG{l+s+si}{:}\PYG{l+s+s1}{\PYGZpc{}B \PYGZpc{}Y}\PYG{l+s+si}{\PYGZcb{}}\PYG{l+s+s1}{\PYGZsq{}}\PYG{p}{)}
\PYG{n}{plt}\PYG{o}{.}\PYG{n}{show}\PYG{p}{(}\PYG{p}{)}
\end{sphinxVerbatim}

\end{sphinxuseclass}\end{sphinxVerbatimInput}
\begin{sphinxVerbatimOutput}

\begin{sphinxuseclass}{cell_output}
\noindent\sphinxincludegraphics{{1e3e4e06c9f6bb45e17f2536aa0d95244c5e54f6917451b7174c8149d6b3dd57}.png}

\end{sphinxuseclass}\end{sphinxVerbatimOutput}

\end{sphinxuseclass}

\subsubsection{Repeat the earnings exercise with \sphinxstyleemphasis{excess returns} of the S\&P 100 Stocks}
\label{\detokenize{mckinney_08_practice_03:repeat-the-earnings-exercise-with-excess-returns-of-the-s-p-100-stocks}}
\sphinxAtStartPar
Excess returns are returns minus market returns.
We need to add a timezone and the closing time to the market return from Fama and French.

\begin{sphinxuseclass}{cell}\begin{sphinxVerbatimInput}

\begin{sphinxuseclass}{cell_input}
\begin{sphinxVerbatim}[commandchars=\\\{\}]
\PYG{n}{Mkt} \PYG{o}{=} \PYG{n}{ff}\PYG{p}{[}\PYG{l+s+s1}{\PYGZsq{}}\PYG{l+s+s1}{Mkt\PYGZhy{}RF}\PYG{l+s+s1}{\PYGZsq{}}\PYG{p}{]}\PYG{o}{.}\PYG{n}{add}\PYG{p}{(}\PYG{n}{ff}\PYG{p}{[}\PYG{l+s+s1}{\PYGZsq{}}\PYG{l+s+s1}{RF}\PYG{l+s+s1}{\PYGZsq{}}\PYG{p}{]}\PYG{p}{)}
\PYG{n}{Mkt}\PYG{o}{.}\PYG{n}{index} \PYG{o}{=} \PYG{n}{Mkt}\PYG{o}{.}\PYG{n}{index}\PYG{o}{.}\PYG{n}{tz\PYGZus{}localize}\PYG{p}{(}\PYG{l+s+s1}{\PYGZsq{}}\PYG{l+s+s1}{America/New\PYGZus{}York}\PYG{l+s+s1}{\PYGZsq{}}\PYG{p}{)} \PYG{o}{+} \PYG{n}{pd}\PYG{o}{.}\PYG{n}{to\PYGZus{}timedelta}\PYG{p}{(}\PYG{l+m+mi}{16}\PYG{p}{,} \PYG{n}{unit}\PYG{o}{=}\PYG{l+s+s1}{\PYGZsq{}}\PYG{l+s+s1}{H}\PYG{l+s+s1}{\PYGZsq{}}\PYG{p}{)}
\PYG{n}{returns\PYGZus{}3} \PYG{o}{=} \PYG{n}{returns\PYGZus{}2}\PYG{o}{.}\PYG{n}{sub}\PYG{p}{(}\PYG{n}{Mkt}\PYG{p}{,} \PYG{n}{axis}\PYG{o}{=}\PYG{l+m+mi}{0}\PYG{p}{)}
\end{sphinxVerbatim}

\end{sphinxuseclass}\end{sphinxVerbatimInput}

\end{sphinxuseclass}
\begin{sphinxuseclass}{cell}\begin{sphinxVerbatimInput}

\begin{sphinxuseclass}{cell_input}
\begin{sphinxVerbatim}[commandchars=\\\{\}]
\PYG{n}{surprises\PYGZus{}3} \PYG{o}{=} \PYG{p}{(}
    \PYG{n}{pd}\PYG{o}{.}\PYG{n}{merge\PYGZus{}asof}\PYG{p}{(}
        \PYG{n}{left}\PYG{o}{=}\PYG{n}{earnings\PYGZus{}2}\PYG{o}{.}\PYG{n}{sort\PYGZus{}index}\PYG{p}{(}\PYG{n}{level}\PYG{o}{=}\PYG{p}{[}\PYG{l+s+s1}{\PYGZsq{}}\PYG{l+s+s1}{Date}\PYG{l+s+s1}{\PYGZsq{}}\PYG{p}{,} \PYG{l+s+s1}{\PYGZsq{}}\PYG{l+s+s1}{Ticker}\PYG{l+s+s1}{\PYGZsq{}}\PYG{p}{]}\PYG{p}{)}\PYG{p}{,}
        \PYG{n}{right}\PYG{o}{=}\PYG{n}{returns\PYGZus{}3}\PYG{o}{.}\PYG{n}{stack}\PYG{p}{(}\PYG{p}{)}\PYG{o}{.}\PYG{n}{to\PYGZus{}frame}\PYG{p}{(}\PYG{l+s+s1}{\PYGZsq{}}\PYG{l+s+s1}{Excess Returns}\PYG{l+s+s1}{\PYGZsq{}}\PYG{p}{)}\PYG{o}{.}\PYG{n}{swaplevel}\PYG{p}{(}\PYG{p}{)}\PYG{o}{.}\PYG{n}{sort\PYGZus{}index}\PYG{p}{(}\PYG{n}{level}\PYG{o}{=}\PYG{p}{[}\PYG{l+s+s1}{\PYGZsq{}}\PYG{l+s+s1}{Date}\PYG{l+s+s1}{\PYGZsq{}}\PYG{p}{,} \PYG{l+s+s1}{\PYGZsq{}}\PYG{l+s+s1}{Ticker}\PYG{l+s+s1}{\PYGZsq{}}\PYG{p}{]}\PYG{p}{)}\PYG{p}{,}
        \PYG{n}{on}\PYG{o}{=}\PYG{l+s+s1}{\PYGZsq{}}\PYG{l+s+s1}{Date}\PYG{l+s+s1}{\PYGZsq{}}\PYG{p}{,}
        \PYG{n}{by}\PYG{o}{=}\PYG{l+s+s1}{\PYGZsq{}}\PYG{l+s+s1}{Ticker}\PYG{l+s+s1}{\PYGZsq{}}\PYG{p}{,}
        \PYG{n}{direction}\PYG{o}{=}\PYG{l+s+s1}{\PYGZsq{}}\PYG{l+s+s1}{forward}\PYG{l+s+s1}{\PYGZsq{}}\PYG{p}{,}
        \PYG{n}{allow\PYGZus{}exact\PYGZus{}matches}\PYG{o}{=}\PYG{k+kc}{False}
    \PYG{p}{)}
    \PYG{o}{.}\PYG{n}{dropna}\PYG{p}{(}\PYG{p}{)}
    \PYG{o}{.}\PYG{n}{set\PYGZus{}index}\PYG{p}{(}\PYG{p}{[}\PYG{l+s+s1}{\PYGZsq{}}\PYG{l+s+s1}{Date}\PYG{l+s+s1}{\PYGZsq{}}\PYG{p}{,} \PYG{l+s+s1}{\PYGZsq{}}\PYG{l+s+s1}{Ticker}\PYG{l+s+s1}{\PYGZsq{}}\PYG{p}{]}\PYG{p}{)}
\PYG{p}{)}
\end{sphinxVerbatim}

\end{sphinxuseclass}\end{sphinxVerbatimInput}

\end{sphinxuseclass}
\begin{sphinxuseclass}{cell}\begin{sphinxVerbatimInput}

\begin{sphinxuseclass}{cell_input}
\begin{sphinxVerbatim}[commandchars=\\\{\}]
\PYG{n}{sns}\PYG{o}{.}\PYG{n}{barplot}\PYG{p}{(}
    \PYG{n}{x}\PYG{o}{=}\PYG{l+s+s1}{\PYGZsq{}}\PYG{l+s+s1}{ESQ}\PYG{l+s+s1}{\PYGZsq{}}\PYG{p}{,}
    \PYG{n}{y}\PYG{o}{=}\PYG{l+s+s1}{\PYGZsq{}}\PYG{l+s+s1}{Excess Returns}\PYG{l+s+s1}{\PYGZsq{}}\PYG{p}{,}
    \PYG{n}{data}\PYG{o}{=}\PYG{p}{(}
        \PYG{n}{surprises\PYGZus{}3}
        \PYG{p}{[}\PYG{p}{[}\PYG{l+s+s1}{\PYGZsq{}}\PYG{l+s+s1}{Surprise(}\PYG{l+s+s1}{\PYGZpc{}}\PYG{l+s+s1}{)}\PYG{l+s+s1}{\PYGZsq{}}\PYG{p}{,} \PYG{l+s+s1}{\PYGZsq{}}\PYG{l+s+s1}{Excess Returns}\PYG{l+s+s1}{\PYGZsq{}}\PYG{p}{]}\PYG{p}{]}
        \PYG{o}{.}\PYG{n}{mul}\PYG{p}{(}\PYG{l+m+mi}{100}\PYG{p}{)}
        \PYG{o}{.}\PYG{n}{assign}\PYG{p}{(}\PYG{n}{ESQ} \PYG{o}{=} \PYG{k}{lambda} \PYG{n}{x}\PYG{p}{:} \PYG{n}{pd}\PYG{o}{.}\PYG{n}{qcut}\PYG{p}{(}\PYG{n}{x}\PYG{o}{=}\PYG{n}{x}\PYG{p}{[}\PYG{l+s+s1}{\PYGZsq{}}\PYG{l+s+s1}{Surprise(}\PYG{l+s+s1}{\PYGZpc{}}\PYG{l+s+s1}{)}\PYG{l+s+s1}{\PYGZsq{}}\PYG{p}{]}\PYG{p}{,} \PYG{n}{q}\PYG{o}{=}\PYG{l+m+mi}{5}\PYG{p}{,} \PYG{n}{labels}\PYG{o}{=}\PYG{k+kc}{False}\PYG{p}{)}\PYG{p}{)}
    \PYG{p}{)}
\PYG{p}{)}

\PYG{n}{plt}\PYG{o}{.}\PYG{n}{xlabel}\PYG{p}{(}\PYG{l+s+s1}{\PYGZsq{}}\PYG{l+s+s1}{Earnings Suprise Portfolio}\PYG{l+s+s1}{\PYGZsq{}}\PYG{p}{)}
\PYG{n}{plt}\PYG{o}{.}\PYG{n}{ylabel}\PYG{p}{(}\PYG{l+s+s1}{\PYGZsq{}}\PYG{l+s+s1}{Announcement Excess Return (}\PYG{l+s+s1}{\PYGZpc{}}\PYG{l+s+s1}{)}\PYG{l+s+s1}{\PYGZsq{}}\PYG{p}{)}

\PYG{n}{\PYGZus{}\PYGZus{}} \PYG{o}{=} \PYG{n}{surprises\PYGZus{}3}\PYG{o}{.}\PYG{n}{index}\PYG{o}{.}\PYG{n}{get\PYGZus{}level\PYGZus{}values}\PYG{p}{(}\PYG{l+s+s1}{\PYGZsq{}}\PYG{l+s+s1}{Date}\PYG{l+s+s1}{\PYGZsq{}}\PYG{p}{)}
\PYG{n}{plt}\PYG{o}{.}\PYG{n}{title}\PYG{p}{(}\PYG{l+s+sa}{f}\PYG{l+s+s1}{\PYGZsq{}}\PYG{l+s+s1}{Earnings Announcements for S\PYGZam{}P 100 Stocks}\PYG{l+s+se}{\PYGZbs{}n}\PYG{l+s+s1}{ from }\PYG{l+s+si}{\PYGZob{}}\PYG{n}{\PYGZus{}\PYGZus{}}\PYG{o}{.}\PYG{n}{min}\PYG{p}{(}\PYG{p}{)}\PYG{l+s+si}{:}\PYG{l+s+s1}{\PYGZpc{}B \PYGZpc{}Y}\PYG{l+s+si}{\PYGZcb{}}\PYG{l+s+s1}{ to }\PYG{l+s+si}{\PYGZob{}}\PYG{n}{\PYGZus{}\PYGZus{}}\PYG{o}{.}\PYG{n}{max}\PYG{p}{(}\PYG{p}{)}\PYG{l+s+si}{:}\PYG{l+s+s1}{\PYGZpc{}B \PYGZpc{}Y}\PYG{l+s+si}{\PYGZcb{}}\PYG{l+s+s1}{\PYGZsq{}}\PYG{p}{)}
\PYG{n}{plt}\PYG{o}{.}\PYG{n}{show}\PYG{p}{(}\PYG{p}{)}
\end{sphinxVerbatim}

\end{sphinxuseclass}\end{sphinxVerbatimInput}
\begin{sphinxVerbatimOutput}

\begin{sphinxuseclass}{cell_output}
\noindent\sphinxincludegraphics{{8fe7732e2c96cf3b18b9c8213a60e89cbd79a5d8630f6e0f5fd3ee112055621a}.png}

\end{sphinxuseclass}\end{sphinxVerbatimOutput}

\end{sphinxuseclass}

\subsubsection{Improve your \sphinxstyleliteralintitle{\sphinxupquote{download()}} function from above}
\label{\detokenize{mckinney_08_practice_03:improve-your-download-function-from-above}}
\sphinxAtStartPar
Modify \sphinxcode{\sphinxupquote{download()}} to accept one or more than one ticker.
Since we will not use the advanced functionality of the tickers object that \sphinxcode{\sphinxupquote{yf.Tickers()}} creates, we will use \sphinxcode{\sphinxupquote{yf.download()}}.
The current version of \sphinxcode{\sphinxupquote{yf.download()}} does not accept a \sphinxcode{\sphinxupquote{session=}} argument.

\begin{sphinxuseclass}{cell}\begin{sphinxVerbatimInput}

\begin{sphinxuseclass}{cell_input}
\begin{sphinxVerbatim}[commandchars=\\\{\}]
\PYG{k}{def} \PYG{n+nf}{download}\PYG{p}{(}\PYG{n}{tickers}\PYG{p}{)}\PYG{p}{:}

    \PYG{n}{histories} \PYG{o}{=} \PYG{p}{(}
        \PYG{n}{yf}\PYG{o}{.}\PYG{n}{download}\PYG{p}{(}\PYG{n}{tickers}\PYG{p}{,} \PYG{n}{progress}\PYG{o}{=}\PYG{k+kc}{False}\PYG{p}{)}
        \PYG{o}{.}\PYG{n}{assign}\PYG{p}{(}\PYG{n}{Date}\PYG{o}{=}\PYG{k}{lambda} \PYG{n}{x}\PYG{p}{:} \PYG{n}{x}\PYG{o}{.}\PYG{n}{index}\PYG{o}{.}\PYG{n}{tz\PYGZus{}localize}\PYG{p}{(}\PYG{k+kc}{None}\PYG{p}{)}\PYG{p}{)}
        \PYG{o}{.}\PYG{n}{set\PYGZus{}index}\PYG{p}{(}\PYG{l+s+s1}{\PYGZsq{}}\PYG{l+s+s1}{Date}\PYG{l+s+s1}{\PYGZsq{}}\PYG{p}{)}
    \PYG{p}{)}

    \PYG{n}{factors} \PYG{o}{=} \PYG{p}{(}
        \PYG{n}{pdr}\PYG{o}{.}\PYG{n}{DataReader}\PYG{p}{(}
            \PYG{n}{name}\PYG{o}{=}\PYG{l+s+s1}{\PYGZsq{}}\PYG{l+s+s1}{F\PYGZhy{}F\PYGZus{}Research\PYGZus{}Data\PYGZus{}Factors\PYGZus{}daily}\PYG{l+s+s1}{\PYGZsq{}}\PYG{p}{,}
            \PYG{n}{data\PYGZus{}source}\PYG{o}{=}\PYG{l+s+s1}{\PYGZsq{}}\PYG{l+s+s1}{famafrench}\PYG{l+s+s1}{\PYGZsq{}}\PYG{p}{,}
            \PYG{n}{start}\PYG{o}{=}\PYG{l+s+s1}{\PYGZsq{}}\PYG{l+s+s1}{1900}\PYG{l+s+s1}{\PYGZsq{}}\PYG{p}{,}
            \PYG{n}{session}\PYG{o}{=}\PYG{n}{session}
        \PYG{p}{)}
        \PYG{p}{[}\PYG{l+m+mi}{0}\PYG{p}{]}
        \PYG{o}{.}\PYG{n}{div}\PYG{p}{(}\PYG{l+m+mi}{100}\PYG{p}{)}
    \PYG{p}{)}

    \PYG{k}{if} \PYG{n+nb}{type}\PYG{p}{(}\PYG{n}{histories}\PYG{o}{.}\PYG{n}{columns}\PYG{p}{)} \PYG{o+ow}{is} \PYG{n}{pd}\PYG{o}{.}\PYG{n}{MultiIndex}\PYG{p}{:}
        \PYG{n}{\PYGZus{}} \PYG{o}{=} \PYG{n}{pd}\PYG{o}{.}\PYG{n}{MultiIndex}\PYG{o}{.}\PYG{n}{from\PYGZus{}product}\PYG{p}{(}\PYG{p}{[}\PYG{p}{[}\PYG{l+s+s1}{\PYGZsq{}}\PYG{l+s+s1}{Returns}\PYG{l+s+s1}{\PYGZsq{}}\PYG{p}{]}\PYG{p}{,} \PYG{n}{histories}\PYG{p}{[}\PYG{l+s+s1}{\PYGZsq{}}\PYG{l+s+s1}{Adj Close}\PYG{l+s+s1}{\PYGZsq{}}\PYG{p}{]}\PYG{o}{.}\PYG{n}{columns}\PYG{p}{]}\PYG{p}{)}
        \PYG{n}{histories}\PYG{p}{[}\PYG{n}{\PYGZus{}}\PYG{p}{]} \PYG{o}{=} \PYG{n}{histories}\PYG{p}{[}\PYG{l+s+s1}{\PYGZsq{}}\PYG{l+s+s1}{Adj Close}\PYG{l+s+s1}{\PYGZsq{}}\PYG{p}{]}\PYG{o}{.}\PYG{n}{pct\PYGZus{}change}\PYG{p}{(}\PYG{p}{)}

        \PYG{n}{\PYGZus{}} \PYG{o}{=} \PYG{n}{pd}\PYG{o}{.}\PYG{n}{MultiIndex}\PYG{o}{.}\PYG{n}{from\PYGZus{}product}\PYG{p}{(}\PYG{p}{[}\PYG{p}{[}\PYG{l+s+s1}{\PYGZsq{}}\PYG{l+s+s1}{Factors}\PYG{l+s+s1}{\PYGZsq{}}\PYG{p}{]}\PYG{p}{,} \PYG{n}{factors}\PYG{o}{.}\PYG{n}{columns}\PYG{p}{]}\PYG{p}{)}
        \PYG{n}{histories}\PYG{p}{[}\PYG{n}{\PYGZus{}}\PYG{p}{]} \PYG{o}{=} \PYG{n}{factors}

        \PYG{k}{return} \PYG{n}{histories}\PYG{o}{.}\PYG{n}{rename\PYGZus{}axis}\PYG{p}{(}\PYG{n}{columns}\PYG{o}{=}\PYG{p}{[}\PYG{l+s+s1}{\PYGZsq{}}\PYG{l+s+s1}{Variable}\PYG{l+s+s1}{\PYGZsq{}}\PYG{p}{,} \PYG{l+s+s1}{\PYGZsq{}}\PYG{l+s+s1}{Ticker}\PYG{l+s+s1}{\PYGZsq{}}\PYG{p}{]}\PYG{p}{)}

    \PYG{k}{elif} \PYG{n+nb}{type}\PYG{p}{(}\PYG{n}{histories}\PYG{o}{.}\PYG{n}{columns}\PYG{p}{)} \PYG{o+ow}{is} \PYG{n}{pd}\PYG{o}{.}\PYG{n}{Index}\PYG{p}{:}
        \PYG{k}{return} \PYG{n}{histories}\PYG{o}{.}\PYG{n}{join}\PYG{p}{(}\PYG{n}{ff}\PYG{p}{)}\PYG{o}{.}\PYG{n}{rename\PYGZus{}axis}\PYG{p}{(}\PYG{n}{columns}\PYG{o}{=}\PYG{p}{[}\PYG{l+s+s1}{\PYGZsq{}}\PYG{l+s+s1}{Variable}\PYG{l+s+s1}{\PYGZsq{}}\PYG{p}{]}\PYG{p}{)}
\end{sphinxVerbatim}

\end{sphinxuseclass}\end{sphinxVerbatimInput}

\end{sphinxuseclass}
\begin{sphinxuseclass}{cell}\begin{sphinxVerbatimInput}

\begin{sphinxuseclass}{cell_input}
\begin{sphinxVerbatim}[commandchars=\\\{\}]
\PYG{n}{download}\PYG{p}{(}\PYG{n}{tickers}\PYG{o}{=}\PYG{l+s+s1}{\PYGZsq{}}\PYG{l+s+s1}{AAPL}\PYG{l+s+s1}{\PYGZsq{}}\PYG{p}{)}\PYG{o}{.}\PYG{n}{head}\PYG{p}{(}\PYG{p}{)}
\end{sphinxVerbatim}

\end{sphinxuseclass}\end{sphinxVerbatimInput}
\begin{sphinxVerbatimOutput}

\begin{sphinxuseclass}{cell_output}
\begin{sphinxVerbatim}[commandchars=\\\{\}]
[*********************100\PYGZpc{}***********************]  1 of 1 completed
\end{sphinxVerbatim}

\begin{sphinxVerbatim}[commandchars=\\\{\}]
Variable     Open   High    Low  Close  Adj Close     Volume  Mkt\PYGZhy{}RF     SMB  \PYGZbs{}
Date                                                                           
1980\PYGZhy{}12\PYGZhy{}12 0.1283 0.1289 0.1283 0.1283     0.0997  469033600  0.0138 \PYGZhy{}0.0001   
1980\PYGZhy{}12\PYGZhy{}15 0.1222 0.1222 0.1217 0.1217     0.0945  175884800  0.0011  0.0025   
1980\PYGZhy{}12\PYGZhy{}16 0.1133 0.1133 0.1127 0.1127     0.0876  105728000  0.0071 \PYGZhy{}0.0075   
1980\PYGZhy{}12\PYGZhy{}17 0.1155 0.1161 0.1155 0.1155     0.0897   86441600  0.0152 \PYGZhy{}0.0086   
1980\PYGZhy{}12\PYGZhy{}18 0.1189 0.1194 0.1189 0.1189     0.0924   73449600  0.0041  0.0022   

Variable       HML     RF  
Date                       
1980\PYGZhy{}12\PYGZhy{}12 \PYGZhy{}0.0105 0.0006  
1980\PYGZhy{}12\PYGZhy{}15 \PYGZhy{}0.0046 0.0006  
1980\PYGZhy{}12\PYGZhy{}16 \PYGZhy{}0.0047 0.0006  
1980\PYGZhy{}12\PYGZhy{}17 \PYGZhy{}0.0034 0.0006  
1980\PYGZhy{}12\PYGZhy{}18  0.0126 0.0006  
\end{sphinxVerbatim}

\end{sphinxuseclass}\end{sphinxVerbatimOutput}

\end{sphinxuseclass}
\begin{sphinxuseclass}{cell}\begin{sphinxVerbatimInput}

\begin{sphinxuseclass}{cell_input}
\begin{sphinxVerbatim}[commandchars=\\\{\}]
\PYG{n}{download}\PYG{p}{(}\PYG{n}{tickers}\PYG{o}{=}\PYG{l+s+s1}{\PYGZsq{}}\PYG{l+s+s1}{AAPL TSLA}\PYG{l+s+s1}{\PYGZsq{}}\PYG{p}{)}\PYG{o}{.}\PYG{n}{head}\PYG{p}{(}\PYG{p}{)}
\end{sphinxVerbatim}

\end{sphinxuseclass}\end{sphinxVerbatimInput}
\begin{sphinxVerbatimOutput}

\begin{sphinxuseclass}{cell_output}
\begin{sphinxVerbatim}[commandchars=\\\{\}]
[*********************100\PYGZpc{}***********************]  2 of 2 completed
\end{sphinxVerbatim}

\begin{sphinxVerbatim}[commandchars=\\\{\}]
Variable   Adj Close       Close        High         Low        Open       \PYGZbs{}
Ticker          AAPL TSLA   AAPL TSLA   AAPL TSLA   AAPL TSLA   AAPL TSLA   
Date                                                                        
1980\PYGZhy{}12\PYGZhy{}12    0.0997  NaN 0.1283  NaN 0.1289  NaN 0.1283  NaN 0.1283  NaN   
1980\PYGZhy{}12\PYGZhy{}15    0.0945  NaN 0.1217  NaN 0.1222  NaN 0.1217  NaN 0.1222  NaN   
1980\PYGZhy{}12\PYGZhy{}16    0.0876  NaN 0.1127  NaN 0.1133  NaN 0.1127  NaN 0.1133  NaN   
1980\PYGZhy{}12\PYGZhy{}17    0.0897  NaN 0.1155  NaN 0.1161  NaN 0.1155  NaN 0.1155  NaN   
1980\PYGZhy{}12\PYGZhy{}18    0.0924  NaN 0.1189  NaN 0.1194  NaN 0.1189  NaN 0.1189  NaN   

Variable       Volume      Returns      Factors                         
Ticker           AAPL TSLA    AAPL TSLA  Mkt\PYGZhy{}RF     SMB     HML     RF  
Date                                                                    
1980\PYGZhy{}12\PYGZhy{}12  469033600  NaN     NaN  NaN  0.0138 \PYGZhy{}0.0001 \PYGZhy{}0.0105 0.0006  
1980\PYGZhy{}12\PYGZhy{}15  175884800  NaN \PYGZhy{}0.0522  NaN  0.0011  0.0025 \PYGZhy{}0.0046 0.0006  
1980\PYGZhy{}12\PYGZhy{}16  105728000  NaN \PYGZhy{}0.0734  NaN  0.0071 \PYGZhy{}0.0075 \PYGZhy{}0.0047 0.0006  
1980\PYGZhy{}12\PYGZhy{}17   86441600  NaN  0.0248  NaN  0.0152 \PYGZhy{}0.0086 \PYGZhy{}0.0034 0.0006  
1980\PYGZhy{}12\PYGZhy{}18   73449600  NaN  0.0290  NaN  0.0041  0.0022  0.0126 0.0006  
\end{sphinxVerbatim}

\end{sphinxuseclass}\end{sphinxVerbatimOutput}

\end{sphinxuseclass}
\sphinxstepscope


\section{McKinney Chapter 8 \sphinxhyphen{} Practice (Wednesday 11:45 AM, Section 4)}
\label{\detokenize{mckinney_08_practice_04:mckinney-chapter-8-practice-wednesday-11-45-am-section-4}}\label{\detokenize{mckinney_08_practice_04::doc}}

\subsection{Announcements}
\label{\detokenize{mckinney_08_practice_04:announcements}}\begin{itemize}
\item {} 
\sphinxAtStartPar
Quiz 3 mean was \$80\%\$

\item {} 
\sphinxAtStartPar
I posted project 1 to Canvas

\end{itemize}


\subsection{Practice}
\label{\detokenize{mckinney_08_practice_04:practice}}

\subsubsection{Download data from Yahoo! Finance for BAC, C, GS, JPM, MS, and PNC and assign to data frame \sphinxstyleliteralintitle{\sphinxupquote{stocks}}.}
\label{\detokenize{mckinney_08_practice_04:download-data-from-yahoo-finance-for-bac-c-gs-jpm-ms-and-pnc-and-assign-to-data-frame-stocks}}
\sphinxAtStartPar
Use \sphinxcode{\sphinxupquote{stocks.columns.names}} to assign the names \sphinxcode{\sphinxupquote{Variable}} and \sphinxcode{\sphinxupquote{Ticker}} to the column multi index.

\begin{sphinxuseclass}{cell}\begin{sphinxVerbatimInput}

\begin{sphinxuseclass}{cell_input}
\begin{sphinxVerbatim}[commandchars=\\\{\}]
\PYG{k+kn}{import} \PYG{n+nn}{pandas} \PYG{k}{as} \PYG{n+nn}{pd}
\PYG{k+kn}{import} \PYG{n+nn}{numpy} \PYG{k}{as} \PYG{n+nn}{np}
\PYG{k+kn}{import} \PYG{n+nn}{matplotlib}\PYG{n+nn}{.}\PYG{n+nn}{pyplot} \PYG{k}{as} \PYG{n+nn}{plt}
\end{sphinxVerbatim}

\end{sphinxuseclass}\end{sphinxVerbatimInput}

\end{sphinxuseclass}
\begin{sphinxuseclass}{cell}\begin{sphinxVerbatimInput}

\begin{sphinxuseclass}{cell_input}
\begin{sphinxVerbatim}[commandchars=\\\{\}]
\PYG{o}{\PYGZpc{}}\PYG{k}{config} InlineBackend.figure\PYGZus{}format = \PYGZsq{}retina\PYGZsq{}
\PYG{o}{\PYGZpc{}}\PYG{k}{precision} 4
\PYG{n}{pd}\PYG{o}{.}\PYG{n}{options}\PYG{o}{.}\PYG{n}{display}\PYG{o}{.}\PYG{n}{float\PYGZus{}format} \PYG{o}{=} \PYG{l+s+s1}{\PYGZsq{}}\PYG{l+s+si}{\PYGZob{}:.4f\PYGZcb{}}\PYG{l+s+s1}{\PYGZsq{}}\PYG{o}{.}\PYG{n}{format}
\end{sphinxVerbatim}

\end{sphinxuseclass}\end{sphinxVerbatimInput}

\end{sphinxuseclass}
\begin{sphinxuseclass}{cell}\begin{sphinxVerbatimInput}

\begin{sphinxuseclass}{cell_input}
\begin{sphinxVerbatim}[commandchars=\\\{\}]
\PYG{k+kn}{import} \PYG{n+nn}{yfinance} \PYG{k}{as} \PYG{n+nn}{yf}
\PYG{k+kn}{import} \PYG{n+nn}{pandas\PYGZus{}datareader} \PYG{k}{as} \PYG{n+nn}{pdr}
\PYG{k+kn}{import} \PYG{n+nn}{requests\PYGZus{}cache}
\PYG{n}{session} \PYG{o}{=} \PYG{n}{requests\PYGZus{}cache}\PYG{o}{.}\PYG{n}{CachedSession}\PYG{p}{(}\PYG{p}{)}
\end{sphinxVerbatim}

\end{sphinxuseclass}\end{sphinxVerbatimInput}

\end{sphinxuseclass}
\begin{sphinxuseclass}{cell}\begin{sphinxVerbatimInput}

\begin{sphinxuseclass}{cell_input}
\begin{sphinxVerbatim}[commandchars=\\\{\}]
\PYG{n}{tickers} \PYG{o}{=} \PYG{n}{yf}\PYG{o}{.}\PYG{n}{Tickers}\PYG{p}{(}\PYG{n}{tickers}\PYG{o}{=}\PYG{l+s+s1}{\PYGZsq{}}\PYG{l+s+s1}{BAC C GS JPM MS PNC}\PYG{l+s+s1}{\PYGZsq{}}\PYG{p}{,} \PYG{n}{session}\PYG{o}{=}\PYG{n}{session}\PYG{p}{)}
\PYG{n}{stocks} \PYG{o}{=} \PYG{n}{tickers}\PYG{o}{.}\PYG{n}{history}\PYG{p}{(}\PYG{n}{period}\PYG{o}{=}\PYG{l+s+s1}{\PYGZsq{}}\PYG{l+s+s1}{max}\PYG{l+s+s1}{\PYGZsq{}}\PYG{p}{,} \PYG{n}{auto\PYGZus{}adjust}\PYG{o}{=}\PYG{k+kc}{False}\PYG{p}{,} \PYG{n}{progress}\PYG{o}{=}\PYG{k+kc}{False}\PYG{p}{)}
\PYG{n}{stocks}\PYG{o}{.}\PYG{n}{index} \PYG{o}{=} \PYG{n}{stocks}\PYG{o}{.}\PYG{n}{index}\PYG{o}{.}\PYG{n}{tz\PYGZus{}localize}\PYG{p}{(}\PYG{k+kc}{None}\PYG{p}{)}
\PYG{n}{stocks}\PYG{o}{.}\PYG{n}{columns}\PYG{o}{.}\PYG{n}{names} \PYG{o}{=} \PYG{p}{[}\PYG{l+s+s1}{\PYGZsq{}}\PYG{l+s+s1}{Variable}\PYG{l+s+s1}{\PYGZsq{}}\PYG{p}{,} \PYG{l+s+s1}{\PYGZsq{}}\PYG{l+s+s1}{Ticker}\PYG{l+s+s1}{\PYGZsq{}}\PYG{p}{]}
\PYG{n}{stocks}\PYG{o}{.}\PYG{n}{head}\PYG{p}{(}\PYG{p}{)}
\end{sphinxVerbatim}

\end{sphinxuseclass}\end{sphinxVerbatimInput}
\begin{sphinxVerbatimOutput}

\begin{sphinxuseclass}{cell_output}
\begin{sphinxVerbatim}[commandchars=\\\{\}]
[*********************100\PYGZpc{}***********************]  6 of 6 completed
\end{sphinxVerbatim}

\begin{sphinxVerbatim}[commandchars=\\\{\}]
Variable   Adj Close                      Close              ... Stock Splits  \PYGZbs{}
Ticker           BAC   C  GS JPM  MS PNC    BAC   C  GS JPM  ...           GS   
Date                                                         ...                
1973\PYGZhy{}02\PYGZhy{}21    1.6311 NaN NaN NaN NaN NaN 4.6250 NaN NaN NaN  ...          NaN   
1973\PYGZhy{}02\PYGZhy{}22    1.6366 NaN NaN NaN NaN NaN 4.6406 NaN NaN NaN  ...          NaN   
1973\PYGZhy{}02\PYGZhy{}23    1.6311 NaN NaN NaN NaN NaN 4.6250 NaN NaN NaN  ...          NaN   
1973\PYGZhy{}02\PYGZhy{}26    1.6311 NaN NaN NaN NaN NaN 4.6250 NaN NaN NaN  ...          NaN   
1973\PYGZhy{}02\PYGZhy{}27    1.6311 NaN NaN NaN NaN NaN 4.6250 NaN NaN NaN  ...          NaN   

Variable                Volume                      
Ticker     JPM  MS PNC     BAC   C  GS JPM  MS PNC  
Date                                                
1973\PYGZhy{}02\PYGZhy{}21 NaN NaN NaN   99200 NaN NaN NaN NaN NaN  
1973\PYGZhy{}02\PYGZhy{}22 NaN NaN NaN   47200 NaN NaN NaN NaN NaN  
1973\PYGZhy{}02\PYGZhy{}23 NaN NaN NaN  133600 NaN NaN NaN NaN NaN  
1973\PYGZhy{}02\PYGZhy{}26 NaN NaN NaN   24000 NaN NaN NaN NaN NaN  
1973\PYGZhy{}02\PYGZhy{}27 NaN NaN NaN   41600 NaN NaN NaN NaN NaN  

[5 rows x 48 columns]
\end{sphinxVerbatim}

\end{sphinxuseclass}\end{sphinxVerbatimOutput}

\end{sphinxuseclass}

\subsubsection{Reshape \sphinxstyleliteralintitle{\sphinxupquote{stocks}} from wide to long with dates and tickers as row indexes and assign to data frame \sphinxstyleliteralintitle{\sphinxupquote{stocks\_long}}.}
\label{\detokenize{mckinney_08_practice_04:reshape-stocks-from-wide-to-long-with-dates-and-tickers-as-row-indexes-and-assign-to-data-frame-stocks-long}}
\begin{sphinxuseclass}{cell}\begin{sphinxVerbatimInput}

\begin{sphinxuseclass}{cell_input}
\begin{sphinxVerbatim}[commandchars=\\\{\}]
\PYG{n}{stocks\PYGZus{}long} \PYG{o}{=} \PYG{n}{stocks}\PYG{o}{.}\PYG{n}{stack}\PYG{p}{(}\PYG{p}{)}
\PYG{n}{stocks\PYGZus{}long}\PYG{o}{.}\PYG{n}{head}\PYG{p}{(}\PYG{p}{)}
\end{sphinxVerbatim}

\end{sphinxuseclass}\end{sphinxVerbatimInput}
\begin{sphinxVerbatimOutput}

\begin{sphinxuseclass}{cell_output}
\begin{sphinxVerbatim}[commandchars=\\\{\}]
Variable           Adj Close  Close  Dividends   High    Low   Open  \PYGZbs{}
Date       Ticker                                                     
1973\PYGZhy{}02\PYGZhy{}21 BAC        1.6311 4.6250     0.0000 4.6250 4.6250 4.6250   
1973\PYGZhy{}02\PYGZhy{}22 BAC        1.6366 4.6406     0.0000 4.6406 4.6406 4.6406   
1973\PYGZhy{}02\PYGZhy{}23 BAC        1.6311 4.6250     0.0000 4.6250 4.6250 4.6250   
1973\PYGZhy{}02\PYGZhy{}26 BAC        1.6311 4.6250     0.0000 4.6250 4.6250 4.6250   
1973\PYGZhy{}02\PYGZhy{}27 BAC        1.6311 4.6250     0.0000 4.6250 4.6250 4.6250   

Variable           Stock Splits      Volume  
Date       Ticker                            
1973\PYGZhy{}02\PYGZhy{}21 BAC           0.0000  99200.0000  
1973\PYGZhy{}02\PYGZhy{}22 BAC           0.0000  47200.0000  
1973\PYGZhy{}02\PYGZhy{}23 BAC           0.0000 133600.0000  
1973\PYGZhy{}02\PYGZhy{}26 BAC           0.0000  24000.0000  
1973\PYGZhy{}02\PYGZhy{}27 BAC           0.0000  41600.0000  
\end{sphinxVerbatim}

\end{sphinxuseclass}\end{sphinxVerbatimOutput}

\end{sphinxuseclass}

\subsubsection{Add daily returns for each stock to data frames \sphinxstyleliteralintitle{\sphinxupquote{stocks}} and \sphinxstyleliteralintitle{\sphinxupquote{stocks\_long}}.}
\label{\detokenize{mckinney_08_practice_04:add-daily-returns-for-each-stock-to-data-frames-stocks-and-stocks-long}}
\sphinxAtStartPar
Name the returns variable \sphinxcode{\sphinxupquote{Returns}}, and maintain all multi indexes.
\sphinxstyleemphasis{Hint:} Use \sphinxcode{\sphinxupquote{pd.MultiIndex()}} to create a multi index for the the wide data frame \sphinxcode{\sphinxupquote{stocks}}.

\begin{sphinxuseclass}{cell}\begin{sphinxVerbatimInput}

\begin{sphinxuseclass}{cell_input}
\begin{sphinxVerbatim}[commandchars=\\\{\}]
\PYG{n}{\PYGZus{}} \PYG{o}{=} \PYG{n}{pd}\PYG{o}{.}\PYG{n}{MultiIndex}\PYG{o}{.}\PYG{n}{from\PYGZus{}product}\PYG{p}{(}\PYG{p}{[}\PYG{p}{[}\PYG{l+s+s1}{\PYGZsq{}}\PYG{l+s+s1}{Returns}\PYG{l+s+s1}{\PYGZsq{}}\PYG{p}{]}\PYG{p}{,} \PYG{n}{stocks}\PYG{p}{[}\PYG{l+s+s1}{\PYGZsq{}}\PYG{l+s+s1}{Adj Close}\PYG{l+s+s1}{\PYGZsq{}}\PYG{p}{]}\PYG{p}{]}\PYG{p}{)}
\PYG{n}{stocks}\PYG{p}{[}\PYG{n}{\PYGZus{}}\PYG{p}{]} \PYG{o}{=} \PYG{n}{stocks}\PYG{p}{[}\PYG{l+s+s1}{\PYGZsq{}}\PYG{l+s+s1}{Adj Close}\PYG{l+s+s1}{\PYGZsq{}}\PYG{p}{]}\PYG{o}{.}\PYG{n}{pct\PYGZus{}change}\PYG{p}{(}\PYG{p}{)}
\PYG{n}{stocks}\PYG{o}{.}\PYG{n}{head}\PYG{p}{(}\PYG{p}{)}
\end{sphinxVerbatim}

\end{sphinxuseclass}\end{sphinxVerbatimInput}
\begin{sphinxVerbatimOutput}

\begin{sphinxuseclass}{cell_output}
\begin{sphinxVerbatim}[commandchars=\\\{\}]
Variable   Adj Close                      Close              ... Volume      \PYGZbs{}
Ticker           BAC   C  GS JPM  MS PNC    BAC   C  GS JPM  ...     GS JPM   
Date                                                         ...              
1973\PYGZhy{}02\PYGZhy{}21    1.6311 NaN NaN NaN NaN NaN 4.6250 NaN NaN NaN  ...    NaN NaN   
1973\PYGZhy{}02\PYGZhy{}22    1.6366 NaN NaN NaN NaN NaN 4.6406 NaN NaN NaN  ...    NaN NaN   
1973\PYGZhy{}02\PYGZhy{}23    1.6311 NaN NaN NaN NaN NaN 4.6250 NaN NaN NaN  ...    NaN NaN   
1973\PYGZhy{}02\PYGZhy{}26    1.6311 NaN NaN NaN NaN NaN 4.6250 NaN NaN NaN  ...    NaN NaN   
1973\PYGZhy{}02\PYGZhy{}27    1.6311 NaN NaN NaN NaN NaN 4.6250 NaN NaN NaN  ...    NaN NaN   

Variable           Returns                      
Ticker      MS PNC     BAC   C  GS JPM  MS PNC  
Date                                            
1973\PYGZhy{}02\PYGZhy{}21 NaN NaN     NaN NaN NaN NaN NaN NaN  
1973\PYGZhy{}02\PYGZhy{}22 NaN NaN  0.0034 NaN NaN NaN NaN NaN  
1973\PYGZhy{}02\PYGZhy{}23 NaN NaN \PYGZhy{}0.0034 NaN NaN NaN NaN NaN  
1973\PYGZhy{}02\PYGZhy{}26 NaN NaN  0.0000 NaN NaN NaN NaN NaN  
1973\PYGZhy{}02\PYGZhy{}27 NaN NaN  0.0000 NaN NaN NaN NaN NaN  

[5 rows x 54 columns]
\end{sphinxVerbatim}

\end{sphinxuseclass}\end{sphinxVerbatimOutput}

\end{sphinxuseclass}
\sphinxAtStartPar
The easiest way to add returns to long data frame \sphinxcode{\sphinxupquote{stocks\_long}} is to \sphinxcode{\sphinxupquote{.stack()}} wide data frame \sphinxcode{\sphinxupquote{stocks}}!
We could sort \sphinxcode{\sphinxupquote{stocks\_long}} by ticker and date (to sort chronologically within each ticker), then use \sphinxcode{\sphinxupquote{.pct\_change()}}.
However, this approach miscalculates the first return for every ticker except for the first ticker.
The easiest and safest solution is to \sphinxcode{\sphinxupquote{.stack()}} the wide data frame \sphinxcode{\sphinxupquote{stocks}}!

\begin{sphinxuseclass}{cell}\begin{sphinxVerbatimInput}

\begin{sphinxuseclass}{cell_input}
\begin{sphinxVerbatim}[commandchars=\\\{\}]
\PYG{c+c1}{\PYGZsh{} see that the first return for C is wrong}
\PYG{c+c1}{\PYGZsh{} stocks\PYGZus{}long[\PYGZsq{}Adj Close\PYGZsq{}].sort\PYGZus{}index(level=[\PYGZsq{}Ticker\PYGZsq{}, \PYGZsq{}Date\PYGZsq{}]).pct\PYGZus{}change().loc[(slice(None), \PYGZsq{}C\PYGZsq{})]}
\end{sphinxVerbatim}

\end{sphinxuseclass}\end{sphinxVerbatimInput}

\end{sphinxuseclass}
\begin{sphinxuseclass}{cell}\begin{sphinxVerbatimInput}

\begin{sphinxuseclass}{cell_input}
\begin{sphinxVerbatim}[commandchars=\\\{\}]
\PYG{n}{stocks\PYGZus{}long} \PYG{o}{=} \PYG{n}{stocks}\PYG{o}{.}\PYG{n}{stack}\PYG{p}{(}\PYG{p}{)}
\end{sphinxVerbatim}

\end{sphinxuseclass}\end{sphinxVerbatimInput}

\end{sphinxuseclass}

\subsubsection{Download the daily benchmark return factors from Ken French’s data library.}
\label{\detokenize{mckinney_08_practice_04:download-the-daily-benchmark-return-factors-from-ken-french-s-data-library}}
\begin{sphinxuseclass}{cell}\begin{sphinxVerbatimInput}

\begin{sphinxuseclass}{cell_input}
\begin{sphinxVerbatim}[commandchars=\\\{\}]
\PYG{n}{pdr}\PYG{o}{.}\PYG{n}{famafrench}\PYG{o}{.}\PYG{n}{get\PYGZus{}available\PYGZus{}datasets}\PYG{p}{(}\PYG{p}{)}\PYG{p}{[}\PYG{p}{:}\PYG{l+m+mi}{5}\PYG{p}{]}
\end{sphinxVerbatim}

\end{sphinxuseclass}\end{sphinxVerbatimInput}
\begin{sphinxVerbatimOutput}

\begin{sphinxuseclass}{cell_output}
\begin{sphinxVerbatim}[commandchars=\\\{\}]
[\PYGZsq{}F\PYGZhy{}F\PYGZus{}Research\PYGZus{}Data\PYGZus{}Factors\PYGZsq{},
 \PYGZsq{}F\PYGZhy{}F\PYGZus{}Research\PYGZus{}Data\PYGZus{}Factors\PYGZus{}weekly\PYGZsq{},
 \PYGZsq{}F\PYGZhy{}F\PYGZus{}Research\PYGZus{}Data\PYGZus{}Factors\PYGZus{}daily\PYGZsq{},
 \PYGZsq{}F\PYGZhy{}F\PYGZus{}Research\PYGZus{}Data\PYGZus{}5\PYGZus{}Factors\PYGZus{}2x3\PYGZsq{},
 \PYGZsq{}F\PYGZhy{}F\PYGZus{}Research\PYGZus{}Data\PYGZus{}5\PYGZus{}Factors\PYGZus{}2x3\PYGZus{}daily\PYGZsq{}]
\end{sphinxVerbatim}

\end{sphinxuseclass}\end{sphinxVerbatimOutput}

\end{sphinxuseclass}
\begin{sphinxuseclass}{cell}\begin{sphinxVerbatimInput}

\begin{sphinxuseclass}{cell_input}
\begin{sphinxVerbatim}[commandchars=\\\{\}]
\PYG{n}{ff} \PYG{o}{=} \PYG{p}{(}
    \PYG{n}{pdr}\PYG{o}{.}\PYG{n}{DataReader}\PYG{p}{(}
        \PYG{n}{name}\PYG{o}{=}\PYG{l+s+s1}{\PYGZsq{}}\PYG{l+s+s1}{F\PYGZhy{}F\PYGZus{}Research\PYGZus{}Data\PYGZus{}Factors\PYGZus{}daily}\PYG{l+s+s1}{\PYGZsq{}}\PYG{p}{,}
        \PYG{n}{data\PYGZus{}source}\PYG{o}{=}\PYG{l+s+s1}{\PYGZsq{}}\PYG{l+s+s1}{famafrench}\PYG{l+s+s1}{\PYGZsq{}}\PYG{p}{,}
        \PYG{n}{start}\PYG{o}{=}\PYG{l+s+s1}{\PYGZsq{}}\PYG{l+s+s1}{1900}\PYG{l+s+s1}{\PYGZsq{}}\PYG{p}{,}
        \PYG{n}{session}\PYG{o}{=}\PYG{n}{session}
    \PYG{p}{)}
    \PYG{p}{[}\PYG{l+m+mi}{0}\PYG{p}{]}
    \PYG{o}{.}\PYG{n}{div}\PYG{p}{(}\PYG{l+m+mi}{100}\PYG{p}{)}
\PYG{p}{)}

\PYG{n}{ff}\PYG{o}{.}\PYG{n}{head}\PYG{p}{(}\PYG{p}{)}
\end{sphinxVerbatim}

\end{sphinxuseclass}\end{sphinxVerbatimInput}
\begin{sphinxVerbatimOutput}

\begin{sphinxuseclass}{cell_output}
\begin{sphinxVerbatim}[commandchars=\\\{\}]
            Mkt\PYGZhy{}RF     SMB     HML     RF
Date                                     
1926\PYGZhy{}07\PYGZhy{}01  0.0010 \PYGZhy{}0.0025 \PYGZhy{}0.0027 0.0001
1926\PYGZhy{}07\PYGZhy{}02  0.0045 \PYGZhy{}0.0033 \PYGZhy{}0.0006 0.0001
1926\PYGZhy{}07\PYGZhy{}06  0.0017  0.0030 \PYGZhy{}0.0039 0.0001
1926\PYGZhy{}07\PYGZhy{}07  0.0009 \PYGZhy{}0.0058  0.0002 0.0001
1926\PYGZhy{}07\PYGZhy{}08  0.0021 \PYGZhy{}0.0038  0.0019 0.0001
\end{sphinxVerbatim}

\end{sphinxuseclass}\end{sphinxVerbatimOutput}

\end{sphinxuseclass}

\subsubsection{Add the daily benchmark return factors to \sphinxstyleliteralintitle{\sphinxupquote{stocks}} and \sphinxstyleliteralintitle{\sphinxupquote{stocks\_long}}.}
\label{\detokenize{mckinney_08_practice_04:add-the-daily-benchmark-return-factors-to-stocks-and-stocks-long}}
\sphinxAtStartPar
For the wide data frame \sphinxcode{\sphinxupquote{stocks}}, use the outer index name \sphinxcode{\sphinxupquote{Factors}}.

\begin{sphinxuseclass}{cell}\begin{sphinxVerbatimInput}

\begin{sphinxuseclass}{cell_input}
\begin{sphinxVerbatim}[commandchars=\\\{\}]
\PYG{n}{\PYGZus{}} \PYG{o}{=} \PYG{n}{pd}\PYG{o}{.}\PYG{n}{MultiIndex}\PYG{o}{.}\PYG{n}{from\PYGZus{}product}\PYG{p}{(}\PYG{p}{[}\PYG{p}{[}\PYG{l+s+s1}{\PYGZsq{}}\PYG{l+s+s1}{Factors}\PYG{l+s+s1}{\PYGZsq{}}\PYG{p}{]}\PYG{p}{,} \PYG{n}{ff}\PYG{p}{]}\PYG{p}{)}
\PYG{n}{stocks}\PYG{p}{[}\PYG{n}{\PYGZus{}}\PYG{p}{]} \PYG{o}{=} \PYG{n}{ff}
\PYG{n}{stocks}\PYG{o}{.}\PYG{n}{head}\PYG{p}{(}\PYG{p}{)}
\end{sphinxVerbatim}

\end{sphinxuseclass}\end{sphinxVerbatimInput}
\begin{sphinxVerbatimOutput}

\begin{sphinxuseclass}{cell_output}
\begin{sphinxVerbatim}[commandchars=\\\{\}]
Variable   Adj Close                      Close              ... Returns      \PYGZbs{}
Ticker           BAC   C  GS JPM  MS PNC    BAC   C  GS JPM  ...     BAC   C   
Date                                                         ...               
1973\PYGZhy{}02\PYGZhy{}21    1.6311 NaN NaN NaN NaN NaN 4.6250 NaN NaN NaN  ...     NaN NaN   
1973\PYGZhy{}02\PYGZhy{}22    1.6366 NaN NaN NaN NaN NaN 4.6406 NaN NaN NaN  ...  0.0034 NaN   
1973\PYGZhy{}02\PYGZhy{}23    1.6311 NaN NaN NaN NaN NaN 4.6250 NaN NaN NaN  ... \PYGZhy{}0.0034 NaN   
1973\PYGZhy{}02\PYGZhy{}26    1.6311 NaN NaN NaN NaN NaN 4.6250 NaN NaN NaN  ...  0.0000 NaN   
1973\PYGZhy{}02\PYGZhy{}27    1.6311 NaN NaN NaN NaN NaN 4.6250 NaN NaN NaN  ...  0.0000 NaN   

Variable                   Factors                        
Ticker      GS JPM  MS PNC  Mkt\PYGZhy{}RF     SMB    HML     RF  
Date                                                      
1973\PYGZhy{}02\PYGZhy{}21 NaN NaN NaN NaN \PYGZhy{}0.0074 \PYGZhy{}0.0039 0.0054 0.0002  
1973\PYGZhy{}02\PYGZhy{}22 NaN NaN NaN NaN \PYGZhy{}0.0030 \PYGZhy{}0.0037 0.0022 0.0002  
1973\PYGZhy{}02\PYGZhy{}23 NaN NaN NaN NaN \PYGZhy{}0.0108 \PYGZhy{}0.0019 0.0054 0.0002  
1973\PYGZhy{}02\PYGZhy{}26 NaN NaN NaN NaN \PYGZhy{}0.0088 \PYGZhy{}0.0050 0.0054 0.0002  
1973\PYGZhy{}02\PYGZhy{}27 NaN NaN NaN NaN \PYGZhy{}0.0115 \PYGZhy{}0.0018 0.0064 0.0002  

[5 rows x 58 columns]
\end{sphinxVerbatim}

\end{sphinxuseclass}\end{sphinxVerbatimOutput}

\end{sphinxuseclass}
\sphinxAtStartPar
We can use \sphinxcode{\sphinxupquote{.join()}} even though \sphinxcode{\sphinxupquote{stocks\_long}} has a multi index.
\sphinxstyleemphasis{\sphinxstylestrong{Note that re\sphinxhyphen{}running a “self join” can create duplicate columns.}}
We should be careful to run self joins only once!

\begin{sphinxuseclass}{cell}\begin{sphinxVerbatimInput}

\begin{sphinxuseclass}{cell_input}
\begin{sphinxVerbatim}[commandchars=\\\{\}]
\PYG{n}{stocks\PYGZus{}long} \PYG{o}{=} \PYG{n}{stocks\PYGZus{}long}\PYG{o}{.}\PYG{n}{join}\PYG{p}{(}\PYG{n}{ff}\PYG{p}{)}
\PYG{n}{stocks\PYGZus{}long}\PYG{o}{.}\PYG{n}{head}\PYG{p}{(}\PYG{p}{)}
\end{sphinxVerbatim}

\end{sphinxuseclass}\end{sphinxVerbatimInput}
\begin{sphinxVerbatimOutput}

\begin{sphinxuseclass}{cell_output}
\begin{sphinxVerbatim}[commandchars=\\\{\}]
                   Adj Close  Close  Dividends   High    Low   Open  \PYGZbs{}
Date       Ticker                                                     
1973\PYGZhy{}02\PYGZhy{}21 BAC        1.6311 4.6250     0.0000 4.6250 4.6250 4.6250   
1973\PYGZhy{}02\PYGZhy{}22 BAC        1.6366 4.6406     0.0000 4.6406 4.6406 4.6406   
1973\PYGZhy{}02\PYGZhy{}23 BAC        1.6311 4.6250     0.0000 4.6250 4.6250 4.6250   
1973\PYGZhy{}02\PYGZhy{}26 BAC        1.6311 4.6250     0.0000 4.6250 4.6250 4.6250   
1973\PYGZhy{}02\PYGZhy{}27 BAC        1.6311 4.6250     0.0000 4.6250 4.6250 4.6250   

                   Stock Splits      Volume  Returns  Mkt\PYGZhy{}RF     SMB    HML  \PYGZbs{}
Date       Ticker                                                             
1973\PYGZhy{}02\PYGZhy{}21 BAC           0.0000  99200.0000      NaN \PYGZhy{}0.0074 \PYGZhy{}0.0039 0.0054   
1973\PYGZhy{}02\PYGZhy{}22 BAC           0.0000  47200.0000   0.0034 \PYGZhy{}0.0030 \PYGZhy{}0.0037 0.0022   
1973\PYGZhy{}02\PYGZhy{}23 BAC           0.0000 133600.0000  \PYGZhy{}0.0034 \PYGZhy{}0.0108 \PYGZhy{}0.0019 0.0054   
1973\PYGZhy{}02\PYGZhy{}26 BAC           0.0000  24000.0000   0.0000 \PYGZhy{}0.0088 \PYGZhy{}0.0050 0.0054   
1973\PYGZhy{}02\PYGZhy{}27 BAC           0.0000  41600.0000   0.0000 \PYGZhy{}0.0115 \PYGZhy{}0.0018 0.0064   

                      RF  
Date       Ticker         
1973\PYGZhy{}02\PYGZhy{}21 BAC    0.0002  
1973\PYGZhy{}02\PYGZhy{}22 BAC    0.0002  
1973\PYGZhy{}02\PYGZhy{}23 BAC    0.0002  
1973\PYGZhy{}02\PYGZhy{}26 BAC    0.0002  
1973\PYGZhy{}02\PYGZhy{}27 BAC    0.0002  
\end{sphinxVerbatim}

\end{sphinxuseclass}\end{sphinxVerbatimOutput}

\end{sphinxuseclass}

\subsubsection{Write a function \sphinxstyleliteralintitle{\sphinxupquote{download()}} that accepts tickers and returns a wide data frame of returns with the daily benchmark return factors.}
\label{\detokenize{mckinney_08_practice_04:write-a-function-download-that-accepts-tickers-and-returns-a-wide-data-frame-of-returns-with-the-daily-benchmark-return-factors}}
\sphinxAtStartPar
\sphinxstyleemphasis{\sphinxstylestrong{See advanced solution below.}}
I had planned this as an in\sphinxhyphen{}class exercise, but decided to spend more time on the earnings surprise exercise.


\subsubsection{Download earnings per share for the stocks in \sphinxstyleliteralintitle{\sphinxupquote{stocks}} and combine to a long data frame \sphinxstyleliteralintitle{\sphinxupquote{earnings}}.}
\label{\detokenize{mckinney_08_practice_04:download-earnings-per-share-for-the-stocks-in-stocks-and-combine-to-a-long-data-frame-earnings}}
\sphinxAtStartPar
Use the \sphinxcode{\sphinxupquote{.earnings\_dates}} method described \sphinxhref{https://pypi.org/project/yfinance/}{here}.
Use \sphinxcode{\sphinxupquote{pd.concat()}} to combine the result of each the \sphinxcode{\sphinxupquote{.earnings\_date}} data frames and assign them to a new data frame \sphinxcode{\sphinxupquote{earnings}}.
Name the row indexes \sphinxcode{\sphinxupquote{Ticker}} and \sphinxcode{\sphinxupquote{Date}} and swap to match the order of the row index in \sphinxcode{\sphinxupquote{stocks\_long}}.

\begin{sphinxuseclass}{cell}\begin{sphinxVerbatimInput}

\begin{sphinxuseclass}{cell_input}
\begin{sphinxVerbatim}[commandchars=\\\{\}]
\PYG{c+c1}{\PYGZsh{} some students had to update yfinance to use the .earnings\PYGZus{}dates atrtibute}
\PYG{c+c1}{\PYGZsh{} \PYGZpc{}pip install \PYGZhy{}U yfinance}
\end{sphinxVerbatim}

\end{sphinxuseclass}\end{sphinxVerbatimInput}

\end{sphinxuseclass}
\begin{sphinxuseclass}{cell}\begin{sphinxVerbatimInput}

\begin{sphinxuseclass}{cell_input}
\begin{sphinxVerbatim}[commandchars=\\\{\}]
\PYG{n}{tickers}\PYG{o}{.}\PYG{n}{tickers}\PYG{p}{[}\PYG{l+s+s1}{\PYGZsq{}}\PYG{l+s+s1}{BAC}\PYG{l+s+s1}{\PYGZsq{}}\PYG{p}{]}\PYG{o}{.}\PYG{n}{earnings\PYGZus{}dates}\PYG{o}{.}\PYG{n}{head}\PYG{p}{(}\PYG{l+m+mi}{2}\PYG{p}{)}
\end{sphinxVerbatim}

\end{sphinxuseclass}\end{sphinxVerbatimInput}
\begin{sphinxVerbatimOutput}

\begin{sphinxuseclass}{cell_output}
\begin{sphinxVerbatim}[commandchars=\\\{\}]
                           EPS Estimate  Reported EPS  Surprise(\PYGZpc{})
Earnings Date                                                     
2024\PYGZhy{}01\PYGZhy{}11 08:00:00\PYGZhy{}05:00           NaN           NaN          NaN
2023\PYGZhy{}10\PYGZhy{}17 08:00:00\PYGZhy{}04:00           NaN           NaN          NaN
\end{sphinxVerbatim}

\end{sphinxuseclass}\end{sphinxVerbatimOutput}

\end{sphinxuseclass}
\begin{sphinxuseclass}{cell}\begin{sphinxVerbatimInput}

\begin{sphinxuseclass}{cell_input}
\begin{sphinxVerbatim}[commandchars=\\\{\}]
\PYG{n}{earnings} \PYG{o}{=} \PYG{p}{(}
    \PYG{n}{pd}\PYG{o}{.}\PYG{n}{concat}\PYG{p}{(}
        \PYG{n}{objs}\PYG{o}{=}\PYG{p}{[}\PYG{n}{tickers}\PYG{o}{.}\PYG{n}{tickers}\PYG{p}{[}\PYG{n}{t}\PYG{p}{]}\PYG{o}{.}\PYG{n}{earnings\PYGZus{}dates} \PYG{k}{for} \PYG{n}{t} \PYG{o+ow}{in} \PYG{n}{tickers}\PYG{o}{.}\PYG{n}{tickers}\PYG{p}{]}\PYG{p}{,}
        \PYG{n}{keys}\PYG{o}{=}\PYG{n}{tickers}\PYG{o}{.}\PYG{n}{tickers}\PYG{p}{,}
        \PYG{n}{names}\PYG{o}{=}\PYG{p}{[}\PYG{l+s+s1}{\PYGZsq{}}\PYG{l+s+s1}{Ticker}\PYG{l+s+s1}{\PYGZsq{}}\PYG{p}{,} \PYG{l+s+s1}{\PYGZsq{}}\PYG{l+s+s1}{Date}\PYG{l+s+s1}{\PYGZsq{}}\PYG{p}{]}
    \PYG{p}{)}
    \PYG{o}{.}\PYG{n}{swaplevel}\PYG{p}{(}\PYG{p}{)}
    \PYG{o}{.}\PYG{n}{rename\PYGZus{}axis}\PYG{p}{(}\PYG{n}{columns}\PYG{o}{=}\PYG{l+s+s1}{\PYGZsq{}}\PYG{l+s+s1}{Variable}\PYG{l+s+s1}{\PYGZsq{}}\PYG{p}{)}
\PYG{p}{)}

\PYG{n}{earnings}\PYG{o}{.}\PYG{n}{head}\PYG{p}{(}\PYG{l+m+mi}{2}\PYG{p}{)}
\end{sphinxVerbatim}

\end{sphinxuseclass}\end{sphinxVerbatimInput}
\begin{sphinxVerbatimOutput}

\begin{sphinxuseclass}{cell_output}
\begin{sphinxVerbatim}[commandchars=\\\{\}]
Variable                          EPS Estimate  Reported EPS  Surprise(\PYGZpc{})
Date                      Ticker                                         
2024\PYGZhy{}01\PYGZhy{}11 08:00:00\PYGZhy{}05:00 BAC              NaN           NaN          NaN
2023\PYGZhy{}10\PYGZhy{}17 08:00:00\PYGZhy{}04:00 BAC              NaN           NaN          NaN
\end{sphinxVerbatim}

\end{sphinxuseclass}\end{sphinxVerbatimOutput}

\end{sphinxuseclass}

\subsubsection{Combine \sphinxstyleliteralintitle{\sphinxupquote{earnings}} with the returns from \sphinxstyleliteralintitle{\sphinxupquote{stocks\_long}}.}
\label{\detokenize{mckinney_08_practice_04:combine-earnings-with-the-returns-from-stocks-long}}
\sphinxAtStartPar
\sphinxstyleemphasis{\sphinxstylestrong{It is easier to leave \sphinxcode{\sphinxupquote{stocks}} and \sphinxcode{\sphinxupquote{stocks\_long}} as\sphinxhyphen{}is and work with slices \sphinxcode{\sphinxupquote{returns}} and \sphinxcode{\sphinxupquote{returns\_long}}.}}
Use the \sphinxcode{\sphinxupquote{tz\_localize('America/New\_York')}} method add time zone information back to \sphinxcode{\sphinxupquote{returns.index}} and use \sphinxcode{\sphinxupquote{pd.to\_timedelta(16, unit='h')}} to set time to the market close in New York City.
Use \sphinxhref{https://pandas.pydata.org/pandas-docs/stable/reference/api/pandas.merge\_asof.html}{\sphinxcode{\sphinxupquote{pd.merge\_asof()}}} to match earnings announcement dates and times to appropriate return periods.
For example, if a firm announces earnings after the close at 5 PM on February 7, we want to match the return period from 4 PM on February 7 to 4 PM on February 8.

\begin{sphinxuseclass}{cell}\begin{sphinxVerbatimInput}

\begin{sphinxuseclass}{cell_input}
\begin{sphinxVerbatim}[commandchars=\\\{\}]
\PYG{n}{returns} \PYG{o}{=} \PYG{n}{stocks}\PYG{p}{[}\PYG{l+s+s1}{\PYGZsq{}}\PYG{l+s+s1}{Returns}\PYG{l+s+s1}{\PYGZsq{}}\PYG{p}{]}
\PYG{n}{returns}\PYG{o}{.}\PYG{n}{index} \PYG{o}{=} \PYG{n}{returns}\PYG{o}{.}\PYG{n}{index}\PYG{o}{.}\PYG{n}{tz\PYGZus{}localize}\PYG{p}{(}\PYG{l+s+s1}{\PYGZsq{}}\PYG{l+s+s1}{America/New\PYGZus{}York}\PYG{l+s+s1}{\PYGZsq{}}\PYG{p}{)} \PYG{o}{+} \PYG{n}{pd}\PYG{o}{.}\PYG{n}{to\PYGZus{}timedelta}\PYG{p}{(}\PYG{l+m+mi}{16}\PYG{p}{,} \PYG{n}{unit}\PYG{o}{=}\PYG{l+s+s1}{\PYGZsq{}}\PYG{l+s+s1}{H}\PYG{l+s+s1}{\PYGZsq{}}\PYG{p}{)}
\PYG{n}{returns\PYGZus{}long} \PYG{o}{=} \PYG{n}{returns}\PYG{o}{.}\PYG{n}{stack}\PYG{p}{(}\PYG{p}{)}\PYG{o}{.}\PYG{n}{to\PYGZus{}frame}\PYG{p}{(}\PYG{l+s+s1}{\PYGZsq{}}\PYG{l+s+s1}{Returns}\PYG{l+s+s1}{\PYGZsq{}}\PYG{p}{)}
\PYG{n}{returns\PYGZus{}long}\PYG{o}{.}\PYG{n}{head}\PYG{p}{(}\PYG{p}{)}
\end{sphinxVerbatim}

\end{sphinxuseclass}\end{sphinxVerbatimInput}
\begin{sphinxVerbatimOutput}

\begin{sphinxuseclass}{cell_output}
\begin{sphinxVerbatim}[commandchars=\\\{\}]
                                  Returns
Date                      Ticker         
1973\PYGZhy{}02\PYGZhy{}22 16:00:00\PYGZhy{}05:00 BAC      0.0034
1973\PYGZhy{}02\PYGZhy{}23 16:00:00\PYGZhy{}05:00 BAC     \PYGZhy{}0.0034
1973\PYGZhy{}02\PYGZhy{}26 16:00:00\PYGZhy{}05:00 BAC      0.0000
1973\PYGZhy{}02\PYGZhy{}27 16:00:00\PYGZhy{}05:00 BAC      0.0000
1973\PYGZhy{}02\PYGZhy{}28 16:00:00\PYGZhy{}05:00 BAC      0.0000
\end{sphinxVerbatim}

\end{sphinxuseclass}\end{sphinxVerbatimOutput}

\end{sphinxuseclass}
\begin{sphinxuseclass}{cell}\begin{sphinxVerbatimInput}

\begin{sphinxuseclass}{cell_input}
\begin{sphinxVerbatim}[commandchars=\\\{\}]
\PYG{n}{surprises} \PYG{o}{=} \PYG{p}{(}
    \PYG{n}{pd}\PYG{o}{.}\PYG{n}{merge\PYGZus{}asof}\PYG{p}{(}
        \PYG{n}{left}\PYG{o}{=}\PYG{n}{earnings}\PYG{o}{.}\PYG{n}{sort\PYGZus{}index}\PYG{p}{(}\PYG{n}{level}\PYG{o}{=}\PYG{p}{[}\PYG{l+s+s1}{\PYGZsq{}}\PYG{l+s+s1}{Date}\PYG{l+s+s1}{\PYGZsq{}}\PYG{p}{,} \PYG{l+s+s1}{\PYGZsq{}}\PYG{l+s+s1}{Ticker}\PYG{l+s+s1}{\PYGZsq{}}\PYG{p}{]}\PYG{p}{)}\PYG{p}{,}
        \PYG{n}{right}\PYG{o}{=}\PYG{n}{returns\PYGZus{}long}\PYG{o}{.}\PYG{n}{sort\PYGZus{}index}\PYG{p}{(}\PYG{n}{level}\PYG{o}{=}\PYG{p}{[}\PYG{l+s+s1}{\PYGZsq{}}\PYG{l+s+s1}{Date}\PYG{l+s+s1}{\PYGZsq{}}\PYG{p}{,} \PYG{l+s+s1}{\PYGZsq{}}\PYG{l+s+s1}{Ticker}\PYG{l+s+s1}{\PYGZsq{}}\PYG{p}{]}\PYG{p}{)}\PYG{p}{,}
        \PYG{n}{on}\PYG{o}{=}\PYG{l+s+s1}{\PYGZsq{}}\PYG{l+s+s1}{Date}\PYG{l+s+s1}{\PYGZsq{}}\PYG{p}{,}
        \PYG{n}{by}\PYG{o}{=}\PYG{l+s+s1}{\PYGZsq{}}\PYG{l+s+s1}{Ticker}\PYG{l+s+s1}{\PYGZsq{}}\PYG{p}{,}
        \PYG{n}{direction}\PYG{o}{=}\PYG{l+s+s1}{\PYGZsq{}}\PYG{l+s+s1}{forward}\PYG{l+s+s1}{\PYGZsq{}}\PYG{p}{,}
        \PYG{n}{allow\PYGZus{}exact\PYGZus{}matches}\PYG{o}{=}\PYG{k+kc}{False}
    \PYG{p}{)}
    \PYG{o}{.}\PYG{n}{set\PYGZus{}index}\PYG{p}{(}\PYG{p}{[}\PYG{l+s+s1}{\PYGZsq{}}\PYG{l+s+s1}{Date}\PYG{l+s+s1}{\PYGZsq{}}\PYG{p}{,} \PYG{l+s+s1}{\PYGZsq{}}\PYG{l+s+s1}{Ticker}\PYG{l+s+s1}{\PYGZsq{}}\PYG{p}{]}\PYG{p}{)}
\PYG{p}{)}

\PYG{n}{surprises}\PYG{o}{.}\PYG{n}{head}\PYG{p}{(}\PYG{p}{)}
\end{sphinxVerbatim}

\end{sphinxuseclass}\end{sphinxVerbatimInput}
\begin{sphinxVerbatimOutput}

\begin{sphinxuseclass}{cell_output}
\begin{sphinxVerbatim}[commandchars=\\\{\}]
                                  EPS Estimate  Reported EPS  Surprise(\PYGZpc{})  \PYGZbs{}
Date                      Ticker                                            
2021\PYGZhy{}04\PYGZhy{}14 03:00:00\PYGZhy{}04:00 GS           10.2200       18.6000       0.8194   
2021\PYGZhy{}04\PYGZhy{}16 03:00:00\PYGZhy{}04:00 MS            1.7000        2.1900       0.2890   
2021\PYGZhy{}07\PYGZhy{}13 03:00:00\PYGZhy{}04:00 GS           10.2400       15.0200       0.4668   
2021\PYGZhy{}07\PYGZhy{}15 03:00:00\PYGZhy{}04:00 MS            1.6500        1.8500       0.1185   
2021\PYGZhy{}10\PYGZhy{}14 03:00:00\PYGZhy{}04:00 MS            1.6900        1.9800       0.1751   

                                  Returns  
Date                      Ticker           
2021\PYGZhy{}04\PYGZhy{}14 03:00:00\PYGZhy{}04:00 GS       0.0234  
2021\PYGZhy{}04\PYGZhy{}16 03:00:00\PYGZhy{}04:00 MS      \PYGZhy{}0.0276  
2021\PYGZhy{}07\PYGZhy{}13 03:00:00\PYGZhy{}04:00 GS      \PYGZhy{}0.0119  
2021\PYGZhy{}07\PYGZhy{}15 03:00:00\PYGZhy{}04:00 MS       0.0018  
2021\PYGZhy{}10\PYGZhy{}14 03:00:00\PYGZhy{}04:00 MS       0.0248  
\end{sphinxVerbatim}

\end{sphinxuseclass}\end{sphinxVerbatimOutput}

\end{sphinxuseclass}
\begin{sphinxuseclass}{cell}\begin{sphinxVerbatimInput}

\begin{sphinxuseclass}{cell_input}
\begin{sphinxVerbatim}[commandchars=\\\{\}]
\PYG{n}{surprises}\PYG{o}{.}\PYG{n}{corr}\PYG{p}{(}\PYG{p}{)}
\end{sphinxVerbatim}

\end{sphinxuseclass}\end{sphinxVerbatimInput}
\begin{sphinxVerbatimOutput}

\begin{sphinxuseclass}{cell_output}
\begin{sphinxVerbatim}[commandchars=\\\{\}]
              EPS Estimate  Reported EPS  Surprise(\PYGZpc{})  Returns
EPS Estimate        1.0000        0.9444       0.3620  \PYGZhy{}0.2239
Reported EPS        0.9444        1.0000       0.6171  \PYGZhy{}0.0615
Surprise(\PYGZpc{})         0.3620        0.6171       1.0000   0.4397
Returns            \PYGZhy{}0.2239       \PYGZhy{}0.0615       0.4397   1.0000
\end{sphinxVerbatim}

\end{sphinxuseclass}\end{sphinxVerbatimOutput}

\end{sphinxuseclass}
\begin{sphinxuseclass}{cell}\begin{sphinxVerbatimInput}

\begin{sphinxuseclass}{cell_input}
\begin{sphinxVerbatim}[commandchars=\\\{\}]
\PYG{n}{tickers}\PYG{o}{.}\PYG{n}{tickers}\PYG{p}{[}\PYG{l+s+s1}{\PYGZsq{}}\PYG{l+s+s1}{C}\PYG{l+s+s1}{\PYGZsq{}}\PYG{p}{]}\PYG{o}{.}\PYG{n}{earnings\PYGZus{}dates}
\end{sphinxVerbatim}

\end{sphinxuseclass}\end{sphinxVerbatimInput}
\begin{sphinxVerbatimOutput}

\begin{sphinxuseclass}{cell_output}
\begin{sphinxVerbatim}[commandchars=\\\{\}]
                           EPS Estimate  Reported EPS  Surprise(\PYGZpc{})
Earnings Date                                                     
2024\PYGZhy{}01\PYGZhy{}17 11:00:00\PYGZhy{}05:00           NaN           NaN          NaN
2024\PYGZhy{}01\PYGZhy{}17 08:00:00\PYGZhy{}05:00           NaN           NaN          NaN
2023\PYGZhy{}10\PYGZhy{}13 11:00:00\PYGZhy{}04:00           NaN           NaN          NaN
2023\PYGZhy{}10\PYGZhy{}13 08:00:00\PYGZhy{}04:00           NaN           NaN          NaN
2023\PYGZhy{}07\PYGZhy{}14 11:00:00\PYGZhy{}04:00           NaN           NaN          NaN
2023\PYGZhy{}07\PYGZhy{}14 08:00:00\PYGZhy{}04:00           NaN           NaN          NaN
2023\PYGZhy{}04\PYGZhy{}14 11:00:00\PYGZhy{}04:00        1.6700           NaN          NaN
2023\PYGZhy{}04\PYGZhy{}14 08:00:00\PYGZhy{}04:00        1.6700           NaN          NaN
2023\PYGZhy{}01\PYGZhy{}13 03:00:00\PYGZhy{}05:00        1.1400        1.1600       0.0149
2022\PYGZhy{}10\PYGZhy{}14 04:00:00\PYGZhy{}04:00        1.4200        1.6300       0.1471
2022\PYGZhy{}07\PYGZhy{}15 04:00:00\PYGZhy{}04:00        1.6800        2.1900       0.3020
2022\PYGZhy{}04\PYGZhy{}14 04:00:00\PYGZhy{}04:00        1.5500        2.0200       0.3058
\end{sphinxVerbatim}

\end{sphinxuseclass}\end{sphinxVerbatimOutput}

\end{sphinxuseclass}

\subsubsection{Plot the relation between daily returns and earnings surprises}
\label{\detokenize{mckinney_08_practice_04:plot-the-relation-between-daily-returns-and-earnings-surprises}}
\sphinxAtStartPar
Three options in increasing difficulty:
\begin{enumerate}
\sphinxsetlistlabels{\arabic}{enumi}{enumii}{}{.}%
\item {} 
\sphinxAtStartPar
Scatter plot

\item {} 
\sphinxAtStartPar
Scatter plot with a best\sphinxhyphen{}fit line using \sphinxcode{\sphinxupquote{regplot()}} from the seaborn package

\item {} 
\sphinxAtStartPar
Bar plot using \sphinxcode{\sphinxupquote{barplot()}} from the seaborn package after using \sphinxcode{\sphinxupquote{pd.qcut()}} to form five groups on earnings surprises

\end{enumerate}

\begin{sphinxuseclass}{cell}\begin{sphinxVerbatimInput}

\begin{sphinxuseclass}{cell_input}
\begin{sphinxVerbatim}[commandchars=\\\{\}]
\PYG{p}{(}
    \PYG{n}{surprises}
    \PYG{p}{[}\PYG{p}{[}\PYG{l+s+s1}{\PYGZsq{}}\PYG{l+s+s1}{Surprise(}\PYG{l+s+s1}{\PYGZpc{}}\PYG{l+s+s1}{)}\PYG{l+s+s1}{\PYGZsq{}}\PYG{p}{,} \PYG{l+s+s1}{\PYGZsq{}}\PYG{l+s+s1}{Returns}\PYG{l+s+s1}{\PYGZsq{}}\PYG{p}{]}\PYG{p}{]}
    \PYG{o}{.}\PYG{n}{mul}\PYG{p}{(}\PYG{l+m+mi}{100}\PYG{p}{)}
    \PYG{o}{.}\PYG{n}{plot}\PYG{p}{(}\PYG{n}{x}\PYG{o}{=}\PYG{l+s+s1}{\PYGZsq{}}\PYG{l+s+s1}{Surprise(}\PYG{l+s+s1}{\PYGZpc{}}\PYG{l+s+s1}{)}\PYG{l+s+s1}{\PYGZsq{}}\PYG{p}{,} \PYG{n}{y}\PYG{o}{=}\PYG{l+s+s1}{\PYGZsq{}}\PYG{l+s+s1}{Returns}\PYG{l+s+s1}{\PYGZsq{}}\PYG{p}{,} \PYG{n}{kind}\PYG{o}{=}\PYG{l+s+s1}{\PYGZsq{}}\PYG{l+s+s1}{scatter}\PYG{l+s+s1}{\PYGZsq{}}\PYG{p}{)}
\PYG{p}{)}
\PYG{n}{plt}\PYG{o}{.}\PYG{n}{xlabel}\PYG{p}{(}\PYG{l+s+s1}{\PYGZsq{}}\PYG{l+s+s1}{Earnings Suprise (}\PYG{l+s+s1}{\PYGZpc{}}\PYG{l+s+s1}{)}\PYG{l+s+s1}{\PYGZsq{}}\PYG{p}{)}
\PYG{n}{plt}\PYG{o}{.}\PYG{n}{ylabel}\PYG{p}{(}\PYG{l+s+s1}{\PYGZsq{}}\PYG{l+s+s1}{Announcement Return (}\PYG{l+s+s1}{\PYGZpc{}}\PYG{l+s+s1}{)}\PYG{l+s+s1}{\PYGZsq{}}\PYG{p}{)}

\PYG{n}{\PYGZus{}} \PYG{o}{=} \PYG{l+s+s1}{\PYGZsq{}}\PYG{l+s+s1}{ }\PYG{l+s+s1}{\PYGZsq{}}\PYG{o}{.}\PYG{n}{join}\PYG{p}{(}\PYG{n}{surprises}\PYG{o}{.}\PYG{n}{index}\PYG{o}{.}\PYG{n}{get\PYGZus{}level\PYGZus{}values}\PYG{p}{(}\PYG{l+s+s1}{\PYGZsq{}}\PYG{l+s+s1}{Ticker}\PYG{l+s+s1}{\PYGZsq{}}\PYG{p}{)}\PYG{o}{.}\PYG{n}{unique}\PYG{p}{(}\PYG{p}{)}\PYG{p}{)}
\PYG{n}{\PYGZus{}\PYGZus{}} \PYG{o}{=} \PYG{n}{surprises}\PYG{o}{.}\PYG{n}{index}\PYG{o}{.}\PYG{n}{get\PYGZus{}level\PYGZus{}values}\PYG{p}{(}\PYG{l+s+s1}{\PYGZsq{}}\PYG{l+s+s1}{Date}\PYG{l+s+s1}{\PYGZsq{}}\PYG{p}{)}
\PYG{n}{plt}\PYG{o}{.}\PYG{n}{title}\PYG{p}{(}\PYG{l+s+sa}{f}\PYG{l+s+s1}{\PYGZsq{}}\PYG{l+s+s1}{Earnings Announcements}\PYG{l+s+se}{\PYGZbs{}n}\PYG{l+s+s1}{ for }\PYG{l+s+si}{\PYGZob{}}\PYG{n}{\PYGZus{}}\PYG{l+s+si}{\PYGZcb{}}\PYG{l+s+se}{\PYGZbs{}n}\PYG{l+s+s1}{ from }\PYG{l+s+si}{\PYGZob{}}\PYG{n}{\PYGZus{}\PYGZus{}}\PYG{o}{.}\PYG{n}{min}\PYG{p}{(}\PYG{p}{)}\PYG{l+s+si}{:}\PYG{l+s+s1}{\PYGZpc{}B \PYGZpc{}Y}\PYG{l+s+si}{\PYGZcb{}}\PYG{l+s+s1}{ to }\PYG{l+s+si}{\PYGZob{}}\PYG{n}{\PYGZus{}\PYGZus{}}\PYG{o}{.}\PYG{n}{max}\PYG{p}{(}\PYG{p}{)}\PYG{l+s+si}{:}\PYG{l+s+s1}{\PYGZpc{}B \PYGZpc{}Y}\PYG{l+s+si}{\PYGZcb{}}\PYG{l+s+s1}{\PYGZsq{}}\PYG{p}{)}
\PYG{n}{plt}\PYG{o}{.}\PYG{n}{show}\PYG{p}{(}\PYG{p}{)}
\end{sphinxVerbatim}

\end{sphinxuseclass}\end{sphinxVerbatimInput}
\begin{sphinxVerbatimOutput}

\begin{sphinxuseclass}{cell_output}
\noindent\sphinxincludegraphics{{ddb8d2140998e68e00f41f701aac3893910abc69a528374a6462b885622c8173}.png}

\end{sphinxuseclass}\end{sphinxVerbatimOutput}

\end{sphinxuseclass}
\begin{sphinxuseclass}{cell}\begin{sphinxVerbatimInput}

\begin{sphinxuseclass}{cell_input}
\begin{sphinxVerbatim}[commandchars=\\\{\}]
\PYG{k+kn}{import} \PYG{n+nn}{seaborn} \PYG{k}{as} \PYG{n+nn}{sns}
\end{sphinxVerbatim}

\end{sphinxuseclass}\end{sphinxVerbatimInput}

\end{sphinxuseclass}
\begin{sphinxuseclass}{cell}\begin{sphinxVerbatimInput}

\begin{sphinxuseclass}{cell_input}
\begin{sphinxVerbatim}[commandchars=\\\{\}]
\PYG{n}{sns}\PYG{o}{.}\PYG{n}{regplot}\PYG{p}{(}
    \PYG{n}{x}\PYG{o}{=}\PYG{l+s+s1}{\PYGZsq{}}\PYG{l+s+s1}{Surprise(}\PYG{l+s+s1}{\PYGZpc{}}\PYG{l+s+s1}{)}\PYG{l+s+s1}{\PYGZsq{}}\PYG{p}{,}
    \PYG{n}{y}\PYG{o}{=} \PYG{l+s+s1}{\PYGZsq{}}\PYG{l+s+s1}{Returns}\PYG{l+s+s1}{\PYGZsq{}}\PYG{p}{,}
    \PYG{n}{data}\PYG{o}{=}\PYG{n}{surprises}\PYG{p}{[}\PYG{p}{[}\PYG{l+s+s1}{\PYGZsq{}}\PYG{l+s+s1}{Surprise(}\PYG{l+s+s1}{\PYGZpc{}}\PYG{l+s+s1}{)}\PYG{l+s+s1}{\PYGZsq{}}\PYG{p}{,} \PYG{l+s+s1}{\PYGZsq{}}\PYG{l+s+s1}{Returns}\PYG{l+s+s1}{\PYGZsq{}}\PYG{p}{]}\PYG{p}{]}\PYG{o}{.}\PYG{n}{mul}\PYG{p}{(}\PYG{l+m+mi}{100}\PYG{p}{)}
\PYG{p}{)}

\PYG{n}{plt}\PYG{o}{.}\PYG{n}{xlabel}\PYG{p}{(}\PYG{l+s+s1}{\PYGZsq{}}\PYG{l+s+s1}{Earnings Suprise (}\PYG{l+s+s1}{\PYGZpc{}}\PYG{l+s+s1}{)}\PYG{l+s+s1}{\PYGZsq{}}\PYG{p}{)}
\PYG{n}{plt}\PYG{o}{.}\PYG{n}{ylabel}\PYG{p}{(}\PYG{l+s+s1}{\PYGZsq{}}\PYG{l+s+s1}{Announcement Return (}\PYG{l+s+s1}{\PYGZpc{}}\PYG{l+s+s1}{)}\PYG{l+s+s1}{\PYGZsq{}}\PYG{p}{)}

\PYG{n}{\PYGZus{}} \PYG{o}{=} \PYG{l+s+s1}{\PYGZsq{}}\PYG{l+s+s1}{ }\PYG{l+s+s1}{\PYGZsq{}}\PYG{o}{.}\PYG{n}{join}\PYG{p}{(}\PYG{n}{surprises}\PYG{o}{.}\PYG{n}{index}\PYG{o}{.}\PYG{n}{get\PYGZus{}level\PYGZus{}values}\PYG{p}{(}\PYG{l+s+s1}{\PYGZsq{}}\PYG{l+s+s1}{Ticker}\PYG{l+s+s1}{\PYGZsq{}}\PYG{p}{)}\PYG{o}{.}\PYG{n}{unique}\PYG{p}{(}\PYG{p}{)}\PYG{p}{)}
\PYG{n}{\PYGZus{}\PYGZus{}} \PYG{o}{=} \PYG{n}{surprises}\PYG{o}{.}\PYG{n}{index}\PYG{o}{.}\PYG{n}{get\PYGZus{}level\PYGZus{}values}\PYG{p}{(}\PYG{l+s+s1}{\PYGZsq{}}\PYG{l+s+s1}{Date}\PYG{l+s+s1}{\PYGZsq{}}\PYG{p}{)}
\PYG{n}{plt}\PYG{o}{.}\PYG{n}{title}\PYG{p}{(}\PYG{l+s+sa}{f}\PYG{l+s+s1}{\PYGZsq{}}\PYG{l+s+s1}{Earnings Announcements}\PYG{l+s+se}{\PYGZbs{}n}\PYG{l+s+s1}{ for }\PYG{l+s+si}{\PYGZob{}}\PYG{n}{\PYGZus{}}\PYG{l+s+si}{\PYGZcb{}}\PYG{l+s+se}{\PYGZbs{}n}\PYG{l+s+s1}{ from }\PYG{l+s+si}{\PYGZob{}}\PYG{n}{\PYGZus{}\PYGZus{}}\PYG{o}{.}\PYG{n}{min}\PYG{p}{(}\PYG{p}{)}\PYG{l+s+si}{:}\PYG{l+s+s1}{\PYGZpc{}B \PYGZpc{}Y}\PYG{l+s+si}{\PYGZcb{}}\PYG{l+s+s1}{ to }\PYG{l+s+si}{\PYGZob{}}\PYG{n}{\PYGZus{}\PYGZus{}}\PYG{o}{.}\PYG{n}{max}\PYG{p}{(}\PYG{p}{)}\PYG{l+s+si}{:}\PYG{l+s+s1}{\PYGZpc{}B \PYGZpc{}Y}\PYG{l+s+si}{\PYGZcb{}}\PYG{l+s+s1}{\PYGZsq{}}\PYG{p}{)}
\PYG{n}{plt}\PYG{o}{.}\PYG{n}{show}\PYG{p}{(}\PYG{p}{)}
\end{sphinxVerbatim}

\end{sphinxuseclass}\end{sphinxVerbatimInput}
\begin{sphinxVerbatimOutput}

\begin{sphinxuseclass}{cell_output}
\noindent\sphinxincludegraphics{{2659346e1958bfdceffab52180fae7b67d8867e7bbcd6492e94d7882e034cb83}.png}

\end{sphinxuseclass}\end{sphinxVerbatimOutput}

\end{sphinxuseclass}
\begin{sphinxuseclass}{cell}\begin{sphinxVerbatimInput}

\begin{sphinxuseclass}{cell_input}
\begin{sphinxVerbatim}[commandchars=\\\{\}]
\PYG{n}{surprises}\PYG{p}{[}\PYG{l+s+s1}{\PYGZsq{}}\PYG{l+s+s1}{ESQ}\PYG{l+s+s1}{\PYGZsq{}}\PYG{p}{]} \PYG{o}{=} \PYG{n}{pd}\PYG{o}{.}\PYG{n}{qcut}\PYG{p}{(}\PYG{n}{x}\PYG{o}{=}\PYG{n}{surprises}\PYG{p}{[}\PYG{l+s+s1}{\PYGZsq{}}\PYG{l+s+s1}{Surprise(}\PYG{l+s+s1}{\PYGZpc{}}\PYG{l+s+s1}{)}\PYG{l+s+s1}{\PYGZsq{}}\PYG{p}{]}\PYG{p}{,} \PYG{n}{q}\PYG{o}{=}\PYG{l+m+mi}{5}\PYG{p}{,} \PYG{n}{labels}\PYG{o}{=}\PYG{k+kc}{False}\PYG{p}{)}
\end{sphinxVerbatim}

\end{sphinxuseclass}\end{sphinxVerbatimInput}

\end{sphinxuseclass}
\begin{sphinxuseclass}{cell}\begin{sphinxVerbatimInput}

\begin{sphinxuseclass}{cell_input}
\begin{sphinxVerbatim}[commandchars=\\\{\}]
\PYG{n}{sns}\PYG{o}{.}\PYG{n}{barplot}\PYG{p}{(}
    \PYG{n}{x}\PYG{o}{=}\PYG{l+s+s1}{\PYGZsq{}}\PYG{l+s+s1}{ESQ}\PYG{l+s+s1}{\PYGZsq{}}\PYG{p}{,}
    \PYG{n}{y}\PYG{o}{=} \PYG{l+s+s1}{\PYGZsq{}}\PYG{l+s+s1}{Returns}\PYG{l+s+s1}{\PYGZsq{}}\PYG{p}{,}
    \PYG{n}{data}\PYG{o}{=}\PYG{p}{(}
        \PYG{n}{surprises}
        \PYG{p}{[}\PYG{p}{[}\PYG{l+s+s1}{\PYGZsq{}}\PYG{l+s+s1}{Surprise(}\PYG{l+s+s1}{\PYGZpc{}}\PYG{l+s+s1}{)}\PYG{l+s+s1}{\PYGZsq{}}\PYG{p}{,} \PYG{l+s+s1}{\PYGZsq{}}\PYG{l+s+s1}{Returns}\PYG{l+s+s1}{\PYGZsq{}}\PYG{p}{]}\PYG{p}{]}
        \PYG{o}{.}\PYG{n}{mul}\PYG{p}{(}\PYG{l+m+mi}{100}\PYG{p}{)}
        \PYG{o}{.}\PYG{n}{assign}\PYG{p}{(}\PYG{n}{ESQ} \PYG{o}{=} \PYG{k}{lambda} \PYG{n}{x}\PYG{p}{:} \PYG{n}{pd}\PYG{o}{.}\PYG{n}{qcut}\PYG{p}{(}\PYG{n}{x}\PYG{o}{=}\PYG{n}{x}\PYG{p}{[}\PYG{l+s+s1}{\PYGZsq{}}\PYG{l+s+s1}{Surprise(}\PYG{l+s+s1}{\PYGZpc{}}\PYG{l+s+s1}{)}\PYG{l+s+s1}{\PYGZsq{}}\PYG{p}{]}\PYG{p}{,} \PYG{n}{q}\PYG{o}{=}\PYG{l+m+mi}{5}\PYG{p}{,} \PYG{n}{labels}\PYG{o}{=}\PYG{k+kc}{False}\PYG{p}{)}\PYG{p}{)}
    \PYG{p}{)}
\PYG{p}{)}

\PYG{n}{plt}\PYG{o}{.}\PYG{n}{xlabel}\PYG{p}{(}\PYG{l+s+s1}{\PYGZsq{}}\PYG{l+s+s1}{Earnings Suprise Portfolio}\PYG{l+s+s1}{\PYGZsq{}}\PYG{p}{)}
\PYG{n}{plt}\PYG{o}{.}\PYG{n}{ylabel}\PYG{p}{(}\PYG{l+s+s1}{\PYGZsq{}}\PYG{l+s+s1}{Announcement Return (}\PYG{l+s+s1}{\PYGZpc{}}\PYG{l+s+s1}{)}\PYG{l+s+s1}{\PYGZsq{}}\PYG{p}{)}

\PYG{n}{\PYGZus{}} \PYG{o}{=} \PYG{l+s+s1}{\PYGZsq{}}\PYG{l+s+s1}{ }\PYG{l+s+s1}{\PYGZsq{}}\PYG{o}{.}\PYG{n}{join}\PYG{p}{(}\PYG{n}{surprises}\PYG{o}{.}\PYG{n}{index}\PYG{o}{.}\PYG{n}{get\PYGZus{}level\PYGZus{}values}\PYG{p}{(}\PYG{l+s+s1}{\PYGZsq{}}\PYG{l+s+s1}{Ticker}\PYG{l+s+s1}{\PYGZsq{}}\PYG{p}{)}\PYG{o}{.}\PYG{n}{unique}\PYG{p}{(}\PYG{p}{)}\PYG{p}{)}
\PYG{n}{\PYGZus{}\PYGZus{}} \PYG{o}{=} \PYG{n}{surprises}\PYG{o}{.}\PYG{n}{index}\PYG{o}{.}\PYG{n}{get\PYGZus{}level\PYGZus{}values}\PYG{p}{(}\PYG{l+s+s1}{\PYGZsq{}}\PYG{l+s+s1}{Date}\PYG{l+s+s1}{\PYGZsq{}}\PYG{p}{)}
\PYG{n}{plt}\PYG{o}{.}\PYG{n}{title}\PYG{p}{(}\PYG{l+s+sa}{f}\PYG{l+s+s1}{\PYGZsq{}}\PYG{l+s+s1}{Earnings Announcements}\PYG{l+s+se}{\PYGZbs{}n}\PYG{l+s+s1}{ for }\PYG{l+s+si}{\PYGZob{}}\PYG{n}{\PYGZus{}}\PYG{l+s+si}{\PYGZcb{}}\PYG{l+s+se}{\PYGZbs{}n}\PYG{l+s+s1}{ from }\PYG{l+s+si}{\PYGZob{}}\PYG{n}{\PYGZus{}\PYGZus{}}\PYG{o}{.}\PYG{n}{min}\PYG{p}{(}\PYG{p}{)}\PYG{l+s+si}{:}\PYG{l+s+s1}{\PYGZpc{}B \PYGZpc{}Y}\PYG{l+s+si}{\PYGZcb{}}\PYG{l+s+s1}{ to }\PYG{l+s+si}{\PYGZob{}}\PYG{n}{\PYGZus{}\PYGZus{}}\PYG{o}{.}\PYG{n}{max}\PYG{p}{(}\PYG{p}{)}\PYG{l+s+si}{:}\PYG{l+s+s1}{\PYGZpc{}B \PYGZpc{}Y}\PYG{l+s+si}{\PYGZcb{}}\PYG{l+s+s1}{\PYGZsq{}}\PYG{p}{)}
\PYG{n}{plt}\PYG{o}{.}\PYG{n}{show}\PYG{p}{(}\PYG{p}{)}
\end{sphinxVerbatim}

\end{sphinxuseclass}\end{sphinxVerbatimInput}
\begin{sphinxVerbatimOutput}

\begin{sphinxuseclass}{cell_output}
\noindent\sphinxincludegraphics{{c4d4d23a0843ae749360981cd673e7eea2624bc04efa2986c45a1b9f2f328dde}.png}

\end{sphinxuseclass}\end{sphinxVerbatimOutput}

\end{sphinxuseclass}
\sphinxAtStartPar
\sphinxstyleemphasis{\sphinxstylestrong{There is a positive relation between announcment returns and earnings surprises!}}
Of course, to say more we need more data and to control for market movements, but this analaysis is a start!


\subsubsection{Repeat the earnings exercise with the S\&P 100 stocks}
\label{\detokenize{mckinney_08_practice_04:repeat-the-earnings-exercise-with-the-s-p-100-stocks}}
\begin{sphinxuseclass}{cell}\begin{sphinxVerbatimInput}

\begin{sphinxuseclass}{cell_input}
\begin{sphinxVerbatim}[commandchars=\\\{\}]
\PYG{n}{wiki} \PYG{o}{=} \PYG{n}{pd}\PYG{o}{.}\PYG{n}{read\PYGZus{}html}\PYG{p}{(}\PYG{l+s+s1}{\PYGZsq{}}\PYG{l+s+s1}{https://en.wikipedia.org/wiki/S}\PYG{l+s+s1}{\PYGZpc{}}\PYG{l+s+s1}{26P\PYGZus{}100}\PYG{l+s+s1}{\PYGZsq{}}\PYG{p}{)}
\end{sphinxVerbatim}

\end{sphinxuseclass}\end{sphinxVerbatimInput}

\end{sphinxuseclass}
\begin{sphinxuseclass}{cell}\begin{sphinxVerbatimInput}

\begin{sphinxuseclass}{cell_input}
\begin{sphinxVerbatim}[commandchars=\\\{\}]
\PYG{n}{symbols} \PYG{o}{=} \PYG{n}{wiki}\PYG{p}{[}\PYG{l+m+mi}{2}\PYG{p}{]}\PYG{p}{[}\PYG{l+s+s1}{\PYGZsq{}}\PYG{l+s+s1}{Symbol}\PYG{l+s+s1}{\PYGZsq{}}\PYG{p}{]}\PYG{o}{.}\PYG{n}{str}\PYG{o}{.}\PYG{n}{replace}\PYG{p}{(}\PYG{l+s+s1}{\PYGZsq{}}\PYG{l+s+s1}{.}\PYG{l+s+s1}{\PYGZsq{}}\PYG{p}{,} \PYG{l+s+s1}{\PYGZsq{}}\PYG{l+s+s1}{\PYGZhy{}}\PYG{l+s+s1}{\PYGZsq{}}\PYG{p}{,} \PYG{n}{regex}\PYG{o}{=}\PYG{k+kc}{False}\PYG{p}{)}\PYG{o}{.}\PYG{n}{to\PYGZus{}list}\PYG{p}{(}\PYG{p}{)}
\end{sphinxVerbatim}

\end{sphinxuseclass}\end{sphinxVerbatimInput}

\end{sphinxuseclass}
\begin{sphinxuseclass}{cell}\begin{sphinxVerbatimInput}

\begin{sphinxuseclass}{cell_input}
\begin{sphinxVerbatim}[commandchars=\\\{\}]
\PYG{n}{tickers\PYGZus{}2} \PYG{o}{=} \PYG{n}{yf}\PYG{o}{.}\PYG{n}{Tickers}\PYG{p}{(}\PYG{n}{tickers}\PYG{o}{=}\PYG{n}{symbols}\PYG{p}{,} \PYG{n}{session}\PYG{o}{=}\PYG{n}{session}\PYG{p}{)}
\end{sphinxVerbatim}

\end{sphinxuseclass}\end{sphinxVerbatimInput}

\end{sphinxuseclass}
\begin{sphinxuseclass}{cell}\begin{sphinxVerbatimInput}

\begin{sphinxuseclass}{cell_input}
\begin{sphinxVerbatim}[commandchars=\\\{\}]
\PYG{n}{returns\PYGZus{}2} \PYG{o}{=} \PYG{p}{(}
    \PYG{n}{tickers\PYGZus{}2}
    \PYG{o}{.}\PYG{n}{history}\PYG{p}{(}\PYG{n}{period}\PYG{o}{=}\PYG{l+s+s1}{\PYGZsq{}}\PYG{l+s+s1}{max}\PYG{l+s+s1}{\PYGZsq{}}\PYG{p}{,} \PYG{n}{auto\PYGZus{}adjust}\PYG{o}{=}\PYG{k+kc}{False}\PYG{p}{,} \PYG{n}{progress}\PYG{o}{=}\PYG{k+kc}{False}\PYG{p}{)}
    \PYG{o}{.}\PYG{n}{rename\PYGZus{}axis}\PYG{p}{(}\PYG{n}{columns}\PYG{o}{=}\PYG{p}{[}\PYG{l+s+s1}{\PYGZsq{}}\PYG{l+s+s1}{Variable}\PYG{l+s+s1}{\PYGZsq{}}\PYG{p}{,} \PYG{l+s+s1}{\PYGZsq{}}\PYG{l+s+s1}{Ticker}\PYG{l+s+s1}{\PYGZsq{}}\PYG{p}{]}\PYG{p}{)}
    \PYG{p}{[}\PYG{l+s+s1}{\PYGZsq{}}\PYG{l+s+s1}{Adj Close}\PYG{l+s+s1}{\PYGZsq{}}\PYG{p}{]}
    \PYG{o}{.}\PYG{n}{pct\PYGZus{}change}\PYG{p}{(}\PYG{p}{)}
    \PYG{o}{.}\PYG{n}{assign}\PYG{p}{(}\PYG{n}{Date}\PYG{o}{=}\PYG{k}{lambda} \PYG{n}{x}\PYG{p}{:} \PYG{n}{x}\PYG{o}{.}\PYG{n}{index}\PYG{o}{.}\PYG{n}{tz\PYGZus{}localize}\PYG{p}{(}\PYG{l+s+s1}{\PYGZsq{}}\PYG{l+s+s1}{America/New\PYGZus{}York}\PYG{l+s+s1}{\PYGZsq{}}\PYG{p}{)} \PYG{o}{+} \PYG{n}{pd}\PYG{o}{.}\PYG{n}{to\PYGZus{}timedelta}\PYG{p}{(}\PYG{l+m+mi}{16}\PYG{p}{,} \PYG{n}{unit}\PYG{o}{=}\PYG{l+s+s1}{\PYGZsq{}}\PYG{l+s+s1}{H}\PYG{l+s+s1}{\PYGZsq{}}\PYG{p}{)}\PYG{p}{)}
    \PYG{o}{.}\PYG{n}{set\PYGZus{}index}\PYG{p}{(}\PYG{l+s+s1}{\PYGZsq{}}\PYG{l+s+s1}{Date}\PYG{l+s+s1}{\PYGZsq{}}\PYG{p}{)}
\PYG{p}{)}

\PYG{n}{returns\PYGZus{}2}\PYG{o}{.}\PYG{n}{head}\PYG{p}{(}\PYG{p}{)}
\end{sphinxVerbatim}

\end{sphinxuseclass}\end{sphinxVerbatimInput}
\begin{sphinxVerbatimOutput}

\begin{sphinxuseclass}{cell_output}
\begin{sphinxVerbatim}[commandchars=\\\{\}]
Ticker                     AAPL  ABBV  ABT  ACN  ADBE  AIG  AMD  AMGN  AMT  \PYGZbs{}
Date                                                                         
1962\PYGZhy{}01\PYGZhy{}02 16:00:00\PYGZhy{}05:00   NaN   NaN  NaN  NaN   NaN  NaN  NaN   NaN  NaN   
1962\PYGZhy{}01\PYGZhy{}03 16:00:00\PYGZhy{}05:00   NaN   NaN  NaN  NaN   NaN  NaN  NaN   NaN  NaN   
1962\PYGZhy{}01\PYGZhy{}04 16:00:00\PYGZhy{}05:00   NaN   NaN  NaN  NaN   NaN  NaN  NaN   NaN  NaN   
1962\PYGZhy{}01\PYGZhy{}05 16:00:00\PYGZhy{}05:00   NaN   NaN  NaN  NaN   NaN  NaN  NaN   NaN  NaN   
1962\PYGZhy{}01\PYGZhy{}08 16:00:00\PYGZhy{}05:00   NaN   NaN  NaN  NaN   NaN  NaN  NaN   NaN  NaN   

Ticker                     AMZN  ...  UNH  UNP  UPS  USB   V  VZ  WBA  WFC  \PYGZbs{}
Date                             ...                                         
1962\PYGZhy{}01\PYGZhy{}02 16:00:00\PYGZhy{}05:00   NaN  ...  NaN  NaN  NaN  NaN NaN NaN  NaN  NaN   
1962\PYGZhy{}01\PYGZhy{}03 16:00:00\PYGZhy{}05:00   NaN  ...  NaN  NaN  NaN  NaN NaN NaN  NaN  NaN   
1962\PYGZhy{}01\PYGZhy{}04 16:00:00\PYGZhy{}05:00   NaN  ...  NaN  NaN  NaN  NaN NaN NaN  NaN  NaN   
1962\PYGZhy{}01\PYGZhy{}05 16:00:00\PYGZhy{}05:00   NaN  ...  NaN  NaN  NaN  NaN NaN NaN  NaN  NaN   
1962\PYGZhy{}01\PYGZhy{}08 16:00:00\PYGZhy{}05:00   NaN  ...  NaN  NaN  NaN  NaN NaN NaN  NaN  NaN   

Ticker                     WMT     XOM  
Date                                    
1962\PYGZhy{}01\PYGZhy{}02 16:00:00\PYGZhy{}05:00  NaN     NaN  
1962\PYGZhy{}01\PYGZhy{}03 16:00:00\PYGZhy{}05:00  NaN  0.0149  
1962\PYGZhy{}01\PYGZhy{}04 16:00:00\PYGZhy{}05:00  NaN  0.0024  
1962\PYGZhy{}01\PYGZhy{}05 16:00:00\PYGZhy{}05:00  NaN \PYGZhy{}0.0219  
1962\PYGZhy{}01\PYGZhy{}08 16:00:00\PYGZhy{}05:00  NaN \PYGZhy{}0.0025  

[5 rows x 101 columns]
\end{sphinxVerbatim}

\end{sphinxuseclass}\end{sphinxVerbatimOutput}

\end{sphinxuseclass}
\begin{sphinxuseclass}{cell}\begin{sphinxVerbatimInput}

\begin{sphinxuseclass}{cell_input}
\begin{sphinxVerbatim}[commandchars=\\\{\}]
\PYG{n}{earnings\PYGZus{}2} \PYG{o}{=} \PYG{p}{(}
    \PYG{n}{pd}\PYG{o}{.}\PYG{n}{concat}\PYG{p}{(}
        \PYG{n}{objs}\PYG{o}{=}\PYG{p}{[}\PYG{n}{tickers\PYGZus{}2}\PYG{o}{.}\PYG{n}{tickers}\PYG{p}{[}\PYG{n}{t}\PYG{p}{]}\PYG{o}{.}\PYG{n}{earnings\PYGZus{}dates} \PYG{k}{for} \PYG{n}{t} \PYG{o+ow}{in} \PYG{n}{tickers\PYGZus{}2}\PYG{o}{.}\PYG{n}{tickers}\PYG{p}{]}\PYG{p}{,}
        \PYG{n}{keys}\PYG{o}{=}\PYG{n}{tickers\PYGZus{}2}\PYG{o}{.}\PYG{n}{tickers}\PYG{p}{,}
        \PYG{n}{names}\PYG{o}{=}\PYG{p}{[}\PYG{l+s+s1}{\PYGZsq{}}\PYG{l+s+s1}{Ticker}\PYG{l+s+s1}{\PYGZsq{}}\PYG{p}{,} \PYG{l+s+s1}{\PYGZsq{}}\PYG{l+s+s1}{Date}\PYG{l+s+s1}{\PYGZsq{}}\PYG{p}{]}
    \PYG{p}{)}
    \PYG{o}{.}\PYG{n}{rename\PYGZus{}axis}\PYG{p}{(}\PYG{n}{columns}\PYG{o}{=}\PYG{l+s+s1}{\PYGZsq{}}\PYG{l+s+s1}{Variable}\PYG{l+s+s1}{\PYGZsq{}}\PYG{p}{)}
\PYG{p}{)}
\end{sphinxVerbatim}

\end{sphinxuseclass}\end{sphinxVerbatimInput}

\end{sphinxuseclass}
\begin{sphinxuseclass}{cell}\begin{sphinxVerbatimInput}

\begin{sphinxuseclass}{cell_input}
\begin{sphinxVerbatim}[commandchars=\\\{\}]
\PYG{n}{surprises\PYGZus{}2} \PYG{o}{=} \PYG{p}{(}
    \PYG{n}{pd}\PYG{o}{.}\PYG{n}{merge\PYGZus{}asof}\PYG{p}{(}
        \PYG{n}{left}\PYG{o}{=}\PYG{n}{earnings\PYGZus{}2}\PYG{o}{.}\PYG{n}{sort\PYGZus{}index}\PYG{p}{(}\PYG{n}{level}\PYG{o}{=}\PYG{p}{[}\PYG{l+s+s1}{\PYGZsq{}}\PYG{l+s+s1}{Date}\PYG{l+s+s1}{\PYGZsq{}}\PYG{p}{,} \PYG{l+s+s1}{\PYGZsq{}}\PYG{l+s+s1}{Ticker}\PYG{l+s+s1}{\PYGZsq{}}\PYG{p}{]}\PYG{p}{)}\PYG{p}{,}
        \PYG{n}{right}\PYG{o}{=}\PYG{n}{returns\PYGZus{}2}\PYG{o}{.}\PYG{n}{stack}\PYG{p}{(}\PYG{p}{)}\PYG{o}{.}\PYG{n}{to\PYGZus{}frame}\PYG{p}{(}\PYG{l+s+s1}{\PYGZsq{}}\PYG{l+s+s1}{Returns}\PYG{l+s+s1}{\PYGZsq{}}\PYG{p}{)}\PYG{o}{.}\PYG{n}{swaplevel}\PYG{p}{(}\PYG{p}{)}\PYG{o}{.}\PYG{n}{sort\PYGZus{}index}\PYG{p}{(}\PYG{n}{level}\PYG{o}{=}\PYG{p}{[}\PYG{l+s+s1}{\PYGZsq{}}\PYG{l+s+s1}{Date}\PYG{l+s+s1}{\PYGZsq{}}\PYG{p}{,} \PYG{l+s+s1}{\PYGZsq{}}\PYG{l+s+s1}{Ticker}\PYG{l+s+s1}{\PYGZsq{}}\PYG{p}{]}\PYG{p}{)}\PYG{p}{,}
        \PYG{n}{on}\PYG{o}{=}\PYG{l+s+s1}{\PYGZsq{}}\PYG{l+s+s1}{Date}\PYG{l+s+s1}{\PYGZsq{}}\PYG{p}{,}
        \PYG{n}{by}\PYG{o}{=}\PYG{l+s+s1}{\PYGZsq{}}\PYG{l+s+s1}{Ticker}\PYG{l+s+s1}{\PYGZsq{}}\PYG{p}{,}
        \PYG{n}{direction}\PYG{o}{=}\PYG{l+s+s1}{\PYGZsq{}}\PYG{l+s+s1}{forward}\PYG{l+s+s1}{\PYGZsq{}}\PYG{p}{,}
        \PYG{n}{allow\PYGZus{}exact\PYGZus{}matches}\PYG{o}{=}\PYG{k+kc}{False}
    \PYG{p}{)}
    \PYG{o}{.}\PYG{n}{dropna}\PYG{p}{(}\PYG{p}{)}
    \PYG{o}{.}\PYG{n}{set\PYGZus{}index}\PYG{p}{(}\PYG{p}{[}\PYG{l+s+s1}{\PYGZsq{}}\PYG{l+s+s1}{Date}\PYG{l+s+s1}{\PYGZsq{}}\PYG{p}{,} \PYG{l+s+s1}{\PYGZsq{}}\PYG{l+s+s1}{Ticker}\PYG{l+s+s1}{\PYGZsq{}}\PYG{p}{]}\PYG{p}{)}
\PYG{p}{)}
\end{sphinxVerbatim}

\end{sphinxuseclass}\end{sphinxVerbatimInput}

\end{sphinxuseclass}
\begin{sphinxuseclass}{cell}\begin{sphinxVerbatimInput}

\begin{sphinxuseclass}{cell_input}
\begin{sphinxVerbatim}[commandchars=\\\{\}]
\PYG{n}{sns}\PYG{o}{.}\PYG{n}{barplot}\PYG{p}{(}
    \PYG{n}{x}\PYG{o}{=}\PYG{l+s+s1}{\PYGZsq{}}\PYG{l+s+s1}{ESQ}\PYG{l+s+s1}{\PYGZsq{}}\PYG{p}{,}
    \PYG{n}{y}\PYG{o}{=} \PYG{l+s+s1}{\PYGZsq{}}\PYG{l+s+s1}{Returns}\PYG{l+s+s1}{\PYGZsq{}}\PYG{p}{,}
    \PYG{n}{data}\PYG{o}{=}\PYG{p}{(}
        \PYG{n}{surprises\PYGZus{}2}
        \PYG{p}{[}\PYG{p}{[}\PYG{l+s+s1}{\PYGZsq{}}\PYG{l+s+s1}{Surprise(}\PYG{l+s+s1}{\PYGZpc{}}\PYG{l+s+s1}{)}\PYG{l+s+s1}{\PYGZsq{}}\PYG{p}{,} \PYG{l+s+s1}{\PYGZsq{}}\PYG{l+s+s1}{Returns}\PYG{l+s+s1}{\PYGZsq{}}\PYG{p}{]}\PYG{p}{]}
        \PYG{o}{.}\PYG{n}{mul}\PYG{p}{(}\PYG{l+m+mi}{100}\PYG{p}{)}
        \PYG{o}{.}\PYG{n}{assign}\PYG{p}{(}\PYG{n}{ESQ} \PYG{o}{=} \PYG{k}{lambda} \PYG{n}{x}\PYG{p}{:} \PYG{n}{pd}\PYG{o}{.}\PYG{n}{qcut}\PYG{p}{(}\PYG{n}{x}\PYG{o}{=}\PYG{n}{x}\PYG{p}{[}\PYG{l+s+s1}{\PYGZsq{}}\PYG{l+s+s1}{Surprise(}\PYG{l+s+s1}{\PYGZpc{}}\PYG{l+s+s1}{)}\PYG{l+s+s1}{\PYGZsq{}}\PYG{p}{]}\PYG{p}{,} \PYG{n}{q}\PYG{o}{=}\PYG{l+m+mi}{5}\PYG{p}{,} \PYG{n}{labels}\PYG{o}{=}\PYG{k+kc}{False}\PYG{p}{)}\PYG{p}{)}
    \PYG{p}{)}
\PYG{p}{)}

\PYG{n}{plt}\PYG{o}{.}\PYG{n}{xlabel}\PYG{p}{(}\PYG{l+s+s1}{\PYGZsq{}}\PYG{l+s+s1}{Earnings Suprise Portfolio}\PYG{l+s+s1}{\PYGZsq{}}\PYG{p}{)}
\PYG{n}{plt}\PYG{o}{.}\PYG{n}{ylabel}\PYG{p}{(}\PYG{l+s+s1}{\PYGZsq{}}\PYG{l+s+s1}{Announcement Return (}\PYG{l+s+s1}{\PYGZpc{}}\PYG{l+s+s1}{)}\PYG{l+s+s1}{\PYGZsq{}}\PYG{p}{)}

\PYG{n}{\PYGZus{}\PYGZus{}} \PYG{o}{=} \PYG{n}{surprises\PYGZus{}2}\PYG{o}{.}\PYG{n}{index}\PYG{o}{.}\PYG{n}{get\PYGZus{}level\PYGZus{}values}\PYG{p}{(}\PYG{l+s+s1}{\PYGZsq{}}\PYG{l+s+s1}{Date}\PYG{l+s+s1}{\PYGZsq{}}\PYG{p}{)}
\PYG{n}{plt}\PYG{o}{.}\PYG{n}{title}\PYG{p}{(}\PYG{l+s+sa}{f}\PYG{l+s+s1}{\PYGZsq{}}\PYG{l+s+s1}{Earnings Announcements for S\PYGZam{}P 100 Stocks }\PYG{l+s+se}{\PYGZbs{}n}\PYG{l+s+s1}{ from }\PYG{l+s+si}{\PYGZob{}}\PYG{n}{\PYGZus{}\PYGZus{}}\PYG{o}{.}\PYG{n}{min}\PYG{p}{(}\PYG{p}{)}\PYG{l+s+si}{:}\PYG{l+s+s1}{\PYGZpc{}B \PYGZpc{}Y}\PYG{l+s+si}{\PYGZcb{}}\PYG{l+s+s1}{ to }\PYG{l+s+si}{\PYGZob{}}\PYG{n}{\PYGZus{}\PYGZus{}}\PYG{o}{.}\PYG{n}{max}\PYG{p}{(}\PYG{p}{)}\PYG{l+s+si}{:}\PYG{l+s+s1}{\PYGZpc{}B \PYGZpc{}Y}\PYG{l+s+si}{\PYGZcb{}}\PYG{l+s+s1}{\PYGZsq{}}\PYG{p}{)}
\PYG{n}{plt}\PYG{o}{.}\PYG{n}{show}\PYG{p}{(}\PYG{p}{)}
\end{sphinxVerbatim}

\end{sphinxuseclass}\end{sphinxVerbatimInput}
\begin{sphinxVerbatimOutput}

\begin{sphinxuseclass}{cell_output}
\noindent\sphinxincludegraphics{{81ce56b56ef7415cd6a8c5cff6aa043d4c00ed3e73f278041ecda4c5b0c33692}.png}

\end{sphinxuseclass}\end{sphinxVerbatimOutput}

\end{sphinxuseclass}

\subsubsection{Repeat the earnings exercise with \sphinxstyleemphasis{excess returns} of the S\&P 100 Stocks}
\label{\detokenize{mckinney_08_practice_04:repeat-the-earnings-exercise-with-excess-returns-of-the-s-p-100-stocks}}
\sphinxAtStartPar
Excess returns are returns minus market returns.
We need to add a timezone and the closing time to the market return from Fama and French.

\begin{sphinxuseclass}{cell}\begin{sphinxVerbatimInput}

\begin{sphinxuseclass}{cell_input}
\begin{sphinxVerbatim}[commandchars=\\\{\}]
\PYG{n}{Mkt} \PYG{o}{=} \PYG{n}{ff}\PYG{p}{[}\PYG{l+s+s1}{\PYGZsq{}}\PYG{l+s+s1}{Mkt\PYGZhy{}RF}\PYG{l+s+s1}{\PYGZsq{}}\PYG{p}{]}\PYG{o}{.}\PYG{n}{add}\PYG{p}{(}\PYG{n}{ff}\PYG{p}{[}\PYG{l+s+s1}{\PYGZsq{}}\PYG{l+s+s1}{RF}\PYG{l+s+s1}{\PYGZsq{}}\PYG{p}{]}\PYG{p}{)}
\PYG{n}{Mkt}\PYG{o}{.}\PYG{n}{index} \PYG{o}{=} \PYG{n}{Mkt}\PYG{o}{.}\PYG{n}{index}\PYG{o}{.}\PYG{n}{tz\PYGZus{}localize}\PYG{p}{(}\PYG{l+s+s1}{\PYGZsq{}}\PYG{l+s+s1}{America/New\PYGZus{}York}\PYG{l+s+s1}{\PYGZsq{}}\PYG{p}{)} \PYG{o}{+} \PYG{n}{pd}\PYG{o}{.}\PYG{n}{to\PYGZus{}timedelta}\PYG{p}{(}\PYG{l+m+mi}{16}\PYG{p}{,} \PYG{n}{unit}\PYG{o}{=}\PYG{l+s+s1}{\PYGZsq{}}\PYG{l+s+s1}{H}\PYG{l+s+s1}{\PYGZsq{}}\PYG{p}{)}
\PYG{n}{returns\PYGZus{}3} \PYG{o}{=} \PYG{n}{returns\PYGZus{}2}\PYG{o}{.}\PYG{n}{sub}\PYG{p}{(}\PYG{n}{Mkt}\PYG{p}{,} \PYG{n}{axis}\PYG{o}{=}\PYG{l+m+mi}{0}\PYG{p}{)}
\end{sphinxVerbatim}

\end{sphinxuseclass}\end{sphinxVerbatimInput}

\end{sphinxuseclass}
\begin{sphinxuseclass}{cell}\begin{sphinxVerbatimInput}

\begin{sphinxuseclass}{cell_input}
\begin{sphinxVerbatim}[commandchars=\\\{\}]
\PYG{n}{surprises\PYGZus{}3} \PYG{o}{=} \PYG{p}{(}
    \PYG{n}{pd}\PYG{o}{.}\PYG{n}{merge\PYGZus{}asof}\PYG{p}{(}
        \PYG{n}{left}\PYG{o}{=}\PYG{n}{earnings\PYGZus{}2}\PYG{o}{.}\PYG{n}{sort\PYGZus{}index}\PYG{p}{(}\PYG{n}{level}\PYG{o}{=}\PYG{p}{[}\PYG{l+s+s1}{\PYGZsq{}}\PYG{l+s+s1}{Date}\PYG{l+s+s1}{\PYGZsq{}}\PYG{p}{,} \PYG{l+s+s1}{\PYGZsq{}}\PYG{l+s+s1}{Ticker}\PYG{l+s+s1}{\PYGZsq{}}\PYG{p}{]}\PYG{p}{)}\PYG{p}{,}
        \PYG{n}{right}\PYG{o}{=}\PYG{n}{returns\PYGZus{}3}\PYG{o}{.}\PYG{n}{stack}\PYG{p}{(}\PYG{p}{)}\PYG{o}{.}\PYG{n}{to\PYGZus{}frame}\PYG{p}{(}\PYG{l+s+s1}{\PYGZsq{}}\PYG{l+s+s1}{Excess Returns}\PYG{l+s+s1}{\PYGZsq{}}\PYG{p}{)}\PYG{o}{.}\PYG{n}{swaplevel}\PYG{p}{(}\PYG{p}{)}\PYG{o}{.}\PYG{n}{sort\PYGZus{}index}\PYG{p}{(}\PYG{n}{level}\PYG{o}{=}\PYG{p}{[}\PYG{l+s+s1}{\PYGZsq{}}\PYG{l+s+s1}{Date}\PYG{l+s+s1}{\PYGZsq{}}\PYG{p}{,} \PYG{l+s+s1}{\PYGZsq{}}\PYG{l+s+s1}{Ticker}\PYG{l+s+s1}{\PYGZsq{}}\PYG{p}{]}\PYG{p}{)}\PYG{p}{,}
        \PYG{n}{on}\PYG{o}{=}\PYG{l+s+s1}{\PYGZsq{}}\PYG{l+s+s1}{Date}\PYG{l+s+s1}{\PYGZsq{}}\PYG{p}{,}
        \PYG{n}{by}\PYG{o}{=}\PYG{l+s+s1}{\PYGZsq{}}\PYG{l+s+s1}{Ticker}\PYG{l+s+s1}{\PYGZsq{}}\PYG{p}{,}
        \PYG{n}{direction}\PYG{o}{=}\PYG{l+s+s1}{\PYGZsq{}}\PYG{l+s+s1}{forward}\PYG{l+s+s1}{\PYGZsq{}}\PYG{p}{,}
        \PYG{n}{allow\PYGZus{}exact\PYGZus{}matches}\PYG{o}{=}\PYG{k+kc}{False}
    \PYG{p}{)}
    \PYG{o}{.}\PYG{n}{dropna}\PYG{p}{(}\PYG{p}{)}
    \PYG{o}{.}\PYG{n}{set\PYGZus{}index}\PYG{p}{(}\PYG{p}{[}\PYG{l+s+s1}{\PYGZsq{}}\PYG{l+s+s1}{Date}\PYG{l+s+s1}{\PYGZsq{}}\PYG{p}{,} \PYG{l+s+s1}{\PYGZsq{}}\PYG{l+s+s1}{Ticker}\PYG{l+s+s1}{\PYGZsq{}}\PYG{p}{]}\PYG{p}{)}
\PYG{p}{)}
\end{sphinxVerbatim}

\end{sphinxuseclass}\end{sphinxVerbatimInput}

\end{sphinxuseclass}
\begin{sphinxuseclass}{cell}\begin{sphinxVerbatimInput}

\begin{sphinxuseclass}{cell_input}
\begin{sphinxVerbatim}[commandchars=\\\{\}]
\PYG{n}{sns}\PYG{o}{.}\PYG{n}{barplot}\PYG{p}{(}
    \PYG{n}{x}\PYG{o}{=}\PYG{l+s+s1}{\PYGZsq{}}\PYG{l+s+s1}{ESQ}\PYG{l+s+s1}{\PYGZsq{}}\PYG{p}{,}
    \PYG{n}{y}\PYG{o}{=}\PYG{l+s+s1}{\PYGZsq{}}\PYG{l+s+s1}{Excess Returns}\PYG{l+s+s1}{\PYGZsq{}}\PYG{p}{,}
    \PYG{n}{data}\PYG{o}{=}\PYG{p}{(}
        \PYG{n}{surprises\PYGZus{}3}
        \PYG{p}{[}\PYG{p}{[}\PYG{l+s+s1}{\PYGZsq{}}\PYG{l+s+s1}{Surprise(}\PYG{l+s+s1}{\PYGZpc{}}\PYG{l+s+s1}{)}\PYG{l+s+s1}{\PYGZsq{}}\PYG{p}{,} \PYG{l+s+s1}{\PYGZsq{}}\PYG{l+s+s1}{Excess Returns}\PYG{l+s+s1}{\PYGZsq{}}\PYG{p}{]}\PYG{p}{]}
        \PYG{o}{.}\PYG{n}{mul}\PYG{p}{(}\PYG{l+m+mi}{100}\PYG{p}{)}
        \PYG{o}{.}\PYG{n}{assign}\PYG{p}{(}\PYG{n}{ESQ} \PYG{o}{=} \PYG{k}{lambda} \PYG{n}{x}\PYG{p}{:} \PYG{n}{pd}\PYG{o}{.}\PYG{n}{qcut}\PYG{p}{(}\PYG{n}{x}\PYG{o}{=}\PYG{n}{x}\PYG{p}{[}\PYG{l+s+s1}{\PYGZsq{}}\PYG{l+s+s1}{Surprise(}\PYG{l+s+s1}{\PYGZpc{}}\PYG{l+s+s1}{)}\PYG{l+s+s1}{\PYGZsq{}}\PYG{p}{]}\PYG{p}{,} \PYG{n}{q}\PYG{o}{=}\PYG{l+m+mi}{5}\PYG{p}{,} \PYG{n}{labels}\PYG{o}{=}\PYG{k+kc}{False}\PYG{p}{)}\PYG{p}{)}
    \PYG{p}{)}
\PYG{p}{)}

\PYG{n}{plt}\PYG{o}{.}\PYG{n}{xlabel}\PYG{p}{(}\PYG{l+s+s1}{\PYGZsq{}}\PYG{l+s+s1}{Earnings Suprise Portfolio}\PYG{l+s+s1}{\PYGZsq{}}\PYG{p}{)}
\PYG{n}{plt}\PYG{o}{.}\PYG{n}{ylabel}\PYG{p}{(}\PYG{l+s+s1}{\PYGZsq{}}\PYG{l+s+s1}{Announcement Excess Return (}\PYG{l+s+s1}{\PYGZpc{}}\PYG{l+s+s1}{)}\PYG{l+s+s1}{\PYGZsq{}}\PYG{p}{)}

\PYG{n}{\PYGZus{}\PYGZus{}} \PYG{o}{=} \PYG{n}{surprises\PYGZus{}3}\PYG{o}{.}\PYG{n}{index}\PYG{o}{.}\PYG{n}{get\PYGZus{}level\PYGZus{}values}\PYG{p}{(}\PYG{l+s+s1}{\PYGZsq{}}\PYG{l+s+s1}{Date}\PYG{l+s+s1}{\PYGZsq{}}\PYG{p}{)}
\PYG{n}{plt}\PYG{o}{.}\PYG{n}{title}\PYG{p}{(}\PYG{l+s+sa}{f}\PYG{l+s+s1}{\PYGZsq{}}\PYG{l+s+s1}{Earnings Announcements for S\PYGZam{}P 100 Stocks}\PYG{l+s+se}{\PYGZbs{}n}\PYG{l+s+s1}{ from }\PYG{l+s+si}{\PYGZob{}}\PYG{n}{\PYGZus{}\PYGZus{}}\PYG{o}{.}\PYG{n}{min}\PYG{p}{(}\PYG{p}{)}\PYG{l+s+si}{:}\PYG{l+s+s1}{\PYGZpc{}B \PYGZpc{}Y}\PYG{l+s+si}{\PYGZcb{}}\PYG{l+s+s1}{ to }\PYG{l+s+si}{\PYGZob{}}\PYG{n}{\PYGZus{}\PYGZus{}}\PYG{o}{.}\PYG{n}{max}\PYG{p}{(}\PYG{p}{)}\PYG{l+s+si}{:}\PYG{l+s+s1}{\PYGZpc{}B \PYGZpc{}Y}\PYG{l+s+si}{\PYGZcb{}}\PYG{l+s+s1}{\PYGZsq{}}\PYG{p}{)}
\PYG{n}{plt}\PYG{o}{.}\PYG{n}{show}\PYG{p}{(}\PYG{p}{)}
\end{sphinxVerbatim}

\end{sphinxuseclass}\end{sphinxVerbatimInput}
\begin{sphinxVerbatimOutput}

\begin{sphinxuseclass}{cell_output}
\noindent\sphinxincludegraphics{{d07eabf2bb46f6242d93f191be3eb6675d51d0ab36f934152934937aa1f5d7be}.png}

\end{sphinxuseclass}\end{sphinxVerbatimOutput}

\end{sphinxuseclass}

\subsubsection{Improve your \sphinxstyleliteralintitle{\sphinxupquote{download()}} function from above}
\label{\detokenize{mckinney_08_practice_04:improve-your-download-function-from-above}}
\sphinxAtStartPar
Modify \sphinxcode{\sphinxupquote{download()}} to accept one or more than one ticker.
Since we will not use the advanced functionality of the tickers object that \sphinxcode{\sphinxupquote{yf.Tickers()}} creates, we will use \sphinxcode{\sphinxupquote{yf.download()}}.
The current version of \sphinxcode{\sphinxupquote{yf.download()}} does not accept a \sphinxcode{\sphinxupquote{session=}} argument.

\begin{sphinxuseclass}{cell}\begin{sphinxVerbatimInput}

\begin{sphinxuseclass}{cell_input}
\begin{sphinxVerbatim}[commandchars=\\\{\}]
\PYG{k}{def} \PYG{n+nf}{download}\PYG{p}{(}\PYG{n}{tickers}\PYG{p}{)}\PYG{p}{:}

    \PYG{n}{histories} \PYG{o}{=} \PYG{p}{(}
        \PYG{n}{yf}\PYG{o}{.}\PYG{n}{download}\PYG{p}{(}\PYG{n}{tickers}\PYG{p}{,} \PYG{n}{progress}\PYG{o}{=}\PYG{k+kc}{False}\PYG{p}{)}
        \PYG{o}{.}\PYG{n}{assign}\PYG{p}{(}\PYG{n}{Date}\PYG{o}{=}\PYG{k}{lambda} \PYG{n}{x}\PYG{p}{:} \PYG{n}{x}\PYG{o}{.}\PYG{n}{index}\PYG{o}{.}\PYG{n}{tz\PYGZus{}localize}\PYG{p}{(}\PYG{k+kc}{None}\PYG{p}{)}\PYG{p}{)}
        \PYG{o}{.}\PYG{n}{set\PYGZus{}index}\PYG{p}{(}\PYG{l+s+s1}{\PYGZsq{}}\PYG{l+s+s1}{Date}\PYG{l+s+s1}{\PYGZsq{}}\PYG{p}{)}
    \PYG{p}{)}

    \PYG{n}{factors} \PYG{o}{=} \PYG{p}{(}
        \PYG{n}{pdr}\PYG{o}{.}\PYG{n}{DataReader}\PYG{p}{(}
            \PYG{n}{name}\PYG{o}{=}\PYG{l+s+s1}{\PYGZsq{}}\PYG{l+s+s1}{F\PYGZhy{}F\PYGZus{}Research\PYGZus{}Data\PYGZus{}Factors\PYGZus{}daily}\PYG{l+s+s1}{\PYGZsq{}}\PYG{p}{,}
            \PYG{n}{data\PYGZus{}source}\PYG{o}{=}\PYG{l+s+s1}{\PYGZsq{}}\PYG{l+s+s1}{famafrench}\PYG{l+s+s1}{\PYGZsq{}}\PYG{p}{,}
            \PYG{n}{start}\PYG{o}{=}\PYG{l+s+s1}{\PYGZsq{}}\PYG{l+s+s1}{1900}\PYG{l+s+s1}{\PYGZsq{}}\PYG{p}{,}
            \PYG{n}{session}\PYG{o}{=}\PYG{n}{session}
        \PYG{p}{)}
        \PYG{p}{[}\PYG{l+m+mi}{0}\PYG{p}{]}
        \PYG{o}{.}\PYG{n}{div}\PYG{p}{(}\PYG{l+m+mi}{100}\PYG{p}{)}
    \PYG{p}{)}

    \PYG{k}{if} \PYG{n+nb}{type}\PYG{p}{(}\PYG{n}{histories}\PYG{o}{.}\PYG{n}{columns}\PYG{p}{)} \PYG{o+ow}{is} \PYG{n}{pd}\PYG{o}{.}\PYG{n}{MultiIndex}\PYG{p}{:}
        \PYG{n}{\PYGZus{}} \PYG{o}{=} \PYG{n}{pd}\PYG{o}{.}\PYG{n}{MultiIndex}\PYG{o}{.}\PYG{n}{from\PYGZus{}product}\PYG{p}{(}\PYG{p}{[}\PYG{p}{[}\PYG{l+s+s1}{\PYGZsq{}}\PYG{l+s+s1}{Returns}\PYG{l+s+s1}{\PYGZsq{}}\PYG{p}{]}\PYG{p}{,} \PYG{n}{histories}\PYG{p}{[}\PYG{l+s+s1}{\PYGZsq{}}\PYG{l+s+s1}{Adj Close}\PYG{l+s+s1}{\PYGZsq{}}\PYG{p}{]}\PYG{o}{.}\PYG{n}{columns}\PYG{p}{]}\PYG{p}{)}
        \PYG{n}{histories}\PYG{p}{[}\PYG{n}{\PYGZus{}}\PYG{p}{]} \PYG{o}{=} \PYG{n}{histories}\PYG{p}{[}\PYG{l+s+s1}{\PYGZsq{}}\PYG{l+s+s1}{Adj Close}\PYG{l+s+s1}{\PYGZsq{}}\PYG{p}{]}\PYG{o}{.}\PYG{n}{pct\PYGZus{}change}\PYG{p}{(}\PYG{p}{)}

        \PYG{n}{\PYGZus{}} \PYG{o}{=} \PYG{n}{pd}\PYG{o}{.}\PYG{n}{MultiIndex}\PYG{o}{.}\PYG{n}{from\PYGZus{}product}\PYG{p}{(}\PYG{p}{[}\PYG{p}{[}\PYG{l+s+s1}{\PYGZsq{}}\PYG{l+s+s1}{Factors}\PYG{l+s+s1}{\PYGZsq{}}\PYG{p}{]}\PYG{p}{,} \PYG{n}{factors}\PYG{o}{.}\PYG{n}{columns}\PYG{p}{]}\PYG{p}{)}
        \PYG{n}{histories}\PYG{p}{[}\PYG{n}{\PYGZus{}}\PYG{p}{]} \PYG{o}{=} \PYG{n}{factors}

        \PYG{k}{return} \PYG{n}{histories}\PYG{o}{.}\PYG{n}{rename\PYGZus{}axis}\PYG{p}{(}\PYG{n}{columns}\PYG{o}{=}\PYG{p}{[}\PYG{l+s+s1}{\PYGZsq{}}\PYG{l+s+s1}{Variable}\PYG{l+s+s1}{\PYGZsq{}}\PYG{p}{,} \PYG{l+s+s1}{\PYGZsq{}}\PYG{l+s+s1}{Ticker}\PYG{l+s+s1}{\PYGZsq{}}\PYG{p}{]}\PYG{p}{)}

    \PYG{k}{elif} \PYG{n+nb}{type}\PYG{p}{(}\PYG{n}{histories}\PYG{o}{.}\PYG{n}{columns}\PYG{p}{)} \PYG{o+ow}{is} \PYG{n}{pd}\PYG{o}{.}\PYG{n}{Index}\PYG{p}{:}
        \PYG{k}{return} \PYG{n}{histories}\PYG{o}{.}\PYG{n}{join}\PYG{p}{(}\PYG{n}{ff}\PYG{p}{)}\PYG{o}{.}\PYG{n}{rename\PYGZus{}axis}\PYG{p}{(}\PYG{n}{columns}\PYG{o}{=}\PYG{p}{[}\PYG{l+s+s1}{\PYGZsq{}}\PYG{l+s+s1}{Variable}\PYG{l+s+s1}{\PYGZsq{}}\PYG{p}{]}\PYG{p}{)}
\end{sphinxVerbatim}

\end{sphinxuseclass}\end{sphinxVerbatimInput}

\end{sphinxuseclass}
\begin{sphinxuseclass}{cell}\begin{sphinxVerbatimInput}

\begin{sphinxuseclass}{cell_input}
\begin{sphinxVerbatim}[commandchars=\\\{\}]
\PYG{n}{download}\PYG{p}{(}\PYG{n}{tickers}\PYG{o}{=}\PYG{l+s+s1}{\PYGZsq{}}\PYG{l+s+s1}{AAPL}\PYG{l+s+s1}{\PYGZsq{}}\PYG{p}{)}\PYG{o}{.}\PYG{n}{head}\PYG{p}{(}\PYG{p}{)}
\end{sphinxVerbatim}

\end{sphinxuseclass}\end{sphinxVerbatimInput}
\begin{sphinxVerbatimOutput}

\begin{sphinxuseclass}{cell_output}
\begin{sphinxVerbatim}[commandchars=\\\{\}]
[*********************100\PYGZpc{}***********************]  1 of 1 completed
\end{sphinxVerbatim}

\begin{sphinxVerbatim}[commandchars=\\\{\}]
Variable     Open   High    Low  Close  Adj Close     Volume  Mkt\PYGZhy{}RF     SMB  \PYGZbs{}
Date                                                                           
1980\PYGZhy{}12\PYGZhy{}12 0.1283 0.1289 0.1283 0.1283     0.0997  469033600  0.0138 \PYGZhy{}0.0001   
1980\PYGZhy{}12\PYGZhy{}15 0.1222 0.1222 0.1217 0.1217     0.0945  175884800  0.0011  0.0025   
1980\PYGZhy{}12\PYGZhy{}16 0.1133 0.1133 0.1127 0.1127     0.0876  105728000  0.0071 \PYGZhy{}0.0075   
1980\PYGZhy{}12\PYGZhy{}17 0.1155 0.1161 0.1155 0.1155     0.0897   86441600  0.0152 \PYGZhy{}0.0086   
1980\PYGZhy{}12\PYGZhy{}18 0.1189 0.1194 0.1189 0.1189     0.0924   73449600  0.0041  0.0022   

Variable       HML     RF  
Date                       
1980\PYGZhy{}12\PYGZhy{}12 \PYGZhy{}0.0105 0.0006  
1980\PYGZhy{}12\PYGZhy{}15 \PYGZhy{}0.0046 0.0006  
1980\PYGZhy{}12\PYGZhy{}16 \PYGZhy{}0.0047 0.0006  
1980\PYGZhy{}12\PYGZhy{}17 \PYGZhy{}0.0034 0.0006  
1980\PYGZhy{}12\PYGZhy{}18  0.0126 0.0006  
\end{sphinxVerbatim}

\end{sphinxuseclass}\end{sphinxVerbatimOutput}

\end{sphinxuseclass}
\begin{sphinxuseclass}{cell}\begin{sphinxVerbatimInput}

\begin{sphinxuseclass}{cell_input}
\begin{sphinxVerbatim}[commandchars=\\\{\}]
\PYG{n}{download}\PYG{p}{(}\PYG{n}{tickers}\PYG{o}{=}\PYG{l+s+s1}{\PYGZsq{}}\PYG{l+s+s1}{AAPL TSLA}\PYG{l+s+s1}{\PYGZsq{}}\PYG{p}{)}\PYG{o}{.}\PYG{n}{head}\PYG{p}{(}\PYG{p}{)}
\end{sphinxVerbatim}

\end{sphinxuseclass}\end{sphinxVerbatimInput}
\begin{sphinxVerbatimOutput}

\begin{sphinxuseclass}{cell_output}
\begin{sphinxVerbatim}[commandchars=\\\{\}]
[*********************100\PYGZpc{}***********************]  2 of 2 completed
\end{sphinxVerbatim}

\begin{sphinxVerbatim}[commandchars=\\\{\}]
Variable   Adj Close       Close        High         Low        Open       \PYGZbs{}
Ticker          AAPL TSLA   AAPL TSLA   AAPL TSLA   AAPL TSLA   AAPL TSLA   
Date                                                                        
1980\PYGZhy{}12\PYGZhy{}12    0.0997  NaN 0.1283  NaN 0.1289  NaN 0.1283  NaN 0.1283  NaN   
1980\PYGZhy{}12\PYGZhy{}15    0.0945  NaN 0.1217  NaN 0.1222  NaN 0.1217  NaN 0.1222  NaN   
1980\PYGZhy{}12\PYGZhy{}16    0.0876  NaN 0.1127  NaN 0.1133  NaN 0.1127  NaN 0.1133  NaN   
1980\PYGZhy{}12\PYGZhy{}17    0.0897  NaN 0.1155  NaN 0.1161  NaN 0.1155  NaN 0.1155  NaN   
1980\PYGZhy{}12\PYGZhy{}18    0.0924  NaN 0.1189  NaN 0.1194  NaN 0.1189  NaN 0.1189  NaN   

Variable       Volume      Returns      Factors                         
Ticker           AAPL TSLA    AAPL TSLA  Mkt\PYGZhy{}RF     SMB     HML     RF  
Date                                                                    
1980\PYGZhy{}12\PYGZhy{}12  469033600  NaN     NaN  NaN  0.0138 \PYGZhy{}0.0001 \PYGZhy{}0.0105 0.0006  
1980\PYGZhy{}12\PYGZhy{}15  175884800  NaN \PYGZhy{}0.0522  NaN  0.0011  0.0025 \PYGZhy{}0.0046 0.0006  
1980\PYGZhy{}12\PYGZhy{}16  105728000  NaN \PYGZhy{}0.0734  NaN  0.0071 \PYGZhy{}0.0075 \PYGZhy{}0.0047 0.0006  
1980\PYGZhy{}12\PYGZhy{}17   86441600  NaN  0.0248  NaN  0.0152 \PYGZhy{}0.0086 \PYGZhy{}0.0034 0.0006  
1980\PYGZhy{}12\PYGZhy{}18   73449600  NaN  0.0290  NaN  0.0041  0.0022  0.0126 0.0006  
\end{sphinxVerbatim}

\end{sphinxuseclass}\end{sphinxVerbatimOutput}

\end{sphinxuseclass}
\sphinxstepscope


\section{McKinney Chapter 8 \sphinxhyphen{} Practice (Wednesday 2:45 PM, Section 2)}
\label{\detokenize{mckinney_08_practice_02:mckinney-chapter-8-practice-wednesday-2-45-pm-section-2}}\label{\detokenize{mckinney_08_practice_02::doc}}

\subsection{Announcements}
\label{\detokenize{mckinney_08_practice_02:announcements}}\begin{itemize}
\item {} 
\sphinxAtStartPar
Quiz 3 mean was \$\textbackslash{}approx 80\%\$

\item {} 
\sphinxAtStartPar
I posted project 1 to Canvas

\end{itemize}


\subsection{Practice}
\label{\detokenize{mckinney_08_practice_02:practice}}

\subsubsection{Download data from Yahoo! Finance for BAC, C, GS, JPM, MS, and PNC and assign to data frame \sphinxstyleliteralintitle{\sphinxupquote{stocks}}.}
\label{\detokenize{mckinney_08_practice_02:download-data-from-yahoo-finance-for-bac-c-gs-jpm-ms-and-pnc-and-assign-to-data-frame-stocks}}
\sphinxAtStartPar
Use \sphinxcode{\sphinxupquote{stocks.columns.names}} to assign the names \sphinxcode{\sphinxupquote{Variable}} and \sphinxcode{\sphinxupquote{Ticker}} to the column multi index.

\begin{sphinxuseclass}{cell}\begin{sphinxVerbatimInput}

\begin{sphinxuseclass}{cell_input}
\begin{sphinxVerbatim}[commandchars=\\\{\}]
\PYG{k+kn}{import} \PYG{n+nn}{pandas} \PYG{k}{as} \PYG{n+nn}{pd}
\PYG{k+kn}{import} \PYG{n+nn}{numpy} \PYG{k}{as} \PYG{n+nn}{np}
\PYG{k+kn}{import} \PYG{n+nn}{matplotlib}\PYG{n+nn}{.}\PYG{n+nn}{pyplot} \PYG{k}{as} \PYG{n+nn}{plt}
\end{sphinxVerbatim}

\end{sphinxuseclass}\end{sphinxVerbatimInput}

\end{sphinxuseclass}
\begin{sphinxuseclass}{cell}\begin{sphinxVerbatimInput}

\begin{sphinxuseclass}{cell_input}
\begin{sphinxVerbatim}[commandchars=\\\{\}]
\PYG{o}{\PYGZpc{}}\PYG{k}{config} InlineBackend.figure\PYGZus{}format = \PYGZsq{}retina\PYGZsq{}
\PYG{o}{\PYGZpc{}}\PYG{k}{precision} 4
\PYG{n}{pd}\PYG{o}{.}\PYG{n}{options}\PYG{o}{.}\PYG{n}{display}\PYG{o}{.}\PYG{n}{float\PYGZus{}format} \PYG{o}{=} \PYG{l+s+s1}{\PYGZsq{}}\PYG{l+s+si}{\PYGZob{}:.4f\PYGZcb{}}\PYG{l+s+s1}{\PYGZsq{}}\PYG{o}{.}\PYG{n}{format}
\end{sphinxVerbatim}

\end{sphinxuseclass}\end{sphinxVerbatimInput}

\end{sphinxuseclass}
\begin{sphinxuseclass}{cell}\begin{sphinxVerbatimInput}

\begin{sphinxuseclass}{cell_input}
\begin{sphinxVerbatim}[commandchars=\\\{\}]
\PYG{k+kn}{import} \PYG{n+nn}{yfinance} \PYG{k}{as} \PYG{n+nn}{yf}
\PYG{k+kn}{import} \PYG{n+nn}{pandas\PYGZus{}datareader} \PYG{k}{as} \PYG{n+nn}{pdr}
\PYG{k+kn}{import} \PYG{n+nn}{requests\PYGZus{}cache}
\PYG{n}{session} \PYG{o}{=} \PYG{n}{requests\PYGZus{}cache}\PYG{o}{.}\PYG{n}{CachedSession}\PYG{p}{(}\PYG{p}{)}
\end{sphinxVerbatim}

\end{sphinxuseclass}\end{sphinxVerbatimInput}

\end{sphinxuseclass}
\begin{sphinxuseclass}{cell}\begin{sphinxVerbatimInput}

\begin{sphinxuseclass}{cell_input}
\begin{sphinxVerbatim}[commandchars=\\\{\}]
\PYG{n}{tickers} \PYG{o}{=} \PYG{n}{yf}\PYG{o}{.}\PYG{n}{Tickers}\PYG{p}{(}\PYG{n}{tickers}\PYG{o}{=}\PYG{l+s+s1}{\PYGZsq{}}\PYG{l+s+s1}{BAC C GS JPM MS PNC}\PYG{l+s+s1}{\PYGZsq{}}\PYG{p}{,} \PYG{n}{session}\PYG{o}{=}\PYG{n}{session}\PYG{p}{)}
\PYG{n}{stocks} \PYG{o}{=} \PYG{n}{tickers}\PYG{o}{.}\PYG{n}{history}\PYG{p}{(}\PYG{n}{period}\PYG{o}{=}\PYG{l+s+s1}{\PYGZsq{}}\PYG{l+s+s1}{max}\PYG{l+s+s1}{\PYGZsq{}}\PYG{p}{,} \PYG{n}{auto\PYGZus{}adjust}\PYG{o}{=}\PYG{k+kc}{False}\PYG{p}{,} \PYG{n}{progress}\PYG{o}{=}\PYG{k+kc}{False}\PYG{p}{)}
\PYG{n}{stocks}\PYG{o}{.}\PYG{n}{index} \PYG{o}{=} \PYG{n}{stocks}\PYG{o}{.}\PYG{n}{index}\PYG{o}{.}\PYG{n}{tz\PYGZus{}localize}\PYG{p}{(}\PYG{k+kc}{None}\PYG{p}{)}
\PYG{n}{stocks}\PYG{o}{.}\PYG{n}{columns}\PYG{o}{.}\PYG{n}{names} \PYG{o}{=} \PYG{p}{[}\PYG{l+s+s1}{\PYGZsq{}}\PYG{l+s+s1}{Variable}\PYG{l+s+s1}{\PYGZsq{}}\PYG{p}{,} \PYG{l+s+s1}{\PYGZsq{}}\PYG{l+s+s1}{Ticker}\PYG{l+s+s1}{\PYGZsq{}}\PYG{p}{]}
\PYG{n}{stocks}\PYG{o}{.}\PYG{n}{head}\PYG{p}{(}\PYG{p}{)}
\end{sphinxVerbatim}

\end{sphinxuseclass}\end{sphinxVerbatimInput}
\begin{sphinxVerbatimOutput}

\begin{sphinxuseclass}{cell_output}
\begin{sphinxVerbatim}[commandchars=\\\{\}]
Variable   Adj Close                      Close              ... Stock Splits  \PYGZbs{}
Ticker           BAC   C  GS JPM  MS PNC    BAC   C  GS JPM  ...           GS   
Date                                                         ...                
1973\PYGZhy{}02\PYGZhy{}21    1.6206 NaN NaN NaN NaN NaN 4.6250 NaN NaN NaN  ...          NaN   
1973\PYGZhy{}02\PYGZhy{}22    1.6260 NaN NaN NaN NaN NaN 4.6406 NaN NaN NaN  ...          NaN   
1973\PYGZhy{}02\PYGZhy{}23    1.6206 NaN NaN NaN NaN NaN 4.6250 NaN NaN NaN  ...          NaN   
1973\PYGZhy{}02\PYGZhy{}26    1.6206 NaN NaN NaN NaN NaN 4.6250 NaN NaN NaN  ...          NaN   
1973\PYGZhy{}02\PYGZhy{}27    1.6206 NaN NaN NaN NaN NaN 4.6250 NaN NaN NaN  ...          NaN   

Variable                    Volume                      
Ticker     JPM  MS PNC         BAC   C  GS JPM  MS PNC  
Date                                                    
1973\PYGZhy{}02\PYGZhy{}21 NaN NaN NaN  99200.0000 NaN NaN NaN NaN NaN  
1973\PYGZhy{}02\PYGZhy{}22 NaN NaN NaN  47200.0000 NaN NaN NaN NaN NaN  
1973\PYGZhy{}02\PYGZhy{}23 NaN NaN NaN 133600.0000 NaN NaN NaN NaN NaN  
1973\PYGZhy{}02\PYGZhy{}26 NaN NaN NaN  24000.0000 NaN NaN NaN NaN NaN  
1973\PYGZhy{}02\PYGZhy{}27 NaN NaN NaN  41600.0000 NaN NaN NaN NaN NaN  

[5 rows x 48 columns]
\end{sphinxVerbatim}

\end{sphinxuseclass}\end{sphinxVerbatimOutput}

\end{sphinxuseclass}

\subsubsection{Reshape \sphinxstyleliteralintitle{\sphinxupquote{stocks}} from wide to long with dates and tickers as row indexes and assign to data frame \sphinxstyleliteralintitle{\sphinxupquote{stocks\_long}}.}
\label{\detokenize{mckinney_08_practice_02:reshape-stocks-from-wide-to-long-with-dates-and-tickers-as-row-indexes-and-assign-to-data-frame-stocks-long}}
\begin{sphinxuseclass}{cell}\begin{sphinxVerbatimInput}

\begin{sphinxuseclass}{cell_input}
\begin{sphinxVerbatim}[commandchars=\\\{\}]
\PYG{n}{stocks}\PYG{o}{.}\PYG{n}{stack}\PYG{p}{(}\PYG{l+s+s1}{\PYGZsq{}}\PYG{l+s+s1}{Ticker}\PYG{l+s+s1}{\PYGZsq{}}\PYG{p}{)}\PYG{o}{.}\PYG{n}{head}\PYG{p}{(}\PYG{p}{)}
\end{sphinxVerbatim}

\end{sphinxuseclass}\end{sphinxVerbatimInput}
\begin{sphinxVerbatimOutput}

\begin{sphinxuseclass}{cell_output}
\begin{sphinxVerbatim}[commandchars=\\\{\}]
Variable           Adj Close  Close  Dividends   High    Low   Open  \PYGZbs{}
Date       Ticker                                                     
1973\PYGZhy{}02\PYGZhy{}21 BAC        1.6206 4.6250     0.0000 4.6250 4.6250 4.6250   
1973\PYGZhy{}02\PYGZhy{}22 BAC        1.6260 4.6406     0.0000 4.6406 4.6406 4.6406   
1973\PYGZhy{}02\PYGZhy{}23 BAC        1.6206 4.6250     0.0000 4.6250 4.6250 4.6250   
1973\PYGZhy{}02\PYGZhy{}26 BAC        1.6206 4.6250     0.0000 4.6250 4.6250 4.6250   
1973\PYGZhy{}02\PYGZhy{}27 BAC        1.6206 4.6250     0.0000 4.6250 4.6250 4.6250   

Variable           Stock Splits      Volume  
Date       Ticker                            
1973\PYGZhy{}02\PYGZhy{}21 BAC           0.0000  99200.0000  
1973\PYGZhy{}02\PYGZhy{}22 BAC           0.0000  47200.0000  
1973\PYGZhy{}02\PYGZhy{}23 BAC           0.0000 133600.0000  
1973\PYGZhy{}02\PYGZhy{}26 BAC           0.0000  24000.0000  
1973\PYGZhy{}02\PYGZhy{}27 BAC           0.0000  41600.0000  
\end{sphinxVerbatim}

\end{sphinxuseclass}\end{sphinxVerbatimOutput}

\end{sphinxuseclass}
\begin{sphinxuseclass}{cell}\begin{sphinxVerbatimInput}

\begin{sphinxuseclass}{cell_input}
\begin{sphinxVerbatim}[commandchars=\\\{\}]
\PYG{n}{stocks}\PYG{o}{.}\PYG{n}{stack}\PYG{p}{(}\PYG{l+s+s1}{\PYGZsq{}}\PYG{l+s+s1}{Variable}\PYG{l+s+s1}{\PYGZsq{}}\PYG{p}{)}\PYG{o}{.}\PYG{n}{head}\PYG{p}{(}\PYG{p}{)}
\end{sphinxVerbatim}

\end{sphinxuseclass}\end{sphinxVerbatimInput}
\begin{sphinxVerbatimOutput}

\begin{sphinxuseclass}{cell_output}
\begin{sphinxVerbatim}[commandchars=\\\{\}]
Ticker                  BAC   C  GS  JPM  MS  PNC
Date       Variable                              
1973\PYGZhy{}02\PYGZhy{}21 Adj Close 1.6206 NaN NaN  NaN NaN  NaN
           Close     4.6250 NaN NaN  NaN NaN  NaN
           Dividends 0.0000 NaN NaN  NaN NaN  NaN
           High      4.6250 NaN NaN  NaN NaN  NaN
           Low       4.6250 NaN NaN  NaN NaN  NaN
\end{sphinxVerbatim}

\end{sphinxuseclass}\end{sphinxVerbatimOutput}

\end{sphinxuseclass}
\begin{sphinxuseclass}{cell}\begin{sphinxVerbatimInput}

\begin{sphinxuseclass}{cell_input}
\begin{sphinxVerbatim}[commandchars=\\\{\}]
\PYG{n}{stocks\PYGZus{}long} \PYG{o}{=} \PYG{n}{stocks}\PYG{o}{.}\PYG{n}{stack}\PYG{p}{(}\PYG{p}{)}
\PYG{n}{stocks\PYGZus{}long}\PYG{o}{.}\PYG{n}{head}\PYG{p}{(}\PYG{p}{)}
\end{sphinxVerbatim}

\end{sphinxuseclass}\end{sphinxVerbatimInput}
\begin{sphinxVerbatimOutput}

\begin{sphinxuseclass}{cell_output}
\begin{sphinxVerbatim}[commandchars=\\\{\}]
Variable           Adj Close  Close  Dividends   High    Low   Open  \PYGZbs{}
Date       Ticker                                                     
1973\PYGZhy{}02\PYGZhy{}21 BAC        1.6206 4.6250     0.0000 4.6250 4.6250 4.6250   
1973\PYGZhy{}02\PYGZhy{}22 BAC        1.6260 4.6406     0.0000 4.6406 4.6406 4.6406   
1973\PYGZhy{}02\PYGZhy{}23 BAC        1.6206 4.6250     0.0000 4.6250 4.6250 4.6250   
1973\PYGZhy{}02\PYGZhy{}26 BAC        1.6206 4.6250     0.0000 4.6250 4.6250 4.6250   
1973\PYGZhy{}02\PYGZhy{}27 BAC        1.6206 4.6250     0.0000 4.6250 4.6250 4.6250   

Variable           Stock Splits      Volume  
Date       Ticker                            
1973\PYGZhy{}02\PYGZhy{}21 BAC           0.0000  99200.0000  
1973\PYGZhy{}02\PYGZhy{}22 BAC           0.0000  47200.0000  
1973\PYGZhy{}02\PYGZhy{}23 BAC           0.0000 133600.0000  
1973\PYGZhy{}02\PYGZhy{}26 BAC           0.0000  24000.0000  
1973\PYGZhy{}02\PYGZhy{}27 BAC           0.0000  41600.0000  
\end{sphinxVerbatim}

\end{sphinxuseclass}\end{sphinxVerbatimOutput}

\end{sphinxuseclass}

\subsubsection{Add daily returns for each stock to data frames \sphinxstyleliteralintitle{\sphinxupquote{stocks}} and \sphinxstyleliteralintitle{\sphinxupquote{stocks\_long}}.}
\label{\detokenize{mckinney_08_practice_02:add-daily-returns-for-each-stock-to-data-frames-stocks-and-stocks-long}}
\sphinxAtStartPar
Name the returns variable \sphinxcode{\sphinxupquote{Returns}}, and maintain all multi indexes.
\sphinxstyleemphasis{Hint:} Use \sphinxcode{\sphinxupquote{pd.MultiIndex()}} to create a multi index for the the wide data frame \sphinxcode{\sphinxupquote{stocks}}.

\begin{sphinxuseclass}{cell}\begin{sphinxVerbatimInput}

\begin{sphinxuseclass}{cell_input}
\begin{sphinxVerbatim}[commandchars=\\\{\}]
\PYG{n}{\PYGZus{}} \PYG{o}{=} \PYG{n}{pd}\PYG{o}{.}\PYG{n}{MultiIndex}\PYG{o}{.}\PYG{n}{from\PYGZus{}product}\PYG{p}{(}\PYG{p}{[}\PYG{p}{[}\PYG{l+s+s1}{\PYGZsq{}}\PYG{l+s+s1}{Returns}\PYG{l+s+s1}{\PYGZsq{}}\PYG{p}{]}\PYG{p}{,} \PYG{n}{stocks}\PYG{p}{[}\PYG{l+s+s1}{\PYGZsq{}}\PYG{l+s+s1}{Adj Close}\PYG{l+s+s1}{\PYGZsq{}}\PYG{p}{]}\PYG{p}{]}\PYG{p}{)}
\PYG{n}{stocks}\PYG{p}{[}\PYG{n}{\PYGZus{}}\PYG{p}{]} \PYG{o}{=} \PYG{n}{stocks}\PYG{p}{[}\PYG{l+s+s1}{\PYGZsq{}}\PYG{l+s+s1}{Adj Close}\PYG{l+s+s1}{\PYGZsq{}}\PYG{p}{]}\PYG{o}{.}\PYG{n}{pct\PYGZus{}change}\PYG{p}{(}\PYG{p}{)}
\PYG{n}{stocks}\PYG{o}{.}\PYG{n}{head}\PYG{p}{(}\PYG{p}{)}
\end{sphinxVerbatim}

\end{sphinxuseclass}\end{sphinxVerbatimInput}
\begin{sphinxVerbatimOutput}

\begin{sphinxuseclass}{cell_output}
\begin{sphinxVerbatim}[commandchars=\\\{\}]
Variable   Adj Close                      Close              ... Volume      \PYGZbs{}
Ticker           BAC   C  GS JPM  MS PNC    BAC   C  GS JPM  ...     GS JPM   
Date                                                         ...              
1973\PYGZhy{}02\PYGZhy{}21    1.6206 NaN NaN NaN NaN NaN 4.6250 NaN NaN NaN  ...    NaN NaN   
1973\PYGZhy{}02\PYGZhy{}22    1.6260 NaN NaN NaN NaN NaN 4.6406 NaN NaN NaN  ...    NaN NaN   
1973\PYGZhy{}02\PYGZhy{}23    1.6206 NaN NaN NaN NaN NaN 4.6250 NaN NaN NaN  ...    NaN NaN   
1973\PYGZhy{}02\PYGZhy{}26    1.6206 NaN NaN NaN NaN NaN 4.6250 NaN NaN NaN  ...    NaN NaN   
1973\PYGZhy{}02\PYGZhy{}27    1.6206 NaN NaN NaN NaN NaN 4.6250 NaN NaN NaN  ...    NaN NaN   

Variable           Returns                      
Ticker      MS PNC     BAC   C  GS JPM  MS PNC  
Date                                            
1973\PYGZhy{}02\PYGZhy{}21 NaN NaN     NaN NaN NaN NaN NaN NaN  
1973\PYGZhy{}02\PYGZhy{}22 NaN NaN  0.0034 NaN NaN NaN NaN NaN  
1973\PYGZhy{}02\PYGZhy{}23 NaN NaN \PYGZhy{}0.0034 NaN NaN NaN NaN NaN  
1973\PYGZhy{}02\PYGZhy{}26 NaN NaN  0.0000 NaN NaN NaN NaN NaN  
1973\PYGZhy{}02\PYGZhy{}27 NaN NaN  0.0000 NaN NaN NaN NaN NaN  

[5 rows x 54 columns]
\end{sphinxVerbatim}

\end{sphinxuseclass}\end{sphinxVerbatimOutput}

\end{sphinxuseclass}
\sphinxAtStartPar
The easiest way to add returns to long data frame \sphinxcode{\sphinxupquote{stocks\_long}} is to \sphinxcode{\sphinxupquote{.stack()}} wide data frame \sphinxcode{\sphinxupquote{stocks}}!
We could sort \sphinxcode{\sphinxupquote{stocks\_long}} by ticker and date (to sort chronologically within each ticker), then use \sphinxcode{\sphinxupquote{.pct\_change()}}.
However, this approach miscalculates the first return for every ticker except for the first ticker.
The easiest and safest solution is to \sphinxcode{\sphinxupquote{.stack()}} the wide data frame \sphinxcode{\sphinxupquote{stocks}}!

\begin{sphinxuseclass}{cell}\begin{sphinxVerbatimInput}

\begin{sphinxuseclass}{cell_input}
\begin{sphinxVerbatim}[commandchars=\\\{\}]
\PYG{c+c1}{\PYGZsh{} see that the first return for C is wrong}
\PYG{c+c1}{\PYGZsh{} stocks\PYGZus{}long[\PYGZsq{}Adj Close\PYGZsq{}].sort\PYGZus{}index(level=[\PYGZsq{}Ticker\PYGZsq{}, \PYGZsq{}Date\PYGZsq{}]).pct\PYGZus{}change().loc[(slice(None), \PYGZsq{}C\PYGZsq{})]}
\end{sphinxVerbatim}

\end{sphinxuseclass}\end{sphinxVerbatimInput}

\end{sphinxuseclass}
\begin{sphinxuseclass}{cell}\begin{sphinxVerbatimInput}

\begin{sphinxuseclass}{cell_input}
\begin{sphinxVerbatim}[commandchars=\\\{\}]
\PYG{n}{stocks\PYGZus{}long} \PYG{o}{=} \PYG{n}{stocks}\PYG{o}{.}\PYG{n}{stack}\PYG{p}{(}\PYG{p}{)}
\end{sphinxVerbatim}

\end{sphinxuseclass}\end{sphinxVerbatimInput}

\end{sphinxuseclass}

\subsubsection{Download the daily benchmark return factors from Ken French’s data library.}
\label{\detokenize{mckinney_08_practice_02:download-the-daily-benchmark-return-factors-from-ken-french-s-data-library}}
\begin{sphinxuseclass}{cell}\begin{sphinxVerbatimInput}

\begin{sphinxuseclass}{cell_input}
\begin{sphinxVerbatim}[commandchars=\\\{\}]
\PYG{n}{pdr}\PYG{o}{.}\PYG{n}{famafrench}\PYG{o}{.}\PYG{n}{get\PYGZus{}available\PYGZus{}datasets}\PYG{p}{(}\PYG{p}{)}\PYG{p}{[}\PYG{p}{:}\PYG{l+m+mi}{5}\PYG{p}{]}
\end{sphinxVerbatim}

\end{sphinxuseclass}\end{sphinxVerbatimInput}
\begin{sphinxVerbatimOutput}

\begin{sphinxuseclass}{cell_output}
\begin{sphinxVerbatim}[commandchars=\\\{\}]
[\PYGZsq{}F\PYGZhy{}F\PYGZus{}Research\PYGZus{}Data\PYGZus{}Factors\PYGZsq{},
 \PYGZsq{}F\PYGZhy{}F\PYGZus{}Research\PYGZus{}Data\PYGZus{}Factors\PYGZus{}weekly\PYGZsq{},
 \PYGZsq{}F\PYGZhy{}F\PYGZus{}Research\PYGZus{}Data\PYGZus{}Factors\PYGZus{}daily\PYGZsq{},
 \PYGZsq{}F\PYGZhy{}F\PYGZus{}Research\PYGZus{}Data\PYGZus{}5\PYGZus{}Factors\PYGZus{}2x3\PYGZsq{},
 \PYGZsq{}F\PYGZhy{}F\PYGZus{}Research\PYGZus{}Data\PYGZus{}5\PYGZus{}Factors\PYGZus{}2x3\PYGZus{}daily\PYGZsq{}]
\end{sphinxVerbatim}

\end{sphinxuseclass}\end{sphinxVerbatimOutput}

\end{sphinxuseclass}
\begin{sphinxuseclass}{cell}\begin{sphinxVerbatimInput}

\begin{sphinxuseclass}{cell_input}
\begin{sphinxVerbatim}[commandchars=\\\{\}]
\PYG{n}{ff} \PYG{o}{=} \PYG{p}{(}
    \PYG{n}{pdr}\PYG{o}{.}\PYG{n}{DataReader}\PYG{p}{(}
        \PYG{n}{name}\PYG{o}{=}\PYG{l+s+s1}{\PYGZsq{}}\PYG{l+s+s1}{F\PYGZhy{}F\PYGZus{}Research\PYGZus{}Data\PYGZus{}Factors\PYGZus{}daily}\PYG{l+s+s1}{\PYGZsq{}}\PYG{p}{,}
        \PYG{n}{data\PYGZus{}source}\PYG{o}{=}\PYG{l+s+s1}{\PYGZsq{}}\PYG{l+s+s1}{famafrench}\PYG{l+s+s1}{\PYGZsq{}}\PYG{p}{,}
        \PYG{n}{start}\PYG{o}{=}\PYG{l+s+s1}{\PYGZsq{}}\PYG{l+s+s1}{1900}\PYG{l+s+s1}{\PYGZsq{}}\PYG{p}{,}
        \PYG{n}{session}\PYG{o}{=}\PYG{n}{session}
    \PYG{p}{)}
    \PYG{p}{[}\PYG{l+m+mi}{0}\PYG{p}{]}
    \PYG{o}{.}\PYG{n}{div}\PYG{p}{(}\PYG{l+m+mi}{100}\PYG{p}{)}
\PYG{p}{)}

\PYG{n}{ff}\PYG{o}{.}\PYG{n}{head}\PYG{p}{(}\PYG{p}{)}
\end{sphinxVerbatim}

\end{sphinxuseclass}\end{sphinxVerbatimInput}
\begin{sphinxVerbatimOutput}

\begin{sphinxuseclass}{cell_output}
\begin{sphinxVerbatim}[commandchars=\\\{\}]
            Mkt\PYGZhy{}RF     SMB     HML     RF
Date                                     
1926\PYGZhy{}07\PYGZhy{}01  0.0010 \PYGZhy{}0.0025 \PYGZhy{}0.0027 0.0001
1926\PYGZhy{}07\PYGZhy{}02  0.0045 \PYGZhy{}0.0033 \PYGZhy{}0.0006 0.0001
1926\PYGZhy{}07\PYGZhy{}06  0.0017  0.0030 \PYGZhy{}0.0039 0.0001
1926\PYGZhy{}07\PYGZhy{}07  0.0009 \PYGZhy{}0.0058  0.0002 0.0001
1926\PYGZhy{}07\PYGZhy{}08  0.0021 \PYGZhy{}0.0038  0.0019 0.0001
\end{sphinxVerbatim}

\end{sphinxuseclass}\end{sphinxVerbatimOutput}

\end{sphinxuseclass}

\subsubsection{Add the daily benchmark return factors to \sphinxstyleliteralintitle{\sphinxupquote{stocks}} and \sphinxstyleliteralintitle{\sphinxupquote{stocks\_long}}.}
\label{\detokenize{mckinney_08_practice_02:add-the-daily-benchmark-return-factors-to-stocks-and-stocks-long}}
\sphinxAtStartPar
For the wide data frame \sphinxcode{\sphinxupquote{stocks}}, use the outer index name \sphinxcode{\sphinxupquote{Factors}}.

\begin{sphinxuseclass}{cell}\begin{sphinxVerbatimInput}

\begin{sphinxuseclass}{cell_input}
\begin{sphinxVerbatim}[commandchars=\\\{\}]
\PYG{n}{\PYGZus{}} \PYG{o}{=} \PYG{n}{pd}\PYG{o}{.}\PYG{n}{MultiIndex}\PYG{o}{.}\PYG{n}{from\PYGZus{}product}\PYG{p}{(}\PYG{p}{[}\PYG{p}{[}\PYG{l+s+s1}{\PYGZsq{}}\PYG{l+s+s1}{Factors}\PYG{l+s+s1}{\PYGZsq{}}\PYG{p}{]}\PYG{p}{,} \PYG{n}{ff}\PYG{p}{]}\PYG{p}{)}
\PYG{n}{stocks}\PYG{p}{[}\PYG{n}{\PYGZus{}}\PYG{p}{]} \PYG{o}{=} \PYG{n}{ff}
\PYG{n}{stocks}\PYG{o}{.}\PYG{n}{head}\PYG{p}{(}\PYG{p}{)}
\end{sphinxVerbatim}

\end{sphinxuseclass}\end{sphinxVerbatimInput}
\begin{sphinxVerbatimOutput}

\begin{sphinxuseclass}{cell_output}
\begin{sphinxVerbatim}[commandchars=\\\{\}]
Variable   Adj Close                      Close              ... Returns      \PYGZbs{}
Ticker           BAC   C  GS JPM  MS PNC    BAC   C  GS JPM  ...     BAC   C   
Date                                                         ...               
1973\PYGZhy{}02\PYGZhy{}21    1.6206 NaN NaN NaN NaN NaN 4.6250 NaN NaN NaN  ...     NaN NaN   
1973\PYGZhy{}02\PYGZhy{}22    1.6260 NaN NaN NaN NaN NaN 4.6406 NaN NaN NaN  ...  0.0034 NaN   
1973\PYGZhy{}02\PYGZhy{}23    1.6206 NaN NaN NaN NaN NaN 4.6250 NaN NaN NaN  ... \PYGZhy{}0.0034 NaN   
1973\PYGZhy{}02\PYGZhy{}26    1.6206 NaN NaN NaN NaN NaN 4.6250 NaN NaN NaN  ...  0.0000 NaN   
1973\PYGZhy{}02\PYGZhy{}27    1.6206 NaN NaN NaN NaN NaN 4.6250 NaN NaN NaN  ...  0.0000 NaN   

Variable                   Factors                        
Ticker      GS JPM  MS PNC  Mkt\PYGZhy{}RF     SMB    HML     RF  
Date                                                      
1973\PYGZhy{}02\PYGZhy{}21 NaN NaN NaN NaN \PYGZhy{}0.0074 \PYGZhy{}0.0039 0.0054 0.0002  
1973\PYGZhy{}02\PYGZhy{}22 NaN NaN NaN NaN \PYGZhy{}0.0030 \PYGZhy{}0.0037 0.0022 0.0002  
1973\PYGZhy{}02\PYGZhy{}23 NaN NaN NaN NaN \PYGZhy{}0.0108 \PYGZhy{}0.0019 0.0054 0.0002  
1973\PYGZhy{}02\PYGZhy{}26 NaN NaN NaN NaN \PYGZhy{}0.0088 \PYGZhy{}0.0050 0.0054 0.0002  
1973\PYGZhy{}02\PYGZhy{}27 NaN NaN NaN NaN \PYGZhy{}0.0115 \PYGZhy{}0.0018 0.0064 0.0002  

[5 rows x 58 columns]
\end{sphinxVerbatim}

\end{sphinxuseclass}\end{sphinxVerbatimOutput}

\end{sphinxuseclass}
\sphinxAtStartPar
We can use \sphinxcode{\sphinxupquote{.join()}} even though \sphinxcode{\sphinxupquote{stocks\_long}} has a multi index.
\sphinxstyleemphasis{\sphinxstylestrong{Note that re\sphinxhyphen{}running a “self join” can create duplicate columns.}}
We should be careful to run self joins only once!

\begin{sphinxuseclass}{cell}\begin{sphinxVerbatimInput}

\begin{sphinxuseclass}{cell_input}
\begin{sphinxVerbatim}[commandchars=\\\{\}]
\PYG{n}{stocks\PYGZus{}long} \PYG{o}{=} \PYG{n}{stocks\PYGZus{}long}\PYG{o}{.}\PYG{n}{join}\PYG{p}{(}\PYG{n}{ff}\PYG{p}{)}
\PYG{n}{stocks\PYGZus{}long}\PYG{o}{.}\PYG{n}{head}\PYG{p}{(}\PYG{p}{)}
\end{sphinxVerbatim}

\end{sphinxuseclass}\end{sphinxVerbatimInput}
\begin{sphinxVerbatimOutput}

\begin{sphinxuseclass}{cell_output}
\begin{sphinxVerbatim}[commandchars=\\\{\}]
                   Adj Close  Close  Dividends   High    Low   Open  \PYGZbs{}
Date       Ticker                                                     
1973\PYGZhy{}02\PYGZhy{}21 BAC        1.6206 4.6250     0.0000 4.6250 4.6250 4.6250   
1973\PYGZhy{}02\PYGZhy{}22 BAC        1.6260 4.6406     0.0000 4.6406 4.6406 4.6406   
1973\PYGZhy{}02\PYGZhy{}23 BAC        1.6206 4.6250     0.0000 4.6250 4.6250 4.6250   
1973\PYGZhy{}02\PYGZhy{}26 BAC        1.6206 4.6250     0.0000 4.6250 4.6250 4.6250   
1973\PYGZhy{}02\PYGZhy{}27 BAC        1.6206 4.6250     0.0000 4.6250 4.6250 4.6250   

                   Stock Splits      Volume  Returns  Mkt\PYGZhy{}RF     SMB    HML  \PYGZbs{}
Date       Ticker                                                             
1973\PYGZhy{}02\PYGZhy{}21 BAC           0.0000  99200.0000      NaN \PYGZhy{}0.0074 \PYGZhy{}0.0039 0.0054   
1973\PYGZhy{}02\PYGZhy{}22 BAC           0.0000  47200.0000   0.0034 \PYGZhy{}0.0030 \PYGZhy{}0.0037 0.0022   
1973\PYGZhy{}02\PYGZhy{}23 BAC           0.0000 133600.0000  \PYGZhy{}0.0034 \PYGZhy{}0.0108 \PYGZhy{}0.0019 0.0054   
1973\PYGZhy{}02\PYGZhy{}26 BAC           0.0000  24000.0000   0.0000 \PYGZhy{}0.0088 \PYGZhy{}0.0050 0.0054   
1973\PYGZhy{}02\PYGZhy{}27 BAC           0.0000  41600.0000   0.0000 \PYGZhy{}0.0115 \PYGZhy{}0.0018 0.0064   

                      RF  
Date       Ticker         
1973\PYGZhy{}02\PYGZhy{}21 BAC    0.0002  
1973\PYGZhy{}02\PYGZhy{}22 BAC    0.0002  
1973\PYGZhy{}02\PYGZhy{}23 BAC    0.0002  
1973\PYGZhy{}02\PYGZhy{}26 BAC    0.0002  
1973\PYGZhy{}02\PYGZhy{}27 BAC    0.0002  
\end{sphinxVerbatim}

\end{sphinxuseclass}\end{sphinxVerbatimOutput}

\end{sphinxuseclass}

\subsubsection{Write a function \sphinxstyleliteralintitle{\sphinxupquote{download()}} that accepts tickers and returns a wide data frame of returns with the daily benchmark return factors.}
\label{\detokenize{mckinney_08_practice_02:write-a-function-download-that-accepts-tickers-and-returns-a-wide-data-frame-of-returns-with-the-daily-benchmark-return-factors}}
\sphinxAtStartPar
\sphinxstyleemphasis{\sphinxstylestrong{See advanced solution below.}}
I had planned this as an in\sphinxhyphen{}class exercise, but decided to spend more time on the earnings surprise exercise.


\subsubsection{Download earnings per share for the stocks in \sphinxstyleliteralintitle{\sphinxupquote{stocks}} and combine to a long data frame \sphinxstyleliteralintitle{\sphinxupquote{earnings}}.}
\label{\detokenize{mckinney_08_practice_02:download-earnings-per-share-for-the-stocks-in-stocks-and-combine-to-a-long-data-frame-earnings}}
\sphinxAtStartPar
Use the \sphinxcode{\sphinxupquote{.earnings\_dates}} method described \sphinxhref{https://pypi.org/project/yfinance/}{here}.
Use \sphinxcode{\sphinxupquote{pd.concat()}} to combine the result of each the \sphinxcode{\sphinxupquote{.earnings\_date}} data frames and assign them to a new data frame \sphinxcode{\sphinxupquote{earnings}}.
Name the row indexes \sphinxcode{\sphinxupquote{Ticker}} and \sphinxcode{\sphinxupquote{Date}} and swap to match the order of the row index in \sphinxcode{\sphinxupquote{stocks\_long}}.

\begin{sphinxuseclass}{cell}\begin{sphinxVerbatimInput}

\begin{sphinxuseclass}{cell_input}
\begin{sphinxVerbatim}[commandchars=\\\{\}]
\PYG{c+c1}{\PYGZsh{} some students had to update yfinance to use the .earnings\PYGZus{}dates atrtibute}
\PYG{c+c1}{\PYGZsh{} \PYGZpc{}pip install \PYGZhy{}U yfinance}
\end{sphinxVerbatim}

\end{sphinxuseclass}\end{sphinxVerbatimInput}

\end{sphinxuseclass}
\begin{sphinxuseclass}{cell}\begin{sphinxVerbatimInput}

\begin{sphinxuseclass}{cell_input}
\begin{sphinxVerbatim}[commandchars=\\\{\}]
\PYG{n}{tickers}\PYG{o}{.}\PYG{n}{tickers}\PYG{p}{[}\PYG{l+s+s1}{\PYGZsq{}}\PYG{l+s+s1}{BAC}\PYG{l+s+s1}{\PYGZsq{}}\PYG{p}{]}\PYG{o}{.}\PYG{n}{earnings\PYGZus{}dates}\PYG{o}{.}\PYG{n}{head}\PYG{p}{(}\PYG{l+m+mi}{2}\PYG{p}{)}
\end{sphinxVerbatim}

\end{sphinxuseclass}\end{sphinxVerbatimInput}
\begin{sphinxVerbatimOutput}

\begin{sphinxuseclass}{cell_output}
\begin{sphinxVerbatim}[commandchars=\\\{\}]
                           EPS Estimate  Reported EPS  Surprise(\PYGZpc{})
Earnings Date                                                     
2024\PYGZhy{}01\PYGZhy{}11 08:00:00\PYGZhy{}05:00           NaN           NaN          NaN
2023\PYGZhy{}10\PYGZhy{}17 08:00:00\PYGZhy{}04:00           NaN           NaN          NaN
\end{sphinxVerbatim}

\end{sphinxuseclass}\end{sphinxVerbatimOutput}

\end{sphinxuseclass}
\begin{sphinxuseclass}{cell}\begin{sphinxVerbatimInput}

\begin{sphinxuseclass}{cell_input}
\begin{sphinxVerbatim}[commandchars=\\\{\}]
\PYG{n}{earnings} \PYG{o}{=} \PYG{p}{(}
    \PYG{n}{pd}\PYG{o}{.}\PYG{n}{concat}\PYG{p}{(}
        \PYG{n}{objs}\PYG{o}{=}\PYG{p}{[}\PYG{n}{tickers}\PYG{o}{.}\PYG{n}{tickers}\PYG{p}{[}\PYG{n}{t}\PYG{p}{]}\PYG{o}{.}\PYG{n}{earnings\PYGZus{}dates} \PYG{k}{for} \PYG{n}{t} \PYG{o+ow}{in} \PYG{n}{tickers}\PYG{o}{.}\PYG{n}{tickers}\PYG{p}{]}\PYG{p}{,}
        \PYG{n}{keys}\PYG{o}{=}\PYG{n}{tickers}\PYG{o}{.}\PYG{n}{tickers}\PYG{p}{,}
        \PYG{n}{names}\PYG{o}{=}\PYG{p}{[}\PYG{l+s+s1}{\PYGZsq{}}\PYG{l+s+s1}{Ticker}\PYG{l+s+s1}{\PYGZsq{}}\PYG{p}{,} \PYG{l+s+s1}{\PYGZsq{}}\PYG{l+s+s1}{Date}\PYG{l+s+s1}{\PYGZsq{}}\PYG{p}{]}
    \PYG{p}{)}
    \PYG{o}{.}\PYG{n}{swaplevel}\PYG{p}{(}\PYG{p}{)}
    \PYG{o}{.}\PYG{n}{rename\PYGZus{}axis}\PYG{p}{(}\PYG{n}{columns}\PYG{o}{=}\PYG{l+s+s1}{\PYGZsq{}}\PYG{l+s+s1}{Variable}\PYG{l+s+s1}{\PYGZsq{}}\PYG{p}{)}
\PYG{p}{)}

\PYG{n}{earnings}\PYG{o}{.}\PYG{n}{head}\PYG{p}{(}\PYG{l+m+mi}{2}\PYG{p}{)}
\end{sphinxVerbatim}

\end{sphinxuseclass}\end{sphinxVerbatimInput}
\begin{sphinxVerbatimOutput}

\begin{sphinxuseclass}{cell_output}
\begin{sphinxVerbatim}[commandchars=\\\{\}]
Variable                          EPS Estimate  Reported EPS  Surprise(\PYGZpc{})
Date                      Ticker                                         
2024\PYGZhy{}01\PYGZhy{}11 08:00:00\PYGZhy{}05:00 BAC              NaN           NaN          NaN
2023\PYGZhy{}10\PYGZhy{}17 08:00:00\PYGZhy{}04:00 BAC              NaN           NaN          NaN
\end{sphinxVerbatim}

\end{sphinxuseclass}\end{sphinxVerbatimOutput}

\end{sphinxuseclass}

\subsubsection{Combine \sphinxstyleliteralintitle{\sphinxupquote{earnings}} with the returns from \sphinxstyleliteralintitle{\sphinxupquote{stocks\_long}}.}
\label{\detokenize{mckinney_08_practice_02:combine-earnings-with-the-returns-from-stocks-long}}
\sphinxAtStartPar
\sphinxstyleemphasis{\sphinxstylestrong{It is easier to leave \sphinxcode{\sphinxupquote{stocks}} and \sphinxcode{\sphinxupquote{stocks\_long}} as\sphinxhyphen{}is and work with slices \sphinxcode{\sphinxupquote{returns}} and \sphinxcode{\sphinxupquote{returns\_long}}.}}
Use the \sphinxcode{\sphinxupquote{tz\_localize('America/New\_York')}} method add time zone information back to \sphinxcode{\sphinxupquote{returns.index}} and use \sphinxcode{\sphinxupquote{pd.to\_timedelta(16, unit='h')}} to set time to the market close in New York City.
Use \sphinxhref{https://pandas.pydata.org/pandas-docs/stable/reference/api/pandas.merge\_asof.html}{\sphinxcode{\sphinxupquote{pd.merge\_asof()}}} to match earnings announcement dates and times to appropriate return periods.
For example, if a firm announces earnings after the close at 5 PM on February 7, we want to match the return period from 4 PM on February 7 to 4 PM on February 8.

\begin{sphinxuseclass}{cell}\begin{sphinxVerbatimInput}

\begin{sphinxuseclass}{cell_input}
\begin{sphinxVerbatim}[commandchars=\\\{\}]
\PYG{n}{returns} \PYG{o}{=} \PYG{n}{stocks}\PYG{p}{[}\PYG{l+s+s1}{\PYGZsq{}}\PYG{l+s+s1}{Returns}\PYG{l+s+s1}{\PYGZsq{}}\PYG{p}{]}
\PYG{n}{returns}\PYG{o}{.}\PYG{n}{index} \PYG{o}{=} \PYG{n}{returns}\PYG{o}{.}\PYG{n}{index}\PYG{o}{.}\PYG{n}{tz\PYGZus{}localize}\PYG{p}{(}\PYG{l+s+s1}{\PYGZsq{}}\PYG{l+s+s1}{America/New\PYGZus{}York}\PYG{l+s+s1}{\PYGZsq{}}\PYG{p}{)} \PYG{o}{+} \PYG{n}{pd}\PYG{o}{.}\PYG{n}{to\PYGZus{}timedelta}\PYG{p}{(}\PYG{l+m+mi}{16}\PYG{p}{,} \PYG{n}{unit}\PYG{o}{=}\PYG{l+s+s1}{\PYGZsq{}}\PYG{l+s+s1}{H}\PYG{l+s+s1}{\PYGZsq{}}\PYG{p}{)}
\PYG{n}{returns\PYGZus{}long} \PYG{o}{=} \PYG{n}{returns}\PYG{o}{.}\PYG{n}{stack}\PYG{p}{(}\PYG{p}{)}\PYG{o}{.}\PYG{n}{to\PYGZus{}frame}\PYG{p}{(}\PYG{l+s+s1}{\PYGZsq{}}\PYG{l+s+s1}{Returns}\PYG{l+s+s1}{\PYGZsq{}}\PYG{p}{)}
\PYG{n}{returns\PYGZus{}long}\PYG{o}{.}\PYG{n}{head}\PYG{p}{(}\PYG{p}{)}
\end{sphinxVerbatim}

\end{sphinxuseclass}\end{sphinxVerbatimInput}
\begin{sphinxVerbatimOutput}

\begin{sphinxuseclass}{cell_output}
\begin{sphinxVerbatim}[commandchars=\\\{\}]
                                  Returns
Date                      Ticker         
1973\PYGZhy{}02\PYGZhy{}22 16:00:00\PYGZhy{}05:00 BAC      0.0034
1973\PYGZhy{}02\PYGZhy{}23 16:00:00\PYGZhy{}05:00 BAC     \PYGZhy{}0.0034
1973\PYGZhy{}02\PYGZhy{}26 16:00:00\PYGZhy{}05:00 BAC      0.0000
1973\PYGZhy{}02\PYGZhy{}27 16:00:00\PYGZhy{}05:00 BAC      0.0000
1973\PYGZhy{}02\PYGZhy{}28 16:00:00\PYGZhy{}05:00 BAC      0.0000
\end{sphinxVerbatim}

\end{sphinxuseclass}\end{sphinxVerbatimOutput}

\end{sphinxuseclass}
\begin{sphinxuseclass}{cell}\begin{sphinxVerbatimInput}

\begin{sphinxuseclass}{cell_input}
\begin{sphinxVerbatim}[commandchars=\\\{\}]
\PYG{n}{surprises} \PYG{o}{=} \PYG{p}{(}
    \PYG{n}{pd}\PYG{o}{.}\PYG{n}{merge\PYGZus{}asof}\PYG{p}{(}
        \PYG{n}{left}\PYG{o}{=}\PYG{n}{earnings}\PYG{o}{.}\PYG{n}{sort\PYGZus{}index}\PYG{p}{(}\PYG{n}{level}\PYG{o}{=}\PYG{p}{[}\PYG{l+s+s1}{\PYGZsq{}}\PYG{l+s+s1}{Date}\PYG{l+s+s1}{\PYGZsq{}}\PYG{p}{,} \PYG{l+s+s1}{\PYGZsq{}}\PYG{l+s+s1}{Ticker}\PYG{l+s+s1}{\PYGZsq{}}\PYG{p}{]}\PYG{p}{)}\PYG{p}{,}
        \PYG{n}{right}\PYG{o}{=}\PYG{n}{returns\PYGZus{}long}\PYG{o}{.}\PYG{n}{sort\PYGZus{}index}\PYG{p}{(}\PYG{n}{level}\PYG{o}{=}\PYG{p}{[}\PYG{l+s+s1}{\PYGZsq{}}\PYG{l+s+s1}{Date}\PYG{l+s+s1}{\PYGZsq{}}\PYG{p}{,} \PYG{l+s+s1}{\PYGZsq{}}\PYG{l+s+s1}{Ticker}\PYG{l+s+s1}{\PYGZsq{}}\PYG{p}{]}\PYG{p}{)}\PYG{p}{,}
        \PYG{n}{on}\PYG{o}{=}\PYG{l+s+s1}{\PYGZsq{}}\PYG{l+s+s1}{Date}\PYG{l+s+s1}{\PYGZsq{}}\PYG{p}{,}
        \PYG{n}{by}\PYG{o}{=}\PYG{l+s+s1}{\PYGZsq{}}\PYG{l+s+s1}{Ticker}\PYG{l+s+s1}{\PYGZsq{}}\PYG{p}{,}
        \PYG{n}{direction}\PYG{o}{=}\PYG{l+s+s1}{\PYGZsq{}}\PYG{l+s+s1}{forward}\PYG{l+s+s1}{\PYGZsq{}}\PYG{p}{,}
        \PYG{n}{allow\PYGZus{}exact\PYGZus{}matches}\PYG{o}{=}\PYG{k+kc}{False}
    \PYG{p}{)}
    \PYG{o}{.}\PYG{n}{set\PYGZus{}index}\PYG{p}{(}\PYG{p}{[}\PYG{l+s+s1}{\PYGZsq{}}\PYG{l+s+s1}{Date}\PYG{l+s+s1}{\PYGZsq{}}\PYG{p}{,} \PYG{l+s+s1}{\PYGZsq{}}\PYG{l+s+s1}{Ticker}\PYG{l+s+s1}{\PYGZsq{}}\PYG{p}{]}\PYG{p}{)}
\PYG{p}{)}

\PYG{n}{surprises}\PYG{o}{.}\PYG{n}{head}\PYG{p}{(}\PYG{p}{)}
\end{sphinxVerbatim}

\end{sphinxuseclass}\end{sphinxVerbatimInput}
\begin{sphinxVerbatimOutput}

\begin{sphinxuseclass}{cell_output}
\begin{sphinxVerbatim}[commandchars=\\\{\}]
                                  EPS Estimate  Reported EPS  Surprise(\PYGZpc{})  \PYGZbs{}
Date                      Ticker                                            
2021\PYGZhy{}04\PYGZhy{}14 03:00:00\PYGZhy{}04:00 GS           10.2200       18.6000       0.8194   
2021\PYGZhy{}04\PYGZhy{}16 03:00:00\PYGZhy{}04:00 MS            1.7000        2.1900       0.2890   
2021\PYGZhy{}07\PYGZhy{}13 03:00:00\PYGZhy{}04:00 GS           10.2400       15.0200       0.4668   
2021\PYGZhy{}07\PYGZhy{}15 03:00:00\PYGZhy{}04:00 MS            1.6500        1.8500       0.1185   
2021\PYGZhy{}10\PYGZhy{}14 03:00:00\PYGZhy{}04:00 MS            1.6900        1.9800       0.1751   

                                  Returns  
Date                      Ticker           
2021\PYGZhy{}04\PYGZhy{}14 03:00:00\PYGZhy{}04:00 GS       0.0234  
2021\PYGZhy{}04\PYGZhy{}16 03:00:00\PYGZhy{}04:00 MS      \PYGZhy{}0.0276  
2021\PYGZhy{}07\PYGZhy{}13 03:00:00\PYGZhy{}04:00 GS      \PYGZhy{}0.0119  
2021\PYGZhy{}07\PYGZhy{}15 03:00:00\PYGZhy{}04:00 MS       0.0018  
2021\PYGZhy{}10\PYGZhy{}14 03:00:00\PYGZhy{}04:00 MS       0.0248  
\end{sphinxVerbatim}

\end{sphinxuseclass}\end{sphinxVerbatimOutput}

\end{sphinxuseclass}
\begin{sphinxuseclass}{cell}\begin{sphinxVerbatimInput}

\begin{sphinxuseclass}{cell_input}
\begin{sphinxVerbatim}[commandchars=\\\{\}]
\PYG{n}{surprises}\PYG{o}{.}\PYG{n}{corr}\PYG{p}{(}\PYG{p}{)}
\end{sphinxVerbatim}

\end{sphinxuseclass}\end{sphinxVerbatimInput}
\begin{sphinxVerbatimOutput}

\begin{sphinxuseclass}{cell_output}
\begin{sphinxVerbatim}[commandchars=\\\{\}]
              EPS Estimate  Reported EPS  Surprise(\PYGZpc{})  Returns
EPS Estimate        1.0000        0.9444       0.3620  \PYGZhy{}0.2142
Reported EPS        0.9444        1.0000       0.6171  \PYGZhy{}0.0615
Surprise(\PYGZpc{})         0.3620        0.6171       1.0000   0.4397
Returns            \PYGZhy{}0.2142       \PYGZhy{}0.0615       0.4397   1.0000
\end{sphinxVerbatim}

\end{sphinxuseclass}\end{sphinxVerbatimOutput}

\end{sphinxuseclass}

\subsubsection{Plot the relation between daily returns and earnings surprises}
\label{\detokenize{mckinney_08_practice_02:plot-the-relation-between-daily-returns-and-earnings-surprises}}
\sphinxAtStartPar
Three options in increasing difficulty:
\begin{enumerate}
\sphinxsetlistlabels{\arabic}{enumi}{enumii}{}{.}%
\item {} 
\sphinxAtStartPar
Scatter plot

\item {} 
\sphinxAtStartPar
Scatter plot with a best\sphinxhyphen{}fit line using \sphinxcode{\sphinxupquote{regplot()}} from the seaborn package

\item {} 
\sphinxAtStartPar
Bar plot using \sphinxcode{\sphinxupquote{barplot()}} from the seaborn package after using \sphinxcode{\sphinxupquote{pd.qcut()}} to form five groups on earnings surprises

\end{enumerate}

\begin{sphinxuseclass}{cell}\begin{sphinxVerbatimInput}

\begin{sphinxuseclass}{cell_input}
\begin{sphinxVerbatim}[commandchars=\\\{\}]
\PYG{p}{(}
    \PYG{n}{surprises}
    \PYG{p}{[}\PYG{p}{[}\PYG{l+s+s1}{\PYGZsq{}}\PYG{l+s+s1}{Surprise(}\PYG{l+s+s1}{\PYGZpc{}}\PYG{l+s+s1}{)}\PYG{l+s+s1}{\PYGZsq{}}\PYG{p}{,} \PYG{l+s+s1}{\PYGZsq{}}\PYG{l+s+s1}{Returns}\PYG{l+s+s1}{\PYGZsq{}}\PYG{p}{]}\PYG{p}{]}
    \PYG{o}{.}\PYG{n}{mul}\PYG{p}{(}\PYG{l+m+mi}{100}\PYG{p}{)}
    \PYG{o}{.}\PYG{n}{plot}\PYG{p}{(}\PYG{n}{x}\PYG{o}{=}\PYG{l+s+s1}{\PYGZsq{}}\PYG{l+s+s1}{Surprise(}\PYG{l+s+s1}{\PYGZpc{}}\PYG{l+s+s1}{)}\PYG{l+s+s1}{\PYGZsq{}}\PYG{p}{,} \PYG{n}{y}\PYG{o}{=}\PYG{l+s+s1}{\PYGZsq{}}\PYG{l+s+s1}{Returns}\PYG{l+s+s1}{\PYGZsq{}}\PYG{p}{,} \PYG{n}{kind}\PYG{o}{=}\PYG{l+s+s1}{\PYGZsq{}}\PYG{l+s+s1}{scatter}\PYG{l+s+s1}{\PYGZsq{}}\PYG{p}{)}
\PYG{p}{)}
\PYG{n}{plt}\PYG{o}{.}\PYG{n}{xlabel}\PYG{p}{(}\PYG{l+s+s1}{\PYGZsq{}}\PYG{l+s+s1}{Earnings Suprise (}\PYG{l+s+s1}{\PYGZpc{}}\PYG{l+s+s1}{)}\PYG{l+s+s1}{\PYGZsq{}}\PYG{p}{)}
\PYG{n}{plt}\PYG{o}{.}\PYG{n}{ylabel}\PYG{p}{(}\PYG{l+s+s1}{\PYGZsq{}}\PYG{l+s+s1}{Announcement Return (}\PYG{l+s+s1}{\PYGZpc{}}\PYG{l+s+s1}{)}\PYG{l+s+s1}{\PYGZsq{}}\PYG{p}{)}

\PYG{n}{\PYGZus{}} \PYG{o}{=} \PYG{l+s+s1}{\PYGZsq{}}\PYG{l+s+s1}{ }\PYG{l+s+s1}{\PYGZsq{}}\PYG{o}{.}\PYG{n}{join}\PYG{p}{(}\PYG{n}{surprises}\PYG{o}{.}\PYG{n}{index}\PYG{o}{.}\PYG{n}{get\PYGZus{}level\PYGZus{}values}\PYG{p}{(}\PYG{l+s+s1}{\PYGZsq{}}\PYG{l+s+s1}{Ticker}\PYG{l+s+s1}{\PYGZsq{}}\PYG{p}{)}\PYG{o}{.}\PYG{n}{unique}\PYG{p}{(}\PYG{p}{)}\PYG{p}{)}
\PYG{n}{\PYGZus{}\PYGZus{}} \PYG{o}{=} \PYG{n}{surprises}\PYG{o}{.}\PYG{n}{index}\PYG{o}{.}\PYG{n}{get\PYGZus{}level\PYGZus{}values}\PYG{p}{(}\PYG{l+s+s1}{\PYGZsq{}}\PYG{l+s+s1}{Date}\PYG{l+s+s1}{\PYGZsq{}}\PYG{p}{)}
\PYG{n}{plt}\PYG{o}{.}\PYG{n}{title}\PYG{p}{(}\PYG{l+s+sa}{f}\PYG{l+s+s1}{\PYGZsq{}}\PYG{l+s+s1}{Earnings Announcements}\PYG{l+s+se}{\PYGZbs{}n}\PYG{l+s+s1}{ for }\PYG{l+s+si}{\PYGZob{}}\PYG{n}{\PYGZus{}}\PYG{l+s+si}{\PYGZcb{}}\PYG{l+s+se}{\PYGZbs{}n}\PYG{l+s+s1}{ from }\PYG{l+s+si}{\PYGZob{}}\PYG{n}{\PYGZus{}\PYGZus{}}\PYG{o}{.}\PYG{n}{min}\PYG{p}{(}\PYG{p}{)}\PYG{l+s+si}{:}\PYG{l+s+s1}{\PYGZpc{}B \PYGZpc{}Y}\PYG{l+s+si}{\PYGZcb{}}\PYG{l+s+s1}{ to }\PYG{l+s+si}{\PYGZob{}}\PYG{n}{\PYGZus{}\PYGZus{}}\PYG{o}{.}\PYG{n}{max}\PYG{p}{(}\PYG{p}{)}\PYG{l+s+si}{:}\PYG{l+s+s1}{\PYGZpc{}B \PYGZpc{}Y}\PYG{l+s+si}{\PYGZcb{}}\PYG{l+s+s1}{\PYGZsq{}}\PYG{p}{)}
\PYG{n}{plt}\PYG{o}{.}\PYG{n}{show}\PYG{p}{(}\PYG{p}{)}
\end{sphinxVerbatim}

\end{sphinxuseclass}\end{sphinxVerbatimInput}
\begin{sphinxVerbatimOutput}

\begin{sphinxuseclass}{cell_output}
\noindent\sphinxincludegraphics{{6036190385fcfd813d2f320098a9f90ae11625e3bb4f1e32f1104d20aaa25966}.png}

\end{sphinxuseclass}\end{sphinxVerbatimOutput}

\end{sphinxuseclass}
\begin{sphinxuseclass}{cell}\begin{sphinxVerbatimInput}

\begin{sphinxuseclass}{cell_input}
\begin{sphinxVerbatim}[commandchars=\\\{\}]
\PYG{k+kn}{import} \PYG{n+nn}{seaborn} \PYG{k}{as} \PYG{n+nn}{sns}
\end{sphinxVerbatim}

\end{sphinxuseclass}\end{sphinxVerbatimInput}

\end{sphinxuseclass}
\begin{sphinxuseclass}{cell}\begin{sphinxVerbatimInput}

\begin{sphinxuseclass}{cell_input}
\begin{sphinxVerbatim}[commandchars=\\\{\}]
\PYG{n}{sns}\PYG{o}{.}\PYG{n}{regplot}\PYG{p}{(}
    \PYG{n}{x}\PYG{o}{=}\PYG{l+s+s1}{\PYGZsq{}}\PYG{l+s+s1}{Surprise(}\PYG{l+s+s1}{\PYGZpc{}}\PYG{l+s+s1}{)}\PYG{l+s+s1}{\PYGZsq{}}\PYG{p}{,}
    \PYG{n}{y}\PYG{o}{=} \PYG{l+s+s1}{\PYGZsq{}}\PYG{l+s+s1}{Returns}\PYG{l+s+s1}{\PYGZsq{}}\PYG{p}{,}
    \PYG{n}{data}\PYG{o}{=}\PYG{n}{surprises}\PYG{p}{[}\PYG{p}{[}\PYG{l+s+s1}{\PYGZsq{}}\PYG{l+s+s1}{Surprise(}\PYG{l+s+s1}{\PYGZpc{}}\PYG{l+s+s1}{)}\PYG{l+s+s1}{\PYGZsq{}}\PYG{p}{,} \PYG{l+s+s1}{\PYGZsq{}}\PYG{l+s+s1}{Returns}\PYG{l+s+s1}{\PYGZsq{}}\PYG{p}{]}\PYG{p}{]}\PYG{o}{.}\PYG{n}{mul}\PYG{p}{(}\PYG{l+m+mi}{100}\PYG{p}{)}
\PYG{p}{)}

\PYG{n}{plt}\PYG{o}{.}\PYG{n}{xlabel}\PYG{p}{(}\PYG{l+s+s1}{\PYGZsq{}}\PYG{l+s+s1}{Earnings Suprise (}\PYG{l+s+s1}{\PYGZpc{}}\PYG{l+s+s1}{)}\PYG{l+s+s1}{\PYGZsq{}}\PYG{p}{)}
\PYG{n}{plt}\PYG{o}{.}\PYG{n}{ylabel}\PYG{p}{(}\PYG{l+s+s1}{\PYGZsq{}}\PYG{l+s+s1}{Announcement Return (}\PYG{l+s+s1}{\PYGZpc{}}\PYG{l+s+s1}{)}\PYG{l+s+s1}{\PYGZsq{}}\PYG{p}{)}

\PYG{n}{\PYGZus{}} \PYG{o}{=} \PYG{l+s+s1}{\PYGZsq{}}\PYG{l+s+s1}{ }\PYG{l+s+s1}{\PYGZsq{}}\PYG{o}{.}\PYG{n}{join}\PYG{p}{(}\PYG{n}{surprises}\PYG{o}{.}\PYG{n}{index}\PYG{o}{.}\PYG{n}{get\PYGZus{}level\PYGZus{}values}\PYG{p}{(}\PYG{l+s+s1}{\PYGZsq{}}\PYG{l+s+s1}{Ticker}\PYG{l+s+s1}{\PYGZsq{}}\PYG{p}{)}\PYG{o}{.}\PYG{n}{unique}\PYG{p}{(}\PYG{p}{)}\PYG{p}{)}
\PYG{n}{\PYGZus{}\PYGZus{}} \PYG{o}{=} \PYG{n}{surprises}\PYG{o}{.}\PYG{n}{index}\PYG{o}{.}\PYG{n}{get\PYGZus{}level\PYGZus{}values}\PYG{p}{(}\PYG{l+s+s1}{\PYGZsq{}}\PYG{l+s+s1}{Date}\PYG{l+s+s1}{\PYGZsq{}}\PYG{p}{)}
\PYG{n}{plt}\PYG{o}{.}\PYG{n}{title}\PYG{p}{(}\PYG{l+s+sa}{f}\PYG{l+s+s1}{\PYGZsq{}}\PYG{l+s+s1}{Earnings Announcements}\PYG{l+s+se}{\PYGZbs{}n}\PYG{l+s+s1}{ for }\PYG{l+s+si}{\PYGZob{}}\PYG{n}{\PYGZus{}}\PYG{l+s+si}{\PYGZcb{}}\PYG{l+s+se}{\PYGZbs{}n}\PYG{l+s+s1}{ from }\PYG{l+s+si}{\PYGZob{}}\PYG{n}{\PYGZus{}\PYGZus{}}\PYG{o}{.}\PYG{n}{min}\PYG{p}{(}\PYG{p}{)}\PYG{l+s+si}{:}\PYG{l+s+s1}{\PYGZpc{}B \PYGZpc{}Y}\PYG{l+s+si}{\PYGZcb{}}\PYG{l+s+s1}{ to }\PYG{l+s+si}{\PYGZob{}}\PYG{n}{\PYGZus{}\PYGZus{}}\PYG{o}{.}\PYG{n}{max}\PYG{p}{(}\PYG{p}{)}\PYG{l+s+si}{:}\PYG{l+s+s1}{\PYGZpc{}B \PYGZpc{}Y}\PYG{l+s+si}{\PYGZcb{}}\PYG{l+s+s1}{\PYGZsq{}}\PYG{p}{)}
\PYG{n}{plt}\PYG{o}{.}\PYG{n}{show}\PYG{p}{(}\PYG{p}{)}
\end{sphinxVerbatim}

\end{sphinxuseclass}\end{sphinxVerbatimInput}
\begin{sphinxVerbatimOutput}

\begin{sphinxuseclass}{cell_output}
\noindent\sphinxincludegraphics{{dc1ca221051ce4ce030134b7024344be192a02c46806937ccd8cd7f2947c9654}.png}

\end{sphinxuseclass}\end{sphinxVerbatimOutput}

\end{sphinxuseclass}
\begin{sphinxuseclass}{cell}\begin{sphinxVerbatimInput}

\begin{sphinxuseclass}{cell_input}
\begin{sphinxVerbatim}[commandchars=\\\{\}]
\PYG{n}{surprises}\PYG{p}{[}\PYG{l+s+s1}{\PYGZsq{}}\PYG{l+s+s1}{ESQ}\PYG{l+s+s1}{\PYGZsq{}}\PYG{p}{]} \PYG{o}{=} \PYG{n}{pd}\PYG{o}{.}\PYG{n}{qcut}\PYG{p}{(}\PYG{n}{x}\PYG{o}{=}\PYG{n}{surprises}\PYG{p}{[}\PYG{l+s+s1}{\PYGZsq{}}\PYG{l+s+s1}{Surprise(}\PYG{l+s+s1}{\PYGZpc{}}\PYG{l+s+s1}{)}\PYG{l+s+s1}{\PYGZsq{}}\PYG{p}{]}\PYG{p}{,} \PYG{n}{q}\PYG{o}{=}\PYG{l+m+mi}{5}\PYG{p}{,} \PYG{n}{labels}\PYG{o}{=}\PYG{k+kc}{False}\PYG{p}{)}
\end{sphinxVerbatim}

\end{sphinxuseclass}\end{sphinxVerbatimInput}

\end{sphinxuseclass}
\begin{sphinxuseclass}{cell}\begin{sphinxVerbatimInput}

\begin{sphinxuseclass}{cell_input}
\begin{sphinxVerbatim}[commandchars=\\\{\}]
\PYG{n}{sns}\PYG{o}{.}\PYG{n}{barplot}\PYG{p}{(}
    \PYG{n}{x}\PYG{o}{=}\PYG{l+s+s1}{\PYGZsq{}}\PYG{l+s+s1}{ESQ}\PYG{l+s+s1}{\PYGZsq{}}\PYG{p}{,}
    \PYG{n}{y}\PYG{o}{=} \PYG{l+s+s1}{\PYGZsq{}}\PYG{l+s+s1}{Returns}\PYG{l+s+s1}{\PYGZsq{}}\PYG{p}{,}
    \PYG{n}{data}\PYG{o}{=}\PYG{p}{(}
        \PYG{n}{surprises}
        \PYG{p}{[}\PYG{p}{[}\PYG{l+s+s1}{\PYGZsq{}}\PYG{l+s+s1}{Surprise(}\PYG{l+s+s1}{\PYGZpc{}}\PYG{l+s+s1}{)}\PYG{l+s+s1}{\PYGZsq{}}\PYG{p}{,} \PYG{l+s+s1}{\PYGZsq{}}\PYG{l+s+s1}{Returns}\PYG{l+s+s1}{\PYGZsq{}}\PYG{p}{]}\PYG{p}{]}
        \PYG{o}{.}\PYG{n}{mul}\PYG{p}{(}\PYG{l+m+mi}{100}\PYG{p}{)}
        \PYG{o}{.}\PYG{n}{assign}\PYG{p}{(}\PYG{n}{ESQ} \PYG{o}{=} \PYG{k}{lambda} \PYG{n}{x}\PYG{p}{:} \PYG{n}{pd}\PYG{o}{.}\PYG{n}{qcut}\PYG{p}{(}\PYG{n}{x}\PYG{o}{=}\PYG{n}{x}\PYG{p}{[}\PYG{l+s+s1}{\PYGZsq{}}\PYG{l+s+s1}{Surprise(}\PYG{l+s+s1}{\PYGZpc{}}\PYG{l+s+s1}{)}\PYG{l+s+s1}{\PYGZsq{}}\PYG{p}{]}\PYG{p}{,} \PYG{n}{q}\PYG{o}{=}\PYG{l+m+mi}{5}\PYG{p}{,} \PYG{n}{labels}\PYG{o}{=}\PYG{k+kc}{False}\PYG{p}{)}\PYG{p}{)}
    \PYG{p}{)}
\PYG{p}{)}

\PYG{n}{plt}\PYG{o}{.}\PYG{n}{xlabel}\PYG{p}{(}\PYG{l+s+s1}{\PYGZsq{}}\PYG{l+s+s1}{Earnings Suprise Portfolio}\PYG{l+s+s1}{\PYGZsq{}}\PYG{p}{)}
\PYG{n}{plt}\PYG{o}{.}\PYG{n}{ylabel}\PYG{p}{(}\PYG{l+s+s1}{\PYGZsq{}}\PYG{l+s+s1}{Announcement Return (}\PYG{l+s+s1}{\PYGZpc{}}\PYG{l+s+s1}{)}\PYG{l+s+s1}{\PYGZsq{}}\PYG{p}{)}

\PYG{n}{\PYGZus{}} \PYG{o}{=} \PYG{l+s+s1}{\PYGZsq{}}\PYG{l+s+s1}{ }\PYG{l+s+s1}{\PYGZsq{}}\PYG{o}{.}\PYG{n}{join}\PYG{p}{(}\PYG{n}{surprises}\PYG{o}{.}\PYG{n}{index}\PYG{o}{.}\PYG{n}{get\PYGZus{}level\PYGZus{}values}\PYG{p}{(}\PYG{l+s+s1}{\PYGZsq{}}\PYG{l+s+s1}{Ticker}\PYG{l+s+s1}{\PYGZsq{}}\PYG{p}{)}\PYG{o}{.}\PYG{n}{unique}\PYG{p}{(}\PYG{p}{)}\PYG{p}{)}
\PYG{n}{\PYGZus{}\PYGZus{}} \PYG{o}{=} \PYG{n}{surprises}\PYG{o}{.}\PYG{n}{index}\PYG{o}{.}\PYG{n}{get\PYGZus{}level\PYGZus{}values}\PYG{p}{(}\PYG{l+s+s1}{\PYGZsq{}}\PYG{l+s+s1}{Date}\PYG{l+s+s1}{\PYGZsq{}}\PYG{p}{)}
\PYG{n}{plt}\PYG{o}{.}\PYG{n}{title}\PYG{p}{(}\PYG{l+s+sa}{f}\PYG{l+s+s1}{\PYGZsq{}}\PYG{l+s+s1}{Earnings Announcements}\PYG{l+s+se}{\PYGZbs{}n}\PYG{l+s+s1}{ for }\PYG{l+s+si}{\PYGZob{}}\PYG{n}{\PYGZus{}}\PYG{l+s+si}{\PYGZcb{}}\PYG{l+s+se}{\PYGZbs{}n}\PYG{l+s+s1}{ from }\PYG{l+s+si}{\PYGZob{}}\PYG{n}{\PYGZus{}\PYGZus{}}\PYG{o}{.}\PYG{n}{min}\PYG{p}{(}\PYG{p}{)}\PYG{l+s+si}{:}\PYG{l+s+s1}{\PYGZpc{}B \PYGZpc{}Y}\PYG{l+s+si}{\PYGZcb{}}\PYG{l+s+s1}{ to }\PYG{l+s+si}{\PYGZob{}}\PYG{n}{\PYGZus{}\PYGZus{}}\PYG{o}{.}\PYG{n}{max}\PYG{p}{(}\PYG{p}{)}\PYG{l+s+si}{:}\PYG{l+s+s1}{\PYGZpc{}B \PYGZpc{}Y}\PYG{l+s+si}{\PYGZcb{}}\PYG{l+s+s1}{\PYGZsq{}}\PYG{p}{)}
\PYG{n}{plt}\PYG{o}{.}\PYG{n}{show}\PYG{p}{(}\PYG{p}{)}
\end{sphinxVerbatim}

\end{sphinxuseclass}\end{sphinxVerbatimInput}
\begin{sphinxVerbatimOutput}

\begin{sphinxuseclass}{cell_output}
\noindent\sphinxincludegraphics{{a80d0270dc1479b187d29731d35f264e28734c5b7c6dce2c31830f489c01a710}.png}

\end{sphinxuseclass}\end{sphinxVerbatimOutput}

\end{sphinxuseclass}
\sphinxAtStartPar
\sphinxstyleemphasis{\sphinxstylestrong{There is a positive relation between announcment returns and earnings surprises!}}
Of course, to say more we need more data and to control for market movements, but this analaysis is a start!


\subsubsection{Repeat the earnings exercise with the S\&P 100 stocks}
\label{\detokenize{mckinney_08_practice_02:repeat-the-earnings-exercise-with-the-s-p-100-stocks}}
\begin{sphinxuseclass}{cell}\begin{sphinxVerbatimInput}

\begin{sphinxuseclass}{cell_input}
\begin{sphinxVerbatim}[commandchars=\\\{\}]
\PYG{n}{wiki} \PYG{o}{=} \PYG{n}{pd}\PYG{o}{.}\PYG{n}{read\PYGZus{}html}\PYG{p}{(}\PYG{l+s+s1}{\PYGZsq{}}\PYG{l+s+s1}{https://en.wikipedia.org/wiki/S}\PYG{l+s+s1}{\PYGZpc{}}\PYG{l+s+s1}{26P\PYGZus{}100}\PYG{l+s+s1}{\PYGZsq{}}\PYG{p}{)}
\end{sphinxVerbatim}

\end{sphinxuseclass}\end{sphinxVerbatimInput}

\end{sphinxuseclass}
\begin{sphinxuseclass}{cell}\begin{sphinxVerbatimInput}

\begin{sphinxuseclass}{cell_input}
\begin{sphinxVerbatim}[commandchars=\\\{\}]
\PYG{n}{symbols} \PYG{o}{=} \PYG{n}{wiki}\PYG{p}{[}\PYG{l+m+mi}{2}\PYG{p}{]}\PYG{p}{[}\PYG{l+s+s1}{\PYGZsq{}}\PYG{l+s+s1}{Symbol}\PYG{l+s+s1}{\PYGZsq{}}\PYG{p}{]}\PYG{o}{.}\PYG{n}{str}\PYG{o}{.}\PYG{n}{replace}\PYG{p}{(}\PYG{l+s+s1}{\PYGZsq{}}\PYG{l+s+s1}{.}\PYG{l+s+s1}{\PYGZsq{}}\PYG{p}{,} \PYG{l+s+s1}{\PYGZsq{}}\PYG{l+s+s1}{\PYGZhy{}}\PYG{l+s+s1}{\PYGZsq{}}\PYG{p}{,} \PYG{n}{regex}\PYG{o}{=}\PYG{k+kc}{False}\PYG{p}{)}\PYG{o}{.}\PYG{n}{to\PYGZus{}list}\PYG{p}{(}\PYG{p}{)}
\end{sphinxVerbatim}

\end{sphinxuseclass}\end{sphinxVerbatimInput}

\end{sphinxuseclass}
\begin{sphinxuseclass}{cell}\begin{sphinxVerbatimInput}

\begin{sphinxuseclass}{cell_input}
\begin{sphinxVerbatim}[commandchars=\\\{\}]
\PYG{n}{tickers\PYGZus{}2} \PYG{o}{=} \PYG{n}{yf}\PYG{o}{.}\PYG{n}{Tickers}\PYG{p}{(}\PYG{n}{tickers}\PYG{o}{=}\PYG{n}{symbols}\PYG{p}{,} \PYG{n}{session}\PYG{o}{=}\PYG{n}{session}\PYG{p}{)}
\end{sphinxVerbatim}

\end{sphinxuseclass}\end{sphinxVerbatimInput}

\end{sphinxuseclass}
\begin{sphinxuseclass}{cell}\begin{sphinxVerbatimInput}

\begin{sphinxuseclass}{cell_input}
\begin{sphinxVerbatim}[commandchars=\\\{\}]
\PYG{n}{returns\PYGZus{}2} \PYG{o}{=} \PYG{p}{(}
    \PYG{n}{tickers\PYGZus{}2}
    \PYG{o}{.}\PYG{n}{history}\PYG{p}{(}\PYG{n}{period}\PYG{o}{=}\PYG{l+s+s1}{\PYGZsq{}}\PYG{l+s+s1}{max}\PYG{l+s+s1}{\PYGZsq{}}\PYG{p}{,} \PYG{n}{auto\PYGZus{}adjust}\PYG{o}{=}\PYG{k+kc}{False}\PYG{p}{,} \PYG{n}{progress}\PYG{o}{=}\PYG{k+kc}{False}\PYG{p}{)}
    \PYG{o}{.}\PYG{n}{rename\PYGZus{}axis}\PYG{p}{(}\PYG{n}{columns}\PYG{o}{=}\PYG{p}{[}\PYG{l+s+s1}{\PYGZsq{}}\PYG{l+s+s1}{Variable}\PYG{l+s+s1}{\PYGZsq{}}\PYG{p}{,} \PYG{l+s+s1}{\PYGZsq{}}\PYG{l+s+s1}{Ticker}\PYG{l+s+s1}{\PYGZsq{}}\PYG{p}{]}\PYG{p}{)}
    \PYG{p}{[}\PYG{l+s+s1}{\PYGZsq{}}\PYG{l+s+s1}{Adj Close}\PYG{l+s+s1}{\PYGZsq{}}\PYG{p}{]}
    \PYG{o}{.}\PYG{n}{pct\PYGZus{}change}\PYG{p}{(}\PYG{p}{)}
    \PYG{o}{.}\PYG{n}{assign}\PYG{p}{(}\PYG{n}{Date}\PYG{o}{=}\PYG{k}{lambda} \PYG{n}{x}\PYG{p}{:} \PYG{n}{x}\PYG{o}{.}\PYG{n}{index}\PYG{o}{.}\PYG{n}{tz\PYGZus{}localize}\PYG{p}{(}\PYG{l+s+s1}{\PYGZsq{}}\PYG{l+s+s1}{America/New\PYGZus{}York}\PYG{l+s+s1}{\PYGZsq{}}\PYG{p}{)} \PYG{o}{+} \PYG{n}{pd}\PYG{o}{.}\PYG{n}{to\PYGZus{}timedelta}\PYG{p}{(}\PYG{l+m+mi}{16}\PYG{p}{,} \PYG{n}{unit}\PYG{o}{=}\PYG{l+s+s1}{\PYGZsq{}}\PYG{l+s+s1}{H}\PYG{l+s+s1}{\PYGZsq{}}\PYG{p}{)}\PYG{p}{)}
    \PYG{o}{.}\PYG{n}{set\PYGZus{}index}\PYG{p}{(}\PYG{l+s+s1}{\PYGZsq{}}\PYG{l+s+s1}{Date}\PYG{l+s+s1}{\PYGZsq{}}\PYG{p}{)}
\PYG{p}{)}

\PYG{n}{returns\PYGZus{}2}\PYG{o}{.}\PYG{n}{head}\PYG{p}{(}\PYG{p}{)}
\end{sphinxVerbatim}

\end{sphinxuseclass}\end{sphinxVerbatimInput}
\begin{sphinxVerbatimOutput}

\begin{sphinxuseclass}{cell_output}
\begin{sphinxVerbatim}[commandchars=\\\{\}]
Ticker                     AAPL  ABBV  ABT  ACN  ADBE  AIG  AMD  AMGN  AMT  \PYGZbs{}
Date                                                                         
1962\PYGZhy{}01\PYGZhy{}02 16:00:00\PYGZhy{}05:00   NaN   NaN  NaN  NaN   NaN  NaN  NaN   NaN  NaN   
1962\PYGZhy{}01\PYGZhy{}03 16:00:00\PYGZhy{}05:00   NaN   NaN  NaN  NaN   NaN  NaN  NaN   NaN  NaN   
1962\PYGZhy{}01\PYGZhy{}04 16:00:00\PYGZhy{}05:00   NaN   NaN  NaN  NaN   NaN  NaN  NaN   NaN  NaN   
1962\PYGZhy{}01\PYGZhy{}05 16:00:00\PYGZhy{}05:00   NaN   NaN  NaN  NaN   NaN  NaN  NaN   NaN  NaN   
1962\PYGZhy{}01\PYGZhy{}08 16:00:00\PYGZhy{}05:00   NaN   NaN  NaN  NaN   NaN  NaN  NaN   NaN  NaN   

Ticker                     AMZN  ...  UNH  UNP  UPS  USB   V  VZ  WBA  WFC  \PYGZbs{}
Date                             ...                                         
1962\PYGZhy{}01\PYGZhy{}02 16:00:00\PYGZhy{}05:00   NaN  ...  NaN  NaN  NaN  NaN NaN NaN  NaN  NaN   
1962\PYGZhy{}01\PYGZhy{}03 16:00:00\PYGZhy{}05:00   NaN  ...  NaN  NaN  NaN  NaN NaN NaN  NaN  NaN   
1962\PYGZhy{}01\PYGZhy{}04 16:00:00\PYGZhy{}05:00   NaN  ...  NaN  NaN  NaN  NaN NaN NaN  NaN  NaN   
1962\PYGZhy{}01\PYGZhy{}05 16:00:00\PYGZhy{}05:00   NaN  ...  NaN  NaN  NaN  NaN NaN NaN  NaN  NaN   
1962\PYGZhy{}01\PYGZhy{}08 16:00:00\PYGZhy{}05:00   NaN  ...  NaN  NaN  NaN  NaN NaN NaN  NaN  NaN   

Ticker                     WMT     XOM  
Date                                    
1962\PYGZhy{}01\PYGZhy{}02 16:00:00\PYGZhy{}05:00  NaN     NaN  
1962\PYGZhy{}01\PYGZhy{}03 16:00:00\PYGZhy{}05:00  NaN  0.0149  
1962\PYGZhy{}01\PYGZhy{}04 16:00:00\PYGZhy{}05:00  NaN  0.0024  
1962\PYGZhy{}01\PYGZhy{}05 16:00:00\PYGZhy{}05:00  NaN \PYGZhy{}0.0219  
1962\PYGZhy{}01\PYGZhy{}08 16:00:00\PYGZhy{}05:00  NaN \PYGZhy{}0.0025  

[5 rows x 101 columns]
\end{sphinxVerbatim}

\end{sphinxuseclass}\end{sphinxVerbatimOutput}

\end{sphinxuseclass}
\begin{sphinxuseclass}{cell}\begin{sphinxVerbatimInput}

\begin{sphinxuseclass}{cell_input}
\begin{sphinxVerbatim}[commandchars=\\\{\}]
\PYG{n}{earnings\PYGZus{}2} \PYG{o}{=} \PYG{p}{(}
    \PYG{n}{pd}\PYG{o}{.}\PYG{n}{concat}\PYG{p}{(}
        \PYG{n}{objs}\PYG{o}{=}\PYG{p}{[}\PYG{n}{tickers\PYGZus{}2}\PYG{o}{.}\PYG{n}{tickers}\PYG{p}{[}\PYG{n}{t}\PYG{p}{]}\PYG{o}{.}\PYG{n}{earnings\PYGZus{}dates} \PYG{k}{for} \PYG{n}{t} \PYG{o+ow}{in} \PYG{n}{tickers\PYGZus{}2}\PYG{o}{.}\PYG{n}{tickers}\PYG{p}{]}\PYG{p}{,}
        \PYG{n}{keys}\PYG{o}{=}\PYG{n}{tickers\PYGZus{}2}\PYG{o}{.}\PYG{n}{tickers}\PYG{p}{,}
        \PYG{n}{names}\PYG{o}{=}\PYG{p}{[}\PYG{l+s+s1}{\PYGZsq{}}\PYG{l+s+s1}{Ticker}\PYG{l+s+s1}{\PYGZsq{}}\PYG{p}{,} \PYG{l+s+s1}{\PYGZsq{}}\PYG{l+s+s1}{Date}\PYG{l+s+s1}{\PYGZsq{}}\PYG{p}{]}
    \PYG{p}{)}
    \PYG{o}{.}\PYG{n}{rename\PYGZus{}axis}\PYG{p}{(}\PYG{n}{columns}\PYG{o}{=}\PYG{l+s+s1}{\PYGZsq{}}\PYG{l+s+s1}{Variable}\PYG{l+s+s1}{\PYGZsq{}}\PYG{p}{)}
\PYG{p}{)}

\PYG{n}{earnings\PYGZus{}2}\PYG{o}{.}\PYG{n}{head}\PYG{p}{(}\PYG{p}{)}
\end{sphinxVerbatim}

\end{sphinxuseclass}\end{sphinxVerbatimInput}
\begin{sphinxVerbatimOutput}

\begin{sphinxuseclass}{cell_output}
\begin{sphinxVerbatim}[commandchars=\\\{\}]
Variable                          EPS Estimate  Reported EPS  Surprise(\PYGZpc{})
Ticker Date                                                              
AAPL   2024\PYGZhy{}01\PYGZhy{}31 16:00:00\PYGZhy{}05:00           NaN           NaN          NaN
       2024\PYGZhy{}01\PYGZhy{}31 05:00:00\PYGZhy{}05:00           NaN           NaN          NaN
       2023\PYGZhy{}10\PYGZhy{}25 06:00:00\PYGZhy{}04:00           NaN           NaN          NaN
       2023\PYGZhy{}07\PYGZhy{}26 16:00:00\PYGZhy{}04:00           NaN           NaN          NaN
       2023\PYGZhy{}04\PYGZhy{}26 06:00:00\PYGZhy{}04:00        1.4300           NaN          NaN
\end{sphinxVerbatim}

\end{sphinxuseclass}\end{sphinxVerbatimOutput}

\end{sphinxuseclass}
\begin{sphinxuseclass}{cell}\begin{sphinxVerbatimInput}

\begin{sphinxuseclass}{cell_input}
\begin{sphinxVerbatim}[commandchars=\\\{\}]
\PYG{n}{surprises\PYGZus{}2} \PYG{o}{=} \PYG{p}{(}
    \PYG{n}{pd}\PYG{o}{.}\PYG{n}{merge\PYGZus{}asof}\PYG{p}{(}
        \PYG{n}{left}\PYG{o}{=}\PYG{n}{earnings\PYGZus{}2}\PYG{o}{.}\PYG{n}{sort\PYGZus{}index}\PYG{p}{(}\PYG{n}{level}\PYG{o}{=}\PYG{p}{[}\PYG{l+s+s1}{\PYGZsq{}}\PYG{l+s+s1}{Date}\PYG{l+s+s1}{\PYGZsq{}}\PYG{p}{,} \PYG{l+s+s1}{\PYGZsq{}}\PYG{l+s+s1}{Ticker}\PYG{l+s+s1}{\PYGZsq{}}\PYG{p}{]}\PYG{p}{)}\PYG{p}{,}
        \PYG{n}{right}\PYG{o}{=}\PYG{n}{returns\PYGZus{}2}\PYG{o}{.}\PYG{n}{stack}\PYG{p}{(}\PYG{p}{)}\PYG{o}{.}\PYG{n}{to\PYGZus{}frame}\PYG{p}{(}\PYG{l+s+s1}{\PYGZsq{}}\PYG{l+s+s1}{Returns}\PYG{l+s+s1}{\PYGZsq{}}\PYG{p}{)}\PYG{o}{.}\PYG{n}{swaplevel}\PYG{p}{(}\PYG{p}{)}\PYG{o}{.}\PYG{n}{sort\PYGZus{}index}\PYG{p}{(}\PYG{n}{level}\PYG{o}{=}\PYG{p}{[}\PYG{l+s+s1}{\PYGZsq{}}\PYG{l+s+s1}{Date}\PYG{l+s+s1}{\PYGZsq{}}\PYG{p}{,} \PYG{l+s+s1}{\PYGZsq{}}\PYG{l+s+s1}{Ticker}\PYG{l+s+s1}{\PYGZsq{}}\PYG{p}{]}\PYG{p}{)}\PYG{p}{,}
        \PYG{n}{on}\PYG{o}{=}\PYG{l+s+s1}{\PYGZsq{}}\PYG{l+s+s1}{Date}\PYG{l+s+s1}{\PYGZsq{}}\PYG{p}{,}
        \PYG{n}{by}\PYG{o}{=}\PYG{l+s+s1}{\PYGZsq{}}\PYG{l+s+s1}{Ticker}\PYG{l+s+s1}{\PYGZsq{}}\PYG{p}{,}
        \PYG{n}{direction}\PYG{o}{=}\PYG{l+s+s1}{\PYGZsq{}}\PYG{l+s+s1}{forward}\PYG{l+s+s1}{\PYGZsq{}}\PYG{p}{,}
        \PYG{n}{allow\PYGZus{}exact\PYGZus{}matches}\PYG{o}{=}\PYG{k+kc}{False}
    \PYG{p}{)}
    \PYG{o}{.}\PYG{n}{dropna}\PYG{p}{(}\PYG{p}{)}
    \PYG{o}{.}\PYG{n}{set\PYGZus{}index}\PYG{p}{(}\PYG{p}{[}\PYG{l+s+s1}{\PYGZsq{}}\PYG{l+s+s1}{Date}\PYG{l+s+s1}{\PYGZsq{}}\PYG{p}{,} \PYG{l+s+s1}{\PYGZsq{}}\PYG{l+s+s1}{Ticker}\PYG{l+s+s1}{\PYGZsq{}}\PYG{p}{]}\PYG{p}{)}
\PYG{p}{)}
\end{sphinxVerbatim}

\end{sphinxuseclass}\end{sphinxVerbatimInput}

\end{sphinxuseclass}
\begin{sphinxuseclass}{cell}\begin{sphinxVerbatimInput}

\begin{sphinxuseclass}{cell_input}
\begin{sphinxVerbatim}[commandchars=\\\{\}]
\PYG{n}{sns}\PYG{o}{.}\PYG{n}{barplot}\PYG{p}{(}
    \PYG{n}{x}\PYG{o}{=}\PYG{l+s+s1}{\PYGZsq{}}\PYG{l+s+s1}{ESQ}\PYG{l+s+s1}{\PYGZsq{}}\PYG{p}{,}
    \PYG{n}{y}\PYG{o}{=} \PYG{l+s+s1}{\PYGZsq{}}\PYG{l+s+s1}{Returns}\PYG{l+s+s1}{\PYGZsq{}}\PYG{p}{,}
    \PYG{n}{data}\PYG{o}{=}\PYG{p}{(}
        \PYG{n}{surprises\PYGZus{}2}
        \PYG{p}{[}\PYG{p}{[}\PYG{l+s+s1}{\PYGZsq{}}\PYG{l+s+s1}{Surprise(}\PYG{l+s+s1}{\PYGZpc{}}\PYG{l+s+s1}{)}\PYG{l+s+s1}{\PYGZsq{}}\PYG{p}{,} \PYG{l+s+s1}{\PYGZsq{}}\PYG{l+s+s1}{Returns}\PYG{l+s+s1}{\PYGZsq{}}\PYG{p}{]}\PYG{p}{]}
        \PYG{o}{.}\PYG{n}{mul}\PYG{p}{(}\PYG{l+m+mi}{100}\PYG{p}{)}
        \PYG{o}{.}\PYG{n}{assign}\PYG{p}{(}\PYG{n}{ESQ} \PYG{o}{=} \PYG{k}{lambda} \PYG{n}{x}\PYG{p}{:} \PYG{n}{pd}\PYG{o}{.}\PYG{n}{qcut}\PYG{p}{(}\PYG{n}{x}\PYG{o}{=}\PYG{n}{x}\PYG{p}{[}\PYG{l+s+s1}{\PYGZsq{}}\PYG{l+s+s1}{Surprise(}\PYG{l+s+s1}{\PYGZpc{}}\PYG{l+s+s1}{)}\PYG{l+s+s1}{\PYGZsq{}}\PYG{p}{]}\PYG{p}{,} \PYG{n}{q}\PYG{o}{=}\PYG{l+m+mi}{5}\PYG{p}{,} \PYG{n}{labels}\PYG{o}{=}\PYG{k+kc}{False}\PYG{p}{)}\PYG{p}{)}
    \PYG{p}{)}
\PYG{p}{)}

\PYG{n}{plt}\PYG{o}{.}\PYG{n}{xlabel}\PYG{p}{(}\PYG{l+s+s1}{\PYGZsq{}}\PYG{l+s+s1}{Earnings Suprise Portfolio}\PYG{l+s+s1}{\PYGZsq{}}\PYG{p}{)}
\PYG{n}{plt}\PYG{o}{.}\PYG{n}{ylabel}\PYG{p}{(}\PYG{l+s+s1}{\PYGZsq{}}\PYG{l+s+s1}{Announcement Return (}\PYG{l+s+s1}{\PYGZpc{}}\PYG{l+s+s1}{)}\PYG{l+s+s1}{\PYGZsq{}}\PYG{p}{)}

\PYG{n}{\PYGZus{}\PYGZus{}} \PYG{o}{=} \PYG{n}{surprises\PYGZus{}2}\PYG{o}{.}\PYG{n}{index}\PYG{o}{.}\PYG{n}{get\PYGZus{}level\PYGZus{}values}\PYG{p}{(}\PYG{l+s+s1}{\PYGZsq{}}\PYG{l+s+s1}{Date}\PYG{l+s+s1}{\PYGZsq{}}\PYG{p}{)}
\PYG{n}{plt}\PYG{o}{.}\PYG{n}{title}\PYG{p}{(}\PYG{l+s+sa}{f}\PYG{l+s+s1}{\PYGZsq{}}\PYG{l+s+s1}{Earnings Announcements for S\PYGZam{}P 100 Stocks }\PYG{l+s+se}{\PYGZbs{}n}\PYG{l+s+s1}{ from }\PYG{l+s+si}{\PYGZob{}}\PYG{n}{\PYGZus{}\PYGZus{}}\PYG{o}{.}\PYG{n}{min}\PYG{p}{(}\PYG{p}{)}\PYG{l+s+si}{:}\PYG{l+s+s1}{\PYGZpc{}B \PYGZpc{}Y}\PYG{l+s+si}{\PYGZcb{}}\PYG{l+s+s1}{ to }\PYG{l+s+si}{\PYGZob{}}\PYG{n}{\PYGZus{}\PYGZus{}}\PYG{o}{.}\PYG{n}{max}\PYG{p}{(}\PYG{p}{)}\PYG{l+s+si}{:}\PYG{l+s+s1}{\PYGZpc{}B \PYGZpc{}Y}\PYG{l+s+si}{\PYGZcb{}}\PYG{l+s+s1}{\PYGZsq{}}\PYG{p}{)}
\PYG{n}{plt}\PYG{o}{.}\PYG{n}{show}\PYG{p}{(}\PYG{p}{)}
\end{sphinxVerbatim}

\end{sphinxuseclass}\end{sphinxVerbatimInput}
\begin{sphinxVerbatimOutput}

\begin{sphinxuseclass}{cell_output}
\noindent\sphinxincludegraphics{{55dd04017f0e88f20e021110b03dcd8f474ab8365abd21c508a3586830e1b664}.png}

\end{sphinxuseclass}\end{sphinxVerbatimOutput}

\end{sphinxuseclass}

\subsubsection{Repeat the earnings exercise with \sphinxstyleemphasis{excess returns} of the S\&P 100 Stocks}
\label{\detokenize{mckinney_08_practice_02:repeat-the-earnings-exercise-with-excess-returns-of-the-s-p-100-stocks}}
\sphinxAtStartPar
Excess returns are returns minus market returns.
We need to add a timezone and the closing time to the market return from Fama and French.

\begin{sphinxuseclass}{cell}\begin{sphinxVerbatimInput}

\begin{sphinxuseclass}{cell_input}
\begin{sphinxVerbatim}[commandchars=\\\{\}]
\PYG{n}{Mkt} \PYG{o}{=} \PYG{n}{ff}\PYG{p}{[}\PYG{l+s+s1}{\PYGZsq{}}\PYG{l+s+s1}{Mkt\PYGZhy{}RF}\PYG{l+s+s1}{\PYGZsq{}}\PYG{p}{]}\PYG{o}{.}\PYG{n}{add}\PYG{p}{(}\PYG{n}{ff}\PYG{p}{[}\PYG{l+s+s1}{\PYGZsq{}}\PYG{l+s+s1}{RF}\PYG{l+s+s1}{\PYGZsq{}}\PYG{p}{]}\PYG{p}{)}
\PYG{n}{Mkt}\PYG{o}{.}\PYG{n}{index} \PYG{o}{=} \PYG{n}{Mkt}\PYG{o}{.}\PYG{n}{index}\PYG{o}{.}\PYG{n}{tz\PYGZus{}localize}\PYG{p}{(}\PYG{l+s+s1}{\PYGZsq{}}\PYG{l+s+s1}{America/New\PYGZus{}York}\PYG{l+s+s1}{\PYGZsq{}}\PYG{p}{)} \PYG{o}{+} \PYG{n}{pd}\PYG{o}{.}\PYG{n}{to\PYGZus{}timedelta}\PYG{p}{(}\PYG{l+m+mi}{16}\PYG{p}{,} \PYG{n}{unit}\PYG{o}{=}\PYG{l+s+s1}{\PYGZsq{}}\PYG{l+s+s1}{H}\PYG{l+s+s1}{\PYGZsq{}}\PYG{p}{)}
\PYG{n}{returns\PYGZus{}3} \PYG{o}{=} \PYG{n}{returns\PYGZus{}2}\PYG{o}{.}\PYG{n}{sub}\PYG{p}{(}\PYG{n}{Mkt}\PYG{p}{,} \PYG{n}{axis}\PYG{o}{=}\PYG{l+m+mi}{0}\PYG{p}{)}
\end{sphinxVerbatim}

\end{sphinxuseclass}\end{sphinxVerbatimInput}

\end{sphinxuseclass}
\begin{sphinxuseclass}{cell}\begin{sphinxVerbatimInput}

\begin{sphinxuseclass}{cell_input}
\begin{sphinxVerbatim}[commandchars=\\\{\}]
\PYG{n}{surprises\PYGZus{}3} \PYG{o}{=} \PYG{p}{(}
    \PYG{n}{pd}\PYG{o}{.}\PYG{n}{merge\PYGZus{}asof}\PYG{p}{(}
        \PYG{n}{left}\PYG{o}{=}\PYG{n}{earnings\PYGZus{}2}\PYG{o}{.}\PYG{n}{sort\PYGZus{}index}\PYG{p}{(}\PYG{n}{level}\PYG{o}{=}\PYG{p}{[}\PYG{l+s+s1}{\PYGZsq{}}\PYG{l+s+s1}{Date}\PYG{l+s+s1}{\PYGZsq{}}\PYG{p}{,} \PYG{l+s+s1}{\PYGZsq{}}\PYG{l+s+s1}{Ticker}\PYG{l+s+s1}{\PYGZsq{}}\PYG{p}{]}\PYG{p}{)}\PYG{p}{,}
        \PYG{n}{right}\PYG{o}{=}\PYG{n}{returns\PYGZus{}3}\PYG{o}{.}\PYG{n}{stack}\PYG{p}{(}\PYG{p}{)}\PYG{o}{.}\PYG{n}{to\PYGZus{}frame}\PYG{p}{(}\PYG{l+s+s1}{\PYGZsq{}}\PYG{l+s+s1}{Excess Returns}\PYG{l+s+s1}{\PYGZsq{}}\PYG{p}{)}\PYG{o}{.}\PYG{n}{swaplevel}\PYG{p}{(}\PYG{p}{)}\PYG{o}{.}\PYG{n}{sort\PYGZus{}index}\PYG{p}{(}\PYG{n}{level}\PYG{o}{=}\PYG{p}{[}\PYG{l+s+s1}{\PYGZsq{}}\PYG{l+s+s1}{Date}\PYG{l+s+s1}{\PYGZsq{}}\PYG{p}{,} \PYG{l+s+s1}{\PYGZsq{}}\PYG{l+s+s1}{Ticker}\PYG{l+s+s1}{\PYGZsq{}}\PYG{p}{]}\PYG{p}{)}\PYG{p}{,}
        \PYG{n}{on}\PYG{o}{=}\PYG{l+s+s1}{\PYGZsq{}}\PYG{l+s+s1}{Date}\PYG{l+s+s1}{\PYGZsq{}}\PYG{p}{,}
        \PYG{n}{by}\PYG{o}{=}\PYG{l+s+s1}{\PYGZsq{}}\PYG{l+s+s1}{Ticker}\PYG{l+s+s1}{\PYGZsq{}}\PYG{p}{,}
        \PYG{n}{direction}\PYG{o}{=}\PYG{l+s+s1}{\PYGZsq{}}\PYG{l+s+s1}{forward}\PYG{l+s+s1}{\PYGZsq{}}\PYG{p}{,}
        \PYG{n}{allow\PYGZus{}exact\PYGZus{}matches}\PYG{o}{=}\PYG{k+kc}{False}
    \PYG{p}{)}
    \PYG{o}{.}\PYG{n}{dropna}\PYG{p}{(}\PYG{p}{)}
    \PYG{o}{.}\PYG{n}{set\PYGZus{}index}\PYG{p}{(}\PYG{p}{[}\PYG{l+s+s1}{\PYGZsq{}}\PYG{l+s+s1}{Date}\PYG{l+s+s1}{\PYGZsq{}}\PYG{p}{,} \PYG{l+s+s1}{\PYGZsq{}}\PYG{l+s+s1}{Ticker}\PYG{l+s+s1}{\PYGZsq{}}\PYG{p}{]}\PYG{p}{)}
\PYG{p}{)}
\end{sphinxVerbatim}

\end{sphinxuseclass}\end{sphinxVerbatimInput}

\end{sphinxuseclass}
\begin{sphinxuseclass}{cell}\begin{sphinxVerbatimInput}

\begin{sphinxuseclass}{cell_input}
\begin{sphinxVerbatim}[commandchars=\\\{\}]
\PYG{n}{sns}\PYG{o}{.}\PYG{n}{barplot}\PYG{p}{(}
    \PYG{n}{x}\PYG{o}{=}\PYG{l+s+s1}{\PYGZsq{}}\PYG{l+s+s1}{ESQ}\PYG{l+s+s1}{\PYGZsq{}}\PYG{p}{,}
    \PYG{n}{y}\PYG{o}{=}\PYG{l+s+s1}{\PYGZsq{}}\PYG{l+s+s1}{Excess Returns}\PYG{l+s+s1}{\PYGZsq{}}\PYG{p}{,}
    \PYG{n}{data}\PYG{o}{=}\PYG{p}{(}
        \PYG{n}{surprises\PYGZus{}3}
        \PYG{p}{[}\PYG{p}{[}\PYG{l+s+s1}{\PYGZsq{}}\PYG{l+s+s1}{Surprise(}\PYG{l+s+s1}{\PYGZpc{}}\PYG{l+s+s1}{)}\PYG{l+s+s1}{\PYGZsq{}}\PYG{p}{,} \PYG{l+s+s1}{\PYGZsq{}}\PYG{l+s+s1}{Excess Returns}\PYG{l+s+s1}{\PYGZsq{}}\PYG{p}{]}\PYG{p}{]}
        \PYG{o}{.}\PYG{n}{mul}\PYG{p}{(}\PYG{l+m+mi}{100}\PYG{p}{)}
        \PYG{o}{.}\PYG{n}{assign}\PYG{p}{(}\PYG{n}{ESQ} \PYG{o}{=} \PYG{k}{lambda} \PYG{n}{x}\PYG{p}{:} \PYG{n}{pd}\PYG{o}{.}\PYG{n}{qcut}\PYG{p}{(}\PYG{n}{x}\PYG{o}{=}\PYG{n}{x}\PYG{p}{[}\PYG{l+s+s1}{\PYGZsq{}}\PYG{l+s+s1}{Surprise(}\PYG{l+s+s1}{\PYGZpc{}}\PYG{l+s+s1}{)}\PYG{l+s+s1}{\PYGZsq{}}\PYG{p}{]}\PYG{p}{,} \PYG{n}{q}\PYG{o}{=}\PYG{l+m+mi}{5}\PYG{p}{,} \PYG{n}{labels}\PYG{o}{=}\PYG{k+kc}{False}\PYG{p}{)}\PYG{p}{)}
    \PYG{p}{)}
\PYG{p}{)}

\PYG{n}{plt}\PYG{o}{.}\PYG{n}{xlabel}\PYG{p}{(}\PYG{l+s+s1}{\PYGZsq{}}\PYG{l+s+s1}{Earnings Suprise Portfolio}\PYG{l+s+s1}{\PYGZsq{}}\PYG{p}{)}
\PYG{n}{plt}\PYG{o}{.}\PYG{n}{ylabel}\PYG{p}{(}\PYG{l+s+s1}{\PYGZsq{}}\PYG{l+s+s1}{Announcement Excess Return (}\PYG{l+s+s1}{\PYGZpc{}}\PYG{l+s+s1}{)}\PYG{l+s+s1}{\PYGZsq{}}\PYG{p}{)}

\PYG{n}{\PYGZus{}\PYGZus{}} \PYG{o}{=} \PYG{n}{surprises\PYGZus{}3}\PYG{o}{.}\PYG{n}{index}\PYG{o}{.}\PYG{n}{get\PYGZus{}level\PYGZus{}values}\PYG{p}{(}\PYG{l+s+s1}{\PYGZsq{}}\PYG{l+s+s1}{Date}\PYG{l+s+s1}{\PYGZsq{}}\PYG{p}{)}
\PYG{n}{plt}\PYG{o}{.}\PYG{n}{title}\PYG{p}{(}\PYG{l+s+sa}{f}\PYG{l+s+s1}{\PYGZsq{}}\PYG{l+s+s1}{Earnings Announcements for S\PYGZam{}P 100 Stocks}\PYG{l+s+se}{\PYGZbs{}n}\PYG{l+s+s1}{ from }\PYG{l+s+si}{\PYGZob{}}\PYG{n}{\PYGZus{}\PYGZus{}}\PYG{o}{.}\PYG{n}{min}\PYG{p}{(}\PYG{p}{)}\PYG{l+s+si}{:}\PYG{l+s+s1}{\PYGZpc{}B \PYGZpc{}Y}\PYG{l+s+si}{\PYGZcb{}}\PYG{l+s+s1}{ to }\PYG{l+s+si}{\PYGZob{}}\PYG{n}{\PYGZus{}\PYGZus{}}\PYG{o}{.}\PYG{n}{max}\PYG{p}{(}\PYG{p}{)}\PYG{l+s+si}{:}\PYG{l+s+s1}{\PYGZpc{}B \PYGZpc{}Y}\PYG{l+s+si}{\PYGZcb{}}\PYG{l+s+s1}{\PYGZsq{}}\PYG{p}{)}
\PYG{n}{plt}\PYG{o}{.}\PYG{n}{show}\PYG{p}{(}\PYG{p}{)}
\end{sphinxVerbatim}

\end{sphinxuseclass}\end{sphinxVerbatimInput}
\begin{sphinxVerbatimOutput}

\begin{sphinxuseclass}{cell_output}
\noindent\sphinxincludegraphics{{04b2e7f0e79a9ade61cd3035cee7ac3cc2afe13377ce408628b1e0f53124d2c4}.png}

\end{sphinxuseclass}\end{sphinxVerbatimOutput}

\end{sphinxuseclass}

\subsubsection{Improve your \sphinxstyleliteralintitle{\sphinxupquote{download()}} function from above}
\label{\detokenize{mckinney_08_practice_02:improve-your-download-function-from-above}}
\sphinxAtStartPar
Modify \sphinxcode{\sphinxupquote{download()}} to accept one or more than one ticker.
Since we will not use the advanced functionality of the tickers object that \sphinxcode{\sphinxupquote{yf.Tickers()}} creates, we will use \sphinxcode{\sphinxupquote{yf.download()}}.
The current version of \sphinxcode{\sphinxupquote{yf.download()}} does not accept a \sphinxcode{\sphinxupquote{session=}} argument.

\begin{sphinxuseclass}{cell}\begin{sphinxVerbatimInput}

\begin{sphinxuseclass}{cell_input}
\begin{sphinxVerbatim}[commandchars=\\\{\}]
\PYG{k}{def} \PYG{n+nf}{download}\PYG{p}{(}\PYG{n}{tickers}\PYG{p}{)}\PYG{p}{:}

    \PYG{n}{histories} \PYG{o}{=} \PYG{p}{(}
        \PYG{n}{yf}\PYG{o}{.}\PYG{n}{download}\PYG{p}{(}\PYG{n}{tickers}\PYG{p}{)}
        \PYG{o}{.}\PYG{n}{assign}\PYG{p}{(}\PYG{n}{Date}\PYG{o}{=}\PYG{k}{lambda} \PYG{n}{x}\PYG{p}{:} \PYG{n}{x}\PYG{o}{.}\PYG{n}{index}\PYG{o}{.}\PYG{n}{tz\PYGZus{}localize}\PYG{p}{(}\PYG{k+kc}{None}\PYG{p}{)}\PYG{p}{)}
        \PYG{o}{.}\PYG{n}{set\PYGZus{}index}\PYG{p}{(}\PYG{l+s+s1}{\PYGZsq{}}\PYG{l+s+s1}{Date}\PYG{l+s+s1}{\PYGZsq{}}\PYG{p}{)}
    \PYG{p}{)}

    \PYG{n}{factors} \PYG{o}{=} \PYG{p}{(}
        \PYG{n}{pdr}\PYG{o}{.}\PYG{n}{DataReader}\PYG{p}{(}
            \PYG{n}{name}\PYG{o}{=}\PYG{l+s+s1}{\PYGZsq{}}\PYG{l+s+s1}{F\PYGZhy{}F\PYGZus{}Research\PYGZus{}Data\PYGZus{}Factors\PYGZus{}daily}\PYG{l+s+s1}{\PYGZsq{}}\PYG{p}{,}
            \PYG{n}{data\PYGZus{}source}\PYG{o}{=}\PYG{l+s+s1}{\PYGZsq{}}\PYG{l+s+s1}{famafrench}\PYG{l+s+s1}{\PYGZsq{}}\PYG{p}{,}
            \PYG{n}{start}\PYG{o}{=}\PYG{l+s+s1}{\PYGZsq{}}\PYG{l+s+s1}{1900}\PYG{l+s+s1}{\PYGZsq{}}\PYG{p}{,}
            \PYG{n}{session}\PYG{o}{=}\PYG{n}{session}
        \PYG{p}{)}
        \PYG{p}{[}\PYG{l+m+mi}{0}\PYG{p}{]}
        \PYG{o}{.}\PYG{n}{div}\PYG{p}{(}\PYG{l+m+mi}{100}\PYG{p}{)}
    \PYG{p}{)}

    \PYG{k}{if} \PYG{n+nb}{type}\PYG{p}{(}\PYG{n}{histories}\PYG{o}{.}\PYG{n}{columns}\PYG{p}{)} \PYG{o+ow}{is} \PYG{n}{pd}\PYG{o}{.}\PYG{n}{MultiIndex}\PYG{p}{:}
        \PYG{n}{\PYGZus{}} \PYG{o}{=} \PYG{n}{pd}\PYG{o}{.}\PYG{n}{MultiIndex}\PYG{o}{.}\PYG{n}{from\PYGZus{}product}\PYG{p}{(}\PYG{p}{[}\PYG{p}{[}\PYG{l+s+s1}{\PYGZsq{}}\PYG{l+s+s1}{Returns}\PYG{l+s+s1}{\PYGZsq{}}\PYG{p}{]}\PYG{p}{,} \PYG{n}{histories}\PYG{p}{[}\PYG{l+s+s1}{\PYGZsq{}}\PYG{l+s+s1}{Adj Close}\PYG{l+s+s1}{\PYGZsq{}}\PYG{p}{]}\PYG{o}{.}\PYG{n}{columns}\PYG{p}{]}\PYG{p}{)}
        \PYG{n}{histories}\PYG{p}{[}\PYG{n}{\PYGZus{}}\PYG{p}{]} \PYG{o}{=} \PYG{n}{histories}\PYG{p}{[}\PYG{l+s+s1}{\PYGZsq{}}\PYG{l+s+s1}{Adj Close}\PYG{l+s+s1}{\PYGZsq{}}\PYG{p}{]}\PYG{o}{.}\PYG{n}{pct\PYGZus{}change}\PYG{p}{(}\PYG{p}{)}

        \PYG{n}{\PYGZus{}} \PYG{o}{=} \PYG{n}{pd}\PYG{o}{.}\PYG{n}{MultiIndex}\PYG{o}{.}\PYG{n}{from\PYGZus{}product}\PYG{p}{(}\PYG{p}{[}\PYG{p}{[}\PYG{l+s+s1}{\PYGZsq{}}\PYG{l+s+s1}{Factors}\PYG{l+s+s1}{\PYGZsq{}}\PYG{p}{]}\PYG{p}{,} \PYG{n}{factors}\PYG{o}{.}\PYG{n}{columns}\PYG{p}{]}\PYG{p}{)}
        \PYG{n}{histories}\PYG{p}{[}\PYG{n}{\PYGZus{}}\PYG{p}{]} \PYG{o}{=} \PYG{n}{factors}

        \PYG{k}{return} \PYG{n}{histories}\PYG{o}{.}\PYG{n}{rename\PYGZus{}axis}\PYG{p}{(}\PYG{n}{columns}\PYG{o}{=}\PYG{p}{[}\PYG{l+s+s1}{\PYGZsq{}}\PYG{l+s+s1}{Variable}\PYG{l+s+s1}{\PYGZsq{}}\PYG{p}{,} \PYG{l+s+s1}{\PYGZsq{}}\PYG{l+s+s1}{Ticker}\PYG{l+s+s1}{\PYGZsq{}}\PYG{p}{]}\PYG{p}{)}

    \PYG{k}{elif} \PYG{n+nb}{type}\PYG{p}{(}\PYG{n}{histories}\PYG{o}{.}\PYG{n}{columns}\PYG{p}{)} \PYG{o+ow}{is} \PYG{n}{pd}\PYG{o}{.}\PYG{n}{Index}\PYG{p}{:}
        \PYG{k}{return} \PYG{n}{histories}\PYG{o}{.}\PYG{n}{join}\PYG{p}{(}\PYG{n}{ff}\PYG{p}{)}\PYG{o}{.}\PYG{n}{rename\PYGZus{}axis}\PYG{p}{(}\PYG{n}{columns}\PYG{o}{=}\PYG{p}{[}\PYG{l+s+s1}{\PYGZsq{}}\PYG{l+s+s1}{Variable}\PYG{l+s+s1}{\PYGZsq{}}\PYG{p}{]}\PYG{p}{)}
\end{sphinxVerbatim}

\end{sphinxuseclass}\end{sphinxVerbatimInput}

\end{sphinxuseclass}
\begin{sphinxuseclass}{cell}\begin{sphinxVerbatimInput}

\begin{sphinxuseclass}{cell_input}
\begin{sphinxVerbatim}[commandchars=\\\{\}]
\PYG{n}{download}\PYG{p}{(}\PYG{n}{tickers}\PYG{o}{=}\PYG{l+s+s1}{\PYGZsq{}}\PYG{l+s+s1}{AAPL}\PYG{l+s+s1}{\PYGZsq{}}\PYG{p}{)}\PYG{o}{.}\PYG{n}{head}\PYG{p}{(}\PYG{p}{)}
\end{sphinxVerbatim}

\end{sphinxuseclass}\end{sphinxVerbatimInput}
\begin{sphinxVerbatimOutput}

\begin{sphinxuseclass}{cell_output}
\begin{sphinxVerbatim}[commandchars=\\\{\}]
[*********************100\PYGZpc{}***********************]  1 of 1 completed
\end{sphinxVerbatim}

\begin{sphinxVerbatim}[commandchars=\\\{\}]
Variable     Open   High    Low  Close  Adj Close     Volume  Mkt\PYGZhy{}RF     SMB  \PYGZbs{}
Date                                                                           
1980\PYGZhy{}12\PYGZhy{}12 0.1283 0.1289 0.1283 0.1283     0.0997  469033600  0.0138 \PYGZhy{}0.0001   
1980\PYGZhy{}12\PYGZhy{}15 0.1222 0.1222 0.1217 0.1217     0.0945  175884800  0.0011  0.0025   
1980\PYGZhy{}12\PYGZhy{}16 0.1133 0.1133 0.1127 0.1127     0.0876  105728000  0.0071 \PYGZhy{}0.0075   
1980\PYGZhy{}12\PYGZhy{}17 0.1155 0.1161 0.1155 0.1155     0.0897   86441600  0.0152 \PYGZhy{}0.0086   
1980\PYGZhy{}12\PYGZhy{}18 0.1189 0.1194 0.1189 0.1189     0.0924   73449600  0.0041  0.0022   

Variable       HML     RF  
Date                       
1980\PYGZhy{}12\PYGZhy{}12 \PYGZhy{}0.0105 0.0006  
1980\PYGZhy{}12\PYGZhy{}15 \PYGZhy{}0.0046 0.0006  
1980\PYGZhy{}12\PYGZhy{}16 \PYGZhy{}0.0047 0.0006  
1980\PYGZhy{}12\PYGZhy{}17 \PYGZhy{}0.0034 0.0006  
1980\PYGZhy{}12\PYGZhy{}18  0.0126 0.0006  
\end{sphinxVerbatim}

\end{sphinxuseclass}\end{sphinxVerbatimOutput}

\end{sphinxuseclass}
\begin{sphinxuseclass}{cell}\begin{sphinxVerbatimInput}

\begin{sphinxuseclass}{cell_input}
\begin{sphinxVerbatim}[commandchars=\\\{\}]
\PYG{n}{download}\PYG{p}{(}\PYG{n}{tickers}\PYG{o}{=}\PYG{l+s+s1}{\PYGZsq{}}\PYG{l+s+s1}{AAPL TSLA}\PYG{l+s+s1}{\PYGZsq{}}\PYG{p}{)}\PYG{o}{.}\PYG{n}{head}\PYG{p}{(}\PYG{p}{)}
\end{sphinxVerbatim}

\end{sphinxuseclass}\end{sphinxVerbatimInput}
\begin{sphinxVerbatimOutput}

\begin{sphinxuseclass}{cell_output}
\begin{sphinxVerbatim}[commandchars=\\\{\}]
[*********************100\PYGZpc{}***********************]  2 of 2 completed
\end{sphinxVerbatim}

\begin{sphinxVerbatim}[commandchars=\\\{\}]
Variable   Adj Close       Close        High         Low        Open       \PYGZbs{}
Ticker          AAPL TSLA   AAPL TSLA   AAPL TSLA   AAPL TSLA   AAPL TSLA   
Date                                                                        
1980\PYGZhy{}12\PYGZhy{}12    0.0997  NaN 0.1283  NaN 0.1289  NaN 0.1283  NaN 0.1283  NaN   
1980\PYGZhy{}12\PYGZhy{}15    0.0945  NaN 0.1217  NaN 0.1222  NaN 0.1217  NaN 0.1222  NaN   
1980\PYGZhy{}12\PYGZhy{}16    0.0876  NaN 0.1127  NaN 0.1133  NaN 0.1127  NaN 0.1133  NaN   
1980\PYGZhy{}12\PYGZhy{}17    0.0897  NaN 0.1155  NaN 0.1161  NaN 0.1155  NaN 0.1155  NaN   
1980\PYGZhy{}12\PYGZhy{}18    0.0924  NaN 0.1189  NaN 0.1194  NaN 0.1189  NaN 0.1189  NaN   

Variable       Volume      Returns      Factors                         
Ticker           AAPL TSLA    AAPL TSLA  Mkt\PYGZhy{}RF     SMB     HML     RF  
Date                                                                    
1980\PYGZhy{}12\PYGZhy{}12  469033600  NaN     NaN  NaN  0.0138 \PYGZhy{}0.0001 \PYGZhy{}0.0105 0.0006  
1980\PYGZhy{}12\PYGZhy{}15  175884800  NaN \PYGZhy{}0.0522  NaN  0.0011  0.0025 \PYGZhy{}0.0046 0.0006  
1980\PYGZhy{}12\PYGZhy{}16  105728000  NaN \PYGZhy{}0.0734  NaN  0.0071 \PYGZhy{}0.0075 \PYGZhy{}0.0047 0.0006  
1980\PYGZhy{}12\PYGZhy{}17   86441600  NaN  0.0248  NaN  0.0152 \PYGZhy{}0.0086 \PYGZhy{}0.0034 0.0006  
1980\PYGZhy{}12\PYGZhy{}18   73449600  NaN  0.0290  NaN  0.0041  0.0022  0.0126 0.0006  
\end{sphinxVerbatim}

\end{sphinxuseclass}\end{sphinxVerbatimOutput}

\end{sphinxuseclass}
\sphinxstepscope


\chapter{McKinney Chapter 10 \sphinxhyphen{} Data Aggregation and Group Operations}
\label{\detokenize{mckinney_10_lecture:mckinney-chapter-10-data-aggregation-and-group-operations}}\label{\detokenize{mckinney_10_lecture::doc}}

\section{Introduction}
\label{\detokenize{mckinney_10_lecture:introduction}}
\sphinxAtStartPar
Chapter 10 of Wes McKinney’s \sphinxhref{https://wesmckinney.com/book/}{\sphinxstyleemphasis{Python for Data Analysis}} discusses groupby operations, which are the pandas equivalent of Excel pivot tables.
Pivot tables help us calculate statistics (e.g., sum, mean, and median) for one set of variables by groups of other variables (e.g., weekday or ticker).
For example, we could use a pivot table to calculate mean daily stock returns by weekday.

\sphinxAtStartPar
We will focus on:
\begin{enumerate}
\sphinxsetlistlabels{\arabic}{enumi}{enumii}{}{.}%
\item {} 
\sphinxAtStartPar
Using \sphinxcode{\sphinxupquote{.groupby()}} to group by columns, indexes, and functions

\item {} 
\sphinxAtStartPar
Using \sphinxcode{\sphinxupquote{.agg()}} to aggregate multiple functions

\item {} 
\sphinxAtStartPar
Using pivot tables as an alternative to \sphinxcode{\sphinxupquote{.groupby()}}

\end{enumerate}

\sphinxAtStartPar
\sphinxstyleemphasis{\sphinxstylestrong{Note:}}
Indented block quotes are from McKinney unless otherwise indicated.
The section numbers here differ from McKinney because we will only discuss some topics.

\begin{sphinxuseclass}{cell}\begin{sphinxVerbatimInput}

\begin{sphinxuseclass}{cell_input}
\begin{sphinxVerbatim}[commandchars=\\\{\}]
\PYG{k+kn}{import} \PYG{n+nn}{matplotlib}\PYG{n+nn}{.}\PYG{n+nn}{pyplot} \PYG{k}{as} \PYG{n+nn}{plt}
\PYG{k+kn}{import} \PYG{n+nn}{numpy} \PYG{k}{as} \PYG{n+nn}{np}
\PYG{k+kn}{import} \PYG{n+nn}{pandas} \PYG{k}{as} \PYG{n+nn}{pd}
\end{sphinxVerbatim}

\end{sphinxuseclass}\end{sphinxVerbatimInput}

\end{sphinxuseclass}
\begin{sphinxuseclass}{cell}\begin{sphinxVerbatimInput}

\begin{sphinxuseclass}{cell_input}
\begin{sphinxVerbatim}[commandchars=\\\{\}]
\PYG{o}{\PYGZpc{}}\PYG{k}{config} InlineBackend.figure\PYGZus{}format = \PYGZsq{}retina\PYGZsq{}
\PYG{o}{\PYGZpc{}}\PYG{k}{precision} 4
\PYG{n}{pd}\PYG{o}{.}\PYG{n}{options}\PYG{o}{.}\PYG{n}{display}\PYG{o}{.}\PYG{n}{float\PYGZus{}format} \PYG{o}{=} \PYG{l+s+s1}{\PYGZsq{}}\PYG{l+s+si}{\PYGZob{}:.4f\PYGZcb{}}\PYG{l+s+s1}{\PYGZsq{}}\PYG{o}{.}\PYG{n}{format}
\end{sphinxVerbatim}

\end{sphinxuseclass}\end{sphinxVerbatimInput}

\end{sphinxuseclass}
\begin{sphinxuseclass}{cell}\begin{sphinxVerbatimInput}

\begin{sphinxuseclass}{cell_input}
\begin{sphinxVerbatim}[commandchars=\\\{\}]
\PYG{k+kn}{import} \PYG{n+nn}{yfinance} \PYG{k}{as} \PYG{n+nn}{yf}
\PYG{k+kn}{import} \PYG{n+nn}{pandas\PYGZus{}datareader} \PYG{k}{as} \PYG{n+nn}{pdr}
\PYG{k+kn}{import} \PYG{n+nn}{requests\PYGZus{}cache}
\PYG{n}{session} \PYG{o}{=} \PYG{n}{requests\PYGZus{}cache}\PYG{o}{.}\PYG{n}{CachedSession}\PYG{p}{(}\PYG{p}{)}
\end{sphinxVerbatim}

\end{sphinxuseclass}\end{sphinxVerbatimInput}

\end{sphinxuseclass}

\section{GroupBy Mechanics}
\label{\detokenize{mckinney_10_lecture:groupby-mechanics}}
\sphinxAtStartPar
“Split\sphinxhyphen{}apply\sphinxhyphen{}combine” is an excellent way to describe and visualize pandas groupby operations.
\begin{quote}

\sphinxAtStartPar
Hadley Wickham, an author of many popular packages for the R programming
language, coined the term split\sphinxhyphen{}apply\sphinxhyphen{}combine for describing group operations. In the
first stage of the process, data contained in a pandas object, whether a Series, DataFrame, or otherwise, is split into groups based on one or more keys that you provide.
The splitting is performed on a particular axis of an object. For example, a DataFrame
can be grouped on its rows (axis=0) or its columns (axis=1). Once this is done, a
function is applied to each group, producing a new value. Finally, the results of all
those function applications are combined into a result object. The form of the resulting object will usually depend on what’s being done to the data. See Figure 10\sphinxhyphen{}1 for a
mockup of a simple group aggregation.
\end{quote}

\sphinxAtStartPar
Figure 10\sphinxhyphen{}1 visualizes a split\sphinxhyphen{}apply\sphinxhyphen{}combine operation that:
\begin{enumerate}
\sphinxsetlistlabels{\arabic}{enumi}{enumii}{}{.}%
\item {} 
\sphinxAtStartPar
Splits by the \sphinxcode{\sphinxupquote{key}} column (i.e., “groups by \sphinxcode{\sphinxupquote{key}}”)

\item {} 
\sphinxAtStartPar
Applies the sum operation to the \sphinxcode{\sphinxupquote{data}} column (i.e., “and sums \sphinxcode{\sphinxupquote{data}}”)

\item {} 
\sphinxAtStartPar
Combines the grouped sums

\end{enumerate}

\sphinxAtStartPar
I describe this operation as “sum the \sphinxcode{\sphinxupquote{data}} column by groups formed on the \sphinxcode{\sphinxupquote{key}} column.”

\begin{sphinxuseclass}{cell}\begin{sphinxVerbatimInput}

\begin{sphinxuseclass}{cell_input}
\begin{sphinxVerbatim}[commandchars=\\\{\}]
\PYG{n}{np}\PYG{o}{.}\PYG{n}{random}\PYG{o}{.}\PYG{n}{seed}\PYG{p}{(}\PYG{l+m+mi}{42}\PYG{p}{)}
\PYG{n}{df} \PYG{o}{=} \PYG{n}{pd}\PYG{o}{.}\PYG{n}{DataFrame}\PYG{p}{(}\PYG{p}{\PYGZob{}}\PYG{l+s+s1}{\PYGZsq{}}\PYG{l+s+s1}{key1}\PYG{l+s+s1}{\PYGZsq{}} \PYG{p}{:} \PYG{p}{[}\PYG{l+s+s1}{\PYGZsq{}}\PYG{l+s+s1}{a}\PYG{l+s+s1}{\PYGZsq{}}\PYG{p}{,} \PYG{l+s+s1}{\PYGZsq{}}\PYG{l+s+s1}{a}\PYG{l+s+s1}{\PYGZsq{}}\PYG{p}{,} \PYG{l+s+s1}{\PYGZsq{}}\PYG{l+s+s1}{b}\PYG{l+s+s1}{\PYGZsq{}}\PYG{p}{,} \PYG{l+s+s1}{\PYGZsq{}}\PYG{l+s+s1}{b}\PYG{l+s+s1}{\PYGZsq{}}\PYG{p}{,} \PYG{l+s+s1}{\PYGZsq{}}\PYG{l+s+s1}{a}\PYG{l+s+s1}{\PYGZsq{}}\PYG{p}{]}\PYG{p}{,}
                   \PYG{l+s+s1}{\PYGZsq{}}\PYG{l+s+s1}{key2}\PYG{l+s+s1}{\PYGZsq{}} \PYG{p}{:} \PYG{p}{[}\PYG{l+s+s1}{\PYGZsq{}}\PYG{l+s+s1}{one}\PYG{l+s+s1}{\PYGZsq{}}\PYG{p}{,} \PYG{l+s+s1}{\PYGZsq{}}\PYG{l+s+s1}{two}\PYG{l+s+s1}{\PYGZsq{}}\PYG{p}{,} \PYG{l+s+s1}{\PYGZsq{}}\PYG{l+s+s1}{one}\PYG{l+s+s1}{\PYGZsq{}}\PYG{p}{,} \PYG{l+s+s1}{\PYGZsq{}}\PYG{l+s+s1}{two}\PYG{l+s+s1}{\PYGZsq{}}\PYG{p}{,} \PYG{l+s+s1}{\PYGZsq{}}\PYG{l+s+s1}{one}\PYG{l+s+s1}{\PYGZsq{}}\PYG{p}{]}\PYG{p}{,}
                   \PYG{l+s+s1}{\PYGZsq{}}\PYG{l+s+s1}{data1}\PYG{l+s+s1}{\PYGZsq{}} \PYG{p}{:} \PYG{n}{np}\PYG{o}{.}\PYG{n}{random}\PYG{o}{.}\PYG{n}{randn}\PYG{p}{(}\PYG{l+m+mi}{5}\PYG{p}{)}\PYG{p}{,}
                   \PYG{l+s+s1}{\PYGZsq{}}\PYG{l+s+s1}{data2}\PYG{l+s+s1}{\PYGZsq{}} \PYG{p}{:} \PYG{n}{np}\PYG{o}{.}\PYG{n}{random}\PYG{o}{.}\PYG{n}{randn}\PYG{p}{(}\PYG{l+m+mi}{5}\PYG{p}{)}\PYG{p}{\PYGZcb{}}\PYG{p}{)}

\PYG{n}{df}
\end{sphinxVerbatim}

\end{sphinxuseclass}\end{sphinxVerbatimInput}
\begin{sphinxVerbatimOutput}

\begin{sphinxuseclass}{cell_output}
\begin{sphinxVerbatim}[commandchars=\\\{\}]
  key1 key2   data1   data2
0    a  one  0.4967 \PYGZhy{}0.2341
1    a  two \PYGZhy{}0.1383  1.5792
2    b  one  0.6477  0.7674
3    b  two  1.5230 \PYGZhy{}0.4695
4    a  one \PYGZhy{}0.2342  0.5426
\end{sphinxVerbatim}

\end{sphinxuseclass}\end{sphinxVerbatimOutput}

\end{sphinxuseclass}
\sphinxAtStartPar
Here is one way to calculate the means of \sphinxcode{\sphinxupquote{data1}} by groups formed on \sphinxcode{\sphinxupquote{key1}}.

\begin{sphinxuseclass}{cell}\begin{sphinxVerbatimInput}

\begin{sphinxuseclass}{cell_input}
\begin{sphinxVerbatim}[commandchars=\\\{\}]
\PYG{n}{df}\PYG{o}{.}\PYG{n}{loc}\PYG{p}{[}\PYG{n}{df}\PYG{p}{[}\PYG{l+s+s1}{\PYGZsq{}}\PYG{l+s+s1}{key1}\PYG{l+s+s1}{\PYGZsq{}}\PYG{p}{]} \PYG{o}{==} \PYG{l+s+s1}{\PYGZsq{}}\PYG{l+s+s1}{a}\PYG{l+s+s1}{\PYGZsq{}}\PYG{p}{,} \PYG{l+s+s1}{\PYGZsq{}}\PYG{l+s+s1}{data1}\PYG{l+s+s1}{\PYGZsq{}}\PYG{p}{]}\PYG{o}{.}\PYG{n}{mean}\PYG{p}{(}\PYG{p}{)}
\end{sphinxVerbatim}

\end{sphinxuseclass}\end{sphinxVerbatimInput}
\begin{sphinxVerbatimOutput}

\begin{sphinxuseclass}{cell_output}
\begin{sphinxVerbatim}[commandchars=\\\{\}]
0.0414
\end{sphinxVerbatim}

\end{sphinxuseclass}\end{sphinxVerbatimOutput}

\end{sphinxuseclass}
\begin{sphinxuseclass}{cell}\begin{sphinxVerbatimInput}

\begin{sphinxuseclass}{cell_input}
\begin{sphinxVerbatim}[commandchars=\\\{\}]
\PYG{n}{df}\PYG{o}{.}\PYG{n}{loc}\PYG{p}{[}\PYG{n}{df}\PYG{p}{[}\PYG{l+s+s1}{\PYGZsq{}}\PYG{l+s+s1}{key1}\PYG{l+s+s1}{\PYGZsq{}}\PYG{p}{]} \PYG{o}{==} \PYG{l+s+s1}{\PYGZsq{}}\PYG{l+s+s1}{b}\PYG{l+s+s1}{\PYGZsq{}}\PYG{p}{,} \PYG{l+s+s1}{\PYGZsq{}}\PYG{l+s+s1}{data1}\PYG{l+s+s1}{\PYGZsq{}}\PYG{p}{]}\PYG{o}{.}\PYG{n}{mean}\PYG{p}{(}\PYG{p}{)}
\end{sphinxVerbatim}

\end{sphinxuseclass}\end{sphinxVerbatimInput}
\begin{sphinxVerbatimOutput}

\begin{sphinxuseclass}{cell_output}
\begin{sphinxVerbatim}[commandchars=\\\{\}]
1.0854
\end{sphinxVerbatim}

\end{sphinxuseclass}\end{sphinxVerbatimOutput}

\end{sphinxuseclass}
\sphinxAtStartPar
We can do this calculation more quickly!
\begin{enumerate}
\sphinxsetlistlabels{\arabic}{enumi}{enumii}{}{.}%
\item {} 
\sphinxAtStartPar
Use the \sphinxcode{\sphinxupquote{.groupby()}} method to group by \sphinxcode{\sphinxupquote{key1}}

\item {} 
\sphinxAtStartPar
Use the \sphinxcode{\sphinxupquote{.mean()}} method to sum \sphinxcode{\sphinxupquote{data1}} within each value of \sphinxcode{\sphinxupquote{key1}}

\end{enumerate}

\sphinxAtStartPar
Note that without the \sphinxcode{\sphinxupquote{.mean()}} method, pandas only sets up the grouped object, which can accept the \sphinxcode{\sphinxupquote{.mean()}} method.

\begin{sphinxuseclass}{cell}\begin{sphinxVerbatimInput}

\begin{sphinxuseclass}{cell_input}
\begin{sphinxVerbatim}[commandchars=\\\{\}]
\PYG{n}{grouped} \PYG{o}{=} \PYG{n}{df}\PYG{p}{[}\PYG{l+s+s1}{\PYGZsq{}}\PYG{l+s+s1}{data1}\PYG{l+s+s1}{\PYGZsq{}}\PYG{p}{]}\PYG{o}{.}\PYG{n}{groupby}\PYG{p}{(}\PYG{n}{df}\PYG{p}{[}\PYG{l+s+s1}{\PYGZsq{}}\PYG{l+s+s1}{key1}\PYG{l+s+s1}{\PYGZsq{}}\PYG{p}{]}\PYG{p}{)}
\PYG{n}{grouped}
\end{sphinxVerbatim}

\end{sphinxuseclass}\end{sphinxVerbatimInput}
\begin{sphinxVerbatimOutput}

\begin{sphinxuseclass}{cell_output}
\begin{sphinxVerbatim}[commandchars=\\\{\}]
\PYGZlt{}pandas.core.groupby.generic.SeriesGroupBy object at 0x7f7fbf0ccca0\PYGZgt{}
\end{sphinxVerbatim}

\end{sphinxuseclass}\end{sphinxVerbatimOutput}

\end{sphinxuseclass}
\begin{sphinxuseclass}{cell}\begin{sphinxVerbatimInput}

\begin{sphinxuseclass}{cell_input}
\begin{sphinxVerbatim}[commandchars=\\\{\}]
\PYG{n}{grouped}\PYG{o}{.}\PYG{n}{mean}\PYG{p}{(}\PYG{p}{)}
\end{sphinxVerbatim}

\end{sphinxuseclass}\end{sphinxVerbatimInput}
\begin{sphinxVerbatimOutput}

\begin{sphinxuseclass}{cell_output}
\begin{sphinxVerbatim}[commandchars=\\\{\}]
key1
a   0.0414
b   1.0854
Name: data1, dtype: float64
\end{sphinxVerbatim}

\end{sphinxuseclass}\end{sphinxVerbatimOutput}

\end{sphinxuseclass}
\sphinxAtStartPar
We can can chain the \sphinxcode{\sphinxupquote{.groupby()}} and \sphinxcode{\sphinxupquote{.mean()}} methods!

\begin{sphinxuseclass}{cell}\begin{sphinxVerbatimInput}

\begin{sphinxuseclass}{cell_input}
\begin{sphinxVerbatim}[commandchars=\\\{\}]
\PYG{n}{df}\PYG{p}{[}\PYG{l+s+s1}{\PYGZsq{}}\PYG{l+s+s1}{data1}\PYG{l+s+s1}{\PYGZsq{}}\PYG{p}{]}\PYG{o}{.}\PYG{n}{groupby}\PYG{p}{(}\PYG{n}{df}\PYG{p}{[}\PYG{l+s+s1}{\PYGZsq{}}\PYG{l+s+s1}{key1}\PYG{l+s+s1}{\PYGZsq{}}\PYG{p}{]}\PYG{p}{)}\PYG{o}{.}\PYG{n}{mean}\PYG{p}{(}\PYG{p}{)}
\end{sphinxVerbatim}

\end{sphinxuseclass}\end{sphinxVerbatimInput}
\begin{sphinxVerbatimOutput}

\begin{sphinxuseclass}{cell_output}
\begin{sphinxVerbatim}[commandchars=\\\{\}]
key1
a   0.0414
b   1.0854
Name: data1, dtype: float64
\end{sphinxVerbatim}

\end{sphinxuseclass}\end{sphinxVerbatimOutput}

\end{sphinxuseclass}
\sphinxAtStartPar
If we prefer our result as a dataframe instead of a series, we can wrap \sphinxcode{\sphinxupquote{data1}} with two sets of square brackets.

\begin{sphinxuseclass}{cell}\begin{sphinxVerbatimInput}

\begin{sphinxuseclass}{cell_input}
\begin{sphinxVerbatim}[commandchars=\\\{\}]
\PYG{n}{df}\PYG{p}{[}\PYG{p}{[}\PYG{l+s+s1}{\PYGZsq{}}\PYG{l+s+s1}{data1}\PYG{l+s+s1}{\PYGZsq{}}\PYG{p}{]}\PYG{p}{]}\PYG{o}{.}\PYG{n}{groupby}\PYG{p}{(}\PYG{n}{df}\PYG{p}{[}\PYG{l+s+s1}{\PYGZsq{}}\PYG{l+s+s1}{key1}\PYG{l+s+s1}{\PYGZsq{}}\PYG{p}{]}\PYG{p}{)}\PYG{o}{.}\PYG{n}{mean}\PYG{p}{(}\PYG{p}{)}
\end{sphinxVerbatim}

\end{sphinxuseclass}\end{sphinxVerbatimInput}
\begin{sphinxVerbatimOutput}

\begin{sphinxuseclass}{cell_output}
\begin{sphinxVerbatim}[commandchars=\\\{\}]
      data1
key1       
a    0.0414
b    1.0854
\end{sphinxVerbatim}

\end{sphinxuseclass}\end{sphinxVerbatimOutput}

\end{sphinxuseclass}
\sphinxAtStartPar
We can group by more than one variable.
We get a hierarchical row index (or row multi\sphinxhyphen{}index) when we group by more than one variable.

\begin{sphinxuseclass}{cell}\begin{sphinxVerbatimInput}

\begin{sphinxuseclass}{cell_input}
\begin{sphinxVerbatim}[commandchars=\\\{\}]
\PYG{n}{means} \PYG{o}{=} \PYG{n}{df}\PYG{p}{[}\PYG{l+s+s1}{\PYGZsq{}}\PYG{l+s+s1}{data1}\PYG{l+s+s1}{\PYGZsq{}}\PYG{p}{]}\PYG{o}{.}\PYG{n}{groupby}\PYG{p}{(}\PYG{p}{[}\PYG{n}{df}\PYG{p}{[}\PYG{l+s+s1}{\PYGZsq{}}\PYG{l+s+s1}{key1}\PYG{l+s+s1}{\PYGZsq{}}\PYG{p}{]}\PYG{p}{,} \PYG{n}{df}\PYG{p}{[}\PYG{l+s+s1}{\PYGZsq{}}\PYG{l+s+s1}{key2}\PYG{l+s+s1}{\PYGZsq{}}\PYG{p}{]}\PYG{p}{]}\PYG{p}{)}\PYG{o}{.}\PYG{n}{mean}\PYG{p}{(}\PYG{p}{)}
\PYG{n}{means}
\end{sphinxVerbatim}

\end{sphinxuseclass}\end{sphinxVerbatimInput}
\begin{sphinxVerbatimOutput}

\begin{sphinxuseclass}{cell_output}
\begin{sphinxVerbatim}[commandchars=\\\{\}]
key1  key2
a     one     0.1313
      two    \PYGZhy{}0.1383
b     one     0.6477
      two     1.5230
Name: data1, dtype: float64
\end{sphinxVerbatim}

\end{sphinxuseclass}\end{sphinxVerbatimOutput}

\end{sphinxuseclass}
\sphinxAtStartPar
We can use the \sphinxcode{\sphinxupquote{.unstack()}} method if we want to use both rows and columns to organize data.
Recall the \sphinxcode{\sphinxupquote{.unstack()}} method un\sphinxhyphen{}stacks the inner index level (i.e., \sphinxcode{\sphinxupquote{level = \sphinxhyphen{}1}}) by default so that \sphinxcode{\sphinxupquote{key2}} values become the columns.

\begin{sphinxuseclass}{cell}\begin{sphinxVerbatimInput}

\begin{sphinxuseclass}{cell_input}
\begin{sphinxVerbatim}[commandchars=\\\{\}]
\PYG{n}{means}\PYG{o}{.}\PYG{n}{unstack}\PYG{p}{(}\PYG{p}{)}
\end{sphinxVerbatim}

\end{sphinxuseclass}\end{sphinxVerbatimInput}
\begin{sphinxVerbatimOutput}

\begin{sphinxuseclass}{cell_output}
\begin{sphinxVerbatim}[commandchars=\\\{\}]
key2    one     two
key1               
a    0.1313 \PYGZhy{}0.1383
b    0.6477  1.5230
\end{sphinxVerbatim}

\end{sphinxuseclass}\end{sphinxVerbatimOutput}

\end{sphinxuseclass}
\sphinxAtStartPar
The grouping variables can be columns in the data frame we want to group with the \sphinxcode{\sphinxupquote{.groupby()}} method.
Our grouping variables are typically columns in the data frame we want to group, so this syntax is more compact and easier to understand.

\begin{sphinxuseclass}{cell}\begin{sphinxVerbatimInput}

\begin{sphinxuseclass}{cell_input}
\begin{sphinxVerbatim}[commandchars=\\\{\}]
\PYG{n}{df}\PYG{o}{.}\PYG{n}{groupby}\PYG{p}{(}\PYG{l+s+s1}{\PYGZsq{}}\PYG{l+s+s1}{key1}\PYG{l+s+s1}{\PYGZsq{}}\PYG{p}{)}\PYG{o}{.}\PYG{n}{mean}\PYG{p}{(}\PYG{p}{)}
\end{sphinxVerbatim}

\end{sphinxuseclass}\end{sphinxVerbatimInput}
\begin{sphinxVerbatimOutput}

\begin{sphinxuseclass}{cell_output}
\begin{sphinxVerbatim}[commandchars=\\\{\}]
      data1  data2
key1              
a    0.0414 0.6292
b    1.0854 0.1490
\end{sphinxVerbatim}

\end{sphinxuseclass}\end{sphinxVerbatimOutput}

\end{sphinxuseclass}
\begin{sphinxuseclass}{cell}\begin{sphinxVerbatimInput}

\begin{sphinxuseclass}{cell_input}
\begin{sphinxVerbatim}[commandchars=\\\{\}]
\PYG{n}{df}\PYG{o}{.}\PYG{n}{groupby}\PYG{p}{(}\PYG{p}{[}\PYG{l+s+s1}{\PYGZsq{}}\PYG{l+s+s1}{key1}\PYG{l+s+s1}{\PYGZsq{}}\PYG{p}{,} \PYG{l+s+s1}{\PYGZsq{}}\PYG{l+s+s1}{key2}\PYG{l+s+s1}{\PYGZsq{}}\PYG{p}{]}\PYG{p}{)}\PYG{o}{.}\PYG{n}{mean}\PYG{p}{(}\PYG{p}{)}
\end{sphinxVerbatim}

\end{sphinxuseclass}\end{sphinxVerbatimInput}
\begin{sphinxVerbatimOutput}

\begin{sphinxuseclass}{cell_output}
\begin{sphinxVerbatim}[commandchars=\\\{\}]
            data1   data2
key1 key2                
a    one   0.1313  0.1542
     two  \PYGZhy{}0.1383  1.5792
b    one   0.6477  0.7674
     two   1.5230 \PYGZhy{}0.4695
\end{sphinxVerbatim}

\end{sphinxuseclass}\end{sphinxVerbatimOutput}

\end{sphinxuseclass}
\sphinxAtStartPar
We can use tab completion to reminder ourselves of methods we can apply to grouped series and data frames.


\subsection{Iterating Over Groups}
\label{\detokenize{mckinney_10_lecture:iterating-over-groups}}
\sphinxAtStartPar
We can iterate over groups, too, because the \sphinxcode{\sphinxupquote{.groupby()}} method generates a sequence of tuples.
Each tuples contains the value(s) of the grouping variable(s) and its chunk of the dataframe.
McKinney provides two loops to show how to iterate over groups.

\begin{sphinxuseclass}{cell}\begin{sphinxVerbatimInput}

\begin{sphinxuseclass}{cell_input}
\begin{sphinxVerbatim}[commandchars=\\\{\}]
\PYG{k}{for} \PYG{n}{name}\PYG{p}{,} \PYG{n}{group} \PYG{o+ow}{in} \PYG{n}{df}\PYG{o}{.}\PYG{n}{groupby}\PYG{p}{(}\PYG{l+s+s1}{\PYGZsq{}}\PYG{l+s+s1}{key1}\PYG{l+s+s1}{\PYGZsq{}}\PYG{p}{)}\PYG{p}{:}
    \PYG{n+nb}{print}\PYG{p}{(}\PYG{n}{name}\PYG{p}{,} \PYG{n}{group}\PYG{p}{,} \PYG{n}{sep}\PYG{o}{=}\PYG{l+s+s1}{\PYGZsq{}}\PYG{l+s+se}{\PYGZbs{}n}\PYG{l+s+s1}{\PYGZsq{}}\PYG{p}{)}
\end{sphinxVerbatim}

\end{sphinxuseclass}\end{sphinxVerbatimInput}
\begin{sphinxVerbatimOutput}

\begin{sphinxuseclass}{cell_output}
\begin{sphinxVerbatim}[commandchars=\\\{\}]
a
  key1 key2   data1   data2
0    a  one  0.4967 \PYGZhy{}0.2341
1    a  two \PYGZhy{}0.1383  1.5792
4    a  one \PYGZhy{}0.2342  0.5426
b
  key1 key2  data1   data2
2    b  one 0.6477  0.7674
3    b  two 1.5230 \PYGZhy{}0.4695
\end{sphinxVerbatim}

\end{sphinxuseclass}\end{sphinxVerbatimOutput}

\end{sphinxuseclass}
\begin{sphinxuseclass}{cell}\begin{sphinxVerbatimInput}

\begin{sphinxuseclass}{cell_input}
\begin{sphinxVerbatim}[commandchars=\\\{\}]
\PYG{k}{for} \PYG{p}{(}\PYG{n}{k1}\PYG{p}{,} \PYG{n}{k2}\PYG{p}{)}\PYG{p}{,} \PYG{n}{group} \PYG{o+ow}{in} \PYG{n}{df}\PYG{o}{.}\PYG{n}{groupby}\PYG{p}{(}\PYG{p}{[}\PYG{l+s+s1}{\PYGZsq{}}\PYG{l+s+s1}{key1}\PYG{l+s+s1}{\PYGZsq{}}\PYG{p}{,} \PYG{l+s+s1}{\PYGZsq{}}\PYG{l+s+s1}{key2}\PYG{l+s+s1}{\PYGZsq{}}\PYG{p}{]}\PYG{p}{)}\PYG{p}{:}
    \PYG{n+nb}{print}\PYG{p}{(}\PYG{p}{(}\PYG{n}{k1}\PYG{p}{,} \PYG{n}{k2}\PYG{p}{)}\PYG{p}{,} \PYG{n}{group}\PYG{p}{,} \PYG{n}{sep}\PYG{o}{=}\PYG{l+s+s1}{\PYGZsq{}}\PYG{l+s+se}{\PYGZbs{}n}\PYG{l+s+s1}{\PYGZsq{}}\PYG{p}{)}
\end{sphinxVerbatim}

\end{sphinxuseclass}\end{sphinxVerbatimInput}
\begin{sphinxVerbatimOutput}

\begin{sphinxuseclass}{cell_output}
\begin{sphinxVerbatim}[commandchars=\\\{\}]
(\PYGZsq{}a\PYGZsq{}, \PYGZsq{}one\PYGZsq{})
  key1 key2   data1   data2
0    a  one  0.4967 \PYGZhy{}0.2341
4    a  one \PYGZhy{}0.2342  0.5426
(\PYGZsq{}a\PYGZsq{}, \PYGZsq{}two\PYGZsq{})
  key1 key2   data1  data2
1    a  two \PYGZhy{}0.1383 1.5792
(\PYGZsq{}b\PYGZsq{}, \PYGZsq{}one\PYGZsq{})
  key1 key2  data1  data2
2    b  one 0.6477 0.7674
(\PYGZsq{}b\PYGZsq{}, \PYGZsq{}two\PYGZsq{})
  key1 key2  data1   data2
3    b  two 1.5230 \PYGZhy{}0.4695
\end{sphinxVerbatim}

\end{sphinxuseclass}\end{sphinxVerbatimOutput}

\end{sphinxuseclass}

\subsection{Selecting a Column or Subset of Columns}
\label{\detokenize{mckinney_10_lecture:selecting-a-column-or-subset-of-columns}}
\sphinxAtStartPar
We preview the idea of grouping an entire dataframe above.
However, I want to explain McKinney’s use of the phrase “syntactic sugar.”
Here is the context:
\begin{quote}

\sphinxAtStartPar
Indexing a GroupBy object created from a DataFrame with a column name or array
of column names has the effect of column subsetting for aggregation. This means
that:

\begin{sphinxVerbatim}[commandchars=\\\{\}]
\PYG{n}{df}\PYG{o}{.}\PYG{n}{groupby}\PYG{p}{(}\PYG{l+s+s1}{\PYGZsq{}}\PYG{l+s+s1}{key1}\PYG{l+s+s1}{\PYGZsq{}}\PYG{p}{)}\PYG{p}{[}\PYG{l+s+s1}{\PYGZsq{}}\PYG{l+s+s1}{data1}\PYG{l+s+s1}{\PYGZsq{}}\PYG{p}{]}
\PYG{n}{df}\PYG{o}{.}\PYG{n}{groupby}\PYG{p}{(}\PYG{l+s+s1}{\PYGZsq{}}\PYG{l+s+s1}{key1}\PYG{l+s+s1}{\PYGZsq{}}\PYG{p}{)}\PYG{p}{[}\PYG{p}{[}\PYG{l+s+s1}{\PYGZsq{}}\PYG{l+s+s1}{data2}\PYG{l+s+s1}{\PYGZsq{}}\PYG{p}{]}\PYG{p}{]}
\end{sphinxVerbatim}

\sphinxAtStartPar
are syntactic sugar for

\begin{sphinxVerbatim}[commandchars=\\\{\}]
\PYG{n}{df}\PYG{p}{[}\PYG{l+s+s1}{\PYGZsq{}}\PYG{l+s+s1}{data1}\PYG{l+s+s1}{\PYGZsq{}}\PYG{p}{]}\PYG{o}{.}\PYG{n}{groupby}\PYG{p}{(}\PYG{n}{df}\PYG{p}{[}\PYG{l+s+s1}{\PYGZsq{}}\PYG{l+s+s1}{key1}\PYG{l+s+s1}{\PYGZsq{}}\PYG{p}{]}\PYG{p}{)}
\PYG{n}{df}\PYG{p}{[}\PYG{p}{[}\PYG{l+s+s1}{\PYGZsq{}}\PYG{l+s+s1}{data2}\PYG{l+s+s1}{\PYGZsq{}}\PYG{p}{]}\PYG{p}{]}\PYG{o}{.}\PYG{n}{groupby}\PYG{p}{(}\PYG{n}{df}\PYG{p}{[}\PYG{l+s+s1}{\PYGZsq{}}\PYG{l+s+s1}{key1}\PYG{l+s+s1}{\PYGZsq{}}\PYG{p}{]}\PYG{p}{)}
\end{sphinxVerbatim}
\end{quote}

\sphinxAtStartPar
“Syntactic sugar” makes code easier to type or read without adding functionality.
It makes code “sweeter” for humans to type or read by making it more concise or clear.
The implication is that syntactic sugar makes code faster to type/read but does make code faster to execute.

\begin{sphinxuseclass}{cell}\begin{sphinxVerbatimInput}

\begin{sphinxuseclass}{cell_input}
\begin{sphinxVerbatim}[commandchars=\\\{\}]
\PYG{n}{df}\PYG{o}{.}\PYG{n}{groupby}\PYG{p}{(}\PYG{p}{[}\PYG{l+s+s1}{\PYGZsq{}}\PYG{l+s+s1}{key1}\PYG{l+s+s1}{\PYGZsq{}}\PYG{p}{,} \PYG{l+s+s1}{\PYGZsq{}}\PYG{l+s+s1}{key2}\PYG{l+s+s1}{\PYGZsq{}}\PYG{p}{]}\PYG{p}{)}\PYG{p}{[}\PYG{p}{[}\PYG{l+s+s1}{\PYGZsq{}}\PYG{l+s+s1}{data2}\PYG{l+s+s1}{\PYGZsq{}}\PYG{p}{]}\PYG{p}{]}\PYG{o}{.}\PYG{n}{mean}\PYG{p}{(}\PYG{p}{)}
\end{sphinxVerbatim}

\end{sphinxuseclass}\end{sphinxVerbatimInput}
\begin{sphinxVerbatimOutput}

\begin{sphinxuseclass}{cell_output}
\begin{sphinxVerbatim}[commandchars=\\\{\}]
            data2
key1 key2        
a    one   0.1542
     two   1.5792
b    one   0.7674
     two  \PYGZhy{}0.4695
\end{sphinxVerbatim}

\end{sphinxuseclass}\end{sphinxVerbatimOutput}

\end{sphinxuseclass}

\subsection{Grouping with Functions}
\label{\detokenize{mckinney_10_lecture:grouping-with-functions}}
\sphinxAtStartPar
We can also group with functions.
Below, we group with the \sphinxcode{\sphinxupquote{len}} function, which calculates the length of the first names in the row index.
We could instead add a helper column to \sphinxcode{\sphinxupquote{people}}, but it is easier to pass a function to \sphinxcode{\sphinxupquote{.groupby()}}.

\begin{sphinxuseclass}{cell}\begin{sphinxVerbatimInput}

\begin{sphinxuseclass}{cell_input}
\begin{sphinxVerbatim}[commandchars=\\\{\}]
\PYG{n}{np}\PYG{o}{.}\PYG{n}{random}\PYG{o}{.}\PYG{n}{seed}\PYG{p}{(}\PYG{l+m+mi}{42}\PYG{p}{)}
\PYG{n}{people} \PYG{o}{=} \PYG{n}{pd}\PYG{o}{.}\PYG{n}{DataFrame}\PYG{p}{(}
    \PYG{n}{data}\PYG{o}{=}\PYG{n}{np}\PYG{o}{.}\PYG{n}{random}\PYG{o}{.}\PYG{n}{randn}\PYG{p}{(}\PYG{l+m+mi}{5}\PYG{p}{,} \PYG{l+m+mi}{5}\PYG{p}{)}\PYG{p}{,} 
    \PYG{n}{columns}\PYG{o}{=}\PYG{p}{[}\PYG{l+s+s1}{\PYGZsq{}}\PYG{l+s+s1}{a}\PYG{l+s+s1}{\PYGZsq{}}\PYG{p}{,} \PYG{l+s+s1}{\PYGZsq{}}\PYG{l+s+s1}{b}\PYG{l+s+s1}{\PYGZsq{}}\PYG{p}{,} \PYG{l+s+s1}{\PYGZsq{}}\PYG{l+s+s1}{c}\PYG{l+s+s1}{\PYGZsq{}}\PYG{p}{,} \PYG{l+s+s1}{\PYGZsq{}}\PYG{l+s+s1}{d}\PYG{l+s+s1}{\PYGZsq{}}\PYG{p}{,} \PYG{l+s+s1}{\PYGZsq{}}\PYG{l+s+s1}{e}\PYG{l+s+s1}{\PYGZsq{}}\PYG{p}{]}\PYG{p}{,} 
    \PYG{n}{index}\PYG{o}{=}\PYG{p}{[}\PYG{l+s+s1}{\PYGZsq{}}\PYG{l+s+s1}{Joe}\PYG{l+s+s1}{\PYGZsq{}}\PYG{p}{,} \PYG{l+s+s1}{\PYGZsq{}}\PYG{l+s+s1}{Steve}\PYG{l+s+s1}{\PYGZsq{}}\PYG{p}{,} \PYG{l+s+s1}{\PYGZsq{}}\PYG{l+s+s1}{Wes}\PYG{l+s+s1}{\PYGZsq{}}\PYG{p}{,} \PYG{l+s+s1}{\PYGZsq{}}\PYG{l+s+s1}{Jim}\PYG{l+s+s1}{\PYGZsq{}}\PYG{p}{,} \PYG{l+s+s1}{\PYGZsq{}}\PYG{l+s+s1}{Travis}\PYG{l+s+s1}{\PYGZsq{}}\PYG{p}{]}
\PYG{p}{)}

\PYG{n}{people}
\end{sphinxVerbatim}

\end{sphinxuseclass}\end{sphinxVerbatimInput}
\begin{sphinxVerbatimOutput}

\begin{sphinxuseclass}{cell_output}
\begin{sphinxVerbatim}[commandchars=\\\{\}]
             a       b      c       d       e
Joe     0.4967 \PYGZhy{}0.1383 0.6477  1.5230 \PYGZhy{}0.2342
Steve  \PYGZhy{}0.2341  1.5792 0.7674 \PYGZhy{}0.4695  0.5426
Wes    \PYGZhy{}0.4634 \PYGZhy{}0.4657 0.2420 \PYGZhy{}1.9133 \PYGZhy{}1.7249
Jim    \PYGZhy{}0.5623 \PYGZhy{}1.0128 0.3142 \PYGZhy{}0.9080 \PYGZhy{}1.4123
Travis  1.4656 \PYGZhy{}0.2258 0.0675 \PYGZhy{}1.4247 \PYGZhy{}0.5444
\end{sphinxVerbatim}

\end{sphinxuseclass}\end{sphinxVerbatimOutput}

\end{sphinxuseclass}
\begin{sphinxuseclass}{cell}\begin{sphinxVerbatimInput}

\begin{sphinxuseclass}{cell_input}
\begin{sphinxVerbatim}[commandchars=\\\{\}]
\PYG{n}{people}\PYG{o}{.}\PYG{n}{groupby}\PYG{p}{(}\PYG{n+nb}{len}\PYG{p}{)}\PYG{o}{.}\PYG{n}{sum}\PYG{p}{(}\PYG{p}{)}
\end{sphinxVerbatim}

\end{sphinxuseclass}\end{sphinxVerbatimInput}
\begin{sphinxVerbatimOutput}

\begin{sphinxuseclass}{cell_output}
\begin{sphinxVerbatim}[commandchars=\\\{\}]
        a       b      c       d       e
3 \PYGZhy{}0.5290 \PYGZhy{}1.6168 1.2039 \PYGZhy{}1.2983 \PYGZhy{}3.3714
5 \PYGZhy{}0.2341  1.5792 0.7674 \PYGZhy{}0.4695  0.5426
6  1.4656 \PYGZhy{}0.2258 0.0675 \PYGZhy{}1.4247 \PYGZhy{}0.5444
\end{sphinxVerbatim}

\end{sphinxuseclass}\end{sphinxVerbatimOutput}

\end{sphinxuseclass}
\sphinxAtStartPar
We can mix functions, lists, dictionaries, etc. that we pass to \sphinxcode{\sphinxupquote{.groupby()}}.

\begin{sphinxuseclass}{cell}\begin{sphinxVerbatimInput}

\begin{sphinxuseclass}{cell_input}
\begin{sphinxVerbatim}[commandchars=\\\{\}]
\PYG{n}{key\PYGZus{}list} \PYG{o}{=} \PYG{p}{[}\PYG{l+s+s1}{\PYGZsq{}}\PYG{l+s+s1}{one}\PYG{l+s+s1}{\PYGZsq{}}\PYG{p}{,} \PYG{l+s+s1}{\PYGZsq{}}\PYG{l+s+s1}{one}\PYG{l+s+s1}{\PYGZsq{}}\PYG{p}{,} \PYG{l+s+s1}{\PYGZsq{}}\PYG{l+s+s1}{one}\PYG{l+s+s1}{\PYGZsq{}}\PYG{p}{,} \PYG{l+s+s1}{\PYGZsq{}}\PYG{l+s+s1}{two}\PYG{l+s+s1}{\PYGZsq{}}\PYG{p}{,} \PYG{l+s+s1}{\PYGZsq{}}\PYG{l+s+s1}{two}\PYG{l+s+s1}{\PYGZsq{}}\PYG{p}{]}
\PYG{n}{people}\PYG{o}{.}\PYG{n}{groupby}\PYG{p}{(}\PYG{p}{[}\PYG{n+nb}{len}\PYG{p}{,} \PYG{n}{key\PYGZus{}list}\PYG{p}{]}\PYG{p}{)}\PYG{o}{.}\PYG{n}{min}\PYG{p}{(}\PYG{p}{)}
\end{sphinxVerbatim}

\end{sphinxuseclass}\end{sphinxVerbatimInput}
\begin{sphinxVerbatimOutput}

\begin{sphinxuseclass}{cell_output}
\begin{sphinxVerbatim}[commandchars=\\\{\}]
            a       b      c       d       e
3 one \PYGZhy{}0.4634 \PYGZhy{}0.4657 0.2420 \PYGZhy{}1.9133 \PYGZhy{}1.7249
  two \PYGZhy{}0.5623 \PYGZhy{}1.0128 0.3142 \PYGZhy{}0.9080 \PYGZhy{}1.4123
5 one \PYGZhy{}0.2341  1.5792 0.7674 \PYGZhy{}0.4695  0.5426
6 two  1.4656 \PYGZhy{}0.2258 0.0675 \PYGZhy{}1.4247 \PYGZhy{}0.5444
\end{sphinxVerbatim}

\end{sphinxuseclass}\end{sphinxVerbatimOutput}

\end{sphinxuseclass}
\begin{sphinxuseclass}{cell}\begin{sphinxVerbatimInput}

\begin{sphinxuseclass}{cell_input}
\begin{sphinxVerbatim}[commandchars=\\\{\}]
\PYG{n}{d} \PYG{o}{=} \PYG{p}{\PYGZob{}}\PYG{l+s+s1}{\PYGZsq{}}\PYG{l+s+s1}{Joe}\PYG{l+s+s1}{\PYGZsq{}}\PYG{p}{:} \PYG{l+s+s1}{\PYGZsq{}}\PYG{l+s+s1}{a}\PYG{l+s+s1}{\PYGZsq{}}\PYG{p}{,} \PYG{l+s+s1}{\PYGZsq{}}\PYG{l+s+s1}{Jim}\PYG{l+s+s1}{\PYGZsq{}}\PYG{p}{:} \PYG{l+s+s1}{\PYGZsq{}}\PYG{l+s+s1}{b}\PYG{l+s+s1}{\PYGZsq{}}\PYG{p}{\PYGZcb{}}
\PYG{n}{people}\PYG{o}{.}\PYG{n}{groupby}\PYG{p}{(}\PYG{p}{[}\PYG{n+nb}{len}\PYG{p}{,} \PYG{n}{d}\PYG{p}{]}\PYG{p}{)}\PYG{o}{.}\PYG{n}{min}\PYG{p}{(}\PYG{p}{)}
\end{sphinxVerbatim}

\end{sphinxuseclass}\end{sphinxVerbatimInput}
\begin{sphinxVerbatimOutput}

\begin{sphinxuseclass}{cell_output}
\begin{sphinxVerbatim}[commandchars=\\\{\}]
          a       b      c       d       e
3 a  0.4967 \PYGZhy{}0.1383 0.6477  1.5230 \PYGZhy{}0.2342
  b \PYGZhy{}0.5623 \PYGZhy{}1.0128 0.3142 \PYGZhy{}0.9080 \PYGZhy{}1.4123
\end{sphinxVerbatim}

\end{sphinxuseclass}\end{sphinxVerbatimOutput}

\end{sphinxuseclass}
\begin{sphinxuseclass}{cell}\begin{sphinxVerbatimInput}

\begin{sphinxuseclass}{cell_input}
\begin{sphinxVerbatim}[commandchars=\\\{\}]
\PYG{n}{d\PYGZus{}2} \PYG{o}{=} \PYG{p}{\PYGZob{}}\PYG{l+s+s1}{\PYGZsq{}}\PYG{l+s+s1}{Joe}\PYG{l+s+s1}{\PYGZsq{}}\PYG{p}{:} \PYG{l+s+s1}{\PYGZsq{}}\PYG{l+s+s1}{Cool}\PYG{l+s+s1}{\PYGZsq{}}\PYG{p}{,} \PYG{l+s+s1}{\PYGZsq{}}\PYG{l+s+s1}{Jim}\PYG{l+s+s1}{\PYGZsq{}}\PYG{p}{:} \PYG{l+s+s1}{\PYGZsq{}}\PYG{l+s+s1}{Nerd}\PYG{l+s+s1}{\PYGZsq{}}\PYG{p}{,} \PYG{l+s+s1}{\PYGZsq{}}\PYG{l+s+s1}{Travis}\PYG{l+s+s1}{\PYGZsq{}}\PYG{p}{:} \PYG{l+s+s1}{\PYGZsq{}}\PYG{l+s+s1}{Cool}\PYG{l+s+s1}{\PYGZsq{}}\PYG{p}{\PYGZcb{}}
\PYG{n}{people}\PYG{o}{.}\PYG{n}{groupby}\PYG{p}{(}\PYG{p}{[}\PYG{n+nb}{len}\PYG{p}{,} \PYG{n}{d\PYGZus{}2}\PYG{p}{]}\PYG{p}{)}\PYG{o}{.}\PYG{n}{min}\PYG{p}{(}\PYG{p}{)}
\end{sphinxVerbatim}

\end{sphinxuseclass}\end{sphinxVerbatimInput}
\begin{sphinxVerbatimOutput}

\begin{sphinxuseclass}{cell_output}
\begin{sphinxVerbatim}[commandchars=\\\{\}]
             a       b      c       d       e
3 Cool  0.4967 \PYGZhy{}0.1383 0.6477  1.5230 \PYGZhy{}0.2342
  Nerd \PYGZhy{}0.5623 \PYGZhy{}1.0128 0.3142 \PYGZhy{}0.9080 \PYGZhy{}1.4123
6 Cool  1.4656 \PYGZhy{}0.2258 0.0675 \PYGZhy{}1.4247 \PYGZhy{}0.5444
\end{sphinxVerbatim}

\end{sphinxuseclass}\end{sphinxVerbatimOutput}

\end{sphinxuseclass}

\subsection{Grouping by Index Levels}
\label{\detokenize{mckinney_10_lecture:grouping-by-index-levels}}
\sphinxAtStartPar
We can also group by index levels.
We can specify index levels by either level number or name.

\begin{sphinxuseclass}{cell}\begin{sphinxVerbatimInput}

\begin{sphinxuseclass}{cell_input}
\begin{sphinxVerbatim}[commandchars=\\\{\}]
\PYG{n}{columns} \PYG{o}{=} \PYG{n}{pd}\PYG{o}{.}\PYG{n}{MultiIndex}\PYG{o}{.}\PYG{n}{from\PYGZus{}arrays}\PYG{p}{(}\PYG{p}{[}\PYG{p}{[}\PYG{l+s+s1}{\PYGZsq{}}\PYG{l+s+s1}{US}\PYG{l+s+s1}{\PYGZsq{}}\PYG{p}{,} \PYG{l+s+s1}{\PYGZsq{}}\PYG{l+s+s1}{US}\PYG{l+s+s1}{\PYGZsq{}}\PYG{p}{,} \PYG{l+s+s1}{\PYGZsq{}}\PYG{l+s+s1}{US}\PYG{l+s+s1}{\PYGZsq{}}\PYG{p}{,} \PYG{l+s+s1}{\PYGZsq{}}\PYG{l+s+s1}{JP}\PYG{l+s+s1}{\PYGZsq{}}\PYG{p}{,} \PYG{l+s+s1}{\PYGZsq{}}\PYG{l+s+s1}{JP}\PYG{l+s+s1}{\PYGZsq{}}\PYG{p}{]}\PYG{p}{,}
                                    \PYG{p}{[}\PYG{l+m+mi}{1}\PYG{p}{,} \PYG{l+m+mi}{3}\PYG{p}{,} \PYG{l+m+mi}{5}\PYG{p}{,} \PYG{l+m+mi}{1}\PYG{p}{,} \PYG{l+m+mi}{3}\PYG{p}{]}\PYG{p}{]}\PYG{p}{,}
                                    \PYG{n}{names}\PYG{o}{=}\PYG{p}{[}\PYG{l+s+s1}{\PYGZsq{}}\PYG{l+s+s1}{cty}\PYG{l+s+s1}{\PYGZsq{}}\PYG{p}{,} \PYG{l+s+s1}{\PYGZsq{}}\PYG{l+s+s1}{tenor}\PYG{l+s+s1}{\PYGZsq{}}\PYG{p}{]}\PYG{p}{)}
\PYG{n}{hier\PYGZus{}df} \PYG{o}{=} \PYG{n}{pd}\PYG{o}{.}\PYG{n}{DataFrame}\PYG{p}{(}\PYG{n}{np}\PYG{o}{.}\PYG{n}{random}\PYG{o}{.}\PYG{n}{randn}\PYG{p}{(}\PYG{l+m+mi}{4}\PYG{p}{,} \PYG{l+m+mi}{5}\PYG{p}{)}\PYG{p}{,} \PYG{n}{columns}\PYG{o}{=}\PYG{n}{columns}\PYG{p}{)}

\PYG{n}{hier\PYGZus{}df}
\end{sphinxVerbatim}

\end{sphinxuseclass}\end{sphinxVerbatimInput}
\begin{sphinxVerbatimOutput}

\begin{sphinxuseclass}{cell_output}
\begin{sphinxVerbatim}[commandchars=\\\{\}]
cty        US                      JP        
tenor       1       3       5       1       3
0      0.1109 \PYGZhy{}1.1510  0.3757 \PYGZhy{}0.6006 \PYGZhy{}0.2917
1     \PYGZhy{}0.6017  1.8523 \PYGZhy{}0.0135 \PYGZhy{}1.0577  0.8225
2     \PYGZhy{}1.2208  0.2089 \PYGZhy{}1.9597 \PYGZhy{}1.3282  0.1969
3      0.7385  0.1714 \PYGZhy{}0.1156 \PYGZhy{}0.3011 \PYGZhy{}1.4785
\end{sphinxVerbatim}

\end{sphinxuseclass}\end{sphinxVerbatimOutput}

\end{sphinxuseclass}
\begin{sphinxuseclass}{cell}\begin{sphinxVerbatimInput}

\begin{sphinxuseclass}{cell_input}
\begin{sphinxVerbatim}[commandchars=\\\{\}]
\PYG{n}{hier\PYGZus{}df}\PYG{o}{.}\PYG{n}{groupby}\PYG{p}{(}\PYG{n}{level}\PYG{o}{=}\PYG{l+s+s1}{\PYGZsq{}}\PYG{l+s+s1}{cty}\PYG{l+s+s1}{\PYGZsq{}}\PYG{p}{,} \PYG{n}{axis}\PYG{o}{=}\PYG{l+m+mi}{1}\PYG{p}{)}\PYG{o}{.}\PYG{n}{count}\PYG{p}{(}\PYG{p}{)}
\end{sphinxVerbatim}

\end{sphinxuseclass}\end{sphinxVerbatimInput}
\begin{sphinxVerbatimOutput}

\begin{sphinxuseclass}{cell_output}
\begin{sphinxVerbatim}[commandchars=\\\{\}]
cty  JP  US
0     2   3
1     2   3
2     2   3
3     2   3
\end{sphinxVerbatim}

\end{sphinxuseclass}\end{sphinxVerbatimOutput}

\end{sphinxuseclass}
\begin{sphinxuseclass}{cell}\begin{sphinxVerbatimInput}

\begin{sphinxuseclass}{cell_input}
\begin{sphinxVerbatim}[commandchars=\\\{\}]
\PYG{n}{hier\PYGZus{}df}\PYG{o}{.}\PYG{n}{groupby}\PYG{p}{(}\PYG{n}{level}\PYG{o}{=}\PYG{l+s+s1}{\PYGZsq{}}\PYG{l+s+s1}{cty}\PYG{l+s+s1}{\PYGZsq{}}\PYG{p}{,} \PYG{n}{axis}\PYG{o}{=}\PYG{l+s+s1}{\PYGZsq{}}\PYG{l+s+s1}{columns}\PYG{l+s+s1}{\PYGZsq{}}\PYG{p}{)}\PYG{o}{.}\PYG{n}{count}\PYG{p}{(}\PYG{p}{)}
\end{sphinxVerbatim}

\end{sphinxuseclass}\end{sphinxVerbatimInput}
\begin{sphinxVerbatimOutput}

\begin{sphinxuseclass}{cell_output}
\begin{sphinxVerbatim}[commandchars=\\\{\}]
cty  JP  US
0     2   3
1     2   3
2     2   3
3     2   3
\end{sphinxVerbatim}

\end{sphinxuseclass}\end{sphinxVerbatimOutput}

\end{sphinxuseclass}
\begin{sphinxuseclass}{cell}\begin{sphinxVerbatimInput}

\begin{sphinxuseclass}{cell_input}
\begin{sphinxVerbatim}[commandchars=\\\{\}]
\PYG{n}{hier\PYGZus{}df}\PYG{o}{.}\PYG{n}{groupby}\PYG{p}{(}\PYG{n}{level}\PYG{o}{=}\PYG{l+s+s1}{\PYGZsq{}}\PYG{l+s+s1}{tenor}\PYG{l+s+s1}{\PYGZsq{}}\PYG{p}{,} \PYG{n}{axis}\PYG{o}{=}\PYG{l+m+mi}{1}\PYG{p}{)}\PYG{o}{.}\PYG{n}{count}\PYG{p}{(}\PYG{p}{)}
\end{sphinxVerbatim}

\end{sphinxuseclass}\end{sphinxVerbatimInput}
\begin{sphinxVerbatimOutput}

\begin{sphinxuseclass}{cell_output}
\begin{sphinxVerbatim}[commandchars=\\\{\}]
tenor  1  3  5
0      2  2  1
1      2  2  1
2      2  2  1
3      2  2  1
\end{sphinxVerbatim}

\end{sphinxuseclass}\end{sphinxVerbatimOutput}

\end{sphinxuseclass}

\section{Data Aggregation}
\label{\detokenize{mckinney_10_lecture:data-aggregation}}
\sphinxAtStartPar
Table 10\sphinxhyphen{}1 provides the optimized groupby methods:
\begin{itemize}
\item {} 
\sphinxAtStartPar
\sphinxcode{\sphinxupquote{count}}: Number of non\sphinxhyphen{}NA values in the group

\item {} 
\sphinxAtStartPar
\sphinxcode{\sphinxupquote{sum}}: Sum of non\sphinxhyphen{}NA values

\item {} 
\sphinxAtStartPar
\sphinxcode{\sphinxupquote{mean}}: Mean of non\sphinxhyphen{}NA values

\item {} 
\sphinxAtStartPar
\sphinxcode{\sphinxupquote{median}}: Arithmetic median of non\sphinxhyphen{}NA values

\item {} 
\sphinxAtStartPar
\sphinxcode{\sphinxupquote{std}}, \sphinxcode{\sphinxupquote{var}}: Unbiased (n – 1 denominator) standard deviation and variance

\item {} 
\sphinxAtStartPar
\sphinxcode{\sphinxupquote{min}}, \sphinxcode{\sphinxupquote{max}}: Minimum and maximum of non\sphinxhyphen{}NA values

\item {} 
\sphinxAtStartPar
\sphinxcode{\sphinxupquote{prod}}: Product of non\sphinxhyphen{}NA values

\item {} 
\sphinxAtStartPar
\sphinxcode{\sphinxupquote{first}}, \sphinxcode{\sphinxupquote{last}}: First and last non\sphinxhyphen{}NA values

\end{itemize}

\sphinxAtStartPar
These optimized methods are fast and efficient, but pandas lets us use other, non\sphinxhyphen{}optimized methods.
First, any series method is available.

\begin{sphinxuseclass}{cell}\begin{sphinxVerbatimInput}

\begin{sphinxuseclass}{cell_input}
\begin{sphinxVerbatim}[commandchars=\\\{\}]
\PYG{n}{df}\PYG{o}{.}\PYG{n}{groupby}\PYG{p}{(}\PYG{l+s+s1}{\PYGZsq{}}\PYG{l+s+s1}{key1}\PYG{l+s+s1}{\PYGZsq{}}\PYG{p}{)}\PYG{p}{[}\PYG{l+s+s1}{\PYGZsq{}}\PYG{l+s+s1}{data1}\PYG{l+s+s1}{\PYGZsq{}}\PYG{p}{]}\PYG{o}{.}\PYG{n}{quantile}\PYG{p}{(}\PYG{l+m+mf}{0.9}\PYG{p}{)}
\end{sphinxVerbatim}

\end{sphinxuseclass}\end{sphinxVerbatimInput}
\begin{sphinxVerbatimOutput}

\begin{sphinxuseclass}{cell_output}
\begin{sphinxVerbatim}[commandchars=\\\{\}]
key1
a   0.3697
b   1.4355
Name: data1, dtype: float64
\end{sphinxVerbatim}

\end{sphinxuseclass}\end{sphinxVerbatimOutput}

\end{sphinxuseclass}
\sphinxAtStartPar
Second, we can write our own functions and pass them to the \sphinxcode{\sphinxupquote{.agg()}} method.
These functions should accept an array and returns a single value.

\begin{sphinxuseclass}{cell}\begin{sphinxVerbatimInput}

\begin{sphinxuseclass}{cell_input}
\begin{sphinxVerbatim}[commandchars=\\\{\}]
\PYG{k}{def} \PYG{n+nf}{max\PYGZus{}minus\PYGZus{}min}\PYG{p}{(}\PYG{n}{arr}\PYG{p}{)}\PYG{p}{:}
    \PYG{k}{return} \PYG{n}{arr}\PYG{o}{.}\PYG{n}{max}\PYG{p}{(}\PYG{p}{)} \PYG{o}{\PYGZhy{}} \PYG{n}{arr}\PYG{o}{.}\PYG{n}{min}\PYG{p}{(}\PYG{p}{)}
\end{sphinxVerbatim}

\end{sphinxuseclass}\end{sphinxVerbatimInput}

\end{sphinxuseclass}
\begin{sphinxuseclass}{cell}\begin{sphinxVerbatimInput}

\begin{sphinxuseclass}{cell_input}
\begin{sphinxVerbatim}[commandchars=\\\{\}]
\PYG{n}{df}\PYG{o}{.}\PYG{n}{sort\PYGZus{}values}\PYG{p}{(}\PYG{n}{by}\PYG{o}{=}\PYG{p}{[}\PYG{l+s+s1}{\PYGZsq{}}\PYG{l+s+s1}{key1}\PYG{l+s+s1}{\PYGZsq{}}\PYG{p}{,} \PYG{l+s+s1}{\PYGZsq{}}\PYG{l+s+s1}{data1}\PYG{l+s+s1}{\PYGZsq{}}\PYG{p}{]}\PYG{p}{)}
\end{sphinxVerbatim}

\end{sphinxuseclass}\end{sphinxVerbatimInput}
\begin{sphinxVerbatimOutput}

\begin{sphinxuseclass}{cell_output}
\begin{sphinxVerbatim}[commandchars=\\\{\}]
  key1 key2   data1   data2
4    a  one \PYGZhy{}0.2342  0.5426
1    a  two \PYGZhy{}0.1383  1.5792
0    a  one  0.4967 \PYGZhy{}0.2341
2    b  one  0.6477  0.7674
3    b  two  1.5230 \PYGZhy{}0.4695
\end{sphinxVerbatim}

\end{sphinxuseclass}\end{sphinxVerbatimOutput}

\end{sphinxuseclass}
\begin{sphinxuseclass}{cell}\begin{sphinxVerbatimInput}

\begin{sphinxuseclass}{cell_input}
\begin{sphinxVerbatim}[commandchars=\\\{\}]
\PYG{n}{df}\PYG{o}{.}\PYG{n}{groupby}\PYG{p}{(}\PYG{l+s+s1}{\PYGZsq{}}\PYG{l+s+s1}{key1}\PYG{l+s+s1}{\PYGZsq{}}\PYG{p}{)}\PYG{p}{[}\PYG{l+s+s1}{\PYGZsq{}}\PYG{l+s+s1}{data1}\PYG{l+s+s1}{\PYGZsq{}}\PYG{p}{]}\PYG{o}{.}\PYG{n}{agg}\PYG{p}{(}\PYG{n}{max\PYGZus{}minus\PYGZus{}min}\PYG{p}{)}
\end{sphinxVerbatim}

\end{sphinxuseclass}\end{sphinxVerbatimInput}
\begin{sphinxVerbatimOutput}

\begin{sphinxuseclass}{cell_output}
\begin{sphinxVerbatim}[commandchars=\\\{\}]
key1
a   0.7309
b   0.8753
Name: data1, dtype: float64
\end{sphinxVerbatim}

\end{sphinxuseclass}\end{sphinxVerbatimOutput}

\end{sphinxuseclass}
\sphinxAtStartPar
Some other methods work, too, even if they are do not aggregate an array to a single value.

\begin{sphinxuseclass}{cell}\begin{sphinxVerbatimInput}

\begin{sphinxuseclass}{cell_input}
\begin{sphinxVerbatim}[commandchars=\\\{\}]
\PYG{n}{df}\PYG{o}{.}\PYG{n}{groupby}\PYG{p}{(}\PYG{l+s+s1}{\PYGZsq{}}\PYG{l+s+s1}{key1}\PYG{l+s+s1}{\PYGZsq{}}\PYG{p}{)}\PYG{p}{[}\PYG{l+s+s1}{\PYGZsq{}}\PYG{l+s+s1}{data1}\PYG{l+s+s1}{\PYGZsq{}}\PYG{p}{]}\PYG{o}{.}\PYG{n}{describe}\PYG{p}{(}\PYG{p}{)}
\end{sphinxVerbatim}

\end{sphinxuseclass}\end{sphinxVerbatimInput}
\begin{sphinxVerbatimOutput}

\begin{sphinxuseclass}{cell_output}
\begin{sphinxVerbatim}[commandchars=\\\{\}]
      count   mean    std     min     25\PYGZpc{}     50\PYGZpc{}    75\PYGZpc{}    max
key1                                                           
a    3.0000 0.0414 0.3972 \PYGZhy{}0.2342 \PYGZhy{}0.1862 \PYGZhy{}0.1383 0.1792 0.4967
b    2.0000 1.0854 0.6190  0.6477  0.8665  1.0854 1.3042 1.5230
\end{sphinxVerbatim}

\end{sphinxuseclass}\end{sphinxVerbatimOutput}

\end{sphinxuseclass}

\subsection{Column\sphinxhyphen{}Wise and Multiple Function Application}
\label{\detokenize{mckinney_10_lecture:column-wise-and-multiple-function-application}}
\sphinxAtStartPar
The \sphinxcode{\sphinxupquote{.agg()}} methods provides two more handy features:
\begin{enumerate}
\sphinxsetlistlabels{\arabic}{enumi}{enumii}{}{.}%
\item {} 
\sphinxAtStartPar
We can pass multiple functions to operate on all of the columns

\item {} 
\sphinxAtStartPar
We can pass specific functions to operate on specific columns

\end{enumerate}

\sphinxAtStartPar
Here is an example with multiple functions:

\begin{sphinxuseclass}{cell}\begin{sphinxVerbatimInput}

\begin{sphinxuseclass}{cell_input}
\begin{sphinxVerbatim}[commandchars=\\\{\}]
\PYG{n}{df}\PYG{o}{.}\PYG{n}{groupby}\PYG{p}{(}\PYG{l+s+s1}{\PYGZsq{}}\PYG{l+s+s1}{key1}\PYG{l+s+s1}{\PYGZsq{}}\PYG{p}{)}\PYG{p}{[}\PYG{l+s+s1}{\PYGZsq{}}\PYG{l+s+s1}{data1}\PYG{l+s+s1}{\PYGZsq{}}\PYG{p}{]}\PYG{o}{.}\PYG{n}{agg}\PYG{p}{(}\PYG{p}{[}\PYG{l+s+s1}{\PYGZsq{}}\PYG{l+s+s1}{mean}\PYG{l+s+s1}{\PYGZsq{}}\PYG{p}{,} \PYG{l+s+s1}{\PYGZsq{}}\PYG{l+s+s1}{median}\PYG{l+s+s1}{\PYGZsq{}}\PYG{p}{,} \PYG{l+s+s1}{\PYGZsq{}}\PYG{l+s+s1}{min}\PYG{l+s+s1}{\PYGZsq{}}\PYG{p}{,} \PYG{l+s+s1}{\PYGZsq{}}\PYG{l+s+s1}{max}\PYG{l+s+s1}{\PYGZsq{}}\PYG{p}{]}\PYG{p}{)}
\end{sphinxVerbatim}

\end{sphinxuseclass}\end{sphinxVerbatimInput}
\begin{sphinxVerbatimOutput}

\begin{sphinxuseclass}{cell_output}
\begin{sphinxVerbatim}[commandchars=\\\{\}]
       mean  median     min    max
key1                              
a    0.0414 \PYGZhy{}0.1383 \PYGZhy{}0.2342 0.4967
b    1.0854  1.0854  0.6477 1.5230
\end{sphinxVerbatim}

\end{sphinxuseclass}\end{sphinxVerbatimOutput}

\end{sphinxuseclass}
\begin{sphinxuseclass}{cell}\begin{sphinxVerbatimInput}

\begin{sphinxuseclass}{cell_input}
\begin{sphinxVerbatim}[commandchars=\\\{\}]
\PYG{n}{df}\PYG{o}{.}\PYG{n}{groupby}\PYG{p}{(}\PYG{l+s+s1}{\PYGZsq{}}\PYG{l+s+s1}{key1}\PYG{l+s+s1}{\PYGZsq{}}\PYG{p}{)}\PYG{p}{[}\PYG{p}{[}\PYG{l+s+s1}{\PYGZsq{}}\PYG{l+s+s1}{data1}\PYG{l+s+s1}{\PYGZsq{}}\PYG{p}{,} \PYG{l+s+s1}{\PYGZsq{}}\PYG{l+s+s1}{data2}\PYG{l+s+s1}{\PYGZsq{}}\PYG{p}{]}\PYG{p}{]}\PYG{o}{.}\PYG{n}{agg}\PYG{p}{(}\PYG{p}{[}\PYG{l+s+s1}{\PYGZsq{}}\PYG{l+s+s1}{mean}\PYG{l+s+s1}{\PYGZsq{}}\PYG{p}{,} \PYG{l+s+s1}{\PYGZsq{}}\PYG{l+s+s1}{median}\PYG{l+s+s1}{\PYGZsq{}}\PYG{p}{,} \PYG{l+s+s1}{\PYGZsq{}}\PYG{l+s+s1}{min}\PYG{l+s+s1}{\PYGZsq{}}\PYG{p}{,} \PYG{l+s+s1}{\PYGZsq{}}\PYG{l+s+s1}{max}\PYG{l+s+s1}{\PYGZsq{}}\PYG{p}{]}\PYG{p}{)}
\end{sphinxVerbatim}

\end{sphinxuseclass}\end{sphinxVerbatimInput}
\begin{sphinxVerbatimOutput}

\begin{sphinxuseclass}{cell_output}
\begin{sphinxVerbatim}[commandchars=\\\{\}]
      data1                         data2                      
       mean  median     min    max   mean median     min    max
key1                                                           
a    0.0414 \PYGZhy{}0.1383 \PYGZhy{}0.2342 0.4967 0.6292 0.5426 \PYGZhy{}0.2341 1.5792
b    1.0854  1.0854  0.6477 1.5230 0.1490 0.1490 \PYGZhy{}0.4695 0.7674
\end{sphinxVerbatim}

\end{sphinxuseclass}\end{sphinxVerbatimOutput}

\end{sphinxuseclass}
\sphinxAtStartPar
What if I wanted to calculate the mean of \sphinxcode{\sphinxupquote{data1}} and the median of \sphinxcode{\sphinxupquote{data2}} by \sphinxcode{\sphinxupquote{key1}}?

\begin{sphinxuseclass}{cell}\begin{sphinxVerbatimInput}

\begin{sphinxuseclass}{cell_input}
\begin{sphinxVerbatim}[commandchars=\\\{\}]
\PYG{n}{df}\PYG{o}{.}\PYG{n}{groupby}\PYG{p}{(}\PYG{l+s+s1}{\PYGZsq{}}\PYG{l+s+s1}{key1}\PYG{l+s+s1}{\PYGZsq{}}\PYG{p}{)}\PYG{o}{.}\PYG{n}{agg}\PYG{p}{(}\PYG{p}{\PYGZob{}}\PYG{l+s+s1}{\PYGZsq{}}\PYG{l+s+s1}{data1}\PYG{l+s+s1}{\PYGZsq{}}\PYG{p}{:} \PYG{l+s+s1}{\PYGZsq{}}\PYG{l+s+s1}{mean}\PYG{l+s+s1}{\PYGZsq{}}\PYG{p}{,} \PYG{l+s+s1}{\PYGZsq{}}\PYG{l+s+s1}{data2}\PYG{l+s+s1}{\PYGZsq{}}\PYG{p}{:} \PYG{l+s+s1}{\PYGZsq{}}\PYG{l+s+s1}{median}\PYG{l+s+s1}{\PYGZsq{}}\PYG{p}{\PYGZcb{}}\PYG{p}{)}
\end{sphinxVerbatim}

\end{sphinxuseclass}\end{sphinxVerbatimInput}
\begin{sphinxVerbatimOutput}

\begin{sphinxuseclass}{cell_output}
\begin{sphinxVerbatim}[commandchars=\\\{\}]
      data1  data2
key1              
a    0.0414 0.5426
b    1.0854 0.1490
\end{sphinxVerbatim}

\end{sphinxuseclass}\end{sphinxVerbatimOutput}

\end{sphinxuseclass}
\sphinxAtStartPar
What if I wanted to calculate the mean \sphinxstyleemphasis{and standard deviation} of \sphinxcode{\sphinxupquote{data1}} and the median of \sphinxcode{\sphinxupquote{data2}} by \sphinxcode{\sphinxupquote{key1}}?

\begin{sphinxuseclass}{cell}\begin{sphinxVerbatimInput}

\begin{sphinxuseclass}{cell_input}
\begin{sphinxVerbatim}[commandchars=\\\{\}]
\PYG{n}{df}\PYG{o}{.}\PYG{n}{groupby}\PYG{p}{(}\PYG{l+s+s1}{\PYGZsq{}}\PYG{l+s+s1}{key1}\PYG{l+s+s1}{\PYGZsq{}}\PYG{p}{)}\PYG{o}{.}\PYG{n}{agg}\PYG{p}{(}\PYG{p}{\PYGZob{}}\PYG{l+s+s1}{\PYGZsq{}}\PYG{l+s+s1}{data1}\PYG{l+s+s1}{\PYGZsq{}}\PYG{p}{:} \PYG{p}{[}\PYG{l+s+s1}{\PYGZsq{}}\PYG{l+s+s1}{mean}\PYG{l+s+s1}{\PYGZsq{}}\PYG{p}{,} \PYG{l+s+s1}{\PYGZsq{}}\PYG{l+s+s1}{std}\PYG{l+s+s1}{\PYGZsq{}}\PYG{p}{]}\PYG{p}{,} \PYG{l+s+s1}{\PYGZsq{}}\PYG{l+s+s1}{data2}\PYG{l+s+s1}{\PYGZsq{}}\PYG{p}{:} \PYG{l+s+s1}{\PYGZsq{}}\PYG{l+s+s1}{median}\PYG{l+s+s1}{\PYGZsq{}}\PYG{p}{\PYGZcb{}}\PYG{p}{)}
\end{sphinxVerbatim}

\end{sphinxuseclass}\end{sphinxVerbatimInput}
\begin{sphinxVerbatimOutput}

\begin{sphinxuseclass}{cell_output}
\begin{sphinxVerbatim}[commandchars=\\\{\}]
      data1         data2
       mean    std median
key1                     
a    0.0414 0.3972 0.5426
b    1.0854 0.6190 0.1490
\end{sphinxVerbatim}

\end{sphinxuseclass}\end{sphinxVerbatimOutput}

\end{sphinxuseclass}

\section{Apply: General split\sphinxhyphen{}apply\sphinxhyphen{}combine}
\label{\detokenize{mckinney_10_lecture:apply-general-split-apply-combine}}
\sphinxAtStartPar
The \sphinxcode{\sphinxupquote{.agg()}} method aggrates an array to a single value.
We can use the \sphinxcode{\sphinxupquote{.apply()}} method for more general calculations.

\sphinxAtStartPar
We can combine the \sphinxcode{\sphinxupquote{.groupby()}} and \sphinxcode{\sphinxupquote{.apply()}} methods to:
\begin{enumerate}
\sphinxsetlistlabels{\arabic}{enumi}{enumii}{}{.}%
\item {} 
\sphinxAtStartPar
Split a dataframe by grouping variables

\item {} 
\sphinxAtStartPar
Call the applied function on each chunk of the original dataframe

\item {} 
\sphinxAtStartPar
Recombine the output of the applied function

\end{enumerate}

\begin{sphinxuseclass}{cell}\begin{sphinxVerbatimInput}

\begin{sphinxuseclass}{cell_input}
\begin{sphinxVerbatim}[commandchars=\\\{\}]
\PYG{k}{def} \PYG{n+nf}{top}\PYG{p}{(}\PYG{n}{x}\PYG{p}{,} \PYG{n}{col}\PYG{p}{,} \PYG{n}{n}\PYG{o}{=}\PYG{l+m+mi}{1}\PYG{p}{)}\PYG{p}{:}
    \PYG{k}{return} \PYG{n}{x}\PYG{o}{.}\PYG{n}{sort\PYGZus{}values}\PYG{p}{(}\PYG{n}{col}\PYG{p}{)}\PYG{o}{.}\PYG{n}{head}\PYG{p}{(}\PYG{n}{n}\PYG{p}{)}
\end{sphinxVerbatim}

\end{sphinxuseclass}\end{sphinxVerbatimInput}

\end{sphinxuseclass}
\begin{sphinxuseclass}{cell}\begin{sphinxVerbatimInput}

\begin{sphinxuseclass}{cell_input}
\begin{sphinxVerbatim}[commandchars=\\\{\}]
\PYG{n}{df}\PYG{o}{.}\PYG{n}{groupby}\PYG{p}{(}\PYG{l+s+s1}{\PYGZsq{}}\PYG{l+s+s1}{key1}\PYG{l+s+s1}{\PYGZsq{}}\PYG{p}{)}\PYG{o}{.}\PYG{n}{apply}\PYG{p}{(}\PYG{n}{top}\PYG{p}{,} \PYG{n}{col}\PYG{o}{=}\PYG{l+s+s1}{\PYGZsq{}}\PYG{l+s+s1}{data1}\PYG{l+s+s1}{\PYGZsq{}}\PYG{p}{)}
\end{sphinxVerbatim}

\end{sphinxuseclass}\end{sphinxVerbatimInput}
\begin{sphinxVerbatimOutput}

\begin{sphinxuseclass}{cell_output}
\begin{sphinxVerbatim}[commandchars=\\\{\}]
       key1 key2   data1  data2
key1                           
a    4    a  one \PYGZhy{}0.2342 0.5426
b    2    b  one  0.6477 0.7674
\end{sphinxVerbatim}

\end{sphinxuseclass}\end{sphinxVerbatimOutput}

\end{sphinxuseclass}
\begin{sphinxuseclass}{cell}\begin{sphinxVerbatimInput}

\begin{sphinxuseclass}{cell_input}
\begin{sphinxVerbatim}[commandchars=\\\{\}]
\PYG{n}{df}\PYG{o}{.}\PYG{n}{groupby}\PYG{p}{(}\PYG{l+s+s1}{\PYGZsq{}}\PYG{l+s+s1}{key1}\PYG{l+s+s1}{\PYGZsq{}}\PYG{p}{)}\PYG{o}{.}\PYG{n}{apply}\PYG{p}{(}\PYG{n}{top}\PYG{p}{,} \PYG{n}{col}\PYG{o}{=}\PYG{l+s+s1}{\PYGZsq{}}\PYG{l+s+s1}{data1}\PYG{l+s+s1}{\PYGZsq{}}\PYG{p}{,} \PYG{n}{n}\PYG{o}{=}\PYG{l+m+mi}{2}\PYG{p}{)}
\end{sphinxVerbatim}

\end{sphinxuseclass}\end{sphinxVerbatimInput}
\begin{sphinxVerbatimOutput}

\begin{sphinxuseclass}{cell_output}
\begin{sphinxVerbatim}[commandchars=\\\{\}]
       key1 key2   data1   data2
key1                            
a    4    a  one \PYGZhy{}0.2342  0.5426
     1    a  two \PYGZhy{}0.1383  1.5792
b    2    b  one  0.6477  0.7674
     3    b  two  1.5230 \PYGZhy{}0.4695
\end{sphinxVerbatim}

\end{sphinxuseclass}\end{sphinxVerbatimOutput}

\end{sphinxuseclass}

\section{Pivot Tables and Cross\sphinxhyphen{}Tabulation}
\label{\detokenize{mckinney_10_lecture:pivot-tables-and-cross-tabulation}}
\sphinxAtStartPar
Above we manually made pivot tables with the \sphinxcode{\sphinxupquote{groupby()}}, \sphinxcode{\sphinxupquote{.agg()}}, \sphinxcode{\sphinxupquote{.apply()}} and \sphinxcode{\sphinxupquote{.unstack()}} methods.
pandas provides a literal interpreation of Excel\sphinxhyphen{}style pivot tables with the \sphinxcode{\sphinxupquote{.pivot\_table()}} method and the \sphinxcode{\sphinxupquote{pandas.pivot\_table()}} function.
These also provide row and column totals via “margins”.
It is worthwhile to read\sphinxhyphen{}through the \sphinxcode{\sphinxupquote{.pivot\_table()}} docstring several times.

\begin{sphinxuseclass}{cell}\begin{sphinxVerbatimInput}

\begin{sphinxuseclass}{cell_input}
\begin{sphinxVerbatim}[commandchars=\\\{\}]
\PYG{n}{ind} \PYG{o}{=} \PYG{p}{(}
    \PYG{n}{yf}\PYG{o}{.}\PYG{n}{download}\PYG{p}{(}
        \PYG{n}{tickers}\PYG{o}{=}\PYG{l+s+s1}{\PYGZsq{}}\PYG{l+s+s1}{\PYGZca{}GSPC \PYGZca{}DJI \PYGZca{}IXIC \PYGZca{}FTSE \PYGZca{}N225 \PYGZca{}HSI}\PYG{l+s+s1}{\PYGZsq{}}\PYG{p}{,}
        \PYG{n}{progress}\PYG{o}{=}\PYG{k+kc}{False}
    \PYG{p}{)}
    \PYG{o}{.}\PYG{n}{rename\PYGZus{}axis}\PYG{p}{(}\PYG{n}{columns}\PYG{o}{=}\PYG{p}{[}\PYG{l+s+s1}{\PYGZsq{}}\PYG{l+s+s1}{Variable}\PYG{l+s+s1}{\PYGZsq{}}\PYG{p}{,} \PYG{l+s+s1}{\PYGZsq{}}\PYG{l+s+s1}{Index}\PYG{l+s+s1}{\PYGZsq{}}\PYG{p}{]}\PYG{p}{)}
    \PYG{o}{.}\PYG{n}{stack}\PYG{p}{(}\PYG{p}{)}
\PYG{p}{)}

\PYG{n}{ind}\PYG{o}{.}\PYG{n}{head}\PYG{p}{(}\PYG{p}{)}
\end{sphinxVerbatim}

\end{sphinxuseclass}\end{sphinxVerbatimInput}
\begin{sphinxVerbatimOutput}

\begin{sphinxuseclass}{cell_output}
\begin{sphinxVerbatim}[commandchars=\\\{\}]
Variable                         Adj Close   Close    High     Low    Open  \PYGZbs{}
Date                      Index                                              
1927\PYGZhy{}12\PYGZhy{}30 05:00:00+00:00 \PYGZca{}GSPC    17.6600 17.6600 17.6600 17.6600 17.6600   
1928\PYGZhy{}01\PYGZhy{}03 05:00:00+00:00 \PYGZca{}GSPC    17.7600 17.7600 17.7600 17.7600 17.7600   
1928\PYGZhy{}01\PYGZhy{}04 05:00:00+00:00 \PYGZca{}GSPC    17.7200 17.7200 17.7200 17.7200 17.7200   
1928\PYGZhy{}01\PYGZhy{}05 05:00:00+00:00 \PYGZca{}GSPC    17.5500 17.5500 17.5500 17.5500 17.5500   
1928\PYGZhy{}01\PYGZhy{}06 05:00:00+00:00 \PYGZca{}GSPC    17.6600 17.6600 17.6600 17.6600 17.6600   

Variable                         Volume  
Date                      Index          
1927\PYGZhy{}12\PYGZhy{}30 05:00:00+00:00 \PYGZca{}GSPC  0.0000  
1928\PYGZhy{}01\PYGZhy{}03 05:00:00+00:00 \PYGZca{}GSPC  0.0000  
1928\PYGZhy{}01\PYGZhy{}04 05:00:00+00:00 \PYGZca{}GSPC  0.0000  
1928\PYGZhy{}01\PYGZhy{}05 05:00:00+00:00 \PYGZca{}GSPC  0.0000  
1928\PYGZhy{}01\PYGZhy{}06 05:00:00+00:00 \PYGZca{}GSPC  0.0000  
\end{sphinxVerbatim}

\end{sphinxuseclass}\end{sphinxVerbatimOutput}

\end{sphinxuseclass}
\sphinxAtStartPar
The default aggregation function for \sphinxcode{\sphinxupquote{.pivot\_table()}} is \sphinxcode{\sphinxupquote{mean}}.

\begin{sphinxuseclass}{cell}\begin{sphinxVerbatimInput}

\begin{sphinxuseclass}{cell_input}
\begin{sphinxVerbatim}[commandchars=\\\{\}]
\PYG{n}{ind}\PYG{o}{.}\PYG{n}{loc}\PYG{p}{[}\PYG{l+s+s1}{\PYGZsq{}}\PYG{l+s+s1}{2015}\PYG{l+s+s1}{\PYGZsq{}}\PYG{p}{:}\PYG{p}{]}\PYG{o}{.}\PYG{n}{pivot\PYGZus{}table}\PYG{p}{(}\PYG{n}{index}\PYG{o}{=}\PYG{l+s+s1}{\PYGZsq{}}\PYG{l+s+s1}{Index}\PYG{l+s+s1}{\PYGZsq{}}\PYG{p}{)}
\end{sphinxVerbatim}

\end{sphinxuseclass}\end{sphinxVerbatimInput}
\begin{sphinxVerbatimOutput}

\begin{sphinxuseclass}{cell_output}
\begin{sphinxVerbatim}[commandchars=\\\{\}]
Variable  Adj Close      Close       High        Low       Open  \PYGZbs{}
Index                                                             
\PYGZca{}DJI     25452.9459 25452.9459 25592.6645 25297.7994 25450.0882   
\PYGZca{}FTSE     6978.1045  6978.1045  7019.7872  6934.9874  6977.4345   
\PYGZca{}GSPC     2998.5095  2998.5095  3014.9221  2979.8820  2998.1779   
\PYGZca{}HSI     25086.5497 25086.5497 25250.8833 24913.3281 25102.2198   
\PYGZca{}IXIC     8588.6746  8588.6746  8647.5013  8521.5014  8587.8332   
\PYGZca{}N225    22464.4354 22464.4354 22583.6795 22337.3847 22466.5559   

Variable          Volume  
Index                     
\PYGZca{}DJI      309385044.8349  
\PYGZca{}FTSE     805027770.8861  
\PYGZca{}GSPC    4017399821.2018  
\PYGZca{}HSI     1986284344.3556  
\PYGZca{}IXIC    3064169027.3571  
\PYGZca{}N225      95184340.3827  
\end{sphinxVerbatim}

\end{sphinxuseclass}\end{sphinxVerbatimOutput}

\end{sphinxuseclass}
\begin{sphinxuseclass}{cell}\begin{sphinxVerbatimInput}

\begin{sphinxuseclass}{cell_input}
\begin{sphinxVerbatim}[commandchars=\\\{\}]
\PYG{n}{ind}\PYG{o}{.}\PYG{n}{loc}\PYG{p}{[}\PYG{l+s+s1}{\PYGZsq{}}\PYG{l+s+s1}{2015}\PYG{l+s+s1}{\PYGZsq{}}\PYG{p}{:}\PYG{p}{]}\PYG{o}{.}\PYG{n}{pivot\PYGZus{}table}\PYG{p}{(}\PYG{n}{index}\PYG{o}{=}\PYG{l+s+s1}{\PYGZsq{}}\PYG{l+s+s1}{Index}\PYG{l+s+s1}{\PYGZsq{}}\PYG{p}{,} \PYG{n}{aggfunc}\PYG{o}{=}\PYG{l+s+s1}{\PYGZsq{}}\PYG{l+s+s1}{median}\PYG{l+s+s1}{\PYGZsq{}}\PYG{p}{)}
\end{sphinxVerbatim}

\end{sphinxuseclass}\end{sphinxVerbatimInput}
\begin{sphinxVerbatimOutput}

\begin{sphinxuseclass}{cell_output}
\begin{sphinxVerbatim}[commandchars=\\\{\}]
Variable  Adj Close      Close       High        Low       Open  \PYGZbs{}
Index                                                             
\PYGZca{}DJI     25410.0293 25410.0293 25551.5605 25243.8809 25388.0801   
\PYGZca{}FTSE     7117.8999  7117.8999  7157.6001  7075.3000  7117.2500   
\PYGZca{}GSPC     2808.4800  2808.4800  2821.2400  2794.9900  2809.3999   
\PYGZca{}HSI     25246.0049 25246.0049 25426.9102 25065.5693 25274.1494   
\PYGZca{}IXIC     7725.5898  7725.5898  7760.8301  7665.2998  7702.0498   
\PYGZca{}N225    21847.0352 21847.0352 21955.6143 21732.5146 21863.8301   

Variable          Volume  
Index                     
\PYGZca{}DJI      295530000.0000  
\PYGZca{}FTSE     761379050.0000  
\PYGZca{}GSPC    3811730000.0000  
\PYGZca{}HSI     1823411600.0000  
\PYGZca{}IXIC    2263640000.0000  
\PYGZca{}N225      80850000.0000  
\end{sphinxVerbatim}

\end{sphinxuseclass}\end{sphinxVerbatimOutput}

\end{sphinxuseclass}
\sphinxAtStartPar
We can use
\sphinxcode{\sphinxupquote{values}} to select specific variables,
\sphinxcode{\sphinxupquote{pd.Grouper()}} to sample different date windows,
and
\sphinxcode{\sphinxupquote{aggfunc}} to select specific aggregation functions.

\begin{sphinxuseclass}{cell}\begin{sphinxVerbatimInput}

\begin{sphinxuseclass}{cell_input}
\begin{sphinxVerbatim}[commandchars=\\\{\}]
\PYG{p}{(}
    \PYG{n}{ind}
    \PYG{o}{.}\PYG{n}{loc}\PYG{p}{[}\PYG{l+s+s1}{\PYGZsq{}}\PYG{l+s+s1}{2015}\PYG{l+s+s1}{\PYGZsq{}}\PYG{p}{:}\PYG{p}{]}
    \PYG{o}{.}\PYG{n}{reset\PYGZus{}index}\PYG{p}{(}\PYG{p}{)}
    \PYG{o}{.}\PYG{n}{pivot\PYGZus{}table}\PYG{p}{(}
        \PYG{n}{values}\PYG{o}{=}\PYG{l+s+s1}{\PYGZsq{}}\PYG{l+s+s1}{Close}\PYG{l+s+s1}{\PYGZsq{}}\PYG{p}{,}
        \PYG{n}{index}\PYG{o}{=}\PYG{n}{pd}\PYG{o}{.}\PYG{n}{Grouper}\PYG{p}{(}\PYG{n}{key}\PYG{o}{=}\PYG{l+s+s1}{\PYGZsq{}}\PYG{l+s+s1}{Date}\PYG{l+s+s1}{\PYGZsq{}}\PYG{p}{,} \PYG{n}{freq}\PYG{o}{=}\PYG{l+s+s1}{\PYGZsq{}}\PYG{l+s+s1}{A}\PYG{l+s+s1}{\PYGZsq{}}\PYG{p}{)}\PYG{p}{,}
        \PYG{n}{columns}\PYG{o}{=}\PYG{l+s+s1}{\PYGZsq{}}\PYG{l+s+s1}{Index}\PYG{l+s+s1}{\PYGZsq{}}\PYG{p}{,}
        \PYG{n}{aggfunc}\PYG{o}{=}\PYG{p}{[}\PYG{l+s+s1}{\PYGZsq{}}\PYG{l+s+s1}{min}\PYG{l+s+s1}{\PYGZsq{}}\PYG{p}{,} \PYG{l+s+s1}{\PYGZsq{}}\PYG{l+s+s1}{max}\PYG{l+s+s1}{\PYGZsq{}}\PYG{p}{]}
    \PYG{p}{)}
\PYG{p}{)}
\end{sphinxVerbatim}

\end{sphinxuseclass}\end{sphinxVerbatimInput}
\begin{sphinxVerbatimOutput}

\begin{sphinxuseclass}{cell_output}
\begin{sphinxVerbatim}[commandchars=\\\{\}]
                                 min                                 \PYGZbs{}
Index                           \PYGZca{}DJI     \PYGZca{}FTSE     \PYGZca{}GSPC       \PYGZca{}HSI   
Date                                                                  
2015\PYGZhy{}12\PYGZhy{}31 00:00:00+00:00 15666.4404 5874.1001 1867.6100 20556.5996   
2016\PYGZhy{}12\PYGZhy{}31 00:00:00+00:00 15660.1797 5537.0000 1829.0800 18319.5801   
2017\PYGZhy{}12\PYGZhy{}31 00:00:00+00:00 19732.4004 7099.2002 2257.8301 22134.4707   
2018\PYGZhy{}12\PYGZhy{}31 00:00:00+00:00 21792.1992 6584.7002 2351.1001 24585.5293   
2019\PYGZhy{}12\PYGZhy{}31 00:00:00+00:00 22686.2207 6692.7002 2447.8899 25064.3594   
2020\PYGZhy{}12\PYGZhy{}31 00:00:00+00:00 18591.9297 4993.8999 2237.3999 21696.1309   
2021\PYGZhy{}12\PYGZhy{}31 00:00:00+00:00 29982.6191 6407.5000 3700.6499 22744.8594   
2022\PYGZhy{}12\PYGZhy{}31 00:00:00+00:00 28725.5098 6826.2002 3577.0300 14687.0195   
2023\PYGZhy{}12\PYGZhy{}31 00:00:00+00:00 32930.0781 7554.1001 3808.1001 20145.2891   

                                                       max            \PYGZbs{}
Index                          \PYGZca{}IXIC      \PYGZca{}N225       \PYGZca{}DJI     \PYGZca{}FTSE   
Date                                                                   
2015\PYGZhy{}12\PYGZhy{}31 00:00:00+00:00  4506.4902 16795.9609 18312.3906 7104.0000   
2016\PYGZhy{}12\PYGZhy{}31 00:00:00+00:00  4266.8398 14952.0195 19974.6191 7142.7998   
2017\PYGZhy{}12\PYGZhy{}31 00:00:00+00:00  5429.0801 18335.6309 24837.5098 7687.7998   
2018\PYGZhy{}12\PYGZhy{}31 00:00:00+00:00  6192.9199 19155.7402 26828.3906 7877.5000   
2019\PYGZhy{}12\PYGZhy{}31 00:00:00+00:00  6463.5000 19561.9609 28645.2598 7686.6001   
2020\PYGZhy{}12\PYGZhy{}31 00:00:00+00:00  6860.6699 16552.8301 30606.4805 7674.6001   
2021\PYGZhy{}12\PYGZhy{}31 00:00:00+00:00 12609.1602 27013.2500 36488.6289 7420.7002   
2022\PYGZhy{}12\PYGZhy{}31 00:00:00+00:00 10213.2900 24717.5293 36799.6484 7672.3999   
2023\PYGZhy{}12\PYGZhy{}31 00:00:00+00:00 10305.2402 25716.8594 34302.6094 8012.5000   

                                                                      
Index                         \PYGZca{}GSPC       \PYGZca{}HSI      \PYGZca{}IXIC      \PYGZca{}N225  
Date                                                                  
2015\PYGZhy{}12\PYGZhy{}31 00:00:00+00:00 2130.8201 28442.7500  5218.8599 20868.0293  
2016\PYGZhy{}12\PYGZhy{}31 00:00:00+00:00 2271.7200 24099.6992  5487.4399 19494.5293  
2017\PYGZhy{}12\PYGZhy{}31 00:00:00+00:00 2690.1599 30003.4902  6994.7598 22939.1797  
2018\PYGZhy{}12\PYGZhy{}31 00:00:00+00:00 2930.7500 33154.1211  8109.6899 24270.6191  
2019\PYGZhy{}12\PYGZhy{}31 00:00:00+00:00 3240.0200 30157.4902  9022.3896 24066.1191  
2020\PYGZhy{}12\PYGZhy{}31 00:00:00+00:00 3756.0701 29056.4199 12899.4199 27568.1504  
2021\PYGZhy{}12\PYGZhy{}31 00:00:00+00:00 4793.0601 31084.9395 16057.4404 30670.0996  
2022\PYGZhy{}12\PYGZhy{}31 00:00:00+00:00 4796.5601 24965.5508 15832.7998 29332.1602  
2023\PYGZhy{}12\PYGZhy{}31 00:00:00+00:00 4179.7598 22688.9004 12200.8203 27696.4395  
\end{sphinxVerbatim}

\end{sphinxuseclass}\end{sphinxVerbatimOutput}

\end{sphinxuseclass}
\sphinxstepscope


\section{McKinney Chapter 10 \sphinxhyphen{} Practice (Blank)}
\label{\detokenize{mckinney_10_practice:mckinney-chapter-10-practice-blank}}\label{\detokenize{mckinney_10_practice::doc}}

\subsection{Announcements}
\label{\detokenize{mckinney_10_practice:announcements}}

\subsection{Practice}
\label{\detokenize{mckinney_10_practice:practice}}
\begin{sphinxuseclass}{cell}\begin{sphinxVerbatimInput}

\begin{sphinxuseclass}{cell_input}
\begin{sphinxVerbatim}[commandchars=\\\{\}]
\PYG{k+kn}{import} \PYG{n+nn}{matplotlib}\PYG{n+nn}{.}\PYG{n+nn}{pyplot} \PYG{k}{as} \PYG{n+nn}{plt}
\PYG{k+kn}{import} \PYG{n+nn}{numpy} \PYG{k}{as} \PYG{n+nn}{np}
\PYG{k+kn}{import} \PYG{n+nn}{pandas} \PYG{k}{as} \PYG{n+nn}{pd}
\end{sphinxVerbatim}

\end{sphinxuseclass}\end{sphinxVerbatimInput}

\end{sphinxuseclass}
\begin{sphinxuseclass}{cell}\begin{sphinxVerbatimInput}

\begin{sphinxuseclass}{cell_input}
\begin{sphinxVerbatim}[commandchars=\\\{\}]
\PYG{o}{\PYGZpc{}}\PYG{k}{config} InlineBackend.figure\PYGZus{}format = \PYGZsq{}retina\PYGZsq{}
\PYG{o}{\PYGZpc{}}\PYG{k}{precision} 4
\PYG{n}{pd}\PYG{o}{.}\PYG{n}{options}\PYG{o}{.}\PYG{n}{display}\PYG{o}{.}\PYG{n}{float\PYGZus{}format} \PYG{o}{=} \PYG{l+s+s1}{\PYGZsq{}}\PYG{l+s+si}{\PYGZob{}:.4f\PYGZcb{}}\PYG{l+s+s1}{\PYGZsq{}}\PYG{o}{.}\PYG{n}{format}
\end{sphinxVerbatim}

\end{sphinxuseclass}\end{sphinxVerbatimInput}

\end{sphinxuseclass}
\begin{sphinxuseclass}{cell}\begin{sphinxVerbatimInput}

\begin{sphinxuseclass}{cell_input}
\begin{sphinxVerbatim}[commandchars=\\\{\}]
\PYG{k+kn}{import} \PYG{n+nn}{yfinance} \PYG{k}{as} \PYG{n+nn}{yf}
\PYG{k+kn}{import} \PYG{n+nn}{pandas\PYGZus{}datareader} \PYG{k}{as} \PYG{n+nn}{pdr}
\PYG{k+kn}{import} \PYG{n+nn}{requests\PYGZus{}cache}
\PYG{n}{session} \PYG{o}{=} \PYG{n}{requests\PYGZus{}cache}\PYG{o}{.}\PYG{n}{CachedSession}\PYG{p}{(}\PYG{p}{)}
\end{sphinxVerbatim}

\end{sphinxuseclass}\end{sphinxVerbatimInput}

\end{sphinxuseclass}

\subsubsection{Replicate the following \sphinxstyleliteralintitle{\sphinxupquote{.pivot\_table()}} output with \sphinxstyleliteralintitle{\sphinxupquote{.groupby()}}}
\label{\detokenize{mckinney_10_practice:replicate-the-following-pivot-table-output-with-groupby}}
\begin{sphinxuseclass}{cell}\begin{sphinxVerbatimInput}

\begin{sphinxuseclass}{cell_input}
\begin{sphinxVerbatim}[commandchars=\\\{\}]
\PYG{n}{ind} \PYG{o}{=} \PYG{p}{(}
    \PYG{n}{yf}\PYG{o}{.}\PYG{n}{download}\PYG{p}{(}
        \PYG{n}{tickers}\PYG{o}{=}\PYG{l+s+s1}{\PYGZsq{}}\PYG{l+s+s1}{\PYGZca{}GSPC \PYGZca{}DJI \PYGZca{}IXIC \PYGZca{}FTSE \PYGZca{}N225 \PYGZca{}HSI}\PYG{l+s+s1}{\PYGZsq{}}\PYG{p}{,}
        \PYG{n}{progress}\PYG{o}{=}\PYG{k+kc}{False}
    \PYG{p}{)}
    \PYG{o}{.}\PYG{n}{rename\PYGZus{}axis}\PYG{p}{(}\PYG{n}{columns}\PYG{o}{=}\PYG{p}{[}\PYG{l+s+s1}{\PYGZsq{}}\PYG{l+s+s1}{Variable}\PYG{l+s+s1}{\PYGZsq{}}\PYG{p}{,} \PYG{l+s+s1}{\PYGZsq{}}\PYG{l+s+s1}{Index}\PYG{l+s+s1}{\PYGZsq{}}\PYG{p}{]}\PYG{p}{)}
    \PYG{o}{.}\PYG{n}{stack}\PYG{p}{(}\PYG{p}{)}
\PYG{p}{)}
\end{sphinxVerbatim}

\end{sphinxuseclass}\end{sphinxVerbatimInput}

\end{sphinxuseclass}
\begin{sphinxuseclass}{cell}\begin{sphinxVerbatimInput}

\begin{sphinxuseclass}{cell_input}
\begin{sphinxVerbatim}[commandchars=\\\{\}]
\PYG{p}{(}
    \PYG{n}{ind}
    \PYG{o}{.}\PYG{n}{loc}\PYG{p}{[}\PYG{l+s+s1}{\PYGZsq{}}\PYG{l+s+s1}{2015}\PYG{l+s+s1}{\PYGZsq{}}\PYG{p}{:}\PYG{p}{]}
    \PYG{o}{.}\PYG{n}{reset\PYGZus{}index}\PYG{p}{(}\PYG{p}{)}
    \PYG{o}{.}\PYG{n}{pivot\PYGZus{}table}\PYG{p}{(}
        \PYG{n}{values}\PYG{o}{=}\PYG{l+s+s1}{\PYGZsq{}}\PYG{l+s+s1}{Close}\PYG{l+s+s1}{\PYGZsq{}}\PYG{p}{,}
        \PYG{n}{index}\PYG{o}{=}\PYG{n}{pd}\PYG{o}{.}\PYG{n}{Grouper}\PYG{p}{(}\PYG{n}{key}\PYG{o}{=}\PYG{l+s+s1}{\PYGZsq{}}\PYG{l+s+s1}{Date}\PYG{l+s+s1}{\PYGZsq{}}\PYG{p}{,} \PYG{n}{freq}\PYG{o}{=}\PYG{l+s+s1}{\PYGZsq{}}\PYG{l+s+s1}{A}\PYG{l+s+s1}{\PYGZsq{}}\PYG{p}{)}\PYG{p}{,}
        \PYG{n}{columns}\PYG{o}{=}\PYG{l+s+s1}{\PYGZsq{}}\PYG{l+s+s1}{Index}\PYG{l+s+s1}{\PYGZsq{}}\PYG{p}{,}
        \PYG{n}{aggfunc}\PYG{o}{=}\PYG{p}{[}\PYG{l+s+s1}{\PYGZsq{}}\PYG{l+s+s1}{min}\PYG{l+s+s1}{\PYGZsq{}}\PYG{p}{,} \PYG{l+s+s1}{\PYGZsq{}}\PYG{l+s+s1}{max}\PYG{l+s+s1}{\PYGZsq{}}\PYG{p}{]}
    \PYG{p}{)}
\PYG{p}{)}
\end{sphinxVerbatim}

\end{sphinxuseclass}\end{sphinxVerbatimInput}
\begin{sphinxVerbatimOutput}

\begin{sphinxuseclass}{cell_output}
\begin{sphinxVerbatim}[commandchars=\\\{\}]
                                 min                                 \PYGZbs{}
Index                           \PYGZca{}DJI     \PYGZca{}FTSE     \PYGZca{}GSPC       \PYGZca{}HSI   
Date                                                                  
2015\PYGZhy{}12\PYGZhy{}31 00:00:00+00:00 15666.4404 5874.1001 1867.6100 20556.5996   
2016\PYGZhy{}12\PYGZhy{}31 00:00:00+00:00 15660.1797 5537.0000 1829.0800 18319.5801   
2017\PYGZhy{}12\PYGZhy{}31 00:00:00+00:00 19732.4004 7099.2002 2257.8301 22134.4707   
2018\PYGZhy{}12\PYGZhy{}31 00:00:00+00:00 21792.1992 6584.7002 2351.1001 24585.5293   
2019\PYGZhy{}12\PYGZhy{}31 00:00:00+00:00 22686.2207 6692.7002 2447.8899 25064.3594   
2020\PYGZhy{}12\PYGZhy{}31 00:00:00+00:00 18591.9297 4993.8999 2237.3999 21696.1309   
2021\PYGZhy{}12\PYGZhy{}31 00:00:00+00:00 29982.6191 6407.5000 3700.6499 22744.8594   
2022\PYGZhy{}12\PYGZhy{}31 00:00:00+00:00 28725.5098 6826.2002 3577.0300 14687.0195   
2023\PYGZhy{}12\PYGZhy{}31 00:00:00+00:00 32930.0781 7554.1001 3808.1001 20145.2891   

                                                       max            \PYGZbs{}
Index                          \PYGZca{}IXIC      \PYGZca{}N225       \PYGZca{}DJI     \PYGZca{}FTSE   
Date                                                                   
2015\PYGZhy{}12\PYGZhy{}31 00:00:00+00:00  4506.4902 16795.9609 18312.3906 7104.0000   
2016\PYGZhy{}12\PYGZhy{}31 00:00:00+00:00  4266.8398 14952.0195 19974.6191 7142.7998   
2017\PYGZhy{}12\PYGZhy{}31 00:00:00+00:00  5429.0801 18335.6309 24837.5098 7687.7998   
2018\PYGZhy{}12\PYGZhy{}31 00:00:00+00:00  6192.9199 19155.7402 26828.3906 7877.5000   
2019\PYGZhy{}12\PYGZhy{}31 00:00:00+00:00  6463.5000 19561.9609 28645.2598 7686.6001   
2020\PYGZhy{}12\PYGZhy{}31 00:00:00+00:00  6860.6699 16552.8301 30606.4805 7674.6001   
2021\PYGZhy{}12\PYGZhy{}31 00:00:00+00:00 12609.1602 27013.2500 36488.6289 7420.7002   
2022\PYGZhy{}12\PYGZhy{}31 00:00:00+00:00 10213.2900 24717.5293 36799.6484 7672.3999   
2023\PYGZhy{}12\PYGZhy{}31 00:00:00+00:00 10305.2402 25716.8594 34302.6094 8012.5000   

                                                                      
Index                         \PYGZca{}GSPC       \PYGZca{}HSI      \PYGZca{}IXIC      \PYGZca{}N225  
Date                                                                  
2015\PYGZhy{}12\PYGZhy{}31 00:00:00+00:00 2130.8201 28442.7500  5218.8599 20868.0293  
2016\PYGZhy{}12\PYGZhy{}31 00:00:00+00:00 2271.7200 24099.6992  5487.4399 19494.5293  
2017\PYGZhy{}12\PYGZhy{}31 00:00:00+00:00 2690.1599 30003.4902  6994.7598 22939.1797  
2018\PYGZhy{}12\PYGZhy{}31 00:00:00+00:00 2930.7500 33154.1211  8109.6899 24270.6191  
2019\PYGZhy{}12\PYGZhy{}31 00:00:00+00:00 3240.0200 30157.4902  9022.3896 24066.1191  
2020\PYGZhy{}12\PYGZhy{}31 00:00:00+00:00 3756.0701 29056.4199 12899.4199 27568.1504  
2021\PYGZhy{}12\PYGZhy{}31 00:00:00+00:00 4793.0601 31084.9395 16057.4404 30670.0996  
2022\PYGZhy{}12\PYGZhy{}31 00:00:00+00:00 4796.5601 24965.5508 15832.7998 29332.1602  
2023\PYGZhy{}12\PYGZhy{}31 00:00:00+00:00 4179.7598 22688.9004 12200.8203 27696.4395  
\end{sphinxVerbatim}

\end{sphinxuseclass}\end{sphinxVerbatimOutput}

\end{sphinxuseclass}

\subsubsection{Calulate the mean and standard deviation of returns by ticker for the MATANA (MSFT, AAPL, TSLA, AMZN, NVDA, and GOOG) stocks}
\label{\detokenize{mckinney_10_practice:calulate-the-mean-and-standard-deviation-of-returns-by-ticker-for-the-matana-msft-aapl-tsla-amzn-nvda-and-goog-stocks}}
\sphinxAtStartPar
Consider only dates with complete returns data.
Try this calculation with wide and long data frames, and confirm your results are the same.


\subsubsection{Calculate the mean and standard deviation of returns and the maximum of closing prices by ticker for the MATANA stocks}
\label{\detokenize{mckinney_10_practice:calculate-the-mean-and-standard-deviation-of-returns-and-the-maximum-of-closing-prices-by-ticker-for-the-matana-stocks}}
\sphinxAtStartPar
Again, consider only dates with complete returns data.
Try this calculation with wide and long data frames, and confirm your results are the same.


\subsubsection{Calculate monthly means and volatilities for SPY and GOOG returns}
\label{\detokenize{mckinney_10_practice:calculate-monthly-means-and-volatilities-for-spy-and-goog-returns}}

\subsubsection{Plot the monthly means and volatilities from the previous exercise}
\label{\detokenize{mckinney_10_practice:plot-the-monthly-means-and-volatilities-from-the-previous-exercise}}

\subsubsection{Assign the Dow Jones stocks to five portfolios based on their monthly volatility}
\label{\detokenize{mckinney_10_practice:assign-the-dow-jones-stocks-to-five-portfolios-based-on-their-monthly-volatility}}
\sphinxAtStartPar
First, we need to download Dow Jones stock data and calculate daily returns.
Use data from 2019 through today.


\subsubsection{Plot the time\sphinxhyphen{}series volatilities of these five portfolios}
\label{\detokenize{mckinney_10_practice:plot-the-time-series-volatilities-of-these-five-portfolios}}
\sphinxAtStartPar
How do these portfolio volatilies compare to (1) each other and (2) the mean volatility of their constituent stocks?


\subsubsection{Calculate the \sphinxstyleemphasis{mean} monthly correlation between the Dow Jones stocks}
\label{\detokenize{mckinney_10_practice:calculate-the-mean-monthly-correlation-between-the-dow-jones-stocks}}
\sphinxAtStartPar
Drop duplicate correlations and self correlations (i.e., correlation between AAPL and AAPL), which are 1, by definition.


\subsubsection{Is market volatility higher during wars?}
\label{\detokenize{mckinney_10_practice:is-market-volatility-higher-during-wars}}
\sphinxAtStartPar
Here is some guidance:
\begin{enumerate}
\sphinxsetlistlabels{\arabic}{enumi}{enumii}{}{.}%
\item {} 
\sphinxAtStartPar
Download the daily factor data from Ken French’s website

\item {} 
\sphinxAtStartPar
Calculate daily market returns by summing the market risk premium and risk\sphinxhyphen{}free rates (\sphinxcode{\sphinxupquote{Mkt\sphinxhyphen{}RF}} and \sphinxcode{\sphinxupquote{RF}}, respectively)

\item {} 
\sphinxAtStartPar
Calculate the volatility (standard deviation) of daily returns \sphinxstyleemphasis{every month} by combining \sphinxcode{\sphinxupquote{pd.Grouper()}} and \sphinxcode{\sphinxupquote{.groupby()}})

\item {} 
\sphinxAtStartPar
Multiply by \$\textbackslash{}sqrt\{252\}\$ to annualize these volatilities of daily returns

\item {} 
\sphinxAtStartPar
Plot these annualized volatilities

\end{enumerate}

\sphinxAtStartPar
Is market volatility higher during wars?
Consider the following dates:
\begin{enumerate}
\sphinxsetlistlabels{\arabic}{enumi}{enumii}{}{.}%
\item {} 
\sphinxAtStartPar
WWII: December 1941 to September 1945

\item {} 
\sphinxAtStartPar
Korean War: 1950 to 1953

\item {} 
\sphinxAtStartPar
Viet Nam War: 1959 to 1975

\item {} 
\sphinxAtStartPar
Gulf War: 1990 to 1991

\item {} 
\sphinxAtStartPar
War in Afghanistan: 2001 to 2021

\end{enumerate}

\sphinxstepscope


\section{McKinney Chapter 10 \sphinxhyphen{} Practice (All Sections)}
\label{\detokenize{mckinney_10_practice_all:mckinney-chapter-10-practice-all-sections}}\label{\detokenize{mckinney_10_practice_all::doc}}

\subsection{Announcements}
\label{\detokenize{mckinney_10_practice_all:announcements}}

\subsection{Practice}
\label{\detokenize{mckinney_10_practice_all:practice}}
\begin{sphinxuseclass}{cell}\begin{sphinxVerbatimInput}

\begin{sphinxuseclass}{cell_input}
\begin{sphinxVerbatim}[commandchars=\\\{\}]
\PYG{k+kn}{import} \PYG{n+nn}{matplotlib}\PYG{n+nn}{.}\PYG{n+nn}{pyplot} \PYG{k}{as} \PYG{n+nn}{plt}
\PYG{k+kn}{import} \PYG{n+nn}{numpy} \PYG{k}{as} \PYG{n+nn}{np}
\PYG{k+kn}{import} \PYG{n+nn}{pandas} \PYG{k}{as} \PYG{n+nn}{pd}
\end{sphinxVerbatim}

\end{sphinxuseclass}\end{sphinxVerbatimInput}

\end{sphinxuseclass}
\begin{sphinxuseclass}{cell}\begin{sphinxVerbatimInput}

\begin{sphinxuseclass}{cell_input}
\begin{sphinxVerbatim}[commandchars=\\\{\}]
\PYG{o}{\PYGZpc{}}\PYG{k}{config} InlineBackend.figure\PYGZus{}format = \PYGZsq{}retina\PYGZsq{}
\PYG{o}{\PYGZpc{}}\PYG{k}{precision} 4
\PYG{n}{pd}\PYG{o}{.}\PYG{n}{options}\PYG{o}{.}\PYG{n}{display}\PYG{o}{.}\PYG{n}{float\PYGZus{}format} \PYG{o}{=} \PYG{l+s+s1}{\PYGZsq{}}\PYG{l+s+si}{\PYGZob{}:.4f\PYGZcb{}}\PYG{l+s+s1}{\PYGZsq{}}\PYG{o}{.}\PYG{n}{format}
\end{sphinxVerbatim}

\end{sphinxuseclass}\end{sphinxVerbatimInput}

\end{sphinxuseclass}
\begin{sphinxuseclass}{cell}\begin{sphinxVerbatimInput}

\begin{sphinxuseclass}{cell_input}
\begin{sphinxVerbatim}[commandchars=\\\{\}]
\PYG{k+kn}{import} \PYG{n+nn}{yfinance} \PYG{k}{as} \PYG{n+nn}{yf}
\PYG{k+kn}{import} \PYG{n+nn}{pandas\PYGZus{}datareader} \PYG{k}{as} \PYG{n+nn}{pdr}
\PYG{k+kn}{import} \PYG{n+nn}{requests\PYGZus{}cache}
\PYG{n}{session} \PYG{o}{=} \PYG{n}{requests\PYGZus{}cache}\PYG{o}{.}\PYG{n}{CachedSession}\PYG{p}{(}\PYG{p}{)}
\end{sphinxVerbatim}

\end{sphinxuseclass}\end{sphinxVerbatimInput}

\end{sphinxuseclass}

\subsubsection{Replicate the following \sphinxstyleliteralintitle{\sphinxupquote{.pivot\_table()}} output with \sphinxstyleliteralintitle{\sphinxupquote{.groupby()}}}
\label{\detokenize{mckinney_10_practice_all:replicate-the-following-pivot-table-output-with-groupby}}
\begin{sphinxuseclass}{cell}\begin{sphinxVerbatimInput}

\begin{sphinxuseclass}{cell_input}
\begin{sphinxVerbatim}[commandchars=\\\{\}]
\PYG{n}{ind} \PYG{o}{=} \PYG{p}{(}
    \PYG{n}{yf}\PYG{o}{.}\PYG{n}{download}\PYG{p}{(}
        \PYG{n}{tickers}\PYG{o}{=}\PYG{l+s+s1}{\PYGZsq{}}\PYG{l+s+s1}{\PYGZca{}GSPC \PYGZca{}DJI \PYGZca{}IXIC \PYGZca{}FTSE \PYGZca{}N225 \PYGZca{}HSI}\PYG{l+s+s1}{\PYGZsq{}}\PYG{p}{,}
        \PYG{n}{progress}\PYG{o}{=}\PYG{k+kc}{False}
    \PYG{p}{)}
    \PYG{o}{.}\PYG{n}{rename\PYGZus{}axis}\PYG{p}{(}\PYG{n}{columns}\PYG{o}{=}\PYG{p}{[}\PYG{l+s+s1}{\PYGZsq{}}\PYG{l+s+s1}{Variable}\PYG{l+s+s1}{\PYGZsq{}}\PYG{p}{,} \PYG{l+s+s1}{\PYGZsq{}}\PYG{l+s+s1}{Index}\PYG{l+s+s1}{\PYGZsq{}}\PYG{p}{]}\PYG{p}{)}
    \PYG{o}{.}\PYG{n}{stack}\PYG{p}{(}\PYG{p}{)}
\PYG{p}{)}

\PYG{n}{ind}\PYG{o}{.}\PYG{n}{head}\PYG{p}{(}\PYG{p}{)}
\end{sphinxVerbatim}

\end{sphinxuseclass}\end{sphinxVerbatimInput}
\begin{sphinxVerbatimOutput}

\begin{sphinxuseclass}{cell_output}
\begin{sphinxVerbatim}[commandchars=\\\{\}]
Variable                         Adj Close   Close    High     Low    Open  \PYGZbs{}
Date                      Index                                              
1927\PYGZhy{}12\PYGZhy{}30 05:00:00+00:00 \PYGZca{}GSPC    17.6600 17.6600 17.6600 17.6600 17.6600   
1928\PYGZhy{}01\PYGZhy{}03 05:00:00+00:00 \PYGZca{}GSPC    17.7600 17.7600 17.7600 17.7600 17.7600   
1928\PYGZhy{}01\PYGZhy{}04 05:00:00+00:00 \PYGZca{}GSPC    17.7200 17.7200 17.7200 17.7200 17.7200   
1928\PYGZhy{}01\PYGZhy{}05 05:00:00+00:00 \PYGZca{}GSPC    17.5500 17.5500 17.5500 17.5500 17.5500   
1928\PYGZhy{}01\PYGZhy{}06 05:00:00+00:00 \PYGZca{}GSPC    17.6600 17.6600 17.6600 17.6600 17.6600   

Variable                         Volume  
Date                      Index          
1927\PYGZhy{}12\PYGZhy{}30 05:00:00+00:00 \PYGZca{}GSPC  0.0000  
1928\PYGZhy{}01\PYGZhy{}03 05:00:00+00:00 \PYGZca{}GSPC  0.0000  
1928\PYGZhy{}01\PYGZhy{}04 05:00:00+00:00 \PYGZca{}GSPC  0.0000  
1928\PYGZhy{}01\PYGZhy{}05 05:00:00+00:00 \PYGZca{}GSPC  0.0000  
1928\PYGZhy{}01\PYGZhy{}06 05:00:00+00:00 \PYGZca{}GSPC  0.0000  
\end{sphinxVerbatim}

\end{sphinxuseclass}\end{sphinxVerbatimOutput}

\end{sphinxuseclass}
\begin{sphinxuseclass}{cell}\begin{sphinxVerbatimInput}

\begin{sphinxuseclass}{cell_input}
\begin{sphinxVerbatim}[commandchars=\\\{\}]
\PYG{p}{(}
    \PYG{n}{ind}
    \PYG{o}{.}\PYG{n}{loc}\PYG{p}{[}\PYG{l+s+s1}{\PYGZsq{}}\PYG{l+s+s1}{2015}\PYG{l+s+s1}{\PYGZsq{}}\PYG{p}{:}\PYG{p}{]}
    \PYG{o}{.}\PYG{n}{reset\PYGZus{}index}\PYG{p}{(}\PYG{p}{)}
    \PYG{o}{.}\PYG{n}{pivot\PYGZus{}table}\PYG{p}{(}
        \PYG{n}{values}\PYG{o}{=}\PYG{l+s+s1}{\PYGZsq{}}\PYG{l+s+s1}{Close}\PYG{l+s+s1}{\PYGZsq{}}\PYG{p}{,}
        \PYG{n}{index}\PYG{o}{=}\PYG{n}{pd}\PYG{o}{.}\PYG{n}{Grouper}\PYG{p}{(}\PYG{n}{key}\PYG{o}{=}\PYG{l+s+s1}{\PYGZsq{}}\PYG{l+s+s1}{Date}\PYG{l+s+s1}{\PYGZsq{}}\PYG{p}{,} \PYG{n}{freq}\PYG{o}{=}\PYG{l+s+s1}{\PYGZsq{}}\PYG{l+s+s1}{A}\PYG{l+s+s1}{\PYGZsq{}}\PYG{p}{)}\PYG{p}{,}
        \PYG{n}{columns}\PYG{o}{=}\PYG{l+s+s1}{\PYGZsq{}}\PYG{l+s+s1}{Index}\PYG{l+s+s1}{\PYGZsq{}}\PYG{p}{,}
        \PYG{n}{aggfunc}\PYG{o}{=}\PYG{p}{[}\PYG{l+s+s1}{\PYGZsq{}}\PYG{l+s+s1}{min}\PYG{l+s+s1}{\PYGZsq{}}\PYG{p}{,} \PYG{l+s+s1}{\PYGZsq{}}\PYG{l+s+s1}{max}\PYG{l+s+s1}{\PYGZsq{}}\PYG{p}{]}
    \PYG{p}{)}
\PYG{p}{)}
\end{sphinxVerbatim}

\end{sphinxuseclass}\end{sphinxVerbatimInput}
\begin{sphinxVerbatimOutput}

\begin{sphinxuseclass}{cell_output}
\begin{sphinxVerbatim}[commandchars=\\\{\}]
                                 min                                 \PYGZbs{}
Index                           \PYGZca{}DJI     \PYGZca{}FTSE     \PYGZca{}GSPC       \PYGZca{}HSI   
Date                                                                  
2015\PYGZhy{}12\PYGZhy{}31 00:00:00+00:00 15666.4404 5874.1001 1867.6100 20556.5996   
2016\PYGZhy{}12\PYGZhy{}31 00:00:00+00:00 15660.1797 5537.0000 1829.0800 18319.5801   
2017\PYGZhy{}12\PYGZhy{}31 00:00:00+00:00 19732.4004 7099.2002 2257.8301 22134.4707   
2018\PYGZhy{}12\PYGZhy{}31 00:00:00+00:00 21792.1992 6584.7002 2351.1001 24585.5293   
2019\PYGZhy{}12\PYGZhy{}31 00:00:00+00:00 22686.2207 6692.7002 2447.8899 25064.3594   
2020\PYGZhy{}12\PYGZhy{}31 00:00:00+00:00 18591.9297 4993.8999 2237.3999 21696.1309   
2021\PYGZhy{}12\PYGZhy{}31 00:00:00+00:00 29982.6191 6407.5000 3700.6499 22744.8594   
2022\PYGZhy{}12\PYGZhy{}31 00:00:00+00:00 28725.5098 6826.2002 3577.0300 14687.0195   
2023\PYGZhy{}12\PYGZhy{}31 00:00:00+00:00 32930.0781 7554.1001 3808.1001 20145.2891   

                                                       max            \PYGZbs{}
Index                          \PYGZca{}IXIC      \PYGZca{}N225       \PYGZca{}DJI     \PYGZca{}FTSE   
Date                                                                   
2015\PYGZhy{}12\PYGZhy{}31 00:00:00+00:00  4506.4902 16795.9609 18312.3906 7104.0000   
2016\PYGZhy{}12\PYGZhy{}31 00:00:00+00:00  4266.8398 14952.0195 19974.6191 7142.7998   
2017\PYGZhy{}12\PYGZhy{}31 00:00:00+00:00  5429.0801 18335.6309 24837.5098 7687.7998   
2018\PYGZhy{}12\PYGZhy{}31 00:00:00+00:00  6192.9199 19155.7402 26828.3906 7877.5000   
2019\PYGZhy{}12\PYGZhy{}31 00:00:00+00:00  6463.5000 19561.9609 28645.2598 7686.6001   
2020\PYGZhy{}12\PYGZhy{}31 00:00:00+00:00  6860.6699 16552.8301 30606.4805 7674.6001   
2021\PYGZhy{}12\PYGZhy{}31 00:00:00+00:00 12609.1602 27013.2500 36488.6289 7420.7002   
2022\PYGZhy{}12\PYGZhy{}31 00:00:00+00:00 10213.2900 24717.5293 36799.6484 7672.3999   
2023\PYGZhy{}12\PYGZhy{}31 00:00:00+00:00 10305.2402 25716.8594 34302.6094 8014.2998   

                                                                      
Index                         \PYGZca{}GSPC       \PYGZca{}HSI      \PYGZca{}IXIC      \PYGZca{}N225  
Date                                                                  
2015\PYGZhy{}12\PYGZhy{}31 00:00:00+00:00 2130.8201 28442.7500  5218.8599 20868.0293  
2016\PYGZhy{}12\PYGZhy{}31 00:00:00+00:00 2271.7200 24099.6992  5487.4399 19494.5293  
2017\PYGZhy{}12\PYGZhy{}31 00:00:00+00:00 2690.1599 30003.4902  6994.7598 22939.1797  
2018\PYGZhy{}12\PYGZhy{}31 00:00:00+00:00 2930.7500 33154.1211  8109.6899 24270.6191  
2019\PYGZhy{}12\PYGZhy{}31 00:00:00+00:00 3240.0200 30157.4902  9022.3896 24066.1191  
2020\PYGZhy{}12\PYGZhy{}31 00:00:00+00:00 3756.0701 29056.4199 12899.4199 27568.1504  
2021\PYGZhy{}12\PYGZhy{}31 00:00:00+00:00 4793.0601 31084.9395 16057.4404 30670.0996  
2022\PYGZhy{}12\PYGZhy{}31 00:00:00+00:00 4796.5601 24965.5508 15832.7998 29332.1602  
2023\PYGZhy{}12\PYGZhy{}31 00:00:00+00:00 4179.7598 22688.9004 12200.8203 27696.4395  
\end{sphinxVerbatim}

\end{sphinxuseclass}\end{sphinxVerbatimOutput}

\end{sphinxuseclass}
\sphinxAtStartPar
Here is the \sphinxcode{\sphinxupquote{.groupby()}} solution!

\begin{sphinxuseclass}{cell}\begin{sphinxVerbatimInput}

\begin{sphinxuseclass}{cell_input}
\begin{sphinxVerbatim}[commandchars=\\\{\}]
\PYG{p}{(}
    \PYG{n}{ind}
    \PYG{o}{.}\PYG{n}{loc}\PYG{p}{[}\PYG{l+s+s1}{\PYGZsq{}}\PYG{l+s+s1}{2015}\PYG{l+s+s1}{\PYGZsq{}}\PYG{p}{:}\PYG{p}{,} \PYG{p}{[}\PYG{l+s+s1}{\PYGZsq{}}\PYG{l+s+s1}{Close}\PYG{l+s+s1}{\PYGZsq{}}\PYG{p}{]}\PYG{p}{]}
    \PYG{o}{.}\PYG{n}{reset\PYGZus{}index}\PYG{p}{(}\PYG{l+s+s1}{\PYGZsq{}}\PYG{l+s+s1}{Index}\PYG{l+s+s1}{\PYGZsq{}}\PYG{p}{)}
    \PYG{o}{.}\PYG{n}{groupby}\PYG{p}{(}\PYG{p}{[}\PYG{n}{pd}\PYG{o}{.}\PYG{n}{Grouper}\PYG{p}{(}\PYG{n}{freq}\PYG{o}{=}\PYG{l+s+s1}{\PYGZsq{}}\PYG{l+s+s1}{A}\PYG{l+s+s1}{\PYGZsq{}}\PYG{p}{)}\PYG{p}{,} \PYG{l+s+s1}{\PYGZsq{}}\PYG{l+s+s1}{Index}\PYG{l+s+s1}{\PYGZsq{}}\PYG{p}{]}\PYG{p}{)}
    \PYG{o}{.}\PYG{n}{agg}\PYG{p}{(}\PYG{p}{[}\PYG{l+s+s1}{\PYGZsq{}}\PYG{l+s+s1}{min}\PYG{l+s+s1}{\PYGZsq{}}\PYG{p}{,} \PYG{l+s+s1}{\PYGZsq{}}\PYG{l+s+s1}{max}\PYG{l+s+s1}{\PYGZsq{}}\PYG{p}{]}\PYG{p}{)}
    \PYG{p}{[}\PYG{l+s+s1}{\PYGZsq{}}\PYG{l+s+s1}{Close}\PYG{l+s+s1}{\PYGZsq{}}\PYG{p}{]}
    \PYG{o}{.}\PYG{n}{unstack}\PYG{p}{(}\PYG{p}{)}
\PYG{p}{)}
\end{sphinxVerbatim}

\end{sphinxuseclass}\end{sphinxVerbatimInput}
\begin{sphinxVerbatimOutput}

\begin{sphinxuseclass}{cell_output}
\begin{sphinxVerbatim}[commandchars=\\\{\}]
                                 min                                 \PYGZbs{}
Index                           \PYGZca{}DJI     \PYGZca{}FTSE     \PYGZca{}GSPC       \PYGZca{}HSI   
Date                                                                  
2015\PYGZhy{}12\PYGZhy{}31 00:00:00+00:00 15666.4404 5874.1001 1867.6100 20556.5996   
2016\PYGZhy{}12\PYGZhy{}31 00:00:00+00:00 15660.1797 5537.0000 1829.0800 18319.5801   
2017\PYGZhy{}12\PYGZhy{}31 00:00:00+00:00 19732.4004 7099.2002 2257.8301 22134.4707   
2018\PYGZhy{}12\PYGZhy{}31 00:00:00+00:00 21792.1992 6584.7002 2351.1001 24585.5293   
2019\PYGZhy{}12\PYGZhy{}31 00:00:00+00:00 22686.2207 6692.7002 2447.8899 25064.3594   
2020\PYGZhy{}12\PYGZhy{}31 00:00:00+00:00 18591.9297 4993.8999 2237.3999 21696.1309   
2021\PYGZhy{}12\PYGZhy{}31 00:00:00+00:00 29982.6191 6407.5000 3700.6499 22744.8594   
2022\PYGZhy{}12\PYGZhy{}31 00:00:00+00:00 28725.5098 6826.2002 3577.0300 14687.0195   
2023\PYGZhy{}12\PYGZhy{}31 00:00:00+00:00 32930.0781 7554.1001 3808.1001 20145.2891   

                                                       max            \PYGZbs{}
Index                          \PYGZca{}IXIC      \PYGZca{}N225       \PYGZca{}DJI     \PYGZca{}FTSE   
Date                                                                   
2015\PYGZhy{}12\PYGZhy{}31 00:00:00+00:00  4506.4902 16795.9609 18312.3906 7104.0000   
2016\PYGZhy{}12\PYGZhy{}31 00:00:00+00:00  4266.8398 14952.0195 19974.6191 7142.7998   
2017\PYGZhy{}12\PYGZhy{}31 00:00:00+00:00  5429.0801 18335.6309 24837.5098 7687.7998   
2018\PYGZhy{}12\PYGZhy{}31 00:00:00+00:00  6192.9199 19155.7402 26828.3906 7877.5000   
2019\PYGZhy{}12\PYGZhy{}31 00:00:00+00:00  6463.5000 19561.9609 28645.2598 7686.6001   
2020\PYGZhy{}12\PYGZhy{}31 00:00:00+00:00  6860.6699 16552.8301 30606.4805 7674.6001   
2021\PYGZhy{}12\PYGZhy{}31 00:00:00+00:00 12609.1602 27013.2500 36488.6289 7420.7002   
2022\PYGZhy{}12\PYGZhy{}31 00:00:00+00:00 10213.2900 24717.5293 36799.6484 7672.3999   
2023\PYGZhy{}12\PYGZhy{}31 00:00:00+00:00 10305.2402 25716.8594 34302.6094 8014.2998   

                                                                      
Index                         \PYGZca{}GSPC       \PYGZca{}HSI      \PYGZca{}IXIC      \PYGZca{}N225  
Date                                                                  
2015\PYGZhy{}12\PYGZhy{}31 00:00:00+00:00 2130.8201 28442.7500  5218.8599 20868.0293  
2016\PYGZhy{}12\PYGZhy{}31 00:00:00+00:00 2271.7200 24099.6992  5487.4399 19494.5293  
2017\PYGZhy{}12\PYGZhy{}31 00:00:00+00:00 2690.1599 30003.4902  6994.7598 22939.1797  
2018\PYGZhy{}12\PYGZhy{}31 00:00:00+00:00 2930.7500 33154.1211  8109.6899 24270.6191  
2019\PYGZhy{}12\PYGZhy{}31 00:00:00+00:00 3240.0200 30157.4902  9022.3896 24066.1191  
2020\PYGZhy{}12\PYGZhy{}31 00:00:00+00:00 3756.0701 29056.4199 12899.4199 27568.1504  
2021\PYGZhy{}12\PYGZhy{}31 00:00:00+00:00 4793.0601 31084.9395 16057.4404 30670.0996  
2022\PYGZhy{}12\PYGZhy{}31 00:00:00+00:00 4796.5601 24965.5508 15832.7998 29332.1602  
2023\PYGZhy{}12\PYGZhy{}31 00:00:00+00:00 4179.7598 22688.9004 12200.8203 27696.4395  
\end{sphinxVerbatim}

\end{sphinxuseclass}\end{sphinxVerbatimOutput}

\end{sphinxuseclass}

\subsubsection{Calulate the mean and standard deviation of returns by ticker for the MATANA (MSFT, AAPL, TSLA, AMZN, NVDA, and GOOG) stocks}
\label{\detokenize{mckinney_10_practice_all:calulate-the-mean-and-standard-deviation-of-returns-by-ticker-for-the-matana-msft-aapl-tsla-amzn-nvda-and-goog-stocks}}
\sphinxAtStartPar
Consider only dates with complete returns data.
Try this calculation with wide and long data frames, and confirm your results are the same.

\begin{sphinxuseclass}{cell}\begin{sphinxVerbatimInput}

\begin{sphinxuseclass}{cell_input}
\begin{sphinxVerbatim}[commandchars=\\\{\}]
\PYG{n}{matana} \PYG{o}{=} \PYG{p}{(}
    \PYG{n}{yf}\PYG{o}{.}\PYG{n}{Tickers}\PYG{p}{(}\PYG{n}{tickers}\PYG{o}{=}\PYG{l+s+s1}{\PYGZsq{}}\PYG{l+s+s1}{MSFT AAPL TSLA AMZN NVDA GOOG}\PYG{l+s+s1}{\PYGZsq{}}\PYG{p}{,} \PYG{n}{session}\PYG{o}{=}\PYG{n}{session}\PYG{p}{)}
    \PYG{o}{.}\PYG{n}{history}\PYG{p}{(}\PYG{n}{period}\PYG{o}{=}\PYG{l+s+s1}{\PYGZsq{}}\PYG{l+s+s1}{max}\PYG{l+s+s1}{\PYGZsq{}}\PYG{p}{,} \PYG{n}{auto\PYGZus{}adjust}\PYG{o}{=}\PYG{k+kc}{False}\PYG{p}{,} \PYG{n}{progress}\PYG{o}{=}\PYG{k+kc}{False}\PYG{p}{)}
    \PYG{o}{.}\PYG{n}{rename\PYGZus{}axis}\PYG{p}{(}\PYG{n}{columns}\PYG{o}{=}\PYG{p}{[}\PYG{l+s+s1}{\PYGZsq{}}\PYG{l+s+s1}{Variable}\PYG{l+s+s1}{\PYGZsq{}}\PYG{p}{,} \PYG{l+s+s1}{\PYGZsq{}}\PYG{l+s+s1}{Ticker}\PYG{l+s+s1}{\PYGZsq{}}\PYG{p}{]}\PYG{p}{)}
\PYG{p}{)}
\end{sphinxVerbatim}

\end{sphinxuseclass}\end{sphinxVerbatimInput}

\end{sphinxuseclass}
\begin{sphinxuseclass}{cell}\begin{sphinxVerbatimInput}

\begin{sphinxuseclass}{cell_input}
\begin{sphinxVerbatim}[commandchars=\\\{\}]
\PYG{n}{returns} \PYG{o}{=} \PYG{n}{matana}\PYG{p}{[}\PYG{l+s+s1}{\PYGZsq{}}\PYG{l+s+s1}{Adj Close}\PYG{l+s+s1}{\PYGZsq{}}\PYG{p}{]}\PYG{o}{.}\PYG{n}{pct\PYGZus{}change}\PYG{p}{(}\PYG{p}{)}\PYG{o}{.}\PYG{n}{dropna}\PYG{p}{(}\PYG{p}{)}
\end{sphinxVerbatim}

\end{sphinxuseclass}\end{sphinxVerbatimInput}

\end{sphinxuseclass}
\begin{sphinxuseclass}{cell}\begin{sphinxVerbatimInput}

\begin{sphinxuseclass}{cell_input}
\begin{sphinxVerbatim}[commandchars=\\\{\}]
\PYG{n}{returns}\PYG{o}{.}\PYG{n}{agg}\PYG{p}{(}\PYG{p}{[}\PYG{l+s+s1}{\PYGZsq{}}\PYG{l+s+s1}{mean}\PYG{l+s+s1}{\PYGZsq{}}\PYG{p}{,} \PYG{l+s+s1}{\PYGZsq{}}\PYG{l+s+s1}{std}\PYG{l+s+s1}{\PYGZsq{}}\PYG{p}{]}\PYG{p}{)}
\end{sphinxVerbatim}

\end{sphinxuseclass}\end{sphinxVerbatimInput}
\begin{sphinxVerbatimOutput}

\begin{sphinxuseclass}{cell_output}
\begin{sphinxVerbatim}[commandchars=\\\{\}]
Ticker   AAPL   AMZN   GOOG   MSFT   NVDA   TSLA
mean   0.0011 0.0011 0.0008 0.0010 0.0018 0.0022
std    0.0180 0.0208 0.0172 0.0165 0.0281 0.0362
\end{sphinxVerbatim}

\end{sphinxuseclass}\end{sphinxVerbatimOutput}

\end{sphinxuseclass}
\begin{sphinxuseclass}{cell}\begin{sphinxVerbatimInput}

\begin{sphinxuseclass}{cell_input}
\begin{sphinxVerbatim}[commandchars=\\\{\}]
\PYG{n}{returns}\PYG{o}{.}\PYG{n}{stack}\PYG{p}{(}\PYG{p}{)}\PYG{o}{.}\PYG{n}{groupby}\PYG{p}{(}\PYG{l+s+s1}{\PYGZsq{}}\PYG{l+s+s1}{Ticker}\PYG{l+s+s1}{\PYGZsq{}}\PYG{p}{)}\PYG{o}{.}\PYG{n}{agg}\PYG{p}{(}\PYG{p}{[}\PYG{l+s+s1}{\PYGZsq{}}\PYG{l+s+s1}{mean}\PYG{l+s+s1}{\PYGZsq{}}\PYG{p}{,} \PYG{l+s+s1}{\PYGZsq{}}\PYG{l+s+s1}{std}\PYG{l+s+s1}{\PYGZsq{}}\PYG{p}{]}\PYG{p}{)}\PYG{o}{.}\PYG{n}{T}
\end{sphinxVerbatim}

\end{sphinxuseclass}\end{sphinxVerbatimInput}
\begin{sphinxVerbatimOutput}

\begin{sphinxuseclass}{cell_output}
\begin{sphinxVerbatim}[commandchars=\\\{\}]
Ticker   AAPL   AMZN   GOOG   MSFT   NVDA   TSLA
mean   0.0011 0.0011 0.0008 0.0010 0.0018 0.0022
std    0.0180 0.0208 0.0172 0.0165 0.0281 0.0362
\end{sphinxVerbatim}

\end{sphinxuseclass}\end{sphinxVerbatimOutput}

\end{sphinxuseclass}
\begin{sphinxuseclass}{cell}\begin{sphinxVerbatimInput}

\begin{sphinxuseclass}{cell_input}
\begin{sphinxVerbatim}[commandchars=\\\{\}]
\PYG{n}{np}\PYG{o}{.}\PYG{n}{allclose}\PYG{p}{(}
    \PYG{n}{returns}\PYG{o}{.}\PYG{n}{agg}\PYG{p}{(}\PYG{p}{[}\PYG{l+s+s1}{\PYGZsq{}}\PYG{l+s+s1}{mean}\PYG{l+s+s1}{\PYGZsq{}}\PYG{p}{,} \PYG{l+s+s1}{\PYGZsq{}}\PYG{l+s+s1}{std}\PYG{l+s+s1}{\PYGZsq{}}\PYG{p}{]}\PYG{p}{)}\PYG{p}{,}
    \PYG{n}{returns}\PYG{o}{.}\PYG{n}{stack}\PYG{p}{(}\PYG{p}{)}\PYG{o}{.}\PYG{n}{groupby}\PYG{p}{(}\PYG{l+s+s1}{\PYGZsq{}}\PYG{l+s+s1}{Ticker}\PYG{l+s+s1}{\PYGZsq{}}\PYG{p}{)}\PYG{o}{.}\PYG{n}{agg}\PYG{p}{(}\PYG{p}{[}\PYG{l+s+s1}{\PYGZsq{}}\PYG{l+s+s1}{mean}\PYG{l+s+s1}{\PYGZsq{}}\PYG{p}{,} \PYG{l+s+s1}{\PYGZsq{}}\PYG{l+s+s1}{std}\PYG{l+s+s1}{\PYGZsq{}}\PYG{p}{]}\PYG{p}{)}\PYG{o}{.}\PYG{n}{T}    
\PYG{p}{)}
\end{sphinxVerbatim}

\end{sphinxuseclass}\end{sphinxVerbatimInput}
\begin{sphinxVerbatimOutput}

\begin{sphinxuseclass}{cell_output}
\begin{sphinxVerbatim}[commandchars=\\\{\}]
True
\end{sphinxVerbatim}

\end{sphinxuseclass}\end{sphinxVerbatimOutput}

\end{sphinxuseclass}

\subsubsection{Calculate the mean and standard deviation of returns and the maximum of closing prices by ticker for the MATANA stocks}
\label{\detokenize{mckinney_10_practice_all:calculate-the-mean-and-standard-deviation-of-returns-and-the-maximum-of-closing-prices-by-ticker-for-the-matana-stocks}}
\sphinxAtStartPar
Again, consider only dates with complete returns data.
Try this calculation with wide and long data frames, and confirm your results are the same.

\begin{sphinxuseclass}{cell}\begin{sphinxVerbatimInput}

\begin{sphinxuseclass}{cell_input}
\begin{sphinxVerbatim}[commandchars=\\\{\}]
\PYG{n}{\PYGZus{}} \PYG{o}{=} \PYG{n}{pd}\PYG{o}{.}\PYG{n}{MultiIndex}\PYG{o}{.}\PYG{n}{from\PYGZus{}product}\PYG{p}{(}\PYG{p}{[}\PYG{p}{[}\PYG{l+s+s1}{\PYGZsq{}}\PYG{l+s+s1}{Returns}\PYG{l+s+s1}{\PYGZsq{}}\PYG{p}{]}\PYG{p}{,} \PYG{n}{matana}\PYG{p}{[}\PYG{l+s+s1}{\PYGZsq{}}\PYG{l+s+s1}{Adj Close}\PYG{l+s+s1}{\PYGZsq{}}\PYG{p}{]}\PYG{p}{]}\PYG{p}{)}
\PYG{n}{matana}\PYG{p}{[}\PYG{n}{\PYGZus{}}\PYG{p}{]} \PYG{o}{=} \PYG{n}{matana}\PYG{p}{[}\PYG{l+s+s1}{\PYGZsq{}}\PYG{l+s+s1}{Adj Close}\PYG{l+s+s1}{\PYGZsq{}}\PYG{p}{]}\PYG{o}{.}\PYG{n}{pct\PYGZus{}change}\PYG{p}{(}\PYG{p}{)}
\end{sphinxVerbatim}

\end{sphinxuseclass}\end{sphinxVerbatimInput}

\end{sphinxuseclass}
\begin{sphinxuseclass}{cell}\begin{sphinxVerbatimInput}

\begin{sphinxuseclass}{cell_input}
\begin{sphinxVerbatim}[commandchars=\\\{\}]
\PYG{p}{(}
    \PYG{n}{matana}
    \PYG{o}{.}\PYG{n}{loc}\PYG{p}{[}\PYG{n}{returns}\PYG{o}{.}\PYG{n}{index}\PYG{p}{]}
    \PYG{o}{.}\PYG{n}{stack}\PYG{p}{(}\PYG{p}{)}
    \PYG{o}{.}\PYG{n}{groupby}\PYG{p}{(}\PYG{l+s+s1}{\PYGZsq{}}\PYG{l+s+s1}{Ticker}\PYG{l+s+s1}{\PYGZsq{}}\PYG{p}{)}
    \PYG{o}{.}\PYG{n}{agg}\PYG{p}{(}\PYG{p}{\PYGZob{}}\PYG{l+s+s1}{\PYGZsq{}}\PYG{l+s+s1}{Returns}\PYG{l+s+s1}{\PYGZsq{}}\PYG{p}{:} \PYG{p}{[}\PYG{l+s+s1}{\PYGZsq{}}\PYG{l+s+s1}{mean}\PYG{l+s+s1}{\PYGZsq{}}\PYG{p}{,} \PYG{l+s+s1}{\PYGZsq{}}\PYG{l+s+s1}{std}\PYG{l+s+s1}{\PYGZsq{}}\PYG{p}{]}\PYG{p}{,} \PYG{l+s+s1}{\PYGZsq{}}\PYG{l+s+s1}{Close}\PYG{l+s+s1}{\PYGZsq{}}\PYG{p}{:} \PYG{p}{[}\PYG{l+s+s1}{\PYGZsq{}}\PYG{l+s+s1}{max}\PYG{l+s+s1}{\PYGZsq{}}\PYG{p}{]}\PYG{p}{\PYGZcb{}}\PYG{p}{)}
\PYG{p}{)}
\end{sphinxVerbatim}

\end{sphinxuseclass}\end{sphinxVerbatimInput}
\begin{sphinxVerbatimOutput}

\begin{sphinxuseclass}{cell_output}
\begin{sphinxVerbatim}[commandchars=\\\{\}]
Variable Returns           Close
            mean    std      max
Ticker                          
AAPL      0.0011 0.0180 182.0100
AMZN      0.0011 0.0208 186.5705
GOOG      0.0008 0.0172 150.7090
MSFT      0.0010 0.0165 343.1100
NVDA      0.0018 0.0281 333.7600
TSLA      0.0022 0.0362 409.9700
\end{sphinxVerbatim}

\end{sphinxuseclass}\end{sphinxVerbatimOutput}

\end{sphinxuseclass}

\subsubsection{Calculate monthly means and volatilities for SPY and GOOG returns}
\label{\detokenize{mckinney_10_practice_all:calculate-monthly-means-and-volatilities-for-spy-and-goog-returns}}
\begin{sphinxuseclass}{cell}\begin{sphinxVerbatimInput}

\begin{sphinxuseclass}{cell_input}
\begin{sphinxVerbatim}[commandchars=\\\{\}]
\PYG{n}{spy\PYGZus{}goog} \PYG{o}{=} \PYG{p}{(}
    \PYG{n}{yf}\PYG{o}{.}\PYG{n}{Tickers}\PYG{p}{(}\PYG{n}{tickers}\PYG{o}{=}\PYG{l+s+s1}{\PYGZsq{}}\PYG{l+s+s1}{SPY GOOG}\PYG{l+s+s1}{\PYGZsq{}}\PYG{p}{,} \PYG{n}{session}\PYG{o}{=}\PYG{n}{session}\PYG{p}{)}
    \PYG{o}{.}\PYG{n}{history}\PYG{p}{(}\PYG{n}{period}\PYG{o}{=}\PYG{l+s+s1}{\PYGZsq{}}\PYG{l+s+s1}{max}\PYG{l+s+s1}{\PYGZsq{}}\PYG{p}{,} \PYG{n}{auto\PYGZus{}adjust}\PYG{o}{=}\PYG{k+kc}{False}\PYG{p}{,} \PYG{n}{progress}\PYG{o}{=}\PYG{k+kc}{False}\PYG{p}{)}
    \PYG{o}{.}\PYG{n}{rename\PYGZus{}axis}\PYG{p}{(}\PYG{n}{columns}\PYG{o}{=}\PYG{p}{[}\PYG{l+s+s1}{\PYGZsq{}}\PYG{l+s+s1}{Variable}\PYG{l+s+s1}{\PYGZsq{}}\PYG{p}{,} \PYG{l+s+s1}{\PYGZsq{}}\PYG{l+s+s1}{Ticker}\PYG{l+s+s1}{\PYGZsq{}}\PYG{p}{]}\PYG{p}{)}
\PYG{p}{)}

\PYG{n}{spy\PYGZus{}goog}\PYG{o}{.}\PYG{n}{head}\PYG{p}{(}\PYG{p}{)}
\end{sphinxVerbatim}

\end{sphinxuseclass}\end{sphinxVerbatimInput}
\begin{sphinxVerbatimOutput}

\begin{sphinxuseclass}{cell_output}
\begin{sphinxVerbatim}[commandchars=\\\{\}]
Variable                  Adj Close         Capital Gains Close          \PYGZbs{}
Ticker                         GOOG     SPY           SPY  GOOG     SPY   
Date                                                                      
1993\PYGZhy{}01\PYGZhy{}29 00:00:00\PYGZhy{}05:00       NaN 25.2182        0.0000   NaN 43.9375   
1993\PYGZhy{}02\PYGZhy{}01 00:00:00\PYGZhy{}05:00       NaN 25.3976        0.0000   NaN 44.2500   
1993\PYGZhy{}02\PYGZhy{}02 00:00:00\PYGZhy{}05:00       NaN 25.4514        0.0000   NaN 44.3438   
1993\PYGZhy{}02\PYGZhy{}03 00:00:00\PYGZhy{}05:00       NaN 25.7204        0.0000   NaN 44.8125   
1993\PYGZhy{}02\PYGZhy{}04 00:00:00\PYGZhy{}05:00       NaN 25.8280        0.0000   NaN 45.0000   

Variable                  Dividends        High          Low         Open  \PYGZbs{}
Ticker                         GOOG    SPY GOOG     SPY GOOG     SPY GOOG   
Date                                                                        
1993\PYGZhy{}01\PYGZhy{}29 00:00:00\PYGZhy{}05:00       NaN 0.0000  NaN 43.9688  NaN 43.7500  NaN   
1993\PYGZhy{}02\PYGZhy{}01 00:00:00\PYGZhy{}05:00       NaN 0.0000  NaN 44.2500  NaN 43.9688  NaN   
1993\PYGZhy{}02\PYGZhy{}02 00:00:00\PYGZhy{}05:00       NaN 0.0000  NaN 44.3750  NaN 44.1250  NaN   
1993\PYGZhy{}02\PYGZhy{}03 00:00:00\PYGZhy{}05:00       NaN 0.0000  NaN 44.8438  NaN 44.3750  NaN   
1993\PYGZhy{}02\PYGZhy{}04 00:00:00\PYGZhy{}05:00       NaN 0.0000  NaN 45.0938  NaN 44.4688  NaN   

Variable                          Stock Splits        Volume           
Ticker                        SPY         GOOG    SPY   GOOG      SPY  
Date                                                                   
1993\PYGZhy{}01\PYGZhy{}29 00:00:00\PYGZhy{}05:00 43.9688          NaN 0.0000    NaN  1003200  
1993\PYGZhy{}02\PYGZhy{}01 00:00:00\PYGZhy{}05:00 43.9688          NaN 0.0000    NaN   480500  
1993\PYGZhy{}02\PYGZhy{}02 00:00:00\PYGZhy{}05:00 44.2188          NaN 0.0000    NaN   201300  
1993\PYGZhy{}02\PYGZhy{}03 00:00:00\PYGZhy{}05:00 44.4062          NaN 0.0000    NaN   529400  
1993\PYGZhy{}02\PYGZhy{}04 00:00:00\PYGZhy{}05:00 44.9688          NaN 0.0000    NaN   531500  
\end{sphinxVerbatim}

\end{sphinxuseclass}\end{sphinxVerbatimOutput}

\end{sphinxuseclass}
\begin{sphinxuseclass}{cell}\begin{sphinxVerbatimInput}

\begin{sphinxuseclass}{cell_input}
\begin{sphinxVerbatim}[commandchars=\\\{\}]
\PYG{n}{spy\PYGZus{}goog\PYGZus{}m} \PYG{o}{=} \PYG{p}{(}
    \PYG{n}{spy\PYGZus{}goog}
    \PYG{o}{.}\PYG{n}{loc}\PYG{p}{[}\PYG{l+s+s1}{\PYGZsq{}}\PYG{l+s+s1}{1993\PYGZhy{}02}\PYG{l+s+s1}{\PYGZsq{}}\PYG{p}{:}\PYG{l+s+s1}{\PYGZsq{}}\PYG{l+s+s1}{2023\PYGZhy{}01}\PYG{l+s+s1}{\PYGZsq{}}\PYG{p}{,} \PYG{l+s+s1}{\PYGZsq{}}\PYG{l+s+s1}{Adj Close}\PYG{l+s+s1}{\PYGZsq{}}\PYG{p}{]}
    \PYG{o}{.}\PYG{n}{pct\PYGZus{}change}\PYG{p}{(}\PYG{p}{)}
    \PYG{o}{.}\PYG{n}{groupby}\PYG{p}{(}\PYG{n}{pd}\PYG{o}{.}\PYG{n}{Grouper}\PYG{p}{(}\PYG{n}{freq}\PYG{o}{=}\PYG{l+s+s1}{\PYGZsq{}}\PYG{l+s+s1}{M}\PYG{l+s+s1}{\PYGZsq{}}\PYG{p}{)}\PYG{p}{)}
    \PYG{o}{.}\PYG{n}{agg}\PYG{p}{(}\PYG{p}{[}\PYG{l+s+s1}{\PYGZsq{}}\PYG{l+s+s1}{mean}\PYG{l+s+s1}{\PYGZsq{}}\PYG{p}{,} \PYG{l+s+s1}{\PYGZsq{}}\PYG{l+s+s1}{std}\PYG{l+s+s1}{\PYGZsq{}}\PYG{p}{]}\PYG{p}{)}
\PYG{p}{)}

\PYG{n}{spy\PYGZus{}goog\PYGZus{}m}
\end{sphinxVerbatim}

\end{sphinxuseclass}\end{sphinxVerbatimInput}
\begin{sphinxVerbatimOutput}

\begin{sphinxuseclass}{cell_output}
\begin{sphinxVerbatim}[commandchars=\\\{\}]
Ticker                       GOOG            SPY       
                             mean    std    mean    std
Date                                                   
1993\PYGZhy{}02\PYGZhy{}28 00:00:00\PYGZhy{}05:00     NaN    NaN  0.0002 0.0081
1993\PYGZhy{}03\PYGZhy{}31 00:00:00\PYGZhy{}05:00     NaN    NaN  0.0010 0.0071
1993\PYGZhy{}04\PYGZhy{}30 00:00:00\PYGZhy{}04:00     NaN    NaN \PYGZhy{}0.0012 0.0074
1993\PYGZhy{}05\PYGZhy{}31 00:00:00\PYGZhy{}04:00     NaN    NaN  0.0014 0.0071
1993\PYGZhy{}06\PYGZhy{}30 00:00:00\PYGZhy{}04:00     NaN    NaN  0.0002 0.0060
...                           ...    ...     ...    ...
2022\PYGZhy{}09\PYGZhy{}30 00:00:00\PYGZhy{}04:00 \PYGZhy{}0.0058 0.0198 \PYGZhy{}0.0045 0.0151
2022\PYGZhy{}10\PYGZhy{}31 00:00:00\PYGZhy{}04:00 \PYGZhy{}0.0003 0.0300  0.0039 0.0176
2022\PYGZhy{}11\PYGZhy{}30 00:00:00\PYGZhy{}05:00  0.0038 0.0316  0.0027 0.0176
2022\PYGZhy{}12\PYGZhy{}31 00:00:00\PYGZhy{}05:00 \PYGZhy{}0.0062 0.0173 \PYGZhy{}0.0028 0.0113
2023\PYGZhy{}01\PYGZhy{}31 00:00:00\PYGZhy{}05:00  0.0061 0.0216  0.0031 0.0105

[360 rows x 4 columns]
\end{sphinxVerbatim}

\end{sphinxuseclass}\end{sphinxVerbatimOutput}

\end{sphinxuseclass}

\subsubsection{Plot the monthly means and volatilities from the previous exercise}
\label{\detokenize{mckinney_10_practice_all:plot-the-monthly-means-and-volatilities-from-the-previous-exercise}}
\sphinxAtStartPar
Here is a first go!

\begin{sphinxuseclass}{cell}\begin{sphinxVerbatimInput}

\begin{sphinxuseclass}{cell_input}
\begin{sphinxVerbatim}[commandchars=\\\{\}]
\PYG{n}{spy\PYGZus{}goog\PYGZus{}m}\PYG{o}{.}\PYG{n}{plot}\PYG{p}{(}\PYG{n}{subplots}\PYG{o}{=}\PYG{k+kc}{True}\PYG{p}{)}
\end{sphinxVerbatim}

\end{sphinxuseclass}\end{sphinxVerbatimInput}
\begin{sphinxVerbatimOutput}

\begin{sphinxuseclass}{cell_output}
\begin{sphinxVerbatim}[commandchars=\\\{\}]
array([\PYGZlt{}AxesSubplot:xlabel=\PYGZsq{}Date\PYGZsq{}\PYGZgt{}, \PYGZlt{}AxesSubplot:xlabel=\PYGZsq{}Date\PYGZsq{}\PYGZgt{},
       \PYGZlt{}AxesSubplot:xlabel=\PYGZsq{}Date\PYGZsq{}\PYGZgt{}, \PYGZlt{}AxesSubplot:xlabel=\PYGZsq{}Date\PYGZsq{}\PYGZgt{}],
      dtype=object)
\end{sphinxVerbatim}

\noindent\sphinxincludegraphics{{57ce4aabcb7cfc753ea0f95ee6806e0dbb999e12ffb4af1fd46c7047a562fb1a}.png}

\end{sphinxuseclass}\end{sphinxVerbatimOutput}

\end{sphinxuseclass}
\sphinxAtStartPar
We can use \sphinxcode{\sphinxupquote{plt.subplots()}} to better organize these plots.
Here \sphinxcode{\sphinxupquote{.plt.subplots()}} creates a tuple of empty axes and assign it to \sphinxcode{\sphinxupquote{ax}}.
Then we can use the \sphinxcode{\sphinxupquote{ax=}} argument to assign each plot to each axes.
I suggest ou pick up these tricks as you go instead of trying to read and memorize the matplotlib manual before you have a very specific task as hand.

\begin{sphinxuseclass}{cell}\begin{sphinxVerbatimInput}

\begin{sphinxuseclass}{cell_input}
\begin{sphinxVerbatim}[commandchars=\\\{\}]
\PYG{n}{fig}\PYG{p}{,} \PYG{n}{ax} \PYG{o}{=} \PYG{n}{plt}\PYG{o}{.}\PYG{n}{subplots}\PYG{p}{(}\PYG{l+m+mi}{2}\PYG{p}{,} \PYG{l+m+mi}{1}\PYG{p}{,} \PYG{n}{sharex}\PYG{o}{=}\PYG{k+kc}{True}\PYG{p}{)}
\PYG{n}{spy\PYGZus{}goog\PYGZus{}m}\PYG{o}{.}\PYG{n}{swaplevel}\PYG{p}{(}\PYG{n}{axis}\PYG{o}{=}\PYG{l+m+mi}{1}\PYG{p}{)}\PYG{p}{[}\PYG{l+s+s1}{\PYGZsq{}}\PYG{l+s+s1}{mean}\PYG{l+s+s1}{\PYGZsq{}}\PYG{p}{]}\PYG{o}{.}\PYG{n}{mul}\PYG{p}{(}\PYG{l+m+mi}{100}\PYG{p}{)}\PYG{o}{.}\PYG{n}{plot}\PYG{p}{(}\PYG{n}{ax}\PYG{o}{=}\PYG{n}{ax}\PYG{p}{[}\PYG{l+m+mi}{0}\PYG{p}{]}\PYG{p}{,} \PYG{n}{ylabel}\PYG{o}{=}\PYG{l+s+s1}{\PYGZsq{}}\PYG{l+s+s1}{Mean of Daily Returns (}\PYG{l+s+s1}{\PYGZpc{}}\PYG{l+s+s1}{)}\PYG{l+s+s1}{\PYGZsq{}}\PYG{p}{)}
\PYG{n}{spy\PYGZus{}goog\PYGZus{}m}\PYG{o}{.}\PYG{n}{swaplevel}\PYG{p}{(}\PYG{n}{axis}\PYG{o}{=}\PYG{l+m+mi}{1}\PYG{p}{)}\PYG{p}{[}\PYG{l+s+s1}{\PYGZsq{}}\PYG{l+s+s1}{std}\PYG{l+s+s1}{\PYGZsq{}}\PYG{p}{]}\PYG{o}{.}\PYG{n}{mul}\PYG{p}{(}\PYG{l+m+mi}{100}\PYG{p}{)}\PYG{o}{.}\PYG{n}{plot}\PYG{p}{(}\PYG{n}{ax}\PYG{o}{=}\PYG{n}{ax}\PYG{p}{[}\PYG{l+m+mi}{1}\PYG{p}{]}\PYG{p}{,} \PYG{n}{ylabel}\PYG{o}{=}\PYG{l+s+s1}{\PYGZsq{}}\PYG{l+s+s1}{SD of Daily Returns (}\PYG{l+s+s1}{\PYGZpc{}}\PYG{l+s+s1}{)}\PYG{l+s+s1}{\PYGZsq{}}\PYG{p}{)}
\PYG{n}{plt}\PYG{o}{.}\PYG{n}{suptitle}\PYG{p}{(}\PYG{l+s+s1}{\PYGZsq{}}\PYG{l+s+s1}{Means and Standard Deviations of Daily Returns}\PYG{l+s+s1}{\PYGZsq{}}\PYG{p}{)}
\PYG{n}{plt}\PYG{o}{.}\PYG{n}{show}\PYG{p}{(}\PYG{p}{)}
\end{sphinxVerbatim}

\end{sphinxuseclass}\end{sphinxVerbatimInput}
\begin{sphinxVerbatimOutput}

\begin{sphinxuseclass}{cell_output}
\noindent\sphinxincludegraphics{{868bfa68c2f523b4db7ae547dfdcbc018f186587bcb1c9775c41bed5d95a5395}.png}

\end{sphinxuseclass}\end{sphinxVerbatimOutput}

\end{sphinxuseclass}

\subsubsection{Assign the Dow Jones stocks to five portfolios based on their monthly volatility}
\label{\detokenize{mckinney_10_practice_all:assign-the-dow-jones-stocks-to-five-portfolios-based-on-their-monthly-volatility}}
\sphinxAtStartPar
First, we need to download Dow Jones stock data and calculate daily returns.
Use data from 2020 through today.

\begin{sphinxuseclass}{cell}\begin{sphinxVerbatimInput}

\begin{sphinxuseclass}{cell_input}
\begin{sphinxVerbatim}[commandchars=\\\{\}]
\PYG{n}{wiki} \PYG{o}{=} \PYG{n}{pd}\PYG{o}{.}\PYG{n}{read\PYGZus{}html}\PYG{p}{(}\PYG{l+s+s1}{\PYGZsq{}}\PYG{l+s+s1}{https://en.wikipedia.org/wiki/Dow\PYGZus{}Jones\PYGZus{}Industrial\PYGZus{}Average}\PYG{l+s+s1}{\PYGZsq{}}\PYG{p}{)}
\end{sphinxVerbatim}

\end{sphinxuseclass}\end{sphinxVerbatimInput}

\end{sphinxuseclass}
\begin{sphinxuseclass}{cell}\begin{sphinxVerbatimInput}

\begin{sphinxuseclass}{cell_input}
\begin{sphinxVerbatim}[commandchars=\\\{\}]
\PYG{n}{dj} \PYG{o}{=} \PYG{p}{(}
    \PYG{n}{yf}\PYG{o}{.}\PYG{n}{Tickers}\PYG{p}{(}\PYG{n}{tickers}\PYG{o}{=}\PYG{n}{wiki}\PYG{p}{[}\PYG{l+m+mi}{1}\PYG{p}{]}\PYG{p}{[}\PYG{l+s+s1}{\PYGZsq{}}\PYG{l+s+s1}{Symbol}\PYG{l+s+s1}{\PYGZsq{}}\PYG{p}{]}\PYG{o}{.}\PYG{n}{to\PYGZus{}list}\PYG{p}{(}\PYG{p}{)}\PYG{p}{,} \PYG{n}{session}\PYG{o}{=}\PYG{n}{session}\PYG{p}{)}
    \PYG{o}{.}\PYG{n}{history}\PYG{p}{(}\PYG{n}{period}\PYG{o}{=}\PYG{l+s+s1}{\PYGZsq{}}\PYG{l+s+s1}{max}\PYG{l+s+s1}{\PYGZsq{}}\PYG{p}{,} \PYG{n}{auto\PYGZus{}adjust}\PYG{o}{=}\PYG{k+kc}{False}\PYG{p}{,} \PYG{n}{progress}\PYG{o}{=}\PYG{k+kc}{False}\PYG{p}{)}
    \PYG{o}{.}\PYG{n}{rename\PYGZus{}axis}\PYG{p}{(}\PYG{n}{columns}\PYG{o}{=}\PYG{p}{[}\PYG{l+s+s1}{\PYGZsq{}}\PYG{l+s+s1}{Variable}\PYG{l+s+s1}{\PYGZsq{}}\PYG{p}{,} \PYG{l+s+s1}{\PYGZsq{}}\PYG{l+s+s1}{Ticker}\PYG{l+s+s1}{\PYGZsq{}}\PYG{p}{]}\PYG{p}{)}
\PYG{p}{)}
\end{sphinxVerbatim}

\end{sphinxuseclass}\end{sphinxVerbatimInput}

\end{sphinxuseclass}
\sphinxAtStartPar
Here I add daily returns.

\begin{sphinxuseclass}{cell}\begin{sphinxVerbatimInput}

\begin{sphinxuseclass}{cell_input}
\begin{sphinxVerbatim}[commandchars=\\\{\}]
\PYG{n}{\PYGZus{}} \PYG{o}{=} \PYG{n}{pd}\PYG{o}{.}\PYG{n}{MultiIndex}\PYG{o}{.}\PYG{n}{from\PYGZus{}product}\PYG{p}{(}\PYG{p}{[}\PYG{p}{[}\PYG{l+s+s1}{\PYGZsq{}}\PYG{l+s+s1}{Returns}\PYG{l+s+s1}{\PYGZsq{}}\PYG{p}{]}\PYG{p}{,} \PYG{n}{dj}\PYG{p}{[}\PYG{l+s+s1}{\PYGZsq{}}\PYG{l+s+s1}{Adj Close}\PYG{l+s+s1}{\PYGZsq{}}\PYG{p}{]}\PYG{p}{]}\PYG{p}{)}
\PYG{n}{dj}\PYG{p}{[}\PYG{n}{\PYGZus{}}\PYG{p}{]} \PYG{o}{=} \PYG{n}{dj}\PYG{p}{[}\PYG{l+s+s1}{\PYGZsq{}}\PYG{l+s+s1}{Adj Close}\PYG{l+s+s1}{\PYGZsq{}}\PYG{p}{]}\PYG{o}{.}\PYG{n}{pct\PYGZus{}change}\PYG{p}{(}\PYG{p}{)}
\end{sphinxVerbatim}

\end{sphinxuseclass}\end{sphinxVerbatimInput}

\end{sphinxuseclass}
\sphinxAtStartPar
Here I add the monthly volatility to all the days in each month for each ticker.

\begin{sphinxuseclass}{cell}\begin{sphinxVerbatimInput}

\begin{sphinxuseclass}{cell_input}
\begin{sphinxVerbatim}[commandchars=\\\{\}]
\PYG{n}{\PYGZus{}} \PYG{o}{=} \PYG{n}{pd}\PYG{o}{.}\PYG{n}{MultiIndex}\PYG{o}{.}\PYG{n}{from\PYGZus{}product}\PYG{p}{(}\PYG{p}{[}\PYG{p}{[}\PYG{l+s+s1}{\PYGZsq{}}\PYG{l+s+s1}{Volatility}\PYG{l+s+s1}{\PYGZsq{}}\PYG{p}{]}\PYG{p}{,} \PYG{n}{dj}\PYG{p}{[}\PYG{l+s+s1}{\PYGZsq{}}\PYG{l+s+s1}{Returns}\PYG{l+s+s1}{\PYGZsq{}}\PYG{p}{]}\PYG{p}{]}\PYG{p}{)}
\PYG{n}{dj}\PYG{p}{[}\PYG{n}{\PYGZus{}}\PYG{p}{]} \PYG{o}{=} \PYG{n}{dj}\PYG{p}{[}\PYG{l+s+s1}{\PYGZsq{}}\PYG{l+s+s1}{Returns}\PYG{l+s+s1}{\PYGZsq{}}\PYG{p}{]}\PYG{o}{.}\PYG{n}{groupby}\PYG{p}{(}\PYG{n}{pd}\PYG{o}{.}\PYG{n}{Grouper}\PYG{p}{(}\PYG{n}{freq}\PYG{o}{=}\PYG{l+s+s1}{\PYGZsq{}}\PYG{l+s+s1}{M}\PYG{l+s+s1}{\PYGZsq{}}\PYG{p}{)}\PYG{p}{)}\PYG{o}{.}\PYG{n}{transform}\PYG{p}{(}\PYG{l+s+s1}{\PYGZsq{}}\PYG{l+s+s1}{std}\PYG{l+s+s1}{\PYGZsq{}}\PYG{p}{)}
\end{sphinxVerbatim}

\end{sphinxuseclass}\end{sphinxVerbatimInput}

\end{sphinxuseclass}
\sphinxAtStartPar
Here I assign stocks to portfolios based on their volatility.

\begin{sphinxuseclass}{cell}\begin{sphinxVerbatimInput}

\begin{sphinxuseclass}{cell_input}
\begin{sphinxVerbatim}[commandchars=\\\{\}]
\PYG{l+m+mi}{1} \PYG{o}{+} \PYG{n}{pd}\PYG{o}{.}\PYG{n}{qcut}\PYG{p}{(}\PYG{n}{np}\PYG{o}{.}\PYG{n}{arange}\PYG{p}{(}\PYG{l+m+mi}{10}\PYG{p}{)}\PYG{p}{,} \PYG{n}{q}\PYG{o}{=}\PYG{l+m+mi}{5}\PYG{p}{,} \PYG{n}{labels}\PYG{o}{=}\PYG{k+kc}{False}\PYG{p}{)}
\end{sphinxVerbatim}

\end{sphinxuseclass}\end{sphinxVerbatimInput}
\begin{sphinxVerbatimOutput}

\begin{sphinxuseclass}{cell_output}
\begin{sphinxVerbatim}[commandchars=\\\{\}]
array([1, 1, 2, 2, 3, 3, 4, 4, 5, 5])
\end{sphinxVerbatim}

\end{sphinxuseclass}\end{sphinxVerbatimOutput}

\end{sphinxuseclass}
\begin{sphinxuseclass}{cell}\begin{sphinxVerbatimInput}

\begin{sphinxuseclass}{cell_input}
\begin{sphinxVerbatim}[commandchars=\\\{\}]
\PYG{n}{\PYGZus{}} \PYG{o}{=} \PYG{n}{pd}\PYG{o}{.}\PYG{n}{MultiIndex}\PYG{o}{.}\PYG{n}{from\PYGZus{}product}\PYG{p}{(}\PYG{p}{[}\PYG{p}{[}\PYG{l+s+s1}{\PYGZsq{}}\PYG{l+s+s1}{Portfolio}\PYG{l+s+s1}{\PYGZsq{}}\PYG{p}{]}\PYG{p}{,} \PYG{n}{dj}\PYG{p}{[}\PYG{l+s+s1}{\PYGZsq{}}\PYG{l+s+s1}{Volatility}\PYG{l+s+s1}{\PYGZsq{}}\PYG{p}{]}\PYG{p}{]}\PYG{p}{)}
\PYG{n}{dj}\PYG{p}{[}\PYG{n}{\PYGZus{}}\PYG{p}{]} \PYG{o}{=} \PYG{n}{dj}\PYG{p}{[}\PYG{l+s+s1}{\PYGZsq{}}\PYG{l+s+s1}{Volatility}\PYG{l+s+s1}{\PYGZsq{}}\PYG{p}{]}\PYG{o}{.}\PYG{n}{apply}\PYG{p}{(}\PYG{n}{pd}\PYG{o}{.}\PYG{n}{qcut}\PYG{p}{,} \PYG{n}{q}\PYG{o}{=}\PYG{l+m+mi}{5}\PYG{p}{,} \PYG{n}{labels}\PYG{o}{=}\PYG{k+kc}{False}\PYG{p}{,} \PYG{n}{axis}\PYG{o}{=}\PYG{l+m+mi}{1}\PYG{p}{)}
\end{sphinxVerbatim}

\end{sphinxuseclass}\end{sphinxVerbatimInput}

\end{sphinxuseclass}

\subsubsection{Plot the time\sphinxhyphen{}series volatilities of these five portfolios}
\label{\detokenize{mckinney_10_practice_all:plot-the-time-series-volatilities-of-these-five-portfolios}}
\sphinxAtStartPar
How do these portfolio volatilies compare to (1) each other and (2) the mean volatility of their constituent stocks?

\begin{sphinxuseclass}{cell}\begin{sphinxVerbatimInput}

\begin{sphinxuseclass}{cell_input}
\begin{sphinxVerbatim}[commandchars=\\\{\}]
\PYG{p}{(}
    \PYG{n}{dj}
    \PYG{o}{.}\PYG{n}{stack}\PYG{p}{(}\PYG{p}{)}
    \PYG{o}{.}\PYG{n}{groupby}\PYG{p}{(}\PYG{p}{[}\PYG{l+s+s1}{\PYGZsq{}}\PYG{l+s+s1}{Date}\PYG{l+s+s1}{\PYGZsq{}}\PYG{p}{,} \PYG{l+s+s1}{\PYGZsq{}}\PYG{l+s+s1}{Portfolio}\PYG{l+s+s1}{\PYGZsq{}}\PYG{p}{]}\PYG{p}{)}
    \PYG{p}{[}\PYG{l+s+s1}{\PYGZsq{}}\PYG{l+s+s1}{Returns}\PYG{l+s+s1}{\PYGZsq{}}\PYG{p}{]}
    \PYG{o}{.}\PYG{n}{mean}\PYG{p}{(}\PYG{p}{)}
    \PYG{o}{.}\PYG{n}{reset\PYGZus{}index}\PYG{p}{(}\PYG{l+s+s1}{\PYGZsq{}}\PYG{l+s+s1}{Portfolio}\PYG{l+s+s1}{\PYGZsq{}}\PYG{p}{)}
    \PYG{o}{.}\PYG{n}{groupby}\PYG{p}{(}\PYG{p}{[}\PYG{n}{pd}\PYG{o}{.}\PYG{n}{Grouper}\PYG{p}{(}\PYG{n}{freq}\PYG{o}{=}\PYG{l+s+s1}{\PYGZsq{}}\PYG{l+s+s1}{M}\PYG{l+s+s1}{\PYGZsq{}}\PYG{p}{)}\PYG{p}{,} \PYG{l+s+s1}{\PYGZsq{}}\PYG{l+s+s1}{Portfolio}\PYG{l+s+s1}{\PYGZsq{}}\PYG{p}{]}\PYG{p}{)}
    \PYG{o}{.}\PYG{n}{std}\PYG{p}{(}\PYG{p}{)}
    \PYG{o}{.}\PYG{n}{rename}\PYG{p}{(}\PYG{n}{columns}\PYG{o}{=}\PYG{p}{\PYGZob{}}\PYG{l+s+s1}{\PYGZsq{}}\PYG{l+s+s1}{Returns}\PYG{l+s+s1}{\PYGZsq{}}\PYG{p}{:} \PYG{l+s+s1}{\PYGZsq{}}\PYG{l+s+s1}{Volatility}\PYG{l+s+s1}{\PYGZsq{}}\PYG{p}{\PYGZcb{}}\PYG{p}{)}
    \PYG{o}{.}\PYG{n}{reset\PYGZus{}index}\PYG{p}{(}\PYG{l+s+s1}{\PYGZsq{}}\PYG{l+s+s1}{Portfolio}\PYG{l+s+s1}{\PYGZsq{}}\PYG{p}{)}
    \PYG{o}{.}\PYG{n}{dropna}\PYG{p}{(}\PYG{p}{)} \PYG{c+c1}{\PYGZsh{} drop missing port.}
    \PYG{o}{.}\PYG{n}{assign}\PYG{p}{(}\PYG{n}{Portfolio} \PYG{o}{=} \PYG{k}{lambda} \PYG{n}{x}\PYG{p}{:} \PYG{l+m+mi}{1} \PYG{o}{+} \PYG{n}{x}\PYG{p}{[}\PYG{l+s+s1}{\PYGZsq{}}\PYG{l+s+s1}{Portfolio}\PYG{l+s+s1}{\PYGZsq{}}\PYG{p}{]}\PYG{o}{.}\PYG{n}{astype}\PYG{p}{(}\PYG{n+nb}{int}\PYG{p}{)}\PYG{p}{)} \PYG{c+c1}{\PYGZsh{} w/o missing portf., can convert to integer}
    \PYG{o}{.}\PYG{n}{set\PYGZus{}index}\PYG{p}{(}\PYG{l+s+s1}{\PYGZsq{}}\PYG{l+s+s1}{Portfolio}\PYG{l+s+s1}{\PYGZsq{}}\PYG{p}{,} \PYG{n}{append}\PYG{o}{=}\PYG{k+kc}{True}\PYG{p}{)}
    \PYG{o}{.}\PYG{n}{unstack}\PYG{p}{(}\PYG{p}{)}
    \PYG{p}{[}\PYG{l+s+s1}{\PYGZsq{}}\PYG{l+s+s1}{Volatility}\PYG{l+s+s1}{\PYGZsq{}}\PYG{p}{]}
    \PYG{o}{.}\PYG{n}{mul}\PYG{p}{(}\PYG{l+m+mi}{100}\PYG{p}{)}
    \PYG{o}{.}\PYG{n}{plot}\PYG{p}{(}\PYG{p}{)}
\PYG{p}{)}

\PYG{n}{plt}\PYG{o}{.}\PYG{n}{ylabel}\PYG{p}{(}\PYG{l+s+s1}{\PYGZsq{}}\PYG{l+s+s1}{Volatility of Daily Returns (}\PYG{l+s+s1}{\PYGZpc{}}\PYG{l+s+s1}{)}\PYG{l+s+s1}{\PYGZsq{}}\PYG{p}{)}
\PYG{n}{plt}\PYG{o}{.}\PYG{n}{title}\PYG{p}{(}\PYG{l+s+s1}{\PYGZsq{}}\PYG{l+s+s1}{Volatility of Five Portfolios Formed on Volatility}\PYG{l+s+s1}{\PYGZsq{}}\PYG{p}{)}
\PYG{n}{plt}\PYG{o}{.}\PYG{n}{show}\PYG{p}{(}\PYG{p}{)}
\end{sphinxVerbatim}

\end{sphinxuseclass}\end{sphinxVerbatimInput}
\begin{sphinxVerbatimOutput}

\begin{sphinxuseclass}{cell_output}
\noindent\sphinxincludegraphics{{66a0cea06b6089c3a7fc11433679caabe29ea557c14803edaa812da94ebaefb1}.png}

\end{sphinxuseclass}\end{sphinxVerbatimOutput}

\end{sphinxuseclass}

\subsubsection{Calculate the \sphinxstyleemphasis{mean} monthly correlation between the Dow Jones stocks}
\label{\detokenize{mckinney_10_practice_all:calculate-the-mean-monthly-correlation-between-the-dow-jones-stocks}}
\sphinxAtStartPar
Drop duplicate correlations and self correlations (i.e., correlation between AAPL and AAPL), which are 1, by definition.

\begin{sphinxuseclass}{cell}\begin{sphinxVerbatimInput}

\begin{sphinxuseclass}{cell_input}
\begin{sphinxVerbatim}[commandchars=\\\{\}]
\PYG{k}{def} \PYG{n+nf}{mu\PYGZus{}rho}\PYG{p}{(}\PYG{n}{x}\PYG{p}{)}\PYG{p}{:}
    \PYG{n}{rhos} \PYG{o}{=} \PYG{n}{x}\PYG{o}{.}\PYG{n}{corr}\PYG{p}{(}\PYG{p}{)}
    \PYG{n}{tril} \PYG{o}{=} \PYG{n}{np}\PYG{o}{.}\PYG{n}{tril}\PYG{p}{(}\PYG{n}{rhos}\PYG{p}{)}
    \PYG{k}{return} \PYG{n}{np}\PYG{o}{.}\PYG{n}{nanmean}\PYG{p}{(}\PYG{n}{rhos}\PYG{o}{.}\PYG{n}{values}\PYG{p}{[}\PYG{p}{(}\PYG{n}{tril} \PYG{o}{!=} \PYG{l+m+mf}{0.}\PYG{p}{)} \PYG{o}{\PYGZam{}} \PYG{p}{(}\PYG{n}{tril} \PYG{o}{!=} \PYG{l+m+mf}{1.}\PYG{p}{)}\PYG{p}{]}\PYG{p}{)}
\end{sphinxVerbatim}

\end{sphinxuseclass}\end{sphinxVerbatimInput}

\end{sphinxuseclass}
\begin{sphinxuseclass}{cell}\begin{sphinxVerbatimInput}

\begin{sphinxuseclass}{cell_input}
\begin{sphinxVerbatim}[commandchars=\\\{\}]
\PYG{p}{(}
    \PYG{n}{dj}
    \PYG{p}{[}\PYG{l+s+s1}{\PYGZsq{}}\PYG{l+s+s1}{Returns}\PYG{l+s+s1}{\PYGZsq{}}\PYG{p}{]}
    \PYG{o}{.}\PYG{n}{groupby}\PYG{p}{(}\PYG{n}{pd}\PYG{o}{.}\PYG{n}{Grouper}\PYG{p}{(}\PYG{n}{freq}\PYG{o}{=}\PYG{l+s+s1}{\PYGZsq{}}\PYG{l+s+s1}{Q}\PYG{l+s+s1}{\PYGZsq{}}\PYG{p}{)}\PYG{p}{)}
    \PYG{o}{.}\PYG{n}{apply}\PYG{p}{(}\PYG{n}{mu\PYGZus{}rho}\PYG{p}{)}
    \PYG{o}{.}\PYG{n}{plot}\PYG{p}{(}\PYG{p}{)}
\PYG{p}{)}

\PYG{n}{plt}\PYG{o}{.}\PYG{n}{ylabel}\PYG{p}{(}\PYG{l+s+s1}{\PYGZsq{}}\PYG{l+s+s1}{Mean Correlation}\PYG{l+s+s1}{\PYGZsq{}}\PYG{p}{)}
\PYG{n}{plt}\PYG{o}{.}\PYG{n}{title}\PYG{p}{(}\PYG{l+s+s1}{\PYGZsq{}}\PYG{l+s+s1}{Mean Correlations of Dow\PYGZhy{}Jones Stocks}\PYG{l+s+s1}{\PYGZsq{}}\PYG{p}{)}
\PYG{n}{plt}\PYG{o}{.}\PYG{n}{show}\PYG{p}{(}\PYG{p}{)}
\end{sphinxVerbatim}

\end{sphinxuseclass}\end{sphinxVerbatimInput}
\begin{sphinxVerbatimOutput}

\begin{sphinxuseclass}{cell_output}
\noindent\sphinxincludegraphics{{2a3e82ea052ce3d0d0c56a8fa05c20dd36cd1b32d8f92a60cf56a84d4aaa9006}.png}

\end{sphinxuseclass}\end{sphinxVerbatimOutput}

\end{sphinxuseclass}

\subsubsection{Is market volatility higher during wars?}
\label{\detokenize{mckinney_10_practice_all:is-market-volatility-higher-during-wars}}
\sphinxAtStartPar
Here is some guidance:
\begin{enumerate}
\sphinxsetlistlabels{\arabic}{enumi}{enumii}{}{.}%
\item {} 
\sphinxAtStartPar
Download the daily factor data from Ken French’s website

\item {} 
\sphinxAtStartPar
Calculate daily market returns by summing the market risk premium and risk\sphinxhyphen{}free rates (\sphinxcode{\sphinxupquote{Mkt\sphinxhyphen{}RF}} and \sphinxcode{\sphinxupquote{RF}}, respectively)

\item {} 
\sphinxAtStartPar
Calculate the volatility (standard deviation) of daily returns \sphinxstyleemphasis{every month} by combining \sphinxcode{\sphinxupquote{pd.Grouper()}} and \sphinxcode{\sphinxupquote{.groupby()}})

\item {} 
\sphinxAtStartPar
Multiply by \$\textbackslash{}sqrt\{252\}\$ to annualize these volatilities of daily returns

\item {} 
\sphinxAtStartPar
Plot these annualized volatilities

\end{enumerate}

\sphinxAtStartPar
Is market volatility higher during wars?
Consider the following dates:
\begin{enumerate}
\sphinxsetlistlabels{\arabic}{enumi}{enumii}{}{.}%
\item {} 
\sphinxAtStartPar
WWII: December 1941 to September 1945

\item {} 
\sphinxAtStartPar
Korean War: 1950 to 1953

\item {} 
\sphinxAtStartPar
Viet Nam War: 1959 to 1975

\item {} 
\sphinxAtStartPar
Gulf War: 1990 to 1991

\item {} 
\sphinxAtStartPar
War in Afghanistan: 2001 to 2021

\end{enumerate}

\begin{sphinxuseclass}{cell}\begin{sphinxVerbatimInput}

\begin{sphinxuseclass}{cell_input}
\begin{sphinxVerbatim}[commandchars=\\\{\}]
\PYG{k+kn}{import} \PYG{n+nn}{pandas\PYGZus{}datareader} \PYG{k}{as} \PYG{n+nn}{pdr}
\end{sphinxVerbatim}

\end{sphinxuseclass}\end{sphinxVerbatimInput}

\end{sphinxuseclass}
\begin{sphinxuseclass}{cell}\begin{sphinxVerbatimInput}

\begin{sphinxuseclass}{cell_input}
\begin{sphinxVerbatim}[commandchars=\\\{\}]
\PYG{n}{pdr}\PYG{o}{.}\PYG{n}{famafrench}\PYG{o}{.}\PYG{n}{get\PYGZus{}available\PYGZus{}datasets}\PYG{p}{(}\PYG{p}{)}\PYG{p}{[}\PYG{p}{:}\PYG{l+m+mi}{5}\PYG{p}{]}
\end{sphinxVerbatim}

\end{sphinxuseclass}\end{sphinxVerbatimInput}
\begin{sphinxVerbatimOutput}

\begin{sphinxuseclass}{cell_output}
\begin{sphinxVerbatim}[commandchars=\\\{\}]
[\PYGZsq{}F\PYGZhy{}F\PYGZus{}Research\PYGZus{}Data\PYGZus{}Factors\PYGZsq{},
 \PYGZsq{}F\PYGZhy{}F\PYGZus{}Research\PYGZus{}Data\PYGZus{}Factors\PYGZus{}weekly\PYGZsq{},
 \PYGZsq{}F\PYGZhy{}F\PYGZus{}Research\PYGZus{}Data\PYGZus{}Factors\PYGZus{}daily\PYGZsq{},
 \PYGZsq{}F\PYGZhy{}F\PYGZus{}Research\PYGZus{}Data\PYGZus{}5\PYGZus{}Factors\PYGZus{}2x3\PYGZsq{},
 \PYGZsq{}F\PYGZhy{}F\PYGZus{}Research\PYGZus{}Data\PYGZus{}5\PYGZus{}Factors\PYGZus{}2x3\PYGZus{}daily\PYGZsq{}]
\end{sphinxVerbatim}

\end{sphinxuseclass}\end{sphinxVerbatimOutput}

\end{sphinxuseclass}
\begin{sphinxuseclass}{cell}\begin{sphinxVerbatimInput}

\begin{sphinxuseclass}{cell_input}
\begin{sphinxVerbatim}[commandchars=\\\{\}]
\PYG{n}{ff\PYGZus{}all} \PYG{o}{=} \PYG{n}{pdr}\PYG{o}{.}\PYG{n}{DataReader}\PYG{p}{(}
    \PYG{n}{name}\PYG{o}{=}\PYG{l+s+s1}{\PYGZsq{}}\PYG{l+s+s1}{F\PYGZhy{}F\PYGZus{}Research\PYGZus{}Data\PYGZus{}Factors\PYGZus{}daily}\PYG{l+s+s1}{\PYGZsq{}}\PYG{p}{,}
    \PYG{n}{data\PYGZus{}source}\PYG{o}{=}\PYG{l+s+s1}{\PYGZsq{}}\PYG{l+s+s1}{famafrench}\PYG{l+s+s1}{\PYGZsq{}}\PYG{p}{,}
    \PYG{n}{start}\PYG{o}{=}\PYG{l+s+s1}{\PYGZsq{}}\PYG{l+s+s1}{1900}\PYG{l+s+s1}{\PYGZsq{}}\PYG{p}{,}
    \PYG{n}{session}\PYG{o}{=}\PYG{n}{session}
\PYG{p}{)}
\end{sphinxVerbatim}

\end{sphinxuseclass}\end{sphinxVerbatimInput}

\end{sphinxuseclass}
\begin{sphinxuseclass}{cell}\begin{sphinxVerbatimInput}

\begin{sphinxuseclass}{cell_input}
\begin{sphinxVerbatim}[commandchars=\\\{\}]
\PYG{p}{(}
    \PYG{n}{ff\PYGZus{}all}\PYG{p}{[}\PYG{l+m+mi}{0}\PYG{p}{]}
    \PYG{o}{.}\PYG{n}{assign}\PYG{p}{(}\PYG{n}{Mkt} \PYG{o}{=} \PYG{k}{lambda} \PYG{n}{x}\PYG{p}{:} \PYG{n}{x}\PYG{p}{[}\PYG{l+s+s1}{\PYGZsq{}}\PYG{l+s+s1}{Mkt\PYGZhy{}RF}\PYG{l+s+s1}{\PYGZsq{}}\PYG{p}{]} \PYG{o}{+} \PYG{n}{x}\PYG{p}{[}\PYG{l+s+s1}{\PYGZsq{}}\PYG{l+s+s1}{RF}\PYG{l+s+s1}{\PYGZsq{}}\PYG{p}{]}\PYG{p}{)}
    \PYG{p}{[}\PYG{l+s+s1}{\PYGZsq{}}\PYG{l+s+s1}{Mkt}\PYG{l+s+s1}{\PYGZsq{}}\PYG{p}{]}
    \PYG{o}{.}\PYG{n}{groupby}\PYG{p}{(}\PYG{n}{pd}\PYG{o}{.}\PYG{n}{Grouper}\PYG{p}{(}\PYG{n}{freq}\PYG{o}{=}\PYG{l+s+s1}{\PYGZsq{}}\PYG{l+s+s1}{M}\PYG{l+s+s1}{\PYGZsq{}}\PYG{p}{)}\PYG{p}{)}
    \PYG{o}{.}\PYG{n}{std}\PYG{p}{(}\PYG{p}{)}
    \PYG{o}{.}\PYG{n}{mul}\PYG{p}{(}\PYG{n}{np}\PYG{o}{.}\PYG{n}{sqrt}\PYG{p}{(}\PYG{l+m+mi}{252}\PYG{p}{)}\PYG{p}{)}
    \PYG{o}{.}\PYG{n}{plot}\PYG{p}{(}\PYG{p}{)}
\PYG{p}{)}

\PYG{c+c1}{\PYGZsh{} adds vertical bands for U.S. wars}
\PYG{n}{plt}\PYG{o}{.}\PYG{n}{axvspan}\PYG{p}{(}\PYG{l+s+s1}{\PYGZsq{}}\PYG{l+s+s1}{1941\PYGZhy{}12}\PYG{l+s+s1}{\PYGZsq{}}\PYG{p}{,} \PYG{l+s+s1}{\PYGZsq{}}\PYG{l+s+s1}{1945\PYGZhy{}09}\PYG{l+s+s1}{\PYGZsq{}}\PYG{p}{,} \PYG{n}{alpha}\PYG{o}{=}\PYG{l+m+mf}{0.25}\PYG{p}{)}
\PYG{n}{plt}\PYG{o}{.}\PYG{n}{annotate}\PYG{p}{(}\PYG{l+s+s1}{\PYGZsq{}}\PYG{l+s+s1}{WWII}\PYG{l+s+s1}{\PYGZsq{}}\PYG{p}{,} \PYG{p}{(}\PYG{l+s+s1}{\PYGZsq{}}\PYG{l+s+s1}{1941\PYGZhy{}12}\PYG{l+s+s1}{\PYGZsq{}}\PYG{p}{,} \PYG{l+m+mi}{90}\PYG{p}{)}\PYG{p}{)}
\PYG{n}{plt}\PYG{o}{.}\PYG{n}{axvspan}\PYG{p}{(}\PYG{l+s+s1}{\PYGZsq{}}\PYG{l+s+s1}{1950}\PYG{l+s+s1}{\PYGZsq{}}\PYG{p}{,} \PYG{l+s+s1}{\PYGZsq{}}\PYG{l+s+s1}{1953}\PYG{l+s+s1}{\PYGZsq{}}\PYG{p}{,} \PYG{n}{alpha}\PYG{o}{=}\PYG{l+m+mf}{0.25}\PYG{p}{)}
\PYG{n}{plt}\PYG{o}{.}\PYG{n}{annotate}\PYG{p}{(}\PYG{l+s+s1}{\PYGZsq{}}\PYG{l+s+s1}{Korean}\PYG{l+s+s1}{\PYGZsq{}}\PYG{p}{,} \PYG{p}{(}\PYG{l+s+s1}{\PYGZsq{}}\PYG{l+s+s1}{1950}\PYG{l+s+s1}{\PYGZsq{}}\PYG{p}{,} \PYG{l+m+mi}{80}\PYG{p}{)}\PYG{p}{)}
\PYG{n}{plt}\PYG{o}{.}\PYG{n}{axvspan}\PYG{p}{(}\PYG{l+s+s1}{\PYGZsq{}}\PYG{l+s+s1}{1959}\PYG{l+s+s1}{\PYGZsq{}}\PYG{p}{,} \PYG{l+s+s1}{\PYGZsq{}}\PYG{l+s+s1}{1975}\PYG{l+s+s1}{\PYGZsq{}}\PYG{p}{,} \PYG{n}{alpha}\PYG{o}{=}\PYG{l+m+mf}{0.25}\PYG{p}{)}
\PYG{n}{plt}\PYG{o}{.}\PYG{n}{annotate}\PYG{p}{(}\PYG{l+s+s1}{\PYGZsq{}}\PYG{l+s+s1}{Vietnam}\PYG{l+s+s1}{\PYGZsq{}}\PYG{p}{,} \PYG{p}{(}\PYG{l+s+s1}{\PYGZsq{}}\PYG{l+s+s1}{1959}\PYG{l+s+s1}{\PYGZsq{}}\PYG{p}{,} \PYG{l+m+mi}{90}\PYG{p}{)}\PYG{p}{)}
\PYG{n}{plt}\PYG{o}{.}\PYG{n}{axvspan}\PYG{p}{(}\PYG{l+s+s1}{\PYGZsq{}}\PYG{l+s+s1}{1990}\PYG{l+s+s1}{\PYGZsq{}}\PYG{p}{,} \PYG{l+s+s1}{\PYGZsq{}}\PYG{l+s+s1}{1991}\PYG{l+s+s1}{\PYGZsq{}}\PYG{p}{,} \PYG{n}{alpha}\PYG{o}{=}\PYG{l+m+mf}{0.25}\PYG{p}{)}
\PYG{n}{plt}\PYG{o}{.}\PYG{n}{annotate}\PYG{p}{(}\PYG{l+s+s1}{\PYGZsq{}}\PYG{l+s+s1}{Gulf I}\PYG{l+s+s1}{\PYGZsq{}}\PYG{p}{,} \PYG{p}{(}\PYG{l+s+s1}{\PYGZsq{}}\PYG{l+s+s1}{1990}\PYG{l+s+s1}{\PYGZsq{}}\PYG{p}{,} \PYG{l+m+mi}{80}\PYG{p}{)}\PYG{p}{)}
\PYG{n}{plt}\PYG{o}{.}\PYG{n}{axvspan}\PYG{p}{(}\PYG{l+s+s1}{\PYGZsq{}}\PYG{l+s+s1}{2001}\PYG{l+s+s1}{\PYGZsq{}}\PYG{p}{,} \PYG{l+s+s1}{\PYGZsq{}}\PYG{l+s+s1}{2021}\PYG{l+s+s1}{\PYGZsq{}}\PYG{p}{,} \PYG{n}{alpha}\PYG{o}{=}\PYG{l+m+mf}{0.25}\PYG{p}{)}
\PYG{n}{plt}\PYG{o}{.}\PYG{n}{annotate}\PYG{p}{(}\PYG{l+s+s1}{\PYGZsq{}}\PYG{l+s+s1}{Afghanistan}\PYG{l+s+s1}{\PYGZsq{}}\PYG{p}{,} \PYG{p}{(}\PYG{l+s+s1}{\PYGZsq{}}\PYG{l+s+s1}{2001}\PYG{l+s+s1}{\PYGZsq{}}\PYG{p}{,} \PYG{l+m+mi}{90}\PYG{p}{)}\PYG{p}{)}

\PYG{n}{plt}\PYG{o}{.}\PYG{n}{ylabel}\PYG{p}{(}\PYG{l+s+s1}{\PYGZsq{}}\PYG{l+s+s1}{Annualized Volatility of Daily Returns (}\PYG{l+s+s1}{\PYGZpc{}}\PYG{l+s+s1}{)}\PYG{l+s+s1}{\PYGZsq{}}\PYG{p}{)}
\PYG{n}{plt}\PYG{o}{.}\PYG{n}{title}\PYG{p}{(}\PYG{l+s+s1}{\PYGZsq{}}\PYG{l+s+s1}{Volatility of U.S. Market}\PYG{l+s+s1}{\PYGZsq{}}\PYG{p}{)}
\PYG{n}{plt}\PYG{o}{.}\PYG{n}{show}\PYG{p}{(}\PYG{p}{)}
\end{sphinxVerbatim}

\end{sphinxuseclass}\end{sphinxVerbatimInput}
\begin{sphinxVerbatimOutput}

\begin{sphinxuseclass}{cell_output}
\noindent\sphinxincludegraphics{{5cd98e18bb8a9624d0b83e25c2c67f24eb070943b5d8d7e8d624d2b17bec4673}.png}

\end{sphinxuseclass}\end{sphinxVerbatimOutput}

\end{sphinxuseclass}
\sphinxstepscope


\chapter{McKinney Chapter 11 \sphinxhyphen{} Time Series}
\label{\detokenize{mckinney_11_lecture:mckinney-chapter-11-time-series}}\label{\detokenize{mckinney_11_lecture::doc}}

\section{Introduction}
\label{\detokenize{mckinney_11_lecture:introduction}}
\sphinxAtStartPar
Chapter 11 of Wes McKinney’s \sphinxhref{https://wesmckinney.com/book/}{\sphinxstyleemphasis{Python for Data Analysis}} discusses time series and panel data, which is where pandas \sphinxstyleemphasis{\sphinxstylestrong{shines}}.
We will use these time series and panel tools every day for the rest of the course.

\sphinxAtStartPar
We will focus on:
\begin{enumerate}
\sphinxsetlistlabels{\arabic}{enumi}{enumii}{}{.}%
\item {} 
\sphinxAtStartPar
Slicing a data frame or series by date or date range

\item {} 
\sphinxAtStartPar
Using \sphinxcode{\sphinxupquote{.shift()}} to create leads and lags of variables

\item {} 
\sphinxAtStartPar
Using \sphinxcode{\sphinxupquote{.resample()}} to change the frequency of variables

\item {} 
\sphinxAtStartPar
Using \sphinxcode{\sphinxupquote{.rolling()}} to aggregate data over rolling windows

\end{enumerate}

\sphinxAtStartPar
\sphinxstyleemphasis{\sphinxstylestrong{Note:}}
Indented block quotes are from McKinney unless otherwise indicated.
The section numbers here differ from McKinney because we will only discuss some topics.

\begin{sphinxuseclass}{cell}\begin{sphinxVerbatimInput}

\begin{sphinxuseclass}{cell_input}
\begin{sphinxVerbatim}[commandchars=\\\{\}]
\PYG{k+kn}{import} \PYG{n+nn}{matplotlib}\PYG{n+nn}{.}\PYG{n+nn}{pyplot} \PYG{k}{as} \PYG{n+nn}{plt}
\PYG{k+kn}{import} \PYG{n+nn}{numpy} \PYG{k}{as} \PYG{n+nn}{np}
\PYG{k+kn}{import} \PYG{n+nn}{pandas} \PYG{k}{as} \PYG{n+nn}{pd}
\end{sphinxVerbatim}

\end{sphinxuseclass}\end{sphinxVerbatimInput}

\end{sphinxuseclass}
\begin{sphinxuseclass}{cell}\begin{sphinxVerbatimInput}

\begin{sphinxuseclass}{cell_input}
\begin{sphinxVerbatim}[commandchars=\\\{\}]
\PYG{o}{\PYGZpc{}}\PYG{k}{config} InlineBackend.figure\PYGZus{}format = \PYGZsq{}retina\PYGZsq{}
\PYG{o}{\PYGZpc{}}\PYG{k}{precision} 4
\PYG{n}{pd}\PYG{o}{.}\PYG{n}{options}\PYG{o}{.}\PYG{n}{display}\PYG{o}{.}\PYG{n}{float\PYGZus{}format} \PYG{o}{=} \PYG{l+s+s1}{\PYGZsq{}}\PYG{l+s+si}{\PYGZob{}:.4f\PYGZcb{}}\PYG{l+s+s1}{\PYGZsq{}}\PYG{o}{.}\PYG{n}{format}
\end{sphinxVerbatim}

\end{sphinxuseclass}\end{sphinxVerbatimInput}

\end{sphinxuseclass}
\begin{sphinxuseclass}{cell}\begin{sphinxVerbatimInput}

\begin{sphinxuseclass}{cell_input}
\begin{sphinxVerbatim}[commandchars=\\\{\}]
\PYG{k+kn}{import} \PYG{n+nn}{yfinance} \PYG{k}{as} \PYG{n+nn}{yf}
\PYG{k+kn}{import} \PYG{n+nn}{pandas\PYGZus{}datareader} \PYG{k}{as} \PYG{n+nn}{pdr}
\PYG{k+kn}{import} \PYG{n+nn}{requests\PYGZus{}cache}
\PYG{n}{session} \PYG{o}{=} \PYG{n}{requests\PYGZus{}cache}\PYG{o}{.}\PYG{n}{CachedSession}\PYG{p}{(}\PYG{p}{)}
\end{sphinxVerbatim}

\end{sphinxuseclass}\end{sphinxVerbatimInput}

\end{sphinxuseclass}
\sphinxAtStartPar
McKinney provides an excellent introduction to the concept of time series and panel data:
\begin{quote}

\sphinxAtStartPar
Time series data is an important form of structured data in many different fields, such
as finance, economics, ecology, neuroscience, and physics. Anything that is observed
or measured at many points in time forms a time series. Many time series are fixed
frequency, which is to say that data points occur at regular intervals according to some
rule, such as every 15 seconds, every 5 minutes, or once per month. Time series can
also be irregular without a fixed unit of time or offset between units. How you mark
and refer to time series data depends on the application, and you may have one of the
following:
\begin{itemize}
\item {} 
\sphinxAtStartPar
Timestamps, specific instants in time

\item {} 
\sphinxAtStartPar
Fixed periods, such as the month January 2007 or the full year 2010

\item {} 
\sphinxAtStartPar
Intervals of time, indicated by a start and end timestamp. Periods can be thought
of as special cases of intervals

\item {} 
\sphinxAtStartPar
Experiment or elapsed time; each timestamp is a measure of time relative to a
particular start time (e.g., the diameter of a cookie baking each second since
being placed in the oven)

\end{itemize}

\sphinxAtStartPar
In this chapter, I am mainly concerned with time series in the first three categories,
though many of the techniques can be applied to experimental time series where the
index may be an integer or floating\sphinxhyphen{}point number indicating elapsed time from the
start of the experiment. The simplest and most widely used kind of time series are
those indexed by timestamp.
323

\sphinxAtStartPar
pandas provides many built\sphinxhyphen{}in time series tools and data algorithms. You can effi‐
ciently work with very large time series and easily slice and dice, aggregate, and
resample irregular\sphinxhyphen{} and fixed\sphinxhyphen{}frequency time series. Some of these tools are especially
useful for financial and economics applications, but you could certainly use them to
analyze server log data, too.
\end{quote}


\section{Time Series Basics}
\label{\detokenize{mckinney_11_lecture:time-series-basics}}
\sphinxAtStartPar
Let us create a time series to play with.

\begin{sphinxuseclass}{cell}\begin{sphinxVerbatimInput}

\begin{sphinxuseclass}{cell_input}
\begin{sphinxVerbatim}[commandchars=\\\{\}]
\PYG{k+kn}{from} \PYG{n+nn}{datetime} \PYG{k+kn}{import} \PYG{n}{datetime}
\PYG{n}{dates} \PYG{o}{=} \PYG{p}{[}
    \PYG{n}{datetime}\PYG{p}{(}\PYG{l+m+mi}{2011}\PYG{p}{,} \PYG{l+m+mi}{1}\PYG{p}{,} \PYG{l+m+mi}{2}\PYG{p}{)}\PYG{p}{,} 
    \PYG{n}{datetime}\PYG{p}{(}\PYG{l+m+mi}{2011}\PYG{p}{,} \PYG{l+m+mi}{1}\PYG{p}{,} \PYG{l+m+mi}{5}\PYG{p}{)}\PYG{p}{,}
    \PYG{n}{datetime}\PYG{p}{(}\PYG{l+m+mi}{2011}\PYG{p}{,} \PYG{l+m+mi}{1}\PYG{p}{,} \PYG{l+m+mi}{7}\PYG{p}{)}\PYG{p}{,} 
    \PYG{n}{datetime}\PYG{p}{(}\PYG{l+m+mi}{2011}\PYG{p}{,} \PYG{l+m+mi}{1}\PYG{p}{,} \PYG{l+m+mi}{8}\PYG{p}{)}\PYG{p}{,}
    \PYG{n}{datetime}\PYG{p}{(}\PYG{l+m+mi}{2011}\PYG{p}{,} \PYG{l+m+mi}{1}\PYG{p}{,} \PYG{l+m+mi}{10}\PYG{p}{)}\PYG{p}{,} 
    \PYG{n}{datetime}\PYG{p}{(}\PYG{l+m+mi}{2011}\PYG{p}{,} \PYG{l+m+mi}{1}\PYG{p}{,} \PYG{l+m+mi}{12}\PYG{p}{)}
\PYG{p}{]}
\PYG{n}{np}\PYG{o}{.}\PYG{n}{random}\PYG{o}{.}\PYG{n}{seed}\PYG{p}{(}\PYG{l+m+mi}{42}\PYG{p}{)}
\PYG{n}{ts} \PYG{o}{=} \PYG{n}{pd}\PYG{o}{.}\PYG{n}{Series}\PYG{p}{(}\PYG{n}{np}\PYG{o}{.}\PYG{n}{random}\PYG{o}{.}\PYG{n}{randn}\PYG{p}{(}\PYG{l+m+mi}{6}\PYG{p}{)}\PYG{p}{,} \PYG{n}{index}\PYG{o}{=}\PYG{n}{dates}\PYG{p}{)}

\PYG{n}{ts}
\end{sphinxVerbatim}

\end{sphinxuseclass}\end{sphinxVerbatimInput}
\begin{sphinxVerbatimOutput}

\begin{sphinxuseclass}{cell_output}
\begin{sphinxVerbatim}[commandchars=\\\{\}]
2011\PYGZhy{}01\PYGZhy{}02    0.4967
2011\PYGZhy{}01\PYGZhy{}05   \PYGZhy{}0.1383
2011\PYGZhy{}01\PYGZhy{}07    0.6477
2011\PYGZhy{}01\PYGZhy{}08    1.5230
2011\PYGZhy{}01\PYGZhy{}10   \PYGZhy{}0.2342
2011\PYGZhy{}01\PYGZhy{}12   \PYGZhy{}0.2341
dtype: float64
\end{sphinxVerbatim}

\end{sphinxuseclass}\end{sphinxVerbatimOutput}

\end{sphinxuseclass}
\sphinxAtStartPar
Note that pandas converts the \sphinxcode{\sphinxupquote{datetime}} objects to a pandas \sphinxcode{\sphinxupquote{DatetimeIndex}} object and a single index value is a \sphinxcode{\sphinxupquote{Timestamp}} object.

\begin{sphinxuseclass}{cell}\begin{sphinxVerbatimInput}

\begin{sphinxuseclass}{cell_input}
\begin{sphinxVerbatim}[commandchars=\\\{\}]
\PYG{n}{ts}\PYG{o}{.}\PYG{n}{index}
\end{sphinxVerbatim}

\end{sphinxuseclass}\end{sphinxVerbatimInput}
\begin{sphinxVerbatimOutput}

\begin{sphinxuseclass}{cell_output}
\begin{sphinxVerbatim}[commandchars=\\\{\}]
DatetimeIndex([\PYGZsq{}2011\PYGZhy{}01\PYGZhy{}02\PYGZsq{}, \PYGZsq{}2011\PYGZhy{}01\PYGZhy{}05\PYGZsq{}, \PYGZsq{}2011\PYGZhy{}01\PYGZhy{}07\PYGZsq{}, \PYGZsq{}2011\PYGZhy{}01\PYGZhy{}08\PYGZsq{},
               \PYGZsq{}2011\PYGZhy{}01\PYGZhy{}10\PYGZsq{}, \PYGZsq{}2011\PYGZhy{}01\PYGZhy{}12\PYGZsq{}],
              dtype=\PYGZsq{}datetime64[ns]\PYGZsq{}, freq=None)
\end{sphinxVerbatim}

\end{sphinxuseclass}\end{sphinxVerbatimOutput}

\end{sphinxuseclass}
\begin{sphinxuseclass}{cell}\begin{sphinxVerbatimInput}

\begin{sphinxuseclass}{cell_input}
\begin{sphinxVerbatim}[commandchars=\\\{\}]
\PYG{n}{ts}\PYG{o}{.}\PYG{n}{index}\PYG{p}{[}\PYG{l+m+mi}{0}\PYG{p}{]}
\end{sphinxVerbatim}

\end{sphinxuseclass}\end{sphinxVerbatimInput}
\begin{sphinxVerbatimOutput}

\begin{sphinxuseclass}{cell_output}
\begin{sphinxVerbatim}[commandchars=\\\{\}]
Timestamp(\PYGZsq{}2011\PYGZhy{}01\PYGZhy{}02 00:00:00\PYGZsq{})
\end{sphinxVerbatim}

\end{sphinxuseclass}\end{sphinxVerbatimOutput}

\end{sphinxuseclass}
\sphinxAtStartPar
Recall that arithmetic operations between pandas objects automatically align on indexes.

\begin{sphinxuseclass}{cell}\begin{sphinxVerbatimInput}

\begin{sphinxuseclass}{cell_input}
\begin{sphinxVerbatim}[commandchars=\\\{\}]
\PYG{n}{ts}\PYG{p}{[}\PYG{p}{:}\PYG{p}{:}\PYG{l+m+mi}{2}\PYG{p}{]}
\end{sphinxVerbatim}

\end{sphinxuseclass}\end{sphinxVerbatimInput}
\begin{sphinxVerbatimOutput}

\begin{sphinxuseclass}{cell_output}
\begin{sphinxVerbatim}[commandchars=\\\{\}]
2011\PYGZhy{}01\PYGZhy{}02    0.4967
2011\PYGZhy{}01\PYGZhy{}07    0.6477
2011\PYGZhy{}01\PYGZhy{}10   \PYGZhy{}0.2342
dtype: float64
\end{sphinxVerbatim}

\end{sphinxuseclass}\end{sphinxVerbatimOutput}

\end{sphinxuseclass}
\begin{sphinxuseclass}{cell}\begin{sphinxVerbatimInput}

\begin{sphinxuseclass}{cell_input}
\begin{sphinxVerbatim}[commandchars=\\\{\}]
\PYG{n}{ts} \PYG{o}{+} \PYG{n}{ts}\PYG{p}{[}\PYG{p}{:}\PYG{p}{:}\PYG{l+m+mi}{2}\PYG{p}{]}
\end{sphinxVerbatim}

\end{sphinxuseclass}\end{sphinxVerbatimInput}
\begin{sphinxVerbatimOutput}

\begin{sphinxuseclass}{cell_output}
\begin{sphinxVerbatim}[commandchars=\\\{\}]
2011\PYGZhy{}01\PYGZhy{}02    0.9934
2011\PYGZhy{}01\PYGZhy{}05       NaN
2011\PYGZhy{}01\PYGZhy{}07    1.2954
2011\PYGZhy{}01\PYGZhy{}08       NaN
2011\PYGZhy{}01\PYGZhy{}10   \PYGZhy{}0.4683
2011\PYGZhy{}01\PYGZhy{}12       NaN
dtype: float64
\end{sphinxVerbatim}

\end{sphinxuseclass}\end{sphinxVerbatimOutput}

\end{sphinxuseclass}

\subsection{Indexing, Selection, Subsetting}
\label{\detokenize{mckinney_11_lecture:indexing-selection-subsetting}}
\sphinxAtStartPar
We can use date and time labels to select data.

\begin{sphinxuseclass}{cell}\begin{sphinxVerbatimInput}

\begin{sphinxuseclass}{cell_input}
\begin{sphinxVerbatim}[commandchars=\\\{\}]
\PYG{n}{stamp} \PYG{o}{=} \PYG{n}{ts}\PYG{o}{.}\PYG{n}{index}\PYG{p}{[}\PYG{l+m+mi}{2}\PYG{p}{]}

\PYG{n}{stamp}
\end{sphinxVerbatim}

\end{sphinxuseclass}\end{sphinxVerbatimInput}
\begin{sphinxVerbatimOutput}

\begin{sphinxuseclass}{cell_output}
\begin{sphinxVerbatim}[commandchars=\\\{\}]
Timestamp(\PYGZsq{}2011\PYGZhy{}01\PYGZhy{}07 00:00:00\PYGZsq{})
\end{sphinxVerbatim}

\end{sphinxuseclass}\end{sphinxVerbatimOutput}

\end{sphinxuseclass}
\begin{sphinxuseclass}{cell}\begin{sphinxVerbatimInput}

\begin{sphinxuseclass}{cell_input}
\begin{sphinxVerbatim}[commandchars=\\\{\}]
\PYG{n}{ts}\PYG{p}{[}\PYG{n}{stamp}\PYG{p}{]}
\end{sphinxVerbatim}

\end{sphinxuseclass}\end{sphinxVerbatimInput}
\begin{sphinxVerbatimOutput}

\begin{sphinxuseclass}{cell_output}
\begin{sphinxVerbatim}[commandchars=\\\{\}]
0.6476885381006925
\end{sphinxVerbatim}

\end{sphinxuseclass}\end{sphinxVerbatimOutput}

\end{sphinxuseclass}
\sphinxAtStartPar
pandas uses unambiguous date strings to select data.

\begin{sphinxuseclass}{cell}\begin{sphinxVerbatimInput}

\begin{sphinxuseclass}{cell_input}
\begin{sphinxVerbatim}[commandchars=\\\{\}]
\PYG{n}{ts}\PYG{p}{[}\PYG{l+s+s1}{\PYGZsq{}}\PYG{l+s+s1}{1/10/2011}\PYG{l+s+s1}{\PYGZsq{}}\PYG{p}{]} \PYG{c+c1}{\PYGZsh{} M/D/YYYY}
\end{sphinxVerbatim}

\end{sphinxuseclass}\end{sphinxVerbatimInput}
\begin{sphinxVerbatimOutput}

\begin{sphinxuseclass}{cell_output}
\begin{sphinxVerbatim}[commandchars=\\\{\}]
\PYGZhy{}0.23415337472333597
\end{sphinxVerbatim}

\end{sphinxuseclass}\end{sphinxVerbatimOutput}

\end{sphinxuseclass}
\begin{sphinxuseclass}{cell}\begin{sphinxVerbatimInput}

\begin{sphinxuseclass}{cell_input}
\begin{sphinxVerbatim}[commandchars=\\\{\}]
\PYG{n}{ts}\PYG{p}{[}\PYG{l+s+s1}{\PYGZsq{}}\PYG{l+s+s1}{20110110}\PYG{l+s+s1}{\PYGZsq{}}\PYG{p}{]} \PYG{c+c1}{\PYGZsh{} YYYYMMDD}
\end{sphinxVerbatim}

\end{sphinxuseclass}\end{sphinxVerbatimInput}
\begin{sphinxVerbatimOutput}

\begin{sphinxuseclass}{cell_output}
\begin{sphinxVerbatim}[commandchars=\\\{\}]
\PYGZhy{}0.23415337472333597
\end{sphinxVerbatim}

\end{sphinxuseclass}\end{sphinxVerbatimOutput}

\end{sphinxuseclass}
\begin{sphinxuseclass}{cell}\begin{sphinxVerbatimInput}

\begin{sphinxuseclass}{cell_input}
\begin{sphinxVerbatim}[commandchars=\\\{\}]
\PYG{n}{ts}\PYG{p}{[}\PYG{l+s+s1}{\PYGZsq{}}\PYG{l+s+s1}{2011\PYGZhy{}01\PYGZhy{}10}\PYG{l+s+s1}{\PYGZsq{}}\PYG{p}{]} \PYG{c+c1}{\PYGZsh{} YYYY\PYGZhy{}MM\PYGZhy{}DD}
\end{sphinxVerbatim}

\end{sphinxuseclass}\end{sphinxVerbatimInput}
\begin{sphinxVerbatimOutput}

\begin{sphinxuseclass}{cell_output}
\begin{sphinxVerbatim}[commandchars=\\\{\}]
\PYGZhy{}0.23415337472333597
\end{sphinxVerbatim}

\end{sphinxuseclass}\end{sphinxVerbatimOutput}

\end{sphinxuseclass}
\begin{sphinxuseclass}{cell}\begin{sphinxVerbatimInput}

\begin{sphinxuseclass}{cell_input}
\begin{sphinxVerbatim}[commandchars=\\\{\}]
\PYG{n}{ts}\PYG{p}{[}\PYG{l+s+s1}{\PYGZsq{}}\PYG{l+s+s1}{10\PYGZhy{}Jan\PYGZhy{}2011}\PYG{l+s+s1}{\PYGZsq{}}\PYG{p}{]} \PYG{c+c1}{\PYGZsh{} D\PYGZhy{}Mon\PYGZhy{}YYYY}
\end{sphinxVerbatim}

\end{sphinxuseclass}\end{sphinxVerbatimInput}
\begin{sphinxVerbatimOutput}

\begin{sphinxuseclass}{cell_output}
\begin{sphinxVerbatim}[commandchars=\\\{\}]
\PYGZhy{}0.23415337472333597
\end{sphinxVerbatim}

\end{sphinxuseclass}\end{sphinxVerbatimOutput}

\end{sphinxuseclass}
\begin{sphinxuseclass}{cell}\begin{sphinxVerbatimInput}

\begin{sphinxuseclass}{cell_input}
\begin{sphinxVerbatim}[commandchars=\\\{\}]
\PYG{n}{ts}\PYG{p}{[}\PYG{l+s+s1}{\PYGZsq{}}\PYG{l+s+s1}{Jan\PYGZhy{}10\PYGZhy{}2011}\PYG{l+s+s1}{\PYGZsq{}}\PYG{p}{]} \PYG{c+c1}{\PYGZsh{} Mon\PYGZhy{}D\PYGZhy{}YYYY}
\end{sphinxVerbatim}

\end{sphinxuseclass}\end{sphinxVerbatimInput}
\begin{sphinxVerbatimOutput}

\begin{sphinxuseclass}{cell_output}
\begin{sphinxVerbatim}[commandchars=\\\{\}]
\PYGZhy{}0.23415337472333597
\end{sphinxVerbatim}

\end{sphinxuseclass}\end{sphinxVerbatimOutput}

\end{sphinxuseclass}
\sphinxAtStartPar
But do not use U.K.\sphinxhyphen{}style dates.

\begin{sphinxuseclass}{cell}\begin{sphinxVerbatimInput}

\begin{sphinxuseclass}{cell_input}
\begin{sphinxVerbatim}[commandchars=\\\{\}]
\PYG{c+c1}{\PYGZsh{} ts[\PYGZsq{}10/1/2011\PYGZsq{}] \PYGZsh{} D/M/YYYY}
\end{sphinxVerbatim}

\end{sphinxuseclass}\end{sphinxVerbatimInput}

\end{sphinxuseclass}
\sphinxAtStartPar
Here is a longer time series for longer slices.

\begin{sphinxuseclass}{cell}\begin{sphinxVerbatimInput}

\begin{sphinxuseclass}{cell_input}
\begin{sphinxVerbatim}[commandchars=\\\{\}]
\PYG{n}{np}\PYG{o}{.}\PYG{n}{random}\PYG{o}{.}\PYG{n}{seed}\PYG{p}{(}\PYG{l+m+mi}{42}\PYG{p}{)}
\PYG{n}{longer\PYGZus{}ts} \PYG{o}{=} \PYG{n}{pd}\PYG{o}{.}\PYG{n}{Series}\PYG{p}{(}\PYG{n}{np}\PYG{o}{.}\PYG{n}{random}\PYG{o}{.}\PYG{n}{randn}\PYG{p}{(}\PYG{l+m+mi}{1000}\PYG{p}{)}\PYG{p}{,} \PYG{n}{index}\PYG{o}{=}\PYG{n}{pd}\PYG{o}{.}\PYG{n}{date\PYGZus{}range}\PYG{p}{(}\PYG{l+s+s1}{\PYGZsq{}}\PYG{l+s+s1}{1/1/2000}\PYG{l+s+s1}{\PYGZsq{}}\PYG{p}{,} \PYG{n}{periods}\PYG{o}{=}\PYG{l+m+mi}{1000}\PYG{p}{)}\PYG{p}{)}

\PYG{n}{longer\PYGZus{}ts}
\end{sphinxVerbatim}

\end{sphinxuseclass}\end{sphinxVerbatimInput}
\begin{sphinxVerbatimOutput}

\begin{sphinxuseclass}{cell_output}
\begin{sphinxVerbatim}[commandchars=\\\{\}]
2000\PYGZhy{}01\PYGZhy{}01    0.4967
2000\PYGZhy{}01\PYGZhy{}02   \PYGZhy{}0.1383
2000\PYGZhy{}01\PYGZhy{}03    0.6477
2000\PYGZhy{}01\PYGZhy{}04    1.5230
2000\PYGZhy{}01\PYGZhy{}05   \PYGZhy{}0.2342
               ...  
2002\PYGZhy{}09\PYGZhy{}22   \PYGZhy{}0.2811
2002\PYGZhy{}09\PYGZhy{}23    1.7977
2002\PYGZhy{}09\PYGZhy{}24    0.6408
2002\PYGZhy{}09\PYGZhy{}25   \PYGZhy{}0.5712
2002\PYGZhy{}09\PYGZhy{}26    0.5726
Freq: D, Length: 1000, dtype: float64
\end{sphinxVerbatim}

\end{sphinxuseclass}\end{sphinxVerbatimOutput}

\end{sphinxuseclass}
\sphinxAtStartPar
We can pass a year\sphinxhyphen{}month to slice all of the observations in May of 2001.

\begin{sphinxuseclass}{cell}\begin{sphinxVerbatimInput}

\begin{sphinxuseclass}{cell_input}
\begin{sphinxVerbatim}[commandchars=\\\{\}]
\PYG{n}{longer\PYGZus{}ts}\PYG{p}{[}\PYG{l+s+s1}{\PYGZsq{}}\PYG{l+s+s1}{2001\PYGZhy{}05}\PYG{l+s+s1}{\PYGZsq{}}\PYG{p}{]}
\end{sphinxVerbatim}

\end{sphinxuseclass}\end{sphinxVerbatimInput}
\begin{sphinxVerbatimOutput}

\begin{sphinxuseclass}{cell_output}
\begin{sphinxVerbatim}[commandchars=\\\{\}]
2001\PYGZhy{}05\PYGZhy{}01   \PYGZhy{}0.6466
2001\PYGZhy{}05\PYGZhy{}02   \PYGZhy{}1.0815
2001\PYGZhy{}05\PYGZhy{}03    1.6871
2001\PYGZhy{}05\PYGZhy{}04    0.8816
2001\PYGZhy{}05\PYGZhy{}05   \PYGZhy{}0.0080
2001\PYGZhy{}05\PYGZhy{}06    1.4799
2001\PYGZhy{}05\PYGZhy{}07    0.0774
2001\PYGZhy{}05\PYGZhy{}08   \PYGZhy{}0.8613
2001\PYGZhy{}05\PYGZhy{}09    1.5231
2001\PYGZhy{}05\PYGZhy{}10    0.5389
2001\PYGZhy{}05\PYGZhy{}11   \PYGZhy{}1.0372
2001\PYGZhy{}05\PYGZhy{}12   \PYGZhy{}0.1903
2001\PYGZhy{}05\PYGZhy{}13   \PYGZhy{}0.8756
2001\PYGZhy{}05\PYGZhy{}14   \PYGZhy{}1.3828
2001\PYGZhy{}05\PYGZhy{}15    0.9262
2001\PYGZhy{}05\PYGZhy{}16    1.9094
2001\PYGZhy{}05\PYGZhy{}17   \PYGZhy{}1.3986
2001\PYGZhy{}05\PYGZhy{}18    0.5630
2001\PYGZhy{}05\PYGZhy{}19   \PYGZhy{}0.6506
2001\PYGZhy{}05\PYGZhy{}20   \PYGZhy{}0.4871
2001\PYGZhy{}05\PYGZhy{}21   \PYGZhy{}0.5924
2001\PYGZhy{}05\PYGZhy{}22   \PYGZhy{}0.8640
2001\PYGZhy{}05\PYGZhy{}23    0.0485
2001\PYGZhy{}05\PYGZhy{}24   \PYGZhy{}0.8310
2001\PYGZhy{}05\PYGZhy{}25    0.2705
2001\PYGZhy{}05\PYGZhy{}26   \PYGZhy{}0.0502
2001\PYGZhy{}05\PYGZhy{}27   \PYGZhy{}0.2389
2001\PYGZhy{}05\PYGZhy{}28   \PYGZhy{}0.9076
2001\PYGZhy{}05\PYGZhy{}29   \PYGZhy{}0.5768
2001\PYGZhy{}05\PYGZhy{}30    0.7554
2001\PYGZhy{}05\PYGZhy{}31    0.5009
Freq: D, dtype: float64
\end{sphinxVerbatim}

\end{sphinxuseclass}\end{sphinxVerbatimOutput}

\end{sphinxuseclass}
\sphinxAtStartPar
We can also pass a year to slice all observations in 2001.

\begin{sphinxuseclass}{cell}\begin{sphinxVerbatimInput}

\begin{sphinxuseclass}{cell_input}
\begin{sphinxVerbatim}[commandchars=\\\{\}]
\PYG{n}{longer\PYGZus{}ts}\PYG{p}{[}\PYG{l+s+s1}{\PYGZsq{}}\PYG{l+s+s1}{2001}\PYG{l+s+s1}{\PYGZsq{}}\PYG{p}{]}
\end{sphinxVerbatim}

\end{sphinxuseclass}\end{sphinxVerbatimInput}
\begin{sphinxVerbatimOutput}

\begin{sphinxuseclass}{cell_output}
\begin{sphinxVerbatim}[commandchars=\\\{\}]
2001\PYGZhy{}01\PYGZhy{}01    0.2241
2001\PYGZhy{}01\PYGZhy{}02    0.0126
2001\PYGZhy{}01\PYGZhy{}03    0.0977
2001\PYGZhy{}01\PYGZhy{}04   \PYGZhy{}0.7730
2001\PYGZhy{}01\PYGZhy{}05    0.0245
               ...  
2001\PYGZhy{}12\PYGZhy{}27    0.0184
2001\PYGZhy{}12\PYGZhy{}28    0.3476
2001\PYGZhy{}12\PYGZhy{}29   \PYGZhy{}0.5398
2001\PYGZhy{}12\PYGZhy{}30   \PYGZhy{}0.7783
2001\PYGZhy{}12\PYGZhy{}31    0.1958
Freq: D, Length: 365, dtype: float64
\end{sphinxVerbatim}

\end{sphinxuseclass}\end{sphinxVerbatimOutput}

\end{sphinxuseclass}
\sphinxAtStartPar
If we sort our data chronologically, we can also slice with a range of date strings.

\begin{sphinxuseclass}{cell}\begin{sphinxVerbatimInput}

\begin{sphinxuseclass}{cell_input}
\begin{sphinxVerbatim}[commandchars=\\\{\}]
\PYG{n}{ts}\PYG{p}{[}\PYG{l+s+s1}{\PYGZsq{}}\PYG{l+s+s1}{1/6/2011}\PYG{l+s+s1}{\PYGZsq{}}\PYG{p}{:}\PYG{l+s+s1}{\PYGZsq{}}\PYG{l+s+s1}{1/10/2011}\PYG{l+s+s1}{\PYGZsq{}}\PYG{p}{]}
\end{sphinxVerbatim}

\end{sphinxuseclass}\end{sphinxVerbatimInput}
\begin{sphinxVerbatimOutput}

\begin{sphinxuseclass}{cell_output}
\begin{sphinxVerbatim}[commandchars=\\\{\}]
2011\PYGZhy{}01\PYGZhy{}07    0.6477
2011\PYGZhy{}01\PYGZhy{}08    1.5230
2011\PYGZhy{}01\PYGZhy{}10   \PYGZhy{}0.2342
dtype: float64
\end{sphinxVerbatim}

\end{sphinxuseclass}\end{sphinxVerbatimOutput}

\end{sphinxuseclass}
\sphinxAtStartPar
To use date slices, our data should be sorted by the date index, as above.
The following code works as though our data were sorted, but raises a warning that it will not work in future versions.

\begin{sphinxuseclass}{cell}\begin{sphinxVerbatimInput}

\begin{sphinxuseclass}{cell_input}
\begin{sphinxVerbatim}[commandchars=\\\{\}]
\PYG{n}{ts2} \PYG{o}{=} \PYG{n}{ts}\PYG{o}{.}\PYG{n}{sort\PYGZus{}values}\PYG{p}{(}\PYG{p}{)}

\PYG{n}{ts2}
\end{sphinxVerbatim}

\end{sphinxuseclass}\end{sphinxVerbatimInput}
\begin{sphinxVerbatimOutput}

\begin{sphinxuseclass}{cell_output}
\begin{sphinxVerbatim}[commandchars=\\\{\}]
2011\PYGZhy{}01\PYGZhy{}10   \PYGZhy{}0.2342
2011\PYGZhy{}01\PYGZhy{}12   \PYGZhy{}0.2341
2011\PYGZhy{}01\PYGZhy{}05   \PYGZhy{}0.1383
2011\PYGZhy{}01\PYGZhy{}02    0.4967
2011\PYGZhy{}01\PYGZhy{}07    0.6477
2011\PYGZhy{}01\PYGZhy{}08    1.5230
dtype: float64
\end{sphinxVerbatim}

\end{sphinxuseclass}\end{sphinxVerbatimOutput}

\end{sphinxuseclass}
\sphinxAtStartPar
The following behavior is “deprecated”, meaning it will eventually go away and we should not rely up on it.

\begin{sphinxuseclass}{cell}\begin{sphinxVerbatimInput}

\begin{sphinxuseclass}{cell_input}
\begin{sphinxVerbatim}[commandchars=\\\{\}]
\PYG{n}{ts2}\PYG{p}{[}\PYG{l+s+s1}{\PYGZsq{}}\PYG{l+s+s1}{1/6/2011}\PYG{l+s+s1}{\PYGZsq{}}\PYG{p}{:}\PYG{l+s+s1}{\PYGZsq{}}\PYG{l+s+s1}{1/11/2011}\PYG{l+s+s1}{\PYGZsq{}}\PYG{p}{]}
\end{sphinxVerbatim}

\end{sphinxuseclass}\end{sphinxVerbatimInput}
\begin{sphinxVerbatimOutput}

\begin{sphinxuseclass}{cell_output}
\begin{sphinxVerbatim}[commandchars=\\\{\}]
/tmp/ipykernel\PYGZus{}259324/1099700163.py:1: FutureWarning: Value based partial slicing on non\PYGZhy{}monotonic DatetimeIndexes with non\PYGZhy{}existing keys is deprecated and will raise a KeyError in a future Version.
  ts2[\PYGZsq{}1/6/2011\PYGZsq{}:\PYGZsq{}1/11/2011\PYGZsq{}]
\end{sphinxVerbatim}

\begin{sphinxVerbatim}[commandchars=\\\{\}]
2011\PYGZhy{}01\PYGZhy{}10   \PYGZhy{}0.2342
2011\PYGZhy{}01\PYGZhy{}07    0.6477
2011\PYGZhy{}01\PYGZhy{}08    1.5230
dtype: float64
\end{sphinxVerbatim}

\end{sphinxuseclass}\end{sphinxVerbatimOutput}

\end{sphinxuseclass}
\begin{sphinxuseclass}{cell}\begin{sphinxVerbatimInput}

\begin{sphinxuseclass}{cell_input}
\begin{sphinxVerbatim}[commandchars=\\\{\}]
\PYG{n}{ts2}\PYG{o}{.}\PYG{n}{sort\PYGZus{}index}\PYG{p}{(}\PYG{p}{)}\PYG{p}{[}\PYG{l+s+s1}{\PYGZsq{}}\PYG{l+s+s1}{1/6/2011}\PYG{l+s+s1}{\PYGZsq{}}\PYG{p}{:}\PYG{l+s+s1}{\PYGZsq{}}\PYG{l+s+s1}{1/11/2011}\PYG{l+s+s1}{\PYGZsq{}}\PYG{p}{]}
\end{sphinxVerbatim}

\end{sphinxuseclass}\end{sphinxVerbatimInput}
\begin{sphinxVerbatimOutput}

\begin{sphinxuseclass}{cell_output}
\begin{sphinxVerbatim}[commandchars=\\\{\}]
2011\PYGZhy{}01\PYGZhy{}07    0.6477
2011\PYGZhy{}01\PYGZhy{}08    1.5230
2011\PYGZhy{}01\PYGZhy{}10   \PYGZhy{}0.2342
dtype: float64
\end{sphinxVerbatim}

\end{sphinxuseclass}\end{sphinxVerbatimOutput}

\end{sphinxuseclass}
\sphinxAtStartPar
\sphinxstyleemphasis{\sphinxstylestrong{To be clear, a range of date strings is inclusive on both ends.}}

\begin{sphinxuseclass}{cell}\begin{sphinxVerbatimInput}

\begin{sphinxuseclass}{cell_input}
\begin{sphinxVerbatim}[commandchars=\\\{\}]
\PYG{n}{longer\PYGZus{}ts}\PYG{p}{[}\PYG{l+s+s1}{\PYGZsq{}}\PYG{l+s+s1}{1/6/2001}\PYG{l+s+s1}{\PYGZsq{}}\PYG{p}{:}\PYG{l+s+s1}{\PYGZsq{}}\PYG{l+s+s1}{1/11/2001}\PYG{l+s+s1}{\PYGZsq{}}\PYG{p}{]}
\end{sphinxVerbatim}

\end{sphinxuseclass}\end{sphinxVerbatimInput}
\begin{sphinxVerbatimOutput}

\begin{sphinxuseclass}{cell_output}
\begin{sphinxVerbatim}[commandchars=\\\{\}]
2001\PYGZhy{}01\PYGZhy{}06    0.4980
2001\PYGZhy{}01\PYGZhy{}07    1.4511
2001\PYGZhy{}01\PYGZhy{}08    0.9593
2001\PYGZhy{}01\PYGZhy{}09    2.1532
2001\PYGZhy{}01\PYGZhy{}10   \PYGZhy{}0.7673
2001\PYGZhy{}01\PYGZhy{}11    0.8723
Freq: D, dtype: float64
\end{sphinxVerbatim}

\end{sphinxuseclass}\end{sphinxVerbatimOutput}

\end{sphinxuseclass}
\sphinxAtStartPar
\sphinxstyleemphasis{\sphinxstylestrong{Recall, if we modify a slice, we modify the original series or dataframe.}}
\begin{quote}

\sphinxAtStartPar
Remember that slicing in this manner produces views on the source time series like slicing NumPy arrays. This means that no data is copied and modifications on the slice will be reflected in the original data.
\end{quote}


\subsection{Time Series with Duplicate Indices}
\label{\detokenize{mckinney_11_lecture:time-series-with-duplicate-indices}}
\sphinxAtStartPar
Most data in this course will be well\sphinxhyphen{}formed with one observation per datetime for series or one observation per individual per datetime for dataframes.
However, you may later receive poorly\sphinxhyphen{}formed data with duplicate observations.
The toy data in series \sphinxcode{\sphinxupquote{dup\_ts}} has three observations on February 2nd.

\begin{sphinxuseclass}{cell}\begin{sphinxVerbatimInput}

\begin{sphinxuseclass}{cell_input}
\begin{sphinxVerbatim}[commandchars=\\\{\}]
\PYG{n}{dates} \PYG{o}{=} \PYG{n}{pd}\PYG{o}{.}\PYG{n}{DatetimeIndex}\PYG{p}{(}\PYG{p}{[}\PYG{l+s+s1}{\PYGZsq{}}\PYG{l+s+s1}{1/1/2000}\PYG{l+s+s1}{\PYGZsq{}}\PYG{p}{,} \PYG{l+s+s1}{\PYGZsq{}}\PYG{l+s+s1}{1/2/2000}\PYG{l+s+s1}{\PYGZsq{}}\PYG{p}{,} \PYG{l+s+s1}{\PYGZsq{}}\PYG{l+s+s1}{1/2/2000}\PYG{l+s+s1}{\PYGZsq{}}\PYG{p}{,} \PYG{l+s+s1}{\PYGZsq{}}\PYG{l+s+s1}{1/2/2000}\PYG{l+s+s1}{\PYGZsq{}}\PYG{p}{,} \PYG{l+s+s1}{\PYGZsq{}}\PYG{l+s+s1}{1/3/2000}\PYG{l+s+s1}{\PYGZsq{}}\PYG{p}{]}\PYG{p}{)}
\PYG{n}{dup\PYGZus{}ts} \PYG{o}{=} \PYG{n}{pd}\PYG{o}{.}\PYG{n}{Series}\PYG{p}{(}\PYG{n}{np}\PYG{o}{.}\PYG{n}{arange}\PYG{p}{(}\PYG{l+m+mi}{5}\PYG{p}{)}\PYG{p}{,} \PYG{n}{index}\PYG{o}{=}\PYG{n}{dates}\PYG{p}{)}

\PYG{n}{dup\PYGZus{}ts}
\end{sphinxVerbatim}

\end{sphinxuseclass}\end{sphinxVerbatimInput}
\begin{sphinxVerbatimOutput}

\begin{sphinxuseclass}{cell_output}
\begin{sphinxVerbatim}[commandchars=\\\{\}]
2000\PYGZhy{}01\PYGZhy{}01    0
2000\PYGZhy{}01\PYGZhy{}02    1
2000\PYGZhy{}01\PYGZhy{}02    2
2000\PYGZhy{}01\PYGZhy{}02    3
2000\PYGZhy{}01\PYGZhy{}03    4
dtype: int64
\end{sphinxVerbatim}

\end{sphinxuseclass}\end{sphinxVerbatimOutput}

\end{sphinxuseclass}
\sphinxAtStartPar
The \sphinxcode{\sphinxupquote{.is\_unique}} property tells us if an index is unique.

\begin{sphinxuseclass}{cell}\begin{sphinxVerbatimInput}

\begin{sphinxuseclass}{cell_input}
\begin{sphinxVerbatim}[commandchars=\\\{\}]
\PYG{n}{dup\PYGZus{}ts}\PYG{o}{.}\PYG{n}{index}\PYG{o}{.}\PYG{n}{is\PYGZus{}unique}
\end{sphinxVerbatim}

\end{sphinxuseclass}\end{sphinxVerbatimInput}
\begin{sphinxVerbatimOutput}

\begin{sphinxuseclass}{cell_output}
\begin{sphinxVerbatim}[commandchars=\\\{\}]
False
\end{sphinxVerbatim}

\end{sphinxuseclass}\end{sphinxVerbatimOutput}

\end{sphinxuseclass}
\begin{sphinxuseclass}{cell}\begin{sphinxVerbatimInput}

\begin{sphinxuseclass}{cell_input}
\begin{sphinxVerbatim}[commandchars=\\\{\}]
\PYG{n}{dup\PYGZus{}ts}\PYG{p}{[}\PYG{l+s+s1}{\PYGZsq{}}\PYG{l+s+s1}{1/3/2000}\PYG{l+s+s1}{\PYGZsq{}}\PYG{p}{]}  \PYG{c+c1}{\PYGZsh{} not duplicated}
\end{sphinxVerbatim}

\end{sphinxuseclass}\end{sphinxVerbatimInput}
\begin{sphinxVerbatimOutput}

\begin{sphinxuseclass}{cell_output}
\begin{sphinxVerbatim}[commandchars=\\\{\}]
4
\end{sphinxVerbatim}

\end{sphinxuseclass}\end{sphinxVerbatimOutput}

\end{sphinxuseclass}
\begin{sphinxuseclass}{cell}\begin{sphinxVerbatimInput}

\begin{sphinxuseclass}{cell_input}
\begin{sphinxVerbatim}[commandchars=\\\{\}]
\PYG{n}{dup\PYGZus{}ts}\PYG{p}{[}\PYG{l+s+s1}{\PYGZsq{}}\PYG{l+s+s1}{1/2/2000}\PYG{l+s+s1}{\PYGZsq{}}\PYG{p}{]}  \PYG{c+c1}{\PYGZsh{} duplicated}
\end{sphinxVerbatim}

\end{sphinxuseclass}\end{sphinxVerbatimInput}
\begin{sphinxVerbatimOutput}

\begin{sphinxuseclass}{cell_output}
\begin{sphinxVerbatim}[commandchars=\\\{\}]
2000\PYGZhy{}01\PYGZhy{}02    1
2000\PYGZhy{}01\PYGZhy{}02    2
2000\PYGZhy{}01\PYGZhy{}02    3
dtype: int64
\end{sphinxVerbatim}

\end{sphinxuseclass}\end{sphinxVerbatimOutput}

\end{sphinxuseclass}
\sphinxAtStartPar
The solution to duplicate data depends on the context.
For example, we may want the mean of all observations on a given date.
The \sphinxcode{\sphinxupquote{.groupby()}}  method can help us here.

\begin{sphinxuseclass}{cell}\begin{sphinxVerbatimInput}

\begin{sphinxuseclass}{cell_input}
\begin{sphinxVerbatim}[commandchars=\\\{\}]
\PYG{n}{grouped} \PYG{o}{=} \PYG{n}{dup\PYGZus{}ts}\PYG{o}{.}\PYG{n}{groupby}\PYG{p}{(}\PYG{n}{level}\PYG{o}{=}\PYG{l+m+mi}{0}\PYG{p}{)}
\end{sphinxVerbatim}

\end{sphinxuseclass}\end{sphinxVerbatimInput}

\end{sphinxuseclass}
\begin{sphinxuseclass}{cell}\begin{sphinxVerbatimInput}

\begin{sphinxuseclass}{cell_input}
\begin{sphinxVerbatim}[commandchars=\\\{\}]
\PYG{n}{grouped}\PYG{o}{.}\PYG{n}{mean}\PYG{p}{(}\PYG{p}{)}
\end{sphinxVerbatim}

\end{sphinxuseclass}\end{sphinxVerbatimInput}
\begin{sphinxVerbatimOutput}

\begin{sphinxuseclass}{cell_output}
\begin{sphinxVerbatim}[commandchars=\\\{\}]
2000\PYGZhy{}01\PYGZhy{}01   0.0000
2000\PYGZhy{}01\PYGZhy{}02   2.0000
2000\PYGZhy{}01\PYGZhy{}03   4.0000
dtype: float64
\end{sphinxVerbatim}

\end{sphinxuseclass}\end{sphinxVerbatimOutput}

\end{sphinxuseclass}
\begin{sphinxuseclass}{cell}\begin{sphinxVerbatimInput}

\begin{sphinxuseclass}{cell_input}
\begin{sphinxVerbatim}[commandchars=\\\{\}]
\PYG{n}{grouped}\PYG{o}{.}\PYG{n}{last}\PYG{p}{(}\PYG{p}{)}
\end{sphinxVerbatim}

\end{sphinxuseclass}\end{sphinxVerbatimInput}
\begin{sphinxVerbatimOutput}

\begin{sphinxuseclass}{cell_output}
\begin{sphinxVerbatim}[commandchars=\\\{\}]
2000\PYGZhy{}01\PYGZhy{}01    0
2000\PYGZhy{}01\PYGZhy{}02    3
2000\PYGZhy{}01\PYGZhy{}03    4
dtype: int64
\end{sphinxVerbatim}

\end{sphinxuseclass}\end{sphinxVerbatimOutput}

\end{sphinxuseclass}
\sphinxAtStartPar
Or we may want the number of observations on each date.

\begin{sphinxuseclass}{cell}\begin{sphinxVerbatimInput}

\begin{sphinxuseclass}{cell_input}
\begin{sphinxVerbatim}[commandchars=\\\{\}]
\PYG{n}{grouped}\PYG{o}{.}\PYG{n}{count}\PYG{p}{(}\PYG{p}{)}
\end{sphinxVerbatim}

\end{sphinxuseclass}\end{sphinxVerbatimInput}
\begin{sphinxVerbatimOutput}

\begin{sphinxuseclass}{cell_output}
\begin{sphinxVerbatim}[commandchars=\\\{\}]
2000\PYGZhy{}01\PYGZhy{}01    1
2000\PYGZhy{}01\PYGZhy{}02    3
2000\PYGZhy{}01\PYGZhy{}03    1
dtype: int64
\end{sphinxVerbatim}

\end{sphinxuseclass}\end{sphinxVerbatimOutput}

\end{sphinxuseclass}

\section{Date Ranges, Frequencies, and Shifting}
\label{\detokenize{mckinney_11_lecture:date-ranges-frequencies-and-shifting}}\begin{quote}

\sphinxAtStartPar
Generic time series in pandas are assumed to be irregular; that is, they have no fixed frequency. For many applications this is sufficient. However, it’s often desirable to work relative to a fixed frequency, such as daily, monthly, or every 15 minutes, even if that means introducing missing values into a time series. Fortunately pandas has a full suite of standard time series frequencies and tools for resampling, inferring frequencies, and generating fixed\sphinxhyphen{}frequency date ranges.
\end{quote}

\sphinxAtStartPar
We will skip the sections on creating date ranges or different frequencies so we can focus on shifting data.


\subsection{Shifting (Leading and Lagging) Data}
\label{\detokenize{mckinney_11_lecture:shifting-leading-and-lagging-data}}
\sphinxAtStartPar
\sphinxstyleemphasis{\sphinxstylestrong{Shifting is an important feature!}}
Shifting is moving data backward (or forward) through time.

\begin{sphinxuseclass}{cell}\begin{sphinxVerbatimInput}

\begin{sphinxuseclass}{cell_input}
\begin{sphinxVerbatim}[commandchars=\\\{\}]
\PYG{n}{np}\PYG{o}{.}\PYG{n}{random}\PYG{o}{.}\PYG{n}{seed}\PYG{p}{(}\PYG{l+m+mi}{42}\PYG{p}{)}
\PYG{n}{ts} \PYG{o}{=} \PYG{n}{pd}\PYG{o}{.}\PYG{n}{Series}\PYG{p}{(}\PYG{n}{np}\PYG{o}{.}\PYG{n}{random}\PYG{o}{.}\PYG{n}{randn}\PYG{p}{(}\PYG{l+m+mi}{4}\PYG{p}{)}\PYG{p}{,} \PYG{n}{index}\PYG{o}{=}\PYG{n}{pd}\PYG{o}{.}\PYG{n}{date\PYGZus{}range}\PYG{p}{(}\PYG{l+s+s1}{\PYGZsq{}}\PYG{l+s+s1}{1/1/2000}\PYG{l+s+s1}{\PYGZsq{}}\PYG{p}{,} \PYG{n}{periods}\PYG{o}{=}\PYG{l+m+mi}{4}\PYG{p}{,} \PYG{n}{freq}\PYG{o}{=}\PYG{l+s+s1}{\PYGZsq{}}\PYG{l+s+s1}{M}\PYG{l+s+s1}{\PYGZsq{}}\PYG{p}{)}\PYG{p}{)}

\PYG{n}{ts}
\end{sphinxVerbatim}

\end{sphinxuseclass}\end{sphinxVerbatimInput}
\begin{sphinxVerbatimOutput}

\begin{sphinxuseclass}{cell_output}
\begin{sphinxVerbatim}[commandchars=\\\{\}]
2000\PYGZhy{}01\PYGZhy{}31    0.4967
2000\PYGZhy{}02\PYGZhy{}29   \PYGZhy{}0.1383
2000\PYGZhy{}03\PYGZhy{}31    0.6477
2000\PYGZhy{}04\PYGZhy{}30    1.5230
Freq: M, dtype: float64
\end{sphinxVerbatim}

\end{sphinxuseclass}\end{sphinxVerbatimOutput}

\end{sphinxuseclass}
\sphinxAtStartPar
If we pass a positive integer \$N\$ to the \sphinxcode{\sphinxupquote{.shift()}} method:
\begin{enumerate}
\sphinxsetlistlabels{\arabic}{enumi}{enumii}{}{.}%
\item {} 
\sphinxAtStartPar
The date index remains the same

\item {} 
\sphinxAtStartPar
Values are shifted down \$N\$ observations

\end{enumerate}

\sphinxAtStartPar
“Lag” might be a better name than “shift” since a postive 2 makes the value at any timestamp the value from 2 timestamps above (earlier, since most time\sphinxhyphen{}series data are chronological).

\begin{sphinxuseclass}{cell}\begin{sphinxVerbatimInput}

\begin{sphinxuseclass}{cell_input}
\begin{sphinxVerbatim}[commandchars=\\\{\}]
\PYG{n}{ts}\PYG{o}{.}\PYG{n}{shift}\PYG{p}{(}\PYG{p}{)} \PYG{c+c1}{\PYGZsh{} if we do not specify \PYGZdq{}periods\PYGZdq{}, pandas assumes 1}
\end{sphinxVerbatim}

\end{sphinxuseclass}\end{sphinxVerbatimInput}
\begin{sphinxVerbatimOutput}

\begin{sphinxuseclass}{cell_output}
\begin{sphinxVerbatim}[commandchars=\\\{\}]
2000\PYGZhy{}01\PYGZhy{}31       NaN
2000\PYGZhy{}02\PYGZhy{}29    0.4967
2000\PYGZhy{}03\PYGZhy{}31   \PYGZhy{}0.1383
2000\PYGZhy{}04\PYGZhy{}30    0.6477
Freq: M, dtype: float64
\end{sphinxVerbatim}

\end{sphinxuseclass}\end{sphinxVerbatimOutput}

\end{sphinxuseclass}
\begin{sphinxuseclass}{cell}\begin{sphinxVerbatimInput}

\begin{sphinxuseclass}{cell_input}
\begin{sphinxVerbatim}[commandchars=\\\{\}]
\PYG{n}{ts}\PYG{o}{.}\PYG{n}{shift}\PYG{p}{(}\PYG{l+m+mi}{2}\PYG{p}{)}
\end{sphinxVerbatim}

\end{sphinxuseclass}\end{sphinxVerbatimInput}
\begin{sphinxVerbatimOutput}

\begin{sphinxuseclass}{cell_output}
\begin{sphinxVerbatim}[commandchars=\\\{\}]
2000\PYGZhy{}01\PYGZhy{}31       NaN
2000\PYGZhy{}02\PYGZhy{}29       NaN
2000\PYGZhy{}03\PYGZhy{}31    0.4967
2000\PYGZhy{}04\PYGZhy{}30   \PYGZhy{}0.1383
Freq: M, dtype: float64
\end{sphinxVerbatim}

\end{sphinxuseclass}\end{sphinxVerbatimOutput}

\end{sphinxuseclass}
\sphinxAtStartPar
If we pass a \sphinxstyleemphasis{negative} integer \$N\$ to the \sphinxcode{\sphinxupquote{.shift()}} method, values are shifted \sphinxstyleemphasis{up} \$N\$ observations.

\begin{sphinxuseclass}{cell}\begin{sphinxVerbatimInput}

\begin{sphinxuseclass}{cell_input}
\begin{sphinxVerbatim}[commandchars=\\\{\}]
\PYG{n}{ts}\PYG{o}{.}\PYG{n}{shift}\PYG{p}{(}\PYG{o}{\PYGZhy{}}\PYG{l+m+mi}{2}\PYG{p}{)}
\end{sphinxVerbatim}

\end{sphinxuseclass}\end{sphinxVerbatimInput}
\begin{sphinxVerbatimOutput}

\begin{sphinxuseclass}{cell_output}
\begin{sphinxVerbatim}[commandchars=\\\{\}]
2000\PYGZhy{}01\PYGZhy{}31   0.6477
2000\PYGZhy{}02\PYGZhy{}29   1.5230
2000\PYGZhy{}03\PYGZhy{}31      NaN
2000\PYGZhy{}04\PYGZhy{}30      NaN
Freq: M, dtype: float64
\end{sphinxVerbatim}

\end{sphinxuseclass}\end{sphinxVerbatimOutput}

\end{sphinxuseclass}
\sphinxAtStartPar
We will almost never shift with negative values (i.e., we will almost never bring forward values from the future) to prevent look\sphinxhyphen{}ahead bias.
We do not want to assume that financial market participants have access to future data.
Our most common shift will be to compute the percent change from one period to the next.
We can calculate the percent change two ways.

\begin{sphinxuseclass}{cell}\begin{sphinxVerbatimInput}

\begin{sphinxuseclass}{cell_input}
\begin{sphinxVerbatim}[commandchars=\\\{\}]
\PYG{n}{ts}\PYG{o}{.}\PYG{n}{pct\PYGZus{}change}\PYG{p}{(}\PYG{p}{)}
\end{sphinxVerbatim}

\end{sphinxuseclass}\end{sphinxVerbatimInput}
\begin{sphinxVerbatimOutput}

\begin{sphinxuseclass}{cell_output}
\begin{sphinxVerbatim}[commandchars=\\\{\}]
2000\PYGZhy{}01\PYGZhy{}31       NaN
2000\PYGZhy{}02\PYGZhy{}29   \PYGZhy{}1.2784
2000\PYGZhy{}03\PYGZhy{}31   \PYGZhy{}5.6844
2000\PYGZhy{}04\PYGZhy{}30    1.3515
Freq: M, dtype: float64
\end{sphinxVerbatim}

\end{sphinxuseclass}\end{sphinxVerbatimOutput}

\end{sphinxuseclass}
\begin{sphinxuseclass}{cell}\begin{sphinxVerbatimInput}

\begin{sphinxuseclass}{cell_input}
\begin{sphinxVerbatim}[commandchars=\\\{\}]
\PYG{p}{(}\PYG{n}{ts} \PYG{o}{\PYGZhy{}} \PYG{n}{ts}\PYG{o}{.}\PYG{n}{shift}\PYG{p}{(}\PYG{p}{)}\PYG{p}{)} \PYG{o}{/} \PYG{n}{ts}\PYG{o}{.}\PYG{n}{shift}\PYG{p}{(}\PYG{p}{)}
\end{sphinxVerbatim}

\end{sphinxuseclass}\end{sphinxVerbatimInput}
\begin{sphinxVerbatimOutput}

\begin{sphinxuseclass}{cell_output}
\begin{sphinxVerbatim}[commandchars=\\\{\}]
2000\PYGZhy{}01\PYGZhy{}31       NaN
2000\PYGZhy{}02\PYGZhy{}29   \PYGZhy{}1.2784
2000\PYGZhy{}03\PYGZhy{}31   \PYGZhy{}5.6844
2000\PYGZhy{}04\PYGZhy{}30    1.3515
Freq: M, dtype: float64
\end{sphinxVerbatim}

\end{sphinxuseclass}\end{sphinxVerbatimOutput}

\end{sphinxuseclass}
\begin{sphinxuseclass}{cell}\begin{sphinxVerbatimInput}

\begin{sphinxuseclass}{cell_input}
\begin{sphinxVerbatim}[commandchars=\\\{\}]
\PYG{n}{np}\PYG{o}{.}\PYG{n}{allclose}\PYG{p}{(}
    \PYG{n}{a}\PYG{o}{=}\PYG{n}{ts}\PYG{o}{.}\PYG{n}{pct\PYGZus{}change}\PYG{p}{(}\PYG{p}{)}\PYG{p}{,}
    \PYG{n}{b}\PYG{o}{=}\PYG{p}{(}\PYG{n}{ts} \PYG{o}{\PYGZhy{}} \PYG{n}{ts}\PYG{o}{.}\PYG{n}{shift}\PYG{p}{(}\PYG{p}{)}\PYG{p}{)} \PYG{o}{/} \PYG{n}{ts}\PYG{o}{.}\PYG{n}{shift}\PYG{p}{(}\PYG{p}{)}\PYG{p}{,}
    \PYG{n}{equal\PYGZus{}nan}\PYG{o}{=}\PYG{k+kc}{True}
\PYG{p}{)}
\end{sphinxVerbatim}

\end{sphinxuseclass}\end{sphinxVerbatimInput}
\begin{sphinxVerbatimOutput}

\begin{sphinxuseclass}{cell_output}
\begin{sphinxVerbatim}[commandchars=\\\{\}]
True
\end{sphinxVerbatim}

\end{sphinxuseclass}\end{sphinxVerbatimOutput}

\end{sphinxuseclass}
\sphinxAtStartPar
Two observations on the percent change calculations above:
\begin{enumerate}
\sphinxsetlistlabels{\arabic}{enumi}{enumii}{}{.}%
\item {} 
\sphinxAtStartPar
The first percent change is NaN (missing) because there is no previous value to change from

\item {} 
\sphinxAtStartPar
The default \sphinxcode{\sphinxupquote{periods}} argument for \sphinxcode{\sphinxupquote{.shift()}}  and \sphinxcode{\sphinxupquote{.pct\_change()}} is 1

\end{enumerate}

\sphinxAtStartPar
The naive shift examples above shift by a number of observations, without considering timestamps or their frequencies.
As a result, timestamps are unchanged and values shift down (positive \sphinxcode{\sphinxupquote{periods}} argument) or up (negative \sphinxcode{\sphinxupquote{periods}} argument).
However, we can also pass the \sphinxcode{\sphinxupquote{freq}} argument to respect the timestamps.
With the \sphinxcode{\sphinxupquote{freq}} argument, timestamps shift by a multiple (specified by the \sphinxcode{\sphinxupquote{periods}} argument) of datetime intervals (specified by the \sphinxcode{\sphinxupquote{freq}} argument).
Note that the examples below generate new datetime indexes.

\begin{sphinxuseclass}{cell}\begin{sphinxVerbatimInput}

\begin{sphinxuseclass}{cell_input}
\begin{sphinxVerbatim}[commandchars=\\\{\}]
\PYG{n}{ts}
\end{sphinxVerbatim}

\end{sphinxuseclass}\end{sphinxVerbatimInput}
\begin{sphinxVerbatimOutput}

\begin{sphinxuseclass}{cell_output}
\begin{sphinxVerbatim}[commandchars=\\\{\}]
2000\PYGZhy{}01\PYGZhy{}31    0.4967
2000\PYGZhy{}02\PYGZhy{}29   \PYGZhy{}0.1383
2000\PYGZhy{}03\PYGZhy{}31    0.6477
2000\PYGZhy{}04\PYGZhy{}30    1.5230
Freq: M, dtype: float64
\end{sphinxVerbatim}

\end{sphinxuseclass}\end{sphinxVerbatimOutput}

\end{sphinxuseclass}
\begin{sphinxuseclass}{cell}\begin{sphinxVerbatimInput}

\begin{sphinxuseclass}{cell_input}
\begin{sphinxVerbatim}[commandchars=\\\{\}]
\PYG{n}{ts}\PYG{o}{.}\PYG{n}{shift}\PYG{p}{(}\PYG{l+m+mi}{2}\PYG{p}{,} \PYG{n}{freq}\PYG{o}{=}\PYG{l+s+s1}{\PYGZsq{}}\PYG{l+s+s1}{M}\PYG{l+s+s1}{\PYGZsq{}}\PYG{p}{)}
\end{sphinxVerbatim}

\end{sphinxuseclass}\end{sphinxVerbatimInput}
\begin{sphinxVerbatimOutput}

\begin{sphinxuseclass}{cell_output}
\begin{sphinxVerbatim}[commandchars=\\\{\}]
2000\PYGZhy{}03\PYGZhy{}31    0.4967
2000\PYGZhy{}04\PYGZhy{}30   \PYGZhy{}0.1383
2000\PYGZhy{}05\PYGZhy{}31    0.6477
2000\PYGZhy{}06\PYGZhy{}30    1.5230
Freq: M, dtype: float64
\end{sphinxVerbatim}

\end{sphinxuseclass}\end{sphinxVerbatimOutput}

\end{sphinxuseclass}
\begin{sphinxuseclass}{cell}\begin{sphinxVerbatimInput}

\begin{sphinxuseclass}{cell_input}
\begin{sphinxVerbatim}[commandchars=\\\{\}]
\PYG{n}{ts}\PYG{o}{.}\PYG{n}{shift}\PYG{p}{(}\PYG{l+m+mi}{3}\PYG{p}{,} \PYG{n}{freq}\PYG{o}{=}\PYG{l+s+s1}{\PYGZsq{}}\PYG{l+s+s1}{D}\PYG{l+s+s1}{\PYGZsq{}}\PYG{p}{)}
\end{sphinxVerbatim}

\end{sphinxuseclass}\end{sphinxVerbatimInput}
\begin{sphinxVerbatimOutput}

\begin{sphinxuseclass}{cell_output}
\begin{sphinxVerbatim}[commandchars=\\\{\}]
2000\PYGZhy{}02\PYGZhy{}03    0.4967
2000\PYGZhy{}03\PYGZhy{}03   \PYGZhy{}0.1383
2000\PYGZhy{}04\PYGZhy{}03    0.6477
2000\PYGZhy{}05\PYGZhy{}03    1.5230
dtype: float64
\end{sphinxVerbatim}

\end{sphinxuseclass}\end{sphinxVerbatimOutput}

\end{sphinxuseclass}
\sphinxAtStartPar
\sphinxcode{\sphinxupquote{M}} is already months, so \sphinxcode{\sphinxupquote{T}} is minutes.

\begin{sphinxuseclass}{cell}\begin{sphinxVerbatimInput}

\begin{sphinxuseclass}{cell_input}
\begin{sphinxVerbatim}[commandchars=\\\{\}]
\PYG{n}{ts}\PYG{o}{.}\PYG{n}{shift}\PYG{p}{(}\PYG{l+m+mi}{1}\PYG{p}{,} \PYG{n}{freq}\PYG{o}{=}\PYG{l+s+s1}{\PYGZsq{}}\PYG{l+s+s1}{90T}\PYG{l+s+s1}{\PYGZsq{}}\PYG{p}{)}
\end{sphinxVerbatim}

\end{sphinxuseclass}\end{sphinxVerbatimInput}
\begin{sphinxVerbatimOutput}

\begin{sphinxuseclass}{cell_output}
\begin{sphinxVerbatim}[commandchars=\\\{\}]
2000\PYGZhy{}01\PYGZhy{}31 01:30:00    0.4967
2000\PYGZhy{}02\PYGZhy{}29 01:30:00   \PYGZhy{}0.1383
2000\PYGZhy{}03\PYGZhy{}31 01:30:00    0.6477
2000\PYGZhy{}04\PYGZhy{}30 01:30:00    1.5230
dtype: float64
\end{sphinxVerbatim}

\end{sphinxuseclass}\end{sphinxVerbatimOutput}

\end{sphinxuseclass}

\subsection{Shifting dates with offsets}
\label{\detokenize{mckinney_11_lecture:shifting-dates-with-offsets}}
\sphinxAtStartPar
We can also shift timestamps to the beginning or end of a period or interval.

\begin{sphinxuseclass}{cell}\begin{sphinxVerbatimInput}

\begin{sphinxuseclass}{cell_input}
\begin{sphinxVerbatim}[commandchars=\\\{\}]
\PYG{k+kn}{from} \PYG{n+nn}{pandas}\PYG{n+nn}{.}\PYG{n+nn}{tseries}\PYG{n+nn}{.}\PYG{n+nn}{offsets} \PYG{k+kn}{import} \PYG{n}{Day}\PYG{p}{,} \PYG{n}{MonthEnd}
\PYG{n}{now} \PYG{o}{=} \PYG{n}{datetime}\PYG{p}{(}\PYG{l+m+mi}{2011}\PYG{p}{,} \PYG{l+m+mi}{11}\PYG{p}{,} \PYG{l+m+mi}{17}\PYG{p}{)}
\PYG{n}{now} \PYG{o}{+} \PYG{l+m+mi}{3} \PYG{o}{*} \PYG{n}{Day}\PYG{p}{(}\PYG{p}{)}
\end{sphinxVerbatim}

\end{sphinxuseclass}\end{sphinxVerbatimInput}
\begin{sphinxVerbatimOutput}

\begin{sphinxuseclass}{cell_output}
\begin{sphinxVerbatim}[commandchars=\\\{\}]
Timestamp(\PYGZsq{}2011\PYGZhy{}11\PYGZhy{}20 00:00:00\PYGZsq{})
\end{sphinxVerbatim}

\end{sphinxuseclass}\end{sphinxVerbatimOutput}

\end{sphinxuseclass}
\begin{sphinxuseclass}{cell}\begin{sphinxVerbatimInput}

\begin{sphinxuseclass}{cell_input}
\begin{sphinxVerbatim}[commandchars=\\\{\}]
\PYG{n}{now} \PYG{o}{+} \PYG{n}{MonthEnd}\PYG{p}{(}\PYG{l+m+mi}{0}\PYG{p}{)} \PYG{c+c1}{\PYGZsh{} 0 is for move to the end of the month, but never leave the month}
\end{sphinxVerbatim}

\end{sphinxuseclass}\end{sphinxVerbatimInput}
\begin{sphinxVerbatimOutput}

\begin{sphinxuseclass}{cell_output}
\begin{sphinxVerbatim}[commandchars=\\\{\}]
Timestamp(\PYGZsq{}2011\PYGZhy{}11\PYGZhy{}30 00:00:00\PYGZsq{})
\end{sphinxVerbatim}

\end{sphinxuseclass}\end{sphinxVerbatimOutput}

\end{sphinxuseclass}
\begin{sphinxuseclass}{cell}\begin{sphinxVerbatimInput}

\begin{sphinxuseclass}{cell_input}
\begin{sphinxVerbatim}[commandchars=\\\{\}]
\PYG{n}{now} \PYG{o}{+} \PYG{n}{MonthEnd}\PYG{p}{(}\PYG{l+m+mi}{1}\PYG{p}{)} \PYG{c+c1}{\PYGZsh{} 1 is for move to the end of the month, if already at end, move to the next end}
\end{sphinxVerbatim}

\end{sphinxuseclass}\end{sphinxVerbatimInput}
\begin{sphinxVerbatimOutput}

\begin{sphinxuseclass}{cell_output}
\begin{sphinxVerbatim}[commandchars=\\\{\}]
Timestamp(\PYGZsq{}2011\PYGZhy{}11\PYGZhy{}30 00:00:00\PYGZsq{})
\end{sphinxVerbatim}

\end{sphinxuseclass}\end{sphinxVerbatimOutput}

\end{sphinxuseclass}
\begin{sphinxuseclass}{cell}\begin{sphinxVerbatimInput}

\begin{sphinxuseclass}{cell_input}
\begin{sphinxVerbatim}[commandchars=\\\{\}]
\PYG{n}{now} \PYG{o}{+} \PYG{n}{MonthEnd}\PYG{p}{(}\PYG{l+m+mi}{2}\PYG{p}{)}
\end{sphinxVerbatim}

\end{sphinxuseclass}\end{sphinxVerbatimInput}
\begin{sphinxVerbatimOutput}

\begin{sphinxuseclass}{cell_output}
\begin{sphinxVerbatim}[commandchars=\\\{\}]
Timestamp(\PYGZsq{}2011\PYGZhy{}12\PYGZhy{}31 00:00:00\PYGZsq{})
\end{sphinxVerbatim}

\end{sphinxuseclass}\end{sphinxVerbatimOutput}

\end{sphinxuseclass}
\sphinxAtStartPar
Date offsets can help us align data for presentation or merging.
\sphinxstyleemphasis{\sphinxstylestrong{But, be careful!}}
The default argument is 1, but we typically want 0.

\begin{sphinxuseclass}{cell}\begin{sphinxVerbatimInput}

\begin{sphinxuseclass}{cell_input}
\begin{sphinxVerbatim}[commandchars=\\\{\}]
\PYG{n}{datetime}\PYG{p}{(}\PYG{l+m+mi}{2021}\PYG{p}{,} \PYG{l+m+mi}{10}\PYG{p}{,} \PYG{l+m+mi}{30}\PYG{p}{)} \PYG{o}{+} \PYG{n}{MonthEnd}\PYG{p}{(}\PYG{l+m+mi}{0}\PYG{p}{)}
\end{sphinxVerbatim}

\end{sphinxuseclass}\end{sphinxVerbatimInput}
\begin{sphinxVerbatimOutput}

\begin{sphinxuseclass}{cell_output}
\begin{sphinxVerbatim}[commandchars=\\\{\}]
Timestamp(\PYGZsq{}2021\PYGZhy{}10\PYGZhy{}31 00:00:00\PYGZsq{})
\end{sphinxVerbatim}

\end{sphinxuseclass}\end{sphinxVerbatimOutput}

\end{sphinxuseclass}
\begin{sphinxuseclass}{cell}\begin{sphinxVerbatimInput}

\begin{sphinxuseclass}{cell_input}
\begin{sphinxVerbatim}[commandchars=\\\{\}]
\PYG{n}{datetime}\PYG{p}{(}\PYG{l+m+mi}{2021}\PYG{p}{,} \PYG{l+m+mi}{10}\PYG{p}{,} \PYG{l+m+mi}{30}\PYG{p}{)} \PYG{o}{+} \PYG{n}{MonthEnd}\PYG{p}{(}\PYG{l+m+mi}{1}\PYG{p}{)}
\end{sphinxVerbatim}

\end{sphinxuseclass}\end{sphinxVerbatimInput}
\begin{sphinxVerbatimOutput}

\begin{sphinxuseclass}{cell_output}
\begin{sphinxVerbatim}[commandchars=\\\{\}]
Timestamp(\PYGZsq{}2021\PYGZhy{}10\PYGZhy{}31 00:00:00\PYGZsq{})
\end{sphinxVerbatim}

\end{sphinxuseclass}\end{sphinxVerbatimOutput}

\end{sphinxuseclass}
\begin{sphinxuseclass}{cell}\begin{sphinxVerbatimInput}

\begin{sphinxuseclass}{cell_input}
\begin{sphinxVerbatim}[commandchars=\\\{\}]
\PYG{n}{datetime}\PYG{p}{(}\PYG{l+m+mi}{2021}\PYG{p}{,} \PYG{l+m+mi}{10}\PYG{p}{,} \PYG{l+m+mi}{31}\PYG{p}{)} \PYG{o}{+} \PYG{n}{MonthEnd}\PYG{p}{(}\PYG{l+m+mi}{0}\PYG{p}{)}
\end{sphinxVerbatim}

\end{sphinxuseclass}\end{sphinxVerbatimInput}
\begin{sphinxVerbatimOutput}

\begin{sphinxuseclass}{cell_output}
\begin{sphinxVerbatim}[commandchars=\\\{\}]
Timestamp(\PYGZsq{}2021\PYGZhy{}10\PYGZhy{}31 00:00:00\PYGZsq{})
\end{sphinxVerbatim}

\end{sphinxuseclass}\end{sphinxVerbatimOutput}

\end{sphinxuseclass}
\begin{sphinxuseclass}{cell}\begin{sphinxVerbatimInput}

\begin{sphinxuseclass}{cell_input}
\begin{sphinxVerbatim}[commandchars=\\\{\}]
\PYG{n}{datetime}\PYG{p}{(}\PYG{l+m+mi}{2021}\PYG{p}{,} \PYG{l+m+mi}{10}\PYG{p}{,} \PYG{l+m+mi}{31}\PYG{p}{)} \PYG{o}{+} \PYG{n}{MonthEnd}\PYG{p}{(}\PYG{l+m+mi}{1}\PYG{p}{)}
\end{sphinxVerbatim}

\end{sphinxuseclass}\end{sphinxVerbatimInput}
\begin{sphinxVerbatimOutput}

\begin{sphinxuseclass}{cell_output}
\begin{sphinxVerbatim}[commandchars=\\\{\}]
Timestamp(\PYGZsq{}2021\PYGZhy{}11\PYGZhy{}30 00:00:00\PYGZsq{})
\end{sphinxVerbatim}

\end{sphinxuseclass}\end{sphinxVerbatimOutput}

\end{sphinxuseclass}

\section{Resampling and Frequency Conversion}
\label{\detokenize{mckinney_11_lecture:resampling-and-frequency-conversion}}
\sphinxAtStartPar
\sphinxstyleemphasis{\sphinxstylestrong{Resampling is an important feature!}}
\begin{quote}

\sphinxAtStartPar
Resampling refers to the process of converting a time series from one frequency to
another. Aggregating higher frequency data to lower frequency is called
downsampling, while converting lower frequency to higher frequency is called upsampling. Not
all resampling falls into either of these categories; for example, converting W\sphinxhyphen{}WED
(weekly on Wednesday) to W\sphinxhyphen{}FRI is neither upsampling nor downsampling.
\end{quote}

\sphinxAtStartPar
We can resample both series and data frames.
The \sphinxcode{\sphinxupquote{.resample()}} method syntax is similar to the \sphinxcode{\sphinxupquote{.groupby()}} method syntax.
This similarity is because \sphinxcode{\sphinxupquote{.resample()}} is syntactic sugar for \sphinxcode{\sphinxupquote{.groupby()}}.


\subsection{Downsampling}
\label{\detokenize{mckinney_11_lecture:downsampling}}\begin{quote}

\sphinxAtStartPar
Aggregating data to a regular, lower frequency is a pretty normal time series task. The
data you’re aggregating doesn’t need to be fixed frequently; the desired frequency
defines bin edges that are used to slice the time series into pieces to aggregate. For
example, to convert to monthly, ‘M’ or ‘BM’, you need to chop up the data into
one\sphinxhyphen{}month intervals. Each interval is said to be half\sphinxhyphen{}open; a data point can only belong to
one interval, and the union of the intervals must make up the whole time frame.
There are a couple things to think about when using resample to downsample data:
\begin{itemize}
\item {} 
\sphinxAtStartPar
Which side of each interval is closed

\item {} 
\sphinxAtStartPar
How to label each aggregated bin, either with the start of the interval or the end

\end{itemize}
\end{quote}

\begin{sphinxuseclass}{cell}\begin{sphinxVerbatimInput}

\begin{sphinxuseclass}{cell_input}
\begin{sphinxVerbatim}[commandchars=\\\{\}]
\PYG{n}{rng} \PYG{o}{=} \PYG{n}{pd}\PYG{o}{.}\PYG{n}{date\PYGZus{}range}\PYG{p}{(}\PYG{l+s+s1}{\PYGZsq{}}\PYG{l+s+s1}{2000\PYGZhy{}01\PYGZhy{}01}\PYG{l+s+s1}{\PYGZsq{}}\PYG{p}{,} \PYG{n}{periods}\PYG{o}{=}\PYG{l+m+mi}{12}\PYG{p}{,} \PYG{n}{freq}\PYG{o}{=}\PYG{l+s+s1}{\PYGZsq{}}\PYG{l+s+s1}{T}\PYG{l+s+s1}{\PYGZsq{}}\PYG{p}{)}
\PYG{n}{ts} \PYG{o}{=} \PYG{n}{pd}\PYG{o}{.}\PYG{n}{Series}\PYG{p}{(}\PYG{n}{np}\PYG{o}{.}\PYG{n}{arange}\PYG{p}{(}\PYG{l+m+mi}{12}\PYG{p}{)}\PYG{p}{,} \PYG{n}{index}\PYG{o}{=}\PYG{n}{rng}\PYG{p}{)}

\PYG{n}{ts}
\end{sphinxVerbatim}

\end{sphinxuseclass}\end{sphinxVerbatimInput}
\begin{sphinxVerbatimOutput}

\begin{sphinxuseclass}{cell_output}
\begin{sphinxVerbatim}[commandchars=\\\{\}]
2000\PYGZhy{}01\PYGZhy{}01 00:00:00     0
2000\PYGZhy{}01\PYGZhy{}01 00:01:00     1
2000\PYGZhy{}01\PYGZhy{}01 00:02:00     2
2000\PYGZhy{}01\PYGZhy{}01 00:03:00     3
2000\PYGZhy{}01\PYGZhy{}01 00:04:00     4
2000\PYGZhy{}01\PYGZhy{}01 00:05:00     5
2000\PYGZhy{}01\PYGZhy{}01 00:06:00     6
2000\PYGZhy{}01\PYGZhy{}01 00:07:00     7
2000\PYGZhy{}01\PYGZhy{}01 00:08:00     8
2000\PYGZhy{}01\PYGZhy{}01 00:09:00     9
2000\PYGZhy{}01\PYGZhy{}01 00:10:00    10
2000\PYGZhy{}01\PYGZhy{}01 00:11:00    11
Freq: T, dtype: int64
\end{sphinxVerbatim}

\end{sphinxuseclass}\end{sphinxVerbatimOutput}

\end{sphinxuseclass}
\sphinxAtStartPar
We can aggregate the one\sphinxhyphen{}minute frequency data above to a five\sphinxhyphen{}minute frequency.
Resampling requires and aggregation method, and here McKinney chooses the \sphinxcode{\sphinxupquote{.sum()}} method.

\begin{sphinxuseclass}{cell}\begin{sphinxVerbatimInput}

\begin{sphinxuseclass}{cell_input}
\begin{sphinxVerbatim}[commandchars=\\\{\}]
\PYG{n}{ts}\PYG{o}{.}\PYG{n}{resample}\PYG{p}{(}\PYG{l+s+s1}{\PYGZsq{}}\PYG{l+s+s1}{5min}\PYG{l+s+s1}{\PYGZsq{}}\PYG{p}{)}\PYG{o}{.}\PYG{n}{sum}\PYG{p}{(}\PYG{p}{)}
\end{sphinxVerbatim}

\end{sphinxuseclass}\end{sphinxVerbatimInput}
\begin{sphinxVerbatimOutput}

\begin{sphinxuseclass}{cell_output}
\begin{sphinxVerbatim}[commandchars=\\\{\}]
2000\PYGZhy{}01\PYGZhy{}01 00:00:00    10
2000\PYGZhy{}01\PYGZhy{}01 00:05:00    35
2000\PYGZhy{}01\PYGZhy{}01 00:10:00    21
Freq: 5T, dtype: int64
\end{sphinxVerbatim}

\end{sphinxuseclass}\end{sphinxVerbatimOutput}

\end{sphinxuseclass}
\sphinxAtStartPar
Two observations about the previous resampling example:
\begin{enumerate}
\sphinxsetlistlabels{\arabic}{enumi}{enumii}{}{.}%
\item {} 
\sphinxAtStartPar
For minute\sphinxhyphen{}frequency resampling, the default is that the new data are labeled by the left edge of the resampling interval

\item {} 
\sphinxAtStartPar
For minute\sphinxhyphen{}frequency resampling, the default is that the left edge is closed (included) and the right edge is open (excluded)

\end{enumerate}

\sphinxAtStartPar
As a result, the first value of 10 at midnight is the sum of values at midnight and to the right of midnight, not including the value at 00:05 (i.e., \$10 = 0+1+2+3+4\$ at 00:00 and \$35 = 5+6+7+8+9\$  at 00:05).
We can use the \sphinxcode{\sphinxupquote{closed}} and \sphinxcode{\sphinxupquote{label}} arguments to change this behavior.

\sphinxAtStartPar
In finance, we prefer \sphinxcode{\sphinxupquote{closed='right'}} and \sphinxcode{\sphinxupquote{label='right'}}.

\begin{sphinxuseclass}{cell}\begin{sphinxVerbatimInput}

\begin{sphinxuseclass}{cell_input}
\begin{sphinxVerbatim}[commandchars=\\\{\}]
\PYG{n}{ts}\PYG{o}{.}\PYG{n}{resample}\PYG{p}{(}\PYG{l+s+s1}{\PYGZsq{}}\PYG{l+s+s1}{5min}\PYG{l+s+s1}{\PYGZsq{}}\PYG{p}{,} \PYG{n}{closed}\PYG{o}{=}\PYG{l+s+s1}{\PYGZsq{}}\PYG{l+s+s1}{right}\PYG{l+s+s1}{\PYGZsq{}}\PYG{p}{,} \PYG{n}{label}\PYG{o}{=}\PYG{l+s+s1}{\PYGZsq{}}\PYG{l+s+s1}{right}\PYG{l+s+s1}{\PYGZsq{}}\PYG{p}{)}\PYG{o}{.}\PYG{n}{sum}\PYG{p}{(}\PYG{p}{)} 
\end{sphinxVerbatim}

\end{sphinxuseclass}\end{sphinxVerbatimInput}
\begin{sphinxVerbatimOutput}

\begin{sphinxuseclass}{cell_output}
\begin{sphinxVerbatim}[commandchars=\\\{\}]
2000\PYGZhy{}01\PYGZhy{}01 00:00:00     0
2000\PYGZhy{}01\PYGZhy{}01 00:05:00    15
2000\PYGZhy{}01\PYGZhy{}01 00:10:00    40
2000\PYGZhy{}01\PYGZhy{}01 00:15:00    11
Freq: 5T, dtype: int64
\end{sphinxVerbatim}

\end{sphinxuseclass}\end{sphinxVerbatimOutput}

\end{sphinxuseclass}
\sphinxAtStartPar
Mixed combinations of \sphinxcode{\sphinxupquote{closed}} and \sphinxcode{\sphinxupquote{label}} are possible but confusing.

\begin{sphinxuseclass}{cell}\begin{sphinxVerbatimInput}

\begin{sphinxuseclass}{cell_input}
\begin{sphinxVerbatim}[commandchars=\\\{\}]
\PYG{n}{ts}\PYG{o}{.}\PYG{n}{resample}\PYG{p}{(}\PYG{l+s+s1}{\PYGZsq{}}\PYG{l+s+s1}{5min}\PYG{l+s+s1}{\PYGZsq{}}\PYG{p}{,} \PYG{n}{closed}\PYG{o}{=}\PYG{l+s+s1}{\PYGZsq{}}\PYG{l+s+s1}{right}\PYG{l+s+s1}{\PYGZsq{}}\PYG{p}{,} \PYG{n}{label}\PYG{o}{=}\PYG{l+s+s1}{\PYGZsq{}}\PYG{l+s+s1}{left}\PYG{l+s+s1}{\PYGZsq{}}\PYG{p}{)}\PYG{o}{.}\PYG{n}{sum}\PYG{p}{(}\PYG{p}{)} 
\end{sphinxVerbatim}

\end{sphinxuseclass}\end{sphinxVerbatimInput}
\begin{sphinxVerbatimOutput}

\begin{sphinxuseclass}{cell_output}
\begin{sphinxVerbatim}[commandchars=\\\{\}]
1999\PYGZhy{}12\PYGZhy{}31 23:55:00     0
2000\PYGZhy{}01\PYGZhy{}01 00:00:00    15
2000\PYGZhy{}01\PYGZhy{}01 00:05:00    40
2000\PYGZhy{}01\PYGZhy{}01 00:10:00    11
Freq: 5T, dtype: int64
\end{sphinxVerbatim}

\end{sphinxuseclass}\end{sphinxVerbatimOutput}

\end{sphinxuseclass}
\sphinxAtStartPar
These defaults for minute\sphinxhyphen{}frequency data may seem odd, but any choice is arbitrary.
I suggest you do the following when you use the \sphinxcode{\sphinxupquote{.resample()}} method:
\begin{enumerate}
\sphinxsetlistlabels{\arabic}{enumi}{enumii}{}{.}%
\item {} 
\sphinxAtStartPar
Read the docstring

\item {} 
\sphinxAtStartPar
Check your output

\end{enumerate}

\sphinxAtStartPar
pandas (and the \sphinxcode{\sphinxupquote{.resample()}} method) are mature and widely used, so the defaults are typically reasonable.


\subsection{Upsampling and Interpolation}
\label{\detokenize{mckinney_11_lecture:upsampling-and-interpolation}}
\sphinxAtStartPar
To downsample (i.e., resample from higher frequency to lower frequency), we have to choose an aggregation method (e.g., \sphinxcode{\sphinxupquote{.mean()}}, \sphinxcode{\sphinxupquote{.sum()}}, \sphinxcode{\sphinxupquote{.first()}}, or \sphinxcode{\sphinxupquote{.last()}}).
To upsample (i.e., resample from lower frequency to higher frequency), we do not have to choose an aggregation method.

\begin{sphinxuseclass}{cell}\begin{sphinxVerbatimInput}

\begin{sphinxuseclass}{cell_input}
\begin{sphinxVerbatim}[commandchars=\\\{\}]
\PYG{n}{np}\PYG{o}{.}\PYG{n}{random}\PYG{o}{.}\PYG{n}{seed}\PYG{p}{(}\PYG{l+m+mi}{42}\PYG{p}{)}
\PYG{n}{frame} \PYG{o}{=} \PYG{n}{pd}\PYG{o}{.}\PYG{n}{DataFrame}\PYG{p}{(}\PYG{n}{np}\PYG{o}{.}\PYG{n}{random}\PYG{o}{.}\PYG{n}{randn}\PYG{p}{(}\PYG{l+m+mi}{2}\PYG{p}{,} \PYG{l+m+mi}{4}\PYG{p}{)}\PYG{p}{,}
                     \PYG{n}{index}\PYG{o}{=}\PYG{n}{pd}\PYG{o}{.}\PYG{n}{date\PYGZus{}range}\PYG{p}{(}\PYG{l+s+s1}{\PYGZsq{}}\PYG{l+s+s1}{1/1/2000}\PYG{l+s+s1}{\PYGZsq{}}\PYG{p}{,} \PYG{n}{periods}\PYG{o}{=}\PYG{l+m+mi}{2}\PYG{p}{,} \PYG{n}{freq}\PYG{o}{=}\PYG{l+s+s1}{\PYGZsq{}}\PYG{l+s+s1}{W\PYGZhy{}WED}\PYG{l+s+s1}{\PYGZsq{}}\PYG{p}{)}\PYG{p}{,}
                     \PYG{n}{columns}\PYG{o}{=}\PYG{p}{[}\PYG{l+s+s1}{\PYGZsq{}}\PYG{l+s+s1}{Colorado}\PYG{l+s+s1}{\PYGZsq{}}\PYG{p}{,} \PYG{l+s+s1}{\PYGZsq{}}\PYG{l+s+s1}{Texas}\PYG{l+s+s1}{\PYGZsq{}}\PYG{p}{,} \PYG{l+s+s1}{\PYGZsq{}}\PYG{l+s+s1}{New York}\PYG{l+s+s1}{\PYGZsq{}}\PYG{p}{,} \PYG{l+s+s1}{\PYGZsq{}}\PYG{l+s+s1}{Ohio}\PYG{l+s+s1}{\PYGZsq{}}\PYG{p}{]}\PYG{p}{)}

\PYG{n}{frame}
\end{sphinxVerbatim}

\end{sphinxuseclass}\end{sphinxVerbatimInput}
\begin{sphinxVerbatimOutput}

\begin{sphinxuseclass}{cell_output}
\begin{sphinxVerbatim}[commandchars=\\\{\}]
            Colorado   Texas  New York   Ohio
2000\PYGZhy{}01\PYGZhy{}05    0.4967 \PYGZhy{}0.1383    0.6477 1.5230
2000\PYGZhy{}01\PYGZhy{}12   \PYGZhy{}0.2342 \PYGZhy{}0.2341    1.5792 0.7674
\end{sphinxVerbatim}

\end{sphinxuseclass}\end{sphinxVerbatimOutput}

\end{sphinxuseclass}
\sphinxAtStartPar
We can use the \sphinxcode{\sphinxupquote{.asfreq()}} method to convert to the new frequency “as is”.

\begin{sphinxuseclass}{cell}\begin{sphinxVerbatimInput}

\begin{sphinxuseclass}{cell_input}
\begin{sphinxVerbatim}[commandchars=\\\{\}]
\PYG{n}{df\PYGZus{}daily} \PYG{o}{=} \PYG{n}{frame}\PYG{o}{.}\PYG{n}{resample}\PYG{p}{(}\PYG{l+s+s1}{\PYGZsq{}}\PYG{l+s+s1}{D}\PYG{l+s+s1}{\PYGZsq{}}\PYG{p}{)}\PYG{o}{.}\PYG{n}{asfreq}\PYG{p}{(}\PYG{p}{)}

\PYG{n}{df\PYGZus{}daily}
\end{sphinxVerbatim}

\end{sphinxuseclass}\end{sphinxVerbatimInput}
\begin{sphinxVerbatimOutput}

\begin{sphinxuseclass}{cell_output}
\begin{sphinxVerbatim}[commandchars=\\\{\}]
            Colorado   Texas  New York   Ohio
2000\PYGZhy{}01\PYGZhy{}05    0.4967 \PYGZhy{}0.1383    0.6477 1.5230
2000\PYGZhy{}01\PYGZhy{}06       NaN     NaN       NaN    NaN
2000\PYGZhy{}01\PYGZhy{}07       NaN     NaN       NaN    NaN
2000\PYGZhy{}01\PYGZhy{}08       NaN     NaN       NaN    NaN
2000\PYGZhy{}01\PYGZhy{}09       NaN     NaN       NaN    NaN
2000\PYGZhy{}01\PYGZhy{}10       NaN     NaN       NaN    NaN
2000\PYGZhy{}01\PYGZhy{}11       NaN     NaN       NaN    NaN
2000\PYGZhy{}01\PYGZhy{}12   \PYGZhy{}0.2342 \PYGZhy{}0.2341    1.5792 0.7674
\end{sphinxVerbatim}

\end{sphinxuseclass}\end{sphinxVerbatimOutput}

\end{sphinxuseclass}
\sphinxAtStartPar
We do not \sphinxstyleemphasis{have} to choose an aggregation (disaggregation?) method, but we may want to choose a method to fill in the missing values.

\begin{sphinxuseclass}{cell}\begin{sphinxVerbatimInput}

\begin{sphinxuseclass}{cell_input}
\begin{sphinxVerbatim}[commandchars=\\\{\}]
\PYG{n}{frame}\PYG{o}{.}\PYG{n}{resample}\PYG{p}{(}\PYG{l+s+s1}{\PYGZsq{}}\PYG{l+s+s1}{D}\PYG{l+s+s1}{\PYGZsq{}}\PYG{p}{)}\PYG{o}{.}\PYG{n}{ffill}\PYG{p}{(}\PYG{p}{)}
\end{sphinxVerbatim}

\end{sphinxuseclass}\end{sphinxVerbatimInput}
\begin{sphinxVerbatimOutput}

\begin{sphinxuseclass}{cell_output}
\begin{sphinxVerbatim}[commandchars=\\\{\}]
            Colorado   Texas  New York   Ohio
2000\PYGZhy{}01\PYGZhy{}05    0.4967 \PYGZhy{}0.1383    0.6477 1.5230
2000\PYGZhy{}01\PYGZhy{}06    0.4967 \PYGZhy{}0.1383    0.6477 1.5230
2000\PYGZhy{}01\PYGZhy{}07    0.4967 \PYGZhy{}0.1383    0.6477 1.5230
2000\PYGZhy{}01\PYGZhy{}08    0.4967 \PYGZhy{}0.1383    0.6477 1.5230
2000\PYGZhy{}01\PYGZhy{}09    0.4967 \PYGZhy{}0.1383    0.6477 1.5230
2000\PYGZhy{}01\PYGZhy{}10    0.4967 \PYGZhy{}0.1383    0.6477 1.5230
2000\PYGZhy{}01\PYGZhy{}11    0.4967 \PYGZhy{}0.1383    0.6477 1.5230
2000\PYGZhy{}01\PYGZhy{}12   \PYGZhy{}0.2342 \PYGZhy{}0.2341    1.5792 0.7674
\end{sphinxVerbatim}

\end{sphinxuseclass}\end{sphinxVerbatimOutput}

\end{sphinxuseclass}
\begin{sphinxuseclass}{cell}\begin{sphinxVerbatimInput}

\begin{sphinxuseclass}{cell_input}
\begin{sphinxVerbatim}[commandchars=\\\{\}]
\PYG{n}{frame}\PYG{o}{.}\PYG{n}{resample}\PYG{p}{(}\PYG{l+s+s1}{\PYGZsq{}}\PYG{l+s+s1}{D}\PYG{l+s+s1}{\PYGZsq{}}\PYG{p}{)}\PYG{o}{.}\PYG{n}{ffill}\PYG{p}{(}\PYG{n}{limit}\PYG{o}{=}\PYG{l+m+mi}{2}\PYG{p}{)}
\end{sphinxVerbatim}

\end{sphinxuseclass}\end{sphinxVerbatimInput}
\begin{sphinxVerbatimOutput}

\begin{sphinxuseclass}{cell_output}
\begin{sphinxVerbatim}[commandchars=\\\{\}]
            Colorado   Texas  New York   Ohio
2000\PYGZhy{}01\PYGZhy{}05    0.4967 \PYGZhy{}0.1383    0.6477 1.5230
2000\PYGZhy{}01\PYGZhy{}06    0.4967 \PYGZhy{}0.1383    0.6477 1.5230
2000\PYGZhy{}01\PYGZhy{}07    0.4967 \PYGZhy{}0.1383    0.6477 1.5230
2000\PYGZhy{}01\PYGZhy{}08       NaN     NaN       NaN    NaN
2000\PYGZhy{}01\PYGZhy{}09       NaN     NaN       NaN    NaN
2000\PYGZhy{}01\PYGZhy{}10       NaN     NaN       NaN    NaN
2000\PYGZhy{}01\PYGZhy{}11       NaN     NaN       NaN    NaN
2000\PYGZhy{}01\PYGZhy{}12   \PYGZhy{}0.2342 \PYGZhy{}0.2341    1.5792 0.7674
\end{sphinxVerbatim}

\end{sphinxuseclass}\end{sphinxVerbatimOutput}

\end{sphinxuseclass}
\begin{sphinxuseclass}{cell}\begin{sphinxVerbatimInput}

\begin{sphinxuseclass}{cell_input}
\begin{sphinxVerbatim}[commandchars=\\\{\}]
\PYG{n}{frame}\PYG{o}{.}\PYG{n}{resample}\PYG{p}{(}\PYG{l+s+s1}{\PYGZsq{}}\PYG{l+s+s1}{W\PYGZhy{}THU}\PYG{l+s+s1}{\PYGZsq{}}\PYG{p}{)}\PYG{o}{.}\PYG{n}{ffill}\PYG{p}{(}\PYG{p}{)}
\end{sphinxVerbatim}

\end{sphinxuseclass}\end{sphinxVerbatimInput}
\begin{sphinxVerbatimOutput}

\begin{sphinxuseclass}{cell_output}
\begin{sphinxVerbatim}[commandchars=\\\{\}]
            Colorado   Texas  New York   Ohio
2000\PYGZhy{}01\PYGZhy{}06    0.4967 \PYGZhy{}0.1383    0.6477 1.5230
2000\PYGZhy{}01\PYGZhy{}13   \PYGZhy{}0.2342 \PYGZhy{}0.2341    1.5792 0.7674
\end{sphinxVerbatim}

\end{sphinxuseclass}\end{sphinxVerbatimOutput}

\end{sphinxuseclass}

\section{Moving Window Functions}
\label{\detokenize{mckinney_11_lecture:moving-window-functions}}
\sphinxAtStartPar
\sphinxstyleemphasis{\sphinxstylestrong{Moving window (or rolling window) functions are one of the neatest features of pandas, and we will frequently use moving window functions.}}
We will use data similar, but not identical, to the book data.
\sphinxstyleemphasis{\sphinxstylestrong{We must remove the timezone if we want to merge with Fama\sphinxhyphen{}French data.}}

\begin{sphinxuseclass}{cell}\begin{sphinxVerbatimInput}

\begin{sphinxuseclass}{cell_input}
\begin{sphinxVerbatim}[commandchars=\\\{\}]
\PYG{n}{df} \PYG{o}{=} \PYG{p}{(}
    \PYG{n}{yf}\PYG{o}{.}\PYG{n}{Tickers}\PYG{p}{(}\PYG{n}{tickers}\PYG{o}{=}\PYG{p}{[}\PYG{l+s+s1}{\PYGZsq{}}\PYG{l+s+s1}{AAPL}\PYG{l+s+s1}{\PYGZsq{}}\PYG{p}{,} \PYG{l+s+s1}{\PYGZsq{}}\PYG{l+s+s1}{MSFT}\PYG{l+s+s1}{\PYGZsq{}}\PYG{p}{,} \PYG{l+s+s1}{\PYGZsq{}}\PYG{l+s+s1}{SPY}\PYG{l+s+s1}{\PYGZsq{}}\PYG{p}{]}\PYG{p}{,} \PYG{n}{session}\PYG{o}{=}\PYG{n}{session}\PYG{p}{)}
    \PYG{o}{.}\PYG{n}{history}\PYG{p}{(}\PYG{n}{period}\PYG{o}{=}\PYG{l+s+s1}{\PYGZsq{}}\PYG{l+s+s1}{max}\PYG{l+s+s1}{\PYGZsq{}}\PYG{p}{,} \PYG{n}{auto\PYGZus{}adjust}\PYG{o}{=}\PYG{k+kc}{False}\PYG{p}{,} \PYG{n}{progress}\PYG{o}{=}\PYG{k+kc}{False}\PYG{p}{)}
    \PYG{o}{.}\PYG{n}{assign}\PYG{p}{(}\PYG{n}{Date} \PYG{o}{=} \PYG{k}{lambda} \PYG{n}{x}\PYG{p}{:} \PYG{n}{x}\PYG{o}{.}\PYG{n}{index}\PYG{o}{.}\PYG{n}{tz\PYGZus{}localize}\PYG{p}{(}\PYG{k+kc}{None}\PYG{p}{)}\PYG{p}{)}
    \PYG{o}{.}\PYG{n}{set\PYGZus{}index}\PYG{p}{(}\PYG{l+s+s1}{\PYGZsq{}}\PYG{l+s+s1}{Date}\PYG{l+s+s1}{\PYGZsq{}}\PYG{p}{)}
    \PYG{o}{.}\PYG{n}{rename\PYGZus{}axis}\PYG{p}{(}\PYG{n}{columns}\PYG{o}{=}\PYG{p}{[}\PYG{l+s+s1}{\PYGZsq{}}\PYG{l+s+s1}{Variable}\PYG{l+s+s1}{\PYGZsq{}}\PYG{p}{,} \PYG{l+s+s1}{\PYGZsq{}}\PYG{l+s+s1}{Ticker}\PYG{l+s+s1}{\PYGZsq{}}\PYG{p}{]}\PYG{p}{)}
\PYG{p}{)}

\PYG{n}{df}\PYG{o}{.}\PYG{n}{head}\PYG{p}{(}\PYG{p}{)}
\end{sphinxVerbatim}

\end{sphinxuseclass}\end{sphinxVerbatimInput}
\begin{sphinxVerbatimOutput}

\begin{sphinxuseclass}{cell_output}
\begin{sphinxVerbatim}[commandchars=\\\{\}]
Variable   Adj Close          Capital Gains  Close          Dividends       \PYGZbs{}
Ticker          AAPL MSFT SPY           SPY   AAPL MSFT SPY      AAPL MSFT   
Date                                                                         
1980\PYGZhy{}12\PYGZhy{}12    0.0997  NaN NaN           NaN 0.1283  NaN NaN    0.0000  NaN   
1980\PYGZhy{}12\PYGZhy{}15    0.0945  NaN NaN           NaN 0.1217  NaN NaN    0.0000  NaN   
1980\PYGZhy{}12\PYGZhy{}16    0.0876  NaN NaN           NaN 0.1127  NaN NaN    0.0000  NaN   
1980\PYGZhy{}12\PYGZhy{}17    0.0897  NaN NaN           NaN 0.1155  NaN NaN    0.0000  NaN   
1980\PYGZhy{}12\PYGZhy{}18    0.0924  NaN NaN           NaN 0.1189  NaN NaN    0.0000  NaN   

Variable        ... Low   Open          Stock Splits              Volume       \PYGZbs{}
Ticker     SPY  ... SPY   AAPL MSFT SPY         AAPL MSFT SPY       AAPL MSFT   
Date            ...                                                             
1980\PYGZhy{}12\PYGZhy{}12 NaN  ... NaN 0.1283  NaN NaN       0.0000  NaN NaN  469033600  NaN   
1980\PYGZhy{}12\PYGZhy{}15 NaN  ... NaN 0.1222  NaN NaN       0.0000  NaN NaN  175884800  NaN   
1980\PYGZhy{}12\PYGZhy{}16 NaN  ... NaN 0.1133  NaN NaN       0.0000  NaN NaN  105728000  NaN   
1980\PYGZhy{}12\PYGZhy{}17 NaN  ... NaN 0.1155  NaN NaN       0.0000  NaN NaN   86441600  NaN   
1980\PYGZhy{}12\PYGZhy{}18 NaN  ... NaN 0.1189  NaN NaN       0.0000  NaN NaN   73449600  NaN   

Variable        
Ticker     SPY  
Date            
1980\PYGZhy{}12\PYGZhy{}12 NaN  
1980\PYGZhy{}12\PYGZhy{}15 NaN  
1980\PYGZhy{}12\PYGZhy{}16 NaN  
1980\PYGZhy{}12\PYGZhy{}17 NaN  
1980\PYGZhy{}12\PYGZhy{}18 NaN  

[5 rows x 25 columns]
\end{sphinxVerbatim}

\end{sphinxuseclass}\end{sphinxVerbatimOutput}

\end{sphinxuseclass}
\sphinxAtStartPar
The \sphinxcode{\sphinxupquote{.rolling()}} method is similar to the \sphinxcode{\sphinxupquote{.groupby()}} and \sphinxcode{\sphinxupquote{.resample()}} methods.
The \sphinxcode{\sphinxupquote{.rolling()}} method accepts a window\sphinxhyphen{}width and requires an aggregation method.
The next example calculates and plots the 252\sphinxhyphen{}trading day moving average of AAPL’s price alongside the daily price.

\begin{sphinxuseclass}{cell}\begin{sphinxVerbatimInput}

\begin{sphinxuseclass}{cell_input}
\begin{sphinxVerbatim}[commandchars=\\\{\}]
\PYG{n}{aapl} \PYG{o}{=} \PYG{n}{df}\PYG{o}{.}\PYG{n}{loc}\PYG{p}{[}\PYG{l+s+s1}{\PYGZsq{}}\PYG{l+s+s1}{2012}\PYG{l+s+s1}{\PYGZsq{}}\PYG{p}{:}\PYG{p}{,} \PYG{p}{(}\PYG{l+s+s1}{\PYGZsq{}}\PYG{l+s+s1}{Adj Close}\PYG{l+s+s1}{\PYGZsq{}}\PYG{p}{,} \PYG{l+s+s1}{\PYGZsq{}}\PYG{l+s+s1}{AAPL}\PYG{l+s+s1}{\PYGZsq{}}\PYG{p}{)}\PYG{p}{]}
\PYG{n}{aapl}\PYG{o}{.}\PYG{n}{plot}\PYG{p}{(}\PYG{n}{label}\PYG{o}{=}\PYG{l+s+s1}{\PYGZsq{}}\PYG{l+s+s1}{Observed}\PYG{l+s+s1}{\PYGZsq{}}\PYG{p}{)}
\PYG{n}{aapl}\PYG{o}{.}\PYG{n}{rolling}\PYG{p}{(}\PYG{l+m+mi}{252}\PYG{p}{)}\PYG{o}{.}\PYG{n}{mean}\PYG{p}{(}\PYG{p}{)}\PYG{o}{.}\PYG{n}{plot}\PYG{p}{(}\PYG{n}{label}\PYG{o}{=}\PYG{l+s+s1}{\PYGZsq{}}\PYG{l+s+s1}{252 Trading Day Mean}\PYG{l+s+s1}{\PYGZsq{}}\PYG{p}{)} \PYG{c+c1}{\PYGZsh{} min\PYGZus{}periods defaults to 252}
\PYG{n}{aapl}\PYG{o}{.}\PYG{n}{rolling}\PYG{p}{(}\PYG{l+s+s1}{\PYGZsq{}}\PYG{l+s+s1}{365D}\PYG{l+s+s1}{\PYGZsq{}}\PYG{p}{)}\PYG{o}{.}\PYG{n}{mean}\PYG{p}{(}\PYG{p}{)}\PYG{o}{.}\PYG{n}{plot}\PYG{p}{(}\PYG{n}{label}\PYG{o}{=}\PYG{l+s+s1}{\PYGZsq{}}\PYG{l+s+s1}{365 Calendar Day Mean}\PYG{l+s+s1}{\PYGZsq{}}\PYG{p}{)} \PYG{c+c1}{\PYGZsh{} min\PYGZus{}periods defaults to 1}
\PYG{n}{aapl}\PYG{o}{.}\PYG{n}{resample}\PYG{p}{(}\PYG{l+s+s1}{\PYGZsq{}}\PYG{l+s+s1}{A}\PYG{l+s+s1}{\PYGZsq{}}\PYG{p}{)}\PYG{o}{.}\PYG{n}{mean}\PYG{p}{(}\PYG{p}{)}\PYG{o}{.}\PYG{n}{plot}\PYG{p}{(}\PYG{n}{style}\PYG{o}{=}\PYG{l+s+s1}{\PYGZsq{}}\PYG{l+s+s1}{.}\PYG{l+s+s1}{\PYGZsq{}}\PYG{p}{,} \PYG{n}{label}\PYG{o}{=}\PYG{l+s+s1}{\PYGZsq{}}\PYG{l+s+s1}{Calendar Year Mean}\PYG{l+s+s1}{\PYGZsq{}}\PYG{p}{)}
\PYG{n}{plt}\PYG{o}{.}\PYG{n}{legend}\PYG{p}{(}\PYG{p}{)}
\PYG{n}{plt}\PYG{o}{.}\PYG{n}{ylabel}\PYG{p}{(}\PYG{l+s+s1}{\PYGZsq{}}\PYG{l+s+s1}{AAPL Adjusted Close (\PYGZdl{})}\PYG{l+s+s1}{\PYGZsq{}}\PYG{p}{)}
\PYG{n}{plt}\PYG{o}{.}\PYG{n}{title}\PYG{p}{(}\PYG{l+s+s1}{\PYGZsq{}}\PYG{l+s+s1}{Comparison of Rolling and Resampling Means}\PYG{l+s+s1}{\PYGZsq{}}\PYG{p}{)}
\PYG{n}{plt}\PYG{o}{.}\PYG{n}{show}\PYG{p}{(}\PYG{p}{)}
\end{sphinxVerbatim}

\end{sphinxuseclass}\end{sphinxVerbatimInput}
\begin{sphinxVerbatimOutput}

\begin{sphinxuseclass}{cell_output}
\noindent\sphinxincludegraphics{{73306499bfed819657f6608b4c2fc4b0b074c7dbb1219a1d79291a0d258024ae}.png}

\end{sphinxuseclass}\end{sphinxVerbatimOutput}

\end{sphinxuseclass}
\sphinxAtStartPar
Two observations:
\begin{enumerate}
\sphinxsetlistlabels{\arabic}{enumi}{enumii}{}{.}%
\item {} 
\sphinxAtStartPar
If we pass the window\sphinxhyphen{}width as an integer, the window\sphinxhyphen{}width is based on the number of observations and ignores time stamps

\item {} 
\sphinxAtStartPar
If we pass the window\sphinxhyphen{}width as an integer, the \sphinxcode{\sphinxupquote{.rolling()}} method requires that number of observations for all windows (i.e., note that the moving average starts 251 trading days after the first daily price

\end{enumerate}

\sphinxAtStartPar
We can use the \sphinxcode{\sphinxupquote{min\_periods}} argument to allow incomplete windows.
For integer window widths, \sphinxcode{\sphinxupquote{min\_periods defaults}} to the given integer window width.
For string date offsets, \sphinxcode{\sphinxupquote{min\_periods defaults}} to \sphinxcode{\sphinxupquote{1}}.


\subsection{Binary Moving Window Functions}
\label{\detokenize{mckinney_11_lecture:binary-moving-window-functions}}
\sphinxAtStartPar
Binary moving window functions accept two inputs.
The most common example is the rolling correlation between two returns series.

\begin{sphinxuseclass}{cell}\begin{sphinxVerbatimInput}

\begin{sphinxuseclass}{cell_input}
\begin{sphinxVerbatim}[commandchars=\\\{\}]
\PYG{n}{returns} \PYG{o}{=} \PYG{n}{df}\PYG{p}{[}\PYG{l+s+s1}{\PYGZsq{}}\PYG{l+s+s1}{Adj Close}\PYG{l+s+s1}{\PYGZsq{}}\PYG{p}{]}\PYG{o}{.}\PYG{n}{pct\PYGZus{}change}\PYG{p}{(}\PYG{p}{)}

\PYG{n}{returns}\PYG{o}{.}\PYG{n}{head}\PYG{p}{(}\PYG{p}{)}
\end{sphinxVerbatim}

\end{sphinxuseclass}\end{sphinxVerbatimInput}
\begin{sphinxVerbatimOutput}

\begin{sphinxuseclass}{cell_output}
\begin{sphinxVerbatim}[commandchars=\\\{\}]
Ticker        AAPL  MSFT  SPY
Date                         
1980\PYGZhy{}12\PYGZhy{}12     NaN   NaN  NaN
1980\PYGZhy{}12\PYGZhy{}15 \PYGZhy{}0.0522   NaN  NaN
1980\PYGZhy{}12\PYGZhy{}16 \PYGZhy{}0.0734   NaN  NaN
1980\PYGZhy{}12\PYGZhy{}17  0.0248   NaN  NaN
1980\PYGZhy{}12\PYGZhy{}18  0.0290   NaN  NaN
\end{sphinxVerbatim}

\end{sphinxuseclass}\end{sphinxVerbatimOutput}

\end{sphinxuseclass}
\begin{sphinxuseclass}{cell}\begin{sphinxVerbatimInput}

\begin{sphinxuseclass}{cell_input}
\begin{sphinxVerbatim}[commandchars=\\\{\}]
\PYG{n}{returns}\PYG{p}{[}\PYG{l+s+s1}{\PYGZsq{}}\PYG{l+s+s1}{AAPL}\PYG{l+s+s1}{\PYGZsq{}}\PYG{p}{]}\PYG{o}{.}\PYG{n}{rolling}\PYG{p}{(}\PYG{l+m+mi}{126}\PYG{p}{,} \PYG{n}{min\PYGZus{}periods}\PYG{o}{=}\PYG{l+m+mi}{100}\PYG{p}{)}\PYG{o}{.}\PYG{n}{corr}\PYG{p}{(}\PYG{n}{returns}\PYG{p}{[}\PYG{l+s+s1}{\PYGZsq{}}\PYG{l+s+s1}{SPY}\PYG{l+s+s1}{\PYGZsq{}}\PYG{p}{]}\PYG{p}{)}\PYG{o}{.}\PYG{n}{plot}\PYG{p}{(}\PYG{p}{)}
\PYG{n}{plt}\PYG{o}{.}\PYG{n}{ylabel}\PYG{p}{(}\PYG{l+s+s1}{\PYGZsq{}}\PYG{l+s+s1}{Correlation(AAPL, SPY)}\PYG{l+s+s1}{\PYGZsq{}}\PYG{p}{)}
\PYG{n}{plt}\PYG{o}{.}\PYG{n}{title}\PYG{p}{(}\PYG{l+s+s1}{\PYGZsq{}}\PYG{l+s+s1}{Rolling Correlation between AAPL and SPY}\PYG{l+s+se}{\PYGZbs{}n}\PYG{l+s+s1}{ (126\PYGZhy{}Day Window w/ 100\PYGZhy{}Day Minimum)}\PYG{l+s+s1}{\PYGZsq{}}\PYG{p}{)}
\PYG{n}{plt}\PYG{o}{.}\PYG{n}{show}\PYG{p}{(}\PYG{p}{)}
\end{sphinxVerbatim}

\end{sphinxuseclass}\end{sphinxVerbatimInput}
\begin{sphinxVerbatimOutput}

\begin{sphinxuseclass}{cell_output}
\noindent\sphinxincludegraphics{{b70bab101a835c6d363dea9990d1fe4a40dfa16554097fbc77efeb1775d784da}.png}

\end{sphinxuseclass}\end{sphinxVerbatimOutput}

\end{sphinxuseclass}
\begin{sphinxuseclass}{cell}\begin{sphinxVerbatimInput}

\begin{sphinxuseclass}{cell_input}
\begin{sphinxVerbatim}[commandchars=\\\{\}]
\PYG{n}{returns}\PYG{p}{[}\PYG{p}{[}\PYG{l+s+s1}{\PYGZsq{}}\PYG{l+s+s1}{AAPL}\PYG{l+s+s1}{\PYGZsq{}}\PYG{p}{,} \PYG{l+s+s1}{\PYGZsq{}}\PYG{l+s+s1}{MSFT}\PYG{l+s+s1}{\PYGZsq{}}\PYG{p}{]}\PYG{p}{]}\PYG{o}{.}\PYG{n}{rolling}\PYG{p}{(}\PYG{l+m+mi}{126}\PYG{p}{,} \PYG{n}{min\PYGZus{}periods}\PYG{o}{=}\PYG{l+m+mi}{100}\PYG{p}{)}\PYG{o}{.}\PYG{n}{corr}\PYG{p}{(}\PYG{n}{returns}\PYG{p}{[}\PYG{l+s+s1}{\PYGZsq{}}\PYG{l+s+s1}{SPY}\PYG{l+s+s1}{\PYGZsq{}}\PYG{p}{]}\PYG{p}{)}\PYG{o}{.}\PYG{n}{plot}\PYG{p}{(}\PYG{p}{)}
\PYG{n}{plt}\PYG{o}{.}\PYG{n}{ylabel}\PYG{p}{(}\PYG{l+s+s1}{\PYGZsq{}}\PYG{l+s+s1}{Correlation with SPY}\PYG{l+s+s1}{\PYGZsq{}}\PYG{p}{)}
\PYG{n}{plt}\PYG{o}{.}\PYG{n}{title}\PYG{p}{(}\PYG{l+s+s1}{\PYGZsq{}}\PYG{l+s+s1}{Rolling Correlation with SPY}\PYG{l+s+se}{\PYGZbs{}n}\PYG{l+s+s1}{ (126\PYGZhy{}Day Window w/ 100\PYGZhy{}Day Minimum)}\PYG{l+s+s1}{\PYGZsq{}}\PYG{p}{)}
\PYG{n}{plt}\PYG{o}{.}\PYG{n}{show}\PYG{p}{(}\PYG{p}{)}
\end{sphinxVerbatim}

\end{sphinxuseclass}\end{sphinxVerbatimInput}
\begin{sphinxVerbatimOutput}

\begin{sphinxuseclass}{cell_output}
\noindent\sphinxincludegraphics{{9c0b888b16fbfe7a5c50183037a64f8369a508e910bf4aad8d1a485830894dbf}.png}

\end{sphinxuseclass}\end{sphinxVerbatimOutput}

\end{sphinxuseclass}

\subsection{User\sphinxhyphen{}Defined Moving Window Functions}
\label{\detokenize{mckinney_11_lecture:user-defined-moving-window-functions}}
\sphinxAtStartPar
Finally, we can define our own moving window functions and use the \sphinxcode{\sphinxupquote{.apply()}} method to apply them
However, note that \sphinxcode{\sphinxupquote{.apply()}} will be much slower than the the optimized moving window functions (e.g., \sphinxcode{\sphinxupquote{.mean()}}, \sphinxcode{\sphinxupquote{.std()}}, etc.).

\sphinxAtStartPar
McKinney provides an abstract example here, but we will discuss a simpler example that calculates rolling volatility.
Also, calculating rolling volatility with the \sphinxcode{\sphinxupquote{.apply()}} method provides us a chance to benchmark it against the optimized version.

\begin{sphinxuseclass}{cell}\begin{sphinxVerbatimInput}

\begin{sphinxuseclass}{cell_input}
\begin{sphinxVerbatim}[commandchars=\\\{\}]
\PYG{n}{returns}\PYG{p}{[}\PYG{l+s+s1}{\PYGZsq{}}\PYG{l+s+s1}{AAPL}\PYG{l+s+s1}{\PYGZsq{}}\PYG{p}{]}\PYG{o}{.}\PYG{n}{rolling}\PYG{p}{(}\PYG{l+m+mi}{252}\PYG{p}{)}\PYG{o}{.}\PYG{n}{apply}\PYG{p}{(}\PYG{n}{np}\PYG{o}{.}\PYG{n}{std}\PYG{p}{)}\PYG{o}{.}\PYG{n}{mul}\PYG{p}{(}\PYG{n}{np}\PYG{o}{.}\PYG{n}{sqrt}\PYG{p}{(}\PYG{l+m+mi}{252}\PYG{p}{)} \PYG{o}{*} \PYG{l+m+mi}{100}\PYG{p}{)}\PYG{o}{.}\PYG{n}{plot}\PYG{p}{(}\PYG{p}{)} \PYG{c+c1}{\PYGZsh{} annualize and convert to percent}
\PYG{n}{plt}\PYG{o}{.}\PYG{n}{ylabel}\PYG{p}{(}\PYG{l+s+s1}{\PYGZsq{}}\PYG{l+s+s1}{Volatility (}\PYG{l+s+s1}{\PYGZpc{}}\PYG{l+s+s1}{)}\PYG{l+s+s1}{\PYGZsq{}}\PYG{p}{)}
\PYG{n}{plt}\PYG{o}{.}\PYG{n}{title}\PYG{p}{(}\PYG{l+s+s1}{\PYGZsq{}}\PYG{l+s+s1}{Rolling Volatility}\PYG{l+s+se}{\PYGZbs{}n}\PYG{l+s+s1}{ (252\PYGZhy{}Day Window w/ 252\PYGZhy{}Day Minimum)}\PYG{l+s+s1}{\PYGZsq{}}\PYG{p}{)}
\PYG{n}{plt}\PYG{o}{.}\PYG{n}{show}\PYG{p}{(}\PYG{p}{)}
\end{sphinxVerbatim}

\end{sphinxuseclass}\end{sphinxVerbatimInput}
\begin{sphinxVerbatimOutput}

\begin{sphinxuseclass}{cell_output}
\noindent\sphinxincludegraphics{{b08139df980f12c04bebffed82b1e46447b329c989e36b2d32c5137f9b6461ef}.png}

\end{sphinxuseclass}\end{sphinxVerbatimOutput}

\end{sphinxuseclass}
\sphinxAtStartPar
Do not be afraid to use \sphinxcode{\sphinxupquote{.apply()}}, but realize that \sphinxcode{\sphinxupquote{.apply()}} is typically 1000\sphinxhyphen{}times slower than the pre\sphinxhyphen{}built method.

\begin{sphinxuseclass}{cell}\begin{sphinxVerbatimInput}

\begin{sphinxuseclass}{cell_input}
\begin{sphinxVerbatim}[commandchars=\\\{\}]
\PYG{o}{\PYGZpc{}}\PYG{k}{timeit} returns[\PYGZsq{}AAPL\PYGZsq{}].rolling(252).apply(np.std)
\end{sphinxVerbatim}

\end{sphinxuseclass}\end{sphinxVerbatimInput}
\begin{sphinxVerbatimOutput}

\begin{sphinxuseclass}{cell_output}
\begin{sphinxVerbatim}[commandchars=\\\{\}]
571 ms ± 3.68 ms per loop (mean ± std. dev. of 7 runs, 1 loop each)
\end{sphinxVerbatim}

\end{sphinxuseclass}\end{sphinxVerbatimOutput}

\end{sphinxuseclass}
\begin{sphinxuseclass}{cell}\begin{sphinxVerbatimInput}

\begin{sphinxuseclass}{cell_input}
\begin{sphinxVerbatim}[commandchars=\\\{\}]
\PYG{o}{\PYGZpc{}}\PYG{k}{timeit} returns[\PYGZsq{}AAPL\PYGZsq{}].rolling(252).std()
\end{sphinxVerbatim}

\end{sphinxuseclass}\end{sphinxVerbatimInput}
\begin{sphinxVerbatimOutput}

\begin{sphinxuseclass}{cell_output}
\begin{sphinxVerbatim}[commandchars=\\\{\}]
230 µs ± 8.46 µs per loop (mean ± std. dev. of 7 runs, 1000 loops each)
\end{sphinxVerbatim}

\end{sphinxuseclass}\end{sphinxVerbatimOutput}

\end{sphinxuseclass}
\sphinxstepscope


\section{McKinney Chapter 11 \sphinxhyphen{} Practice (Blank)}
\label{\detokenize{mckinney_11_practice:mckinney-chapter-11-practice-blank}}\label{\detokenize{mckinney_11_practice::doc}}

\subsection{Announcements}
\label{\detokenize{mckinney_11_practice:announcements}}

\subsection{Practice}
\label{\detokenize{mckinney_11_practice:practice}}

\subsubsection{Which are larger, overnight or intraday returns?}
\label{\detokenize{mckinney_11_practice:which-are-larger-overnight-or-intraday-returns}}
\sphinxAtStartPar
Yahoo! Finance provides easy acces to high\sphinxhyphen{}quality open, high, low, close (OHLC) and adjusted close price data.
However, Yahoo! Finance does not provide overnight or instraday returns directly.
Therefore, we need to use math to decompose daily returns into overnight and intraday returns.

\sphinxAtStartPar
Daily returns are defined as (adjusted) closing price to (adjusted) closing price returns.
Therefore, daily returns consist of overnight returns compounded with the intraday returns from the next day \$(1 + R\_\{daily\}) = (1 + R\_\{overnight\}) \textbackslash{}times (1 + R\_\{intraday\})\$ which we can rearrange to calculate overnight returns as \$\textbackslash{}frac\{1 + R\_\{daily\}\}\{1 + R\_\{intraday\}\} \sphinxhyphen{} 1 = R\_\{overnight\}\$.

\sphinxAtStartPar
We can calculate daily and intraday returns from Yahoo! Finance data as \$R\_\{daily\} = \textbackslash{}frac\{Adj\textbackslash{} Close\_\{t\} \sphinxhyphen{} Adj\textbackslash{} Close\_\{t\sphinxhyphen{}1\}\}\{Adj\textbackslash{} Close\_\{t\sphinxhyphen{}1\}\}\$ and \$R\_\{intraday\} = \textbackslash{}frac\{Close \sphinxhyphen{} Open\}\{Open\}\$.

\sphinxAtStartPar
Compare the following for the SPY ETF:
\begin{enumerate}
\sphinxsetlistlabels{\arabic}{enumi}{enumii}{}{.}%
\item {} 
\sphinxAtStartPar
Cumulative returns with all available data

\item {} 
\sphinxAtStartPar
Total returns for each calendar year

\item {} 
\sphinxAtStartPar
Total returns over rolling 252\sphinxhyphen{}trading\sphinxhyphen{}day windows

\item {} 
\sphinxAtStartPar
Total returns over rolling 12\sphinxhyphen{}months windows after calculating monthly returns

\item {} 
\sphinxAtStartPar
Sharpe Ratios for each calendar year

\end{enumerate}


\paragraph{Cumulative returns with all available data}
\label{\detokenize{mckinney_11_practice:cumulative-returns-with-all-available-data}}

\paragraph{Total returns for each calendar year}
\label{\detokenize{mckinney_11_practice:total-returns-for-each-calendar-year}}

\paragraph{Total returns over rolling 252\sphinxhyphen{}trading\sphinxhyphen{}day windows}
\label{\detokenize{mckinney_11_practice:total-returns-over-rolling-252-trading-day-windows}}

\paragraph{Total returns over rolling 12\sphinxhyphen{}months windows after calculating monthly returns}
\label{\detokenize{mckinney_11_practice:total-returns-over-rolling-12-months-windows-after-calculating-monthly-returns}}

\paragraph{Sharpe Ratios for each calendar year}
\label{\detokenize{mckinney_11_practice:sharpe-ratios-for-each-calendar-year}}

\subsubsection{Calculate rolling betas}
\label{\detokenize{mckinney_11_practice:calculate-rolling-betas}}
\sphinxAtStartPar
Calculate rolling capital asset pricing model (CAPM) betas for the MATANA stocks.

\sphinxAtStartPar
The CAPM says the risk premium on a stock depends on the risk\sphinxhyphen{}free rate, beta, and the risk premium on the market: \$E(R\_\{stock\}) = R\_f + \textbackslash{}beta\_\{stock\} \textbackslash{}times (E(R\_\{market\}) \sphinxhyphen{} R\_f)\$.
We can calculate CAPM betas as: \$\textbackslash{}beta\_\{stock\} = \textbackslash{}frac\{Cov(R\_\{stock\} \sphinxhyphen{} R\_f, R\_\{market\} \sphinxhyphen{} R\_f)\}\{Var(R\_\{market\} \sphinxhyphen{} R\_f)\}\$.


\subsubsection{Calculate rolling Sharpe Ratios}
\label{\detokenize{mckinney_11_practice:calculate-rolling-sharpe-ratios}}
\sphinxAtStartPar
Calculate rolling Sharpe Ratios for the MATANA stocks.

\sphinxAtStartPar
The Sharpe Ratio is often used to evaluate fund managers.
The Sharpe Ratio is \$SR\_i = \textbackslash{}frac\{\textbackslash{}overline\{R\_i \sphinxhyphen{} R\_f\}\}\{\textbackslash{}sigma\}\$, where \$\textbackslash{}overline\{R\_i\sphinxhyphen{}R\_f\}\$ is mean fund return relative to the risk\sphinxhyphen{}free rate over some period and \$\textbackslash{}sigma\$ is the standard deviation of \$R\_i\sphinxhyphen{}R\_f\$ over the same period.
While the Sharpe Ratio is typically used for funds, we can apply it to a single stock to test our knowledge of the \sphinxcode{\sphinxupquote{.rolling()}} method.
Calculate and plot the one\sphinxhyphen{}year rolling Sharpe Ratio for the MATANA stocks using all available daily data.


\subsubsection{Does more frequent rebalancing increase or decrease returns?}
\label{\detokenize{mckinney_11_practice:does-more-frequent-rebalancing-increase-or-decrease-returns}}
\sphinxAtStartPar
Compare decade\sphinxhyphen{}total returns for the following rebalancing frequencies:
\begin{enumerate}
\sphinxsetlistlabels{\arabic}{enumi}{enumii}{}{.}%
\item {} 
\sphinxAtStartPar
Daily rebalancing

\item {} 
\sphinxAtStartPar
Monthly rebalancing

\item {} 
\sphinxAtStartPar
Annual rebalancing

\item {} 
\sphinxAtStartPar
Decade rebalancing

\end{enumerate}

\sphinxAtStartPar
Use equally\sphinxhyphen{}weighted portfolios of industry\sphinxhyphen{}level daily returns from French’s website: \sphinxcode{\sphinxupquote{'17\_Industry\_Portfolios\_daily'}}.

\sphinxstepscope


\section{McKinney Chapter 11 \sphinxhyphen{} Practice (Monday 2:45 PM, Section 3)}
\label{\detokenize{mckinney_11_practice_03:mckinney-chapter-11-practice-monday-2-45-pm-section-3}}\label{\detokenize{mckinney_11_practice_03::doc}}

\subsection{Announcements}
\label{\detokenize{mckinney_11_practice_03:announcements}}\begin{itemize}
\item {} 
\sphinxAtStartPar
Quiz 4 \sphinxhyphen{} mean was \$87\%\$

\item {} 
\sphinxAtStartPar
Project 1 and Teammates Reviews on Friday at 11:59 PM

\end{itemize}


\subsection{Practice}
\label{\detokenize{mckinney_11_practice_03:practice}}
\begin{sphinxuseclass}{cell}\begin{sphinxVerbatimInput}

\begin{sphinxuseclass}{cell_input}
\begin{sphinxVerbatim}[commandchars=\\\{\}]
\PYG{k+kn}{import} \PYG{n+nn}{matplotlib}\PYG{n+nn}{.}\PYG{n+nn}{pyplot} \PYG{k}{as} \PYG{n+nn}{plt}
\PYG{k+kn}{import} \PYG{n+nn}{numpy} \PYG{k}{as} \PYG{n+nn}{np}
\PYG{k+kn}{import} \PYG{n+nn}{pandas} \PYG{k}{as} \PYG{n+nn}{pd}
\end{sphinxVerbatim}

\end{sphinxuseclass}\end{sphinxVerbatimInput}

\end{sphinxuseclass}
\begin{sphinxuseclass}{cell}\begin{sphinxVerbatimInput}

\begin{sphinxuseclass}{cell_input}
\begin{sphinxVerbatim}[commandchars=\\\{\}]
\PYG{o}{\PYGZpc{}}\PYG{k}{config} InlineBackend.figure\PYGZus{}format = \PYGZsq{}retina\PYGZsq{}
\PYG{o}{\PYGZpc{}}\PYG{k}{precision} 4
\PYG{n}{pd}\PYG{o}{.}\PYG{n}{options}\PYG{o}{.}\PYG{n}{display}\PYG{o}{.}\PYG{n}{float\PYGZus{}format} \PYG{o}{=} \PYG{l+s+s1}{\PYGZsq{}}\PYG{l+s+si}{\PYGZob{}:.4f\PYGZcb{}}\PYG{l+s+s1}{\PYGZsq{}}\PYG{o}{.}\PYG{n}{format}
\end{sphinxVerbatim}

\end{sphinxuseclass}\end{sphinxVerbatimInput}

\end{sphinxuseclass}
\begin{sphinxuseclass}{cell}\begin{sphinxVerbatimInput}

\begin{sphinxuseclass}{cell_input}
\begin{sphinxVerbatim}[commandchars=\\\{\}]
\PYG{k+kn}{import} \PYG{n+nn}{requests\PYGZus{}cache}
\PYG{n}{session} \PYG{o}{=} \PYG{n}{requests\PYGZus{}cache}\PYG{o}{.}\PYG{n}{CachedSession}\PYG{p}{(}\PYG{p}{)}
\PYG{k+kn}{import} \PYG{n+nn}{yfinance} \PYG{k}{as} \PYG{n+nn}{yf}
\PYG{k+kn}{import} \PYG{n+nn}{pandas\PYGZus{}datareader} \PYG{k}{as} \PYG{n+nn}{pdr}
\end{sphinxVerbatim}

\end{sphinxuseclass}\end{sphinxVerbatimInput}

\end{sphinxuseclass}

\subsubsection{Which are larger, overnight or intraday returns?}
\label{\detokenize{mckinney_11_practice_03:which-are-larger-overnight-or-intraday-returns}}
\sphinxAtStartPar
Yahoo! Finance provides easy acces to high\sphinxhyphen{}quality open, high, low, close (OHLC) and adjusted close price data.
However, Yahoo! Finance does not provide overnight or instraday returns directly.
Therefore, we need to use math to decompose daily returns into overnight and intraday returns.

\sphinxAtStartPar
Daily returns are defined as (adjusted) closing price to (adjusted) closing price returns.
Therefore, daily returns consist of overnight returns compounded with the intraday returns from the next day \$(1 + R\_\{daily\}) = (1 + R\_\{overnight\}) \textbackslash{}times (1 + R\_\{intraday\})\$ which we can rearrange to calculate overnight returns as \$\textbackslash{}frac\{1 + R\_\{daily\}\}\{1 + R\_\{intraday\}\} \sphinxhyphen{} 1 = R\_\{overnight\}\$.

\sphinxAtStartPar
We can calculate daily and intraday returns from Yahoo! Finance data as \$R\_\{daily\} = \textbackslash{}frac\{Adj\textbackslash{} Close\_\{t\} \sphinxhyphen{} Adj\textbackslash{} Close\_\{t\sphinxhyphen{}1\}\}\{Adj\textbackslash{} Close\_\{t\sphinxhyphen{}1\}\}\$ and \$R\_\{intraday\} = \textbackslash{}frac\{Close \sphinxhyphen{} Open\}\{Open\}\$.

\sphinxAtStartPar
Compare the following for the SPY ETF:
\begin{enumerate}
\sphinxsetlistlabels{\arabic}{enumi}{enumii}{}{.}%
\item {} 
\sphinxAtStartPar
Cumulative returns with all available data

\item {} 
\sphinxAtStartPar
Total returns for each calendar year

\item {} 
\sphinxAtStartPar
Total returns over rolling 252\sphinxhyphen{}trading\sphinxhyphen{}day windows

\item {} 
\sphinxAtStartPar
Total returns over rolling 12\sphinxhyphen{}months windows after calculating monthly returns

\item {} 
\sphinxAtStartPar
Sharpe Ratios for each calendar year

\end{enumerate}


\paragraph{Intraday and overnight return decomposition}
\label{\detokenize{mckinney_11_practice_03:intraday-and-overnight-return-decomposition}}
\begin{sphinxuseclass}{cell}\begin{sphinxVerbatimInput}

\begin{sphinxuseclass}{cell_input}
\begin{sphinxVerbatim}[commandchars=\\\{\}]
\PYG{n}{spy} \PYG{o}{=} \PYG{n}{yf}\PYG{o}{.}\PYG{n}{download}\PYG{p}{(}\PYG{n}{tickers}\PYG{o}{=}\PYG{l+s+s1}{\PYGZsq{}}\PYG{l+s+s1}{SPY}\PYG{l+s+s1}{\PYGZsq{}}\PYG{p}{,} \PYG{n}{progress}\PYG{o}{=}\PYG{k+kc}{False}\PYG{p}{)}
\end{sphinxVerbatim}

\end{sphinxuseclass}\end{sphinxVerbatimInput}
\begin{sphinxVerbatimOutput}

\begin{sphinxuseclass}{cell_output}
\begin{sphinxVerbatim}[commandchars=\\\{\}]
[*********************100\PYGZpc{}***********************]  1 of 1 completed
\end{sphinxVerbatim}

\end{sphinxuseclass}\end{sphinxVerbatimOutput}

\end{sphinxuseclass}
\begin{sphinxuseclass}{cell}\begin{sphinxVerbatimInput}

\begin{sphinxuseclass}{cell_input}
\begin{sphinxVerbatim}[commandchars=\\\{\}]
\PYG{n}{returns} \PYG{o}{=} \PYG{p}{(}
    \PYG{n}{spy}
    \PYG{o}{.}\PYG{n}{assign}\PYG{p}{(}
        \PYG{n}{Date} \PYG{o}{=} \PYG{k}{lambda} \PYG{n}{x}\PYG{p}{:} \PYG{n}{x}\PYG{o}{.}\PYG{n}{index}\PYG{o}{.}\PYG{n}{tz\PYGZus{}localize}\PYG{p}{(}\PYG{k+kc}{None}\PYG{p}{)}\PYG{p}{,} \PYG{c+c1}{\PYGZsh{} remove time zone from date index}
        \PYG{n}{Total} \PYG{o}{=} \PYG{k}{lambda} \PYG{n}{x}\PYG{p}{:} \PYG{n}{x}\PYG{p}{[}\PYG{l+s+s1}{\PYGZsq{}}\PYG{l+s+s1}{Adj Close}\PYG{l+s+s1}{\PYGZsq{}}\PYG{p}{]}\PYG{o}{.}\PYG{n}{pct\PYGZus{}change}\PYG{p}{(}\PYG{p}{)}\PYG{p}{,} \PYG{c+c1}{\PYGZsh{} close\PYGZhy{}to\PYGZhy{}close}
        \PYG{n}{Intraday} \PYG{o}{=} \PYG{k}{lambda} \PYG{n}{x}\PYG{p}{:} \PYG{p}{(}\PYG{n}{x}\PYG{p}{[}\PYG{l+s+s1}{\PYGZsq{}}\PYG{l+s+s1}{Close}\PYG{l+s+s1}{\PYGZsq{}}\PYG{p}{]} \PYG{o}{/} \PYG{n}{x}\PYG{p}{[}\PYG{l+s+s1}{\PYGZsq{}}\PYG{l+s+s1}{Open}\PYG{l+s+s1}{\PYGZsq{}}\PYG{p}{]}\PYG{p}{)} \PYG{o}{\PYGZhy{}} \PYG{l+m+mi}{1}\PYG{p}{,} \PYG{c+c1}{\PYGZsh{} open\PYGZhy{}to\PYGZhy{}close}
        \PYG{n}{Overnight} \PYG{o}{=} \PYG{k}{lambda} \PYG{n}{x}\PYG{p}{:} \PYG{p}{(}\PYG{p}{(}\PYG{l+m+mi}{1} \PYG{o}{+} \PYG{n}{x}\PYG{p}{[}\PYG{l+s+s1}{\PYGZsq{}}\PYG{l+s+s1}{Total}\PYG{l+s+s1}{\PYGZsq{}}\PYG{p}{]}\PYG{p}{)} \PYG{o}{/} \PYG{p}{(}\PYG{l+m+mi}{1} \PYG{o}{+} \PYG{n}{x}\PYG{p}{[}\PYG{l+s+s1}{\PYGZsq{}}\PYG{l+s+s1}{Intraday}\PYG{l+s+s1}{\PYGZsq{}}\PYG{p}{]}\PYG{p}{)}\PYG{p}{)} \PYG{o}{\PYGZhy{}} \PYG{l+m+mi}{1} \PYG{c+c1}{\PYGZsh{} remainder is close\PYGZhy{}to\PYGZhy{}open}
    \PYG{p}{)}
    \PYG{o}{.}\PYG{n}{set\PYGZus{}index}\PYG{p}{(}\PYG{l+s+s1}{\PYGZsq{}}\PYG{l+s+s1}{Date}\PYG{l+s+s1}{\PYGZsq{}}\PYG{p}{)} \PYG{c+c1}{\PYGZsh{} replace index with time\PYGZhy{}zome\PYGZhy{}less index}
    \PYG{o}{.}\PYG{n}{dropna}\PYG{p}{(}\PYG{p}{)} \PYG{c+c1}{\PYGZsh{} drop first day with incomplete returns}
    \PYG{p}{[}\PYG{p}{[}\PYG{l+s+s1}{\PYGZsq{}}\PYG{l+s+s1}{Total}\PYG{l+s+s1}{\PYGZsq{}}\PYG{p}{,} \PYG{l+s+s1}{\PYGZsq{}}\PYG{l+s+s1}{Intraday}\PYG{l+s+s1}{\PYGZsq{}}\PYG{p}{,} \PYG{l+s+s1}{\PYGZsq{}}\PYG{l+s+s1}{Overnight}\PYG{l+s+s1}{\PYGZsq{}}\PYG{p}{]}\PYG{p}{]} \PYG{c+c1}{\PYGZsh{} slice returns columns}
    \PYG{o}{.}\PYG{n}{rename\PYGZus{}axis}\PYG{p}{(}\PYG{n}{columns}\PYG{o}{=}\PYG{l+s+s1}{\PYGZsq{}}\PYG{l+s+s1}{Return}\PYG{l+s+s1}{\PYGZsq{}}\PYG{p}{)} \PYG{c+c1}{\PYGZsh{} name columns axis}
\PYG{p}{)}
\end{sphinxVerbatim}

\end{sphinxuseclass}\end{sphinxVerbatimInput}

\end{sphinxuseclass}

\subsubsection{Cumulative returns with all available data}
\label{\detokenize{mckinney_11_practice_03:cumulative-returns-with-all-available-data}}
\begin{sphinxuseclass}{cell}\begin{sphinxVerbatimInput}

\begin{sphinxuseclass}{cell_input}
\begin{sphinxVerbatim}[commandchars=\\\{\}]
\PYG{n}{returns}\PYG{o}{.}\PYG{n}{add}\PYG{p}{(}\PYG{l+m+mi}{1}\PYG{p}{)}\PYG{o}{.}\PYG{n}{cumprod}\PYG{p}{(}\PYG{p}{)}\PYG{o}{.}\PYG{n}{sub}\PYG{p}{(}\PYG{l+m+mi}{1}\PYG{p}{)}\PYG{o}{.}\PYG{n}{mul}\PYG{p}{(}\PYG{l+m+mi}{100}\PYG{p}{)}\PYG{o}{.}\PYG{n}{plot}\PYG{p}{(}\PYG{p}{)}
\PYG{n}{plt}\PYG{o}{.}\PYG{n}{ylabel}\PYG{p}{(}\PYG{l+s+s1}{\PYGZsq{}}\PYG{l+s+s1}{Cumulative Return (}\PYG{l+s+s1}{\PYGZpc{}}\PYG{l+s+s1}{)}\PYG{l+s+s1}{\PYGZsq{}}\PYG{p}{)}
\PYG{n}{plt}\PYG{o}{.}\PYG{n}{title}\PYG{p}{(}\PYG{l+s+s1}{\PYGZsq{}}\PYG{l+s+s1}{Intraday and Overnight Return Decomposition}\PYG{l+s+se}{\PYGZbs{}n}\PYG{l+s+s1}{Cumulative Returns}\PYG{l+s+s1}{\PYGZsq{}}\PYG{p}{)}
\PYG{n}{plt}\PYG{o}{.}\PYG{n}{show}\PYG{p}{(}\PYG{p}{)}
\end{sphinxVerbatim}

\end{sphinxuseclass}\end{sphinxVerbatimInput}
\begin{sphinxVerbatimOutput}

\begin{sphinxuseclass}{cell_output}
\noindent\sphinxincludegraphics{{5d313b70d512f93a5ef89e4e8703b2ff33bf3ce524a2d3d58feac73d97c54918}.png}

\end{sphinxuseclass}\end{sphinxVerbatimOutput}

\end{sphinxuseclass}
\sphinxAtStartPar
We want to calcuate total returns at several different aggregation levels.
A helper function makes this code more compact and easier to troubleshoot.

\begin{sphinxuseclass}{cell}\begin{sphinxVerbatimInput}

\begin{sphinxuseclass}{cell_input}
\begin{sphinxVerbatim}[commandchars=\\\{\}]
\PYG{k}{def} \PYG{n+nf}{totret}\PYG{p}{(}\PYG{n}{x}\PYG{p}{)}\PYG{p}{:}
    \PYG{k}{return} \PYG{p}{(}\PYG{l+m+mi}{1} \PYG{o}{+} \PYG{n}{x}\PYG{p}{)}\PYG{o}{.}\PYG{n}{prod}\PYG{p}{(}\PYG{p}{)} \PYG{o}{\PYGZhy{}} \PYG{l+m+mi}{1}
\end{sphinxVerbatim}

\end{sphinxuseclass}\end{sphinxVerbatimInput}

\end{sphinxuseclass}

\paragraph{Total returns for each calendar year}
\label{\detokenize{mckinney_11_practice_03:total-returns-for-each-calendar-year}}
\sphinxAtStartPar
First, total returns for each calendar year.
Note, the first and last years are partial (i.e., do not include about 252 trading days).

\begin{sphinxuseclass}{cell}\begin{sphinxVerbatimInput}

\begin{sphinxuseclass}{cell_input}
\begin{sphinxVerbatim}[commandchars=\\\{\}]
\PYG{n}{returns\PYGZus{}a} \PYG{o}{=} \PYG{n}{returns}\PYG{o}{.}\PYG{n}{resample}\PYG{p}{(}\PYG{l+s+s1}{\PYGZsq{}}\PYG{l+s+s1}{A}\PYG{l+s+s1}{\PYGZsq{}}\PYG{p}{,} \PYG{n}{kind}\PYG{o}{=}\PYG{l+s+s1}{\PYGZsq{}}\PYG{l+s+s1}{period}\PYG{l+s+s1}{\PYGZsq{}}\PYG{p}{)}\PYG{o}{.}\PYG{n}{apply}\PYG{p}{(}\PYG{n}{totret}\PYG{p}{)}

\PYG{n}{returns\PYGZus{}a}\PYG{o}{.}\PYG{n}{head}\PYG{p}{(}\PYG{p}{)}
\end{sphinxVerbatim}

\end{sphinxuseclass}\end{sphinxVerbatimInput}
\begin{sphinxVerbatimOutput}

\begin{sphinxuseclass}{cell_output}
\begin{sphinxVerbatim}[commandchars=\\\{\}]
Return  Total  Intraday  Overnight
Date                              
1993   0.0871   \PYGZhy{}0.0443     0.1374
1994   0.0040   \PYGZhy{}0.0774     0.0881
1995   0.3805    0.2857     0.0738
1996   0.2250    0.0236     0.1968
1997   0.3348   \PYGZhy{}0.0017     0.3370
\end{sphinxVerbatim}

\end{sphinxuseclass}\end{sphinxVerbatimOutput}

\end{sphinxuseclass}
\begin{sphinxuseclass}{cell}\begin{sphinxVerbatimInput}

\begin{sphinxuseclass}{cell_input}
\begin{sphinxVerbatim}[commandchars=\\\{\}]
\PYG{n}{returns\PYGZus{}a}\PYG{o}{.}\PYG{n}{iloc}\PYG{p}{[}\PYG{l+m+mi}{1}\PYG{p}{:}\PYG{o}{\PYGZhy{}}\PYG{l+m+mi}{1}\PYG{p}{]}\PYG{o}{.}\PYG{n}{mul}\PYG{p}{(}\PYG{l+m+mi}{100}\PYG{p}{)}\PYG{o}{.}\PYG{n}{plot}\PYG{p}{(}\PYG{n}{kind}\PYG{o}{=}\PYG{l+s+s1}{\PYGZsq{}}\PYG{l+s+s1}{bar}\PYG{l+s+s1}{\PYGZsq{}}\PYG{p}{)}
\PYG{n}{plt}\PYG{o}{.}\PYG{n}{ylabel}\PYG{p}{(}\PYG{l+s+s1}{\PYGZsq{}}\PYG{l+s+s1}{Total Return (}\PYG{l+s+s1}{\PYGZpc{}}\PYG{l+s+s1}{)}\PYG{l+s+s1}{\PYGZsq{}}\PYG{p}{)}
\PYG{n}{plt}\PYG{o}{.}\PYG{n}{title}\PYG{p}{(}\PYG{l+s+s1}{\PYGZsq{}}\PYG{l+s+s1}{Intraday and Overnight Return Decomposition}\PYG{l+s+se}{\PYGZbs{}n}\PYG{l+s+s1}{Calendar Year Total Returns}\PYG{l+s+s1}{\PYGZsq{}}\PYG{p}{)}
\PYG{n}{plt}\PYG{o}{.}\PYG{n}{show}\PYG{p}{(}\PYG{p}{)}
\end{sphinxVerbatim}

\end{sphinxuseclass}\end{sphinxVerbatimInput}
\begin{sphinxVerbatimOutput}

\begin{sphinxuseclass}{cell_output}
\noindent\sphinxincludegraphics{{aeab7e9e979fc7392abaeb58760bf5396929ac555716285bbe6441bbe45b13eb}.png}

\end{sphinxuseclass}\end{sphinxVerbatimOutput}

\end{sphinxuseclass}
\sphinxAtStartPar
We can get the same result without defining a \sphinxcode{\sphinxupquote{totret()}} function, but then we use a lambda function, which can be tedious.

\begin{sphinxuseclass}{cell}\begin{sphinxVerbatimInput}

\begin{sphinxuseclass}{cell_input}
\begin{sphinxVerbatim}[commandchars=\\\{\}]
\PYG{n}{returns\PYGZus{}a\PYGZus{}alt} \PYG{o}{=} \PYG{n}{returns}\PYG{o}{.}\PYG{n}{resample}\PYG{p}{(}\PYG{l+s+s1}{\PYGZsq{}}\PYG{l+s+s1}{A}\PYG{l+s+s1}{\PYGZsq{}}\PYG{p}{,} \PYG{n}{kind}\PYG{o}{=}\PYG{l+s+s1}{\PYGZsq{}}\PYG{l+s+s1}{period}\PYG{l+s+s1}{\PYGZsq{}}\PYG{p}{)}\PYG{o}{.}\PYG{n}{apply}\PYG{p}{(}\PYG{k}{lambda} \PYG{n}{x}\PYG{p}{:} \PYG{p}{(}\PYG{l+m+mi}{1} \PYG{o}{+} \PYG{n}{x}\PYG{p}{)}\PYG{o}{.}\PYG{n}{prod}\PYG{p}{(}\PYG{p}{)} \PYG{o}{\PYGZhy{}} \PYG{l+m+mi}{1}\PYG{p}{)}

\PYG{n}{returns\PYGZus{}a\PYGZus{}alt}\PYG{o}{.}\PYG{n}{head}\PYG{p}{(}\PYG{p}{)}
\end{sphinxVerbatim}

\end{sphinxuseclass}\end{sphinxVerbatimInput}
\begin{sphinxVerbatimOutput}

\begin{sphinxuseclass}{cell_output}
\begin{sphinxVerbatim}[commandchars=\\\{\}]
Return  Total  Intraday  Overnight
Date                              
1993   0.0871   \PYGZhy{}0.0443     0.1374
1994   0.0040   \PYGZhy{}0.0774     0.0881
1995   0.3805    0.2857     0.0738
1996   0.2250    0.0236     0.1968
1997   0.3348   \PYGZhy{}0.0017     0.3370
\end{sphinxVerbatim}

\end{sphinxuseclass}\end{sphinxVerbatimOutput}

\end{sphinxuseclass}
\begin{sphinxuseclass}{cell}\begin{sphinxVerbatimInput}

\begin{sphinxuseclass}{cell_input}
\begin{sphinxVerbatim}[commandchars=\\\{\}]
\PYG{n}{np}\PYG{o}{.}\PYG{n}{allclose}\PYG{p}{(}
    \PYG{n}{returns\PYGZus{}a}\PYG{p}{,}
    \PYG{n}{returns\PYGZus{}a\PYGZus{}alt}
\PYG{p}{)}
\end{sphinxVerbatim}

\end{sphinxuseclass}\end{sphinxVerbatimInput}
\begin{sphinxVerbatimOutput}

\begin{sphinxuseclass}{cell_output}
\begin{sphinxVerbatim}[commandchars=\\\{\}]
True
\end{sphinxVerbatim}

\end{sphinxuseclass}\end{sphinxVerbatimOutput}

\end{sphinxuseclass}
\sphinxAtStartPar
If we only want the total return for \sphinxcode{\sphinxupquote{Intraday}}, we can slice \sphinxcode{\sphinxupquote{returns}} before we \sphinxcode{\sphinxupquote{.resample()}}.

\begin{sphinxuseclass}{cell}\begin{sphinxVerbatimInput}

\begin{sphinxuseclass}{cell_input}
\begin{sphinxVerbatim}[commandchars=\\\{\}]
\PYG{n}{returns}\PYG{p}{[}\PYG{l+s+s1}{\PYGZsq{}}\PYG{l+s+s1}{Intraday}\PYG{l+s+s1}{\PYGZsq{}}\PYG{p}{]}\PYG{o}{.}\PYG{n}{resample}\PYG{p}{(}\PYG{l+s+s1}{\PYGZsq{}}\PYG{l+s+s1}{A}\PYG{l+s+s1}{\PYGZsq{}}\PYG{p}{,} \PYG{n}{kind}\PYG{o}{=}\PYG{l+s+s1}{\PYGZsq{}}\PYG{l+s+s1}{period}\PYG{l+s+s1}{\PYGZsq{}}\PYG{p}{)}\PYG{o}{.}\PYG{n}{apply}\PYG{p}{(}\PYG{n}{totret}\PYG{p}{)}\PYG{o}{.}\PYG{n}{head}\PYG{p}{(}\PYG{p}{)}
\end{sphinxVerbatim}

\end{sphinxuseclass}\end{sphinxVerbatimInput}
\begin{sphinxVerbatimOutput}

\begin{sphinxuseclass}{cell_output}
\begin{sphinxVerbatim}[commandchars=\\\{\}]
Date
1993   \PYGZhy{}0.0443
1994   \PYGZhy{}0.0774
1995    0.2857
1996    0.0236
1997   \PYGZhy{}0.0017
Freq: A\PYGZhy{}DEC, Name: Intraday, dtype: float64
\end{sphinxVerbatim}

\end{sphinxuseclass}\end{sphinxVerbatimOutput}

\end{sphinxuseclass}

\paragraph{Total returns over rolling 252\sphinxhyphen{}trading\sphinxhyphen{}day windows}
\label{\detokenize{mckinney_11_practice_03:total-returns-over-rolling-252-trading-day-windows}}
\sphinxAtStartPar
We can repeat the total return calculation for 252\sphinxhyphen{}trading\sphinxhyphen{}day rolling windows.

\begin{sphinxuseclass}{cell}\begin{sphinxVerbatimInput}

\begin{sphinxuseclass}{cell_input}
\begin{sphinxVerbatim}[commandchars=\\\{\}]
\PYG{n}{returns\PYGZus{}r} \PYG{o}{=} \PYG{n}{returns}\PYG{o}{.}\PYG{n}{rolling}\PYG{p}{(}\PYG{l+m+mi}{252}\PYG{p}{)}\PYG{o}{.}\PYG{n}{apply}\PYG{p}{(}\PYG{n}{totret}\PYG{p}{)}

\PYG{n}{returns\PYGZus{}r}\PYG{o}{.}\PYG{n}{head}\PYG{p}{(}\PYG{p}{)}
\end{sphinxVerbatim}

\end{sphinxuseclass}\end{sphinxVerbatimInput}
\begin{sphinxVerbatimOutput}

\begin{sphinxuseclass}{cell_output}
\begin{sphinxVerbatim}[commandchars=\\\{\}]
Return      Total  Intraday  Overnight
Date                                  
1993\PYGZhy{}02\PYGZhy{}01    NaN       NaN        NaN
1993\PYGZhy{}02\PYGZhy{}02    NaN       NaN        NaN
1993\PYGZhy{}02\PYGZhy{}03    NaN       NaN        NaN
1993\PYGZhy{}02\PYGZhy{}04    NaN       NaN        NaN
1993\PYGZhy{}02\PYGZhy{}05    NaN       NaN        NaN
\end{sphinxVerbatim}

\end{sphinxuseclass}\end{sphinxVerbatimOutput}

\end{sphinxuseclass}
\begin{sphinxuseclass}{cell}\begin{sphinxVerbatimInput}

\begin{sphinxuseclass}{cell_input}
\begin{sphinxVerbatim}[commandchars=\\\{\}]
\PYG{n}{returns\PYGZus{}r}\PYG{o}{.}\PYG{n}{mul}\PYG{p}{(}\PYG{l+m+mi}{100}\PYG{p}{)}\PYG{o}{.}\PYG{n}{plot}\PYG{p}{(}\PYG{p}{)}
\PYG{n}{plt}\PYG{o}{.}\PYG{n}{ylabel}\PYG{p}{(}\PYG{l+s+s1}{\PYGZsq{}}\PYG{l+s+s1}{Total Return (}\PYG{l+s+s1}{\PYGZpc{}}\PYG{l+s+s1}{)}\PYG{l+s+s1}{\PYGZsq{}}\PYG{p}{)}
\PYG{n}{plt}\PYG{o}{.}\PYG{n}{title}\PYG{p}{(}\PYG{l+s+s1}{\PYGZsq{}}\PYG{l+s+s1}{Intraday and Overnight Return Decomposition}\PYG{l+s+se}{\PYGZbs{}n}\PYG{l+s+s1}{252\PYGZhy{}Trading\PYGZhy{}Day Rolling Windows}\PYG{l+s+s1}{\PYGZsq{}}\PYG{p}{)}
\PYG{n}{plt}\PYG{o}{.}\PYG{n}{show}\PYG{p}{(}\PYG{p}{)}
\end{sphinxVerbatim}

\end{sphinxuseclass}\end{sphinxVerbatimInput}
\begin{sphinxVerbatimOutput}

\begin{sphinxuseclass}{cell_output}
\noindent\sphinxincludegraphics{{b81e52c04650c8539929407e5f87ab6dbfc7c1fedb554277c6ccbbf5a53adcf3}.png}

\end{sphinxuseclass}\end{sphinxVerbatimOutput}

\end{sphinxuseclass}

\paragraph{Total returns over rolling 12\sphinxhyphen{}months windows after calculating monthly returns}
\label{\detokenize{mckinney_11_practice_03:total-returns-over-rolling-12-months-windows-after-calculating-monthly-returns}}
\sphinxAtStartPar
We can chain this operation!
Note the plot has the same general appearance, but is less noisy because we aggregate to monthly total returns first!

\begin{sphinxuseclass}{cell}\begin{sphinxVerbatimInput}

\begin{sphinxuseclass}{cell_input}
\begin{sphinxVerbatim}[commandchars=\\\{\}]
\PYG{n}{returns\PYGZus{}mr} \PYG{o}{=} \PYG{p}{(}
    \PYG{n}{returns}
    \PYG{o}{.}\PYG{n}{resample}\PYG{p}{(}\PYG{n}{rule}\PYG{o}{=}\PYG{l+s+s1}{\PYGZsq{}}\PYG{l+s+s1}{M}\PYG{l+s+s1}{\PYGZsq{}}\PYG{p}{,} \PYG{n}{kind}\PYG{o}{=}\PYG{l+s+s1}{\PYGZsq{}}\PYG{l+s+s1}{period}\PYG{l+s+s1}{\PYGZsq{}}\PYG{p}{)} \PYG{c+c1}{\PYGZsh{} aggregate from daily to monthly}
    \PYG{o}{.}\PYG{n}{apply}\PYG{p}{(}\PYG{n}{totret}\PYG{p}{)} \PYG{c+c1}{\PYGZsh{} ...calculate total returns}
    \PYG{o}{.}\PYG{n}{rolling}\PYG{p}{(}\PYG{l+m+mi}{12}\PYG{p}{)} \PYG{c+c1}{\PYGZsh{} aggregate into 12\PYGZhy{}month rolling windows}
    \PYG{o}{.}\PYG{n}{apply}\PYG{p}{(}\PYG{n}{totret}\PYG{p}{)} \PYG{c+c1}{\PYGZsh{} ...calculate total returns}
\PYG{p}{)}
\end{sphinxVerbatim}

\end{sphinxuseclass}\end{sphinxVerbatimInput}

\end{sphinxuseclass}
\begin{sphinxuseclass}{cell}\begin{sphinxVerbatimInput}

\begin{sphinxuseclass}{cell_input}
\begin{sphinxVerbatim}[commandchars=\\\{\}]
\PYG{n}{returns\PYGZus{}mr}\PYG{o}{.}\PYG{n}{mul}\PYG{p}{(}\PYG{l+m+mi}{100}\PYG{p}{)}\PYG{o}{.}\PYG{n}{plot}\PYG{p}{(}\PYG{p}{)}
\PYG{n}{plt}\PYG{o}{.}\PYG{n}{ylabel}\PYG{p}{(}\PYG{l+s+s1}{\PYGZsq{}}\PYG{l+s+s1}{Total Return (}\PYG{l+s+s1}{\PYGZpc{}}\PYG{l+s+s1}{)}\PYG{l+s+s1}{\PYGZsq{}}\PYG{p}{)}
\PYG{n}{plt}\PYG{o}{.}\PYG{n}{title}\PYG{p}{(}\PYG{l+s+s1}{\PYGZsq{}}\PYG{l+s+s1}{Intraday and Overnight Return Decomposition}\PYG{l+s+se}{\PYGZbs{}n}\PYG{l+s+s1}{12\PYGZhy{}Month Rolling Windows}\PYG{l+s+s1}{\PYGZsq{}}\PYG{p}{)}
\PYG{n}{plt}\PYG{o}{.}\PYG{n}{show}\PYG{p}{(}\PYG{p}{)}
\end{sphinxVerbatim}

\end{sphinxuseclass}\end{sphinxVerbatimInput}
\begin{sphinxVerbatimOutput}

\begin{sphinxuseclass}{cell_output}
\noindent\sphinxincludegraphics{{1d74ede469744e32262f36e9aa684004d2cba25347fac629fffa1ce961b3feec}.png}

\end{sphinxuseclass}\end{sphinxVerbatimOutput}

\end{sphinxuseclass}

\paragraph{Sharpe Ratios for each calendar year}
\label{\detokenize{mckinney_11_practice_03:sharpe-ratios-for-each-calendar-year}}
\sphinxAtStartPar
We need the risk\sphinxhyphen{}free rate \sphinxcode{\sphinxupquote{RF}} from Ken French’s daily benchmark factors.

\begin{sphinxuseclass}{cell}\begin{sphinxVerbatimInput}

\begin{sphinxuseclass}{cell_input}
\begin{sphinxVerbatim}[commandchars=\\\{\}]
\PYG{n}{ff} \PYG{o}{=} \PYG{p}{(}
    \PYG{n}{pdr}\PYG{o}{.}\PYG{n}{DataReader}\PYG{p}{(}
        \PYG{n}{name}\PYG{o}{=}\PYG{l+s+s1}{\PYGZsq{}}\PYG{l+s+s1}{F\PYGZhy{}F\PYGZus{}Research\PYGZus{}Data\PYGZus{}Factors\PYGZus{}daily}\PYG{l+s+s1}{\PYGZsq{}}\PYG{p}{,}
        \PYG{n}{data\PYGZus{}source}\PYG{o}{=}\PYG{l+s+s1}{\PYGZsq{}}\PYG{l+s+s1}{famafrench}\PYG{l+s+s1}{\PYGZsq{}}\PYG{p}{,}
        \PYG{n}{start}\PYG{o}{=}\PYG{l+s+s1}{\PYGZsq{}}\PYG{l+s+s1}{1900}\PYG{l+s+s1}{\PYGZsq{}}\PYG{p}{,}
        \PYG{n}{session}\PYG{o}{=}\PYG{n}{session}
    \PYG{p}{)}
    \PYG{p}{[}\PYG{l+m+mi}{0}\PYG{p}{]}
    \PYG{o}{.}\PYG{n}{div}\PYG{p}{(}\PYG{l+m+mi}{100}\PYG{p}{)}
\PYG{p}{)}
\end{sphinxVerbatim}

\end{sphinxuseclass}\end{sphinxVerbatimInput}

\end{sphinxuseclass}
\sphinxAtStartPar
I think this easiest (and fastest approach) is to:
\begin{enumerate}
\sphinxsetlistlabels{\arabic}{enumi}{enumii}{}{.}%
\item {} 
\sphinxAtStartPar
Calculate a data frame of excess returns

\item {} 
\sphinxAtStartPar
Use an anonymous (lambda) function

\end{enumerate}

\begin{sphinxuseclass}{cell}\begin{sphinxVerbatimInput}

\begin{sphinxuseclass}{cell_input}
\begin{sphinxVerbatim}[commandchars=\\\{\}]
\PYG{n}{returns\PYGZus{}excess} \PYG{o}{=} \PYG{p}{(}
    \PYG{n}{returns}
    \PYG{o}{.}\PYG{n}{sub}\PYG{p}{(}\PYG{n}{ff}\PYG{p}{[}\PYG{l+s+s1}{\PYGZsq{}}\PYG{l+s+s1}{RF}\PYG{l+s+s1}{\PYGZsq{}}\PYG{p}{]}\PYG{p}{,} \PYG{n}{axis}\PYG{o}{=}\PYG{l+m+mi}{0}\PYG{p}{)} \PYG{c+c1}{\PYGZsh{} note axis=0 for column\PYGZhy{}wise subtraction}
    \PYG{o}{.}\PYG{n}{dropna}\PYG{p}{(}\PYG{p}{)}
\PYG{p}{)}
\end{sphinxVerbatim}

\end{sphinxuseclass}\end{sphinxVerbatimInput}

\end{sphinxuseclass}
\begin{sphinxuseclass}{cell}\begin{sphinxVerbatimInput}

\begin{sphinxuseclass}{cell_input}
\begin{sphinxVerbatim}[commandchars=\\\{\}]
\PYG{n}{sharpes\PYGZus{}a} \PYG{o}{=} \PYG{p}{(}
    \PYG{n}{returns\PYGZus{}excess}
    \PYG{o}{.}\PYG{n}{resample}\PYG{p}{(}\PYG{n}{rule}\PYG{o}{=}\PYG{l+s+s1}{\PYGZsq{}}\PYG{l+s+s1}{A}\PYG{l+s+s1}{\PYGZsq{}}\PYG{p}{,} \PYG{n}{kind}\PYG{o}{=}\PYG{l+s+s1}{\PYGZsq{}}\PYG{l+s+s1}{period}\PYG{l+s+s1}{\PYGZsq{}}\PYG{p}{)}
    \PYG{o}{.}\PYG{n}{apply}\PYG{p}{(}\PYG{k}{lambda} \PYG{n}{x}\PYG{p}{:} \PYG{n}{np}\PYG{o}{.}\PYG{n}{sqrt}\PYG{p}{(}\PYG{l+m+mi}{252}\PYG{p}{)} \PYG{o}{*} \PYG{n}{x}\PYG{o}{.}\PYG{n}{mean}\PYG{p}{(}\PYG{p}{)} \PYG{o}{/} \PYG{n}{x}\PYG{o}{.}\PYG{n}{std}\PYG{p}{(}\PYG{p}{)}\PYG{p}{)}
\PYG{p}{)}

\PYG{n}{sharpes\PYGZus{}a}\PYG{o}{.}\PYG{n}{head}\PYG{p}{(}\PYG{p}{)}
\end{sphinxVerbatim}

\end{sphinxuseclass}\end{sphinxVerbatimInput}
\begin{sphinxVerbatimOutput}

\begin{sphinxuseclass}{cell_output}
\begin{sphinxVerbatim}[commandchars=\\\{\}]
Return   Total  Intraday  Overnight
Date                               
1993    0.7233   \PYGZhy{}0.9172     2.8057
1994   \PYGZhy{}0.2732   \PYGZhy{}1.1871     0.8819
1995    3.1845    2.5275     0.4197
1996    1.2247   \PYGZhy{}0.1691     1.6658
1997    1.2796   \PYGZhy{}0.1781     2.5564
\end{sphinxVerbatim}

\end{sphinxuseclass}\end{sphinxVerbatimOutput}

\end{sphinxuseclass}
\begin{sphinxuseclass}{cell}\begin{sphinxVerbatimInput}

\begin{sphinxuseclass}{cell_input}
\begin{sphinxVerbatim}[commandchars=\\\{\}]
\PYG{n}{sharpes\PYGZus{}a}\PYG{o}{.}\PYG{n}{iloc}\PYG{p}{[}\PYG{l+m+mi}{1}\PYG{p}{:}\PYG{o}{\PYGZhy{}}\PYG{l+m+mi}{1}\PYG{p}{]}\PYG{o}{.}\PYG{n}{plot}\PYG{p}{(}\PYG{n}{kind}\PYG{o}{=}\PYG{l+s+s1}{\PYGZsq{}}\PYG{l+s+s1}{bar}\PYG{l+s+s1}{\PYGZsq{}}\PYG{p}{)}
\PYG{n}{plt}\PYG{o}{.}\PYG{n}{ylabel}\PYG{p}{(}\PYG{l+s+s1}{\PYGZsq{}}\PYG{l+s+s1}{Sharpe Ratio}\PYG{l+s+s1}{\PYGZsq{}}\PYG{p}{)}
\PYG{n}{plt}\PYG{o}{.}\PYG{n}{title}\PYG{p}{(}\PYG{l+s+s1}{\PYGZsq{}}\PYG{l+s+s1}{Intraday and Overnight Return Decomposition}\PYG{l+s+se}{\PYGZbs{}n}\PYG{l+s+s1}{Calendar Year Sharpe Ratios}\PYG{l+s+s1}{\PYGZsq{}}\PYG{p}{)}
\PYG{n}{plt}\PYG{o}{.}\PYG{n}{show}\PYG{p}{(}\PYG{p}{)}
\end{sphinxVerbatim}

\end{sphinxuseclass}\end{sphinxVerbatimInput}
\begin{sphinxVerbatimOutput}

\begin{sphinxuseclass}{cell_output}
\noindent\sphinxincludegraphics{{f8172c21e8cd023d83053cd9bd88c6b3e49803b9a3ee0adf753555311c6f75fe}.png}

\end{sphinxuseclass}\end{sphinxVerbatimOutput}

\end{sphinxuseclass}

\subsubsection{Calculate rolling betas}
\label{\detokenize{mckinney_11_practice_03:calculate-rolling-betas}}
\sphinxAtStartPar
Calculate rolling capital asset pricing model (CAPM) betas for the MATANA stocks.

\sphinxAtStartPar
The CAPM says the risk premium on a stock depends on the risk\sphinxhyphen{}free rate, beta, and the risk premium on the market: \$E(R\_\{stock\}) = R\_f + \textbackslash{}beta\_\{stock\} \textbackslash{}times (E(R\_\{market\}) \sphinxhyphen{} R\_f)\$.
We can calculate CAPM betas as: \$\textbackslash{}beta\_\{stock\} = \textbackslash{}frac\{Cov(R\_\{stock\} \sphinxhyphen{} R\_f, R\_\{market\} \sphinxhyphen{} R\_f)\}\{Var(R\_\{market\} \sphinxhyphen{} R\_f)\}\$.

\begin{sphinxuseclass}{cell}\begin{sphinxVerbatimInput}

\begin{sphinxuseclass}{cell_input}
\begin{sphinxVerbatim}[commandchars=\\\{\}]
\PYG{n}{matana} \PYG{o}{=} \PYG{p}{(}
    \PYG{n}{yf}\PYG{o}{.}\PYG{n}{download}\PYG{p}{(}\PYG{n}{tickers}\PYG{o}{=}\PYG{l+s+s1}{\PYGZsq{}}\PYG{l+s+s1}{MSFT AAPL TSLA AMZN NVDA GOOG}\PYG{l+s+s1}{\PYGZsq{}}\PYG{p}{,} \PYG{n}{progress}\PYG{o}{=}\PYG{k+kc}{False}\PYG{p}{)}
    \PYG{o}{.}\PYG{n}{assign}\PYG{p}{(}
        \PYG{n}{Date} \PYG{o}{=} \PYG{k}{lambda} \PYG{n}{x}\PYG{p}{:} \PYG{n}{x}\PYG{o}{.}\PYG{n}{index}\PYG{o}{.}\PYG{n}{tz\PYGZus{}localize}\PYG{p}{(}\PYG{k+kc}{None}\PYG{p}{)}\PYG{p}{,}
    \PYG{p}{)}
    \PYG{o}{.}\PYG{n}{set\PYGZus{}index}\PYG{p}{(}\PYG{l+s+s1}{\PYGZsq{}}\PYG{l+s+s1}{Date}\PYG{l+s+s1}{\PYGZsq{}}\PYG{p}{)}
    \PYG{p}{[}\PYG{l+s+s1}{\PYGZsq{}}\PYG{l+s+s1}{Adj Close}\PYG{l+s+s1}{\PYGZsq{}}\PYG{p}{]}
    \PYG{o}{.}\PYG{n}{pct\PYGZus{}change}\PYG{p}{(}\PYG{p}{)}
    \PYG{o}{.}\PYG{n}{rename\PYGZus{}axis}\PYG{p}{(}\PYG{n}{columns}\PYG{o}{=}\PYG{l+s+s1}{\PYGZsq{}}\PYG{l+s+s1}{Ticker}\PYG{l+s+s1}{\PYGZsq{}}\PYG{p}{)}
\PYG{p}{)}

\PYG{n}{matana}\PYG{o}{.}\PYG{n}{head}\PYG{p}{(}\PYG{p}{)}
\end{sphinxVerbatim}

\end{sphinxuseclass}\end{sphinxVerbatimInput}
\begin{sphinxVerbatimOutput}

\begin{sphinxuseclass}{cell_output}
\begin{sphinxVerbatim}[commandchars=\\\{\}]
[*********************100\PYGZpc{}***********************]  6 of 6 completed
\end{sphinxVerbatim}

\begin{sphinxVerbatim}[commandchars=\\\{\}]
Ticker        AAPL  AMZN  GOOG  MSFT  NVDA  TSLA
Date                                            
1980\PYGZhy{}12\PYGZhy{}12     NaN   NaN   NaN   NaN   NaN   NaN
1980\PYGZhy{}12\PYGZhy{}15 \PYGZhy{}0.0522   NaN   NaN   NaN   NaN   NaN
1980\PYGZhy{}12\PYGZhy{}16 \PYGZhy{}0.0734   NaN   NaN   NaN   NaN   NaN
1980\PYGZhy{}12\PYGZhy{}17  0.0248   NaN   NaN   NaN   NaN   NaN
1980\PYGZhy{}12\PYGZhy{}18  0.0290   NaN   NaN   NaN   NaN   NaN
\end{sphinxVerbatim}

\end{sphinxuseclass}\end{sphinxVerbatimOutput}

\end{sphinxuseclass}
\begin{sphinxuseclass}{cell}\begin{sphinxVerbatimInput}

\begin{sphinxuseclass}{cell_input}
\begin{sphinxVerbatim}[commandchars=\\\{\}]
\PYG{k}{def} \PYG{n+nf}{beta}\PYG{p}{(}\PYG{n}{ri}\PYG{p}{,} \PYG{n}{rf}\PYG{o}{=}\PYG{n}{ff}\PYG{p}{[}\PYG{l+s+s1}{\PYGZsq{}}\PYG{l+s+s1}{RF}\PYG{l+s+s1}{\PYGZsq{}}\PYG{p}{]}\PYG{p}{,} \PYG{n}{rm\PYGZus{}rf}\PYG{o}{=}\PYG{n}{ff}\PYG{p}{[}\PYG{l+s+s1}{\PYGZsq{}}\PYG{l+s+s1}{Mkt\PYGZhy{}RF}\PYG{l+s+s1}{\PYGZsq{}}\PYG{p}{]}\PYG{p}{)}\PYG{p}{:}
    \PYG{n}{ri\PYGZus{}rf} \PYG{o}{=} \PYG{n}{ri}\PYG{o}{.}\PYG{n}{sub}\PYG{p}{(}\PYG{n}{rf}\PYG{p}{)}\PYG{o}{.}\PYG{n}{dropna}\PYG{p}{(}\PYG{p}{)}
    \PYG{k}{return} \PYG{n}{ri\PYGZus{}rf}\PYG{o}{.}\PYG{n}{cov}\PYG{p}{(}\PYG{n}{rm\PYGZus{}rf}\PYG{p}{)} \PYG{o}{/} \PYG{n}{rm\PYGZus{}rf}\PYG{o}{.}\PYG{n}{loc}\PYG{p}{[}\PYG{n}{ri\PYGZus{}rf}\PYG{o}{.}\PYG{n}{index}\PYG{p}{]}\PYG{o}{.}\PYG{n}{var}\PYG{p}{(}\PYG{p}{)}
\end{sphinxVerbatim}

\end{sphinxuseclass}\end{sphinxVerbatimInput}

\end{sphinxuseclass}
\sphinxAtStartPar
A few notes:
\begin{enumerate}
\sphinxsetlistlabels{\arabic}{enumi}{enumii}{}{.}%
\item {} 
\sphinxAtStartPar
Here the \sphinxcode{\sphinxupquote{beta()}} function is too smart for the \sphinxcode{\sphinxupquote{.rolling()}} method.
Our \sphinxcode{\sphinxupquote{beta()}} function matches stock and market data, but may not use 252 trading days at the end of the sample.
For example, in February 2023, we can slice a 252\sphinxhyphen{}trading\sphinxhyphen{}day window from the \sphinxcode{\sphinxupquote{matana}} data frame.
However, the \sphinxcode{\sphinxupquote{beta()}} function does not use January and Februrary 2023 data because they are missing from the \sphinxcode{\sphinxupquote{ff}} data frame.
The easiest fix is to slice the \sphinxcode{\sphinxupquote{matana}} data frame to remove 2023 data before we apply the \sphinxcode{\sphinxupquote{.rolling()}} method.

\item {} 
\sphinxAtStartPar
The \sphinxcode{\sphinxupquote{.apply()}} method and \sphinxcode{\sphinxupquote{beta()}} function are \sphinxstyleemphasis{very slow} because they slice inefficiently!
We will write faster—but uglier—code below.

\end{enumerate}

\begin{sphinxuseclass}{cell}\begin{sphinxVerbatimInput}

\begin{sphinxuseclass}{cell_input}
\begin{sphinxVerbatim}[commandchars=\\\{\}]
\PYG{o}{\PYGZpc{}\PYGZpc{}time}
\PYG{n}{betas\PYGZus{}1} \PYG{o}{=} \PYG{n}{matana}\PYG{o}{.}\PYG{n}{dropna}\PYG{p}{(}\PYG{p}{)}\PYG{o}{.}\PYG{n}{loc}\PYG{p}{[}\PYG{p}{:}\PYG{l+s+s1}{\PYGZsq{}}\PYG{l+s+s1}{2022}\PYG{l+s+s1}{\PYGZsq{}}\PYG{p}{]}\PYG{o}{.}\PYG{n}{rolling}\PYG{p}{(}\PYG{l+m+mi}{252}\PYG{p}{)}\PYG{o}{.}\PYG{n}{apply}\PYG{p}{(}\PYG{n}{beta}\PYG{p}{)}
\end{sphinxVerbatim}

\end{sphinxuseclass}\end{sphinxVerbatimInput}
\begin{sphinxVerbatimOutput}

\begin{sphinxuseclass}{cell_output}
\begin{sphinxVerbatim}[commandchars=\\\{\}]
CPU times: user 14.1 s, sys: 3.57 ms, total: 14.1 s
Wall time: 14.1 s
\end{sphinxVerbatim}

\end{sphinxuseclass}\end{sphinxVerbatimOutput}

\end{sphinxuseclass}
\sphinxAtStartPar
Here is the fast solution, which uses the optimized \sphinxcode{\sphinxupquote{.cov()}} and \sphinxcode{\sphinxupquote{.var()}} to calculate the numerator and denominator for \$\textbackslash{}beta\$.

\begin{sphinxuseclass}{cell}\begin{sphinxVerbatimInput}

\begin{sphinxuseclass}{cell_input}
\begin{sphinxVerbatim}[commandchars=\\\{\}]
\PYG{o}{\PYGZpc{}\PYGZpc{}time}
\PYG{n}{cov\PYGZus{}2} \PYG{o}{=} \PYG{n}{matana}\PYG{o}{.}\PYG{n}{dropna}\PYG{p}{(}\PYG{p}{)}\PYG{o}{.}\PYG{n}{loc}\PYG{p}{[}\PYG{p}{:}\PYG{l+s+s1}{\PYGZsq{}}\PYG{l+s+s1}{2022}\PYG{l+s+s1}{\PYGZsq{}}\PYG{p}{]}\PYG{o}{.}\PYG{n}{sub}\PYG{p}{(}\PYG{n}{ff}\PYG{p}{[}\PYG{l+s+s1}{\PYGZsq{}}\PYG{l+s+s1}{RF}\PYG{l+s+s1}{\PYGZsq{}}\PYG{p}{]}\PYG{p}{,} \PYG{n}{axis}\PYG{o}{=}\PYG{l+m+mi}{0}\PYG{p}{)}\PYG{o}{.}\PYG{n}{rolling}\PYG{p}{(}\PYG{l+m+mi}{252}\PYG{p}{)}\PYG{o}{.}\PYG{n}{cov}\PYG{p}{(}\PYG{n}{ff}\PYG{p}{[}\PYG{l+s+s1}{\PYGZsq{}}\PYG{l+s+s1}{Mkt\PYGZhy{}RF}\PYG{l+s+s1}{\PYGZsq{}}\PYG{p}{]}\PYG{p}{)}
\PYG{n}{var\PYGZus{}2} \PYG{o}{=} \PYG{n}{ff}\PYG{p}{[}\PYG{l+s+s1}{\PYGZsq{}}\PYG{l+s+s1}{Mkt\PYGZhy{}RF}\PYG{l+s+s1}{\PYGZsq{}}\PYG{p}{]}\PYG{o}{.}\PYG{n}{rolling}\PYG{p}{(}\PYG{l+m+mi}{252}\PYG{p}{)}\PYG{o}{.}\PYG{n}{var}\PYG{p}{(}\PYG{p}{)}
\PYG{n}{betas\PYGZus{}2} \PYG{o}{=} \PYG{n}{cov\PYGZus{}2}\PYG{o}{.}\PYG{n}{div}\PYG{p}{(}\PYG{n}{var\PYGZus{}2}\PYG{p}{,} \PYG{n}{axis}\PYG{o}{=}\PYG{l+m+mi}{0}\PYG{p}{)}
\end{sphinxVerbatim}

\end{sphinxuseclass}\end{sphinxVerbatimInput}
\begin{sphinxVerbatimOutput}

\begin{sphinxuseclass}{cell_output}
\begin{sphinxVerbatim}[commandchars=\\\{\}]
CPU times: user 9.54 ms, sys: 0 ns, total: 9.54 ms
Wall time: 9.03 ms
\end{sphinxVerbatim}

\end{sphinxuseclass}\end{sphinxVerbatimOutput}

\end{sphinxuseclass}
\sphinxAtStartPar
These solutions are identical!
Even though \sphinxcode{\sphinxupquote{beta\_2}} is 1000 times faster than \sphinxcode{\sphinxupquote{beta\_1}}, you should use the approach that makes the most sense to you!

\begin{sphinxuseclass}{cell}\begin{sphinxVerbatimInput}

\begin{sphinxuseclass}{cell_input}
\begin{sphinxVerbatim}[commandchars=\\\{\}]
\PYG{n}{np}\PYG{o}{.}\PYG{n}{allclose}\PYG{p}{(}\PYG{n}{betas\PYGZus{}1}\PYG{o}{.}\PYG{n}{dropna}\PYG{p}{(}\PYG{p}{)}\PYG{p}{,} \PYG{n}{betas\PYGZus{}2}\PYG{o}{.}\PYG{n}{dropna}\PYG{p}{(}\PYG{p}{)}\PYG{p}{)}
\end{sphinxVerbatim}

\end{sphinxuseclass}\end{sphinxVerbatimInput}
\begin{sphinxVerbatimOutput}

\begin{sphinxuseclass}{cell_output}
\begin{sphinxVerbatim}[commandchars=\\\{\}]
True
\end{sphinxVerbatim}

\end{sphinxuseclass}\end{sphinxVerbatimOutput}

\end{sphinxuseclass}
\begin{sphinxuseclass}{cell}\begin{sphinxVerbatimInput}

\begin{sphinxuseclass}{cell_input}
\begin{sphinxVerbatim}[commandchars=\\\{\}]
\PYG{n}{betas\PYGZus{}1}\PYG{o}{.}\PYG{n}{plot}\PYG{p}{(}\PYG{p}{)}
\PYG{n}{plt}\PYG{o}{.}\PYG{n}{ylabel}\PYG{p}{(}\PYG{l+s+s1}{\PYGZsq{}}\PYG{l+s+s1}{Beta}\PYG{l+s+s1}{\PYGZsq{}}\PYG{p}{)}
\PYG{n}{plt}\PYG{o}{.}\PYG{n}{title}\PYG{p}{(}\PYG{l+s+s1}{\PYGZsq{}}\PYG{l+s+s1}{Betas from Daily Returns}\PYG{l+s+se}{\PYGZbs{}n}\PYG{l+s+s1}{252\PYGZhy{}Trading\PYGZhy{}Day Rolling Windows}\PYG{l+s+s1}{\PYGZsq{}}\PYG{p}{)}
\PYG{n}{plt}\PYG{o}{.}\PYG{n}{show}\PYG{p}{(}\PYG{p}{)}
\end{sphinxVerbatim}

\end{sphinxuseclass}\end{sphinxVerbatimInput}
\begin{sphinxVerbatimOutput}

\begin{sphinxuseclass}{cell_output}
\noindent\sphinxincludegraphics{{e22791d627abdd8be9438a05436284c4dabaf66d191aa7299a1af0f6c87d4441}.png}

\end{sphinxuseclass}\end{sphinxVerbatimOutput}

\end{sphinxuseclass}

\subsubsection{Calculate rolling Sharpe Ratios}
\label{\detokenize{mckinney_11_practice_03:calculate-rolling-sharpe-ratios}}
\sphinxAtStartPar
Calculate rolling Sharpe Ratios for the MATANA stocks.

\sphinxAtStartPar
The Sharpe Ratio is often used to evaluate fund managers.
The Sharpe Ratio is \$SR\_i = \textbackslash{}frac\{\textbackslash{}overline\{R\_i \sphinxhyphen{} R\_f\}\}\{\textbackslash{}sigma\}\$, where \$\textbackslash{}overline\{R\_i\sphinxhyphen{}R\_f\}\$ is mean fund return relative to the risk\sphinxhyphen{}free rate over some period and \$\textbackslash{}sigma\$ is the standard deviation of \$R\_i\sphinxhyphen{}R\_f\$ over the same period.
While the Sharpe Ratio is typically used for funds, we can apply it to a single stock to test our knowledge of the \sphinxcode{\sphinxupquote{.rolling()}} method.
Calculate and plot the one\sphinxhyphen{}year rolling Sharpe Ratio for the MATANA stocks using all available daily data.

\begin{sphinxuseclass}{cell}\begin{sphinxVerbatimInput}

\begin{sphinxuseclass}{cell_input}
\begin{sphinxVerbatim}[commandchars=\\\{\}]
\PYG{k}{def} \PYG{n+nf}{sharpe}\PYG{p}{(}\PYG{n}{ri}\PYG{p}{,} \PYG{n}{rf}\PYG{o}{=}\PYG{n}{ff}\PYG{p}{[}\PYG{l+s+s1}{\PYGZsq{}}\PYG{l+s+s1}{RF}\PYG{l+s+s1}{\PYGZsq{}}\PYG{p}{]}\PYG{p}{,} \PYG{n}{ann\PYGZus{}fac}\PYG{o}{=}\PYG{n}{np}\PYG{o}{.}\PYG{n}{sqrt}\PYG{p}{(}\PYG{l+m+mi}{252}\PYG{p}{)}\PYG{p}{)}\PYG{p}{:}
    \PYG{n}{ri\PYGZus{}rf} \PYG{o}{=} \PYG{n}{ri}\PYG{o}{.}\PYG{n}{sub}\PYG{p}{(}\PYG{n}{rf}\PYG{p}{)}\PYG{o}{.}\PYG{n}{dropna}\PYG{p}{(}\PYG{p}{)}
    \PYG{k}{return} \PYG{n}{ann\PYGZus{}fac} \PYG{o}{*} \PYG{n}{ri\PYGZus{}rf}\PYG{o}{.}\PYG{n}{mean}\PYG{p}{(}\PYG{p}{)} \PYG{o}{/} \PYG{n}{ri\PYGZus{}rf}\PYG{o}{.}\PYG{n}{std}\PYG{p}{(}\PYG{p}{)}
\end{sphinxVerbatim}

\end{sphinxuseclass}\end{sphinxVerbatimInput}

\end{sphinxuseclass}
\sphinxAtStartPar
See the discussion in the beta exercise above about how we can end up with a result based on fewer than 252 trading days of data because the \sphinxcode{\sphinxupquote{sharpe()}} function matches the \sphinxcode{\sphinxupquote{matana}} and \sphinxcode{\sphinxupquote{ff}} data frames \sphinxstyleemphasis{after} the \sphinxcode{\sphinxupquote{.rolling()}} mehtod slices a 252 trading day window.
The simplest solution is the remove the 2023 data from the \sphinxcode{\sphinxupquote{matana}} data frame before we apply the \sphinxcode{\sphinxupquote{.rolling()}} method.

\begin{sphinxuseclass}{cell}\begin{sphinxVerbatimInput}

\begin{sphinxuseclass}{cell_input}
\begin{sphinxVerbatim}[commandchars=\\\{\}]
\PYG{o}{\PYGZpc{}\PYGZpc{}time}
\PYG{n}{sharpes\PYGZus{}1} \PYG{o}{=} \PYG{n}{matana}\PYG{o}{.}\PYG{n}{dropna}\PYG{p}{(}\PYG{p}{)}\PYG{o}{.}\PYG{n}{loc}\PYG{p}{[}\PYG{p}{:}\PYG{l+s+s1}{\PYGZsq{}}\PYG{l+s+s1}{2022}\PYG{l+s+s1}{\PYGZsq{}}\PYG{p}{]}\PYG{o}{.}\PYG{n}{rolling}\PYG{p}{(}\PYG{l+m+mi}{252}\PYG{p}{)}\PYG{o}{.}\PYG{n}{apply}\PYG{p}{(}\PYG{n}{sharpe}\PYG{p}{)}
\end{sphinxVerbatim}

\end{sphinxuseclass}\end{sphinxVerbatimInput}
\begin{sphinxVerbatimOutput}

\begin{sphinxuseclass}{cell_output}
\begin{sphinxVerbatim}[commandchars=\\\{\}]
CPU times: user 6.35 s, sys: 0 ns, total: 6.35 s
Wall time: 6.35 s
\end{sphinxVerbatim}

\end{sphinxuseclass}\end{sphinxVerbatimOutput}

\end{sphinxuseclass}
\sphinxAtStartPar
Here is the fast solution, which:
\begin{enumerate}
\sphinxsetlistlabels{\arabic}{enumi}{enumii}{}{.}%
\item {} 
\sphinxAtStartPar
Uses the \sphinxcode{\sphinxupquote{.sub()}} method to calculate excess returns

\item {} 
\sphinxAtStartPar
Then uses the optimized \sphinxcode{\sphinxupquote{.mean()}} and \sphinxcode{\sphinxupquote{.std()}} methods to calculate the numerator and denominator for the Sharpe Ratios

\end{enumerate}

\begin{sphinxuseclass}{cell}\begin{sphinxVerbatimInput}

\begin{sphinxuseclass}{cell_input}
\begin{sphinxVerbatim}[commandchars=\\\{\}]
\PYG{n}{mean\PYGZus{}2} \PYG{o}{=} \PYG{n}{matana}\PYG{o}{.}\PYG{n}{dropna}\PYG{p}{(}\PYG{p}{)}\PYG{o}{.}\PYG{n}{loc}\PYG{p}{[}\PYG{p}{:}\PYG{l+s+s1}{\PYGZsq{}}\PYG{l+s+s1}{2022}\PYG{l+s+s1}{\PYGZsq{}}\PYG{p}{]}\PYG{o}{.}\PYG{n}{sub}\PYG{p}{(}\PYG{n}{ff}\PYG{p}{[}\PYG{l+s+s1}{\PYGZsq{}}\PYG{l+s+s1}{RF}\PYG{l+s+s1}{\PYGZsq{}}\PYG{p}{]}\PYG{p}{,} \PYG{n}{axis}\PYG{o}{=}\PYG{l+m+mi}{0}\PYG{p}{)}\PYG{o}{.}\PYG{n}{rolling}\PYG{p}{(}\PYG{l+m+mi}{252}\PYG{p}{)}\PYG{o}{.}\PYG{n}{mean}\PYG{p}{(}\PYG{p}{)}
\PYG{n}{std\PYGZus{}2} \PYG{o}{=} \PYG{n}{matana}\PYG{o}{.}\PYG{n}{dropna}\PYG{p}{(}\PYG{p}{)}\PYG{o}{.}\PYG{n}{loc}\PYG{p}{[}\PYG{p}{:}\PYG{l+s+s1}{\PYGZsq{}}\PYG{l+s+s1}{2022}\PYG{l+s+s1}{\PYGZsq{}}\PYG{p}{]}\PYG{o}{.}\PYG{n}{sub}\PYG{p}{(}\PYG{n}{ff}\PYG{p}{[}\PYG{l+s+s1}{\PYGZsq{}}\PYG{l+s+s1}{RF}\PYG{l+s+s1}{\PYGZsq{}}\PYG{p}{]}\PYG{p}{,} \PYG{n}{axis}\PYG{o}{=}\PYG{l+m+mi}{0}\PYG{p}{)}\PYG{o}{.}\PYG{n}{rolling}\PYG{p}{(}\PYG{l+m+mi}{252}\PYG{p}{)}\PYG{o}{.}\PYG{n}{std}\PYG{p}{(}\PYG{p}{)}
\PYG{n}{sharpes\PYGZus{}2} \PYG{o}{=} \PYG{n}{mean\PYGZus{}2}\PYG{o}{.}\PYG{n}{div}\PYG{p}{(}\PYG{n}{std\PYGZus{}2}\PYG{p}{,} \PYG{n}{axis}\PYG{o}{=}\PYG{l+m+mi}{0}\PYG{p}{)}\PYG{o}{.}\PYG{n}{mul}\PYG{p}{(}\PYG{n}{np}\PYG{o}{.}\PYG{n}{sqrt}\PYG{p}{(}\PYG{l+m+mi}{252}\PYG{p}{)}\PYG{p}{)}
\end{sphinxVerbatim}

\end{sphinxuseclass}\end{sphinxVerbatimInput}

\end{sphinxuseclass}
\begin{sphinxuseclass}{cell}\begin{sphinxVerbatimInput}

\begin{sphinxuseclass}{cell_input}
\begin{sphinxVerbatim}[commandchars=\\\{\}]
\PYG{n}{np}\PYG{o}{.}\PYG{n}{allclose}\PYG{p}{(}\PYG{n}{sharpes\PYGZus{}1}\PYG{o}{.}\PYG{n}{dropna}\PYG{p}{(}\PYG{p}{)}\PYG{p}{,} \PYG{n}{sharpes\PYGZus{}2}\PYG{o}{.}\PYG{n}{dropna}\PYG{p}{(}\PYG{p}{)}\PYG{p}{)}
\end{sphinxVerbatim}

\end{sphinxuseclass}\end{sphinxVerbatimInput}
\begin{sphinxVerbatimOutput}

\begin{sphinxuseclass}{cell_output}
\begin{sphinxVerbatim}[commandchars=\\\{\}]
True
\end{sphinxVerbatim}

\end{sphinxuseclass}\end{sphinxVerbatimOutput}

\end{sphinxuseclass}
\begin{sphinxuseclass}{cell}\begin{sphinxVerbatimInput}

\begin{sphinxuseclass}{cell_input}
\begin{sphinxVerbatim}[commandchars=\\\{\}]
\PYG{n}{sharpes\PYGZus{}1}\PYG{o}{.}\PYG{n}{plot}\PYG{p}{(}\PYG{p}{)}
\PYG{n}{plt}\PYG{o}{.}\PYG{n}{ylabel}\PYG{p}{(}\PYG{l+s+s1}{\PYGZsq{}}\PYG{l+s+s1}{Sharpe Ratio}\PYG{l+s+s1}{\PYGZsq{}}\PYG{p}{)}
\PYG{n}{plt}\PYG{o}{.}\PYG{n}{title}\PYG{p}{(}\PYG{l+s+s1}{\PYGZsq{}}\PYG{l+s+s1}{Sharpe Ratios from Daily Returns}\PYG{l+s+se}{\PYGZbs{}n}\PYG{l+s+s1}{252\PYGZhy{}Trading\PYGZhy{}Day Rolling Windows}\PYG{l+s+s1}{\PYGZsq{}}\PYG{p}{)}
\PYG{n}{plt}\PYG{o}{.}\PYG{n}{show}\PYG{p}{(}\PYG{p}{)}
\end{sphinxVerbatim}

\end{sphinxuseclass}\end{sphinxVerbatimInput}
\begin{sphinxVerbatimOutput}

\begin{sphinxuseclass}{cell_output}
\noindent\sphinxincludegraphics{{1e88030c2fe69e86f6bfbf839a93d52d41e5af5a1e288c401bc22751840eba9d}.png}

\end{sphinxuseclass}\end{sphinxVerbatimOutput}

\end{sphinxuseclass}

\subsubsection{Does more frequent rebalancing increase or decrease returns?}
\label{\detokenize{mckinney_11_practice_03:does-more-frequent-rebalancing-increase-or-decrease-returns}}
\sphinxAtStartPar
Compare decade\sphinxhyphen{}total returns for the following rebalancing frequencies:
\begin{enumerate}
\sphinxsetlistlabels{\arabic}{enumi}{enumii}{}{.}%
\item {} 
\sphinxAtStartPar
Daily rebalancing

\item {} 
\sphinxAtStartPar
Monthly rebalancing

\item {} 
\sphinxAtStartPar
Annual rebalancing

\item {} 
\sphinxAtStartPar
Decade rebalancing

\end{enumerate}

\sphinxAtStartPar
Use equally\sphinxhyphen{}weighted portfolios of industry\sphinxhyphen{}level daily returns from French’s website: \sphinxcode{\sphinxupquote{'17\_Industry\_Portfolios\_daily'}}.

\sphinxAtStartPar
\sphinxstyleemphasis{\sphinxstylestrong{This exercise is a little early, and we will revisit later in the course!}}

\sphinxstepscope


\section{McKinney Chapter 11 \sphinxhyphen{} Practice (Wednesday 11:45 AM, Section 4)}
\label{\detokenize{mckinney_11_practice_04:mckinney-chapter-11-practice-wednesday-11-45-am-section-4}}\label{\detokenize{mckinney_11_practice_04::doc}}

\subsection{Announcements}
\label{\detokenize{mckinney_11_practice_04:announcements}}\begin{itemize}
\item {} 
\sphinxAtStartPar
Quiz 4 mean was \$87\%\$

\item {} 
\sphinxAtStartPar
Team Project 1 and Teammates Review 1 are due at 11:59 PM on Friday

\end{itemize}


\subsection{Practice}
\label{\detokenize{mckinney_11_practice_04:practice}}
\begin{sphinxuseclass}{cell}\begin{sphinxVerbatimInput}

\begin{sphinxuseclass}{cell_input}
\begin{sphinxVerbatim}[commandchars=\\\{\}]
\PYG{k+kn}{import} \PYG{n+nn}{pandas} \PYG{k}{as} \PYG{n+nn}{pd}
\PYG{k+kn}{import} \PYG{n+nn}{numpy} \PYG{k}{as} \PYG{n+nn}{np}
\PYG{k+kn}{import} \PYG{n+nn}{matplotlib}\PYG{n+nn}{.}\PYG{n+nn}{pyplot} \PYG{k}{as} \PYG{n+nn}{plt}
\end{sphinxVerbatim}

\end{sphinxuseclass}\end{sphinxVerbatimInput}

\end{sphinxuseclass}
\begin{sphinxuseclass}{cell}\begin{sphinxVerbatimInput}

\begin{sphinxuseclass}{cell_input}
\begin{sphinxVerbatim}[commandchars=\\\{\}]
\PYG{o}{\PYGZpc{}}\PYG{k}{config} InlineBackend.figure\PYGZus{}format = \PYGZsq{}retina\PYGZsq{}
\PYG{o}{\PYGZpc{}}\PYG{k}{precision} 4
\PYG{n}{pd}\PYG{o}{.}\PYG{n}{options}\PYG{o}{.}\PYG{n}{display}\PYG{o}{.}\PYG{n}{float\PYGZus{}format} \PYG{o}{=} \PYG{l+s+s1}{\PYGZsq{}}\PYG{l+s+si}{\PYGZob{}:.4f\PYGZcb{}}\PYG{l+s+s1}{\PYGZsq{}}\PYG{o}{.}\PYG{n}{format}
\end{sphinxVerbatim}

\end{sphinxuseclass}\end{sphinxVerbatimInput}

\end{sphinxuseclass}
\begin{sphinxuseclass}{cell}\begin{sphinxVerbatimInput}

\begin{sphinxuseclass}{cell_input}
\begin{sphinxVerbatim}[commandchars=\\\{\}]
\PYG{k+kn}{import} \PYG{n+nn}{yfinance} \PYG{k}{as} \PYG{n+nn}{yf}
\PYG{k+kn}{import} \PYG{n+nn}{pandas\PYGZus{}datareader} \PYG{k}{as} \PYG{n+nn}{pdr}
\PYG{k+kn}{import} \PYG{n+nn}{requests\PYGZus{}cache}
\PYG{n}{session} \PYG{o}{=} \PYG{n}{requests\PYGZus{}cache}\PYG{o}{.}\PYG{n}{CachedSession}\PYG{p}{(}\PYG{p}{)}
\end{sphinxVerbatim}

\end{sphinxuseclass}\end{sphinxVerbatimInput}

\end{sphinxuseclass}

\subsubsection{Which are larger, overnight or intraday returns?}
\label{\detokenize{mckinney_11_practice_04:which-are-larger-overnight-or-intraday-returns}}
\sphinxAtStartPar
Yahoo! Finance provides easy acces to high\sphinxhyphen{}quality open, high, low, close (OHLC) and adjusted close price data.
However, Yahoo! Finance does not provide overnight or instraday returns directly.
Therefore, we need to use math to decompose daily returns into overnight and intraday returns.

\sphinxAtStartPar
Daily returns are defined as (adjusted) closing price to (adjusted) closing price returns.
Therefore, daily returns consist of overnight returns compounded with the intraday returns from the next day \$(1 + R\_\{daily\}) = (1 + R\_\{overnight\}) \textbackslash{}times (1 + R\_\{intraday\})\$ which we can rearrange to calculate overnight returns as \$\textbackslash{}frac\{1 + R\_\{daily\}\}\{1 + R\_\{intraday\}\} \sphinxhyphen{} 1 = R\_\{overnight\}\$.

\sphinxAtStartPar
We can calculate daily and intraday returns from Yahoo! Finance data as \$R\_\{daily\} = \textbackslash{}frac\{Adj\textbackslash{} Close\_\{t\} \sphinxhyphen{} Adj\textbackslash{} Close\_\{t\sphinxhyphen{}1\}\}\{Adj\textbackslash{} Close\_\{t\sphinxhyphen{}1\}\}\$ and \$R\_\{intraday\} = \textbackslash{}frac\{Close \sphinxhyphen{} Open\}\{Open\}\$.

\sphinxAtStartPar
Compare the following for the SPY ETF:
\begin{enumerate}
\sphinxsetlistlabels{\arabic}{enumi}{enumii}{}{.}%
\item {} 
\sphinxAtStartPar
Cumulative returns with all available data

\item {} 
\sphinxAtStartPar
Total returns for each calendar year

\item {} 
\sphinxAtStartPar
Total returns over rolling 252\sphinxhyphen{}trading\sphinxhyphen{}day windows

\item {} 
\sphinxAtStartPar
Total returns over rolling 12\sphinxhyphen{}months windows after calculating monthly returns
. Sharpe Ratios for each calendar year

\end{enumerate}


\paragraph{Cumulative returns with all available data}
\label{\detokenize{mckinney_11_practice_04:cumulative-returns-with-all-available-data}}
\begin{sphinxuseclass}{cell}\begin{sphinxVerbatimInput}

\begin{sphinxuseclass}{cell_input}
\begin{sphinxVerbatim}[commandchars=\\\{\}]
\PYG{n}{spy} \PYG{o}{=} \PYG{n}{yf}\PYG{o}{.}\PYG{n}{download}\PYG{p}{(}\PYG{n}{tickers}\PYG{o}{=}\PYG{l+s+s1}{\PYGZsq{}}\PYG{l+s+s1}{SPY}\PYG{l+s+s1}{\PYGZsq{}}\PYG{p}{,} \PYG{n}{progress}\PYG{o}{=}\PYG{k+kc}{False}\PYG{p}{)}
\end{sphinxVerbatim}

\end{sphinxuseclass}\end{sphinxVerbatimInput}
\begin{sphinxVerbatimOutput}

\begin{sphinxuseclass}{cell_output}
\begin{sphinxVerbatim}[commandchars=\\\{\}]
[*********************100\PYGZpc{}***********************]  1 of 1 completed
\end{sphinxVerbatim}

\end{sphinxuseclass}\end{sphinxVerbatimOutput}

\end{sphinxuseclass}
\begin{sphinxuseclass}{cell}\begin{sphinxVerbatimInput}

\begin{sphinxuseclass}{cell_input}
\begin{sphinxVerbatim}[commandchars=\\\{\}]
\PYG{n}{returns} \PYG{o}{=} \PYG{p}{(}
    \PYG{n}{spy}
    \PYG{o}{.}\PYG{n}{assign}\PYG{p}{(}
        \PYG{n}{Date} \PYG{o}{=} \PYG{k}{lambda} \PYG{n}{x}\PYG{p}{:} \PYG{n}{x}\PYG{o}{.}\PYG{n}{index}\PYG{o}{.}\PYG{n}{tz\PYGZus{}localize}\PYG{p}{(}\PYG{k+kc}{None}\PYG{p}{)}\PYG{p}{,}
        \PYG{n}{Total} \PYG{o}{=} \PYG{k}{lambda} \PYG{n}{x}\PYG{p}{:} \PYG{n}{x}\PYG{p}{[}\PYG{l+s+s1}{\PYGZsq{}}\PYG{l+s+s1}{Adj Close}\PYG{l+s+s1}{\PYGZsq{}}\PYG{p}{]}\PYG{o}{.}\PYG{n}{pct\PYGZus{}change}\PYG{p}{(}\PYG{p}{)}\PYG{p}{,}
        \PYG{n}{Intraday} \PYG{o}{=} \PYG{k}{lambda} \PYG{n}{x}\PYG{p}{:} \PYG{n}{x}\PYG{p}{[}\PYG{l+s+s1}{\PYGZsq{}}\PYG{l+s+s1}{Close}\PYG{l+s+s1}{\PYGZsq{}}\PYG{p}{]}\PYG{o}{.}\PYG{n}{div}\PYG{p}{(}\PYG{n}{x}\PYG{p}{[}\PYG{l+s+s1}{\PYGZsq{}}\PYG{l+s+s1}{Open}\PYG{l+s+s1}{\PYGZsq{}}\PYG{p}{]}\PYG{p}{)}\PYG{o}{.}\PYG{n}{sub}\PYG{p}{(}\PYG{l+m+mi}{1}\PYG{p}{)}\PYG{p}{,}
        \PYG{n}{Overnight} \PYG{o}{=} \PYG{k}{lambda} \PYG{n}{x}\PYG{p}{:} \PYG{p}{(}\PYG{l+m+mi}{1} \PYG{o}{+} \PYG{n}{x}\PYG{p}{[}\PYG{l+s+s1}{\PYGZsq{}}\PYG{l+s+s1}{Total}\PYG{l+s+s1}{\PYGZsq{}}\PYG{p}{]}\PYG{p}{)} \PYG{o}{/} \PYG{p}{(}\PYG{l+m+mi}{1} \PYG{o}{+} \PYG{n}{x}\PYG{p}{[}\PYG{l+s+s1}{\PYGZsq{}}\PYG{l+s+s1}{Intraday}\PYG{l+s+s1}{\PYGZsq{}}\PYG{p}{]}\PYG{p}{)} \PYG{o}{\PYGZhy{}} \PYG{l+m+mi}{1}
    \PYG{p}{)}
    \PYG{o}{.}\PYG{n}{set\PYGZus{}index}\PYG{p}{(}\PYG{l+s+s1}{\PYGZsq{}}\PYG{l+s+s1}{Date}\PYG{l+s+s1}{\PYGZsq{}}\PYG{p}{)}
    \PYG{p}{[}\PYG{p}{[}\PYG{l+s+s1}{\PYGZsq{}}\PYG{l+s+s1}{Intraday}\PYG{l+s+s1}{\PYGZsq{}}\PYG{p}{,} \PYG{l+s+s1}{\PYGZsq{}}\PYG{l+s+s1}{Overnight}\PYG{l+s+s1}{\PYGZsq{}}\PYG{p}{,} \PYG{l+s+s1}{\PYGZsq{}}\PYG{l+s+s1}{Total}\PYG{l+s+s1}{\PYGZsq{}}\PYG{p}{]}\PYG{p}{]}
    \PYG{o}{.}\PYG{n}{rename\PYGZus{}axis}\PYG{p}{(}\PYG{n}{columns}\PYG{o}{=}\PYG{l+s+s1}{\PYGZsq{}}\PYG{l+s+s1}{Return}\PYG{l+s+s1}{\PYGZsq{}}\PYG{p}{)}
    \PYG{o}{.}\PYG{n}{dropna}\PYG{p}{(}\PYG{p}{)}
\PYG{p}{)}

\PYG{n}{returns}\PYG{o}{.}\PYG{n}{head}\PYG{p}{(}\PYG{p}{)}
\end{sphinxVerbatim}

\end{sphinxuseclass}\end{sphinxVerbatimInput}
\begin{sphinxVerbatimOutput}

\begin{sphinxuseclass}{cell_output}
\begin{sphinxVerbatim}[commandchars=\\\{\}]
Return      Intraday  Overnight   Total
Date                                   
1993\PYGZhy{}02\PYGZhy{}01    0.0064     0.0007  0.0071
1993\PYGZhy{}02\PYGZhy{}02    0.0028    \PYGZhy{}0.0007  0.0021
1993\PYGZhy{}02\PYGZhy{}03    0.0091     0.0014  0.0106
1993\PYGZhy{}02\PYGZhy{}04    0.0007     0.0035  0.0042
1993\PYGZhy{}02\PYGZhy{}05    0.0000    \PYGZhy{}0.0007 \PYGZhy{}0.0007
\end{sphinxVerbatim}

\end{sphinxuseclass}\end{sphinxVerbatimOutput}

\end{sphinxuseclass}
\begin{sphinxuseclass}{cell}\begin{sphinxVerbatimInput}

\begin{sphinxuseclass}{cell_input}
\begin{sphinxVerbatim}[commandchars=\\\{\}]
\PYG{n}{returns}\PYG{o}{.}\PYG{n}{describe}\PYG{p}{(}\PYG{p}{)}\PYG{o}{.}\PYG{n}{T}
\end{sphinxVerbatim}

\end{sphinxuseclass}\end{sphinxVerbatimInput}
\begin{sphinxVerbatimOutput}

\begin{sphinxuseclass}{cell_output}
\begin{sphinxVerbatim}[commandchars=\\\{\}]
              count   mean    std     min     25\PYGZpc{}    50\PYGZpc{}    75\PYGZpc{}    max
Return                                                                
Intraday  7575.0000 0.0000 0.0097 \PYGZhy{}0.0899 \PYGZhy{}0.0042 0.0004 0.0047 0.0930
Overnight 7575.0000 0.0004 0.0068 \PYGZhy{}0.1045 \PYGZhy{}0.0021 0.0006 0.0032 0.0613
Total     7575.0000 0.0004 0.0119 \PYGZhy{}0.1094 \PYGZhy{}0.0045 0.0007 0.0059 0.1452
\end{sphinxVerbatim}

\end{sphinxuseclass}\end{sphinxVerbatimOutput}

\end{sphinxuseclass}
\begin{sphinxuseclass}{cell}\begin{sphinxVerbatimInput}

\begin{sphinxuseclass}{cell_input}
\begin{sphinxVerbatim}[commandchars=\\\{\}]
\PYG{n}{returns}\PYG{o}{.}\PYG{n}{add}\PYG{p}{(}\PYG{l+m+mi}{1}\PYG{p}{)}\PYG{o}{.}\PYG{n}{cumprod}\PYG{p}{(}\PYG{p}{)}\PYG{o}{.}\PYG{n}{sub}\PYG{p}{(}\PYG{l+m+mi}{1}\PYG{p}{)}\PYG{o}{.}\PYG{n}{mul}\PYG{p}{(}\PYG{l+m+mi}{100}\PYG{p}{)}\PYG{o}{.}\PYG{n}{plot}\PYG{p}{(}\PYG{p}{)}
\PYG{n}{plt}\PYG{o}{.}\PYG{n}{ylabel}\PYG{p}{(}\PYG{l+s+s1}{\PYGZsq{}}\PYG{l+s+s1}{Cumulative Return (}\PYG{l+s+s1}{\PYGZpc{}}\PYG{l+s+s1}{)}\PYG{l+s+s1}{\PYGZsq{}}\PYG{p}{)}
\PYG{n}{plt}\PYG{o}{.}\PYG{n}{title}\PYG{p}{(}\PYG{l+s+s1}{\PYGZsq{}}\PYG{l+s+s1}{Intraday and Overnight Return Decomposition}\PYG{l+s+se}{\PYGZbs{}n}\PYG{l+s+s1}{Cumulative Returns}\PYG{l+s+s1}{\PYGZsq{}}\PYG{p}{)}
\PYG{n}{plt}\PYG{o}{.}\PYG{n}{show}\PYG{p}{(}\PYG{p}{)}
\end{sphinxVerbatim}

\end{sphinxuseclass}\end{sphinxVerbatimInput}
\begin{sphinxVerbatimOutput}

\begin{sphinxuseclass}{cell_output}
\noindent\sphinxincludegraphics{{334bda14ee5bf292e99376bcadc26355b70b90822e6f8dc561e433c7dbe0fea6}.png}

\end{sphinxuseclass}\end{sphinxVerbatimOutput}

\end{sphinxuseclass}
\begin{sphinxuseclass}{cell}\begin{sphinxVerbatimInput}

\begin{sphinxuseclass}{cell_input}
\begin{sphinxVerbatim}[commandchars=\\\{\}]
\PYG{k}{def} \PYG{n+nf}{totret}\PYG{p}{(}\PYG{n}{x}\PYG{p}{)}\PYG{p}{:}
    \PYG{k}{return} \PYG{p}{(}\PYG{l+m+mi}{1} \PYG{o}{+} \PYG{n}{x}\PYG{p}{)}\PYG{o}{.}\PYG{n}{prod}\PYG{p}{(}\PYG{p}{)} \PYG{o}{\PYGZhy{}} \PYG{l+m+mi}{1}
\end{sphinxVerbatim}

\end{sphinxuseclass}\end{sphinxVerbatimInput}

\end{sphinxuseclass}

\paragraph{Total returns for each calendar year}
\label{\detokenize{mckinney_11_practice_04:total-returns-for-each-calendar-year}}
\begin{sphinxuseclass}{cell}\begin{sphinxVerbatimInput}

\begin{sphinxuseclass}{cell_input}
\begin{sphinxVerbatim}[commandchars=\\\{\}]
\PYG{n}{returns\PYGZus{}a} \PYG{o}{=} \PYG{n}{returns}\PYG{o}{.}\PYG{n}{resample}\PYG{p}{(}\PYG{l+s+s1}{\PYGZsq{}}\PYG{l+s+s1}{Y}\PYG{l+s+s1}{\PYGZsq{}}\PYG{p}{,} \PYG{n}{kind}\PYG{o}{=}\PYG{l+s+s1}{\PYGZsq{}}\PYG{l+s+s1}{period}\PYG{l+s+s1}{\PYGZsq{}}\PYG{p}{)}\PYG{o}{.}\PYG{n}{apply}\PYG{p}{(}\PYG{n}{totret}\PYG{p}{)}
\end{sphinxVerbatim}

\end{sphinxuseclass}\end{sphinxVerbatimInput}

\end{sphinxuseclass}
\sphinxAtStartPar
We can use an anonymous (lambda) function to calculate total returns, too.

\begin{sphinxuseclass}{cell}\begin{sphinxVerbatimInput}

\begin{sphinxuseclass}{cell_input}
\begin{sphinxVerbatim}[commandchars=\\\{\}]
\PYG{n}{np}\PYG{o}{.}\PYG{n}{allclose}\PYG{p}{(}
    \PYG{n}{returns}\PYG{o}{.}\PYG{n}{resample}\PYG{p}{(}\PYG{l+s+s1}{\PYGZsq{}}\PYG{l+s+s1}{Y}\PYG{l+s+s1}{\PYGZsq{}}\PYG{p}{,} \PYG{n}{kind}\PYG{o}{=}\PYG{l+s+s1}{\PYGZsq{}}\PYG{l+s+s1}{period}\PYG{l+s+s1}{\PYGZsq{}}\PYG{p}{)}\PYG{o}{.}\PYG{n}{apply}\PYG{p}{(}\PYG{k}{lambda} \PYG{n}{x}\PYG{p}{:} \PYG{p}{(}\PYG{l+m+mi}{1} \PYG{o}{+} \PYG{n}{x}\PYG{p}{)}\PYG{o}{.}\PYG{n}{prod}\PYG{p}{(}\PYG{p}{)} \PYG{o}{\PYGZhy{}} \PYG{l+m+mi}{1}\PYG{p}{)}\PYG{p}{,}
    \PYG{n}{returns\PYGZus{}a}
\PYG{p}{)}
\end{sphinxVerbatim}

\end{sphinxuseclass}\end{sphinxVerbatimInput}
\begin{sphinxVerbatimOutput}

\begin{sphinxuseclass}{cell_output}
\begin{sphinxVerbatim}[commandchars=\\\{\}]
True
\end{sphinxVerbatim}

\end{sphinxuseclass}\end{sphinxVerbatimOutput}

\end{sphinxuseclass}
\sphinxAtStartPar
We can use \sphinxcode{\sphinxupquote{counts\_a}} to remove years with too few observations.

\begin{sphinxuseclass}{cell}\begin{sphinxVerbatimInput}

\begin{sphinxuseclass}{cell_input}
\begin{sphinxVerbatim}[commandchars=\\\{\}]
\PYG{n}{counts\PYGZus{}a} \PYG{o}{=} \PYG{n}{returns}\PYG{o}{.}\PYG{n}{resample}\PYG{p}{(}\PYG{l+s+s1}{\PYGZsq{}}\PYG{l+s+s1}{Y}\PYG{l+s+s1}{\PYGZsq{}}\PYG{p}{,} \PYG{n}{kind}\PYG{o}{=}\PYG{l+s+s1}{\PYGZsq{}}\PYG{l+s+s1}{period}\PYG{l+s+s1}{\PYGZsq{}}\PYG{p}{)}\PYG{o}{.}\PYG{n}{count}\PYG{p}{(}\PYG{p}{)}
\end{sphinxVerbatim}

\end{sphinxuseclass}\end{sphinxVerbatimInput}

\end{sphinxuseclass}
\begin{sphinxuseclass}{cell}\begin{sphinxVerbatimInput}

\begin{sphinxuseclass}{cell_input}
\begin{sphinxVerbatim}[commandchars=\\\{\}]
\PYG{n}{returns\PYGZus{}a}\PYG{o}{.}\PYG{n}{loc}\PYG{p}{[}\PYG{n}{counts\PYGZus{}a}\PYG{p}{[}\PYG{l+s+s1}{\PYGZsq{}}\PYG{l+s+s1}{Total}\PYG{l+s+s1}{\PYGZsq{}}\PYG{p}{]} \PYG{o}{\PYGZgt{}}\PYG{o}{=} \PYG{l+m+mi}{250}\PYG{p}{]}\PYG{o}{.}\PYG{n}{plot}\PYG{p}{(}\PYG{n}{kind}\PYG{o}{=}\PYG{l+s+s1}{\PYGZsq{}}\PYG{l+s+s1}{bar}\PYG{l+s+s1}{\PYGZsq{}}\PYG{p}{)}
\PYG{n}{plt}\PYG{o}{.}\PYG{n}{ylabel}\PYG{p}{(}\PYG{l+s+s1}{\PYGZsq{}}\PYG{l+s+s1}{Total Return (}\PYG{l+s+s1}{\PYGZpc{}}\PYG{l+s+s1}{)}\PYG{l+s+s1}{\PYGZsq{}}\PYG{p}{)}
\PYG{n}{plt}\PYG{o}{.}\PYG{n}{title}\PYG{p}{(}\PYG{l+s+s1}{\PYGZsq{}}\PYG{l+s+s1}{Intraday and Overnight Return Decomposition}\PYG{l+s+se}{\PYGZbs{}n}\PYG{l+s+s1}{Calendar Year Total Returns}\PYG{l+s+s1}{\PYGZsq{}}\PYG{p}{)}
\PYG{n}{plt}\PYG{o}{.}\PYG{n}{show}\PYG{p}{(}\PYG{p}{)}
\end{sphinxVerbatim}

\end{sphinxuseclass}\end{sphinxVerbatimInput}
\begin{sphinxVerbatimOutput}

\begin{sphinxuseclass}{cell_output}
\noindent\sphinxincludegraphics{{d36763091bd2da5b0bd7f190769e61d1c512f41a0bbe68f89792e899b181c8d9}.png}

\end{sphinxuseclass}\end{sphinxVerbatimOutput}

\end{sphinxuseclass}

\bigskip\hrule\bigskip


\sphinxAtStartPar
What is the difference between \sphinxstyleemphasis{cumulative} and \sphinxstyleemphasis{total} return?
\begin{itemize}
\item {} 
\sphinxAtStartPar
Total return is the return for the entire holding period, or one value for a range of dates

\item {} 
\sphinxAtStartPar
Cumulative returns is the total return at every single date in a range of dates

\end{itemize}

\sphinxAtStartPar
Total Returns:

\begin{sphinxuseclass}{cell}\begin{sphinxVerbatimInput}

\begin{sphinxuseclass}{cell_input}
\begin{sphinxVerbatim}[commandchars=\\\{\}]
\PYG{n}{returns}\PYG{p}{[}\PYG{l+s+s1}{\PYGZsq{}}\PYG{l+s+s1}{Total}\PYG{l+s+s1}{\PYGZsq{}}\PYG{p}{]}\PYG{o}{.}\PYG{n}{add}\PYG{p}{(}\PYG{l+m+mi}{1}\PYG{p}{)}\PYG{o}{.}\PYG{n}{prod}\PYG{p}{(}\PYG{p}{)} \PYG{o}{\PYGZhy{}} \PYG{l+m+mi}{1}
\end{sphinxVerbatim}

\end{sphinxuseclass}\end{sphinxVerbatimInput}
\begin{sphinxVerbatimOutput}

\begin{sphinxuseclass}{cell_output}
\begin{sphinxVerbatim}[commandchars=\\\{\}]
14.652961029877108
\end{sphinxVerbatim}

\end{sphinxuseclass}\end{sphinxVerbatimOutput}

\end{sphinxuseclass}
\sphinxAtStartPar
Cumulative Returns:

\begin{sphinxuseclass}{cell}\begin{sphinxVerbatimInput}

\begin{sphinxuseclass}{cell_input}
\begin{sphinxVerbatim}[commandchars=\\\{\}]
\PYG{n}{returns}\PYG{p}{[}\PYG{l+s+s1}{\PYGZsq{}}\PYG{l+s+s1}{Total}\PYG{l+s+s1}{\PYGZsq{}}\PYG{p}{]}\PYG{o}{.}\PYG{n}{add}\PYG{p}{(}\PYG{l+m+mi}{1}\PYG{p}{)}\PYG{o}{.}\PYG{n}{cumprod}\PYG{p}{(}\PYG{p}{)} \PYG{o}{\PYGZhy{}} \PYG{l+m+mi}{1}
\end{sphinxVerbatim}

\end{sphinxuseclass}\end{sphinxVerbatimInput}
\begin{sphinxVerbatimOutput}

\begin{sphinxuseclass}{cell_output}
\begin{sphinxVerbatim}[commandchars=\\\{\}]
Date
1993\PYGZhy{}02\PYGZhy{}01    0.0071
1993\PYGZhy{}02\PYGZhy{}02    0.0092
1993\PYGZhy{}02\PYGZhy{}03    0.0199
1993\PYGZhy{}02\PYGZhy{}04    0.0242
1993\PYGZhy{}02\PYGZhy{}05    0.0235
               ...  
2023\PYGZhy{}02\PYGZhy{}23   14.8877
2023\PYGZhy{}02\PYGZhy{}24   14.7180
2023\PYGZhy{}02\PYGZhy{}27   14.7715
2023\PYGZhy{}02\PYGZhy{}28   14.7132
2023\PYGZhy{}03\PYGZhy{}01   14.6530
Name: Total, Length: 7575, dtype: float64
\end{sphinxVerbatim}

\end{sphinxuseclass}\end{sphinxVerbatimOutput}

\end{sphinxuseclass}

\bigskip\hrule\bigskip



\paragraph{Total returns over rolling 252\sphinxhyphen{}trading\sphinxhyphen{}day windows}
\label{\detokenize{mckinney_11_practice_04:total-returns-over-rolling-252-trading-day-windows}}
\sphinxAtStartPar
We can repeat the total return calculation for 252\sphinxhyphen{}trading\sphinxhyphen{}day rolling windows.

\begin{sphinxuseclass}{cell}\begin{sphinxVerbatimInput}

\begin{sphinxuseclass}{cell_input}
\begin{sphinxVerbatim}[commandchars=\\\{\}]
\PYG{n}{returns\PYGZus{}r} \PYG{o}{=} \PYG{n}{returns}\PYG{o}{.}\PYG{n}{rolling}\PYG{p}{(}\PYG{l+m+mi}{252}\PYG{p}{)}\PYG{o}{.}\PYG{n}{apply}\PYG{p}{(}\PYG{n}{totret}\PYG{p}{)}

\PYG{n}{returns\PYGZus{}r}\PYG{o}{.}\PYG{n}{head}\PYG{p}{(}\PYG{p}{)}
\end{sphinxVerbatim}

\end{sphinxuseclass}\end{sphinxVerbatimInput}
\begin{sphinxVerbatimOutput}

\begin{sphinxuseclass}{cell_output}
\begin{sphinxVerbatim}[commandchars=\\\{\}]
Return      Intraday  Overnight  Total
Date                                  
1993\PYGZhy{}02\PYGZhy{}01       NaN        NaN    NaN
1993\PYGZhy{}02\PYGZhy{}02       NaN        NaN    NaN
1993\PYGZhy{}02\PYGZhy{}03       NaN        NaN    NaN
1993\PYGZhy{}02\PYGZhy{}04       NaN        NaN    NaN
1993\PYGZhy{}02\PYGZhy{}05       NaN        NaN    NaN
\end{sphinxVerbatim}

\end{sphinxuseclass}\end{sphinxVerbatimOutput}

\end{sphinxuseclass}
\begin{sphinxuseclass}{cell}\begin{sphinxVerbatimInput}

\begin{sphinxuseclass}{cell_input}
\begin{sphinxVerbatim}[commandchars=\\\{\}]
\PYG{n}{returns\PYGZus{}r}\PYG{o}{.}\PYG{n}{mul}\PYG{p}{(}\PYG{l+m+mi}{100}\PYG{p}{)}\PYG{o}{.}\PYG{n}{plot}\PYG{p}{(}\PYG{p}{)}
\PYG{n}{plt}\PYG{o}{.}\PYG{n}{ylabel}\PYG{p}{(}\PYG{l+s+s1}{\PYGZsq{}}\PYG{l+s+s1}{Total Return (}\PYG{l+s+s1}{\PYGZpc{}}\PYG{l+s+s1}{)}\PYG{l+s+s1}{\PYGZsq{}}\PYG{p}{)}
\PYG{n}{plt}\PYG{o}{.}\PYG{n}{title}\PYG{p}{(}\PYG{l+s+s1}{\PYGZsq{}}\PYG{l+s+s1}{Intraday and Overnight Return Decomposition}\PYG{l+s+se}{\PYGZbs{}n}\PYG{l+s+s1}{252\PYGZhy{}Trading\PYGZhy{}Day Rolling Windows}\PYG{l+s+s1}{\PYGZsq{}}\PYG{p}{)}
\PYG{n}{plt}\PYG{o}{.}\PYG{n}{show}\PYG{p}{(}\PYG{p}{)}
\end{sphinxVerbatim}

\end{sphinxuseclass}\end{sphinxVerbatimInput}
\begin{sphinxVerbatimOutput}

\begin{sphinxuseclass}{cell_output}
\noindent\sphinxincludegraphics{{8263a05ad89fb0e64368c006d13dd94c0dfb41bc8644ae40ee0f5557dde0d9eb}.png}

\end{sphinxuseclass}\end{sphinxVerbatimOutput}

\end{sphinxuseclass}

\paragraph{Total returns over rolling 12\sphinxhyphen{}months windows after calculating monthly returns}
\label{\detokenize{mckinney_11_practice_04:total-returns-over-rolling-12-months-windows-after-calculating-monthly-returns}}
\sphinxAtStartPar
We can chain this operation!
Note the plot has the same general appearance, but is less noisy because we aggregate to monthly total returns first!

\begin{sphinxuseclass}{cell}\begin{sphinxVerbatimInput}

\begin{sphinxuseclass}{cell_input}
\begin{sphinxVerbatim}[commandchars=\\\{\}]
\PYG{n}{returns\PYGZus{}mr} \PYG{o}{=} \PYG{p}{(}
    \PYG{n}{returns}
    \PYG{o}{.}\PYG{n}{resample}\PYG{p}{(}\PYG{n}{rule}\PYG{o}{=}\PYG{l+s+s1}{\PYGZsq{}}\PYG{l+s+s1}{M}\PYG{l+s+s1}{\PYGZsq{}}\PYG{p}{,} \PYG{n}{kind}\PYG{o}{=}\PYG{l+s+s1}{\PYGZsq{}}\PYG{l+s+s1}{period}\PYG{l+s+s1}{\PYGZsq{}}\PYG{p}{)} \PYG{c+c1}{\PYGZsh{} aggregate from daily to monthly}
    \PYG{o}{.}\PYG{n}{apply}\PYG{p}{(}\PYG{n}{totret}\PYG{p}{)} \PYG{c+c1}{\PYGZsh{} ...calculate total returns}
    \PYG{o}{.}\PYG{n}{rolling}\PYG{p}{(}\PYG{l+m+mi}{12}\PYG{p}{)} \PYG{c+c1}{\PYGZsh{} aggregate into 12\PYGZhy{}month rolling windows}
    \PYG{o}{.}\PYG{n}{apply}\PYG{p}{(}\PYG{n}{totret}\PYG{p}{)} \PYG{c+c1}{\PYGZsh{} ...calculate total returns}
\PYG{p}{)}
\end{sphinxVerbatim}

\end{sphinxuseclass}\end{sphinxVerbatimInput}

\end{sphinxuseclass}
\begin{sphinxuseclass}{cell}\begin{sphinxVerbatimInput}

\begin{sphinxuseclass}{cell_input}
\begin{sphinxVerbatim}[commandchars=\\\{\}]
\PYG{n}{returns\PYGZus{}mr}\PYG{o}{.}\PYG{n}{mul}\PYG{p}{(}\PYG{l+m+mi}{100}\PYG{p}{)}\PYG{o}{.}\PYG{n}{plot}\PYG{p}{(}\PYG{p}{)}
\PYG{n}{plt}\PYG{o}{.}\PYG{n}{ylabel}\PYG{p}{(}\PYG{l+s+s1}{\PYGZsq{}}\PYG{l+s+s1}{Total Return (}\PYG{l+s+s1}{\PYGZpc{}}\PYG{l+s+s1}{)}\PYG{l+s+s1}{\PYGZsq{}}\PYG{p}{)}
\PYG{n}{plt}\PYG{o}{.}\PYG{n}{title}\PYG{p}{(}\PYG{l+s+s1}{\PYGZsq{}}\PYG{l+s+s1}{Intraday and Overnight Return Decomposition}\PYG{l+s+se}{\PYGZbs{}n}\PYG{l+s+s1}{12\PYGZhy{}Month Rolling Windows}\PYG{l+s+s1}{\PYGZsq{}}\PYG{p}{)}
\PYG{n}{plt}\PYG{o}{.}\PYG{n}{show}\PYG{p}{(}\PYG{p}{)}
\end{sphinxVerbatim}

\end{sphinxuseclass}\end{sphinxVerbatimInput}
\begin{sphinxVerbatimOutput}

\begin{sphinxuseclass}{cell_output}
\noindent\sphinxincludegraphics{{68b3a799c69710aba3b785cdfe0b97208ccdd5253837ff4d1912809d8a0e5109}.png}

\end{sphinxuseclass}\end{sphinxVerbatimOutput}

\end{sphinxuseclass}

\paragraph{Sharpe Ratios for each calendar year}
\label{\detokenize{mckinney_11_practice_04:sharpe-ratios-for-each-calendar-year}}
\sphinxAtStartPar
We need the risk\sphinxhyphen{}free rate \sphinxcode{\sphinxupquote{RF}} from Ken French’s daily benchmark factors.

\begin{sphinxuseclass}{cell}\begin{sphinxVerbatimInput}

\begin{sphinxuseclass}{cell_input}
\begin{sphinxVerbatim}[commandchars=\\\{\}]
\PYG{n}{ff} \PYG{o}{=} \PYG{p}{(}
    \PYG{n}{pdr}\PYG{o}{.}\PYG{n}{DataReader}\PYG{p}{(}
        \PYG{n}{name}\PYG{o}{=}\PYG{l+s+s1}{\PYGZsq{}}\PYG{l+s+s1}{F\PYGZhy{}F\PYGZus{}Research\PYGZus{}Data\PYGZus{}Factors\PYGZus{}daily}\PYG{l+s+s1}{\PYGZsq{}}\PYG{p}{,}
        \PYG{n}{data\PYGZus{}source}\PYG{o}{=}\PYG{l+s+s1}{\PYGZsq{}}\PYG{l+s+s1}{famafrench}\PYG{l+s+s1}{\PYGZsq{}}\PYG{p}{,}
        \PYG{n}{start}\PYG{o}{=}\PYG{l+s+s1}{\PYGZsq{}}\PYG{l+s+s1}{1900}\PYG{l+s+s1}{\PYGZsq{}}\PYG{p}{,}
        \PYG{n}{session}\PYG{o}{=}\PYG{n}{session}
    \PYG{p}{)}
    \PYG{p}{[}\PYG{l+m+mi}{0}\PYG{p}{]}
    \PYG{o}{.}\PYG{n}{div}\PYG{p}{(}\PYG{l+m+mi}{100}\PYG{p}{)}
\PYG{p}{)}
\end{sphinxVerbatim}

\end{sphinxuseclass}\end{sphinxVerbatimInput}

\end{sphinxuseclass}
\sphinxAtStartPar
I think this easiest (and fastest approach) is to:
\begin{enumerate}
\sphinxsetlistlabels{\arabic}{enumi}{enumii}{}{.}%
\item {} 
\sphinxAtStartPar
Calculate a data frame of excess returns

\item {} 
\sphinxAtStartPar
Use an anonymous (lambda) function

\end{enumerate}

\begin{sphinxuseclass}{cell}\begin{sphinxVerbatimInput}

\begin{sphinxuseclass}{cell_input}
\begin{sphinxVerbatim}[commandchars=\\\{\}]
\PYG{n}{returns\PYGZus{}excess} \PYG{o}{=} \PYG{p}{(}
    \PYG{n}{returns}
    \PYG{o}{.}\PYG{n}{sub}\PYG{p}{(}\PYG{n}{ff}\PYG{p}{[}\PYG{l+s+s1}{\PYGZsq{}}\PYG{l+s+s1}{RF}\PYG{l+s+s1}{\PYGZsq{}}\PYG{p}{]}\PYG{p}{,} \PYG{n}{axis}\PYG{o}{=}\PYG{l+m+mi}{0}\PYG{p}{)} \PYG{c+c1}{\PYGZsh{} note axis=0 for column\PYGZhy{}wise subtraction}
    \PYG{o}{.}\PYG{n}{dropna}\PYG{p}{(}\PYG{p}{)}
\PYG{p}{)}
\end{sphinxVerbatim}

\end{sphinxuseclass}\end{sphinxVerbatimInput}

\end{sphinxuseclass}
\begin{sphinxuseclass}{cell}\begin{sphinxVerbatimInput}

\begin{sphinxuseclass}{cell_input}
\begin{sphinxVerbatim}[commandchars=\\\{\}]
\PYG{n}{sharpes\PYGZus{}a} \PYG{o}{=} \PYG{p}{(}
    \PYG{n}{returns\PYGZus{}excess}
    \PYG{o}{.}\PYG{n}{resample}\PYG{p}{(}\PYG{n}{rule}\PYG{o}{=}\PYG{l+s+s1}{\PYGZsq{}}\PYG{l+s+s1}{A}\PYG{l+s+s1}{\PYGZsq{}}\PYG{p}{,} \PYG{n}{kind}\PYG{o}{=}\PYG{l+s+s1}{\PYGZsq{}}\PYG{l+s+s1}{period}\PYG{l+s+s1}{\PYGZsq{}}\PYG{p}{)}
    \PYG{o}{.}\PYG{n}{apply}\PYG{p}{(}\PYG{k}{lambda} \PYG{n}{x}\PYG{p}{:} \PYG{n}{np}\PYG{o}{.}\PYG{n}{sqrt}\PYG{p}{(}\PYG{l+m+mi}{252}\PYG{p}{)} \PYG{o}{*} \PYG{n}{x}\PYG{o}{.}\PYG{n}{mean}\PYG{p}{(}\PYG{p}{)} \PYG{o}{/} \PYG{n}{x}\PYG{o}{.}\PYG{n}{std}\PYG{p}{(}\PYG{p}{)}\PYG{p}{)}
\PYG{p}{)}

\PYG{n}{sharpes\PYGZus{}a}\PYG{o}{.}\PYG{n}{head}\PYG{p}{(}\PYG{p}{)}
\end{sphinxVerbatim}

\end{sphinxuseclass}\end{sphinxVerbatimInput}
\begin{sphinxVerbatimOutput}

\begin{sphinxuseclass}{cell_output}
\begin{sphinxVerbatim}[commandchars=\\\{\}]
Return  Intraday  Overnight   Total
Date                               
1993     \PYGZhy{}0.9172     2.8058  0.7233
1994     \PYGZhy{}1.1871     0.8819 \PYGZhy{}0.2732
1995      2.5275     0.4197  3.1845
1996     \PYGZhy{}0.1691     1.6658  1.2247
1997     \PYGZhy{}0.1781     2.5564  1.2796
\end{sphinxVerbatim}

\end{sphinxuseclass}\end{sphinxVerbatimOutput}

\end{sphinxuseclass}
\begin{sphinxuseclass}{cell}\begin{sphinxVerbatimInput}

\begin{sphinxuseclass}{cell_input}
\begin{sphinxVerbatim}[commandchars=\\\{\}]
\PYG{n}{sharpes\PYGZus{}a}\PYG{o}{.}\PYG{n}{iloc}\PYG{p}{[}\PYG{l+m+mi}{1}\PYG{p}{:}\PYG{o}{\PYGZhy{}}\PYG{l+m+mi}{1}\PYG{p}{]}\PYG{o}{.}\PYG{n}{plot}\PYG{p}{(}\PYG{n}{kind}\PYG{o}{=}\PYG{l+s+s1}{\PYGZsq{}}\PYG{l+s+s1}{bar}\PYG{l+s+s1}{\PYGZsq{}}\PYG{p}{)}
\PYG{n}{plt}\PYG{o}{.}\PYG{n}{ylabel}\PYG{p}{(}\PYG{l+s+s1}{\PYGZsq{}}\PYG{l+s+s1}{Sharpe Ratio}\PYG{l+s+s1}{\PYGZsq{}}\PYG{p}{)}
\PYG{n}{plt}\PYG{o}{.}\PYG{n}{title}\PYG{p}{(}\PYG{l+s+s1}{\PYGZsq{}}\PYG{l+s+s1}{Intraday and Overnight Return Decomposition}\PYG{l+s+se}{\PYGZbs{}n}\PYG{l+s+s1}{Calendar Year Sharpe Ratios}\PYG{l+s+s1}{\PYGZsq{}}\PYG{p}{)}
\PYG{n}{plt}\PYG{o}{.}\PYG{n}{show}\PYG{p}{(}\PYG{p}{)}
\end{sphinxVerbatim}

\end{sphinxuseclass}\end{sphinxVerbatimInput}
\begin{sphinxVerbatimOutput}

\begin{sphinxuseclass}{cell_output}
\noindent\sphinxincludegraphics{{7a0daea46b562bb3d445b9d30abf98b5d7aa24e837cacf360629ff41f861503b}.png}

\end{sphinxuseclass}\end{sphinxVerbatimOutput}

\end{sphinxuseclass}

\subsubsection{Calculate rolling betas}
\label{\detokenize{mckinney_11_practice_04:calculate-rolling-betas}}
\sphinxAtStartPar
Calculate rolling capital asset pricing model (CAPM) betas for the MATANA stocks.

\sphinxAtStartPar
The CAPM says the risk premium on a stock depends on the risk\sphinxhyphen{}free rate, beta, and the risk premium on the market: \$E(R\_\{stock\}) = R\_f + \textbackslash{}beta\_\{stock\} \textbackslash{}times (E(R\_\{market\}) \sphinxhyphen{} R\_f)\$.
We can calculate CAPM betas as: \$\textbackslash{}beta\_\{stock\} = \textbackslash{}frac\{Cov(R\_\{stock\} \sphinxhyphen{} R\_f, R\_\{market\} \sphinxhyphen{} R\_f)\}\{Var(R\_\{market\} \sphinxhyphen{} R\_f)\}\$.

\begin{sphinxuseclass}{cell}\begin{sphinxVerbatimInput}

\begin{sphinxuseclass}{cell_input}
\begin{sphinxVerbatim}[commandchars=\\\{\}]
\PYG{n}{matana} \PYG{o}{=} \PYG{p}{(}
    \PYG{n}{yf}\PYG{o}{.}\PYG{n}{download}\PYG{p}{(}\PYG{n}{tickers}\PYG{o}{=}\PYG{l+s+s1}{\PYGZsq{}}\PYG{l+s+s1}{MSFT AAPL TSLA AMZN NVDA GOOG}\PYG{l+s+s1}{\PYGZsq{}}\PYG{p}{,} \PYG{n}{progress}\PYG{o}{=}\PYG{k+kc}{False}\PYG{p}{)}
    \PYG{o}{.}\PYG{n}{assign}\PYG{p}{(}
        \PYG{n}{Date} \PYG{o}{=} \PYG{k}{lambda} \PYG{n}{x}\PYG{p}{:} \PYG{n}{x}\PYG{o}{.}\PYG{n}{index}\PYG{o}{.}\PYG{n}{tz\PYGZus{}localize}\PYG{p}{(}\PYG{k+kc}{None}\PYG{p}{)}\PYG{p}{,}
    \PYG{p}{)}
    \PYG{o}{.}\PYG{n}{set\PYGZus{}index}\PYG{p}{(}\PYG{l+s+s1}{\PYGZsq{}}\PYG{l+s+s1}{Date}\PYG{l+s+s1}{\PYGZsq{}}\PYG{p}{)}
    \PYG{p}{[}\PYG{l+s+s1}{\PYGZsq{}}\PYG{l+s+s1}{Adj Close}\PYG{l+s+s1}{\PYGZsq{}}\PYG{p}{]}
    \PYG{o}{.}\PYG{n}{pct\PYGZus{}change}\PYG{p}{(}\PYG{p}{)}
    \PYG{o}{.}\PYG{n}{rename\PYGZus{}axis}\PYG{p}{(}\PYG{n}{columns}\PYG{o}{=}\PYG{l+s+s1}{\PYGZsq{}}\PYG{l+s+s1}{Ticker}\PYG{l+s+s1}{\PYGZsq{}}\PYG{p}{)}
\PYG{p}{)}

\PYG{n}{matana}\PYG{o}{.}\PYG{n}{head}\PYG{p}{(}\PYG{p}{)}
\end{sphinxVerbatim}

\end{sphinxuseclass}\end{sphinxVerbatimInput}
\begin{sphinxVerbatimOutput}

\begin{sphinxuseclass}{cell_output}
\begin{sphinxVerbatim}[commandchars=\\\{\}]
Ticker        AAPL  AMZN  GOOG  MSFT  NVDA  TSLA
Date                                            
1980\PYGZhy{}12\PYGZhy{}12     NaN   NaN   NaN   NaN   NaN   NaN
1980\PYGZhy{}12\PYGZhy{}15 \PYGZhy{}0.0522   NaN   NaN   NaN   NaN   NaN
1980\PYGZhy{}12\PYGZhy{}16 \PYGZhy{}0.0734   NaN   NaN   NaN   NaN   NaN
1980\PYGZhy{}12\PYGZhy{}17  0.0248   NaN   NaN   NaN   NaN   NaN
1980\PYGZhy{}12\PYGZhy{}18  0.0290   NaN   NaN   NaN   NaN   NaN
\end{sphinxVerbatim}

\end{sphinxuseclass}\end{sphinxVerbatimOutput}

\end{sphinxuseclass}
\begin{sphinxuseclass}{cell}\begin{sphinxVerbatimInput}

\begin{sphinxuseclass}{cell_input}
\begin{sphinxVerbatim}[commandchars=\\\{\}]
\PYG{k}{def} \PYG{n+nf}{beta}\PYG{p}{(}\PYG{n}{ri}\PYG{p}{,} \PYG{n}{rf}\PYG{o}{=}\PYG{n}{ff}\PYG{p}{[}\PYG{l+s+s1}{\PYGZsq{}}\PYG{l+s+s1}{RF}\PYG{l+s+s1}{\PYGZsq{}}\PYG{p}{]}\PYG{p}{,} \PYG{n}{rm\PYGZus{}rf}\PYG{o}{=}\PYG{n}{ff}\PYG{p}{[}\PYG{l+s+s1}{\PYGZsq{}}\PYG{l+s+s1}{Mkt\PYGZhy{}RF}\PYG{l+s+s1}{\PYGZsq{}}\PYG{p}{]}\PYG{p}{)}\PYG{p}{:}
    \PYG{n}{ri\PYGZus{}rf} \PYG{o}{=} \PYG{n}{ri}\PYG{o}{.}\PYG{n}{sub}\PYG{p}{(}\PYG{n}{rf}\PYG{p}{)}\PYG{o}{.}\PYG{n}{dropna}\PYG{p}{(}\PYG{p}{)}
    \PYG{k}{return} \PYG{n}{ri\PYGZus{}rf}\PYG{o}{.}\PYG{n}{cov}\PYG{p}{(}\PYG{n}{rm\PYGZus{}rf}\PYG{p}{)} \PYG{o}{/} \PYG{n}{rm\PYGZus{}rf}\PYG{o}{.}\PYG{n}{loc}\PYG{p}{[}\PYG{n}{ri\PYGZus{}rf}\PYG{o}{.}\PYG{n}{index}\PYG{p}{]}\PYG{o}{.}\PYG{n}{var}\PYG{p}{(}\PYG{p}{)}
\end{sphinxVerbatim}

\end{sphinxuseclass}\end{sphinxVerbatimInput}

\end{sphinxuseclass}
\sphinxAtStartPar
A few notes:
\begin{enumerate}
\sphinxsetlistlabels{\arabic}{enumi}{enumii}{}{.}%
\item {} 
\sphinxAtStartPar
Here the \sphinxcode{\sphinxupquote{beta()}} function is too smart for the \sphinxcode{\sphinxupquote{.rolling()}} method.
Our \sphinxcode{\sphinxupquote{beta()}} function matches stock and market data, but may not use 252 trading days at the end of the sample.
For example, in February 2023, we can slice a 252\sphinxhyphen{}trading\sphinxhyphen{}day window from the \sphinxcode{\sphinxupquote{matana}} data frame.
However, the \sphinxcode{\sphinxupquote{beta()}} function does not use January and Februrary 2023 data because they are missing from the \sphinxcode{\sphinxupquote{ff}} data frame.
The easiest fix is to slice the \sphinxcode{\sphinxupquote{matana}} data frame to remove 2023 data before we apply the \sphinxcode{\sphinxupquote{.rolling()}} method.

\item {} 
\sphinxAtStartPar
The \sphinxcode{\sphinxupquote{.apply()}} method and \sphinxcode{\sphinxupquote{beta()}} function are \sphinxstyleemphasis{very slow} because they slice inefficiently!
We will write faster—but uglier—code below.

\end{enumerate}

\begin{sphinxuseclass}{cell}\begin{sphinxVerbatimInput}

\begin{sphinxuseclass}{cell_input}
\begin{sphinxVerbatim}[commandchars=\\\{\}]
\PYG{o}{\PYGZpc{}\PYGZpc{}time}
\PYG{n}{betas\PYGZus{}1} \PYG{o}{=} \PYG{n}{matana}\PYG{o}{.}\PYG{n}{dropna}\PYG{p}{(}\PYG{p}{)}\PYG{o}{.}\PYG{n}{loc}\PYG{p}{[}\PYG{p}{:}\PYG{l+s+s1}{\PYGZsq{}}\PYG{l+s+s1}{2022}\PYG{l+s+s1}{\PYGZsq{}}\PYG{p}{]}\PYG{o}{.}\PYG{n}{rolling}\PYG{p}{(}\PYG{l+m+mi}{252}\PYG{p}{)}\PYG{o}{.}\PYG{n}{apply}\PYG{p}{(}\PYG{n}{beta}\PYG{p}{)}
\end{sphinxVerbatim}

\end{sphinxuseclass}\end{sphinxVerbatimInput}
\begin{sphinxVerbatimOutput}

\begin{sphinxuseclass}{cell_output}
\begin{sphinxVerbatim}[commandchars=\\\{\}]
Wall time: 24.8 s
\end{sphinxVerbatim}

\end{sphinxuseclass}\end{sphinxVerbatimOutput}

\end{sphinxuseclass}
\sphinxAtStartPar
Here is the fast solution, which uses the optimized \sphinxcode{\sphinxupquote{.cov()}} and \sphinxcode{\sphinxupquote{.var()}} to calculate the numerator and denominator for \$\textbackslash{}beta\$.

\begin{sphinxuseclass}{cell}\begin{sphinxVerbatimInput}

\begin{sphinxuseclass}{cell_input}
\begin{sphinxVerbatim}[commandchars=\\\{\}]
\PYG{o}{\PYGZpc{}\PYGZpc{}time}
\PYG{n}{cov\PYGZus{}2} \PYG{o}{=} \PYG{n}{matana}\PYG{o}{.}\PYG{n}{dropna}\PYG{p}{(}\PYG{p}{)}\PYG{o}{.}\PYG{n}{loc}\PYG{p}{[}\PYG{p}{:}\PYG{l+s+s1}{\PYGZsq{}}\PYG{l+s+s1}{2022}\PYG{l+s+s1}{\PYGZsq{}}\PYG{p}{]}\PYG{o}{.}\PYG{n}{sub}\PYG{p}{(}\PYG{n}{ff}\PYG{p}{[}\PYG{l+s+s1}{\PYGZsq{}}\PYG{l+s+s1}{RF}\PYG{l+s+s1}{\PYGZsq{}}\PYG{p}{]}\PYG{p}{,} \PYG{n}{axis}\PYG{o}{=}\PYG{l+m+mi}{0}\PYG{p}{)}\PYG{o}{.}\PYG{n}{rolling}\PYG{p}{(}\PYG{l+m+mi}{252}\PYG{p}{)}\PYG{o}{.}\PYG{n}{cov}\PYG{p}{(}\PYG{n}{ff}\PYG{p}{[}\PYG{l+s+s1}{\PYGZsq{}}\PYG{l+s+s1}{Mkt\PYGZhy{}RF}\PYG{l+s+s1}{\PYGZsq{}}\PYG{p}{]}\PYG{p}{)}
\PYG{n}{var\PYGZus{}2} \PYG{o}{=} \PYG{n}{ff}\PYG{p}{[}\PYG{l+s+s1}{\PYGZsq{}}\PYG{l+s+s1}{Mkt\PYGZhy{}RF}\PYG{l+s+s1}{\PYGZsq{}}\PYG{p}{]}\PYG{o}{.}\PYG{n}{rolling}\PYG{p}{(}\PYG{l+m+mi}{252}\PYG{p}{)}\PYG{o}{.}\PYG{n}{var}\PYG{p}{(}\PYG{p}{)}
\PYG{n}{betas\PYGZus{}2} \PYG{o}{=} \PYG{n}{cov\PYGZus{}2}\PYG{o}{.}\PYG{n}{div}\PYG{p}{(}\PYG{n}{var\PYGZus{}2}\PYG{p}{,} \PYG{n}{axis}\PYG{o}{=}\PYG{l+m+mi}{0}\PYG{p}{)}
\end{sphinxVerbatim}

\end{sphinxuseclass}\end{sphinxVerbatimInput}
\begin{sphinxVerbatimOutput}

\begin{sphinxuseclass}{cell_output}
\begin{sphinxVerbatim}[commandchars=\\\{\}]
Wall time: 20.9 ms
\end{sphinxVerbatim}

\end{sphinxuseclass}\end{sphinxVerbatimOutput}

\end{sphinxuseclass}
\sphinxAtStartPar
These solutions are identical!
Even though \sphinxcode{\sphinxupquote{beta\_2}} is 1000 times faster than \sphinxcode{\sphinxupquote{beta\_1}}, you should use the approach that makes the most sense to you!

\begin{sphinxuseclass}{cell}\begin{sphinxVerbatimInput}

\begin{sphinxuseclass}{cell_input}
\begin{sphinxVerbatim}[commandchars=\\\{\}]
\PYG{n}{np}\PYG{o}{.}\PYG{n}{allclose}\PYG{p}{(}\PYG{n}{betas\PYGZus{}1}\PYG{o}{.}\PYG{n}{dropna}\PYG{p}{(}\PYG{p}{)}\PYG{p}{,} \PYG{n}{betas\PYGZus{}2}\PYG{o}{.}\PYG{n}{dropna}\PYG{p}{(}\PYG{p}{)}\PYG{p}{)}
\end{sphinxVerbatim}

\end{sphinxuseclass}\end{sphinxVerbatimInput}
\begin{sphinxVerbatimOutput}

\begin{sphinxuseclass}{cell_output}
\begin{sphinxVerbatim}[commandchars=\\\{\}]
True
\end{sphinxVerbatim}

\end{sphinxuseclass}\end{sphinxVerbatimOutput}

\end{sphinxuseclass}
\begin{sphinxuseclass}{cell}\begin{sphinxVerbatimInput}

\begin{sphinxuseclass}{cell_input}
\begin{sphinxVerbatim}[commandchars=\\\{\}]
\PYG{n}{betas\PYGZus{}1}\PYG{o}{.}\PYG{n}{plot}\PYG{p}{(}\PYG{p}{)}
\PYG{n}{plt}\PYG{o}{.}\PYG{n}{ylabel}\PYG{p}{(}\PYG{l+s+s1}{\PYGZsq{}}\PYG{l+s+s1}{Beta}\PYG{l+s+s1}{\PYGZsq{}}\PYG{p}{)}
\PYG{n}{plt}\PYG{o}{.}\PYG{n}{title}\PYG{p}{(}\PYG{l+s+s1}{\PYGZsq{}}\PYG{l+s+s1}{Betas from Daily Returns}\PYG{l+s+se}{\PYGZbs{}n}\PYG{l+s+s1}{252\PYGZhy{}Trading\PYGZhy{}Day Rolling Windows}\PYG{l+s+s1}{\PYGZsq{}}\PYG{p}{)}
\PYG{n}{plt}\PYG{o}{.}\PYG{n}{show}\PYG{p}{(}\PYG{p}{)}
\end{sphinxVerbatim}

\end{sphinxuseclass}\end{sphinxVerbatimInput}
\begin{sphinxVerbatimOutput}

\begin{sphinxuseclass}{cell_output}
\noindent\sphinxincludegraphics{{fa254b5f87bdfba6bd42ce01883937d07ef250c9d46e61931b206d4369c814f6}.png}

\end{sphinxuseclass}\end{sphinxVerbatimOutput}

\end{sphinxuseclass}

\subsubsection{Calculate rolling Sharpe Ratios}
\label{\detokenize{mckinney_11_practice_04:calculate-rolling-sharpe-ratios}}
\sphinxAtStartPar
Calculate rolling Sharpe Ratios for the MATANA stocks.

\sphinxAtStartPar
The Sharpe Ratio is often used to evaluate fund managers.
The Sharpe Ratio is \$SR\_i = \textbackslash{}frac\{\textbackslash{}overline\{R\_i \sphinxhyphen{} R\_f\}\}\{\textbackslash{}sigma\}\$, where \$\textbackslash{}overline\{R\_i\sphinxhyphen{}R\_f\}\$ is mean fund return relative to the risk\sphinxhyphen{}free rate over some period and \$\textbackslash{}sigma\$ is the standard deviation of \$R\_i\sphinxhyphen{}R\_f\$ over the same period.
While the Sharpe Ratio is typically used for funds, we can apply it to a single stock to test our knowledge of the \sphinxcode{\sphinxupquote{.rolling()}} method.
Calculate and plot the one\sphinxhyphen{}year rolling Sharpe Ratio for the MATANA stocks using all available daily data.

\begin{sphinxuseclass}{cell}\begin{sphinxVerbatimInput}

\begin{sphinxuseclass}{cell_input}
\begin{sphinxVerbatim}[commandchars=\\\{\}]
\PYG{k}{def} \PYG{n+nf}{sharpe}\PYG{p}{(}\PYG{n}{ri}\PYG{p}{,} \PYG{n}{rf}\PYG{o}{=}\PYG{n}{ff}\PYG{p}{[}\PYG{l+s+s1}{\PYGZsq{}}\PYG{l+s+s1}{RF}\PYG{l+s+s1}{\PYGZsq{}}\PYG{p}{]}\PYG{p}{,} \PYG{n}{ann\PYGZus{}fac}\PYG{o}{=}\PYG{n}{np}\PYG{o}{.}\PYG{n}{sqrt}\PYG{p}{(}\PYG{l+m+mi}{252}\PYG{p}{)}\PYG{p}{)}\PYG{p}{:}
    \PYG{n}{ri\PYGZus{}rf} \PYG{o}{=} \PYG{n}{ri}\PYG{o}{.}\PYG{n}{sub}\PYG{p}{(}\PYG{n}{rf}\PYG{p}{)}\PYG{o}{.}\PYG{n}{dropna}\PYG{p}{(}\PYG{p}{)}
    \PYG{k}{return} \PYG{n}{ann\PYGZus{}fac} \PYG{o}{*} \PYG{n}{ri\PYGZus{}rf}\PYG{o}{.}\PYG{n}{mean}\PYG{p}{(}\PYG{p}{)} \PYG{o}{/} \PYG{n}{ri\PYGZus{}rf}\PYG{o}{.}\PYG{n}{std}\PYG{p}{(}\PYG{p}{)}
\end{sphinxVerbatim}

\end{sphinxuseclass}\end{sphinxVerbatimInput}

\end{sphinxuseclass}
\sphinxAtStartPar
See the discussion in the beta exercise above about how we can end up with a result based on fewer than 252 trading days of data because the \sphinxcode{\sphinxupquote{sharpe()}} function matches the \sphinxcode{\sphinxupquote{matana}} and \sphinxcode{\sphinxupquote{ff}} data frames \sphinxstyleemphasis{after} the \sphinxcode{\sphinxupquote{.rolling()}} mehtod slices a 252 trading day window.
The simplest solution is the remove the 2023 data from the \sphinxcode{\sphinxupquote{matana}} data frame before we apply the \sphinxcode{\sphinxupquote{.rolling()}} method.

\begin{sphinxuseclass}{cell}\begin{sphinxVerbatimInput}

\begin{sphinxuseclass}{cell_input}
\begin{sphinxVerbatim}[commandchars=\\\{\}]
\PYG{o}{\PYGZpc{}\PYGZpc{}time}
\PYG{n}{sharpes\PYGZus{}1} \PYG{o}{=} \PYG{n}{matana}\PYG{o}{.}\PYG{n}{dropna}\PYG{p}{(}\PYG{p}{)}\PYG{o}{.}\PYG{n}{loc}\PYG{p}{[}\PYG{p}{:}\PYG{l+s+s1}{\PYGZsq{}}\PYG{l+s+s1}{2022}\PYG{l+s+s1}{\PYGZsq{}}\PYG{p}{]}\PYG{o}{.}\PYG{n}{rolling}\PYG{p}{(}\PYG{l+m+mi}{252}\PYG{p}{)}\PYG{o}{.}\PYG{n}{apply}\PYG{p}{(}\PYG{n}{sharpe}\PYG{p}{)}
\end{sphinxVerbatim}

\end{sphinxuseclass}\end{sphinxVerbatimInput}
\begin{sphinxVerbatimOutput}

\begin{sphinxuseclass}{cell_output}
\begin{sphinxVerbatim}[commandchars=\\\{\}]
Wall time: 13 s
\end{sphinxVerbatim}

\end{sphinxuseclass}\end{sphinxVerbatimOutput}

\end{sphinxuseclass}
\sphinxAtStartPar
Here is the fast solution, which:
\begin{enumerate}
\sphinxsetlistlabels{\arabic}{enumi}{enumii}{}{.}%
\item {} 
\sphinxAtStartPar
Uses the \sphinxcode{\sphinxupquote{.sub()}} method to calculate excess returns

\item {} 
\sphinxAtStartPar
Then uses the optimized \sphinxcode{\sphinxupquote{.mean()}} and \sphinxcode{\sphinxupquote{.std()}} methods to calculate the numerator and denominator for the Sharpe Ratios

\end{enumerate}

\begin{sphinxuseclass}{cell}\begin{sphinxVerbatimInput}

\begin{sphinxuseclass}{cell_input}
\begin{sphinxVerbatim}[commandchars=\\\{\}]
\PYG{n}{mean\PYGZus{}2} \PYG{o}{=} \PYG{n}{matana}\PYG{o}{.}\PYG{n}{dropna}\PYG{p}{(}\PYG{p}{)}\PYG{o}{.}\PYG{n}{loc}\PYG{p}{[}\PYG{p}{:}\PYG{l+s+s1}{\PYGZsq{}}\PYG{l+s+s1}{2022}\PYG{l+s+s1}{\PYGZsq{}}\PYG{p}{]}\PYG{o}{.}\PYG{n}{sub}\PYG{p}{(}\PYG{n}{ff}\PYG{p}{[}\PYG{l+s+s1}{\PYGZsq{}}\PYG{l+s+s1}{RF}\PYG{l+s+s1}{\PYGZsq{}}\PYG{p}{]}\PYG{p}{,} \PYG{n}{axis}\PYG{o}{=}\PYG{l+m+mi}{0}\PYG{p}{)}\PYG{o}{.}\PYG{n}{rolling}\PYG{p}{(}\PYG{l+m+mi}{252}\PYG{p}{)}\PYG{o}{.}\PYG{n}{mean}\PYG{p}{(}\PYG{p}{)}
\PYG{n}{std\PYGZus{}2} \PYG{o}{=} \PYG{n}{matana}\PYG{o}{.}\PYG{n}{dropna}\PYG{p}{(}\PYG{p}{)}\PYG{o}{.}\PYG{n}{loc}\PYG{p}{[}\PYG{p}{:}\PYG{l+s+s1}{\PYGZsq{}}\PYG{l+s+s1}{2022}\PYG{l+s+s1}{\PYGZsq{}}\PYG{p}{]}\PYG{o}{.}\PYG{n}{sub}\PYG{p}{(}\PYG{n}{ff}\PYG{p}{[}\PYG{l+s+s1}{\PYGZsq{}}\PYG{l+s+s1}{RF}\PYG{l+s+s1}{\PYGZsq{}}\PYG{p}{]}\PYG{p}{,} \PYG{n}{axis}\PYG{o}{=}\PYG{l+m+mi}{0}\PYG{p}{)}\PYG{o}{.}\PYG{n}{rolling}\PYG{p}{(}\PYG{l+m+mi}{252}\PYG{p}{)}\PYG{o}{.}\PYG{n}{std}\PYG{p}{(}\PYG{p}{)}
\PYG{n}{sharpes\PYGZus{}2} \PYG{o}{=} \PYG{n}{mean\PYGZus{}2}\PYG{o}{.}\PYG{n}{div}\PYG{p}{(}\PYG{n}{std\PYGZus{}2}\PYG{p}{,} \PYG{n}{axis}\PYG{o}{=}\PYG{l+m+mi}{0}\PYG{p}{)}\PYG{o}{.}\PYG{n}{mul}\PYG{p}{(}\PYG{n}{np}\PYG{o}{.}\PYG{n}{sqrt}\PYG{p}{(}\PYG{l+m+mi}{252}\PYG{p}{)}\PYG{p}{)}
\end{sphinxVerbatim}

\end{sphinxuseclass}\end{sphinxVerbatimInput}

\end{sphinxuseclass}
\begin{sphinxuseclass}{cell}\begin{sphinxVerbatimInput}

\begin{sphinxuseclass}{cell_input}
\begin{sphinxVerbatim}[commandchars=\\\{\}]
\PYG{n}{np}\PYG{o}{.}\PYG{n}{allclose}\PYG{p}{(}\PYG{n}{sharpes\PYGZus{}1}\PYG{o}{.}\PYG{n}{dropna}\PYG{p}{(}\PYG{p}{)}\PYG{p}{,} \PYG{n}{sharpes\PYGZus{}2}\PYG{o}{.}\PYG{n}{dropna}\PYG{p}{(}\PYG{p}{)}\PYG{p}{)}
\end{sphinxVerbatim}

\end{sphinxuseclass}\end{sphinxVerbatimInput}
\begin{sphinxVerbatimOutput}

\begin{sphinxuseclass}{cell_output}
\begin{sphinxVerbatim}[commandchars=\\\{\}]
True
\end{sphinxVerbatim}

\end{sphinxuseclass}\end{sphinxVerbatimOutput}

\end{sphinxuseclass}
\begin{sphinxuseclass}{cell}\begin{sphinxVerbatimInput}

\begin{sphinxuseclass}{cell_input}
\begin{sphinxVerbatim}[commandchars=\\\{\}]
\PYG{n}{sharpes\PYGZus{}1}\PYG{o}{.}\PYG{n}{plot}\PYG{p}{(}\PYG{p}{)}
\PYG{n}{plt}\PYG{o}{.}\PYG{n}{ylabel}\PYG{p}{(}\PYG{l+s+s1}{\PYGZsq{}}\PYG{l+s+s1}{Sharpe Ratio}\PYG{l+s+s1}{\PYGZsq{}}\PYG{p}{)}
\PYG{n}{plt}\PYG{o}{.}\PYG{n}{title}\PYG{p}{(}\PYG{l+s+s1}{\PYGZsq{}}\PYG{l+s+s1}{Sharpe Ratios from Daily Returns}\PYG{l+s+se}{\PYGZbs{}n}\PYG{l+s+s1}{252\PYGZhy{}Trading\PYGZhy{}Day Rolling Windows}\PYG{l+s+s1}{\PYGZsq{}}\PYG{p}{)}
\PYG{n}{plt}\PYG{o}{.}\PYG{n}{show}\PYG{p}{(}\PYG{p}{)}
\end{sphinxVerbatim}

\end{sphinxuseclass}\end{sphinxVerbatimInput}
\begin{sphinxVerbatimOutput}

\begin{sphinxuseclass}{cell_output}
\noindent\sphinxincludegraphics{{1d57bfaf9da951ace6159176d2c88cb345d6b95e852f5a06f7ad29b49a82f99d}.png}

\end{sphinxuseclass}\end{sphinxVerbatimOutput}

\end{sphinxuseclass}

\subsubsection{Does more frequent rebalancing increase or decrease returns?}
\label{\detokenize{mckinney_11_practice_04:does-more-frequent-rebalancing-increase-or-decrease-returns}}
\sphinxAtStartPar
Compare decade\sphinxhyphen{}total returns for the following rebalancing frequencies:
\begin{enumerate}
\sphinxsetlistlabels{\arabic}{enumi}{enumii}{}{.}%
\item {} 
\sphinxAtStartPar
Daily rebalancing

\item {} 
\sphinxAtStartPar
Monthly rebalancing

\item {} 
\sphinxAtStartPar
Annual rebalancing

\item {} 
\sphinxAtStartPar
Decade rebalancing

\end{enumerate}

\sphinxAtStartPar
Use equally\sphinxhyphen{}weighted portfolios of industry\sphinxhyphen{}level daily returns from French’s website: \sphinxcode{\sphinxupquote{'17\_Industry\_Portfolios\_daily'}}.

\sphinxAtStartPar
\sphinxstyleemphasis{\sphinxstylestrong{This exercise is a little early, and we will revisit later in the course!}}

\sphinxstepscope


\section{McKinney Chapter 11 \sphinxhyphen{} Practice (Wednesday 2:45 PM, Section 2)}
\label{\detokenize{mckinney_11_practice_02:mckinney-chapter-11-practice-wednesday-2-45-pm-section-2}}\label{\detokenize{mckinney_11_practice_02::doc}}

\subsection{Announcements}
\label{\detokenize{mckinney_11_practice_02:announcements}}\begin{itemize}
\item {} 
\sphinxAtStartPar
Quiz 4 mean was \$87\%\$

\item {} 
\sphinxAtStartPar
Project 1 and Teammates Review 1 at due Friday at 11:59 PM

\end{itemize}


\subsection{Practice}
\label{\detokenize{mckinney_11_practice_02:practice}}
\begin{sphinxuseclass}{cell}\begin{sphinxVerbatimInput}

\begin{sphinxuseclass}{cell_input}
\begin{sphinxVerbatim}[commandchars=\\\{\}]
\PYG{k+kn}{import} \PYG{n+nn}{pandas} \PYG{k}{as} \PYG{n+nn}{pd}
\PYG{k+kn}{import} \PYG{n+nn}{numpy} \PYG{k}{as} \PYG{n+nn}{np}
\PYG{k+kn}{import} \PYG{n+nn}{matplotlib}\PYG{n+nn}{.}\PYG{n+nn}{pyplot} \PYG{k}{as} \PYG{n+nn}{plt}
\end{sphinxVerbatim}

\end{sphinxuseclass}\end{sphinxVerbatimInput}

\end{sphinxuseclass}
\begin{sphinxuseclass}{cell}\begin{sphinxVerbatimInput}

\begin{sphinxuseclass}{cell_input}
\begin{sphinxVerbatim}[commandchars=\\\{\}]
\PYG{o}{\PYGZpc{}}\PYG{k}{config} InlineBackend.figure\PYGZus{}format = \PYGZsq{}retina\PYGZsq{}
\PYG{o}{\PYGZpc{}}\PYG{k}{precision} 4
\PYG{n}{pd}\PYG{o}{.}\PYG{n}{options}\PYG{o}{.}\PYG{n}{display}\PYG{o}{.}\PYG{n}{float\PYGZus{}format} \PYG{o}{=} \PYG{l+s+s1}{\PYGZsq{}}\PYG{l+s+si}{\PYGZob{}:.4f\PYGZcb{}}\PYG{l+s+s1}{\PYGZsq{}}\PYG{o}{.}\PYG{n}{format}
\end{sphinxVerbatim}

\end{sphinxuseclass}\end{sphinxVerbatimInput}

\end{sphinxuseclass}
\begin{sphinxuseclass}{cell}\begin{sphinxVerbatimInput}

\begin{sphinxuseclass}{cell_input}
\begin{sphinxVerbatim}[commandchars=\\\{\}]
\PYG{k+kn}{import} \PYG{n+nn}{yfinance} \PYG{k}{as} \PYG{n+nn}{yf}
\PYG{k+kn}{import} \PYG{n+nn}{pandas\PYGZus{}datareader} \PYG{k}{as} \PYG{n+nn}{pdr}
\PYG{k+kn}{import} \PYG{n+nn}{requests\PYGZus{}cache}
\PYG{n}{session} \PYG{o}{=} \PYG{n}{requests\PYGZus{}cache}\PYG{o}{.}\PYG{n}{CachedSession}\PYG{p}{(}\PYG{p}{)}
\end{sphinxVerbatim}

\end{sphinxuseclass}\end{sphinxVerbatimInput}

\end{sphinxuseclass}

\subsubsection{Which are larger, overnight or intraday returns?}
\label{\detokenize{mckinney_11_practice_02:which-are-larger-overnight-or-intraday-returns}}
\sphinxAtStartPar
Yahoo! Finance provides easy acces to high\sphinxhyphen{}quality open, high, low, close (OHLC) and adjusted close price data.
However, Yahoo! Finance does not provide overnight or instraday returns directly.
Therefore, we need to use math to decompose daily returns into overnight and intraday returns.

\sphinxAtStartPar
Daily returns are defined as (adjusted) closing price to (adjusted) closing price returns.
Therefore, daily returns consist of overnight returns compounded with the intraday returns from the next day \$(1 + R\_\{daily\}) = (1 + R\_\{overnight\}) \textbackslash{}times (1 + R\_\{intraday\})\$ which we can rearrange to calculate overnight returns as \$\textbackslash{}frac\{1 + R\_\{daily\}\}\{1 + R\_\{intraday\}\} \sphinxhyphen{} 1 = R\_\{overnight\}\$.

\sphinxAtStartPar
We can calculate daily and intraday returns from Yahoo! Finance data as \$R\_\{daily\} = \textbackslash{}frac\{Adj\textbackslash{} Close\_\{t\} \sphinxhyphen{} Adj\textbackslash{} Close\_\{t\sphinxhyphen{}1\}\}\{Adj\textbackslash{} Close\_\{t\sphinxhyphen{}1\}\}\$ and \$R\_\{intraday\} = \textbackslash{}frac\{Close \sphinxhyphen{} Open\}\{Open\}\$.

\sphinxAtStartPar
Compare the following for the SPY ETF:
\begin{enumerate}
\sphinxsetlistlabels{\arabic}{enumi}{enumii}{}{.}%
\item {} 
\sphinxAtStartPar
Cumulative returns with all available data

\item {} 
\sphinxAtStartPar
Total returns for each calendar year

\item {} 
\sphinxAtStartPar
Total returns over rolling 252\sphinxhyphen{}trading\sphinxhyphen{}day windows

\item {} 
\sphinxAtStartPar
Total returns over rolling 12\sphinxhyphen{}months windows after calculating monthly returns

\item {} 
\sphinxAtStartPar
Sharpe Ratios for each calendar year

\end{enumerate}


\paragraph{Cumulative returns with all available data}
\label{\detokenize{mckinney_11_practice_02:cumulative-returns-with-all-available-data}}
\begin{sphinxuseclass}{cell}\begin{sphinxVerbatimInput}

\begin{sphinxuseclass}{cell_input}
\begin{sphinxVerbatim}[commandchars=\\\{\}]
\PYG{n}{spy} \PYG{o}{=} \PYG{n}{yf}\PYG{o}{.}\PYG{n}{download}\PYG{p}{(}\PYG{n}{tickers}\PYG{o}{=}\PYG{l+s+s1}{\PYGZsq{}}\PYG{l+s+s1}{spy}\PYG{l+s+s1}{\PYGZsq{}}\PYG{p}{,} \PYG{n}{progress}\PYG{o}{=}\PYG{k+kc}{False}\PYG{p}{)}
\end{sphinxVerbatim}

\end{sphinxuseclass}\end{sphinxVerbatimInput}

\end{sphinxuseclass}
\begin{sphinxuseclass}{cell}\begin{sphinxVerbatimInput}

\begin{sphinxuseclass}{cell_input}
\begin{sphinxVerbatim}[commandchars=\\\{\}]
\PYG{n}{returns} \PYG{o}{=} \PYG{p}{(}
    \PYG{n}{spy}
    \PYG{o}{.}\PYG{n}{assign}\PYG{p}{(}
        \PYG{n}{Date} \PYG{o}{=} \PYG{k}{lambda} \PYG{n}{x}\PYG{p}{:} \PYG{n}{x}\PYG{o}{.}\PYG{n}{index}\PYG{o}{.}\PYG{n}{tz\PYGZus{}localize}\PYG{p}{(}\PYG{k+kc}{None}\PYG{p}{)}\PYG{p}{,}
        \PYG{n}{Total} \PYG{o}{=} \PYG{k}{lambda} \PYG{n}{x}\PYG{p}{:} \PYG{n}{x}\PYG{p}{[}\PYG{l+s+s1}{\PYGZsq{}}\PYG{l+s+s1}{Adj Close}\PYG{l+s+s1}{\PYGZsq{}}\PYG{p}{]}\PYG{o}{.}\PYG{n}{pct\PYGZus{}change}\PYG{p}{(}\PYG{p}{)}\PYG{p}{,}
        \PYG{n}{Intraday} \PYG{o}{=} \PYG{k}{lambda} \PYG{n}{x}\PYG{p}{:} \PYG{p}{(}\PYG{n}{x}\PYG{p}{[}\PYG{l+s+s1}{\PYGZsq{}}\PYG{l+s+s1}{Close}\PYG{l+s+s1}{\PYGZsq{}}\PYG{p}{]} \PYG{o}{/} \PYG{n}{x}\PYG{p}{[}\PYG{l+s+s1}{\PYGZsq{}}\PYG{l+s+s1}{Open}\PYG{l+s+s1}{\PYGZsq{}}\PYG{p}{]}\PYG{p}{)} \PYG{o}{\PYGZhy{}} \PYG{l+m+mi}{1}\PYG{p}{,}
        \PYG{n}{Overnight} \PYG{o}{=} \PYG{k}{lambda} \PYG{n}{x}\PYG{p}{:} \PYG{p}{(}\PYG{p}{(}\PYG{l+m+mi}{1} \PYG{o}{+} \PYG{n}{x}\PYG{p}{[}\PYG{l+s+s1}{\PYGZsq{}}\PYG{l+s+s1}{Total}\PYG{l+s+s1}{\PYGZsq{}}\PYG{p}{]}\PYG{p}{)} \PYG{o}{/} \PYG{p}{(}\PYG{l+m+mi}{1} \PYG{o}{+} \PYG{n}{x}\PYG{p}{[}\PYG{l+s+s1}{\PYGZsq{}}\PYG{l+s+s1}{Intraday}\PYG{l+s+s1}{\PYGZsq{}}\PYG{p}{]}\PYG{p}{)}\PYG{p}{)} \PYG{o}{\PYGZhy{}} \PYG{l+m+mi}{1}
    \PYG{p}{)}
    \PYG{o}{.}\PYG{n}{set\PYGZus{}index}\PYG{p}{(}\PYG{l+s+s1}{\PYGZsq{}}\PYG{l+s+s1}{Date}\PYG{l+s+s1}{\PYGZsq{}}\PYG{p}{)}
    \PYG{p}{[}\PYG{p}{[}\PYG{l+s+s1}{\PYGZsq{}}\PYG{l+s+s1}{Intraday}\PYG{l+s+s1}{\PYGZsq{}}\PYG{p}{,} \PYG{l+s+s1}{\PYGZsq{}}\PYG{l+s+s1}{Overnight}\PYG{l+s+s1}{\PYGZsq{}}\PYG{p}{,} \PYG{l+s+s1}{\PYGZsq{}}\PYG{l+s+s1}{Total}\PYG{l+s+s1}{\PYGZsq{}}\PYG{p}{]}\PYG{p}{]}
    \PYG{o}{.}\PYG{n}{dropna}\PYG{p}{(}\PYG{p}{)}
    \PYG{o}{.}\PYG{n}{rename\PYGZus{}axis}\PYG{p}{(}\PYG{n}{columns}\PYG{o}{=}\PYG{l+s+s1}{\PYGZsq{}}\PYG{l+s+s1}{Return}\PYG{l+s+s1}{\PYGZsq{}}\PYG{p}{)}
\PYG{p}{)}

\PYG{n}{returns}\PYG{o}{.}\PYG{n}{head}\PYG{p}{(}\PYG{p}{)}
\end{sphinxVerbatim}

\end{sphinxuseclass}\end{sphinxVerbatimInput}
\begin{sphinxVerbatimOutput}

\begin{sphinxuseclass}{cell_output}
\begin{sphinxVerbatim}[commandchars=\\\{\}]
Return      Intraday  Overnight   Total
Date                                   
1993\PYGZhy{}02\PYGZhy{}01    0.0064     0.0007  0.0071
1993\PYGZhy{}02\PYGZhy{}02    0.0028    \PYGZhy{}0.0007  0.0021
1993\PYGZhy{}02\PYGZhy{}03    0.0091     0.0014  0.0106
1993\PYGZhy{}02\PYGZhy{}04    0.0007     0.0035  0.0042
1993\PYGZhy{}02\PYGZhy{}05    0.0000    \PYGZhy{}0.0007 \PYGZhy{}0.0007
\end{sphinxVerbatim}

\end{sphinxuseclass}\end{sphinxVerbatimOutput}

\end{sphinxuseclass}
\begin{sphinxuseclass}{cell}\begin{sphinxVerbatimInput}

\begin{sphinxuseclass}{cell_input}
\begin{sphinxVerbatim}[commandchars=\\\{\}]
\PYG{n}{returns}\PYG{o}{.}\PYG{n}{add}\PYG{p}{(}\PYG{l+m+mi}{1}\PYG{p}{)}\PYG{o}{.}\PYG{n}{prod}\PYG{p}{(}\PYG{p}{)}\PYG{o}{.}\PYG{n}{sub}\PYG{p}{(}\PYG{l+m+mi}{1}\PYG{p}{)}\PYG{o}{.}\PYG{n}{mul}\PYG{p}{(}\PYG{l+m+mi}{100}\PYG{p}{)}
\end{sphinxVerbatim}

\end{sphinxuseclass}\end{sphinxVerbatimInput}
\begin{sphinxVerbatimOutput}

\begin{sphinxuseclass}{cell_output}
\begin{sphinxVerbatim}[commandchars=\\\{\}]
Return
Intraday       1.2913
Overnight   1445.3413
Total       1465.2966
dtype: float64
\end{sphinxVerbatim}

\end{sphinxuseclass}\end{sphinxVerbatimOutput}

\end{sphinxuseclass}
\begin{sphinxuseclass}{cell}\begin{sphinxVerbatimInput}

\begin{sphinxuseclass}{cell_input}
\begin{sphinxVerbatim}[commandchars=\\\{\}]
\PYG{n}{returns}\PYG{o}{.}\PYG{n}{mean}\PYG{p}{(}\PYG{p}{)}\PYG{o}{.}\PYG{n}{mul}\PYG{p}{(}\PYG{l+m+mi}{100}\PYG{p}{)}
\end{sphinxVerbatim}

\end{sphinxuseclass}\end{sphinxVerbatimInput}
\begin{sphinxVerbatimOutput}

\begin{sphinxuseclass}{cell_output}
\begin{sphinxVerbatim}[commandchars=\\\{\}]
Return
Intraday    0.0049
Overnight   0.0384
Total       0.0434
dtype: float64
\end{sphinxVerbatim}

\end{sphinxuseclass}\end{sphinxVerbatimOutput}

\end{sphinxuseclass}
\begin{sphinxuseclass}{cell}\begin{sphinxVerbatimInput}

\begin{sphinxuseclass}{cell_input}
\begin{sphinxVerbatim}[commandchars=\\\{\}]
\PYG{n}{returns}\PYG{o}{.}\PYG{n}{add}\PYG{p}{(}\PYG{l+m+mi}{1}\PYG{p}{)}\PYG{o}{.}\PYG{n}{cumprod}\PYG{p}{(}\PYG{p}{)}\PYG{o}{.}\PYG{n}{sub}\PYG{p}{(}\PYG{l+m+mi}{1}\PYG{p}{)}\PYG{o}{.}\PYG{n}{mul}\PYG{p}{(}\PYG{l+m+mi}{100}\PYG{p}{)}\PYG{o}{.}\PYG{n}{plot}\PYG{p}{(}\PYG{p}{)}
\PYG{n}{plt}\PYG{o}{.}\PYG{n}{ylabel}\PYG{p}{(}\PYG{l+s+s1}{\PYGZsq{}}\PYG{l+s+s1}{Cumulative Return (}\PYG{l+s+s1}{\PYGZpc{}}\PYG{l+s+s1}{)}\PYG{l+s+s1}{\PYGZsq{}}\PYG{p}{)}
\PYG{n}{plt}\PYG{o}{.}\PYG{n}{title}\PYG{p}{(}\PYG{l+s+s1}{\PYGZsq{}}\PYG{l+s+s1}{Intraday and Overnight Return Decomposition}\PYG{l+s+se}{\PYGZbs{}n}\PYG{l+s+s1}{Cumulative Returns}\PYG{l+s+s1}{\PYGZsq{}}\PYG{p}{)}
\PYG{n}{plt}\PYG{o}{.}\PYG{n}{show}\PYG{p}{(}\PYG{p}{)}
\end{sphinxVerbatim}

\end{sphinxuseclass}\end{sphinxVerbatimInput}
\begin{sphinxVerbatimOutput}

\begin{sphinxuseclass}{cell_output}
\noindent\sphinxincludegraphics{{4637dc5fc25d64889a953edb1af421e4e54de699eb4b79cdcedfaf0f259297a1}.png}

\end{sphinxuseclass}\end{sphinxVerbatimOutput}

\end{sphinxuseclass}

\paragraph{Total returns for each calendar year}
\label{\detokenize{mckinney_11_practice_02:total-returns-for-each-calendar-year}}
\begin{sphinxuseclass}{cell}\begin{sphinxVerbatimInput}

\begin{sphinxuseclass}{cell_input}
\begin{sphinxVerbatim}[commandchars=\\\{\}]
\PYG{k}{def} \PYG{n+nf}{totret}\PYG{p}{(}\PYG{n}{x}\PYG{p}{)}\PYG{p}{:}
    \PYG{k}{return} \PYG{p}{(}\PYG{l+m+mi}{1} \PYG{o}{+} \PYG{n}{x}\PYG{p}{)}\PYG{o}{.}\PYG{n}{prod}\PYG{p}{(}\PYG{p}{)} \PYG{o}{\PYGZhy{}} \PYG{l+m+mi}{1}
\end{sphinxVerbatim}

\end{sphinxuseclass}\end{sphinxVerbatimInput}

\end{sphinxuseclass}
\begin{sphinxuseclass}{cell}\begin{sphinxVerbatimInput}

\begin{sphinxuseclass}{cell_input}
\begin{sphinxVerbatim}[commandchars=\\\{\}]
\PYG{n}{returns\PYGZus{}a} \PYG{o}{=} \PYG{n}{returns}\PYG{o}{.}\PYG{n}{resample}\PYG{p}{(}\PYG{l+s+s1}{\PYGZsq{}}\PYG{l+s+s1}{A}\PYG{l+s+s1}{\PYGZsq{}}\PYG{p}{,} \PYG{n}{kind}\PYG{o}{=}\PYG{l+s+s1}{\PYGZsq{}}\PYG{l+s+s1}{period}\PYG{l+s+s1}{\PYGZsq{}}\PYG{p}{)}\PYG{o}{.}\PYG{n}{agg}\PYG{p}{(}\PYG{n}{totret}\PYG{p}{)}
\end{sphinxVerbatim}

\end{sphinxuseclass}\end{sphinxVerbatimInput}

\end{sphinxuseclass}
\sphinxAtStartPar
We can use \sphinxcode{\sphinxupquote{.count()}} to remove years with too few observations.

\begin{sphinxuseclass}{cell}\begin{sphinxVerbatimInput}

\begin{sphinxuseclass}{cell_input}
\begin{sphinxVerbatim}[commandchars=\\\{\}]
\PYG{n}{counts\PYGZus{}a} \PYG{o}{=} \PYG{n}{returns}\PYG{o}{.}\PYG{n}{resample}\PYG{p}{(}\PYG{l+s+s1}{\PYGZsq{}}\PYG{l+s+s1}{A}\PYG{l+s+s1}{\PYGZsq{}}\PYG{p}{,} \PYG{n}{kind}\PYG{o}{=}\PYG{l+s+s1}{\PYGZsq{}}\PYG{l+s+s1}{period}\PYG{l+s+s1}{\PYGZsq{}}\PYG{p}{)}\PYG{o}{.}\PYG{n}{count}\PYG{p}{(}\PYG{p}{)}
\end{sphinxVerbatim}

\end{sphinxuseclass}\end{sphinxVerbatimInput}

\end{sphinxuseclass}
\begin{sphinxuseclass}{cell}\begin{sphinxVerbatimInput}

\begin{sphinxuseclass}{cell_input}
\begin{sphinxVerbatim}[commandchars=\\\{\}]
\PYG{n}{returns\PYGZus{}a}\PYG{o}{.}\PYG{n}{loc}\PYG{p}{[}\PYG{n}{counts\PYGZus{}a}\PYG{p}{[}\PYG{l+s+s1}{\PYGZsq{}}\PYG{l+s+s1}{Total}\PYG{l+s+s1}{\PYGZsq{}}\PYG{p}{]} \PYG{o}{\PYGZgt{}}\PYG{o}{=} \PYG{l+m+mi}{250} \PYG{p}{]}\PYG{o}{.}\PYG{n}{mul}\PYG{p}{(}\PYG{l+m+mi}{100}\PYG{p}{)}\PYG{o}{.}\PYG{n}{plot}\PYG{p}{(}\PYG{n}{kind}\PYG{o}{=}\PYG{l+s+s1}{\PYGZsq{}}\PYG{l+s+s1}{bar}\PYG{l+s+s1}{\PYGZsq{}}\PYG{p}{)}
\PYG{n}{plt}\PYG{o}{.}\PYG{n}{ylabel}\PYG{p}{(}\PYG{l+s+s1}{\PYGZsq{}}\PYG{l+s+s1}{Total Return (}\PYG{l+s+s1}{\PYGZpc{}}\PYG{l+s+s1}{)}\PYG{l+s+s1}{\PYGZsq{}}\PYG{p}{)}
\PYG{n}{plt}\PYG{o}{.}\PYG{n}{title}\PYG{p}{(}\PYG{l+s+s1}{\PYGZsq{}}\PYG{l+s+s1}{Intraday and Overnight Return Decomposition}\PYG{l+s+se}{\PYGZbs{}n}\PYG{l+s+s1}{Calendar Year Total Returns}\PYG{l+s+s1}{\PYGZsq{}}\PYG{p}{)}
\PYG{n}{plt}\PYG{o}{.}\PYG{n}{show}\PYG{p}{(}\PYG{p}{)}
\end{sphinxVerbatim}

\end{sphinxuseclass}\end{sphinxVerbatimInput}
\begin{sphinxVerbatimOutput}

\begin{sphinxuseclass}{cell_output}
\noindent\sphinxincludegraphics{{2e2e24a0c13573a77d5d424c1fa0b0a8d52191dc859ea5d9dae85afca7da1ea5}.png}

\end{sphinxuseclass}\end{sphinxVerbatimOutput}

\end{sphinxuseclass}
\begin{sphinxuseclass}{cell}\begin{sphinxVerbatimInput}

\begin{sphinxuseclass}{cell_input}
\begin{sphinxVerbatim}[commandchars=\\\{\}]
\PYG{p}{(}\PYG{n}{returns\PYGZus{}a}\PYG{p}{[}\PYG{l+s+s1}{\PYGZsq{}}\PYG{l+s+s1}{Overnight}\PYG{l+s+s1}{\PYGZsq{}}\PYG{p}{]} \PYG{o}{\PYGZgt{}} \PYG{n}{returns\PYGZus{}a}\PYG{p}{[}\PYG{l+s+s1}{\PYGZsq{}}\PYG{l+s+s1}{Intraday}\PYG{l+s+s1}{\PYGZsq{}}\PYG{p}{]}\PYG{p}{)}\PYG{o}{.}\PYG{n}{mean}\PYG{p}{(}\PYG{p}{)}
\end{sphinxVerbatim}

\end{sphinxuseclass}\end{sphinxVerbatimInput}
\begin{sphinxVerbatimOutput}

\begin{sphinxuseclass}{cell_output}
\begin{sphinxVerbatim}[commandchars=\\\{\}]
0.7096774193548387
\end{sphinxVerbatim}

\end{sphinxuseclass}\end{sphinxVerbatimOutput}

\end{sphinxuseclass}

\paragraph{Total returns over rolling 252\sphinxhyphen{}trading\sphinxhyphen{}day windows}
\label{\detokenize{mckinney_11_practice_02:total-returns-over-rolling-252-trading-day-windows}}
\sphinxAtStartPar
We can repeat the total return calculation for 252\sphinxhyphen{}trading\sphinxhyphen{}day rolling windows.

\begin{sphinxuseclass}{cell}\begin{sphinxVerbatimInput}

\begin{sphinxuseclass}{cell_input}
\begin{sphinxVerbatim}[commandchars=\\\{\}]
\PYG{n}{returns\PYGZus{}r} \PYG{o}{=} \PYG{n}{returns}\PYG{o}{.}\PYG{n}{rolling}\PYG{p}{(}\PYG{l+m+mi}{252}\PYG{p}{)}\PYG{o}{.}\PYG{n}{apply}\PYG{p}{(}\PYG{n}{totret}\PYG{p}{)}

\PYG{n}{returns\PYGZus{}r}\PYG{o}{.}\PYG{n}{head}\PYG{p}{(}\PYG{p}{)}
\end{sphinxVerbatim}

\end{sphinxuseclass}\end{sphinxVerbatimInput}
\begin{sphinxVerbatimOutput}

\begin{sphinxuseclass}{cell_output}
\begin{sphinxVerbatim}[commandchars=\\\{\}]
Return      Intraday  Overnight  Total
Date                                  
1993\PYGZhy{}02\PYGZhy{}01       NaN        NaN    NaN
1993\PYGZhy{}02\PYGZhy{}02       NaN        NaN    NaN
1993\PYGZhy{}02\PYGZhy{}03       NaN        NaN    NaN
1993\PYGZhy{}02\PYGZhy{}04       NaN        NaN    NaN
1993\PYGZhy{}02\PYGZhy{}05       NaN        NaN    NaN
\end{sphinxVerbatim}

\end{sphinxuseclass}\end{sphinxVerbatimOutput}

\end{sphinxuseclass}
\begin{sphinxuseclass}{cell}\begin{sphinxVerbatimInput}

\begin{sphinxuseclass}{cell_input}
\begin{sphinxVerbatim}[commandchars=\\\{\}]
\PYG{n}{returns\PYGZus{}r}\PYG{o}{.}\PYG{n}{mul}\PYG{p}{(}\PYG{l+m+mi}{100}\PYG{p}{)}\PYG{o}{.}\PYG{n}{plot}\PYG{p}{(}\PYG{p}{)}
\PYG{n}{plt}\PYG{o}{.}\PYG{n}{ylabel}\PYG{p}{(}\PYG{l+s+s1}{\PYGZsq{}}\PYG{l+s+s1}{Total Return (}\PYG{l+s+s1}{\PYGZpc{}}\PYG{l+s+s1}{)}\PYG{l+s+s1}{\PYGZsq{}}\PYG{p}{)}
\PYG{n}{plt}\PYG{o}{.}\PYG{n}{title}\PYG{p}{(}\PYG{l+s+s1}{\PYGZsq{}}\PYG{l+s+s1}{Intraday and Overnight Return Decomposition}\PYG{l+s+se}{\PYGZbs{}n}\PYG{l+s+s1}{252\PYGZhy{}Trading\PYGZhy{}Day Rolling Windows}\PYG{l+s+s1}{\PYGZsq{}}\PYG{p}{)}
\PYG{n}{plt}\PYG{o}{.}\PYG{n}{show}\PYG{p}{(}\PYG{p}{)}
\end{sphinxVerbatim}

\end{sphinxuseclass}\end{sphinxVerbatimInput}
\begin{sphinxVerbatimOutput}

\begin{sphinxuseclass}{cell_output}
\noindent\sphinxincludegraphics{{c9b1c36c6c7d1cbb6121e5ff94e880ecce649bcdc60454fb0588acff52405002}.png}

\end{sphinxuseclass}\end{sphinxVerbatimOutput}

\end{sphinxuseclass}

\paragraph{Total returns over rolling 12\sphinxhyphen{}months windows after calculating monthly returns}
\label{\detokenize{mckinney_11_practice_02:total-returns-over-rolling-12-months-windows-after-calculating-monthly-returns}}
\sphinxAtStartPar
We can chain this operation!
Note the plot has the same general appearance, but is less noisy because we aggregate to monthly total returns first!

\begin{sphinxuseclass}{cell}\begin{sphinxVerbatimInput}

\begin{sphinxuseclass}{cell_input}
\begin{sphinxVerbatim}[commandchars=\\\{\}]
\PYG{n}{returns\PYGZus{}mr} \PYG{o}{=} \PYG{p}{(}
    \PYG{n}{returns}
    \PYG{o}{.}\PYG{n}{resample}\PYG{p}{(}\PYG{n}{rule}\PYG{o}{=}\PYG{l+s+s1}{\PYGZsq{}}\PYG{l+s+s1}{M}\PYG{l+s+s1}{\PYGZsq{}}\PYG{p}{,} \PYG{n}{kind}\PYG{o}{=}\PYG{l+s+s1}{\PYGZsq{}}\PYG{l+s+s1}{period}\PYG{l+s+s1}{\PYGZsq{}}\PYG{p}{)} \PYG{c+c1}{\PYGZsh{} aggregate from daily to monthly}
    \PYG{o}{.}\PYG{n}{apply}\PYG{p}{(}\PYG{n}{totret}\PYG{p}{)} \PYG{c+c1}{\PYGZsh{} ...calculate total returns}
    \PYG{o}{.}\PYG{n}{rolling}\PYG{p}{(}\PYG{l+m+mi}{12}\PYG{p}{)} \PYG{c+c1}{\PYGZsh{} aggregate into 12\PYGZhy{}month rolling windows}
    \PYG{o}{.}\PYG{n}{apply}\PYG{p}{(}\PYG{n}{totret}\PYG{p}{)} \PYG{c+c1}{\PYGZsh{} ...calculate total returns}
\PYG{p}{)}
\end{sphinxVerbatim}

\end{sphinxuseclass}\end{sphinxVerbatimInput}

\end{sphinxuseclass}
\begin{sphinxuseclass}{cell}\begin{sphinxVerbatimInput}

\begin{sphinxuseclass}{cell_input}
\begin{sphinxVerbatim}[commandchars=\\\{\}]
\PYG{n}{returns\PYGZus{}mr}\PYG{o}{.}\PYG{n}{mul}\PYG{p}{(}\PYG{l+m+mi}{100}\PYG{p}{)}\PYG{o}{.}\PYG{n}{plot}\PYG{p}{(}\PYG{p}{)}
\PYG{n}{plt}\PYG{o}{.}\PYG{n}{ylabel}\PYG{p}{(}\PYG{l+s+s1}{\PYGZsq{}}\PYG{l+s+s1}{Total Return (}\PYG{l+s+s1}{\PYGZpc{}}\PYG{l+s+s1}{)}\PYG{l+s+s1}{\PYGZsq{}}\PYG{p}{)}
\PYG{n}{plt}\PYG{o}{.}\PYG{n}{title}\PYG{p}{(}\PYG{l+s+s1}{\PYGZsq{}}\PYG{l+s+s1}{Intraday and Overnight Return Decomposition}\PYG{l+s+se}{\PYGZbs{}n}\PYG{l+s+s1}{12\PYGZhy{}Month Rolling Windows}\PYG{l+s+s1}{\PYGZsq{}}\PYG{p}{)}
\PYG{n}{plt}\PYG{o}{.}\PYG{n}{show}\PYG{p}{(}\PYG{p}{)}
\end{sphinxVerbatim}

\end{sphinxuseclass}\end{sphinxVerbatimInput}
\begin{sphinxVerbatimOutput}

\begin{sphinxuseclass}{cell_output}
\noindent\sphinxincludegraphics{{6e8c1d08b6f6d57e8ab0d326196e4260bb87f18741f3423ae4362abdc0b2c9e9}.png}

\end{sphinxuseclass}\end{sphinxVerbatimOutput}

\end{sphinxuseclass}

\paragraph{Sharpe Ratios for each calendar year}
\label{\detokenize{mckinney_11_practice_02:sharpe-ratios-for-each-calendar-year}}
\sphinxAtStartPar
We need the risk\sphinxhyphen{}free rate \sphinxcode{\sphinxupquote{RF}} from Ken French’s daily benchmark factors.

\begin{sphinxuseclass}{cell}\begin{sphinxVerbatimInput}

\begin{sphinxuseclass}{cell_input}
\begin{sphinxVerbatim}[commandchars=\\\{\}]
\PYG{n}{ff} \PYG{o}{=} \PYG{p}{(}
    \PYG{n}{pdr}\PYG{o}{.}\PYG{n}{DataReader}\PYG{p}{(}
        \PYG{n}{name}\PYG{o}{=}\PYG{l+s+s1}{\PYGZsq{}}\PYG{l+s+s1}{F\PYGZhy{}F\PYGZus{}Research\PYGZus{}Data\PYGZus{}Factors\PYGZus{}daily}\PYG{l+s+s1}{\PYGZsq{}}\PYG{p}{,}
        \PYG{n}{data\PYGZus{}source}\PYG{o}{=}\PYG{l+s+s1}{\PYGZsq{}}\PYG{l+s+s1}{famafrench}\PYG{l+s+s1}{\PYGZsq{}}\PYG{p}{,}
        \PYG{n}{start}\PYG{o}{=}\PYG{l+s+s1}{\PYGZsq{}}\PYG{l+s+s1}{1900}\PYG{l+s+s1}{\PYGZsq{}}\PYG{p}{,}
        \PYG{n}{session}\PYG{o}{=}\PYG{n}{session}
    \PYG{p}{)}
    \PYG{p}{[}\PYG{l+m+mi}{0}\PYG{p}{]}
    \PYG{o}{.}\PYG{n}{div}\PYG{p}{(}\PYG{l+m+mi}{100}\PYG{p}{)}
\PYG{p}{)}
\end{sphinxVerbatim}

\end{sphinxuseclass}\end{sphinxVerbatimInput}

\end{sphinxuseclass}
\sphinxAtStartPar
I think this easiest (and fastest approach) is to:
\begin{enumerate}
\sphinxsetlistlabels{\arabic}{enumi}{enumii}{}{.}%
\item {} 
\sphinxAtStartPar
Calculate a data frame of excess returns

\item {} 
\sphinxAtStartPar
Use an anonymous (lambda) function

\end{enumerate}

\begin{sphinxuseclass}{cell}\begin{sphinxVerbatimInput}

\begin{sphinxuseclass}{cell_input}
\begin{sphinxVerbatim}[commandchars=\\\{\}]
\PYG{n}{returns\PYGZus{}excess} \PYG{o}{=} \PYG{p}{(}
    \PYG{n}{returns}
    \PYG{o}{.}\PYG{n}{sub}\PYG{p}{(}\PYG{n}{ff}\PYG{p}{[}\PYG{l+s+s1}{\PYGZsq{}}\PYG{l+s+s1}{RF}\PYG{l+s+s1}{\PYGZsq{}}\PYG{p}{]}\PYG{p}{,} \PYG{n}{axis}\PYG{o}{=}\PYG{l+m+mi}{0}\PYG{p}{)} \PYG{c+c1}{\PYGZsh{} note axis=0 for column\PYGZhy{}wise subtraction}
    \PYG{o}{.}\PYG{n}{dropna}\PYG{p}{(}\PYG{p}{)}
\PYG{p}{)}
\end{sphinxVerbatim}

\end{sphinxuseclass}\end{sphinxVerbatimInput}

\end{sphinxuseclass}
\begin{sphinxuseclass}{cell}\begin{sphinxVerbatimInput}

\begin{sphinxuseclass}{cell_input}
\begin{sphinxVerbatim}[commandchars=\\\{\}]
\PYG{n}{sharpes\PYGZus{}a} \PYG{o}{=} \PYG{p}{(}
    \PYG{n}{returns\PYGZus{}excess}
    \PYG{o}{.}\PYG{n}{resample}\PYG{p}{(}\PYG{n}{rule}\PYG{o}{=}\PYG{l+s+s1}{\PYGZsq{}}\PYG{l+s+s1}{A}\PYG{l+s+s1}{\PYGZsq{}}\PYG{p}{,} \PYG{n}{kind}\PYG{o}{=}\PYG{l+s+s1}{\PYGZsq{}}\PYG{l+s+s1}{period}\PYG{l+s+s1}{\PYGZsq{}}\PYG{p}{)}
    \PYG{o}{.}\PYG{n}{apply}\PYG{p}{(}\PYG{k}{lambda} \PYG{n}{x}\PYG{p}{:} \PYG{n}{np}\PYG{o}{.}\PYG{n}{sqrt}\PYG{p}{(}\PYG{l+m+mi}{252}\PYG{p}{)} \PYG{o}{*} \PYG{n}{x}\PYG{o}{.}\PYG{n}{mean}\PYG{p}{(}\PYG{p}{)} \PYG{o}{/} \PYG{n}{x}\PYG{o}{.}\PYG{n}{std}\PYG{p}{(}\PYG{p}{)}\PYG{p}{)}
\PYG{p}{)}

\PYG{n}{sharpes\PYGZus{}a}\PYG{o}{.}\PYG{n}{head}\PYG{p}{(}\PYG{p}{)}
\end{sphinxVerbatim}

\end{sphinxuseclass}\end{sphinxVerbatimInput}
\begin{sphinxVerbatimOutput}

\begin{sphinxuseclass}{cell_output}
\begin{sphinxVerbatim}[commandchars=\\\{\}]
Return  Intraday  Overnight   Total
Date                               
1993     \PYGZhy{}0.9172     2.8057  0.7233
1994     \PYGZhy{}1.1871     0.8819 \PYGZhy{}0.2732
1995      2.5275     0.4197  3.1845
1996     \PYGZhy{}0.1691     1.6658  1.2247
1997     \PYGZhy{}0.1781     2.5565  1.2796
\end{sphinxVerbatim}

\end{sphinxuseclass}\end{sphinxVerbatimOutput}

\end{sphinxuseclass}
\begin{sphinxuseclass}{cell}\begin{sphinxVerbatimInput}

\begin{sphinxuseclass}{cell_input}
\begin{sphinxVerbatim}[commandchars=\\\{\}]
\PYG{n}{sharpes\PYGZus{}a}\PYG{o}{.}\PYG{n}{iloc}\PYG{p}{[}\PYG{l+m+mi}{1}\PYG{p}{:}\PYG{o}{\PYGZhy{}}\PYG{l+m+mi}{1}\PYG{p}{]}\PYG{o}{.}\PYG{n}{plot}\PYG{p}{(}\PYG{n}{kind}\PYG{o}{=}\PYG{l+s+s1}{\PYGZsq{}}\PYG{l+s+s1}{bar}\PYG{l+s+s1}{\PYGZsq{}}\PYG{p}{)}
\PYG{n}{plt}\PYG{o}{.}\PYG{n}{ylabel}\PYG{p}{(}\PYG{l+s+s1}{\PYGZsq{}}\PYG{l+s+s1}{Sharpe Ratio}\PYG{l+s+s1}{\PYGZsq{}}\PYG{p}{)}
\PYG{n}{plt}\PYG{o}{.}\PYG{n}{title}\PYG{p}{(}\PYG{l+s+s1}{\PYGZsq{}}\PYG{l+s+s1}{Intraday and Overnight Return Decomposition}\PYG{l+s+se}{\PYGZbs{}n}\PYG{l+s+s1}{Calendar Year Sharpe Ratios}\PYG{l+s+s1}{\PYGZsq{}}\PYG{p}{)}
\PYG{n}{plt}\PYG{o}{.}\PYG{n}{show}\PYG{p}{(}\PYG{p}{)}
\end{sphinxVerbatim}

\end{sphinxuseclass}\end{sphinxVerbatimInput}
\begin{sphinxVerbatimOutput}

\begin{sphinxuseclass}{cell_output}
\noindent\sphinxincludegraphics{{7a0daea46b562bb3d445b9d30abf98b5d7aa24e837cacf360629ff41f861503b}.png}

\end{sphinxuseclass}\end{sphinxVerbatimOutput}

\end{sphinxuseclass}

\subsubsection{Calculate rolling betas}
\label{\detokenize{mckinney_11_practice_02:calculate-rolling-betas}}
\sphinxAtStartPar
Calculate rolling capital asset pricing model (CAPM) betas for the MATANA stocks.

\sphinxAtStartPar
The CAPM says the risk premium on a stock depends on the risk\sphinxhyphen{}free rate, beta, and the risk premium on the market: \$E(R\_\{stock\}) = R\_f + \textbackslash{}beta\_\{stock\} \textbackslash{}times (E(R\_\{market\}) \sphinxhyphen{} R\_f)\$.
We can calculate CAPM betas as: \$\textbackslash{}beta\_\{stock\} = \textbackslash{}frac\{Cov(R\_\{stock\} \sphinxhyphen{} R\_f, R\_\{market\} \sphinxhyphen{} R\_f)\}\{Var(R\_\{market\} \sphinxhyphen{} R\_f)\}\$.

\begin{sphinxuseclass}{cell}\begin{sphinxVerbatimInput}

\begin{sphinxuseclass}{cell_input}
\begin{sphinxVerbatim}[commandchars=\\\{\}]
\PYG{n}{matana} \PYG{o}{=} \PYG{p}{(}
    \PYG{n}{yf}\PYG{o}{.}\PYG{n}{download}\PYG{p}{(}\PYG{n}{tickers}\PYG{o}{=}\PYG{l+s+s1}{\PYGZsq{}}\PYG{l+s+s1}{MSFT AAPL TSLA AMZN NVDA GOOG}\PYG{l+s+s1}{\PYGZsq{}}\PYG{p}{,} \PYG{n}{progress}\PYG{o}{=}\PYG{k+kc}{False}\PYG{p}{)}
    \PYG{o}{.}\PYG{n}{assign}\PYG{p}{(}
        \PYG{n}{Date} \PYG{o}{=} \PYG{k}{lambda} \PYG{n}{x}\PYG{p}{:} \PYG{n}{x}\PYG{o}{.}\PYG{n}{index}\PYG{o}{.}\PYG{n}{tz\PYGZus{}localize}\PYG{p}{(}\PYG{k+kc}{None}\PYG{p}{)}\PYG{p}{,}
    \PYG{p}{)}
    \PYG{o}{.}\PYG{n}{set\PYGZus{}index}\PYG{p}{(}\PYG{l+s+s1}{\PYGZsq{}}\PYG{l+s+s1}{Date}\PYG{l+s+s1}{\PYGZsq{}}\PYG{p}{)}
    \PYG{p}{[}\PYG{l+s+s1}{\PYGZsq{}}\PYG{l+s+s1}{Adj Close}\PYG{l+s+s1}{\PYGZsq{}}\PYG{p}{]}
    \PYG{o}{.}\PYG{n}{pct\PYGZus{}change}\PYG{p}{(}\PYG{p}{)}
    \PYG{o}{.}\PYG{n}{rename\PYGZus{}axis}\PYG{p}{(}\PYG{n}{columns}\PYG{o}{=}\PYG{l+s+s1}{\PYGZsq{}}\PYG{l+s+s1}{Ticker}\PYG{l+s+s1}{\PYGZsq{}}\PYG{p}{)}
\PYG{p}{)}

\PYG{n}{matana}\PYG{o}{.}\PYG{n}{head}\PYG{p}{(}\PYG{p}{)}
\end{sphinxVerbatim}

\end{sphinxuseclass}\end{sphinxVerbatimInput}
\begin{sphinxVerbatimOutput}

\begin{sphinxuseclass}{cell_output}
\begin{sphinxVerbatim}[commandchars=\\\{\}]
Ticker        AAPL  AMZN  GOOG  MSFT  NVDA  TSLA
Date                                            
1980\PYGZhy{}12\PYGZhy{}12     NaN   NaN   NaN   NaN   NaN   NaN
1980\PYGZhy{}12\PYGZhy{}15 \PYGZhy{}0.0522   NaN   NaN   NaN   NaN   NaN
1980\PYGZhy{}12\PYGZhy{}16 \PYGZhy{}0.0734   NaN   NaN   NaN   NaN   NaN
1980\PYGZhy{}12\PYGZhy{}17  0.0248   NaN   NaN   NaN   NaN   NaN
1980\PYGZhy{}12\PYGZhy{}18  0.0290   NaN   NaN   NaN   NaN   NaN
\end{sphinxVerbatim}

\end{sphinxuseclass}\end{sphinxVerbatimOutput}

\end{sphinxuseclass}
\begin{sphinxuseclass}{cell}\begin{sphinxVerbatimInput}

\begin{sphinxuseclass}{cell_input}
\begin{sphinxVerbatim}[commandchars=\\\{\}]
\PYG{k}{def} \PYG{n+nf}{beta}\PYG{p}{(}\PYG{n}{ri}\PYG{p}{,} \PYG{n}{rf}\PYG{o}{=}\PYG{n}{ff}\PYG{p}{[}\PYG{l+s+s1}{\PYGZsq{}}\PYG{l+s+s1}{RF}\PYG{l+s+s1}{\PYGZsq{}}\PYG{p}{]}\PYG{p}{,} \PYG{n}{rm\PYGZus{}rf}\PYG{o}{=}\PYG{n}{ff}\PYG{p}{[}\PYG{l+s+s1}{\PYGZsq{}}\PYG{l+s+s1}{Mkt\PYGZhy{}RF}\PYG{l+s+s1}{\PYGZsq{}}\PYG{p}{]}\PYG{p}{)}\PYG{p}{:}
    \PYG{n}{ri\PYGZus{}rf} \PYG{o}{=} \PYG{n}{ri}\PYG{o}{.}\PYG{n}{sub}\PYG{p}{(}\PYG{n}{rf}\PYG{p}{)}\PYG{o}{.}\PYG{n}{dropna}\PYG{p}{(}\PYG{p}{)}
    \PYG{k}{return} \PYG{n}{ri\PYGZus{}rf}\PYG{o}{.}\PYG{n}{cov}\PYG{p}{(}\PYG{n}{rm\PYGZus{}rf}\PYG{p}{)} \PYG{o}{/} \PYG{n}{rm\PYGZus{}rf}\PYG{o}{.}\PYG{n}{loc}\PYG{p}{[}\PYG{n}{ri\PYGZus{}rf}\PYG{o}{.}\PYG{n}{index}\PYG{p}{]}\PYG{o}{.}\PYG{n}{var}\PYG{p}{(}\PYG{p}{)}
\end{sphinxVerbatim}

\end{sphinxuseclass}\end{sphinxVerbatimInput}

\end{sphinxuseclass}
\sphinxAtStartPar
A few notes:
\begin{enumerate}
\sphinxsetlistlabels{\arabic}{enumi}{enumii}{}{.}%
\item {} 
\sphinxAtStartPar
Here the \sphinxcode{\sphinxupquote{beta()}} function is too smart for the \sphinxcode{\sphinxupquote{.rolling()}} method.
Our \sphinxcode{\sphinxupquote{beta()}} function matches stock and market data, but may not use 252 trading days at the end of the sample.
For example, in February 2023, we can slice a 252\sphinxhyphen{}trading\sphinxhyphen{}day window from the \sphinxcode{\sphinxupquote{matana}} data frame.
However, the \sphinxcode{\sphinxupquote{beta()}} function does not use January and Februrary 2023 data because they are missing from the \sphinxcode{\sphinxupquote{ff}} data frame.
The easiest fix is to slice the \sphinxcode{\sphinxupquote{matana}} data frame to remove 2023 data before we apply the \sphinxcode{\sphinxupquote{.rolling()}} method.

\item {} 
\sphinxAtStartPar
The \sphinxcode{\sphinxupquote{.apply()}} method and \sphinxcode{\sphinxupquote{beta()}} function are \sphinxstyleemphasis{very slow} because they slice inefficiently!
We will write faster—but uglier—code below.

\end{enumerate}

\begin{sphinxuseclass}{cell}\begin{sphinxVerbatimInput}

\begin{sphinxuseclass}{cell_input}
\begin{sphinxVerbatim}[commandchars=\\\{\}]
\PYG{o}{\PYGZpc{}\PYGZpc{}time}
\PYG{n}{betas\PYGZus{}1} \PYG{o}{=} \PYG{n}{matana}\PYG{o}{.}\PYG{n}{dropna}\PYG{p}{(}\PYG{p}{)}\PYG{o}{.}\PYG{n}{loc}\PYG{p}{[}\PYG{p}{:}\PYG{l+s+s1}{\PYGZsq{}}\PYG{l+s+s1}{2022}\PYG{l+s+s1}{\PYGZsq{}}\PYG{p}{]}\PYG{o}{.}\PYG{n}{rolling}\PYG{p}{(}\PYG{l+m+mi}{252}\PYG{p}{)}\PYG{o}{.}\PYG{n}{apply}\PYG{p}{(}\PYG{n}{beta}\PYG{p}{)}
\end{sphinxVerbatim}

\end{sphinxuseclass}\end{sphinxVerbatimInput}
\begin{sphinxVerbatimOutput}

\begin{sphinxuseclass}{cell_output}
\begin{sphinxVerbatim}[commandchars=\\\{\}]
Wall time: 24.4 s
\end{sphinxVerbatim}

\end{sphinxuseclass}\end{sphinxVerbatimOutput}

\end{sphinxuseclass}
\sphinxAtStartPar
Here is the fast solution, which uses the optimized \sphinxcode{\sphinxupquote{.cov()}} and \sphinxcode{\sphinxupquote{.var()}} to calculate the numerator and denominator for \$\textbackslash{}beta\$.

\begin{sphinxuseclass}{cell}\begin{sphinxVerbatimInput}

\begin{sphinxuseclass}{cell_input}
\begin{sphinxVerbatim}[commandchars=\\\{\}]
\PYG{o}{\PYGZpc{}\PYGZpc{}time}
\PYG{n}{cov\PYGZus{}2} \PYG{o}{=} \PYG{n}{matana}\PYG{o}{.}\PYG{n}{dropna}\PYG{p}{(}\PYG{p}{)}\PYG{o}{.}\PYG{n}{loc}\PYG{p}{[}\PYG{p}{:}\PYG{l+s+s1}{\PYGZsq{}}\PYG{l+s+s1}{2022}\PYG{l+s+s1}{\PYGZsq{}}\PYG{p}{]}\PYG{o}{.}\PYG{n}{sub}\PYG{p}{(}\PYG{n}{ff}\PYG{p}{[}\PYG{l+s+s1}{\PYGZsq{}}\PYG{l+s+s1}{RF}\PYG{l+s+s1}{\PYGZsq{}}\PYG{p}{]}\PYG{p}{,} \PYG{n}{axis}\PYG{o}{=}\PYG{l+m+mi}{0}\PYG{p}{)}\PYG{o}{.}\PYG{n}{rolling}\PYG{p}{(}\PYG{l+m+mi}{252}\PYG{p}{)}\PYG{o}{.}\PYG{n}{cov}\PYG{p}{(}\PYG{n}{ff}\PYG{p}{[}\PYG{l+s+s1}{\PYGZsq{}}\PYG{l+s+s1}{Mkt\PYGZhy{}RF}\PYG{l+s+s1}{\PYGZsq{}}\PYG{p}{]}\PYG{p}{)}
\PYG{n}{var\PYGZus{}2} \PYG{o}{=} \PYG{n}{ff}\PYG{p}{[}\PYG{l+s+s1}{\PYGZsq{}}\PYG{l+s+s1}{Mkt\PYGZhy{}RF}\PYG{l+s+s1}{\PYGZsq{}}\PYG{p}{]}\PYG{o}{.}\PYG{n}{rolling}\PYG{p}{(}\PYG{l+m+mi}{252}\PYG{p}{)}\PYG{o}{.}\PYG{n}{var}\PYG{p}{(}\PYG{p}{)}
\PYG{n}{betas\PYGZus{}2} \PYG{o}{=} \PYG{n}{cov\PYGZus{}2}\PYG{o}{.}\PYG{n}{div}\PYG{p}{(}\PYG{n}{var\PYGZus{}2}\PYG{p}{,} \PYG{n}{axis}\PYG{o}{=}\PYG{l+m+mi}{0}\PYG{p}{)}
\end{sphinxVerbatim}

\end{sphinxuseclass}\end{sphinxVerbatimInput}
\begin{sphinxVerbatimOutput}

\begin{sphinxuseclass}{cell_output}
\begin{sphinxVerbatim}[commandchars=\\\{\}]
Wall time: 30.9 ms
\end{sphinxVerbatim}

\end{sphinxuseclass}\end{sphinxVerbatimOutput}

\end{sphinxuseclass}
\sphinxAtStartPar
These solutions are identical!
Even though \sphinxcode{\sphinxupquote{beta\_2}} is 1000 times faster than \sphinxcode{\sphinxupquote{beta\_1}}, you should use the approach that makes the most sense to you!

\begin{sphinxuseclass}{cell}\begin{sphinxVerbatimInput}

\begin{sphinxuseclass}{cell_input}
\begin{sphinxVerbatim}[commandchars=\\\{\}]
\PYG{n}{np}\PYG{o}{.}\PYG{n}{allclose}\PYG{p}{(}\PYG{n}{betas\PYGZus{}1}\PYG{o}{.}\PYG{n}{dropna}\PYG{p}{(}\PYG{p}{)}\PYG{p}{,} \PYG{n}{betas\PYGZus{}2}\PYG{o}{.}\PYG{n}{dropna}\PYG{p}{(}\PYG{p}{)}\PYG{p}{)}
\end{sphinxVerbatim}

\end{sphinxuseclass}\end{sphinxVerbatimInput}
\begin{sphinxVerbatimOutput}

\begin{sphinxuseclass}{cell_output}
\begin{sphinxVerbatim}[commandchars=\\\{\}]
True
\end{sphinxVerbatim}

\end{sphinxuseclass}\end{sphinxVerbatimOutput}

\end{sphinxuseclass}
\begin{sphinxuseclass}{cell}\begin{sphinxVerbatimInput}

\begin{sphinxuseclass}{cell_input}
\begin{sphinxVerbatim}[commandchars=\\\{\}]
\PYG{n}{betas\PYGZus{}1}\PYG{o}{.}\PYG{n}{plot}\PYG{p}{(}\PYG{p}{)}
\PYG{n}{plt}\PYG{o}{.}\PYG{n}{ylabel}\PYG{p}{(}\PYG{l+s+s1}{\PYGZsq{}}\PYG{l+s+s1}{Beta}\PYG{l+s+s1}{\PYGZsq{}}\PYG{p}{)}
\PYG{n}{plt}\PYG{o}{.}\PYG{n}{title}\PYG{p}{(}\PYG{l+s+s1}{\PYGZsq{}}\PYG{l+s+s1}{Betas from Daily Returns}\PYG{l+s+se}{\PYGZbs{}n}\PYG{l+s+s1}{252\PYGZhy{}Trading\PYGZhy{}Day Rolling Windows}\PYG{l+s+s1}{\PYGZsq{}}\PYG{p}{)}
\PYG{n}{plt}\PYG{o}{.}\PYG{n}{show}\PYG{p}{(}\PYG{p}{)}
\end{sphinxVerbatim}

\end{sphinxuseclass}\end{sphinxVerbatimInput}
\begin{sphinxVerbatimOutput}

\begin{sphinxuseclass}{cell_output}
\noindent\sphinxincludegraphics{{0614bba1237a3dd35963b0a5594b60145b782cd7979b90672df38df363f6ef61}.png}

\end{sphinxuseclass}\end{sphinxVerbatimOutput}

\end{sphinxuseclass}

\subsubsection{Calculate rolling Sharpe Ratios}
\label{\detokenize{mckinney_11_practice_02:calculate-rolling-sharpe-ratios}}
\sphinxAtStartPar
Calculate rolling Sharpe Ratios for the MATANA stocks.

\sphinxAtStartPar
The Sharpe Ratio is often used to evaluate fund managers.
The Sharpe Ratio is \$SR\_i = \textbackslash{}frac\{\textbackslash{}overline\{R\_i \sphinxhyphen{} R\_f\}\}\{\textbackslash{}sigma\}\$, where \$\textbackslash{}overline\{R\_i\sphinxhyphen{}R\_f\}\$ is mean fund return relative to the risk\sphinxhyphen{}free rate over some period and \$\textbackslash{}sigma\$ is the standard deviation of \$R\_i\sphinxhyphen{}R\_f\$ over the same period.
While the Sharpe Ratio is typically used for funds, we can apply it to a single stock to test our knowledge of the \sphinxcode{\sphinxupquote{.rolling()}} method.
Calculate and plot the one\sphinxhyphen{}year rolling Sharpe Ratio for the MATANA stocks using all available daily data.

\begin{sphinxuseclass}{cell}\begin{sphinxVerbatimInput}

\begin{sphinxuseclass}{cell_input}
\begin{sphinxVerbatim}[commandchars=\\\{\}]
\PYG{k}{def} \PYG{n+nf}{sharpe}\PYG{p}{(}\PYG{n}{ri}\PYG{p}{,} \PYG{n}{rf}\PYG{o}{=}\PYG{n}{ff}\PYG{p}{[}\PYG{l+s+s1}{\PYGZsq{}}\PYG{l+s+s1}{RF}\PYG{l+s+s1}{\PYGZsq{}}\PYG{p}{]}\PYG{p}{,} \PYG{n}{ann\PYGZus{}fac}\PYG{o}{=}\PYG{n}{np}\PYG{o}{.}\PYG{n}{sqrt}\PYG{p}{(}\PYG{l+m+mi}{252}\PYG{p}{)}\PYG{p}{)}\PYG{p}{:}
    \PYG{n}{ri\PYGZus{}rf} \PYG{o}{=} \PYG{n}{ri}\PYG{o}{.}\PYG{n}{sub}\PYG{p}{(}\PYG{n}{rf}\PYG{p}{)}\PYG{o}{.}\PYG{n}{dropna}\PYG{p}{(}\PYG{p}{)}
    \PYG{k}{return} \PYG{n}{ann\PYGZus{}fac} \PYG{o}{*} \PYG{n}{ri\PYGZus{}rf}\PYG{o}{.}\PYG{n}{mean}\PYG{p}{(}\PYG{p}{)} \PYG{o}{/} \PYG{n}{ri\PYGZus{}rf}\PYG{o}{.}\PYG{n}{std}\PYG{p}{(}\PYG{p}{)}
\end{sphinxVerbatim}

\end{sphinxuseclass}\end{sphinxVerbatimInput}

\end{sphinxuseclass}
\sphinxAtStartPar
See the discussion in the beta exercise above about how we can end up with a result based on fewer than 252 trading days of data because the \sphinxcode{\sphinxupquote{sharpe()}} function matches the \sphinxcode{\sphinxupquote{matana}} and \sphinxcode{\sphinxupquote{ff}} data frames \sphinxstyleemphasis{after} the \sphinxcode{\sphinxupquote{.rolling()}} mehtod slices a 252 trading day window.
The simplest solution is the remove the 2023 data from the \sphinxcode{\sphinxupquote{matana}} data frame before we apply the \sphinxcode{\sphinxupquote{.rolling()}} method.

\begin{sphinxuseclass}{cell}\begin{sphinxVerbatimInput}

\begin{sphinxuseclass}{cell_input}
\begin{sphinxVerbatim}[commandchars=\\\{\}]
\PYG{o}{\PYGZpc{}\PYGZpc{}time}
\PYG{n}{sharpes\PYGZus{}1} \PYG{o}{=} \PYG{n}{matana}\PYG{o}{.}\PYG{n}{dropna}\PYG{p}{(}\PYG{p}{)}\PYG{o}{.}\PYG{n}{loc}\PYG{p}{[}\PYG{p}{:}\PYG{l+s+s1}{\PYGZsq{}}\PYG{l+s+s1}{2022}\PYG{l+s+s1}{\PYGZsq{}}\PYG{p}{]}\PYG{o}{.}\PYG{n}{rolling}\PYG{p}{(}\PYG{l+m+mi}{252}\PYG{p}{)}\PYG{o}{.}\PYG{n}{apply}\PYG{p}{(}\PYG{n}{sharpe}\PYG{p}{)}
\end{sphinxVerbatim}

\end{sphinxuseclass}\end{sphinxVerbatimInput}
\begin{sphinxVerbatimOutput}

\begin{sphinxuseclass}{cell_output}
\begin{sphinxVerbatim}[commandchars=\\\{\}]
Wall time: 12.3 s
\end{sphinxVerbatim}

\end{sphinxuseclass}\end{sphinxVerbatimOutput}

\end{sphinxuseclass}
\sphinxAtStartPar
Here is the fast solution, which:
\begin{enumerate}
\sphinxsetlistlabels{\arabic}{enumi}{enumii}{}{.}%
\item {} 
\sphinxAtStartPar
Uses the \sphinxcode{\sphinxupquote{.sub()}} method to calculate excess returns

\item {} 
\sphinxAtStartPar
Then uses the optimized \sphinxcode{\sphinxupquote{.mean()}} and \sphinxcode{\sphinxupquote{.std()}} methods to calculate the numerator and denominator for the Sharpe Ratios

\end{enumerate}

\begin{sphinxuseclass}{cell}\begin{sphinxVerbatimInput}

\begin{sphinxuseclass}{cell_input}
\begin{sphinxVerbatim}[commandchars=\\\{\}]
\PYG{n}{mean\PYGZus{}2} \PYG{o}{=} \PYG{n}{matana}\PYG{o}{.}\PYG{n}{dropna}\PYG{p}{(}\PYG{p}{)}\PYG{o}{.}\PYG{n}{loc}\PYG{p}{[}\PYG{p}{:}\PYG{l+s+s1}{\PYGZsq{}}\PYG{l+s+s1}{2022}\PYG{l+s+s1}{\PYGZsq{}}\PYG{p}{]}\PYG{o}{.}\PYG{n}{sub}\PYG{p}{(}\PYG{n}{ff}\PYG{p}{[}\PYG{l+s+s1}{\PYGZsq{}}\PYG{l+s+s1}{RF}\PYG{l+s+s1}{\PYGZsq{}}\PYG{p}{]}\PYG{p}{,} \PYG{n}{axis}\PYG{o}{=}\PYG{l+m+mi}{0}\PYG{p}{)}\PYG{o}{.}\PYG{n}{rolling}\PYG{p}{(}\PYG{l+m+mi}{252}\PYG{p}{)}\PYG{o}{.}\PYG{n}{mean}\PYG{p}{(}\PYG{p}{)}
\PYG{n}{std\PYGZus{}2} \PYG{o}{=} \PYG{n}{matana}\PYG{o}{.}\PYG{n}{dropna}\PYG{p}{(}\PYG{p}{)}\PYG{o}{.}\PYG{n}{loc}\PYG{p}{[}\PYG{p}{:}\PYG{l+s+s1}{\PYGZsq{}}\PYG{l+s+s1}{2022}\PYG{l+s+s1}{\PYGZsq{}}\PYG{p}{]}\PYG{o}{.}\PYG{n}{sub}\PYG{p}{(}\PYG{n}{ff}\PYG{p}{[}\PYG{l+s+s1}{\PYGZsq{}}\PYG{l+s+s1}{RF}\PYG{l+s+s1}{\PYGZsq{}}\PYG{p}{]}\PYG{p}{,} \PYG{n}{axis}\PYG{o}{=}\PYG{l+m+mi}{0}\PYG{p}{)}\PYG{o}{.}\PYG{n}{rolling}\PYG{p}{(}\PYG{l+m+mi}{252}\PYG{p}{)}\PYG{o}{.}\PYG{n}{std}\PYG{p}{(}\PYG{p}{)}
\PYG{n}{sharpes\PYGZus{}2} \PYG{o}{=} \PYG{n}{mean\PYGZus{}2}\PYG{o}{.}\PYG{n}{div}\PYG{p}{(}\PYG{n}{std\PYGZus{}2}\PYG{p}{,} \PYG{n}{axis}\PYG{o}{=}\PYG{l+m+mi}{0}\PYG{p}{)}\PYG{o}{.}\PYG{n}{mul}\PYG{p}{(}\PYG{n}{np}\PYG{o}{.}\PYG{n}{sqrt}\PYG{p}{(}\PYG{l+m+mi}{252}\PYG{p}{)}\PYG{p}{)}
\end{sphinxVerbatim}

\end{sphinxuseclass}\end{sphinxVerbatimInput}

\end{sphinxuseclass}
\begin{sphinxuseclass}{cell}\begin{sphinxVerbatimInput}

\begin{sphinxuseclass}{cell_input}
\begin{sphinxVerbatim}[commandchars=\\\{\}]
\PYG{n}{np}\PYG{o}{.}\PYG{n}{allclose}\PYG{p}{(}\PYG{n}{sharpes\PYGZus{}1}\PYG{o}{.}\PYG{n}{dropna}\PYG{p}{(}\PYG{p}{)}\PYG{p}{,} \PYG{n}{sharpes\PYGZus{}2}\PYG{o}{.}\PYG{n}{dropna}\PYG{p}{(}\PYG{p}{)}\PYG{p}{)}
\end{sphinxVerbatim}

\end{sphinxuseclass}\end{sphinxVerbatimInput}
\begin{sphinxVerbatimOutput}

\begin{sphinxuseclass}{cell_output}
\begin{sphinxVerbatim}[commandchars=\\\{\}]
True
\end{sphinxVerbatim}

\end{sphinxuseclass}\end{sphinxVerbatimOutput}

\end{sphinxuseclass}
\begin{sphinxuseclass}{cell}\begin{sphinxVerbatimInput}

\begin{sphinxuseclass}{cell_input}
\begin{sphinxVerbatim}[commandchars=\\\{\}]
\PYG{n}{sharpes\PYGZus{}1}\PYG{o}{.}\PYG{n}{plot}\PYG{p}{(}\PYG{p}{)}
\PYG{n}{plt}\PYG{o}{.}\PYG{n}{ylabel}\PYG{p}{(}\PYG{l+s+s1}{\PYGZsq{}}\PYG{l+s+s1}{Sharpe Ratio}\PYG{l+s+s1}{\PYGZsq{}}\PYG{p}{)}
\PYG{n}{plt}\PYG{o}{.}\PYG{n}{title}\PYG{p}{(}\PYG{l+s+s1}{\PYGZsq{}}\PYG{l+s+s1}{Sharpe Ratios from Daily Returns}\PYG{l+s+se}{\PYGZbs{}n}\PYG{l+s+s1}{252\PYGZhy{}Trading\PYGZhy{}Day Rolling Windows}\PYG{l+s+s1}{\PYGZsq{}}\PYG{p}{)}
\PYG{n}{plt}\PYG{o}{.}\PYG{n}{show}\PYG{p}{(}\PYG{p}{)}
\end{sphinxVerbatim}

\end{sphinxuseclass}\end{sphinxVerbatimInput}
\begin{sphinxVerbatimOutput}

\begin{sphinxuseclass}{cell_output}
\noindent\sphinxincludegraphics{{9e2712e079320e7f44e37d6c7d85689ebf4e5c2c409bbb4e25223614a8b66a97}.png}

\end{sphinxuseclass}\end{sphinxVerbatimOutput}

\end{sphinxuseclass}

\subsubsection{Does more frequent rebalancing increase or decrease returns?}
\label{\detokenize{mckinney_11_practice_02:does-more-frequent-rebalancing-increase-or-decrease-returns}}
\sphinxAtStartPar
Compare decade\sphinxhyphen{}total returns for the following rebalancing frequencies:
\begin{enumerate}
\sphinxsetlistlabels{\arabic}{enumi}{enumii}{}{.}%
\item {} 
\sphinxAtStartPar
Daily rebalancing

\item {} 
\sphinxAtStartPar
Monthly rebalancing

\item {} 
\sphinxAtStartPar
Annual rebalancing

\item {} 
\sphinxAtStartPar
Decade rebalancing

\end{enumerate}

\sphinxAtStartPar
Use equally\sphinxhyphen{}weighted portfolios of industry\sphinxhyphen{}level daily returns from French’s website: \sphinxcode{\sphinxupquote{'17\_Industry\_Portfolios\_daily'}}.

\sphinxAtStartPar
\sphinxstyleemphasis{\sphinxstylestrong{This exercise is a little early, and we will revisit later in the course!}}

\sphinxstepscope


\chapter{Herron Topic 2 \sphinxhyphen{} Trading Strategies}
\label{\detokenize{herron_02_lecture:herron-topic-2-trading-strategies}}\label{\detokenize{herron_02_lecture::doc}}
\sphinxAtStartPar
This notebook covers trading strategies based on technical analysis in three parts:
\begin{enumerate}
\sphinxsetlistlabels{\arabic}{enumi}{enumii}{}{.}%
\item {} 
\sphinxAtStartPar
What is technical analysis?

\item {} 
\sphinxAtStartPar
Why might trading strategies based on technical analysis work (or not work)?

\item {} 
\sphinxAtStartPar
Implement a simple moving average (SMA) trading strategy

\end{enumerate}

\sphinxAtStartPar
I based this lecture notebook on \sphinxhref{https://book.ivo-welch.info/read/source5.mba/12-effbehav.pdf}{chapter 12 of Ivo Welch’s (free) \sphinxstyleemphasis{Corporate Finance} textbook} and chapter 2 of \sphinxhref{https://onesearch.library.northeastern.edu/permalink/01NEU\_INST/i2gqis/alma9952082522901401}{Eryk Lewinson’s \sphinxstyleemphasis{Python for Finance Cookbook}}.
The practice notebook will cover several other trading strategies based on technical analysis.

\begin{sphinxuseclass}{cell}\begin{sphinxVerbatimInput}

\begin{sphinxuseclass}{cell_input}
\begin{sphinxVerbatim}[commandchars=\\\{\}]
\PYG{k+kn}{import} \PYG{n+nn}{matplotlib}\PYG{n+nn}{.}\PYG{n+nn}{pyplot} \PYG{k}{as} \PYG{n+nn}{plt}
\PYG{k+kn}{import} \PYG{n+nn}{numpy} \PYG{k}{as} \PYG{n+nn}{np}
\PYG{k+kn}{import} \PYG{n+nn}{pandas} \PYG{k}{as} \PYG{n+nn}{pd}
\end{sphinxVerbatim}

\end{sphinxuseclass}\end{sphinxVerbatimInput}

\end{sphinxuseclass}
\begin{sphinxuseclass}{cell}\begin{sphinxVerbatimInput}

\begin{sphinxuseclass}{cell_input}
\begin{sphinxVerbatim}[commandchars=\\\{\}]
\PYG{o}{\PYGZpc{}}\PYG{k}{config} InlineBackend.figure\PYGZus{}format = \PYGZsq{}retina\PYGZsq{}
\PYG{o}{\PYGZpc{}}\PYG{k}{precision} 4
\PYG{n}{pd}\PYG{o}{.}\PYG{n}{options}\PYG{o}{.}\PYG{n}{display}\PYG{o}{.}\PYG{n}{float\PYGZus{}format} \PYG{o}{=} \PYG{l+s+s1}{\PYGZsq{}}\PYG{l+s+si}{\PYGZob{}:.4f\PYGZcb{}}\PYG{l+s+s1}{\PYGZsq{}}\PYG{o}{.}\PYG{n}{format}
\end{sphinxVerbatim}

\end{sphinxuseclass}\end{sphinxVerbatimInput}

\end{sphinxuseclass}

\section{What is technical analysis?}
\label{\detokenize{herron_02_lecture:what-is-technical-analysis}}
\sphinxAtStartPar
\sphinxhref{https://en.wikipedia.org/wiki/Technical\_analysis}{Technical analysis} is a methodology that analyzes past market data (e.g., past prices and volume) in an attempt to forecast future price movements.
If technical analysis can predict future price movements, the market is not weak\sphinxhyphen{}form efficient.
Ivo Welch provides the three degrees of market efficiency in section 12.2 of \sphinxhref{https://book.ivo-welch.info/read/source5.mba/12-effbehav.pdf}{chapter 12 of his (free) \sphinxstyleemphasis{Corporate Finance} textbook}:
\begin{quote}

\sphinxAtStartPar
The Traditional Classification
The traditional definition of market efficiency focuses on information. In the traditional classification, market efficiency comes in one of three primary degrees: weak,
semi\sphinxhyphen{}strong, and strong.

\sphinxAtStartPar
\sphinxstylestrong{Weak market efficiency} says that all information in past prices is reflected in today’s
stock prices so that technical analysis (trading based solely on historical price
patterns) cannot be used to beat the market. Put differently, the market is the
best technical analyst.

\sphinxAtStartPar
\sphinxstylestrong{Semistrong market efficiency} says that all public information is reflected in today’s
stock prices, so that neither fundamental trading (based on underlying firm
fundamentals, such as cash flows or discount rates) nor technical analysis can
be used to beat the market. Put differently, the market is both the best technical
and the best fundamental analyst.

\sphinxAtStartPar
\sphinxstylestrong{Strong market efficiency} says that all information, both public and private, is reflected in today’s stock prices, so that nothing — not even private insider
information — can be used to beat the market. Put differently, the market is
the best analyst and cannot be beat.

\sphinxAtStartPar
In this traditional classification, all finance professors
most U.S. financial markets are not strong\sphinxhyphen{}form efficient: Insider trading may be
illegal, but it works. However, there are still arguments as to which markets are only
semi\sphinxhyphen{}strong\sphinxhyphen{}form efficient or even only weak\sphinxhyphen{}form efficient.
\end{quote}

\sphinxAtStartPar
Section 12.2 goes on to provide Welch’s own taxonomy of true, firm, mild, and nonbelievers in market efficiency.
Chapter 12 summarizes market efficiency, classical finance, behavioral finance, arbitrage, limits to arbitrage, and their consequences for managers and investors.
We will focus on technical analysis in this notebook, but chapter 12 is excellent.


\section{Why might trading strategies based on technical analysis work or not?}
\label{\detokenize{herron_02_lecture:why-might-trading-strategies-based-on-technical-analysis-work-or-not}}

\subsection{…Work?}
\label{\detokenize{herron_02_lecture:work}}
\sphinxAtStartPar
Technical analysis relies on a few ideas:
\begin{enumerate}
\sphinxsetlistlabels{\arabic}{enumi}{enumii}{}{.}%
\item {} 
\sphinxAtStartPar
Market prices and volume reflect all relevant information, so we can focus on past prices and volume instead of fundamentals and news.

\item {} 
\sphinxAtStartPar
Market prices move in trends and patterns driven by market participants.

\item {} 
\sphinxAtStartPar
These trends and patterns tend to repeat themselves because market participants create them.

\end{enumerate}


\subsection{…Or Not?}
\label{\detokenize{herron_02_lecture:or-not}}
\sphinxAtStartPar
The logic above is reasonable.
However, if past market prices reflect all relevant information, they should also reflect any prices trends they predict.
Therefore, any patterns should be self\sphinxhyphen{}defeating, and market prices should follow \sphinxhref{https://en.wikipedia.org/wiki/Random\_walk\_hypothesis}{a random walk}.
As well, the signal\sphinxhyphen{}to\sphinxhyphen{}noise ratio in market prices is high!
Still, technical analysis provides an opportunity to learn how to implement and back\sphinxhyphen{}test trading strategies in Python.


\subsubsection{A Random Walk}
\label{\detokenize{herron_02_lecture:a-random-walk}}
\sphinxAtStartPar
In a random walk, the price tomorrow equals the price today plus a tiny drift plus noise.
In math terms, a random walk is \$P\_\{t\} = \textbackslash{}rho P\_\{t\sphinxhyphen{}1\} + m P\_\{t\sphinxhyphen{}1\} + \textbackslash{}varepsilon\$, where \$m\$ is a small drift term and \$E{[}\textbackslash{}varepsilon{]} = 0\$.
If \$\textbackslash{}rho > 1\$, prices would quickly increase, and, if \$\textbackslash{}rho < 1\$, prices would quickly decrease.
Let us examine the historical record.

\begin{sphinxuseclass}{cell}\begin{sphinxVerbatimInput}

\begin{sphinxuseclass}{cell_input}
\begin{sphinxVerbatim}[commandchars=\\\{\}]
\PYG{k+kn}{import} \PYG{n+nn}{pandas\PYGZus{}datareader} \PYG{k}{as} \PYG{n+nn}{pdr}
\PYG{k+kn}{import} \PYG{n+nn}{requests\PYGZus{}cache}
\PYG{n}{session} \PYG{o}{=} \PYG{n}{requests\PYGZus{}cache}\PYG{o}{.}\PYG{n}{CachedSession}\PYG{p}{(}\PYG{p}{)}
\end{sphinxVerbatim}

\end{sphinxuseclass}\end{sphinxVerbatimInput}

\end{sphinxuseclass}
\begin{sphinxuseclass}{cell}\begin{sphinxVerbatimInput}

\begin{sphinxuseclass}{cell_input}
\begin{sphinxVerbatim}[commandchars=\\\{\}]
\PYG{n}{ff} \PYG{o}{=} \PYG{p}{(}
    \PYG{n}{pdr}\PYG{o}{.}\PYG{n}{DataReader}\PYG{p}{(}
        \PYG{n}{name}\PYG{o}{=}\PYG{l+s+s1}{\PYGZsq{}}\PYG{l+s+s1}{F\PYGZhy{}F\PYGZus{}Research\PYGZus{}Data\PYGZus{}Factors\PYGZus{}daily}\PYG{l+s+s1}{\PYGZsq{}}\PYG{p}{,}
        \PYG{n}{data\PYGZus{}source}\PYG{o}{=}\PYG{l+s+s1}{\PYGZsq{}}\PYG{l+s+s1}{famafrench}\PYG{l+s+s1}{\PYGZsq{}}\PYG{p}{,}
        \PYG{n}{start}\PYG{o}{=}\PYG{l+s+s1}{\PYGZsq{}}\PYG{l+s+s1}{1900}\PYG{l+s+s1}{\PYGZsq{}}\PYG{p}{,}
        \PYG{n}{session}\PYG{o}{=}\PYG{n}{session}
    \PYG{p}{)}
    \PYG{p}{[}\PYG{l+m+mi}{0}\PYG{p}{]}
    \PYG{o}{.}\PYG{n}{assign}\PYG{p}{(}\PYG{n}{Mkt} \PYG{o}{=} \PYG{k}{lambda} \PYG{n}{x}\PYG{p}{:} \PYG{n}{x}\PYG{p}{[}\PYG{l+s+s1}{\PYGZsq{}}\PYG{l+s+s1}{Mkt\PYGZhy{}RF}\PYG{l+s+s1}{\PYGZsq{}}\PYG{p}{]} \PYG{o}{+} \PYG{n}{x}\PYG{p}{[}\PYG{l+s+s1}{\PYGZsq{}}\PYG{l+s+s1}{RF}\PYG{l+s+s1}{\PYGZsq{}}\PYG{p}{]}\PYG{p}{)}
    \PYG{o}{.}\PYG{n}{div}\PYG{p}{(}\PYG{l+m+mi}{100}\PYG{p}{)}
\PYG{p}{)}

\PYG{n}{ff}\PYG{o}{.}\PYG{n}{head}\PYG{p}{(}\PYG{p}{)}
\end{sphinxVerbatim}

\end{sphinxuseclass}\end{sphinxVerbatimInput}
\begin{sphinxVerbatimOutput}

\begin{sphinxuseclass}{cell_output}
\begin{sphinxVerbatim}[commandchars=\\\{\}]
            Mkt\PYGZhy{}RF     SMB     HML     RF    Mkt
Date                                            
1926\PYGZhy{}07\PYGZhy{}01  0.0010 \PYGZhy{}0.0025 \PYGZhy{}0.0027 0.0001 0.0011
1926\PYGZhy{}07\PYGZhy{}02  0.0045 \PYGZhy{}0.0033 \PYGZhy{}0.0006 0.0001 0.0046
1926\PYGZhy{}07\PYGZhy{}06  0.0017  0.0030 \PYGZhy{}0.0039 0.0001 0.0018
1926\PYGZhy{}07\PYGZhy{}07  0.0009 \PYGZhy{}0.0058  0.0002 0.0001 0.0010
1926\PYGZhy{}07\PYGZhy{}08  0.0021 \PYGZhy{}0.0038  0.0019 0.0001 0.0022
\end{sphinxVerbatim}

\end{sphinxuseclass}\end{sphinxVerbatimOutput}

\end{sphinxuseclass}
\sphinxAtStartPar
We can use market returns to impute market prices relative to the last day of June 1926.

\begin{sphinxuseclass}{cell}\begin{sphinxVerbatimInput}

\begin{sphinxuseclass}{cell_input}
\begin{sphinxVerbatim}[commandchars=\\\{\}]
\PYG{n}{price} \PYG{o}{=} \PYG{n}{ff}\PYG{p}{[}\PYG{l+s+s1}{\PYGZsq{}}\PYG{l+s+s1}{Mkt}\PYG{l+s+s1}{\PYGZsq{}}\PYG{p}{]}\PYG{o}{.}\PYG{n}{add}\PYG{p}{(}\PYG{l+m+mi}{1}\PYG{p}{)}\PYG{o}{.}\PYG{n}{cumprod}\PYG{p}{(}\PYG{p}{)}

\PYG{n}{price}\PYG{o}{.}\PYG{n}{tail}\PYG{p}{(}\PYG{p}{)}
\end{sphinxVerbatim}

\end{sphinxuseclass}\end{sphinxVerbatimInput}
\begin{sphinxVerbatimOutput}

\begin{sphinxuseclass}{cell_output}
\begin{sphinxVerbatim}[commandchars=\\\{\}]
Date
2022\PYGZhy{}12\PYGZhy{}23   9266.3996
2022\PYGZhy{}12\PYGZhy{}27   9220.6236
2022\PYGZhy{}12\PYGZhy{}28   9108.6852
2022\PYGZhy{}12\PYGZhy{}29   9280.4750
2022\PYGZhy{}12\PYGZhy{}30   9261.5429
Name: Mkt, dtype: float64
\end{sphinxVerbatim}

\end{sphinxuseclass}\end{sphinxVerbatimOutput}

\end{sphinxuseclass}
\begin{sphinxuseclass}{cell}\begin{sphinxVerbatimInput}

\begin{sphinxuseclass}{cell_input}
\begin{sphinxVerbatim}[commandchars=\\\{\}]
\PYG{n}{price}\PYG{o}{.}\PYG{n}{plot}\PYG{p}{(}\PYG{p}{)}
\PYG{n}{plt}\PYG{o}{.}\PYG{n}{title}\PYG{p}{(}\PYG{l+s+s1}{\PYGZsq{}}\PYG{l+s+s1}{Imputed Market Prices}\PYG{l+s+se}{\PYGZbs{}n}\PYG{l+s+s1}{Assuming \PYGZdl{}1 on the Last Day of June 1926}\PYG{l+s+s1}{\PYGZsq{}}\PYG{p}{)}
\PYG{n}{plt}\PYG{o}{.}\PYG{n}{ylabel}\PYG{p}{(}\PYG{l+s+s1}{\PYGZsq{}}\PYG{l+s+s1}{Imputed Market Price (\PYGZdl{})}\PYG{l+s+s1}{\PYGZsq{}}\PYG{p}{)}
\PYG{n}{plt}\PYG{o}{.}\PYG{n}{semilogy}\PYG{p}{(}\PYG{p}{)}
\PYG{n}{plt}\PYG{o}{.}\PYG{n}{show}\PYG{p}{(}\PYG{p}{)}
\end{sphinxVerbatim}

\end{sphinxuseclass}\end{sphinxVerbatimInput}
\begin{sphinxVerbatimOutput}

\begin{sphinxuseclass}{cell_output}
\noindent\sphinxincludegraphics{{74bcadb8c1c6207545ff28a0384a656c7173b0d1bde628305a9a10ce18b15a66}.png}

\end{sphinxuseclass}\end{sphinxVerbatimOutput}

\end{sphinxuseclass}
\sphinxAtStartPar
We need lagged prices to estimate \$\textbackslash{}rho\$.
We will add 10 lags of \$P\$ to help us understand the relation between past and future prices.

\begin{sphinxuseclass}{cell}\begin{sphinxVerbatimInput}

\begin{sphinxuseclass}{cell_input}
\begin{sphinxVerbatim}[commandchars=\\\{\}]
\PYG{n}{price\PYGZus{}lags} \PYG{o}{=} \PYG{p}{(}
    \PYG{n}{pd}\PYG{o}{.}\PYG{n}{concat}\PYG{p}{(}
        \PYG{n}{objs}\PYG{o}{=}\PYG{p}{[}\PYG{n}{price}\PYG{o}{.}\PYG{n}{shift}\PYG{p}{(}\PYG{n}{t}\PYG{p}{)} \PYG{k}{for} \PYG{n}{t} \PYG{o+ow}{in} \PYG{n+nb}{range}\PYG{p}{(}\PYG{l+m+mi}{11}\PYG{p}{)}\PYG{p}{]}\PYG{p}{,}
        \PYG{n}{keys}\PYG{o}{=}\PYG{p}{[}\PYG{l+s+sa}{f}\PYG{l+s+s1}{\PYGZsq{}}\PYG{l+s+s1}{Lag }\PYG{l+s+si}{\PYGZob{}}\PYG{n}{t}\PYG{l+s+si}{\PYGZcb{}}\PYG{l+s+s1}{\PYGZsq{}} \PYG{k}{for} \PYG{n}{t} \PYG{o+ow}{in} \PYG{n+nb}{range}\PYG{p}{(}\PYG{l+m+mi}{11}\PYG{p}{)}\PYG{p}{]}\PYG{p}{,}
        \PYG{n}{names}\PYG{o}{=}\PYG{p}{[}\PYG{l+s+s1}{\PYGZsq{}}\PYG{l+s+s1}{Price}\PYG{l+s+s1}{\PYGZsq{}}\PYG{p}{]}\PYG{p}{,}
        \PYG{n}{axis}\PYG{o}{=}\PYG{l+m+mi}{1}\PYG{p}{,}
    \PYG{p}{)}
\PYG{p}{)}

\PYG{n}{price\PYGZus{}lags}\PYG{o}{.}\PYG{n}{head}\PYG{p}{(}\PYG{p}{)}
\end{sphinxVerbatim}

\end{sphinxuseclass}\end{sphinxVerbatimInput}
\begin{sphinxVerbatimOutput}

\begin{sphinxuseclass}{cell_output}
\begin{sphinxVerbatim}[commandchars=\\\{\}]
Price       Lag 0  Lag 1  Lag 2  Lag 3  Lag 4  Lag 5  Lag 6  Lag 7  Lag 8  \PYGZbs{}
Date                                                                        
1926\PYGZhy{}07\PYGZhy{}01 1.0011    NaN    NaN    NaN    NaN    NaN    NaN    NaN    NaN   
1926\PYGZhy{}07\PYGZhy{}02 1.0057 1.0011    NaN    NaN    NaN    NaN    NaN    NaN    NaN   
1926\PYGZhy{}07\PYGZhy{}06 1.0075 1.0057 1.0011    NaN    NaN    NaN    NaN    NaN    NaN   
1926\PYGZhy{}07\PYGZhy{}07 1.0085 1.0075 1.0057 1.0011    NaN    NaN    NaN    NaN    NaN   
1926\PYGZhy{}07\PYGZhy{}08 1.0107 1.0085 1.0075 1.0057 1.0011    NaN    NaN    NaN    NaN   

Price       Lag 9  Lag 10  
Date                       
1926\PYGZhy{}07\PYGZhy{}01    NaN     NaN  
1926\PYGZhy{}07\PYGZhy{}02    NaN     NaN  
1926\PYGZhy{}07\PYGZhy{}06    NaN     NaN  
1926\PYGZhy{}07\PYGZhy{}07    NaN     NaN  
1926\PYGZhy{}07\PYGZhy{}08    NaN     NaN  
\end{sphinxVerbatim}

\end{sphinxuseclass}\end{sphinxVerbatimOutput}

\end{sphinxuseclass}
\begin{sphinxuseclass}{cell}\begin{sphinxVerbatimInput}

\begin{sphinxuseclass}{cell_input}
\begin{sphinxVerbatim}[commandchars=\\\{\}]
\PYG{p}{(}
    \PYG{n}{price\PYGZus{}lags}
    \PYG{o}{.}\PYG{n}{dropna}\PYG{p}{(}\PYG{p}{)}
    \PYG{o}{.}\PYG{n}{corr}\PYG{p}{(}\PYG{p}{)}
    \PYG{o}{.}\PYG{n}{loc}\PYG{p}{[}\PYG{l+s+s1}{\PYGZsq{}}\PYG{l+s+s1}{Lag 0}\PYG{l+s+s1}{\PYGZsq{}}\PYG{p}{]}
    \PYG{o}{.}\PYG{n}{plot}\PYG{p}{(}\PYG{n}{kind}\PYG{o}{=}\PYG{l+s+s1}{\PYGZsq{}}\PYG{l+s+s1}{bar}\PYG{l+s+s1}{\PYGZsq{}}\PYG{p}{)}
\PYG{p}{)}
\PYG{n}{plt}\PYG{o}{.}\PYG{n}{title}\PYG{p}{(}\PYG{l+s+s1}{\PYGZsq{}}\PYG{l+s+s1}{Correlations with Imputed Market Price}\PYG{l+s+s1}{\PYGZsq{}}\PYG{p}{)}
\PYG{n}{plt}\PYG{o}{.}\PYG{n}{ylabel}\PYG{p}{(}\PYG{l+s+s1}{\PYGZsq{}}\PYG{l+s+s1}{Correlation}\PYG{l+s+s1}{\PYGZsq{}}\PYG{p}{)}
\PYG{n}{plt}\PYG{o}{.}\PYG{n}{show}\PYG{p}{(}\PYG{p}{)}
\end{sphinxVerbatim}

\end{sphinxuseclass}\end{sphinxVerbatimInput}
\begin{sphinxVerbatimOutput}

\begin{sphinxuseclass}{cell_output}
\noindent\sphinxincludegraphics{{eac89e8ff5500058d692b5612039ae3acfea74fe994cd182cee82f9787c996a8}.png}

\end{sphinxuseclass}\end{sphinxVerbatimOutput}

\end{sphinxuseclass}
\sphinxAtStartPar
But these are \sphinxstyleemphasis{pairwise} correlations.
If we estimate \sphinxstyleemphasis{conditional} correlations, we see that all the price information is in the first lag!

\begin{sphinxuseclass}{cell}\begin{sphinxVerbatimInput}

\begin{sphinxuseclass}{cell_input}
\begin{sphinxVerbatim}[commandchars=\\\{\}]
\PYG{k+kn}{import} \PYG{n+nn}{statsmodels}\PYG{n+nn}{.}\PYG{n+nn}{api} \PYG{k}{as} \PYG{n+nn}{sm}
\end{sphinxVerbatim}

\end{sphinxuseclass}\end{sphinxVerbatimInput}

\end{sphinxuseclass}
\begin{sphinxuseclass}{cell}\begin{sphinxVerbatimInput}

\begin{sphinxuseclass}{cell_input}
\begin{sphinxVerbatim}[commandchars=\\\{\}]
\PYG{n}{\PYGZus{}} \PYG{o}{=} \PYG{n}{price\PYGZus{}lags}\PYG{o}{.}\PYG{n}{dropna}\PYG{p}{(}\PYG{p}{)}
\PYG{n}{y} \PYG{o}{=} \PYG{n}{\PYGZus{}}\PYG{p}{[}\PYG{l+s+s1}{\PYGZsq{}}\PYG{l+s+s1}{Lag 0}\PYG{l+s+s1}{\PYGZsq{}}\PYG{p}{]}
\PYG{n}{X} \PYG{o}{=} \PYG{n}{\PYGZus{}}\PYG{o}{.}\PYG{n}{drop}\PYG{p}{(}\PYG{l+s+s1}{\PYGZsq{}}\PYG{l+s+s1}{Lag 0}\PYG{l+s+s1}{\PYGZsq{}}\PYG{p}{,} \PYG{n}{axis}\PYG{o}{=}\PYG{l+m+mi}{1}\PYG{p}{)}\PYG{o}{.}\PYG{n}{pipe}\PYG{p}{(}\PYG{n}{sm}\PYG{o}{.}\PYG{n}{add\PYGZus{}constant}\PYG{p}{)}
\PYG{n}{model} \PYG{o}{=} \PYG{n}{sm}\PYG{o}{.}\PYG{n}{OLS}\PYG{p}{(}\PYG{n}{endog}\PYG{o}{=}\PYG{n}{y}\PYG{p}{,} \PYG{n}{exog}\PYG{o}{=}\PYG{n}{X}\PYG{p}{)}
\PYG{n}{fit} \PYG{o}{=} \PYG{n}{model}\PYG{o}{.}\PYG{n}{fit}\PYG{p}{(}\PYG{n}{cov\PYGZus{}type}\PYG{o}{=}\PYG{l+s+s1}{\PYGZsq{}}\PYG{l+s+s1}{HAC}\PYG{l+s+s1}{\PYGZsq{}}\PYG{p}{,} \PYG{n}{cov\PYGZus{}kwds}\PYG{o}{=}\PYG{p}{\PYGZob{}}\PYG{l+s+s1}{\PYGZsq{}}\PYG{l+s+s1}{maxlags}\PYG{l+s+s1}{\PYGZsq{}}\PYG{p}{:} \PYG{l+m+mi}{10}\PYG{p}{\PYGZcb{}}\PYG{p}{)}
\PYG{n}{fit}\PYG{o}{.}\PYG{n}{summary}\PYG{p}{(}\PYG{p}{)}
\end{sphinxVerbatim}

\end{sphinxuseclass}\end{sphinxVerbatimInput}
\begin{sphinxVerbatimOutput}

\begin{sphinxuseclass}{cell_output}
\begin{sphinxVerbatim}[commandchars=\\\{\}]
\PYGZlt{}class \PYGZsq{}statsmodels.iolib.summary.Summary\PYGZsq{}\PYGZgt{}
\PYGZdq{}\PYGZdq{}\PYGZdq{}
                            OLS Regression Results                            
==============================================================================
Dep. Variable:                  Lag 0   R\PYGZhy{}squared:                       1.000
Model:                            OLS   Adj. R\PYGZhy{}squared:                  1.000
Method:                 Least Squares   F\PYGZhy{}statistic:                 1.505e+06
Date:                Mon, 13 Mar 2023   Prob (F\PYGZhy{}statistic):               0.00
Time:                        14:11:49   Log\PYGZhy{}Likelihood:            \PYGZhy{}1.1988e+05
No. Observations:               25389   AIC:                         2.398e+05
Df Residuals:                   25378   BIC:                         2.399e+05
Df Model:                          10                                         
Covariance Type:                  HAC                                         
==============================================================================
                 coef    std err          z      P\PYGZgt{}|z|      [0.025      0.975]
\PYGZhy{}\PYGZhy{}\PYGZhy{}\PYGZhy{}\PYGZhy{}\PYGZhy{}\PYGZhy{}\PYGZhy{}\PYGZhy{}\PYGZhy{}\PYGZhy{}\PYGZhy{}\PYGZhy{}\PYGZhy{}\PYGZhy{}\PYGZhy{}\PYGZhy{}\PYGZhy{}\PYGZhy{}\PYGZhy{}\PYGZhy{}\PYGZhy{}\PYGZhy{}\PYGZhy{}\PYGZhy{}\PYGZhy{}\PYGZhy{}\PYGZhy{}\PYGZhy{}\PYGZhy{}\PYGZhy{}\PYGZhy{}\PYGZhy{}\PYGZhy{}\PYGZhy{}\PYGZhy{}\PYGZhy{}\PYGZhy{}\PYGZhy{}\PYGZhy{}\PYGZhy{}\PYGZhy{}\PYGZhy{}\PYGZhy{}\PYGZhy{}\PYGZhy{}\PYGZhy{}\PYGZhy{}\PYGZhy{}\PYGZhy{}\PYGZhy{}\PYGZhy{}\PYGZhy{}\PYGZhy{}\PYGZhy{}\PYGZhy{}\PYGZhy{}\PYGZhy{}\PYGZhy{}\PYGZhy{}\PYGZhy{}\PYGZhy{}\PYGZhy{}\PYGZhy{}\PYGZhy{}\PYGZhy{}\PYGZhy{}\PYGZhy{}\PYGZhy{}\PYGZhy{}\PYGZhy{}\PYGZhy{}\PYGZhy{}\PYGZhy{}\PYGZhy{}\PYGZhy{}\PYGZhy{}\PYGZhy{}
const          0.1325      0.155      0.854      0.393      \PYGZhy{}0.171       0.436
Lag 1          0.9377      0.037     25.613      0.000       0.866       1.009
Lag 2          0.0918      0.064      1.445      0.148      \PYGZhy{}0.033       0.216
Lag 3         \PYGZhy{}0.0359      0.041     \PYGZhy{}0.880      0.379      \PYGZhy{}0.116       0.044
Lag 4         \PYGZhy{}0.0331      0.046     \PYGZhy{}0.714      0.475      \PYGZhy{}0.124       0.058
Lag 5          0.0473      0.042      1.122      0.262      \PYGZhy{}0.035       0.130
Lag 6         \PYGZhy{}0.0676      0.045     \PYGZhy{}1.504      0.133      \PYGZhy{}0.156       0.020
Lag 7          0.1294      0.052      2.478      0.013       0.027       0.232
Lag 8         \PYGZhy{}0.1415      0.059     \PYGZhy{}2.400      0.016      \PYGZhy{}0.257      \PYGZhy{}0.026
Lag 9          0.1608      0.050      3.236      0.001       0.063       0.258
Lag 10        \PYGZhy{}0.0887      0.034     \PYGZhy{}2.576      0.010      \PYGZhy{}0.156      \PYGZhy{}0.021
==============================================================================
Omnibus:                    17230.148   Durbin\PYGZhy{}Watson:                   1.995
Prob(Omnibus):                  0.000   Jarque\PYGZhy{}Bera (JB):          6700094.359
Skew:                          \PYGZhy{}2.073   Prob(JB):                         0.00
Kurtosis:                      82.475   Cond. No.                     8.09e+03
==============================================================================

Notes:
[1] Standard Errors are heteroscedasticity and autocorrelation robust (HAC) using 10 lags and without small sample correction
[2] The condition number is large, 8.09e+03. This might indicate that there are
strong multicollinearity or other numerical problems.
\PYGZdq{}\PYGZdq{}\PYGZdq{}
\end{sphinxVerbatim}

\end{sphinxuseclass}\end{sphinxVerbatimOutput}

\end{sphinxuseclass}
\begin{sphinxuseclass}{cell}\begin{sphinxVerbatimInput}

\begin{sphinxuseclass}{cell_input}
\begin{sphinxVerbatim}[commandchars=\\\{\}]
\PYG{n}{plt}\PYG{o}{.}\PYG{n}{bar}\PYG{p}{(}
    \PYG{n}{x}\PYG{o}{=}\PYG{n}{price\PYGZus{}lags}\PYG{o}{.}\PYG{n}{columns}\PYG{p}{[}\PYG{l+m+mi}{1}\PYG{p}{:}\PYG{p}{]}\PYG{p}{,}
    \PYG{n}{height}\PYG{o}{=}\PYG{n}{fit}\PYG{o}{.}\PYG{n}{params}\PYG{p}{[}\PYG{l+m+mi}{1}\PYG{p}{:}\PYG{p}{]}\PYG{p}{,}
    \PYG{n}{yerr}\PYG{o}{=}\PYG{l+m+mi}{2}\PYG{o}{*}\PYG{n}{fit}\PYG{o}{.}\PYG{n}{bse}\PYG{p}{[}\PYG{l+m+mi}{1}\PYG{p}{:}\PYG{p}{]}
\PYG{p}{)}
\PYG{n}{plt}\PYG{o}{.}\PYG{n}{title}\PYG{p}{(}\PYG{l+s+s1}{\PYGZsq{}}\PYG{l+s+s1}{Conditional Correlations with Imputed Market Price}\PYG{l+s+se}{\PYGZbs{}n}\PYG{l+s+s1}{Bars Indicate \PYGZdl{}}\PYG{l+s+s1}{\PYGZbs{}}\PYG{l+s+s1}{pm 2\PYGZdl{} Standard Errors}\PYG{l+s+s1}{\PYGZsq{}}\PYG{p}{)}
\PYG{n}{plt}\PYG{o}{.}\PYG{n}{ylabel}\PYG{p}{(}\PYG{l+s+s1}{\PYGZsq{}}\PYG{l+s+s1}{Conditional Correlation}\PYG{l+s+s1}{\PYGZsq{}}\PYG{p}{)}
\PYG{n}{plt}\PYG{o}{.}\PYG{n}{xlabel}\PYG{p}{(}\PYG{l+s+s1}{\PYGZsq{}}\PYG{l+s+s1}{Price}\PYG{l+s+s1}{\PYGZsq{}}\PYG{p}{)}
\PYG{n}{plt}\PYG{o}{.}\PYG{n}{show}\PYG{p}{(}\PYG{p}{)}
\end{sphinxVerbatim}

\end{sphinxuseclass}\end{sphinxVerbatimInput}
\begin{sphinxVerbatimOutput}

\begin{sphinxuseclass}{cell_output}
\noindent\sphinxincludegraphics{{987afc9d4034d1d563d4820055e62ac2b57f72da7a88925775f24682af7908f2}.png}

\end{sphinxuseclass}\end{sphinxVerbatimOutput}

\end{sphinxuseclass}

\subsubsection{Signal\sphinxhyphen{}to\sphinxhyphen{}Noise Ratio}
\label{\detokenize{herron_02_lecture:signal-to-noise-ratio}}
\sphinxAtStartPar
Recall, we can express a random walk as \$P\_\{t\} = \textbackslash{}rho P\_\{t\sphinxhyphen{}1\} + m P\_\{t\sphinxhyphen{}1\} + \textbackslash{}varepsilon\_t\$.
Since \$\textbackslash{}rho = 1\$, we can subtract \$P\_\{t\sphinxhyphen{}1\}\$ from both sides, then divide by \$P\_\{t\sphinxhyphen{}1\}\$ on both sides.
This transformation expresses a random walk in terms of returns: \$r\_\{t\sphinxhyphen{}1,t\} = m + e\_t\$, where \$E{[}e\_t{]} = 0\$ and \$SD{[}e\_t{]} = s\$, so \$E{[}r\_\{t\sphinxhyphen{}1, t\}{]} = m\$.
We can think of the signal\sphinxhyphen{}to\sphinxhyphen{}noise ratio as \$\textbackslash{}frac\{m\}\{s\}\$.
How high is this ratio?

\begin{sphinxuseclass}{cell}\begin{sphinxVerbatimInput}

\begin{sphinxuseclass}{cell_input}
\begin{sphinxVerbatim}[commandchars=\\\{\}]
\PYG{n}{m}\PYG{p}{,} \PYG{n}{s} \PYG{o}{=} \PYG{n}{ff}\PYG{p}{[}\PYG{l+s+s1}{\PYGZsq{}}\PYG{l+s+s1}{Mkt}\PYG{l+s+s1}{\PYGZsq{}}\PYG{p}{]}\PYG{o}{.}\PYG{n}{mean}\PYG{p}{(}\PYG{p}{)}\PYG{p}{,} \PYG{n}{ff}\PYG{p}{[}\PYG{l+s+s1}{\PYGZsq{}}\PYG{l+s+s1}{Mkt}\PYG{l+s+s1}{\PYGZsq{}}\PYG{p}{]}\PYG{o}{.}\PYG{n}{std}\PYG{p}{(}\PYG{p}{)}
\end{sphinxVerbatim}

\end{sphinxuseclass}\end{sphinxVerbatimInput}

\end{sphinxuseclass}
\sphinxAtStartPar
Here \$m\$ is about 4 basis points per day!

\begin{sphinxuseclass}{cell}\begin{sphinxVerbatimInput}

\begin{sphinxuseclass}{cell_input}
\begin{sphinxVerbatim}[commandchars=\\\{\}]
\PYG{n}{m}
\end{sphinxVerbatim}

\end{sphinxuseclass}\end{sphinxVerbatimInput}
\begin{sphinxVerbatimOutput}

\begin{sphinxuseclass}{cell_output}
\begin{sphinxVerbatim}[commandchars=\\\{\}]
0.0004
\end{sphinxVerbatim}

\end{sphinxuseclass}\end{sphinxVerbatimOutput}

\end{sphinxuseclass}
\sphinxAtStartPar
However, \$s\$ is about 108 basis points per day!

\begin{sphinxuseclass}{cell}\begin{sphinxVerbatimInput}

\begin{sphinxuseclass}{cell_input}
\begin{sphinxVerbatim}[commandchars=\\\{\}]
\PYG{n}{s}
\end{sphinxVerbatim}

\end{sphinxuseclass}\end{sphinxVerbatimInput}
\begin{sphinxVerbatimOutput}

\begin{sphinxuseclass}{cell_output}
\begin{sphinxVerbatim}[commandchars=\\\{\}]
0.0108
\end{sphinxVerbatim}

\end{sphinxuseclass}\end{sphinxVerbatimOutput}

\end{sphinxuseclass}
\begin{sphinxuseclass}{cell}\begin{sphinxVerbatimInput}

\begin{sphinxuseclass}{cell_input}
\begin{sphinxVerbatim}[commandchars=\\\{\}]
\PYG{n}{m}\PYG{o}{/}\PYG{n}{s}
\end{sphinxVerbatim}

\end{sphinxuseclass}\end{sphinxVerbatimInput}
\begin{sphinxVerbatimOutput}

\begin{sphinxuseclass}{cell_output}
\begin{sphinxVerbatim}[commandchars=\\\{\}]
0.0386
\end{sphinxVerbatim}

\end{sphinxuseclass}\end{sphinxVerbatimOutput}

\end{sphinxuseclass}
\sphinxAtStartPar
Means grow linearly with time, while standard deviations growth with the square\sphinxhyphen{}root of time.
Therefore, even with a 1 basis point signal, we need a lot of data to make sure this 1 basis point signal is real!
For example, if we want \$\textbackslash{}sqrt\{t\} \textbackslash{}times \textbackslash{}frac\{1\}\{s\} \textbackslash{}geq 2\$, we need \$t \textbackslash{}geq \textbackslash{}left(2 \textbackslash{}times \textbackslash{}frac\{s\}\{1\} \textbackslash{}right)\textasciicircum{}2\$ days!
Even with market noise, which is diversified and low, we need more than 100 years of data!

\begin{sphinxuseclass}{cell}\begin{sphinxVerbatimInput}

\begin{sphinxuseclass}{cell_input}
\begin{sphinxVerbatim}[commandchars=\\\{\}]
\PYG{p}{(}\PYG{l+m+mi}{2} \PYG{o}{*} \PYG{n}{s} \PYG{o}{/} \PYG{l+m+mf}{0.0001}\PYG{p}{)}\PYG{o}{*}\PYG{o}{*}\PYG{l+m+mi}{2} \PYG{o}{/} \PYG{l+m+mi}{365}
\end{sphinxVerbatim}

\end{sphinxuseclass}\end{sphinxVerbatimInput}
\begin{sphinxVerbatimOutput}

\begin{sphinxuseclass}{cell_output}
\begin{sphinxVerbatim}[commandchars=\\\{\}]
128.4479
\end{sphinxVerbatim}

\end{sphinxuseclass}\end{sphinxVerbatimOutput}

\end{sphinxuseclass}

\section{Implement a simple moving average (SMA) trading strategy}
\label{\detokenize{herron_02_lecture:implement-a-simple-moving-average-sma-trading-strategy}}
\sphinxAtStartPar
One goal of technical analysis is to “buy low, and sell high”.
The \$n\$\sphinxhyphen{}day SMA reduces noise in market prices, removing market fluctuations and providing estimates of “true” prices.
Market prices below their SMA might be buying opportunities, and market prices below their SMA might be selling opportunities.
Here, we will implement a long\sphinxhyphen{}only 20\sphinxhyphen{}day SMA (SMA(20)) strategy with Bitcoin:
\begin{enumerate}
\sphinxsetlistlabels{\arabic}{enumi}{enumii}{}{.}%
\item {} 
\sphinxAtStartPar
Buy when the closing price crosses SMA(20) from below

\item {} 
\sphinxAtStartPar
Sell when the closing price crosses SMA(20) from above

\item {} 
\sphinxAtStartPar
No short\sphinxhyphen{}selling

\end{enumerate}

\sphinxAtStartPar
Because we will not sell short, we can simplify this strategy to “long if above SMA(20), otherwise neutral”.
First, we will need Bitcoin returns data.

\begin{sphinxuseclass}{cell}\begin{sphinxVerbatimInput}

\begin{sphinxuseclass}{cell_input}
\begin{sphinxVerbatim}[commandchars=\\\{\}]
\PYG{k+kn}{import} \PYG{n+nn}{yfinance} \PYG{k}{as} \PYG{n+nn}{yf}
\end{sphinxVerbatim}

\end{sphinxuseclass}\end{sphinxVerbatimInput}

\end{sphinxuseclass}
\begin{sphinxuseclass}{cell}\begin{sphinxVerbatimInput}

\begin{sphinxuseclass}{cell_input}
\begin{sphinxVerbatim}[commandchars=\\\{\}]
\PYG{n}{btc} \PYG{o}{=} \PYG{p}{(}
    \PYG{n}{yf}\PYG{o}{.}\PYG{n}{download}\PYG{p}{(}\PYG{n}{tickers}\PYG{o}{=}\PYG{l+s+s1}{\PYGZsq{}}\PYG{l+s+s1}{BTC\PYGZhy{}USD}\PYG{l+s+s1}{\PYGZsq{}}\PYG{p}{,} \PYG{n}{progress}\PYG{o}{=}\PYG{k+kc}{False}\PYG{p}{)}
    \PYG{o}{.}\PYG{n}{assign}\PYG{p}{(}
        \PYG{n}{Date} \PYG{o}{=} \PYG{k}{lambda} \PYG{n}{x}\PYG{p}{:} \PYG{n}{x}\PYG{o}{.}\PYG{n}{index}\PYG{o}{.}\PYG{n}{tz\PYGZus{}localize}\PYG{p}{(}\PYG{k+kc}{None}\PYG{p}{)}\PYG{p}{,}
        \PYG{n}{Return} \PYG{o}{=} \PYG{k}{lambda} \PYG{n}{x}\PYG{p}{:} \PYG{n}{x}\PYG{p}{[}\PYG{l+s+s1}{\PYGZsq{}}\PYG{l+s+s1}{Adj Close}\PYG{l+s+s1}{\PYGZsq{}}\PYG{p}{]}\PYG{o}{.}\PYG{n}{pct\PYGZus{}change}\PYG{p}{(}\PYG{p}{)}\PYG{p}{,}
    \PYG{p}{)}
    \PYG{o}{.}\PYG{n}{set\PYGZus{}index}\PYG{p}{(}\PYG{l+s+s1}{\PYGZsq{}}\PYG{l+s+s1}{Date}\PYG{l+s+s1}{\PYGZsq{}}\PYG{p}{)}
    \PYG{o}{.}\PYG{n}{rename\PYGZus{}axis}\PYG{p}{(}\PYG{n}{columns}\PYG{o}{=}\PYG{l+s+s1}{\PYGZsq{}}\PYG{l+s+s1}{Variable}\PYG{l+s+s1}{\PYGZsq{}}\PYG{p}{)}
\PYG{p}{)}

\PYG{n}{btc}\PYG{o}{.}\PYG{n}{head}\PYG{p}{(}\PYG{p}{)}
\end{sphinxVerbatim}

\end{sphinxuseclass}\end{sphinxVerbatimInput}
\begin{sphinxVerbatimOutput}

\begin{sphinxuseclass}{cell_output}
\begin{sphinxVerbatim}[commandchars=\\\{\}]
Variable       Open     High      Low    Close  Adj Close    Volume  Return
Date                                                                       
2014\PYGZhy{}09\PYGZhy{}17 465.8640 468.1740 452.4220 457.3340   457.3340  21056800     NaN
2014\PYGZhy{}09\PYGZhy{}18 456.8600 456.8600 413.1040 424.4400   424.4400  34483200 \PYGZhy{}0.0719
2014\PYGZhy{}09\PYGZhy{}19 424.1030 427.8350 384.5320 394.7960   394.7960  37919700 \PYGZhy{}0.0698
2014\PYGZhy{}09\PYGZhy{}20 394.6730 423.2960 389.8830 408.9040   408.9040  36863600  0.0357
2014\PYGZhy{}09\PYGZhy{}21 408.0850 412.4260 393.1810 398.8210   398.8210  26580100 \PYGZhy{}0.0247
\end{sphinxVerbatim}

\end{sphinxuseclass}\end{sphinxVerbatimOutput}

\end{sphinxuseclass}
\sphinxAtStartPar
Next we:
\begin{enumerate}
\sphinxsetlistlabels{\arabic}{enumi}{enumii}{}{.}%
\item {} 
\sphinxAtStartPar
Use \sphinxcode{\sphinxupquote{.rolling(20).mean()}} to add a \sphinxcode{\sphinxupquote{SMA20}} column containing SMA(20) to our \sphinxcode{\sphinxupquote{btc}} data frame

\item {} 
\sphinxAtStartPar
Use \sphinxcode{\sphinxupquote{np.select()}} to add a \sphinxcode{\sphinxupquote{Position}} column containing:
\begin{enumerate}
\sphinxsetlistlabels{\arabic}{enumii}{enumiii}{}{.}%
\item {} 
\sphinxAtStartPar
\sphinxcode{\sphinxupquote{1}} (long) when the adjusted close is greater than SMA(20)

\item {} 
\sphinxAtStartPar
\sphinxcode{\sphinxupquote{0}} (neutral) when the adjusted close is less than (or equal to) SMA(20)

\item {} 
\sphinxAtStartPar
We could use \sphinxcode{\sphinxupquote{np.where()}} instead of \sphinxcode{\sphinxupquote{np.select()}}, but using \sphinxcode{\sphinxupquote{np.select()}} provides a more flexible framework for more complex examples

\item {} 
\sphinxAtStartPar
\sphinxstylestrong{We use \sphinxcode{\sphinxupquote{.shift()}} to compare yesterday’s closing prices, avoiding a look\sphinxhyphen{}ahead bias}

\end{enumerate}

\item {} 
\sphinxAtStartPar
Add a \sphinxcode{\sphinxupquote{Strategy}} column containing:
\begin{enumerate}
\sphinxsetlistlabels{\arabic}{enumii}{enumiii}{}{.}%
\item {} 
\sphinxAtStartPar
\sphinxcode{\sphinxupquote{Return}} if \sphinxcode{\sphinxupquote{Position == 1}}

\item {} 
\sphinxAtStartPar
\sphinxcode{\sphinxupquote{0}} if \sphinxcode{\sphinxupquote{Position == 0}}

\item {} 
\sphinxAtStartPar
We could earn the risk\sphinxhyphen{}free rate instead of 0 percent, but earning 0 percent simplifies this example

\end{enumerate}

\end{enumerate}

\begin{sphinxuseclass}{cell}\begin{sphinxVerbatimInput}

\begin{sphinxuseclass}{cell_input}
\begin{sphinxVerbatim}[commandchars=\\\{\}]
\PYG{n}{btc} \PYG{o}{=} \PYG{p}{(}
    \PYG{n}{btc}
    \PYG{o}{.}\PYG{n}{assign}\PYG{p}{(}
        \PYG{n}{SMA20} \PYG{o}{=} \PYG{k}{lambda} \PYG{n}{x}\PYG{p}{:} \PYG{n}{x}\PYG{p}{[}\PYG{l+s+s1}{\PYGZsq{}}\PYG{l+s+s1}{Adj Close}\PYG{l+s+s1}{\PYGZsq{}}\PYG{p}{]}\PYG{o}{.}\PYG{n}{rolling}\PYG{p}{(}\PYG{l+m+mi}{20}\PYG{p}{)}\PYG{o}{.}\PYG{n}{mean}\PYG{p}{(}\PYG{p}{)}\PYG{p}{,}
        \PYG{n}{Position} \PYG{o}{=} \PYG{k}{lambda} \PYG{n}{x}\PYG{p}{:} \PYG{n}{np}\PYG{o}{.}\PYG{n}{select}\PYG{p}{(}
            \PYG{n}{condlist}\PYG{o}{=}\PYG{p}{[}\PYG{n}{x}\PYG{p}{[}\PYG{l+s+s1}{\PYGZsq{}}\PYG{l+s+s1}{Adj Close}\PYG{l+s+s1}{\PYGZsq{}}\PYG{p}{]}\PYG{o}{.}\PYG{n}{shift}\PYG{p}{(}\PYG{p}{)} \PYG{o}{\PYGZgt{}} \PYG{n}{x}\PYG{p}{[}\PYG{l+s+s1}{\PYGZsq{}}\PYG{l+s+s1}{SMA20}\PYG{l+s+s1}{\PYGZsq{}}\PYG{p}{]}\PYG{o}{.}\PYG{n}{shift}\PYG{p}{(}\PYG{p}{)}\PYG{p}{,} \PYG{n}{x}\PYG{p}{[}\PYG{l+s+s1}{\PYGZsq{}}\PYG{l+s+s1}{Adj Close}\PYG{l+s+s1}{\PYGZsq{}}\PYG{p}{]}\PYG{o}{.}\PYG{n}{shift}\PYG{p}{(}\PYG{p}{)} \PYG{o}{\PYGZlt{}}\PYG{o}{=} \PYG{n}{x}\PYG{p}{[}\PYG{l+s+s1}{\PYGZsq{}}\PYG{l+s+s1}{SMA20}\PYG{l+s+s1}{\PYGZsq{}}\PYG{p}{]}\PYG{o}{.}\PYG{n}{shift}\PYG{p}{(}\PYG{p}{)}\PYG{p}{]}\PYG{p}{,}
            \PYG{n}{choicelist}\PYG{o}{=}\PYG{p}{[}\PYG{l+m+mi}{1}\PYG{p}{,} \PYG{l+m+mi}{0}\PYG{p}{]}\PYG{p}{,}
            \PYG{n}{default}\PYG{o}{=}\PYG{n}{np}\PYG{o}{.}\PYG{n}{nan}
        \PYG{p}{)}\PYG{p}{,}
        \PYG{n}{Strategy} \PYG{o}{=} \PYG{k}{lambda} \PYG{n}{x}\PYG{p}{:} \PYG{n}{x}\PYG{p}{[}\PYG{l+s+s1}{\PYGZsq{}}\PYG{l+s+s1}{Position}\PYG{l+s+s1}{\PYGZsq{}}\PYG{p}{]} \PYG{o}{*} \PYG{n}{x}\PYG{p}{[}\PYG{l+s+s1}{\PYGZsq{}}\PYG{l+s+s1}{Return}\PYG{l+s+s1}{\PYGZsq{}}\PYG{p}{]}
    \PYG{p}{)}
\PYG{p}{)}

\PYG{n}{btc}\PYG{o}{.}\PYG{n}{tail}\PYG{p}{(}\PYG{p}{)}
\end{sphinxVerbatim}

\end{sphinxuseclass}\end{sphinxVerbatimInput}
\begin{sphinxVerbatimOutput}

\begin{sphinxuseclass}{cell_output}
\begin{sphinxVerbatim}[commandchars=\\\{\}]
Variable         Open       High        Low      Close  Adj Close  \PYGZbs{}
Date                                                                
2023\PYGZhy{}03\PYGZhy{}09 21720.0801 21802.7168 20210.3066 20363.0215 20363.0215   
2023\PYGZhy{}03\PYGZhy{}10 20367.0020 20370.5957 19628.2539 20187.2441 20187.2441   
2023\PYGZhy{}03\PYGZhy{}11 20187.8770 20792.5254 20068.6602 20632.4102 20632.4102   
2023\PYGZhy{}03\PYGZhy{}12 20628.0293 22185.0312 20448.8066 22163.9492 22163.9492   
2023\PYGZhy{}03\PYGZhy{}13 22178.5801 24427.3906 21918.1992 24229.8164 24229.8164   

Variable         Volume  Return      SMA20  Position  Strategy  
Date                                                            
2023\PYGZhy{}03\PYGZhy{}09  30364664171 \PYGZhy{}0.0624 23198.9943    0.0000   \PYGZhy{}0.0000  
2023\PYGZhy{}03\PYGZhy{}10  39578257695 \PYGZhy{}0.0086 22976.2927    0.0000   \PYGZhy{}0.0000  
2023\PYGZhy{}03\PYGZhy{}11  30180288176  0.0221 22791.5311    0.0000    0.0000  
2023\PYGZhy{}03\PYGZhy{}12  29279035521  0.0742 22658.2711    0.0000    0.0000  
2023\PYGZhy{}03\PYGZhy{}13  54919491584  0.0932 22647.9442    0.0000    0.0000  
\end{sphinxVerbatim}

\end{sphinxuseclass}\end{sphinxVerbatimOutput}

\end{sphinxuseclass}
\sphinxAtStartPar
I find it helpful to plot \sphinxcode{\sphinxupquote{Adj Close}}, \sphinxcode{\sphinxupquote{SMA20}}, and \sphinxcode{\sphinxupquote{Position}} for a sort window with one or more crossings.

\begin{sphinxuseclass}{cell}\begin{sphinxVerbatimInput}

\begin{sphinxuseclass}{cell_input}
\begin{sphinxVerbatim}[commandchars=\\\{\}]
\PYG{n}{fig}\PYG{p}{,} \PYG{n}{ax} \PYG{o}{=} \PYG{n}{plt}\PYG{o}{.}\PYG{n}{subplots}\PYG{p}{(}\PYG{l+m+mi}{2}\PYG{p}{,} \PYG{l+m+mi}{1}\PYG{p}{,} \PYG{n}{sharex}\PYG{o}{=}\PYG{k+kc}{True}\PYG{p}{)}
\PYG{n}{\PYGZus{}} \PYG{o}{=} \PYG{n}{btc}\PYG{o}{.}\PYG{n}{loc}\PYG{p}{[}\PYG{l+s+s1}{\PYGZsq{}}\PYG{l+s+s1}{2023\PYGZhy{}02}\PYG{l+s+s1}{\PYGZsq{}}\PYG{p}{]}\PYG{o}{.}\PYG{n}{iloc}\PYG{p}{[}\PYG{o}{\PYGZhy{}}\PYG{l+m+mi}{15}\PYG{p}{:}\PYG{p}{]}
\PYG{n}{\PYGZus{}}\PYG{p}{[}\PYG{p}{[}\PYG{l+s+s1}{\PYGZsq{}}\PYG{l+s+s1}{Adj Close}\PYG{l+s+s1}{\PYGZsq{}}\PYG{p}{,} \PYG{l+s+s1}{\PYGZsq{}}\PYG{l+s+s1}{SMA20}\PYG{l+s+s1}{\PYGZsq{}}\PYG{p}{]}\PYG{p}{]}\PYG{o}{.}\PYG{n}{plot}\PYG{p}{(}\PYG{n}{ax}\PYG{o}{=}\PYG{n}{ax}\PYG{p}{[}\PYG{l+m+mi}{0}\PYG{p}{]}\PYG{p}{,} \PYG{n}{ylabel}\PYG{o}{=}\PYG{l+s+s1}{\PYGZsq{}}\PYG{l+s+s1}{BTC\PYGZhy{}USD (\PYGZdl{})}\PYG{l+s+s1}{\PYGZsq{}}\PYG{p}{)}
\PYG{n}{\PYGZus{}}\PYG{p}{[}\PYG{p}{[}\PYG{l+s+s1}{\PYGZsq{}}\PYG{l+s+s1}{Position}\PYG{l+s+s1}{\PYGZsq{}}\PYG{p}{]}\PYG{p}{]}\PYG{o}{.}\PYG{n}{plot}\PYG{p}{(}\PYG{n}{ax}\PYG{o}{=}\PYG{n}{ax}\PYG{p}{[}\PYG{l+m+mi}{1}\PYG{p}{]}\PYG{p}{,} \PYG{n}{ylabel}\PYG{o}{=}\PYG{l+s+s1}{\PYGZsq{}}\PYG{l+s+s1}{Position}\PYG{l+s+s1}{\PYGZsq{}}\PYG{p}{,} \PYG{n}{legend}\PYG{o}{=}\PYG{k+kc}{False}\PYG{p}{)}
\PYG{n}{plt}\PYG{o}{.}\PYG{n}{suptitle}\PYG{p}{(}\PYG{l+s+s1}{\PYGZsq{}}\PYG{l+s+s1}{Bitcoin SMA(20) Strategy}\PYG{l+s+s1}{\PYGZsq{}}\PYG{p}{)}
\PYG{n}{plt}\PYG{o}{.}\PYG{n}{show}\PYG{p}{(}\PYG{p}{)}
\end{sphinxVerbatim}

\end{sphinxuseclass}\end{sphinxVerbatimInput}
\begin{sphinxVerbatimOutput}

\begin{sphinxuseclass}{cell_output}
\noindent\sphinxincludegraphics{{b1c6220705b1230b69ccf69d0f0ef53cc41b2587a3049a7a2219d61dcf4d0adc}.png}

\end{sphinxuseclass}\end{sphinxVerbatimOutput}

\end{sphinxuseclass}
\sphinxAtStartPar
We can compare the long\sphinxhyphen{}run performance of buy\sphinxhyphen{}and\sphinxhyphen{}hold and SMA(20).

\begin{sphinxuseclass}{cell}\begin{sphinxVerbatimInput}

\begin{sphinxuseclass}{cell_input}
\begin{sphinxVerbatim}[commandchars=\\\{\}]
\PYG{n}{\PYGZus{}} \PYG{o}{=} \PYG{n}{btc}\PYG{p}{[}\PYG{p}{[}\PYG{l+s+s1}{\PYGZsq{}}\PYG{l+s+s1}{Return}\PYG{l+s+s1}{\PYGZsq{}}\PYG{p}{,} \PYG{l+s+s1}{\PYGZsq{}}\PYG{l+s+s1}{Strategy}\PYG{l+s+s1}{\PYGZsq{}}\PYG{p}{]}\PYG{p}{]}\PYG{o}{.}\PYG{n}{dropna}\PYG{p}{(}\PYG{p}{)}

\PYG{p}{(}
    \PYG{n}{\PYGZus{}}
    \PYG{o}{.}\PYG{n}{add}\PYG{p}{(}\PYG{l+m+mi}{1}\PYG{p}{)}
    \PYG{o}{.}\PYG{n}{cumprod}\PYG{p}{(}\PYG{p}{)}
    \PYG{o}{.}\PYG{n}{rename\PYGZus{}axis}\PYG{p}{(}\PYG{n}{columns}\PYG{o}{=}\PYG{l+s+s1}{\PYGZsq{}}\PYG{l+s+s1}{Strategy}\PYG{l+s+s1}{\PYGZsq{}}\PYG{p}{)}
    \PYG{o}{.}\PYG{n}{rename}\PYG{p}{(}\PYG{n}{columns}\PYG{o}{=}\PYG{p}{\PYGZob{}}\PYG{l+s+s1}{\PYGZsq{}}\PYG{l+s+s1}{Return}\PYG{l+s+s1}{\PYGZsq{}}\PYG{p}{:} \PYG{l+s+s1}{\PYGZsq{}}\PYG{l+s+s1}{Buy\PYGZhy{}And\PYGZhy{}Hold}\PYG{l+s+s1}{\PYGZsq{}}\PYG{p}{,} \PYG{l+s+s1}{\PYGZsq{}}\PYG{l+s+s1}{Strategy}\PYG{l+s+s1}{\PYGZsq{}}\PYG{p}{:} \PYG{l+s+s1}{\PYGZsq{}}\PYG{l+s+s1}{SMA(20)}\PYG{l+s+s1}{\PYGZsq{}}\PYG{p}{\PYGZcb{}}\PYG{p}{)}
    \PYG{o}{.}\PYG{n}{plot}\PYG{p}{(}\PYG{p}{)}
\PYG{p}{)}
\PYG{n}{plt}\PYG{o}{.}\PYG{n}{ylabel}\PYG{p}{(}\PYG{l+s+s1}{\PYGZsq{}}\PYG{l+s+s1}{Value (\PYGZdl{})}\PYG{l+s+s1}{\PYGZsq{}}\PYG{p}{)}
\PYG{n}{plt}\PYG{o}{.}\PYG{n}{title}\PYG{p}{(}\PYG{l+s+sa}{f}\PYG{l+s+s1}{\PYGZsq{}}\PYG{l+s+s1}{Value of \PYGZdl{}1 Invested at Close on }\PYG{l+s+si}{\PYGZob{}}\PYG{n}{\PYGZus{}}\PYG{o}{.}\PYG{n}{index}\PYG{p}{[}\PYG{l+m+mi}{0}\PYG{p}{]} \PYG{o}{\PYGZhy{}} \PYG{n}{pd}\PYG{o}{.}\PYG{n}{offsets}\PYG{o}{.}\PYG{n}{Day}\PYG{p}{(}\PYG{l+m+mi}{1}\PYG{p}{)}\PYG{l+s+si}{:}\PYG{l+s+s1}{\PYGZpc{}B \PYGZpc{}d, \PYGZpc{}Y}\PYG{l+s+si}{\PYGZcb{}}\PYG{l+s+s1}{\PYGZsq{}}\PYG{p}{)}
\PYG{n}{plt}\PYG{o}{.}\PYG{n}{show}\PYG{p}{(}\PYG{p}{)}
\end{sphinxVerbatim}

\end{sphinxuseclass}\end{sphinxVerbatimInput}
\begin{sphinxVerbatimOutput}

\begin{sphinxuseclass}{cell_output}
\noindent\sphinxincludegraphics{{0cad2f7d1526ceebabe827f97c13ba848fbb804f644aca517c641836eded983d}.png}

\end{sphinxuseclass}\end{sphinxVerbatimOutput}

\end{sphinxuseclass}
\sphinxAtStartPar
What about the rolling and full\sphinxhyphen{}sample Sharpe Ratios of these two strategies?
For simplicity, we will assume \$R\_F = 0\$.

\begin{sphinxuseclass}{cell}\begin{sphinxVerbatimInput}

\begin{sphinxuseclass}{cell_input}
\begin{sphinxVerbatim}[commandchars=\\\{\}]
\PYG{p}{(}
    \PYG{n}{btc}
    \PYG{p}{[}\PYG{p}{[}\PYG{l+s+s1}{\PYGZsq{}}\PYG{l+s+s1}{Return}\PYG{l+s+s1}{\PYGZsq{}}\PYG{p}{,} \PYG{l+s+s1}{\PYGZsq{}}\PYG{l+s+s1}{Strategy}\PYG{l+s+s1}{\PYGZsq{}}\PYG{p}{]}\PYG{p}{]}
    \PYG{o}{.}\PYG{n}{dropna}\PYG{p}{(}\PYG{p}{)}
    \PYG{o}{.}\PYG{n}{rolling}\PYG{p}{(}\PYG{l+m+mi}{252}\PYG{p}{)}
    \PYG{o}{.}\PYG{n}{apply}\PYG{p}{(}\PYG{k}{lambda} \PYG{n}{x}\PYG{p}{:} \PYG{n}{np}\PYG{o}{.}\PYG{n}{sqrt}\PYG{p}{(}\PYG{l+m+mi}{252}\PYG{p}{)} \PYG{o}{*} \PYG{n}{x}\PYG{o}{.}\PYG{n}{mean}\PYG{p}{(}\PYG{p}{)} \PYG{o}{/} \PYG{n}{x}\PYG{o}{.}\PYG{n}{std}\PYG{p}{(}\PYG{p}{)}\PYG{p}{)}
    \PYG{o}{.}\PYG{n}{rename\PYGZus{}axis}\PYG{p}{(}\PYG{n}{columns}\PYG{o}{=}\PYG{l+s+s1}{\PYGZsq{}}\PYG{l+s+s1}{Strategy}\PYG{l+s+s1}{\PYGZsq{}}\PYG{p}{)}
    \PYG{o}{.}\PYG{n}{rename}\PYG{p}{(}\PYG{n}{columns}\PYG{o}{=}\PYG{p}{\PYGZob{}}\PYG{l+s+s1}{\PYGZsq{}}\PYG{l+s+s1}{Return}\PYG{l+s+s1}{\PYGZsq{}}\PYG{p}{:} \PYG{l+s+s1}{\PYGZsq{}}\PYG{l+s+s1}{Buy\PYGZhy{}And\PYGZhy{}Hold}\PYG{l+s+s1}{\PYGZsq{}}\PYG{p}{,} \PYG{l+s+s1}{\PYGZsq{}}\PYG{l+s+s1}{Strategy}\PYG{l+s+s1}{\PYGZsq{}}\PYG{p}{:} \PYG{l+s+s1}{\PYGZsq{}}\PYG{l+s+s1}{SMA(20)}\PYG{l+s+s1}{\PYGZsq{}}\PYG{p}{\PYGZcb{}}\PYG{p}{)}
    \PYG{o}{.}\PYG{n}{plot}\PYG{p}{(}\PYG{p}{)}
\PYG{p}{)}
\PYG{n}{plt}\PYG{o}{.}\PYG{n}{ylabel}\PYG{p}{(}\PYG{l+s+s1}{\PYGZsq{}}\PYG{l+s+s1}{Sharpe Ratio}\PYG{l+s+s1}{\PYGZsq{}}\PYG{p}{)}
\PYG{n}{plt}\PYG{o}{.}\PYG{n}{title}\PYG{p}{(}\PYG{l+s+s1}{\PYGZsq{}}\PYG{l+s+s1}{Sharpe Ratios for 252 Trading Day Rolling Windows}\PYG{l+s+s1}{\PYGZsq{}}\PYG{p}{)}
\PYG{n}{plt}\PYG{o}{.}\PYG{n}{show}\PYG{p}{(}\PYG{p}{)}
\end{sphinxVerbatim}

\end{sphinxuseclass}\end{sphinxVerbatimInput}
\begin{sphinxVerbatimOutput}

\begin{sphinxuseclass}{cell_output}
\noindent\sphinxincludegraphics{{80eda870eabd782d081806c4e7e4861495b9e691a8beac56ad81eec443090200}.png}

\end{sphinxuseclass}\end{sphinxVerbatimOutput}

\end{sphinxuseclass}
\begin{sphinxuseclass}{cell}\begin{sphinxVerbatimInput}

\begin{sphinxuseclass}{cell_input}
\begin{sphinxVerbatim}[commandchars=\\\{\}]
\PYG{p}{(}
    \PYG{n}{btc}
    \PYG{p}{[}\PYG{p}{[}\PYG{l+s+s1}{\PYGZsq{}}\PYG{l+s+s1}{Return}\PYG{l+s+s1}{\PYGZsq{}}\PYG{p}{,} \PYG{l+s+s1}{\PYGZsq{}}\PYG{l+s+s1}{Strategy}\PYG{l+s+s1}{\PYGZsq{}}\PYG{p}{]}\PYG{p}{]}
    \PYG{o}{.}\PYG{n}{dropna}\PYG{p}{(}\PYG{p}{)}
    \PYG{o}{.}\PYG{n}{agg}\PYG{p}{(}\PYG{k}{lambda} \PYG{n}{x}\PYG{p}{:} \PYG{n}{np}\PYG{o}{.}\PYG{n}{sqrt}\PYG{p}{(}\PYG{l+m+mi}{252}\PYG{p}{)} \PYG{o}{*} \PYG{n}{x}\PYG{o}{.}\PYG{n}{mean}\PYG{p}{(}\PYG{p}{)} \PYG{o}{/} \PYG{n}{x}\PYG{o}{.}\PYG{n}{std}\PYG{p}{(}\PYG{p}{)}\PYG{p}{)}
    \PYG{o}{.}\PYG{n}{rename\PYGZus{}axis}\PYG{p}{(}\PYG{l+s+s1}{\PYGZsq{}}\PYG{l+s+s1}{Sharpe Ratio}\PYG{l+s+s1}{\PYGZsq{}}\PYG{p}{)}
    \PYG{o}{.}\PYG{n}{rename}\PYG{p}{(}\PYG{p}{\PYGZob{}}\PYG{l+s+s1}{\PYGZsq{}}\PYG{l+s+s1}{Return}\PYG{l+s+s1}{\PYGZsq{}}\PYG{p}{:} \PYG{l+s+s1}{\PYGZsq{}}\PYG{l+s+s1}{Buy\PYGZhy{}And\PYGZhy{}Hold}\PYG{l+s+s1}{\PYGZsq{}}\PYG{p}{,} \PYG{l+s+s1}{\PYGZsq{}}\PYG{l+s+s1}{Strategy}\PYG{l+s+s1}{\PYGZsq{}}\PYG{p}{:} \PYG{l+s+s1}{\PYGZsq{}}\PYG{l+s+s1}{SMA(20)}\PYG{l+s+s1}{\PYGZsq{}}\PYG{p}{\PYGZcb{}}\PYG{p}{)}
\PYG{p}{)}
\end{sphinxVerbatim}

\end{sphinxuseclass}\end{sphinxVerbatimInput}
\begin{sphinxVerbatimOutput}

\begin{sphinxuseclass}{cell_output}
\begin{sphinxVerbatim}[commandchars=\\\{\}]
Sharpe Ratio
Buy\PYGZhy{}And\PYGZhy{}Hold   0.8868
SMA(20)        1.1869
dtype: float64
\end{sphinxVerbatim}

\end{sphinxuseclass}\end{sphinxVerbatimOutput}

\end{sphinxuseclass}
\sphinxAtStartPar
In the practice notebook, we will dig deeper on this strategy and others.

\sphinxstepscope


\section{Herron Topic 2 \sphinxhyphen{} Practice (Blank)}
\label{\detokenize{herron_02_practice:herron-topic-2-practice-blank}}\label{\detokenize{herron_02_practice::doc}}

\subsection{Announcements}
\label{\detokenize{herron_02_practice:announcements}}

\subsection{Practice}
\label{\detokenize{herron_02_practice:practice}}

\subsubsection{Implement the SMA(20) strategy with Bitcoin from the lecture notebook}
\label{\detokenize{herron_02_practice:implement-the-sma-20-strategy-with-bitcoin-from-the-lecture-notebook}}
\sphinxAtStartPar
Try to create the \sphinxcode{\sphinxupquote{btc}} data frame in one code cell with one assignment (i.e., one \sphinxcode{\sphinxupquote{=}}).


\subsubsection{How does SMA(20) outperform buy\sphinxhyphen{}and\sphinxhyphen{}hold with this sample?}
\label{\detokenize{herron_02_practice:how-does-sma-20-outperform-buy-and-hold-with-this-sample}}
\sphinxAtStartPar
Consider the following:
\begin{enumerate}
\sphinxsetlistlabels{\arabic}{enumi}{enumii}{}{.}%
\item {} 
\sphinxAtStartPar
Does SMA(20) avoid the worst performing days? How many of the worst 20 days does SMA(20) avoid? Try the \sphinxcode{\sphinxupquote{.sort\_values()}} or \sphinxcode{\sphinxupquote{.nlargest()}} method.

\item {} 
\sphinxAtStartPar
Does SMA(20) preferentially avoid low\sphinxhyphen{}return days? Try to combine the \sphinxcode{\sphinxupquote{.groupby()}} method and \sphinxcode{\sphinxupquote{pd.qcut()}} function.

\item {} 
\sphinxAtStartPar
Does SMA(20) preferentially avoid high\sphinxhyphen{}volatility days? Try to combine the \sphinxcode{\sphinxupquote{.groupby()}} method and \sphinxcode{\sphinxupquote{pd.qcut()}} function.

\end{enumerate}


\subsubsection{Implement the SMA(20) strategy with the market factor from French}
\label{\detokenize{herron_02_practice:implement-the-sma-20-strategy-with-the-market-factor-from-french}}
\sphinxAtStartPar
We need to impute a market price before we calculate SMA(20).


\subsubsection{How often does SMA(20) outperform buy\sphinxhyphen{}and\sphinxhyphen{}hold with 10\sphinxhyphen{}year rolling windows?}
\label{\detokenize{herron_02_practice:how-often-does-sma-20-outperform-buy-and-hold-with-10-year-rolling-windows}}

\subsubsection{Implement a long\sphinxhyphen{}only BB(20, 2) strategy with Bitcoin}
\label{\detokenize{herron_02_practice:implement-a-long-only-bb-20-2-strategy-with-bitcoin}}
\sphinxAtStartPar
More on Bollinger Bands \sphinxhref{https://www.bollingerbands.com/bollinger-bands}{here} and \sphinxhref{https://www.bollingerbands.com/bollinger-band-rules}{here}.
In short, Bollinger Bands are bands around a trend, typically defined in terms of simple moving averages and volatilities.
Here, long\sphinxhyphen{}only BB(20, 2) implies we have upper and lower bands at 2 standard deviations above and below SMA(20):
\begin{enumerate}
\sphinxsetlistlabels{\arabic}{enumi}{enumii}{}{.}%
\item {} 
\sphinxAtStartPar
Buy when the closing price crosses LB(20) from below, where LB(20) is SMA(20) minus 2 sigma

\item {} 
\sphinxAtStartPar
Sell when the closing price crosses UB(20) from above, where UB(20) is SMA(20) plus 2 sigma

\item {} 
\sphinxAtStartPar
No short\sphinxhyphen{}selling

\end{enumerate}

\sphinxAtStartPar
The long\sphinxhyphen{}only BB(20, 2) is more difficult to implement than the long\sphinxhyphen{}only SMA(20) because we need to track buys and sells.
For example, if the closing price is between LB(20) and BB(20), we need to know if our last trade was a buy or a sell.
Further, if the closing price is below LB(20), we can still be long because we sell when the closing price crosses UB(20) from above.


\subsubsection{Implement a long\sphinxhyphen{}short RSI(14) strategy with Bitcoin}
\label{\detokenize{herron_02_practice:implement-a-long-short-rsi-14-strategy-with-bitcoin}}
\sphinxAtStartPar
From \sphinxhref{https://www.fidelity.com/learning-center/trading-investing/technical-analysis/technical-indicator-guide/rsi}{Fidelity}:
\begin{quote}

\sphinxAtStartPar
The Relative Strength Index (RSI), developed by J. Welles Wilder, is a momentum oscillator that measures the speed and change of price movements. The RSI oscillates between zero and 100. Traditionally the RSI is considered overbought when above 70 and oversold when below 30. Signals can be generated by looking for divergences and failure swings. RSI can also be used to identify the general trend.
\end{quote}

\sphinxAtStartPar
Here is the RSI formula: \$RSI(n) = 100 \sphinxhyphen{} \textbackslash{}frac\{100\}\{1 + RS(n)\}\$, where \$RS(n) = \textbackslash{}frac\{SMA(U, n)\}\{SMA(D, n)\}\$.
For “up days”, \$U = \textbackslash{}Delta Adj\textbackslash{} Close\$ and \$D = 0\$, and, for “down days”, \$U = 0\$ and \$D = \sphinxhyphen{} \textbackslash{}Delta Adj\textbackslash{} Close\$.
Therefore, \$U\$ and \$D\$ are always non\sphinxhyphen{}negative.
We can learn more about RSI \sphinxhref{https://en.wikipedia.org/wiki/Relative\_strength\_index}{here}.

\sphinxAtStartPar
We will implement a long\sphinxhyphen{}short RSI(14) as follows:
\begin{enumerate}
\sphinxsetlistlabels{\arabic}{enumi}{enumii}{}{.}%
\item {} 
\sphinxAtStartPar
Enter a long position when  the RSI crosses 30 from below, and exit the position when the RSI crosses 50 from below

\item {} 
\sphinxAtStartPar
Enter a short position when the RSI crosses 70 from above, and exit the position when the RSI crosses 50 from above

\end{enumerate}

\sphinxstepscope


\section{Herron Topic 2 \sphinxhyphen{} Practice (Monday 2:45 PM, Section 3)}
\label{\detokenize{herron_02_practice_03:herron-topic-2-practice-monday-2-45-pm-section-3}}\label{\detokenize{herron_02_practice_03::doc}}

\subsection{Announcements}
\label{\detokenize{herron_02_practice_03:announcements}}\begin{itemize}
\item {} 
\sphinxAtStartPar
I will finish grading projects this week/weekend

\item {} 
\sphinxAtStartPar
Quiz 5 due Friday at 11:59 PM
\begin{itemize}
\item {} 
\sphinxAtStartPar
A handful of students have submitted identical quizzes

\item {} 
\sphinxAtStartPar
Quizzes are individual efforts

\item {} 
\sphinxAtStartPar
Do not assume it is hard to for me to compare quiz and project submissions

\end{itemize}

\item {} 
\sphinxAtStartPar
DataCamp 20,000 XP due \sphinxstyleemphasis{next} Friday at 11:59 PM

\item {} 
\sphinxAtStartPar
Attendance and participation account for 5\% of your grade

\end{itemize}


\subsection{Practice}
\label{\detokenize{herron_02_practice_03:practice}}

\subsubsection{Implement the SMA(20) strategy with Bitcoin from the lecture notebook}
\label{\detokenize{herron_02_practice_03:implement-the-sma-20-strategy-with-bitcoin-from-the-lecture-notebook}}
\sphinxAtStartPar
Try to create the \sphinxcode{\sphinxupquote{btc}} data frame in one code cell with one assignment (i.e., one \sphinxcode{\sphinxupquote{=}}).

\sphinxAtStartPar
\sphinxstyleemphasis{\sphinxstylestrong{After class, I wrapped this chained operation into a function named \sphinxcode{\sphinxupquote{sma()}}.}}
This will make it easier to try different windows and separate the trading strategies in this notebook.

\begin{sphinxuseclass}{cell}\begin{sphinxVerbatimInput}

\begin{sphinxuseclass}{cell_input}
\begin{sphinxVerbatim}[commandchars=\\\{\}]
\PYG{k+kn}{import} \PYG{n+nn}{matplotlib}\PYG{n+nn}{.}\PYG{n+nn}{pyplot} \PYG{k}{as} \PYG{n+nn}{plt}
\PYG{k+kn}{import} \PYG{n+nn}{numpy} \PYG{k}{as} \PYG{n+nn}{np}
\PYG{k+kn}{import} \PYG{n+nn}{pandas} \PYG{k}{as} \PYG{n+nn}{pd}
\end{sphinxVerbatim}

\end{sphinxuseclass}\end{sphinxVerbatimInput}

\end{sphinxuseclass}
\begin{sphinxuseclass}{cell}\begin{sphinxVerbatimInput}

\begin{sphinxuseclass}{cell_input}
\begin{sphinxVerbatim}[commandchars=\\\{\}]
\PYG{o}{\PYGZpc{}}\PYG{k}{config} InlineBackend.figure\PYGZus{}format = \PYGZsq{}retina\PYGZsq{}
\PYG{o}{\PYGZpc{}}\PYG{k}{precision} 4
\PYG{n}{pd}\PYG{o}{.}\PYG{n}{options}\PYG{o}{.}\PYG{n}{display}\PYG{o}{.}\PYG{n}{float\PYGZus{}format} \PYG{o}{=} \PYG{l+s+s1}{\PYGZsq{}}\PYG{l+s+si}{\PYGZob{}:.4f\PYGZcb{}}\PYG{l+s+s1}{\PYGZsq{}}\PYG{o}{.}\PYG{n}{format}
\end{sphinxVerbatim}

\end{sphinxuseclass}\end{sphinxVerbatimInput}

\end{sphinxuseclass}
\begin{sphinxuseclass}{cell}\begin{sphinxVerbatimInput}

\begin{sphinxuseclass}{cell_input}
\begin{sphinxVerbatim}[commandchars=\\\{\}]
\PYG{k+kn}{import} \PYG{n+nn}{yfinance} \PYG{k}{as} \PYG{n+nn}{yf}
\end{sphinxVerbatim}

\end{sphinxuseclass}\end{sphinxVerbatimInput}

\end{sphinxuseclass}
\begin{sphinxuseclass}{cell}\begin{sphinxVerbatimInput}

\begin{sphinxuseclass}{cell_input}
\begin{sphinxVerbatim}[commandchars=\\\{\}]
\PYG{n}{btc} \PYG{o}{=} \PYG{p}{(}
    \PYG{n}{yf}\PYG{o}{.}\PYG{n}{download}\PYG{p}{(}\PYG{n}{tickers}\PYG{o}{=}\PYG{l+s+s1}{\PYGZsq{}}\PYG{l+s+s1}{BTC\PYGZhy{}USD}\PYG{l+s+s1}{\PYGZsq{}}\PYG{p}{,} \PYG{n}{progress}\PYG{o}{=}\PYG{k+kc}{False}\PYG{p}{)}
    \PYG{o}{.}\PYG{n}{assign}\PYG{p}{(}\PYG{n}{Date} \PYG{o}{=} \PYG{k}{lambda} \PYG{n}{x}\PYG{p}{:} \PYG{n}{x}\PYG{o}{.}\PYG{n}{index}\PYG{o}{.}\PYG{n}{tz\PYGZus{}localize}\PYG{p}{(}\PYG{k+kc}{None}\PYG{p}{)}\PYG{p}{)}
    \PYG{o}{.}\PYG{n}{set\PYGZus{}index}\PYG{p}{(}\PYG{l+s+s1}{\PYGZsq{}}\PYG{l+s+s1}{Date}\PYG{l+s+s1}{\PYGZsq{}}\PYG{p}{)}
    \PYG{o}{.}\PYG{n}{rename\PYGZus{}axis}\PYG{p}{(}\PYG{n}{columns}\PYG{o}{=}\PYG{l+s+s1}{\PYGZsq{}}\PYG{l+s+s1}{Variable}\PYG{l+s+s1}{\PYGZsq{}}\PYG{p}{)}
\PYG{p}{)}

\PYG{n}{btc}\PYG{o}{.}\PYG{n}{head}\PYG{p}{(}\PYG{p}{)}
\end{sphinxVerbatim}

\end{sphinxuseclass}\end{sphinxVerbatimInput}
\begin{sphinxVerbatimOutput}

\begin{sphinxuseclass}{cell_output}
\begin{sphinxVerbatim}[commandchars=\\\{\}]
Variable       Open     High      Low    Close  Adj Close    Volume
Date                                                               
2014\PYGZhy{}09\PYGZhy{}17 465.8640 468.1740 452.4220 457.3340   457.3340  21056800
2014\PYGZhy{}09\PYGZhy{}18 456.8600 456.8600 413.1040 424.4400   424.4400  34483200
2014\PYGZhy{}09\PYGZhy{}19 424.1030 427.8350 384.5320 394.7960   394.7960  37919700
2014\PYGZhy{}09\PYGZhy{}20 394.6730 423.2960 389.8830 408.9040   408.9040  36863600
2014\PYGZhy{}09\PYGZhy{}21 408.0850 412.4260 393.1810 398.8210   398.8210  26580100
\end{sphinxVerbatim}

\end{sphinxuseclass}\end{sphinxVerbatimOutput}

\end{sphinxuseclass}
\begin{sphinxuseclass}{cell}\begin{sphinxVerbatimInput}

\begin{sphinxuseclass}{cell_input}
\begin{sphinxVerbatim}[commandchars=\\\{\}]
\PYG{k}{def} \PYG{n+nf}{sma}\PYG{p}{(}\PYG{n}{df}\PYG{p}{,} \PYG{n}{n}\PYG{o}{=}\PYG{l+m+mi}{20}\PYG{p}{)}\PYG{p}{:}
    \PYG{k}{return} \PYG{p}{(}
        \PYG{n}{df}
        \PYG{o}{.}\PYG{n}{assign}\PYG{p}{(}
            \PYG{n}{Return} \PYG{o}{=} \PYG{k}{lambda} \PYG{n}{x}\PYG{p}{:} \PYG{n}{x}\PYG{p}{[}\PYG{l+s+s1}{\PYGZsq{}}\PYG{l+s+s1}{Adj Close}\PYG{l+s+s1}{\PYGZsq{}}\PYG{p}{]}\PYG{o}{.}\PYG{n}{pct\PYGZus{}change}\PYG{p}{(}\PYG{p}{)}\PYG{p}{,}
            \PYG{n}{SMA20} \PYG{o}{=} \PYG{k}{lambda} \PYG{n}{x}\PYG{p}{:} \PYG{n}{x}\PYG{p}{[}\PYG{l+s+s1}{\PYGZsq{}}\PYG{l+s+s1}{Adj Close}\PYG{l+s+s1}{\PYGZsq{}}\PYG{p}{]}\PYG{o}{.}\PYG{n}{rolling}\PYG{p}{(}\PYG{n}{n}\PYG{p}{)}\PYG{o}{.}\PYG{n}{mean}\PYG{p}{(}\PYG{p}{)}\PYG{p}{,}
            \PYG{n}{Position} \PYG{o}{=} \PYG{k}{lambda} \PYG{n}{x}\PYG{p}{:} \PYG{n}{np}\PYG{o}{.}\PYG{n}{select}\PYG{p}{(}
                \PYG{n}{condlist}\PYG{o}{=}\PYG{p}{[}
                    \PYG{n}{x}\PYG{p}{[}\PYG{l+s+s1}{\PYGZsq{}}\PYG{l+s+s1}{Adj Close}\PYG{l+s+s1}{\PYGZsq{}}\PYG{p}{]}\PYG{o}{.}\PYG{n}{shift}\PYG{p}{(}\PYG{p}{)} \PYG{o}{\PYGZgt{}} \PYG{n}{x}\PYG{p}{[}\PYG{l+s+s1}{\PYGZsq{}}\PYG{l+s+s1}{SMA20}\PYG{l+s+s1}{\PYGZsq{}}\PYG{p}{]}\PYG{o}{.}\PYG{n}{shift}\PYG{p}{(}\PYG{p}{)}\PYG{p}{,} 
                    \PYG{n}{x}\PYG{p}{[}\PYG{l+s+s1}{\PYGZsq{}}\PYG{l+s+s1}{Adj Close}\PYG{l+s+s1}{\PYGZsq{}}\PYG{p}{]}\PYG{o}{.}\PYG{n}{shift}\PYG{p}{(}\PYG{p}{)} \PYG{o}{\PYGZlt{}}\PYG{o}{=} \PYG{n}{x}\PYG{p}{[}\PYG{l+s+s1}{\PYGZsq{}}\PYG{l+s+s1}{SMA20}\PYG{l+s+s1}{\PYGZsq{}}\PYG{p}{]}\PYG{o}{.}\PYG{n}{shift}\PYG{p}{(}\PYG{p}{)}
                \PYG{p}{]}\PYG{p}{,}
                \PYG{n}{choicelist}\PYG{o}{=}\PYG{p}{[}
                    \PYG{l+m+mi}{1}\PYG{p}{,} 
                    \PYG{l+m+mi}{0}
                \PYG{p}{]}\PYG{p}{,}
                \PYG{n}{default}\PYG{o}{=}\PYG{n}{np}\PYG{o}{.}\PYG{n}{nan}
            \PYG{p}{)}\PYG{p}{,}
            \PYG{n}{Strategy} \PYG{o}{=} \PYG{k}{lambda} \PYG{n}{x}\PYG{p}{:} \PYG{n}{x}\PYG{p}{[}\PYG{l+s+s1}{\PYGZsq{}}\PYG{l+s+s1}{Position}\PYG{l+s+s1}{\PYGZsq{}}\PYG{p}{]} \PYG{o}{*} \PYG{n}{x}\PYG{p}{[}\PYG{l+s+s1}{\PYGZsq{}}\PYG{l+s+s1}{Return}\PYG{l+s+s1}{\PYGZsq{}}\PYG{p}{]}
        \PYG{p}{)}
    \PYG{p}{)}
\end{sphinxVerbatim}

\end{sphinxuseclass}\end{sphinxVerbatimInput}

\end{sphinxuseclass}
\begin{sphinxuseclass}{cell}\begin{sphinxVerbatimInput}

\begin{sphinxuseclass}{cell_input}
\begin{sphinxVerbatim}[commandchars=\\\{\}]
\PYG{n}{btc\PYGZus{}sma} \PYG{o}{=} \PYG{n}{btc}\PYG{o}{.}\PYG{n}{pipe}\PYG{p}{(}\PYG{n}{sma}\PYG{p}{,} \PYG{n}{n}\PYG{o}{=}\PYG{l+m+mi}{20}\PYG{p}{)}

\PYG{n}{btc\PYGZus{}sma}\PYG{o}{.}\PYG{n}{tail}\PYG{p}{(}\PYG{p}{)}
\end{sphinxVerbatim}

\end{sphinxuseclass}\end{sphinxVerbatimInput}
\begin{sphinxVerbatimOutput}

\begin{sphinxuseclass}{cell_output}
\begin{sphinxVerbatim}[commandchars=\\\{\}]
Variable         Open       High        Low      Close  Adj Close  \PYGZbs{}
Date                                                                
2023\PYGZhy{}03\PYGZhy{}11 20187.8770 20792.5254 20068.6602 20632.4102 20632.4102   
2023\PYGZhy{}03\PYGZhy{}12 20628.0293 22185.0312 20448.8066 22163.9492 22163.9492   
2023\PYGZhy{}03\PYGZhy{}13 22156.4062 24550.8379 21918.1992 24197.5332 24197.5332   
2023\PYGZhy{}03\PYGZhy{}14 24201.7656 26514.7168 24081.1836 24746.0742 24746.0742   
2023\PYGZhy{}03\PYGZhy{}15 24734.6523 25169.7305 24048.3516 24617.3047 24617.3047   

Variable         Volume  Return      SMA20  Position  Strategy  
Date                                                            
2023\PYGZhy{}03\PYGZhy{}11  30180288176  0.0221 22791.5311    0.0000    0.0000  
2023\PYGZhy{}03\PYGZhy{}12  29279035521  0.0742 22658.2711    0.0000    0.0000  
2023\PYGZhy{}03\PYGZhy{}13  49466362688  0.0918 22646.3301    0.0000    0.0000  
2023\PYGZhy{}03\PYGZhy{}14  54622230164  0.0227 22674.1916    1.0000    0.0227  
2023\PYGZhy{}03\PYGZhy{}15  44081737728 \PYGZhy{}0.0052 22707.6822    1.0000   \PYGZhy{}0.0052  
\end{sphinxVerbatim}

\end{sphinxuseclass}\end{sphinxVerbatimOutput}

\end{sphinxuseclass}
\begin{sphinxuseclass}{cell}\begin{sphinxVerbatimInput}

\begin{sphinxuseclass}{cell_input}
\begin{sphinxVerbatim}[commandchars=\\\{\}]
\PYG{n}{\PYGZus{}} \PYG{o}{=} \PYG{n}{btc\PYGZus{}sma}\PYG{p}{[}\PYG{p}{[}\PYG{l+s+s1}{\PYGZsq{}}\PYG{l+s+s1}{Return}\PYG{l+s+s1}{\PYGZsq{}}\PYG{p}{,} \PYG{l+s+s1}{\PYGZsq{}}\PYG{l+s+s1}{Strategy}\PYG{l+s+s1}{\PYGZsq{}}\PYG{p}{]}\PYG{p}{]}\PYG{o}{.}\PYG{n}{dropna}\PYG{p}{(}\PYG{p}{)}

\PYG{p}{(}
    \PYG{n}{\PYGZus{}}
    \PYG{o}{.}\PYG{n}{add}\PYG{p}{(}\PYG{l+m+mi}{1}\PYG{p}{)}
    \PYG{o}{.}\PYG{n}{cumprod}\PYG{p}{(}\PYG{p}{)}
    \PYG{o}{.}\PYG{n}{rename\PYGZus{}axis}\PYG{p}{(}\PYG{n}{columns}\PYG{o}{=}\PYG{l+s+s1}{\PYGZsq{}}\PYG{l+s+s1}{Strategy}\PYG{l+s+s1}{\PYGZsq{}}\PYG{p}{)}
    \PYG{o}{.}\PYG{n}{rename}\PYG{p}{(}\PYG{n}{columns}\PYG{o}{=}\PYG{p}{\PYGZob{}}\PYG{l+s+s1}{\PYGZsq{}}\PYG{l+s+s1}{Return}\PYG{l+s+s1}{\PYGZsq{}}\PYG{p}{:} \PYG{l+s+s1}{\PYGZsq{}}\PYG{l+s+s1}{Buy\PYGZhy{}And\PYGZhy{}Hold}\PYG{l+s+s1}{\PYGZsq{}}\PYG{p}{,} \PYG{l+s+s1}{\PYGZsq{}}\PYG{l+s+s1}{Strategy}\PYG{l+s+s1}{\PYGZsq{}}\PYG{p}{:} \PYG{l+s+s1}{\PYGZsq{}}\PYG{l+s+s1}{SMA(20)}\PYG{l+s+s1}{\PYGZsq{}}\PYG{p}{\PYGZcb{}}\PYG{p}{)}
    \PYG{o}{.}\PYG{n}{plot}\PYG{p}{(}\PYG{p}{)}
\PYG{p}{)}
\PYG{n}{plt}\PYG{o}{.}\PYG{n}{ylabel}\PYG{p}{(}\PYG{l+s+s1}{\PYGZsq{}}\PYG{l+s+s1}{Value (\PYGZdl{})}\PYG{l+s+s1}{\PYGZsq{}}\PYG{p}{)}
\PYG{n}{plt}\PYG{o}{.}\PYG{n}{title}\PYG{p}{(}\PYG{l+s+sa}{f}\PYG{l+s+s1}{\PYGZsq{}}\PYG{l+s+s1}{Value of \PYGZdl{}1 Invested at Close on }\PYG{l+s+si}{\PYGZob{}}\PYG{n}{\PYGZus{}}\PYG{o}{.}\PYG{n}{index}\PYG{p}{[}\PYG{l+m+mi}{0}\PYG{p}{]} \PYG{o}{\PYGZhy{}} \PYG{n}{pd}\PYG{o}{.}\PYG{n}{offsets}\PYG{o}{.}\PYG{n}{Day}\PYG{p}{(}\PYG{l+m+mi}{1}\PYG{p}{)}\PYG{l+s+si}{:}\PYG{l+s+s1}{\PYGZpc{}B \PYGZpc{}d, \PYGZpc{}Y}\PYG{l+s+si}{\PYGZcb{}}\PYG{l+s+s1}{\PYGZsq{}}\PYG{p}{)}
\PYG{n}{plt}\PYG{o}{.}\PYG{n}{show}\PYG{p}{(}\PYG{p}{)}
\end{sphinxVerbatim}

\end{sphinxuseclass}\end{sphinxVerbatimInput}
\begin{sphinxVerbatimOutput}

\begin{sphinxuseclass}{cell_output}
\noindent\sphinxincludegraphics{{0465375d524760e8f7e6f5cf74e7afd4fb8b2e32201f977ff1caf134a14848b0}.png}

\end{sphinxuseclass}\end{sphinxVerbatimOutput}

\end{sphinxuseclass}

\subsubsection{How does SMA(20) outperform buy\sphinxhyphen{}and\sphinxhyphen{}hold with this sample?}
\label{\detokenize{herron_02_practice_03:how-does-sma-20-outperform-buy-and-hold-with-this-sample}}
\sphinxAtStartPar
Consider the following:
\begin{enumerate}
\sphinxsetlistlabels{\arabic}{enumi}{enumii}{}{.}%
\item {} 
\sphinxAtStartPar
Does SMA(20) avoid the worst performing days? How many of the worst 20 days does SMA(20) avoid? Try the \sphinxcode{\sphinxupquote{.sort\_values()}} or \sphinxcode{\sphinxupquote{.nlargest()}} method.

\item {} 
\sphinxAtStartPar
Does SMA(20) preferentially avoid low\sphinxhyphen{}return days? Try to combine the \sphinxcode{\sphinxupquote{.groupby()}} method and \sphinxcode{\sphinxupquote{pd.qcut()}} function.

\item {} 
\sphinxAtStartPar
Does SMA(20) preferentially avoid high\sphinxhyphen{}volatility days? Try to combine the \sphinxcode{\sphinxupquote{.groupby()}} method and \sphinxcode{\sphinxupquote{pd.qcut()}} function.

\end{enumerate}

\sphinxAtStartPar
By chance, the SMA(20) strategy avoids all but three of the worst days.

\begin{sphinxuseclass}{cell}\begin{sphinxVerbatimInput}

\begin{sphinxuseclass}{cell_input}
\begin{sphinxVerbatim}[commandchars=\\\{\}]
\PYG{n}{btc\PYGZus{}sma}\PYG{o}{.}\PYG{n}{sort\PYGZus{}values}\PYG{p}{(}\PYG{l+s+s1}{\PYGZsq{}}\PYG{l+s+s1}{Return}\PYG{l+s+s1}{\PYGZsq{}}\PYG{p}{)}\PYG{p}{[}\PYG{p}{[}\PYG{l+s+s1}{\PYGZsq{}}\PYG{l+s+s1}{Position}\PYG{l+s+s1}{\PYGZsq{}}\PYG{p}{]}\PYG{p}{]}\PYG{o}{.}\PYG{n}{head}\PYG{p}{(}\PYG{l+m+mi}{20}\PYG{p}{)}\PYG{o}{.}\PYG{n}{value\PYGZus{}counts}\PYG{p}{(}\PYG{p}{)}
\end{sphinxVerbatim}

\end{sphinxuseclass}\end{sphinxVerbatimInput}
\begin{sphinxVerbatimOutput}

\begin{sphinxuseclass}{cell_output}
\begin{sphinxVerbatim}[commandchars=\\\{\}]
Position
0.0000      17
1.0000       3
dtype: int64
\end{sphinxVerbatim}

\end{sphinxuseclass}\end{sphinxVerbatimOutput}

\end{sphinxuseclass}
\sphinxAtStartPar
However, SMA(20) does not avoid the best days, again by chance.

\begin{sphinxuseclass}{cell}\begin{sphinxVerbatimInput}

\begin{sphinxuseclass}{cell_input}
\begin{sphinxVerbatim}[commandchars=\\\{\}]
\PYG{n}{btc\PYGZus{}sma}\PYG{o}{.}\PYG{n}{sort\PYGZus{}values}\PYG{p}{(}\PYG{l+s+s1}{\PYGZsq{}}\PYG{l+s+s1}{Return}\PYG{l+s+s1}{\PYGZsq{}}\PYG{p}{,} \PYG{n}{ascending}\PYG{o}{=}\PYG{k+kc}{False}\PYG{p}{)}\PYG{p}{[}\PYG{p}{[}\PYG{l+s+s1}{\PYGZsq{}}\PYG{l+s+s1}{Position}\PYG{l+s+s1}{\PYGZsq{}}\PYG{p}{]}\PYG{p}{]}\PYG{o}{.}\PYG{n}{head}\PYG{p}{(}\PYG{l+m+mi}{20}\PYG{p}{)}\PYG{o}{.}\PYG{n}{value\PYGZus{}counts}\PYG{p}{(}\PYG{p}{)}
\end{sphinxVerbatim}

\end{sphinxuseclass}\end{sphinxVerbatimInput}
\begin{sphinxVerbatimOutput}

\begin{sphinxuseclass}{cell_output}
\begin{sphinxVerbatim}[commandchars=\\\{\}]
Position
0.0000      10
1.0000      10
dtype: int64
\end{sphinxVerbatim}

\end{sphinxuseclass}\end{sphinxVerbatimOutput}

\end{sphinxuseclass}
\sphinxAtStartPar
The SMA(20) strategy has a slight edge in picking high\sphinxhyphen{}return days, again by chance.

\begin{sphinxuseclass}{cell}\begin{sphinxVerbatimInput}

\begin{sphinxuseclass}{cell_input}
\begin{sphinxVerbatim}[commandchars=\\\{\}]
\PYG{p}{(}
    \PYG{n}{btc\PYGZus{}sma}
    \PYG{o}{.}\PYG{n}{assign}\PYG{p}{(}\PYG{n}{q5\PYGZus{}return} \PYG{o}{=} \PYG{k}{lambda} \PYG{n}{x}\PYG{p}{:} \PYG{l+m+mi}{1} \PYG{o}{+} \PYG{n}{pd}\PYG{o}{.}\PYG{n}{qcut}\PYG{p}{(}\PYG{n}{x}\PYG{p}{[}\PYG{l+s+s1}{\PYGZsq{}}\PYG{l+s+s1}{Return}\PYG{l+s+s1}{\PYGZsq{}}\PYG{p}{]}\PYG{p}{,} \PYG{n}{q}\PYG{o}{=}\PYG{l+m+mi}{5}\PYG{p}{,} \PYG{n}{labels}\PYG{o}{=}\PYG{k+kc}{False}\PYG{p}{)}\PYG{p}{)}
    \PYG{o}{.}\PYG{n}{groupby}\PYG{p}{(}\PYG{l+s+s1}{\PYGZsq{}}\PYG{l+s+s1}{q5\PYGZus{}return}\PYG{l+s+s1}{\PYGZsq{}}\PYG{p}{)}
    \PYG{p}{[}\PYG{l+s+s1}{\PYGZsq{}}\PYG{l+s+s1}{Position}\PYG{l+s+s1}{\PYGZsq{}}\PYG{p}{]}
    \PYG{o}{.}\PYG{n}{mean}\PYG{p}{(}\PYG{p}{)}
    \PYG{o}{.}\PYG{n}{plot}\PYG{p}{(}\PYG{n}{kind}\PYG{o}{=}\PYG{l+s+s1}{\PYGZsq{}}\PYG{l+s+s1}{bar}\PYG{l+s+s1}{\PYGZsq{}}\PYG{p}{)}
\PYG{p}{)}

\PYG{n}{plt}\PYG{o}{.}\PYG{n}{xticks}\PYG{p}{(}\PYG{n}{rotation}\PYG{o}{=}\PYG{l+m+mi}{0}\PYG{p}{)}
\PYG{n}{plt}\PYG{o}{.}\PYG{n}{xlabel}\PYG{p}{(}\PYG{l+s+s1}{\PYGZsq{}}\PYG{l+s+s1}{Return Bin (1 is Lowest, 5 is Highest)}\PYG{l+s+s1}{\PYGZsq{}}\PYG{p}{)}
\PYG{n}{plt}\PYG{o}{.}\PYG{n}{ylabel}\PYG{p}{(}\PYG{l+s+s1}{\PYGZsq{}}\PYG{l+s+s1}{Fraction of Days Strategy is Long Bitcoin}\PYG{l+s+s1}{\PYGZsq{}}\PYG{p}{)}
\PYG{n}{plt}\PYG{o}{.}\PYG{n}{title}\PYG{p}{(}\PYG{l+s+s1}{\PYGZsq{}}\PYG{l+s+s1}{Mean Position by Return Bin}\PYG{l+s+s1}{\PYGZsq{}}\PYG{p}{)}
\PYG{n}{plt}\PYG{o}{.}\PYG{n}{show}\PYG{p}{(}\PYG{p}{)}
\end{sphinxVerbatim}

\end{sphinxuseclass}\end{sphinxVerbatimInput}
\begin{sphinxVerbatimOutput}

\begin{sphinxuseclass}{cell_output}
\noindent\sphinxincludegraphics{{77fcce6036e279a25a7c191e860fc4caa4bd9ab43c234b7935bfcdb356dabc6a}.png}

\end{sphinxuseclass}\end{sphinxVerbatimOutput}

\end{sphinxuseclass}
\sphinxAtStartPar
However, the SMA(20) \sphinxstyleemphasis{does} avoid the high volatility days that create \sphinxhref{https://www.kitces.com/blog/volatility-drag-variance-drain-mean-arithmetic-vs-geometric-average-investment-returns/}{volatility drag}.

\begin{sphinxuseclass}{cell}\begin{sphinxVerbatimInput}

\begin{sphinxuseclass}{cell_input}
\begin{sphinxVerbatim}[commandchars=\\\{\}]
\PYG{p}{(}
    \PYG{n}{btc\PYGZus{}sma}
    \PYG{o}{.}\PYG{n}{assign}\PYG{p}{(}
        \PYG{n}{Volatility} \PYG{o}{=} \PYG{k}{lambda} \PYG{n}{x}\PYG{p}{:} \PYG{n}{x}\PYG{p}{[}\PYG{l+s+s1}{\PYGZsq{}}\PYG{l+s+s1}{Return}\PYG{l+s+s1}{\PYGZsq{}}\PYG{p}{]}\PYG{o}{.}\PYG{n}{rolling}\PYG{p}{(}\PYG{l+m+mi}{63}\PYG{p}{)}\PYG{o}{.}\PYG{n}{std}\PYG{p}{(}\PYG{p}{)}\PYG{p}{,}
        \PYG{n}{q5\PYGZus{}volatility} \PYG{o}{=} \PYG{k}{lambda} \PYG{n}{x}\PYG{p}{:} \PYG{l+m+mi}{1} \PYG{o}{+} \PYG{n}{pd}\PYG{o}{.}\PYG{n}{qcut}\PYG{p}{(}\PYG{n}{x}\PYG{p}{[}\PYG{l+s+s1}{\PYGZsq{}}\PYG{l+s+s1}{Volatility}\PYG{l+s+s1}{\PYGZsq{}}\PYG{p}{]}\PYG{p}{,} \PYG{n}{q}\PYG{o}{=}\PYG{l+m+mi}{5}\PYG{p}{,} \PYG{n}{labels}\PYG{o}{=}\PYG{k+kc}{False}\PYG{p}{)}
    \PYG{p}{)}
    \PYG{o}{.}\PYG{n}{groupby}\PYG{p}{(}\PYG{l+s+s1}{\PYGZsq{}}\PYG{l+s+s1}{q5\PYGZus{}volatility}\PYG{l+s+s1}{\PYGZsq{}}\PYG{p}{)}
    \PYG{p}{[}\PYG{l+s+s1}{\PYGZsq{}}\PYG{l+s+s1}{Position}\PYG{l+s+s1}{\PYGZsq{}}\PYG{p}{]}
    \PYG{o}{.}\PYG{n}{mean}\PYG{p}{(}\PYG{p}{)}
    \PYG{o}{.}\PYG{n}{plot}\PYG{p}{(}\PYG{n}{kind}\PYG{o}{=}\PYG{l+s+s1}{\PYGZsq{}}\PYG{l+s+s1}{bar}\PYG{l+s+s1}{\PYGZsq{}}\PYG{p}{)}
\PYG{p}{)}

\PYG{n}{plt}\PYG{o}{.}\PYG{n}{xticks}\PYG{p}{(}\PYG{n}{rotation}\PYG{o}{=}\PYG{l+m+mi}{0}\PYG{p}{)}
\PYG{n}{plt}\PYG{o}{.}\PYG{n}{xlabel}\PYG{p}{(}\PYG{l+s+s1}{\PYGZsq{}}\PYG{l+s+s1}{63\PYGZhy{}Day Rolling Volatility Bin (1 is Lowest, 5 is Highest)}\PYG{l+s+s1}{\PYGZsq{}}\PYG{p}{)}
\PYG{n}{plt}\PYG{o}{.}\PYG{n}{ylabel}\PYG{p}{(}\PYG{l+s+s1}{\PYGZsq{}}\PYG{l+s+s1}{Fraction of Days Strategy is Long Bitcoin}\PYG{l+s+s1}{\PYGZsq{}}\PYG{p}{)}
\PYG{n}{plt}\PYG{o}{.}\PYG{n}{title}\PYG{p}{(}\PYG{l+s+s1}{\PYGZsq{}}\PYG{l+s+s1}{Mean Position by 63\PYGZhy{}Day Rolling Volatility Bin}\PYG{l+s+s1}{\PYGZsq{}}\PYG{p}{)}
\PYG{n}{plt}\PYG{o}{.}\PYG{n}{show}\PYG{p}{(}\PYG{p}{)}
\end{sphinxVerbatim}

\end{sphinxuseclass}\end{sphinxVerbatimInput}
\begin{sphinxVerbatimOutput}

\begin{sphinxuseclass}{cell_output}
\noindent\sphinxincludegraphics{{63059ddf0e219dc18f519507934fef116e57ea00861bca604dcdc2281f99d135}.png}

\end{sphinxuseclass}\end{sphinxVerbatimOutput}

\end{sphinxuseclass}
\sphinxAtStartPar
Recall that \$Arith\textbackslash{} Mean \textbackslash{}approx Geom\textbackslash{} Mean + \textbackslash{}frac\{\textbackslash{}sigma\textasciicircum{}2\}\{2\}\$, so avoiding high volatility (high variance) days, reduced the drag on the  cumulative returns that intermediate\sphinxhyphen{}term and long\sphinxhyphen{}term investors care about!

\begin{sphinxuseclass}{cell}\begin{sphinxVerbatimInput}

\begin{sphinxuseclass}{cell_input}
\begin{sphinxVerbatim}[commandchars=\\\{\}]
\PYG{p}{(}
    \PYG{n}{btc\PYGZus{}sma}
    \PYG{o}{.}\PYG{n}{groupby}\PYG{p}{(}\PYG{l+s+s1}{\PYGZsq{}}\PYG{l+s+s1}{Position}\PYG{l+s+s1}{\PYGZsq{}}\PYG{p}{)}
    \PYG{p}{[}\PYG{l+s+s1}{\PYGZsq{}}\PYG{l+s+s1}{Return}\PYG{l+s+s1}{\PYGZsq{}}\PYG{p}{]}
    \PYG{o}{.}\PYG{n}{agg}\PYG{p}{(}\PYG{p}{[}\PYG{l+s+s1}{\PYGZsq{}}\PYG{l+s+s1}{std}\PYG{l+s+s1}{\PYGZsq{}}\PYG{p}{,} \PYG{l+s+s1}{\PYGZsq{}}\PYG{l+s+s1}{mean}\PYG{l+s+s1}{\PYGZsq{}}\PYG{p}{,} \PYG{k}{lambda} \PYG{n}{x}\PYG{p}{:} \PYG{p}{(}\PYG{l+m+mi}{1} \PYG{o}{+} \PYG{n}{x}\PYG{p}{)}\PYG{o}{.}\PYG{n}{prod}\PYG{p}{(}\PYG{p}{)}\PYG{o}{*}\PYG{o}{*}\PYG{p}{(}\PYG{l+m+mi}{1} \PYG{o}{/} \PYG{n}{x}\PYG{o}{.}\PYG{n}{count}\PYG{p}{(}\PYG{p}{)}\PYG{p}{)} \PYG{o}{\PYGZhy{}} \PYG{l+m+mi}{1}\PYG{p}{]}\PYG{p}{)}
    \PYG{o}{.}\PYG{n}{mul}\PYG{p}{(}\PYG{l+m+mi}{100}\PYG{p}{)}
    \PYG{o}{.}\PYG{n}{rename}\PYG{p}{(}\PYG{n}{columns}\PYG{o}{=}\PYG{p}{\PYGZob{}}\PYG{l+s+s1}{\PYGZsq{}}\PYG{l+s+s1}{std}\PYG{l+s+s1}{\PYGZsq{}}\PYG{p}{:} \PYG{l+s+s1}{\PYGZsq{}}\PYG{l+s+s1}{Volatility}\PYG{l+s+s1}{\PYGZsq{}}\PYG{p}{,} \PYG{l+s+s1}{\PYGZsq{}}\PYG{l+s+s1}{mean}\PYG{l+s+s1}{\PYGZsq{}}\PYG{p}{:} \PYG{l+s+s1}{\PYGZsq{}}\PYG{l+s+s1}{Arith Mean}\PYG{l+s+s1}{\PYGZsq{}}\PYG{p}{,} \PYG{l+s+s1}{\PYGZsq{}}\PYG{l+s+s1}{\PYGZlt{}lambda\PYGZus{}0\PYGZgt{}}\PYG{l+s+s1}{\PYGZsq{}}\PYG{p}{:} \PYG{l+s+s1}{\PYGZsq{}}\PYG{l+s+s1}{Geom Mean}\PYG{l+s+s1}{\PYGZsq{}}\PYG{p}{\PYGZcb{}}\PYG{p}{)}
\PYG{p}{)}
\end{sphinxVerbatim}

\end{sphinxuseclass}\end{sphinxVerbatimInput}
\begin{sphinxVerbatimOutput}

\begin{sphinxuseclass}{cell_output}
\begin{sphinxVerbatim}[commandchars=\\\{\}]
          Volatility  Arith Mean  Geom Mean
Position                                   
0.0000        4.1518      0.0433    \PYGZhy{}0.0452
1.0000        3.4901      0.3553     0.2952
\end{sphinxVerbatim}

\end{sphinxuseclass}\end{sphinxVerbatimOutput}

\end{sphinxuseclass}

\subsubsection{Implement the SMA(20) strategy with the market factor from French}
\label{\detokenize{herron_02_practice_03:implement-the-sma-20-strategy-with-the-market-factor-from-french}}
\sphinxAtStartPar
We need to impute a market price before we calculate SMA(20).

\begin{sphinxuseclass}{cell}\begin{sphinxVerbatimInput}

\begin{sphinxuseclass}{cell_input}
\begin{sphinxVerbatim}[commandchars=\\\{\}]
\PYG{k+kn}{import} \PYG{n+nn}{pandas\PYGZus{}datareader} \PYG{k}{as} \PYG{n+nn}{pdr}
\PYG{k+kn}{import} \PYG{n+nn}{requests\PYGZus{}cache}
\PYG{n}{session} \PYG{o}{=} \PYG{n}{requests\PYGZus{}cache}\PYG{o}{.}\PYG{n}{CachedSession}\PYG{p}{(}\PYG{p}{)}
\end{sphinxVerbatim}

\end{sphinxuseclass}\end{sphinxVerbatimInput}

\end{sphinxuseclass}
\begin{sphinxuseclass}{cell}\begin{sphinxVerbatimInput}

\begin{sphinxuseclass}{cell_input}
\begin{sphinxVerbatim}[commandchars=\\\{\}]
\PYG{n}{ff} \PYG{o}{=} \PYG{p}{(}
    \PYG{n}{pdr}\PYG{o}{.}\PYG{n}{DataReader}\PYG{p}{(}
        \PYG{n}{name}\PYG{o}{=}\PYG{l+s+s1}{\PYGZsq{}}\PYG{l+s+s1}{F\PYGZhy{}F\PYGZus{}Research\PYGZus{}Data\PYGZus{}Factors\PYGZus{}daily}\PYG{l+s+s1}{\PYGZsq{}}\PYG{p}{,}
        \PYG{n}{data\PYGZus{}source}\PYG{o}{=}\PYG{l+s+s1}{\PYGZsq{}}\PYG{l+s+s1}{famafrench}\PYG{l+s+s1}{\PYGZsq{}}\PYG{p}{,}
        \PYG{n}{start}\PYG{o}{=}\PYG{l+s+s1}{\PYGZsq{}}\PYG{l+s+s1}{1900}\PYG{l+s+s1}{\PYGZsq{}}\PYG{p}{,}
        \PYG{n}{session}\PYG{o}{=}\PYG{n}{session}
    \PYG{p}{)}
    \PYG{p}{[}\PYG{l+m+mi}{0}\PYG{p}{]}
    \PYG{o}{.}\PYG{n}{div}\PYG{p}{(}\PYG{l+m+mi}{100}\PYG{p}{)}
    \PYG{o}{.}\PYG{n}{assign}\PYG{p}{(}
        \PYG{n}{Mkt} \PYG{o}{=} \PYG{k}{lambda} \PYG{n}{x}\PYG{p}{:} \PYG{n}{x}\PYG{p}{[}\PYG{l+s+s1}{\PYGZsq{}}\PYG{l+s+s1}{Mkt\PYGZhy{}RF}\PYG{l+s+s1}{\PYGZsq{}}\PYG{p}{]} \PYG{o}{+} \PYG{n}{x}\PYG{p}{[}\PYG{l+s+s1}{\PYGZsq{}}\PYG{l+s+s1}{RF}\PYG{l+s+s1}{\PYGZsq{}}\PYG{p}{]}\PYG{p}{,}
        \PYG{n}{Price} \PYG{o}{=} \PYG{k}{lambda} \PYG{n}{x}\PYG{p}{:} \PYG{n}{x}\PYG{p}{[}\PYG{l+s+s1}{\PYGZsq{}}\PYG{l+s+s1}{Mkt}\PYG{l+s+s1}{\PYGZsq{}}\PYG{p}{]}\PYG{o}{.}\PYG{n}{add}\PYG{p}{(}\PYG{l+m+mi}{1}\PYG{p}{)}\PYG{o}{.}\PYG{n}{cumprod}\PYG{p}{(}\PYG{p}{)}
    \PYG{p}{)}
\PYG{p}{)}
\end{sphinxVerbatim}

\end{sphinxuseclass}\end{sphinxVerbatimInput}

\end{sphinxuseclass}
\begin{sphinxuseclass}{cell}\begin{sphinxVerbatimInput}

\begin{sphinxuseclass}{cell_input}
\begin{sphinxVerbatim}[commandchars=\\\{\}]
\PYG{n}{ff\PYGZus{}sma} \PYG{o}{=} \PYG{p}{(}
    \PYG{n}{ff}
    \PYG{o}{.}\PYG{n}{rename}\PYG{p}{(}\PYG{n}{columns}\PYG{o}{=}\PYG{p}{\PYGZob{}}\PYG{l+s+s1}{\PYGZsq{}}\PYG{l+s+s1}{Price}\PYG{l+s+s1}{\PYGZsq{}}\PYG{p}{:} \PYG{l+s+s1}{\PYGZsq{}}\PYG{l+s+s1}{Adj Close}\PYG{l+s+s1}{\PYGZsq{}}\PYG{p}{\PYGZcb{}}\PYG{p}{)}
    \PYG{o}{.}\PYG{n}{pipe}\PYG{p}{(}\PYG{n}{sma}\PYG{p}{,} \PYG{n}{n}\PYG{o}{=}\PYG{l+m+mi}{20}\PYG{p}{)}
\PYG{p}{)}

\PYG{n}{ff\PYGZus{}sma}\PYG{o}{.}\PYG{n}{tail}\PYG{p}{(}\PYG{p}{)}
\end{sphinxVerbatim}

\end{sphinxuseclass}\end{sphinxVerbatimInput}
\begin{sphinxVerbatimOutput}

\begin{sphinxuseclass}{cell_output}
\begin{sphinxVerbatim}[commandchars=\\\{\}]
            Mkt\PYGZhy{}RF     SMB     HML     RF     Mkt  Adj Close  Return  \PYGZbs{}
Date                                                                   
2022\PYGZhy{}12\PYGZhy{}23  0.0051 \PYGZhy{}0.0060  0.0115 0.0002  0.0053  9266.3996  0.0053   
2022\PYGZhy{}12\PYGZhy{}27 \PYGZhy{}0.0051 \PYGZhy{}0.0073  0.0142 0.0002 \PYGZhy{}0.0049  9220.6236 \PYGZhy{}0.0049   
2022\PYGZhy{}12\PYGZhy{}28 \PYGZhy{}0.0123 \PYGZhy{}0.0025 \PYGZhy{}0.0029 0.0002 \PYGZhy{}0.0121  9108.6852 \PYGZhy{}0.0121   
2022\PYGZhy{}12\PYGZhy{}29  0.0187  0.0127 \PYGZhy{}0.0107 0.0002  0.0189  9280.4750  0.0189   
2022\PYGZhy{}12\PYGZhy{}30 \PYGZhy{}0.0022  0.0011 \PYGZhy{}0.0003 0.0002 \PYGZhy{}0.0020  9261.5429 \PYGZhy{}0.0020   

               SMA20  Position  Strategy  
Date                                      
2022\PYGZhy{}12\PYGZhy{}23 9515.5406    0.0000    0.0000  
2022\PYGZhy{}12\PYGZhy{}27 9497.8127    0.0000   \PYGZhy{}0.0000  
2022\PYGZhy{}12\PYGZhy{}28 9475.2827    0.0000   \PYGZhy{}0.0000  
2022\PYGZhy{}12\PYGZhy{}29 9446.3627    0.0000    0.0000  
2022\PYGZhy{}12\PYGZhy{}30 9416.5159    0.0000   \PYGZhy{}0.0000  
\end{sphinxVerbatim}

\end{sphinxuseclass}\end{sphinxVerbatimOutput}

\end{sphinxuseclass}
\begin{sphinxuseclass}{cell}\begin{sphinxVerbatimInput}

\begin{sphinxuseclass}{cell_input}
\begin{sphinxVerbatim}[commandchars=\\\{\}]
\PYG{n}{\PYGZus{}} \PYG{o}{=} \PYG{n}{ff\PYGZus{}sma}\PYG{p}{[}\PYG{p}{[}\PYG{l+s+s1}{\PYGZsq{}}\PYG{l+s+s1}{Return}\PYG{l+s+s1}{\PYGZsq{}}\PYG{p}{,} \PYG{l+s+s1}{\PYGZsq{}}\PYG{l+s+s1}{Strategy}\PYG{l+s+s1}{\PYGZsq{}}\PYG{p}{]}\PYG{p}{]}\PYG{o}{.}\PYG{n}{dropna}\PYG{p}{(}\PYG{p}{)}

\PYG{p}{(}
    \PYG{n}{\PYGZus{}}
    \PYG{o}{.}\PYG{n}{add}\PYG{p}{(}\PYG{l+m+mi}{1}\PYG{p}{)}
    \PYG{o}{.}\PYG{n}{cumprod}\PYG{p}{(}\PYG{p}{)}
    \PYG{o}{.}\PYG{n}{rename\PYGZus{}axis}\PYG{p}{(}\PYG{n}{columns}\PYG{o}{=}\PYG{l+s+s1}{\PYGZsq{}}\PYG{l+s+s1}{Strategy}\PYG{l+s+s1}{\PYGZsq{}}\PYG{p}{)}
    \PYG{o}{.}\PYG{n}{rename}\PYG{p}{(}\PYG{n}{columns}\PYG{o}{=}\PYG{p}{\PYGZob{}}\PYG{l+s+s1}{\PYGZsq{}}\PYG{l+s+s1}{Return}\PYG{l+s+s1}{\PYGZsq{}}\PYG{p}{:} \PYG{l+s+s1}{\PYGZsq{}}\PYG{l+s+s1}{Buy\PYGZhy{}And\PYGZhy{}Hold}\PYG{l+s+s1}{\PYGZsq{}}\PYG{p}{,} \PYG{l+s+s1}{\PYGZsq{}}\PYG{l+s+s1}{Strategy}\PYG{l+s+s1}{\PYGZsq{}}\PYG{p}{:} \PYG{l+s+s1}{\PYGZsq{}}\PYG{l+s+s1}{SMA(20)}\PYG{l+s+s1}{\PYGZsq{}}\PYG{p}{\PYGZcb{}}\PYG{p}{)}
    \PYG{o}{.}\PYG{n}{plot}\PYG{p}{(}\PYG{p}{)}
\PYG{p}{)}
\PYG{n}{plt}\PYG{o}{.}\PYG{n}{ylabel}\PYG{p}{(}\PYG{l+s+s1}{\PYGZsq{}}\PYG{l+s+s1}{Value (\PYGZdl{})}\PYG{l+s+s1}{\PYGZsq{}}\PYG{p}{)}
\PYG{n}{plt}\PYG{o}{.}\PYG{n}{title}\PYG{p}{(}\PYG{l+s+sa}{f}\PYG{l+s+s1}{\PYGZsq{}}\PYG{l+s+s1}{Value of \PYGZdl{}1 Invested in Market at Close on }\PYG{l+s+si}{\PYGZob{}}\PYG{n}{\PYGZus{}}\PYG{o}{.}\PYG{n}{index}\PYG{p}{[}\PYG{l+m+mi}{0}\PYG{p}{]} \PYG{o}{\PYGZhy{}} \PYG{n}{pd}\PYG{o}{.}\PYG{n}{offsets}\PYG{o}{.}\PYG{n}{Day}\PYG{p}{(}\PYG{l+m+mi}{1}\PYG{p}{)}\PYG{l+s+si}{:}\PYG{l+s+s1}{\PYGZpc{}B \PYGZpc{}d, \PYGZpc{}Y}\PYG{l+s+si}{\PYGZcb{}}\PYG{l+s+s1}{\PYGZsq{}}\PYG{p}{)}
\PYG{n}{plt}\PYG{o}{.}\PYG{n}{show}\PYG{p}{(}\PYG{p}{)}
\end{sphinxVerbatim}

\end{sphinxuseclass}\end{sphinxVerbatimInput}
\begin{sphinxVerbatimOutput}

\begin{sphinxuseclass}{cell_output}
\noindent\sphinxincludegraphics{{131e3565948329ffbfb65b3360199ffe069c47f7c99b79b2edd04f4debb4d99e}.png}

\end{sphinxuseclass}\end{sphinxVerbatimOutput}

\end{sphinxuseclass}

\subsubsection{How often does SMA(20) outperform buy\sphinxhyphen{}and\sphinxhyphen{}hold with 10\sphinxhyphen{}year rolling windows?}
\label{\detokenize{herron_02_practice_03:how-often-does-sma-20-outperform-buy-and-hold-with-10-year-rolling-windows}}
\begin{sphinxuseclass}{cell}\begin{sphinxVerbatimInput}

\begin{sphinxuseclass}{cell_input}
\begin{sphinxVerbatim}[commandchars=\\\{\}]
\PYG{p}{(}
    \PYG{n}{ff\PYGZus{}sma}
    \PYG{p}{[}\PYG{p}{[}\PYG{l+s+s1}{\PYGZsq{}}\PYG{l+s+s1}{Return}\PYG{l+s+s1}{\PYGZsq{}}\PYG{p}{,} \PYG{l+s+s1}{\PYGZsq{}}\PYG{l+s+s1}{Strategy}\PYG{l+s+s1}{\PYGZsq{}}\PYG{p}{]}\PYG{p}{]}
    \PYG{o}{.}\PYG{n}{rolling}\PYG{p}{(}\PYG{l+m+mi}{10} \PYG{o}{*} \PYG{l+m+mi}{252}\PYG{p}{)}
    \PYG{o}{.}\PYG{n}{apply}\PYG{p}{(}\PYG{k}{lambda} \PYG{n}{x}\PYG{p}{:} \PYG{p}{(}\PYG{l+m+mi}{1} \PYG{o}{+} \PYG{n}{x}\PYG{p}{)}\PYG{o}{.}\PYG{n}{prod}\PYG{p}{(}\PYG{p}{)}\PYG{p}{)}
    \PYG{o}{.}\PYG{n}{rename\PYGZus{}axis}\PYG{p}{(}\PYG{n}{columns}\PYG{o}{=}\PYG{l+s+s1}{\PYGZsq{}}\PYG{l+s+s1}{Strategy}\PYG{l+s+s1}{\PYGZsq{}}\PYG{p}{)}
    \PYG{o}{.}\PYG{n}{rename}\PYG{p}{(}\PYG{n}{columns}\PYG{o}{=}\PYG{p}{\PYGZob{}}\PYG{l+s+s1}{\PYGZsq{}}\PYG{l+s+s1}{Return}\PYG{l+s+s1}{\PYGZsq{}}\PYG{p}{:} \PYG{l+s+s1}{\PYGZsq{}}\PYG{l+s+s1}{Buy\PYGZhy{}And\PYGZhy{}Hold}\PYG{l+s+s1}{\PYGZsq{}}\PYG{p}{,} \PYG{l+s+s1}{\PYGZsq{}}\PYG{l+s+s1}{Strategy}\PYG{l+s+s1}{\PYGZsq{}}\PYG{p}{:} \PYG{l+s+s1}{\PYGZsq{}}\PYG{l+s+s1}{SMA(20)}\PYG{l+s+s1}{\PYGZsq{}}\PYG{p}{\PYGZcb{}}\PYG{p}{)}
    \PYG{o}{.}\PYG{n}{plot}\PYG{p}{(}\PYG{p}{)}
\PYG{p}{)}
\PYG{n}{plt}\PYG{o}{.}\PYG{n}{ylabel}\PYG{p}{(}\PYG{l+s+s1}{\PYGZsq{}}\PYG{l+s+s1}{Value (\PYGZdl{})}\PYG{l+s+s1}{\PYGZsq{}}\PYG{p}{)}
\PYG{n}{plt}\PYG{o}{.}\PYG{n}{title}\PYG{p}{(}\PYG{l+s+sa}{f}\PYG{l+s+s1}{\PYGZsq{}}\PYG{l+s+s1}{Value of \PYGZdl{}1 Investments for Rolling 10\PYGZhy{}Year Holding Periods }\PYG{l+s+s1}{\PYGZsq{}}\PYG{p}{)}
\PYG{n}{plt}\PYG{o}{.}\PYG{n}{show}\PYG{p}{(}\PYG{p}{)}
\end{sphinxVerbatim}

\end{sphinxuseclass}\end{sphinxVerbatimInput}
\begin{sphinxVerbatimOutput}

\begin{sphinxuseclass}{cell_output}
\noindent\sphinxincludegraphics{{9f8dc1a1b39e0f46b1d4fda08ef1b2a8f7ae3e2ae45f32a52dcfe96994401f08}.png}

\end{sphinxuseclass}\end{sphinxVerbatimOutput}

\end{sphinxuseclass}
\sphinxAtStartPar
In the previous example, SMA(20) looks amazing!
But over many shorter holding periods, we see the two are comparable.
This is largely because the SMA(20) strategy \sphinxstyleemphasis{by pure chance} avoids big market draw downs!

\begin{sphinxuseclass}{cell}\begin{sphinxVerbatimInput}

\begin{sphinxuseclass}{cell_input}
\begin{sphinxVerbatim}[commandchars=\\\{\}]
\PYG{n}{ff\PYGZus{}sma}\PYG{o}{.}\PYG{n}{sort\PYGZus{}values}\PYG{p}{(}\PYG{l+s+s1}{\PYGZsq{}}\PYG{l+s+s1}{Return}\PYG{l+s+s1}{\PYGZsq{}}\PYG{p}{)}\PYG{p}{[}\PYG{p}{[}\PYG{l+s+s1}{\PYGZsq{}}\PYG{l+s+s1}{Position}\PYG{l+s+s1}{\PYGZsq{}}\PYG{p}{,} \PYG{l+s+s1}{\PYGZsq{}}\PYG{l+s+s1}{Return}\PYG{l+s+s1}{\PYGZsq{}}\PYG{p}{,} \PYG{l+s+s1}{\PYGZsq{}}\PYG{l+s+s1}{Strategy}\PYG{l+s+s1}{\PYGZsq{}}\PYG{p}{]}\PYG{p}{]}\PYG{o}{.}\PYG{n}{head}\PYG{p}{(}\PYG{l+m+mi}{10}\PYG{p}{)}
\end{sphinxVerbatim}

\end{sphinxuseclass}\end{sphinxVerbatimInput}
\begin{sphinxVerbatimOutput}

\begin{sphinxuseclass}{cell_output}
\begin{sphinxVerbatim}[commandchars=\\\{\}]
            Position  Return  Strategy
Date                                  
1987\PYGZhy{}10\PYGZhy{}19    0.0000 \PYGZhy{}0.1741   \PYGZhy{}0.0000
2020\PYGZhy{}03\PYGZhy{}16    0.0000 \PYGZhy{}0.1199   \PYGZhy{}0.0000
1929\PYGZhy{}10\PYGZhy{}29    0.0000 \PYGZhy{}0.1199   \PYGZhy{}0.0000
1929\PYGZhy{}10\PYGZhy{}28    0.0000 \PYGZhy{}0.1127   \PYGZhy{}0.0000
1929\PYGZhy{}11\PYGZhy{}06    0.0000 \PYGZhy{}0.0973   \PYGZhy{}0.0000
2020\PYGZhy{}03\PYGZhy{}12    0.0000 \PYGZhy{}0.0962   \PYGZhy{}0.0000
1933\PYGZhy{}07\PYGZhy{}21    0.0000 \PYGZhy{}0.0921   \PYGZhy{}0.0000
2008\PYGZhy{}12\PYGZhy{}01    1.0000 \PYGZhy{}0.0895   \PYGZhy{}0.0895
2008\PYGZhy{}10\PYGZhy{}15    0.0000 \PYGZhy{}0.0878   \PYGZhy{}0.0000
1933\PYGZhy{}07\PYGZhy{}20    1.0000 \PYGZhy{}0.0849   \PYGZhy{}0.0849
\end{sphinxVerbatim}

\end{sphinxuseclass}\end{sphinxVerbatimOutput}

\end{sphinxuseclass}
\sphinxAtStartPar
SMA(20) also avoids the up days.
However, for this sample, missing the extreme down days helps more than missing the extreme updays hurts.

\begin{sphinxuseclass}{cell}\begin{sphinxVerbatimInput}

\begin{sphinxuseclass}{cell_input}
\begin{sphinxVerbatim}[commandchars=\\\{\}]
\PYG{n}{ff\PYGZus{}sma}\PYG{o}{.}\PYG{n}{sort\PYGZus{}values}\PYG{p}{(}\PYG{l+s+s1}{\PYGZsq{}}\PYG{l+s+s1}{Return}\PYG{l+s+s1}{\PYGZsq{}}\PYG{p}{,} \PYG{n}{ascending}\PYG{o}{=}\PYG{k+kc}{False}\PYG{p}{)}\PYG{p}{[}\PYG{p}{[}\PYG{l+s+s1}{\PYGZsq{}}\PYG{l+s+s1}{Position}\PYG{l+s+s1}{\PYGZsq{}}\PYG{p}{,} \PYG{l+s+s1}{\PYGZsq{}}\PYG{l+s+s1}{Return}\PYG{l+s+s1}{\PYGZsq{}}\PYG{p}{,} \PYG{l+s+s1}{\PYGZsq{}}\PYG{l+s+s1}{Strategy}\PYG{l+s+s1}{\PYGZsq{}}\PYG{p}{]}\PYG{p}{]}\PYG{o}{.}\PYG{n}{head}\PYG{p}{(}\PYG{l+m+mi}{10}\PYG{p}{)}
\end{sphinxVerbatim}

\end{sphinxuseclass}\end{sphinxVerbatimInput}
\begin{sphinxVerbatimOutput}

\begin{sphinxuseclass}{cell_output}
\begin{sphinxVerbatim}[commandchars=\\\{\}]
            Position  Return  Strategy
Date                                  
1933\PYGZhy{}03\PYGZhy{}15    0.0000  0.1576    0.0000
1929\PYGZhy{}10\PYGZhy{}30    0.0000  0.1218    0.0000
2008\PYGZhy{}10\PYGZhy{}13    0.0000  0.1135    0.0000
1931\PYGZhy{}10\PYGZhy{}06    0.0000  0.1116    0.0000
1932\PYGZhy{}09\PYGZhy{}21    0.0000  0.1096    0.0000
2008\PYGZhy{}10\PYGZhy{}28    0.0000  0.0977    0.0000
2020\PYGZhy{}03\PYGZhy{}24    0.0000  0.0935    0.0000
2020\PYGZhy{}03\PYGZhy{}13    0.0000  0.0897    0.0000
1939\PYGZhy{}09\PYGZhy{}05    1.0000  0.0879    0.0879
1937\PYGZhy{}10\PYGZhy{}20    0.0000  0.0867    0.0000
\end{sphinxVerbatim}

\end{sphinxuseclass}\end{sphinxVerbatimOutput}

\end{sphinxuseclass}
\sphinxAtStartPar
We can also think about this problem by decade.
If we want to get proper calendar decades (instead of 10\sphinxhyphen{}year periods that start in 1926), we combine \sphinxcode{\sphinxupquote{.groupby()}} with an anonymous function that converts the date\sphinxhyphen{}time index to a proper calendar decade.
Again, we see that SMA(20) and buy\sphinxhyphen{}and\sphinxhyphen{}hold trade wins, but SMA(20) wins bigs in the 1930s!

\begin{sphinxuseclass}{cell}\begin{sphinxVerbatimInput}

\begin{sphinxuseclass}{cell_input}
\begin{sphinxVerbatim}[commandchars=\\\{\}]
\PYG{p}{(}
    \PYG{n}{ff\PYGZus{}sma}
    \PYG{p}{[}\PYG{p}{[}\PYG{l+s+s1}{\PYGZsq{}}\PYG{l+s+s1}{Return}\PYG{l+s+s1}{\PYGZsq{}}\PYG{p}{,} \PYG{l+s+s1}{\PYGZsq{}}\PYG{l+s+s1}{Strategy}\PYG{l+s+s1}{\PYGZsq{}}\PYG{p}{]}\PYG{p}{]}
    \PYG{o}{.}\PYG{n}{groupby}\PYG{p}{(}\PYG{k}{lambda} \PYG{n}{x}\PYG{p}{:} \PYG{l+s+sa}{f}\PYG{l+s+s1}{\PYGZsq{}}\PYG{l+s+si}{\PYGZob{}}\PYG{p}{(}\PYG{n}{x}\PYG{o}{.}\PYG{n}{year} \PYG{o}{/}\PYG{o}{/} \PYG{l+m+mi}{10}\PYG{p}{)} \PYG{o}{*} \PYG{l+m+mi}{10}\PYG{l+s+si}{\PYGZcb{}}\PYG{l+s+s1}{s}\PYG{l+s+s1}{\PYGZsq{}}\PYG{p}{)}
    \PYG{o}{.}\PYG{n}{apply}\PYG{p}{(}\PYG{k}{lambda} \PYG{n}{x}\PYG{p}{:} \PYG{p}{(}\PYG{l+m+mi}{1} \PYG{o}{+} \PYG{n}{x}\PYG{p}{)}\PYG{o}{.}\PYG{n}{prod}\PYG{p}{(}\PYG{p}{)}\PYG{p}{)}
    \PYG{o}{.}\PYG{n}{rename\PYGZus{}axis}\PYG{p}{(}\PYG{n}{index}\PYG{o}{=}\PYG{l+s+s1}{\PYGZsq{}}\PYG{l+s+s1}{Decade}\PYG{l+s+s1}{\PYGZsq{}}\PYG{p}{,} \PYG{n}{columns}\PYG{o}{=}\PYG{l+s+s1}{\PYGZsq{}}\PYG{l+s+s1}{Strategy}\PYG{l+s+s1}{\PYGZsq{}}\PYG{p}{)}
    \PYG{o}{.}\PYG{n}{rename}\PYG{p}{(}\PYG{n}{columns}\PYG{o}{=}\PYG{p}{\PYGZob{}}\PYG{l+s+s1}{\PYGZsq{}}\PYG{l+s+s1}{Return}\PYG{l+s+s1}{\PYGZsq{}}\PYG{p}{:} \PYG{l+s+s1}{\PYGZsq{}}\PYG{l+s+s1}{Buy\PYGZhy{}And\PYGZhy{}Hold}\PYG{l+s+s1}{\PYGZsq{}}\PYG{p}{,} \PYG{l+s+s1}{\PYGZsq{}}\PYG{l+s+s1}{Strategy}\PYG{l+s+s1}{\PYGZsq{}}\PYG{p}{:} \PYG{l+s+s1}{\PYGZsq{}}\PYG{l+s+s1}{SMA(20)}\PYG{l+s+s1}{\PYGZsq{}}\PYG{p}{\PYGZcb{}}\PYG{p}{)}
    \PYG{o}{.}\PYG{n}{plot}\PYG{p}{(}\PYG{n}{kind}\PYG{o}{=}\PYG{l+s+s1}{\PYGZsq{}}\PYG{l+s+s1}{bar}\PYG{l+s+s1}{\PYGZsq{}}\PYG{p}{)}
\PYG{p}{)}
\PYG{n}{plt}\PYG{o}{.}\PYG{n}{xticks}\PYG{p}{(}\PYG{n}{rotation}\PYG{o}{=}\PYG{l+m+mi}{0}\PYG{p}{)}
\PYG{n}{plt}\PYG{o}{.}\PYG{n}{ylabel}\PYG{p}{(}\PYG{l+s+s1}{\PYGZsq{}}\PYG{l+s+s1}{Value (\PYGZdl{})}\PYG{l+s+s1}{\PYGZsq{}}\PYG{p}{)}
\PYG{n}{plt}\PYG{o}{.}\PYG{n}{title}\PYG{p}{(}\PYG{l+s+sa}{f}\PYG{l+s+s1}{\PYGZsq{}}\PYG{l+s+s1}{Value of \PYGZdl{}1 Investments Add End of 10\PYGZhy{}Year Holding Periods }\PYG{l+s+s1}{\PYGZsq{}}\PYG{p}{)}
\PYG{n}{plt}\PYG{o}{.}\PYG{n}{show}\PYG{p}{(}\PYG{p}{)}
\end{sphinxVerbatim}

\end{sphinxuseclass}\end{sphinxVerbatimInput}
\begin{sphinxVerbatimOutput}

\begin{sphinxuseclass}{cell_output}
\noindent\sphinxincludegraphics{{262e708684b1109abef3e7de77a33f6ae09019d5b276baf05ba5dcbda2c924e9}.png}

\end{sphinxuseclass}\end{sphinxVerbatimOutput}

\end{sphinxuseclass}
\sphinxAtStartPar
In fact, buy\sphinxhyphen{}and\sphinxhyphen{}hold outperforms SMA(20) is we start in 1950.

\begin{sphinxuseclass}{cell}\begin{sphinxVerbatimInput}

\begin{sphinxuseclass}{cell_input}
\begin{sphinxVerbatim}[commandchars=\\\{\}]
\PYG{n}{\PYGZus{}} \PYG{o}{=} \PYG{n}{ff\PYGZus{}sma}\PYG{o}{.}\PYG{n}{loc}\PYG{p}{[}\PYG{l+s+s1}{\PYGZsq{}}\PYG{l+s+s1}{1950}\PYG{l+s+s1}{\PYGZsq{}}\PYG{p}{:}\PYG{p}{,} \PYG{p}{[}\PYG{l+s+s1}{\PYGZsq{}}\PYG{l+s+s1}{Return}\PYG{l+s+s1}{\PYGZsq{}}\PYG{p}{,} \PYG{l+s+s1}{\PYGZsq{}}\PYG{l+s+s1}{Strategy}\PYG{l+s+s1}{\PYGZsq{}}\PYG{p}{]}\PYG{p}{]}\PYG{o}{.}\PYG{n}{dropna}\PYG{p}{(}\PYG{p}{)}

\PYG{p}{(}
    \PYG{n}{\PYGZus{}}
    \PYG{o}{.}\PYG{n}{add}\PYG{p}{(}\PYG{l+m+mi}{1}\PYG{p}{)}
    \PYG{o}{.}\PYG{n}{cumprod}\PYG{p}{(}\PYG{p}{)}
    \PYG{o}{.}\PYG{n}{rename\PYGZus{}axis}\PYG{p}{(}\PYG{n}{columns}\PYG{o}{=}\PYG{l+s+s1}{\PYGZsq{}}\PYG{l+s+s1}{Strategy}\PYG{l+s+s1}{\PYGZsq{}}\PYG{p}{)}
    \PYG{o}{.}\PYG{n}{rename}\PYG{p}{(}\PYG{n}{columns}\PYG{o}{=}\PYG{p}{\PYGZob{}}\PYG{l+s+s1}{\PYGZsq{}}\PYG{l+s+s1}{Return}\PYG{l+s+s1}{\PYGZsq{}}\PYG{p}{:} \PYG{l+s+s1}{\PYGZsq{}}\PYG{l+s+s1}{Buy\PYGZhy{}And\PYGZhy{}Hold}\PYG{l+s+s1}{\PYGZsq{}}\PYG{p}{,} \PYG{l+s+s1}{\PYGZsq{}}\PYG{l+s+s1}{Strategy}\PYG{l+s+s1}{\PYGZsq{}}\PYG{p}{:} \PYG{l+s+s1}{\PYGZsq{}}\PYG{l+s+s1}{SMA(20)}\PYG{l+s+s1}{\PYGZsq{}}\PYG{p}{\PYGZcb{}}\PYG{p}{)}
    \PYG{o}{.}\PYG{n}{plot}\PYG{p}{(}\PYG{p}{)}
\PYG{p}{)}
\PYG{n}{plt}\PYG{o}{.}\PYG{n}{ylabel}\PYG{p}{(}\PYG{l+s+s1}{\PYGZsq{}}\PYG{l+s+s1}{Value (\PYGZdl{})}\PYG{l+s+s1}{\PYGZsq{}}\PYG{p}{)}
\PYG{n}{plt}\PYG{o}{.}\PYG{n}{title}\PYG{p}{(}\PYG{l+s+sa}{f}\PYG{l+s+s1}{\PYGZsq{}}\PYG{l+s+s1}{Value of \PYGZdl{}1 Invested in Market at Close on }\PYG{l+s+si}{\PYGZob{}}\PYG{n}{\PYGZus{}}\PYG{o}{.}\PYG{n}{index}\PYG{p}{[}\PYG{l+m+mi}{0}\PYG{p}{]} \PYG{o}{\PYGZhy{}} \PYG{n}{pd}\PYG{o}{.}\PYG{n}{offsets}\PYG{o}{.}\PYG{n}{Day}\PYG{p}{(}\PYG{l+m+mi}{1}\PYG{p}{)}\PYG{l+s+si}{:}\PYG{l+s+s1}{\PYGZpc{}B \PYGZpc{}d, \PYGZpc{}Y}\PYG{l+s+si}{\PYGZcb{}}\PYG{l+s+s1}{\PYGZsq{}}\PYG{p}{)}
\PYG{n}{plt}\PYG{o}{.}\PYG{n}{show}\PYG{p}{(}\PYG{p}{)}
\end{sphinxVerbatim}

\end{sphinxuseclass}\end{sphinxVerbatimInput}
\begin{sphinxVerbatimOutput}

\begin{sphinxuseclass}{cell_output}
\noindent\sphinxincludegraphics{{de32325d6c5dda85fff71b12216bc00bdfbe78f65cb56ebc4cf40d6abd840d13}.png}

\end{sphinxuseclass}\end{sphinxVerbatimOutput}

\end{sphinxuseclass}

\subsubsection{Implement a long\sphinxhyphen{}only BB(20, 2) strategy with Bitcoin}
\label{\detokenize{herron_02_practice_03:implement-a-long-only-bb-20-2-strategy-with-bitcoin}}
\sphinxAtStartPar
More on Bollinger Bands \sphinxhref{https://www.bollingerbands.com/bollinger-bands}{here} and \sphinxhref{https://www.bollingerbands.com/bollinger-band-rules}{here}.
In short, Bollinger Bands are bands around a trend, typically defined in terms of simple moving averages and volatilities.
Here, long\sphinxhyphen{}only BB(20, 2) implies we have upper and lower bands at 2 standard deviations above and below SMA(20):
\begin{enumerate}
\sphinxsetlistlabels{\arabic}{enumi}{enumii}{}{.}%
\item {} 
\sphinxAtStartPar
Buy when the closing price crosses LB(20) from below, where LB(20) is SMA(20) minus 2 sigma

\item {} 
\sphinxAtStartPar
Sell when the closing price crosses UB(20) from above, where UB(20) is SMA(20) plus 2 sigma

\item {} 
\sphinxAtStartPar
No short\sphinxhyphen{}selling

\end{enumerate}

\sphinxAtStartPar
The long\sphinxhyphen{}only BB(20, 2) is more difficult to implement than the long\sphinxhyphen{}only SMA(20) because we need to track buys and sells.
For example, if the closing price is between LB(20) and BB(20), we need to know if our last trade was a buy or a sell.
Further, if the closing price is below LB(20), we can still be long because we sell when the closing price crosses UB(20) from above.

\sphinxAtStartPar
\sphinxstyleemphasis{\sphinxstylestrong{After class, I wrapped this chained operation into a function named \sphinxcode{\sphinxupquote{bb()}}.}}
This will make it easier to try different windows and separate the trading strategies in this notebook.

\begin{sphinxuseclass}{cell}\begin{sphinxVerbatimInput}

\begin{sphinxuseclass}{cell_input}
\begin{sphinxVerbatim}[commandchars=\\\{\}]
\PYG{k}{def} \PYG{n+nf}{bb}\PYG{p}{(}\PYG{n}{df}\PYG{p}{,} \PYG{n}{n}\PYG{o}{=}\PYG{l+m+mi}{20}\PYG{p}{,} \PYG{n}{m}\PYG{o}{=}\PYG{l+m+mi}{2}\PYG{p}{)}\PYG{p}{:}
    \PYG{k}{return} \PYG{p}{(}
        \PYG{n}{df}
        \PYG{o}{.}\PYG{n}{assign}\PYG{p}{(}
            \PYG{n}{Date} \PYG{o}{=} \PYG{k}{lambda} \PYG{n}{x}\PYG{p}{:} \PYG{n}{x}\PYG{o}{.}\PYG{n}{index}\PYG{o}{.}\PYG{n}{tz\PYGZus{}localize}\PYG{p}{(}\PYG{k+kc}{None}\PYG{p}{)}\PYG{p}{,}
            \PYG{n}{Return} \PYG{o}{=} \PYG{k}{lambda} \PYG{n}{x}\PYG{p}{:} \PYG{n}{x}\PYG{p}{[}\PYG{l+s+s1}{\PYGZsq{}}\PYG{l+s+s1}{Adj Close}\PYG{l+s+s1}{\PYGZsq{}}\PYG{p}{]}\PYG{o}{.}\PYG{n}{pct\PYGZus{}change}\PYG{p}{(}\PYG{p}{)}\PYG{p}{,}
            \PYG{n}{SMA} \PYG{o}{=} \PYG{k}{lambda} \PYG{n}{x}\PYG{p}{:} \PYG{n}{x}\PYG{p}{[}\PYG{l+s+s1}{\PYGZsq{}}\PYG{l+s+s1}{Adj Close}\PYG{l+s+s1}{\PYGZsq{}}\PYG{p}{]}\PYG{o}{.}\PYG{n}{rolling}\PYG{p}{(}\PYG{n}{n}\PYG{p}{)}\PYG{o}{.}\PYG{n}{mean}\PYG{p}{(}\PYG{p}{)}\PYG{p}{,}
            \PYG{n}{SMV} \PYG{o}{=} \PYG{k}{lambda} \PYG{n}{x}\PYG{p}{:} \PYG{n}{x}\PYG{p}{[}\PYG{l+s+s1}{\PYGZsq{}}\PYG{l+s+s1}{Adj Close}\PYG{l+s+s1}{\PYGZsq{}}\PYG{p}{]}\PYG{o}{.}\PYG{n}{rolling}\PYG{p}{(}\PYG{n}{n}\PYG{p}{)}\PYG{o}{.}\PYG{n}{std}\PYG{p}{(}\PYG{p}{)}\PYG{p}{,}
            \PYG{n}{LB} \PYG{o}{=} \PYG{k}{lambda} \PYG{n}{x}\PYG{p}{:} \PYG{n}{x}\PYG{p}{[}\PYG{l+s+s1}{\PYGZsq{}}\PYG{l+s+s1}{SMA}\PYG{l+s+s1}{\PYGZsq{}}\PYG{p}{]} \PYG{o}{\PYGZhy{}} \PYG{n}{m}\PYG{o}{*}\PYG{n}{x}\PYG{p}{[}\PYG{l+s+s1}{\PYGZsq{}}\PYG{l+s+s1}{SMV}\PYG{l+s+s1}{\PYGZsq{}}\PYG{p}{]}\PYG{p}{,}
            \PYG{n}{UB} \PYG{o}{=} \PYG{k}{lambda} \PYG{n}{x}\PYG{p}{:} \PYG{n}{x}\PYG{p}{[}\PYG{l+s+s1}{\PYGZsq{}}\PYG{l+s+s1}{SMA}\PYG{l+s+s1}{\PYGZsq{}}\PYG{p}{]} \PYG{o}{+} \PYG{n}{m}\PYG{o}{*}\PYG{n}{x}\PYG{p}{[}\PYG{l+s+s1}{\PYGZsq{}}\PYG{l+s+s1}{SMV}\PYG{l+s+s1}{\PYGZsq{}}\PYG{p}{]}\PYG{p}{,}
            \PYG{n}{Position\PYGZus{}with\PYGZus{}nan} \PYG{o}{=} \PYG{k}{lambda} \PYG{n}{x}\PYG{p}{:} \PYG{n}{np}\PYG{o}{.}\PYG{n}{select}\PYG{p}{(}
                \PYG{n}{condlist}\PYG{o}{=}\PYG{p}{[}
                    \PYG{p}{(}\PYG{n}{x}\PYG{p}{[}\PYG{l+s+s1}{\PYGZsq{}}\PYG{l+s+s1}{Adj Close}\PYG{l+s+s1}{\PYGZsq{}}\PYG{p}{]}\PYG{o}{.}\PYG{n}{shift}\PYG{p}{(}\PYG{l+m+mi}{1}\PYG{p}{)} \PYG{o}{\PYGZgt{}}\PYG{o}{=} \PYG{n}{x}\PYG{p}{[}\PYG{l+s+s1}{\PYGZsq{}}\PYG{l+s+s1}{LB}\PYG{l+s+s1}{\PYGZsq{}}\PYG{p}{]}\PYG{o}{.}\PYG{n}{shift}\PYG{p}{(}\PYG{l+m+mi}{1}\PYG{p}{)}\PYG{p}{)} \PYG{o}{\PYGZam{}} \PYG{p}{(}\PYG{n}{x}\PYG{p}{[}\PYG{l+s+s1}{\PYGZsq{}}\PYG{l+s+s1}{Adj Close}\PYG{l+s+s1}{\PYGZsq{}}\PYG{p}{]}\PYG{o}{.}\PYG{n}{shift}\PYG{p}{(}\PYG{l+m+mi}{2}\PYG{p}{)} \PYG{o}{\PYGZlt{}} \PYG{n}{x}\PYG{p}{[}\PYG{l+s+s1}{\PYGZsq{}}\PYG{l+s+s1}{LB}\PYG{l+s+s1}{\PYGZsq{}}\PYG{p}{]}\PYG{o}{.}\PYG{n}{shift}\PYG{p}{(}\PYG{l+m+mi}{2}\PYG{p}{)}\PYG{p}{)}\PYG{p}{,} 
                    \PYG{p}{(}\PYG{n}{x}\PYG{p}{[}\PYG{l+s+s1}{\PYGZsq{}}\PYG{l+s+s1}{Adj Close}\PYG{l+s+s1}{\PYGZsq{}}\PYG{p}{]}\PYG{o}{.}\PYG{n}{shift}\PYG{p}{(}\PYG{l+m+mi}{1}\PYG{p}{)} \PYG{o}{\PYGZlt{}}\PYG{o}{=} \PYG{n}{x}\PYG{p}{[}\PYG{l+s+s1}{\PYGZsq{}}\PYG{l+s+s1}{UB}\PYG{l+s+s1}{\PYGZsq{}}\PYG{p}{]}\PYG{o}{.}\PYG{n}{shift}\PYG{p}{(}\PYG{l+m+mi}{1}\PYG{p}{)}\PYG{p}{)} \PYG{o}{\PYGZam{}} \PYG{p}{(}\PYG{n}{x}\PYG{p}{[}\PYG{l+s+s1}{\PYGZsq{}}\PYG{l+s+s1}{Adj Close}\PYG{l+s+s1}{\PYGZsq{}}\PYG{p}{]}\PYG{o}{.}\PYG{n}{shift}\PYG{p}{(}\PYG{l+m+mi}{2}\PYG{p}{)} \PYG{o}{\PYGZgt{}} \PYG{n}{x}\PYG{p}{[}\PYG{l+s+s1}{\PYGZsq{}}\PYG{l+s+s1}{UB}\PYG{l+s+s1}{\PYGZsq{}}\PYG{p}{]}\PYG{o}{.}\PYG{n}{shift}\PYG{p}{(}\PYG{l+m+mi}{2}\PYG{p}{)}\PYG{p}{)}\PYG{p}{,}
                \PYG{p}{]}\PYG{p}{,}
                \PYG{n}{choicelist}\PYG{o}{=}\PYG{p}{[}
                    \PYG{l+m+mi}{1}\PYG{p}{,} 
                    \PYG{l+m+mi}{0}
                \PYG{p}{]}\PYG{p}{,}
                \PYG{n}{default}\PYG{o}{=}\PYG{n}{np}\PYG{o}{.}\PYG{n}{nan}
            \PYG{p}{)}\PYG{p}{,}
            \PYG{n}{Position} \PYG{o}{=} \PYG{k}{lambda} \PYG{n}{x}\PYG{p}{:} \PYG{n}{x}\PYG{p}{[}\PYG{l+s+s1}{\PYGZsq{}}\PYG{l+s+s1}{Position\PYGZus{}with\PYGZus{}nan}\PYG{l+s+s1}{\PYGZsq{}}\PYG{p}{]}\PYG{o}{.}\PYG{n}{fillna}\PYG{p}{(}\PYG{n}{method}\PYG{o}{=}\PYG{l+s+s1}{\PYGZsq{}}\PYG{l+s+s1}{ffill}\PYG{l+s+s1}{\PYGZsq{}}\PYG{p}{)}\PYG{p}{,}
            \PYG{n}{Strategy} \PYG{o}{=} \PYG{k}{lambda} \PYG{n}{x}\PYG{p}{:} \PYG{n}{x}\PYG{p}{[}\PYG{l+s+s1}{\PYGZsq{}}\PYG{l+s+s1}{Position}\PYG{l+s+s1}{\PYGZsq{}}\PYG{p}{]} \PYG{o}{*} \PYG{n}{x}\PYG{p}{[}\PYG{l+s+s1}{\PYGZsq{}}\PYG{l+s+s1}{Return}\PYG{l+s+s1}{\PYGZsq{}}\PYG{p}{]}
        \PYG{p}{)}
    \PYG{p}{)}
\end{sphinxVerbatim}

\end{sphinxuseclass}\end{sphinxVerbatimInput}

\end{sphinxuseclass}
\begin{sphinxuseclass}{cell}\begin{sphinxVerbatimInput}

\begin{sphinxuseclass}{cell_input}
\begin{sphinxVerbatim}[commandchars=\\\{\}]
\PYG{n}{btc\PYGZus{}bb} \PYG{o}{=} \PYG{n}{btc}\PYG{o}{.}\PYG{n}{pipe}\PYG{p}{(}\PYG{n}{bb}\PYG{p}{)}

\PYG{n}{btc\PYGZus{}bb}\PYG{o}{.}\PYG{n}{tail}\PYG{p}{(}\PYG{p}{)}
\end{sphinxVerbatim}

\end{sphinxuseclass}\end{sphinxVerbatimInput}
\begin{sphinxVerbatimOutput}

\begin{sphinxuseclass}{cell_output}
\begin{sphinxVerbatim}[commandchars=\\\{\}]
Variable         Open       High        Low      Close  Adj Close  \PYGZbs{}
Date                                                                
2023\PYGZhy{}03\PYGZhy{}11 20187.8770 20792.5254 20068.6602 20632.4102 20632.4102   
2023\PYGZhy{}03\PYGZhy{}12 20628.0293 22185.0312 20448.8066 22163.9492 22163.9492   
2023\PYGZhy{}03\PYGZhy{}13 22156.4062 24550.8379 21918.1992 24197.5332 24197.5332   
2023\PYGZhy{}03\PYGZhy{}14 24201.7656 26514.7168 24081.1836 24746.0742 24746.0742   
2023\PYGZhy{}03\PYGZhy{}15 24734.6523 25169.7305 24048.3516 24617.3047 24617.3047   

Variable         Volume       Date  Return        SMA       SMV         LB  \PYGZbs{}
Date                                                                         
2023\PYGZhy{}03\PYGZhy{}11  30180288176 2023\PYGZhy{}03\PYGZhy{}11  0.0221 22791.5311 1305.1707 20181.1896   
2023\PYGZhy{}03\PYGZhy{}12  29279035521 2023\PYGZhy{}03\PYGZhy{}12  0.0742 22658.2711 1219.4207 20219.4297   
2023\PYGZhy{}03\PYGZhy{}13  49466362688 2023\PYGZhy{}03\PYGZhy{}13  0.0918 22646.3301 1202.1395 20242.0510   
2023\PYGZhy{}03\PYGZhy{}14  54622230164 2023\PYGZhy{}03\PYGZhy{}14  0.0227 22674.1916 1245.4486 20183.2944   
2023\PYGZhy{}03\PYGZhy{}15  44081737728 2023\PYGZhy{}03\PYGZhy{}15 \PYGZhy{}0.0052 22707.6822 1289.7095 20128.2632   

Variable           UB  Position\PYGZus{}with\PYGZus{}nan  Position  Strategy  
Date                                                          
2023\PYGZhy{}03\PYGZhy{}11 25401.8725                NaN    1.0000    0.0221  
2023\PYGZhy{}03\PYGZhy{}12 25097.1125             1.0000    1.0000    0.0742  
2023\PYGZhy{}03\PYGZhy{}13 25050.6091                NaN    1.0000    0.0918  
2023\PYGZhy{}03\PYGZhy{}14 25165.0888                NaN    1.0000    0.0227  
2023\PYGZhy{}03\PYGZhy{}15 25287.1013                NaN    1.0000   \PYGZhy{}0.0052  
\end{sphinxVerbatim}

\end{sphinxuseclass}\end{sphinxVerbatimOutput}

\end{sphinxuseclass}
\begin{sphinxuseclass}{cell}\begin{sphinxVerbatimInput}

\begin{sphinxuseclass}{cell_input}
\begin{sphinxVerbatim}[commandchars=\\\{\}]
\PYG{n}{\PYGZus{}} \PYG{o}{=} \PYG{n}{btc\PYGZus{}bb}\PYG{p}{[}\PYG{p}{[}\PYG{l+s+s1}{\PYGZsq{}}\PYG{l+s+s1}{Return}\PYG{l+s+s1}{\PYGZsq{}}\PYG{p}{,} \PYG{l+s+s1}{\PYGZsq{}}\PYG{l+s+s1}{Strategy}\PYG{l+s+s1}{\PYGZsq{}}\PYG{p}{]}\PYG{p}{]}\PYG{o}{.}\PYG{n}{dropna}\PYG{p}{(}\PYG{p}{)}

\PYG{p}{(}
    \PYG{n}{\PYGZus{}}
    \PYG{o}{.}\PYG{n}{add}\PYG{p}{(}\PYG{l+m+mi}{1}\PYG{p}{)}
    \PYG{o}{.}\PYG{n}{cumprod}\PYG{p}{(}\PYG{p}{)}
    \PYG{o}{.}\PYG{n}{rename\PYGZus{}axis}\PYG{p}{(}\PYG{n}{columns}\PYG{o}{=}\PYG{l+s+s1}{\PYGZsq{}}\PYG{l+s+s1}{Strategy}\PYG{l+s+s1}{\PYGZsq{}}\PYG{p}{)}
    \PYG{o}{.}\PYG{n}{rename}\PYG{p}{(}\PYG{n}{columns}\PYG{o}{=}\PYG{p}{\PYGZob{}}\PYG{l+s+s1}{\PYGZsq{}}\PYG{l+s+s1}{Return}\PYG{l+s+s1}{\PYGZsq{}}\PYG{p}{:} \PYG{l+s+s1}{\PYGZsq{}}\PYG{l+s+s1}{Buy\PYGZhy{}And\PYGZhy{}Hold}\PYG{l+s+s1}{\PYGZsq{}}\PYG{p}{,} \PYG{l+s+s1}{\PYGZsq{}}\PYG{l+s+s1}{Strategy}\PYG{l+s+s1}{\PYGZsq{}}\PYG{p}{:} \PYG{l+s+s1}{\PYGZsq{}}\PYG{l+s+s1}{BB(20, 2)}\PYG{l+s+s1}{\PYGZsq{}}\PYG{p}{\PYGZcb{}}\PYG{p}{)}
    \PYG{o}{.}\PYG{n}{plot}\PYG{p}{(}\PYG{p}{)}
\PYG{p}{)}
\PYG{n}{plt}\PYG{o}{.}\PYG{n}{ylabel}\PYG{p}{(}\PYG{l+s+s1}{\PYGZsq{}}\PYG{l+s+s1}{Value (\PYGZdl{})}\PYG{l+s+s1}{\PYGZsq{}}\PYG{p}{)}
\PYG{n}{plt}\PYG{o}{.}\PYG{n}{title}\PYG{p}{(}\PYG{l+s+sa}{f}\PYG{l+s+s1}{\PYGZsq{}}\PYG{l+s+s1}{Value of \PYGZdl{}1 Invested at Close on }\PYG{l+s+si}{\PYGZob{}}\PYG{n}{\PYGZus{}}\PYG{o}{.}\PYG{n}{index}\PYG{p}{[}\PYG{l+m+mi}{0}\PYG{p}{]} \PYG{o}{\PYGZhy{}} \PYG{n}{pd}\PYG{o}{.}\PYG{n}{offsets}\PYG{o}{.}\PYG{n}{Day}\PYG{p}{(}\PYG{l+m+mi}{1}\PYG{p}{)}\PYG{l+s+si}{:}\PYG{l+s+s1}{\PYGZpc{}B \PYGZpc{}d, \PYGZpc{}Y}\PYG{l+s+si}{\PYGZcb{}}\PYG{l+s+s1}{\PYGZsq{}}\PYG{p}{)}
\PYG{n}{plt}\PYG{o}{.}\PYG{n}{show}\PYG{p}{(}\PYG{p}{)}
\end{sphinxVerbatim}

\end{sphinxuseclass}\end{sphinxVerbatimInput}
\begin{sphinxVerbatimOutput}

\begin{sphinxuseclass}{cell_output}
\noindent\sphinxincludegraphics{{1712840168ce637c11b63aaf384533ed9deb985f91ac37aa6e4cf240669a1e8e}.png}

\end{sphinxuseclass}\end{sphinxVerbatimOutput}

\end{sphinxuseclass}
\sphinxAtStartPar
For an asset that we know has large positive returns over the sample, “time in the market” beats “timing the market”.


\subsubsection{Implement a long\sphinxhyphen{}short RSI(14) strategy with Bitcoin}
\label{\detokenize{herron_02_practice_03:implement-a-long-short-rsi-14-strategy-with-bitcoin}}
\sphinxAtStartPar
From \sphinxhref{https://www.fidelity.com/learning-center/trading-investing/technical-analysis/technical-indicator-guide/rsi}{Fidelity}:
\begin{quote}

\sphinxAtStartPar
The Relative Strength Index (RSI), developed by J. Welles Wilder, is a momentum oscillator that measures the speed and change of price movements. The RSI oscillates between zero and 100. Traditionally the RSI is considered overbought when above 70 and oversold when below 30. Signals can be generated by looking for divergences and failure swings. RSI can also be used to identify the general trend.
\end{quote}

\sphinxAtStartPar
Here is the RSI formula: \$RSI(n) = 100 \sphinxhyphen{} \textbackslash{}frac\{100\}\{1 + RS(n)\}\$, where \$RS(n) = \textbackslash{}frac\{SMA(U, n)\}\{SMA(D, n)\}\$.
For “up days”, \$U = \textbackslash{}Delta Adj\textbackslash{} Close\$ and \$D = 0\$, and, for “down days”, \$U = 0\$ and \$D = \sphinxhyphen{} \textbackslash{}Delta Adj\textbackslash{} Close\$.
Therefore, \$U\$ and \$D\$ are always non\sphinxhyphen{}negative.
We can learn more about RSI \sphinxhref{https://en.wikipedia.org/wiki/Relative\_strength\_index}{here}.

\sphinxAtStartPar
We will implement a long\sphinxhyphen{}short RSI(14) as follows:
\begin{enumerate}
\sphinxsetlistlabels{\arabic}{enumi}{enumii}{}{.}%
\item {} 
\sphinxAtStartPar
Enter a long position when  the RSI crosses 30 from below, and exit the position when the RSI crosses 50 from below

\item {} 
\sphinxAtStartPar
Enter a short position when the RSI crosses 70 from above, and exit the position when the RSI crosses 50 from above

\end{enumerate}

\sphinxAtStartPar
\sphinxstyleemphasis{\sphinxstylestrong{After class, I replace the \sphinxcode{\sphinxupquote{np.where()}} for \sphinxcode{\sphinxupquote{U}} and \sphinxcode{\sphinxupquote{D}} with \sphinxcode{\sphinxupquote{np.select()}} for (1) consistency and (2) to better handle missing observations.}}

\begin{sphinxuseclass}{cell}\begin{sphinxVerbatimInput}

\begin{sphinxuseclass}{cell_input}
\begin{sphinxVerbatim}[commandchars=\\\{\}]
\PYG{k}{def} \PYG{n+nf}{rsi}\PYG{p}{(}\PYG{n}{df}\PYG{p}{,} \PYG{n}{n}\PYG{o}{=}\PYG{l+m+mi}{14}\PYG{p}{,} \PYG{n}{lb}\PYG{o}{=}\PYG{l+m+mi}{30}\PYG{p}{,} \PYG{n}{mb}\PYG{o}{=}\PYG{l+m+mi}{50}\PYG{p}{,} \PYG{n}{ub}\PYG{o}{=}\PYG{l+m+mi}{70}\PYG{p}{)}\PYG{p}{:}
    \PYG{k}{return} \PYG{n}{df}\PYG{o}{.}\PYG{n}{assign}\PYG{p}{(}
        \PYG{n}{Return} \PYG{o}{=} \PYG{k}{lambda} \PYG{n}{x}\PYG{p}{:} \PYG{n}{x}\PYG{p}{[}\PYG{l+s+s1}{\PYGZsq{}}\PYG{l+s+s1}{Adj Close}\PYG{l+s+s1}{\PYGZsq{}}\PYG{p}{]}\PYG{o}{.}\PYG{n}{pct\PYGZus{}change}\PYG{p}{(}\PYG{p}{)}\PYG{p}{,}
        \PYG{n}{Diff} \PYG{o}{=} \PYG{k}{lambda} \PYG{n}{x}\PYG{p}{:} \PYG{n}{x}\PYG{p}{[}\PYG{l+s+s1}{\PYGZsq{}}\PYG{l+s+s1}{Adj Close}\PYG{l+s+s1}{\PYGZsq{}}\PYG{p}{]}\PYG{o}{.}\PYG{n}{diff}\PYG{p}{(}\PYG{p}{)}\PYG{p}{,}
        \PYG{n}{U} \PYG{o}{=} \PYG{k}{lambda} \PYG{n}{x}\PYG{p}{:} \PYG{n}{np}\PYG{o}{.}\PYG{n}{select}\PYG{p}{(}
            \PYG{n}{condlist}\PYG{o}{=}\PYG{p}{[}\PYG{n}{x}\PYG{p}{[}\PYG{l+s+s1}{\PYGZsq{}}\PYG{l+s+s1}{Diff}\PYG{l+s+s1}{\PYGZsq{}}\PYG{p}{]} \PYG{o}{\PYGZgt{}}\PYG{o}{=} \PYG{l+m+mi}{0}\PYG{p}{,} \PYG{n}{x}\PYG{p}{[}\PYG{l+s+s1}{\PYGZsq{}}\PYG{l+s+s1}{Diff}\PYG{l+s+s1}{\PYGZsq{}}\PYG{p}{]} \PYG{o}{\PYGZlt{}} \PYG{l+m+mi}{0}\PYG{p}{]}\PYG{p}{,}
            \PYG{n}{choicelist}\PYG{o}{=}\PYG{p}{[}\PYG{n}{x}\PYG{p}{[}\PYG{l+s+s1}{\PYGZsq{}}\PYG{l+s+s1}{Diff}\PYG{l+s+s1}{\PYGZsq{}}\PYG{p}{]}\PYG{p}{,} \PYG{l+m+mi}{0}\PYG{p}{]}\PYG{p}{,}
            \PYG{n}{default}\PYG{o}{=}\PYG{n}{np}\PYG{o}{.}\PYG{n}{nan}
        \PYG{p}{)}\PYG{p}{,}
        \PYG{n}{D} \PYG{o}{=} \PYG{k}{lambda} \PYG{n}{x}\PYG{p}{:} \PYG{n}{np}\PYG{o}{.}\PYG{n}{select}\PYG{p}{(}
            \PYG{n}{condlist}\PYG{o}{=}\PYG{p}{[}\PYG{n}{x}\PYG{p}{[}\PYG{l+s+s1}{\PYGZsq{}}\PYG{l+s+s1}{Diff}\PYG{l+s+s1}{\PYGZsq{}}\PYG{p}{]} \PYG{o}{\PYGZlt{}}\PYG{o}{=} \PYG{l+m+mi}{0}\PYG{p}{,} \PYG{n}{x}\PYG{p}{[}\PYG{l+s+s1}{\PYGZsq{}}\PYG{l+s+s1}{Diff}\PYG{l+s+s1}{\PYGZsq{}}\PYG{p}{]} \PYG{o}{\PYGZgt{}} \PYG{l+m+mi}{0}\PYG{p}{]}\PYG{p}{,}
            \PYG{n}{choicelist}\PYG{o}{=}\PYG{p}{[}\PYG{o}{\PYGZhy{}}\PYG{l+m+mi}{1} \PYG{o}{*} \PYG{n}{x}\PYG{p}{[}\PYG{l+s+s1}{\PYGZsq{}}\PYG{l+s+s1}{Diff}\PYG{l+s+s1}{\PYGZsq{}}\PYG{p}{]}\PYG{p}{,} \PYG{l+m+mi}{0}\PYG{p}{]}\PYG{p}{,}
            \PYG{n}{default}\PYG{o}{=}\PYG{n}{np}\PYG{o}{.}\PYG{n}{nan}
        \PYG{p}{)}\PYG{p}{,}
        \PYG{n}{SMAU} \PYG{o}{=} \PYG{k}{lambda} \PYG{n}{x}\PYG{p}{:} \PYG{n}{x}\PYG{p}{[}\PYG{l+s+s1}{\PYGZsq{}}\PYG{l+s+s1}{U}\PYG{l+s+s1}{\PYGZsq{}}\PYG{p}{]}\PYG{o}{.}\PYG{n}{rolling}\PYG{p}{(}\PYG{n}{n}\PYG{p}{)}\PYG{o}{.}\PYG{n}{mean}\PYG{p}{(}\PYG{p}{)}\PYG{p}{,}
        \PYG{n}{SMAD} \PYG{o}{=} \PYG{k}{lambda} \PYG{n}{x}\PYG{p}{:} \PYG{n}{x}\PYG{p}{[}\PYG{l+s+s1}{\PYGZsq{}}\PYG{l+s+s1}{D}\PYG{l+s+s1}{\PYGZsq{}}\PYG{p}{]}\PYG{o}{.}\PYG{n}{rolling}\PYG{p}{(}\PYG{n}{n}\PYG{p}{)}\PYG{o}{.}\PYG{n}{mean}\PYG{p}{(}\PYG{p}{)}\PYG{p}{,}
        \PYG{n}{RS} \PYG{o}{=} \PYG{k}{lambda} \PYG{n}{x}\PYG{p}{:} \PYG{n}{x}\PYG{p}{[}\PYG{l+s+s1}{\PYGZsq{}}\PYG{l+s+s1}{SMAU}\PYG{l+s+s1}{\PYGZsq{}}\PYG{p}{]} \PYG{o}{/} \PYG{n}{x}\PYG{p}{[}\PYG{l+s+s1}{\PYGZsq{}}\PYG{l+s+s1}{SMAD}\PYG{l+s+s1}{\PYGZsq{}}\PYG{p}{]}\PYG{p}{,}
        \PYG{n}{RSI} \PYG{o}{=} \PYG{k}{lambda} \PYG{n}{x}\PYG{p}{:} \PYG{l+m+mi}{100} \PYG{o}{\PYGZhy{}} \PYG{l+m+mi}{100} \PYG{o}{/} \PYG{p}{(}\PYG{l+m+mi}{1} \PYG{o}{+} \PYG{n}{x}\PYG{p}{[}\PYG{l+s+s1}{\PYGZsq{}}\PYG{l+s+s1}{RS}\PYG{l+s+s1}{\PYGZsq{}}\PYG{p}{]}\PYG{p}{)}\PYG{p}{,}
        \PYG{n}{Position\PYGZus{}with\PYGZus{}nan} \PYG{o}{=} \PYG{k}{lambda} \PYG{n}{x}\PYG{p}{:} \PYG{n}{np}\PYG{o}{.}\PYG{n}{select}\PYG{p}{(}
            \PYG{n}{condlist}\PYG{o}{=}\PYG{p}{[}
                \PYG{p}{(}\PYG{n}{x}\PYG{p}{[}\PYG{l+s+s1}{\PYGZsq{}}\PYG{l+s+s1}{RSI}\PYG{l+s+s1}{\PYGZsq{}}\PYG{p}{]}\PYG{o}{.}\PYG{n}{shift}\PYG{p}{(}\PYG{l+m+mi}{1}\PYG{p}{)} \PYG{o}{\PYGZgt{}}\PYG{o}{=} \PYG{n}{lb}\PYG{p}{)} \PYG{o}{\PYGZam{}} \PYG{p}{(}\PYG{n}{x}\PYG{p}{[}\PYG{l+s+s1}{\PYGZsq{}}\PYG{l+s+s1}{RSI}\PYG{l+s+s1}{\PYGZsq{}}\PYG{p}{]}\PYG{o}{.}\PYG{n}{shift}\PYG{p}{(}\PYG{l+m+mi}{2}\PYG{p}{)} \PYG{o}{\PYGZlt{}} \PYG{n}{lb}\PYG{p}{)}\PYG{p}{,} 
                \PYG{p}{(}\PYG{n}{x}\PYG{p}{[}\PYG{l+s+s1}{\PYGZsq{}}\PYG{l+s+s1}{RSI}\PYG{l+s+s1}{\PYGZsq{}}\PYG{p}{]}\PYG{o}{.}\PYG{n}{shift}\PYG{p}{(}\PYG{l+m+mi}{1}\PYG{p}{)} \PYG{o}{\PYGZgt{}}\PYG{o}{=} \PYG{n}{mb}\PYG{p}{)} \PYG{o}{\PYGZam{}} \PYG{p}{(}\PYG{n}{x}\PYG{p}{[}\PYG{l+s+s1}{\PYGZsq{}}\PYG{l+s+s1}{RSI}\PYG{l+s+s1}{\PYGZsq{}}\PYG{p}{]}\PYG{o}{.}\PYG{n}{shift}\PYG{p}{(}\PYG{l+m+mi}{2}\PYG{p}{)} \PYG{o}{\PYGZlt{}} \PYG{n}{mb}\PYG{p}{)}\PYG{p}{,}
                \PYG{p}{(}\PYG{n}{x}\PYG{p}{[}\PYG{l+s+s1}{\PYGZsq{}}\PYG{l+s+s1}{RSI}\PYG{l+s+s1}{\PYGZsq{}}\PYG{p}{]}\PYG{o}{.}\PYG{n}{shift}\PYG{p}{(}\PYG{l+m+mi}{1}\PYG{p}{)} \PYG{o}{\PYGZlt{}}\PYG{o}{=} \PYG{n}{ub}\PYG{p}{)} \PYG{o}{\PYGZam{}} \PYG{p}{(}\PYG{n}{x}\PYG{p}{[}\PYG{l+s+s1}{\PYGZsq{}}\PYG{l+s+s1}{RSI}\PYG{l+s+s1}{\PYGZsq{}}\PYG{p}{]}\PYG{o}{.}\PYG{n}{shift}\PYG{p}{(}\PYG{l+m+mi}{2}\PYG{p}{)} \PYG{o}{\PYGZgt{}} \PYG{n}{ub}\PYG{p}{)}\PYG{p}{,} 
                \PYG{p}{(}\PYG{n}{x}\PYG{p}{[}\PYG{l+s+s1}{\PYGZsq{}}\PYG{l+s+s1}{RSI}\PYG{l+s+s1}{\PYGZsq{}}\PYG{p}{]}\PYG{o}{.}\PYG{n}{shift}\PYG{p}{(}\PYG{l+m+mi}{1}\PYG{p}{)} \PYG{o}{\PYGZlt{}}\PYG{o}{=} \PYG{n}{mb}\PYG{p}{)} \PYG{o}{\PYGZam{}} \PYG{p}{(}\PYG{n}{x}\PYG{p}{[}\PYG{l+s+s1}{\PYGZsq{}}\PYG{l+s+s1}{RSI}\PYG{l+s+s1}{\PYGZsq{}}\PYG{p}{]}\PYG{o}{.}\PYG{n}{shift}\PYG{p}{(}\PYG{l+m+mi}{2}\PYG{p}{)} \PYG{o}{\PYGZgt{}} \PYG{n}{mb}\PYG{p}{)}\PYG{p}{,}
            \PYG{p}{]}\PYG{p}{,}
            \PYG{n}{choicelist}\PYG{o}{=}\PYG{p}{[}
                \PYG{l+m+mi}{1}\PYG{p}{,} 
                \PYG{l+m+mi}{0}\PYG{p}{,}
                \PYG{o}{\PYGZhy{}}\PYG{l+m+mi}{1}\PYG{p}{,}
                \PYG{l+m+mi}{0}
            \PYG{p}{]}\PYG{p}{,}
            \PYG{n}{default}\PYG{o}{=}\PYG{n}{np}\PYG{o}{.}\PYG{n}{nan}
        \PYG{p}{)}\PYG{p}{,}
        \PYG{n}{Position} \PYG{o}{=} \PYG{k}{lambda} \PYG{n}{x}\PYG{p}{:} \PYG{n}{x}\PYG{p}{[}\PYG{l+s+s1}{\PYGZsq{}}\PYG{l+s+s1}{Position\PYGZus{}with\PYGZus{}nan}\PYG{l+s+s1}{\PYGZsq{}}\PYG{p}{]}\PYG{o}{.}\PYG{n}{fillna}\PYG{p}{(}\PYG{n}{method}\PYG{o}{=}\PYG{l+s+s1}{\PYGZsq{}}\PYG{l+s+s1}{ffill}\PYG{l+s+s1}{\PYGZsq{}}\PYG{p}{)}\PYG{p}{,}
        \PYG{n}{Strategy} \PYG{o}{=} \PYG{k}{lambda} \PYG{n}{x}\PYG{p}{:} \PYG{n}{x}\PYG{p}{[}\PYG{l+s+s1}{\PYGZsq{}}\PYG{l+s+s1}{Position}\PYG{l+s+s1}{\PYGZsq{}}\PYG{p}{]} \PYG{o}{*} \PYG{n}{x}\PYG{p}{[}\PYG{l+s+s1}{\PYGZsq{}}\PYG{l+s+s1}{Return}\PYG{l+s+s1}{\PYGZsq{}}\PYG{p}{]}
    \PYG{p}{)}
\end{sphinxVerbatim}

\end{sphinxuseclass}\end{sphinxVerbatimInput}

\end{sphinxuseclass}
\begin{sphinxuseclass}{cell}\begin{sphinxVerbatimInput}

\begin{sphinxuseclass}{cell_input}
\begin{sphinxVerbatim}[commandchars=\\\{\}]
\PYG{n}{btc\PYGZus{}rsi} \PYG{o}{=} \PYG{n}{rsi}\PYG{p}{(}\PYG{n}{btc}\PYG{p}{)}

\PYG{n}{btc\PYGZus{}rsi}\PYG{o}{.}\PYG{n}{tail}\PYG{p}{(}\PYG{p}{)}
\end{sphinxVerbatim}

\end{sphinxuseclass}\end{sphinxVerbatimInput}
\begin{sphinxVerbatimOutput}

\begin{sphinxuseclass}{cell_output}
\begin{sphinxVerbatim}[commandchars=\\\{\}]
Variable         Open       High        Low      Close  Adj Close  \PYGZbs{}
Date                                                                
2023\PYGZhy{}03\PYGZhy{}11 20187.8770 20792.5254 20068.6602 20632.4102 20632.4102   
2023\PYGZhy{}03\PYGZhy{}12 20628.0293 22185.0312 20448.8066 22163.9492 22163.9492   
2023\PYGZhy{}03\PYGZhy{}13 22156.4062 24550.8379 21918.1992 24197.5332 24197.5332   
2023\PYGZhy{}03\PYGZhy{}14 24201.7656 26514.7168 24081.1836 24746.0742 24746.0742   
2023\PYGZhy{}03\PYGZhy{}15 24734.6523 25169.7305 24048.3516 24617.3047 24617.3047   

Variable         Volume  Return      Diff         U        D     SMAU  \PYGZbs{}
Date                                                                    
2023\PYGZhy{}03\PYGZhy{}11  30180288176  0.0221  445.1660  445.1660   0.0000 100.8832   
2023\PYGZhy{}03\PYGZhy{}12  29279035521  0.0742 1531.5391 1531.5391   0.0000 182.7190   
2023\PYGZhy{}03\PYGZhy{}13  49466362688  0.0918 2033.5840 2033.5840   0.0000 327.9750   
2023\PYGZhy{}03\PYGZhy{}14  54622230164  0.0227  548.5410  548.5410   0.0000 367.1565   
2023\PYGZhy{}03\PYGZhy{}15  44081737728 \PYGZhy{}0.0052 \PYGZhy{}128.7695    0.0000 128.7695 331.4996   

Variable       SMAD     RS     RSI  Position\PYGZus{}with\PYGZus{}nan  Position  Strategy  
Date                                                                       
2023\PYGZhy{}03\PYGZhy{}11 282.5236 0.3571 26.3123                NaN    1.0000    0.0221  
2023\PYGZhy{}03\PYGZhy{}12 282.5236 0.6467 39.2739                NaN    1.0000    0.0742  
2023\PYGZhy{}03\PYGZhy{}13 279.7849 1.1722 53.9646             1.0000    1.0000    0.0918  
2023\PYGZhy{}03\PYGZhy{}14 252.9622 1.4514 59.2075             0.0000    0.0000    0.0000  
2023\PYGZhy{}03\PYGZhy{}15 262.1600 1.2645 55.8400                NaN    0.0000   \PYGZhy{}0.0000  
\end{sphinxVerbatim}

\end{sphinxuseclass}\end{sphinxVerbatimOutput}

\end{sphinxuseclass}
\begin{sphinxuseclass}{cell}\begin{sphinxVerbatimInput}

\begin{sphinxuseclass}{cell_input}
\begin{sphinxVerbatim}[commandchars=\\\{\}]
\PYG{n}{\PYGZus{}} \PYG{o}{=} \PYG{n}{btc\PYGZus{}rsi}\PYG{p}{[}\PYG{p}{[}\PYG{l+s+s1}{\PYGZsq{}}\PYG{l+s+s1}{Return}\PYG{l+s+s1}{\PYGZsq{}}\PYG{p}{,} \PYG{l+s+s1}{\PYGZsq{}}\PYG{l+s+s1}{Strategy}\PYG{l+s+s1}{\PYGZsq{}}\PYG{p}{]}\PYG{p}{]}\PYG{o}{.}\PYG{n}{dropna}\PYG{p}{(}\PYG{p}{)}

\PYG{p}{(}
    \PYG{n}{\PYGZus{}}
    \PYG{o}{.}\PYG{n}{add}\PYG{p}{(}\PYG{l+m+mi}{1}\PYG{p}{)}
    \PYG{o}{.}\PYG{n}{cumprod}\PYG{p}{(}\PYG{p}{)}
    \PYG{o}{.}\PYG{n}{rename\PYGZus{}axis}\PYG{p}{(}\PYG{n}{columns}\PYG{o}{=}\PYG{l+s+s1}{\PYGZsq{}}\PYG{l+s+s1}{Strategy}\PYG{l+s+s1}{\PYGZsq{}}\PYG{p}{)}
    \PYG{o}{.}\PYG{n}{rename}\PYG{p}{(}\PYG{n}{columns}\PYG{o}{=}\PYG{p}{\PYGZob{}}\PYG{l+s+s1}{\PYGZsq{}}\PYG{l+s+s1}{Return}\PYG{l+s+s1}{\PYGZsq{}}\PYG{p}{:} \PYG{l+s+s1}{\PYGZsq{}}\PYG{l+s+s1}{Buy\PYGZhy{}And\PYGZhy{}Hold}\PYG{l+s+s1}{\PYGZsq{}}\PYG{p}{,} \PYG{l+s+s1}{\PYGZsq{}}\PYG{l+s+s1}{Strategy}\PYG{l+s+s1}{\PYGZsq{}}\PYG{p}{:} \PYG{l+s+s1}{\PYGZsq{}}\PYG{l+s+s1}{RSI(14)}\PYG{l+s+s1}{\PYGZsq{}}\PYG{p}{\PYGZcb{}}\PYG{p}{)}
    \PYG{o}{.}\PYG{n}{plot}\PYG{p}{(}\PYG{p}{)}
\PYG{p}{)}
\PYG{n}{plt}\PYG{o}{.}\PYG{n}{ylabel}\PYG{p}{(}\PYG{l+s+s1}{\PYGZsq{}}\PYG{l+s+s1}{Value (\PYGZdl{})}\PYG{l+s+s1}{\PYGZsq{}}\PYG{p}{)}
\PYG{n}{plt}\PYG{o}{.}\PYG{n}{title}\PYG{p}{(}\PYG{l+s+sa}{f}\PYG{l+s+s1}{\PYGZsq{}}\PYG{l+s+s1}{Value of \PYGZdl{}1 Invested at Close on }\PYG{l+s+si}{\PYGZob{}}\PYG{n}{\PYGZus{}}\PYG{o}{.}\PYG{n}{index}\PYG{p}{[}\PYG{l+m+mi}{0}\PYG{p}{]} \PYG{o}{\PYGZhy{}} \PYG{n}{pd}\PYG{o}{.}\PYG{n}{offsets}\PYG{o}{.}\PYG{n}{Day}\PYG{p}{(}\PYG{l+m+mi}{1}\PYG{p}{)}\PYG{l+s+si}{:}\PYG{l+s+s1}{\PYGZpc{}B \PYGZpc{}d, \PYGZpc{}Y}\PYG{l+s+si}{\PYGZcb{}}\PYG{l+s+s1}{\PYGZsq{}}\PYG{p}{)}
\PYG{n}{plt}\PYG{o}{.}\PYG{n}{show}\PYG{p}{(}\PYG{p}{)}
\end{sphinxVerbatim}

\end{sphinxuseclass}\end{sphinxVerbatimInput}
\begin{sphinxVerbatimOutput}

\begin{sphinxuseclass}{cell_output}
\noindent\sphinxincludegraphics{{7a408772ea36fc857aee85128d3c0797cc8766924aa0789e89e26ae642d598d7}.png}

\end{sphinxuseclass}\end{sphinxVerbatimOutput}

\end{sphinxuseclass}
\sphinxAtStartPar
We can compare all three!
Shorting Bitcoin has been dangerous, as the poor returns on RSI(14) show!

\begin{sphinxuseclass}{cell}\begin{sphinxVerbatimInput}

\begin{sphinxuseclass}{cell_input}
\begin{sphinxVerbatim}[commandchars=\\\{\}]
\PYG{n}{\PYGZus{}} \PYG{o}{=} \PYG{p}{(}
    \PYG{n}{btc\PYGZus{}sma}\PYG{p}{[}\PYG{p}{[}\PYG{l+s+s1}{\PYGZsq{}}\PYG{l+s+s1}{Return}\PYG{l+s+s1}{\PYGZsq{}}\PYG{p}{,} \PYG{l+s+s1}{\PYGZsq{}}\PYG{l+s+s1}{Strategy}\PYG{l+s+s1}{\PYGZsq{}}\PYG{p}{]}\PYG{p}{]}
    \PYG{o}{.}\PYG{n}{join}\PYG{p}{(}
        \PYG{n}{btc\PYGZus{}bb}\PYG{p}{[}\PYG{p}{[}\PYG{l+s+s1}{\PYGZsq{}}\PYG{l+s+s1}{Strategy}\PYG{l+s+s1}{\PYGZsq{}}\PYG{p}{]}\PYG{p}{]}\PYG{o}{.}\PYG{n}{add\PYGZus{}suffix}\PYG{p}{(}\PYG{l+s+s1}{\PYGZsq{}}\PYG{l+s+s1}{\PYGZus{}BB}\PYG{l+s+s1}{\PYGZsq{}}\PYG{p}{)}\PYG{p}{,} 
    \PYG{p}{)}
    \PYG{o}{.}\PYG{n}{join}\PYG{p}{(}
        \PYG{n}{btc\PYGZus{}rsi}\PYG{p}{[}\PYG{p}{[}\PYG{l+s+s1}{\PYGZsq{}}\PYG{l+s+s1}{Strategy}\PYG{l+s+s1}{\PYGZsq{}}\PYG{p}{]}\PYG{p}{]}\PYG{o}{.}\PYG{n}{add\PYGZus{}suffix}\PYG{p}{(}\PYG{l+s+s1}{\PYGZsq{}}\PYG{l+s+s1}{\PYGZus{}RSI}\PYG{l+s+s1}{\PYGZsq{}}\PYG{p}{)}\PYG{p}{,} 
    \PYG{p}{)}
    \PYG{o}{.}\PYG{n}{dropna}\PYG{p}{(}\PYG{p}{)}
\PYG{p}{)}


\PYG{p}{(}
    \PYG{n}{\PYGZus{}}
    \PYG{o}{.}\PYG{n}{add}\PYG{p}{(}\PYG{l+m+mi}{1}\PYG{p}{)}
    \PYG{o}{.}\PYG{n}{cumprod}\PYG{p}{(}\PYG{p}{)}
    \PYG{o}{.}\PYG{n}{rename\PYGZus{}axis}\PYG{p}{(}\PYG{n}{columns}\PYG{o}{=}\PYG{l+s+s1}{\PYGZsq{}}\PYG{l+s+s1}{Strategy}\PYG{l+s+s1}{\PYGZsq{}}\PYG{p}{)}
    \PYG{o}{.}\PYG{n}{rename}\PYG{p}{(}\PYG{n}{columns}\PYG{o}{=}
            \PYG{p}{\PYGZob{}}
                \PYG{l+s+s1}{\PYGZsq{}}\PYG{l+s+s1}{Return}\PYG{l+s+s1}{\PYGZsq{}}\PYG{p}{:} \PYG{l+s+s1}{\PYGZsq{}}\PYG{l+s+s1}{Buy\PYGZhy{}And\PYGZhy{}Hold}\PYG{l+s+s1}{\PYGZsq{}}\PYG{p}{,} 
                \PYG{l+s+s1}{\PYGZsq{}}\PYG{l+s+s1}{Strategy}\PYG{l+s+s1}{\PYGZsq{}}\PYG{p}{:} \PYG{l+s+s1}{\PYGZsq{}}\PYG{l+s+s1}{SMA(20)}\PYG{l+s+s1}{\PYGZsq{}}\PYG{p}{,}
                \PYG{l+s+s1}{\PYGZsq{}}\PYG{l+s+s1}{Strategy\PYGZus{}BB}\PYG{l+s+s1}{\PYGZsq{}}\PYG{p}{:} \PYG{l+s+s1}{\PYGZsq{}}\PYG{l+s+s1}{BB(20, 2)}\PYG{l+s+s1}{\PYGZsq{}}\PYG{p}{,}
                \PYG{l+s+s1}{\PYGZsq{}}\PYG{l+s+s1}{Strategy\PYGZus{}RSI}\PYG{l+s+s1}{\PYGZsq{}}\PYG{p}{:} \PYG{l+s+s1}{\PYGZsq{}}\PYG{l+s+s1}{RSI(14)}\PYG{l+s+s1}{\PYGZsq{}}\PYG{p}{,}
            \PYG{p}{\PYGZcb{}}
           \PYG{p}{)}
    \PYG{o}{.}\PYG{n}{plot}\PYG{p}{(}\PYG{p}{)}
\PYG{p}{)}
\PYG{n}{plt}\PYG{o}{.}\PYG{n}{semilogy}\PYG{p}{(}\PYG{p}{)}
\PYG{n}{plt}\PYG{o}{.}\PYG{n}{ylabel}\PYG{p}{(}\PYG{l+s+s1}{\PYGZsq{}}\PYG{l+s+s1}{Value (\PYGZdl{})}\PYG{l+s+s1}{\PYGZsq{}}\PYG{p}{)}
\PYG{n}{plt}\PYG{o}{.}\PYG{n}{title}\PYG{p}{(}\PYG{l+s+sa}{f}\PYG{l+s+s1}{\PYGZsq{}}\PYG{l+s+s1}{Value of \PYGZdl{}1 Invested at Close on }\PYG{l+s+si}{\PYGZob{}}\PYG{n}{\PYGZus{}}\PYG{o}{.}\PYG{n}{index}\PYG{p}{[}\PYG{l+m+mi}{0}\PYG{p}{]} \PYG{o}{\PYGZhy{}} \PYG{n}{pd}\PYG{o}{.}\PYG{n}{offsets}\PYG{o}{.}\PYG{n}{Day}\PYG{p}{(}\PYG{l+m+mi}{1}\PYG{p}{)}\PYG{l+s+si}{:}\PYG{l+s+s1}{\PYGZpc{}B \PYGZpc{}d, \PYGZpc{}Y}\PYG{l+s+si}{\PYGZcb{}}\PYG{l+s+s1}{\PYGZsq{}}\PYG{p}{)}
\PYG{n}{plt}\PYG{o}{.}\PYG{n}{show}\PYG{p}{(}\PYG{p}{)}
\end{sphinxVerbatim}

\end{sphinxuseclass}\end{sphinxVerbatimInput}
\begin{sphinxVerbatimOutput}

\begin{sphinxuseclass}{cell_output}
\noindent\sphinxincludegraphics{{bca6fe8a2ee1156bf6d412f316c36d0be9567385beaf9e1190a825a35bc69504}.png}

\end{sphinxuseclass}\end{sphinxVerbatimOutput}

\end{sphinxuseclass}
\sphinxstepscope


\section{Herron Topic 2 \sphinxhyphen{} Practice (Monday 11:45 AM, Section 4)}
\label{\detokenize{herron_02_practice_04:herron-topic-2-practice-monday-11-45-am-section-4}}\label{\detokenize{herron_02_practice_04::doc}}

\subsection{Announcements}
\label{\detokenize{herron_02_practice_04:announcements}}\begin{itemize}
\item {} 
\sphinxAtStartPar
I will finish grading projects this week/weekend

\item {} 
\sphinxAtStartPar
Quiz 5 due \textasciitilde{}\textasciitilde{}Friday at 11:59 PM\textasciitilde{}\textasciitilde{} Sunday at 11:59 PM
\begin{itemize}
\item {} 
\sphinxAtStartPar
A handful of students have submitted identical quizzes

\item {} 
\sphinxAtStartPar
Quizzes are individual efforts

\item {} 
\sphinxAtStartPar
Do not assume it is hard to for me to compare quiz and project submissions

\end{itemize}

\item {} 
\sphinxAtStartPar
DataCamp 20,000 XP due \sphinxstyleemphasis{next} Friday at 11:59 PM

\item {} 
\sphinxAtStartPar
Attendance and participation account for 5\% of your grade

\end{itemize}


\subsection{Practice}
\label{\detokenize{herron_02_practice_04:practice}}
\begin{sphinxuseclass}{cell}\begin{sphinxVerbatimInput}

\begin{sphinxuseclass}{cell_input}
\begin{sphinxVerbatim}[commandchars=\\\{\}]
\PYG{k+kn}{import} \PYG{n+nn}{matplotlib}\PYG{n+nn}{.}\PYG{n+nn}{pyplot} \PYG{k}{as} \PYG{n+nn}{plt}
\PYG{k+kn}{import} \PYG{n+nn}{numpy} \PYG{k}{as} \PYG{n+nn}{np}
\PYG{k+kn}{import} \PYG{n+nn}{pandas} \PYG{k}{as} \PYG{n+nn}{pd}
\end{sphinxVerbatim}

\end{sphinxuseclass}\end{sphinxVerbatimInput}

\end{sphinxuseclass}
\begin{sphinxuseclass}{cell}\begin{sphinxVerbatimInput}

\begin{sphinxuseclass}{cell_input}
\begin{sphinxVerbatim}[commandchars=\\\{\}]
\PYG{o}{\PYGZpc{}}\PYG{k}{config} InlineBackend.figure\PYGZus{}format = \PYGZsq{}retina\PYGZsq{}
\PYG{o}{\PYGZpc{}}\PYG{k}{precision} 4
\PYG{n}{pd}\PYG{o}{.}\PYG{n}{options}\PYG{o}{.}\PYG{n}{display}\PYG{o}{.}\PYG{n}{float\PYGZus{}format} \PYG{o}{=} \PYG{l+s+s1}{\PYGZsq{}}\PYG{l+s+si}{\PYGZob{}:.4f\PYGZcb{}}\PYG{l+s+s1}{\PYGZsq{}}\PYG{o}{.}\PYG{n}{format}
\end{sphinxVerbatim}

\end{sphinxuseclass}\end{sphinxVerbatimInput}

\end{sphinxuseclass}

\subsubsection{Implement the SMA(20) strategy with Bitcoin from the lecture notebook}
\label{\detokenize{herron_02_practice_04:implement-the-sma-20-strategy-with-bitcoin-from-the-lecture-notebook}}
\sphinxAtStartPar
\textasciitilde{}\textasciitilde{}Try to create the \sphinxcode{\sphinxupquote{btc}} data frame in one code cell with one assignment (i.e., one \sphinxcode{\sphinxupquote{=}}).\textasciitilde{}\textasciitilde{}
\sphinxstyleemphasis{\sphinxstylestrong{Write a function \sphinxcode{\sphinxupquote{sma()}} that accepts a data frame \sphinxcode{\sphinxupquote{df}} and window size \sphinxcode{\sphinxupquote{n}}.
Use your \sphinxcode{\sphinxupquote{sma()}} function to implement the SMA(20) strategy and assign it to data frame \sphinxcode{\sphinxupquote{btc\_sma}}.}}

\begin{sphinxuseclass}{cell}\begin{sphinxVerbatimInput}

\begin{sphinxuseclass}{cell_input}
\begin{sphinxVerbatim}[commandchars=\\\{\}]
\PYG{k+kn}{import} \PYG{n+nn}{yfinance} \PYG{k}{as} \PYG{n+nn}{yf}
\end{sphinxVerbatim}

\end{sphinxuseclass}\end{sphinxVerbatimInput}

\end{sphinxuseclass}
\begin{sphinxuseclass}{cell}\begin{sphinxVerbatimInput}

\begin{sphinxuseclass}{cell_input}
\begin{sphinxVerbatim}[commandchars=\\\{\}]
\PYG{n}{btc} \PYG{o}{=} \PYG{p}{(}
    \PYG{n}{yf}\PYG{o}{.}\PYG{n}{download}\PYG{p}{(}\PYG{n}{tickers}\PYG{o}{=}\PYG{l+s+s1}{\PYGZsq{}}\PYG{l+s+s1}{BTC\PYGZhy{}USD}\PYG{l+s+s1}{\PYGZsq{}}\PYG{p}{,} \PYG{n}{progress}\PYG{o}{=}\PYG{k+kc}{False}\PYG{p}{)}
    \PYG{o}{.}\PYG{n}{assign}\PYG{p}{(}\PYG{n}{Date} \PYG{o}{=} \PYG{k}{lambda} \PYG{n}{x}\PYG{p}{:} \PYG{n}{x}\PYG{o}{.}\PYG{n}{index}\PYG{o}{.}\PYG{n}{tz\PYGZus{}localize}\PYG{p}{(}\PYG{k+kc}{None}\PYG{p}{)}\PYG{p}{)}
    \PYG{o}{.}\PYG{n}{set\PYGZus{}index}\PYG{p}{(}\PYG{l+s+s1}{\PYGZsq{}}\PYG{l+s+s1}{Date}\PYG{l+s+s1}{\PYGZsq{}}\PYG{p}{)}
    \PYG{o}{.}\PYG{n}{rename\PYGZus{}axis}\PYG{p}{(}\PYG{n}{columns}\PYG{o}{=}\PYG{l+s+s1}{\PYGZsq{}}\PYG{l+s+s1}{Variable}\PYG{l+s+s1}{\PYGZsq{}}\PYG{p}{)}
\PYG{p}{)}

\PYG{n}{btc}\PYG{o}{.}\PYG{n}{head}\PYG{p}{(}\PYG{p}{)}
\end{sphinxVerbatim}

\end{sphinxuseclass}\end{sphinxVerbatimInput}
\begin{sphinxVerbatimOutput}

\begin{sphinxuseclass}{cell_output}
\begin{sphinxVerbatim}[commandchars=\\\{\}]
Variable       Open     High      Low    Close  Adj Close    Volume
Date                                                               
2014\PYGZhy{}09\PYGZhy{}17 465.8640 468.1740 452.4220 457.3340   457.3340  21056800
2014\PYGZhy{}09\PYGZhy{}18 456.8600 456.8600 413.1040 424.4400   424.4400  34483200
2014\PYGZhy{}09\PYGZhy{}19 424.1030 427.8350 384.5320 394.7960   394.7960  37919700
2014\PYGZhy{}09\PYGZhy{}20 394.6730 423.2960 389.8830 408.9040   408.9040  36863600
2014\PYGZhy{}09\PYGZhy{}21 408.0850 412.4260 393.1810 398.8210   398.8210  26580100
\end{sphinxVerbatim}

\end{sphinxuseclass}\end{sphinxVerbatimOutput}

\end{sphinxuseclass}
\begin{sphinxuseclass}{cell}\begin{sphinxVerbatimInput}

\begin{sphinxuseclass}{cell_input}
\begin{sphinxVerbatim}[commandchars=\\\{\}]
\PYG{k}{def} \PYG{n+nf}{sma}\PYG{p}{(}\PYG{n}{df}\PYG{p}{,} \PYG{n}{n}\PYG{o}{=}\PYG{l+m+mi}{20}\PYG{p}{)}\PYG{p}{:}
    \PYG{k}{return} \PYG{p}{(}
        \PYG{n}{df}
        \PYG{o}{.}\PYG{n}{assign}\PYG{p}{(}
            \PYG{n}{Return} \PYG{o}{=} \PYG{k}{lambda} \PYG{n}{x}\PYG{p}{:} \PYG{n}{x}\PYG{p}{[}\PYG{l+s+s1}{\PYGZsq{}}\PYG{l+s+s1}{Adj Close}\PYG{l+s+s1}{\PYGZsq{}}\PYG{p}{]}\PYG{o}{.}\PYG{n}{pct\PYGZus{}change}\PYG{p}{(}\PYG{p}{)}\PYG{p}{,}
            \PYG{n}{SMA} \PYG{o}{=} \PYG{k}{lambda} \PYG{n}{x}\PYG{p}{:} \PYG{n}{x}\PYG{p}{[}\PYG{l+s+s1}{\PYGZsq{}}\PYG{l+s+s1}{Adj Close}\PYG{l+s+s1}{\PYGZsq{}}\PYG{p}{]}\PYG{o}{.}\PYG{n}{rolling}\PYG{p}{(}\PYG{n}{n}\PYG{p}{)}\PYG{o}{.}\PYG{n}{mean}\PYG{p}{(}\PYG{p}{)}\PYG{p}{,}
            \PYG{n}{Position} \PYG{o}{=} \PYG{k}{lambda} \PYG{n}{x}\PYG{p}{:} \PYG{n}{np}\PYG{o}{.}\PYG{n}{select}\PYG{p}{(}
                \PYG{n}{condlist}\PYG{o}{=}\PYG{p}{[}
                    \PYG{n}{x}\PYG{p}{[}\PYG{l+s+s1}{\PYGZsq{}}\PYG{l+s+s1}{Adj Close}\PYG{l+s+s1}{\PYGZsq{}}\PYG{p}{]}\PYG{o}{.}\PYG{n}{shift}\PYG{p}{(}\PYG{p}{)} \PYG{o}{\PYGZgt{}} \PYG{n}{x}\PYG{p}{[}\PYG{l+s+s1}{\PYGZsq{}}\PYG{l+s+s1}{SMA}\PYG{l+s+s1}{\PYGZsq{}}\PYG{p}{]}\PYG{o}{.}\PYG{n}{shift}\PYG{p}{(}\PYG{p}{)}\PYG{p}{,} 
                    \PYG{n}{x}\PYG{p}{[}\PYG{l+s+s1}{\PYGZsq{}}\PYG{l+s+s1}{Adj Close}\PYG{l+s+s1}{\PYGZsq{}}\PYG{p}{]}\PYG{o}{.}\PYG{n}{shift}\PYG{p}{(}\PYG{p}{)} \PYG{o}{\PYGZlt{}}\PYG{o}{=} \PYG{n}{x}\PYG{p}{[}\PYG{l+s+s1}{\PYGZsq{}}\PYG{l+s+s1}{SMA}\PYG{l+s+s1}{\PYGZsq{}}\PYG{p}{]}\PYG{o}{.}\PYG{n}{shift}\PYG{p}{(}\PYG{p}{)}
                \PYG{p}{]}\PYG{p}{,}
                \PYG{n}{choicelist}\PYG{o}{=}\PYG{p}{[}
                    \PYG{l+m+mi}{1}\PYG{p}{,} 
                    \PYG{l+m+mi}{0}
                \PYG{p}{]}\PYG{p}{,}
                \PYG{n}{default}\PYG{o}{=}\PYG{n}{np}\PYG{o}{.}\PYG{n}{nan}
            \PYG{p}{)}\PYG{p}{,}
            \PYG{n}{Strategy} \PYG{o}{=} \PYG{k}{lambda} \PYG{n}{x}\PYG{p}{:} \PYG{n}{x}\PYG{p}{[}\PYG{l+s+s1}{\PYGZsq{}}\PYG{l+s+s1}{Position}\PYG{l+s+s1}{\PYGZsq{}}\PYG{p}{]} \PYG{o}{*} \PYG{n}{x}\PYG{p}{[}\PYG{l+s+s1}{\PYGZsq{}}\PYG{l+s+s1}{Return}\PYG{l+s+s1}{\PYGZsq{}}\PYG{p}{]}
        \PYG{p}{)}
    \PYG{p}{)}
\end{sphinxVerbatim}

\end{sphinxuseclass}\end{sphinxVerbatimInput}

\end{sphinxuseclass}
\begin{sphinxuseclass}{cell}\begin{sphinxVerbatimInput}

\begin{sphinxuseclass}{cell_input}
\begin{sphinxVerbatim}[commandchars=\\\{\}]
\PYG{n}{btc\PYGZus{}sma} \PYG{o}{=} \PYG{n}{btc}\PYG{o}{.}\PYG{n}{pipe}\PYG{p}{(}\PYG{n}{sma}\PYG{p}{,} \PYG{n}{n}\PYG{o}{=}\PYG{l+m+mi}{20}\PYG{p}{)}

\PYG{n}{btc\PYGZus{}sma}\PYG{o}{.}\PYG{n}{tail}\PYG{p}{(}\PYG{p}{)}
\end{sphinxVerbatim}

\end{sphinxuseclass}\end{sphinxVerbatimInput}
\begin{sphinxVerbatimOutput}

\begin{sphinxuseclass}{cell_output}
\begin{sphinxVerbatim}[commandchars=\\\{\}]
Variable         Open       High        Low      Close  Adj Close  \PYGZbs{}
Date                                                                
2023\PYGZhy{}03\PYGZhy{}11 20187.8770 20792.5254 20068.6602 20632.4102 20632.4102   
2023\PYGZhy{}03\PYGZhy{}12 20628.0293 22185.0312 20448.8066 22163.9492 22163.9492   
2023\PYGZhy{}03\PYGZhy{}13 22156.4062 24550.8379 21918.1992 24197.5332 24197.5332   
2023\PYGZhy{}03\PYGZhy{}14 24201.7656 26514.7168 24081.1836 24746.0742 24746.0742   
2023\PYGZhy{}03\PYGZhy{}15 24734.6523 25169.7305 24048.3516 24617.3047 24617.3047   

Variable         Volume  Return        SMA  Position  Strategy  
Date                                                            
2023\PYGZhy{}03\PYGZhy{}11  30180288176  0.0221 22791.5311    0.0000    0.0000  
2023\PYGZhy{}03\PYGZhy{}12  29279035521  0.0742 22658.2711    0.0000    0.0000  
2023\PYGZhy{}03\PYGZhy{}13  49466362688  0.0918 22646.3301    0.0000    0.0000  
2023\PYGZhy{}03\PYGZhy{}14  54622230164  0.0227 22674.1916    1.0000    0.0227  
2023\PYGZhy{}03\PYGZhy{}15  44081737728 \PYGZhy{}0.0052 22707.6822    1.0000   \PYGZhy{}0.0052  
\end{sphinxVerbatim}

\end{sphinxuseclass}\end{sphinxVerbatimOutput}

\end{sphinxuseclass}
\begin{sphinxuseclass}{cell}\begin{sphinxVerbatimInput}

\begin{sphinxuseclass}{cell_input}
\begin{sphinxVerbatim}[commandchars=\\\{\}]
\PYG{n}{\PYGZus{}} \PYG{o}{=} \PYG{n}{btc\PYGZus{}sma}\PYG{p}{[}\PYG{p}{[}\PYG{l+s+s1}{\PYGZsq{}}\PYG{l+s+s1}{Return}\PYG{l+s+s1}{\PYGZsq{}}\PYG{p}{,} \PYG{l+s+s1}{\PYGZsq{}}\PYG{l+s+s1}{Strategy}\PYG{l+s+s1}{\PYGZsq{}}\PYG{p}{]}\PYG{p}{]}\PYG{o}{.}\PYG{n}{dropna}\PYG{p}{(}\PYG{p}{)}

\PYG{p}{(}
    \PYG{n}{\PYGZus{}}
    \PYG{o}{.}\PYG{n}{add}\PYG{p}{(}\PYG{l+m+mi}{1}\PYG{p}{)}
    \PYG{o}{.}\PYG{n}{cumprod}\PYG{p}{(}\PYG{p}{)}
    \PYG{o}{.}\PYG{n}{rename\PYGZus{}axis}\PYG{p}{(}\PYG{n}{columns}\PYG{o}{=}\PYG{l+s+s1}{\PYGZsq{}}\PYG{l+s+s1}{Strategy}\PYG{l+s+s1}{\PYGZsq{}}\PYG{p}{)}
    \PYG{o}{.}\PYG{n}{rename}\PYG{p}{(}\PYG{n}{columns}\PYG{o}{=}\PYG{p}{\PYGZob{}}\PYG{l+s+s1}{\PYGZsq{}}\PYG{l+s+s1}{Return}\PYG{l+s+s1}{\PYGZsq{}}\PYG{p}{:} \PYG{l+s+s1}{\PYGZsq{}}\PYG{l+s+s1}{Buy\PYGZhy{}And\PYGZhy{}Hold}\PYG{l+s+s1}{\PYGZsq{}}\PYG{p}{,} \PYG{l+s+s1}{\PYGZsq{}}\PYG{l+s+s1}{Strategy}\PYG{l+s+s1}{\PYGZsq{}}\PYG{p}{:} \PYG{l+s+s1}{\PYGZsq{}}\PYG{l+s+s1}{SMA(20)}\PYG{l+s+s1}{\PYGZsq{}}\PYG{p}{\PYGZcb{}}\PYG{p}{)}
    \PYG{o}{.}\PYG{n}{plot}\PYG{p}{(}\PYG{p}{)}
\PYG{p}{)}
\PYG{n}{plt}\PYG{o}{.}\PYG{n}{ylabel}\PYG{p}{(}\PYG{l+s+s1}{\PYGZsq{}}\PYG{l+s+s1}{Value (\PYGZdl{})}\PYG{l+s+s1}{\PYGZsq{}}\PYG{p}{)}
\PYG{n}{plt}\PYG{o}{.}\PYG{n}{title}\PYG{p}{(}\PYG{l+s+sa}{f}\PYG{l+s+s1}{\PYGZsq{}}\PYG{l+s+s1}{Value of \PYGZdl{}1 Invested at Close on }\PYG{l+s+si}{\PYGZob{}}\PYG{n}{\PYGZus{}}\PYG{o}{.}\PYG{n}{index}\PYG{p}{[}\PYG{l+m+mi}{0}\PYG{p}{]} \PYG{o}{\PYGZhy{}} \PYG{n}{pd}\PYG{o}{.}\PYG{n}{offsets}\PYG{o}{.}\PYG{n}{Day}\PYG{p}{(}\PYG{l+m+mi}{1}\PYG{p}{)}\PYG{l+s+si}{:}\PYG{l+s+s1}{\PYGZpc{}B \PYGZpc{}d, \PYGZpc{}Y}\PYG{l+s+si}{\PYGZcb{}}\PYG{l+s+s1}{\PYGZsq{}}\PYG{p}{)}
\PYG{n}{plt}\PYG{o}{.}\PYG{n}{show}\PYG{p}{(}\PYG{p}{)}
\end{sphinxVerbatim}

\end{sphinxuseclass}\end{sphinxVerbatimInput}
\begin{sphinxVerbatimOutput}

\begin{sphinxuseclass}{cell_output}
\noindent\sphinxincludegraphics{{0465375d524760e8f7e6f5cf74e7afd4fb8b2e32201f977ff1caf134a14848b0}.png}

\end{sphinxuseclass}\end{sphinxVerbatimOutput}

\end{sphinxuseclass}

\subsubsection{How does SMA(20) outperform buy\sphinxhyphen{}and\sphinxhyphen{}hold with this sample?}
\label{\detokenize{herron_02_practice_04:how-does-sma-20-outperform-buy-and-hold-with-this-sample}}
\sphinxAtStartPar
Consider the following:
\begin{enumerate}
\sphinxsetlistlabels{\arabic}{enumi}{enumii}{}{.}%
\item {} 
\sphinxAtStartPar
Does SMA(20) avoid the worst performing days? How many of the worst 20 days does SMA(20) avoid? Try the \sphinxcode{\sphinxupquote{.sort\_values()}} or \sphinxcode{\sphinxupquote{.nlargest()}} method.

\item {} 
\sphinxAtStartPar
Does SMA(20) preferentially avoid low\sphinxhyphen{}return days? Try to combine the \sphinxcode{\sphinxupquote{.groupby()}} method and \sphinxcode{\sphinxupquote{pd.qcut()}} function.

\item {} 
\sphinxAtStartPar
Does SMA(20) preferentially avoid high\sphinxhyphen{}volatility days? Try to combine the \sphinxcode{\sphinxupquote{.groupby()}} method and \sphinxcode{\sphinxupquote{pd.qcut()}} function.

\end{enumerate}

\sphinxAtStartPar
By chance, the SMA(20) strategy avoids all but three of the worst days.

\begin{sphinxuseclass}{cell}\begin{sphinxVerbatimInput}

\begin{sphinxuseclass}{cell_input}
\begin{sphinxVerbatim}[commandchars=\\\{\}]
\PYG{n}{btc\PYGZus{}sma}\PYG{o}{.}\PYG{n}{sort\PYGZus{}values}\PYG{p}{(}\PYG{l+s+s1}{\PYGZsq{}}\PYG{l+s+s1}{Return}\PYG{l+s+s1}{\PYGZsq{}}\PYG{p}{)}\PYG{p}{[}\PYG{p}{[}\PYG{l+s+s1}{\PYGZsq{}}\PYG{l+s+s1}{Position}\PYG{l+s+s1}{\PYGZsq{}}\PYG{p}{]}\PYG{p}{]}\PYG{o}{.}\PYG{n}{head}\PYG{p}{(}\PYG{l+m+mi}{20}\PYG{p}{)}\PYG{o}{.}\PYG{n}{value\PYGZus{}counts}\PYG{p}{(}\PYG{p}{)}
\end{sphinxVerbatim}

\end{sphinxuseclass}\end{sphinxVerbatimInput}
\begin{sphinxVerbatimOutput}

\begin{sphinxuseclass}{cell_output}
\begin{sphinxVerbatim}[commandchars=\\\{\}]
Position
0.0000      17
1.0000       3
dtype: int64
\end{sphinxVerbatim}

\end{sphinxuseclass}\end{sphinxVerbatimOutput}

\end{sphinxuseclass}
\sphinxAtStartPar
However, SMA(20) does not avoid the best days, again by chance.

\begin{sphinxuseclass}{cell}\begin{sphinxVerbatimInput}

\begin{sphinxuseclass}{cell_input}
\begin{sphinxVerbatim}[commandchars=\\\{\}]
\PYG{n}{btc\PYGZus{}sma}\PYG{o}{.}\PYG{n}{sort\PYGZus{}values}\PYG{p}{(}\PYG{l+s+s1}{\PYGZsq{}}\PYG{l+s+s1}{Return}\PYG{l+s+s1}{\PYGZsq{}}\PYG{p}{,} \PYG{n}{ascending}\PYG{o}{=}\PYG{k+kc}{False}\PYG{p}{)}\PYG{p}{[}\PYG{p}{[}\PYG{l+s+s1}{\PYGZsq{}}\PYG{l+s+s1}{Position}\PYG{l+s+s1}{\PYGZsq{}}\PYG{p}{]}\PYG{p}{]}\PYG{o}{.}\PYG{n}{head}\PYG{p}{(}\PYG{l+m+mi}{20}\PYG{p}{)}\PYG{o}{.}\PYG{n}{value\PYGZus{}counts}\PYG{p}{(}\PYG{p}{)}
\end{sphinxVerbatim}

\end{sphinxuseclass}\end{sphinxVerbatimInput}
\begin{sphinxVerbatimOutput}

\begin{sphinxuseclass}{cell_output}
\begin{sphinxVerbatim}[commandchars=\\\{\}]
Position
0.0000      10
1.0000      10
dtype: int64
\end{sphinxVerbatim}

\end{sphinxuseclass}\end{sphinxVerbatimOutput}

\end{sphinxuseclass}
\sphinxAtStartPar
The SMA(20) strategy has a slight edge in picking high\sphinxhyphen{}return days, again by chance.

\begin{sphinxuseclass}{cell}\begin{sphinxVerbatimInput}

\begin{sphinxuseclass}{cell_input}
\begin{sphinxVerbatim}[commandchars=\\\{\}]
\PYG{p}{(}
    \PYG{n}{btc\PYGZus{}sma}
    \PYG{o}{.}\PYG{n}{assign}\PYG{p}{(}\PYG{n}{q5\PYGZus{}return} \PYG{o}{=} \PYG{k}{lambda} \PYG{n}{x}\PYG{p}{:} \PYG{l+m+mi}{1} \PYG{o}{+} \PYG{n}{pd}\PYG{o}{.}\PYG{n}{qcut}\PYG{p}{(}\PYG{n}{x}\PYG{p}{[}\PYG{l+s+s1}{\PYGZsq{}}\PYG{l+s+s1}{Return}\PYG{l+s+s1}{\PYGZsq{}}\PYG{p}{]}\PYG{p}{,} \PYG{n}{q}\PYG{o}{=}\PYG{l+m+mi}{5}\PYG{p}{,} \PYG{n}{labels}\PYG{o}{=}\PYG{k+kc}{False}\PYG{p}{)}\PYG{p}{)}
    \PYG{o}{.}\PYG{n}{groupby}\PYG{p}{(}\PYG{l+s+s1}{\PYGZsq{}}\PYG{l+s+s1}{q5\PYGZus{}return}\PYG{l+s+s1}{\PYGZsq{}}\PYG{p}{)}
    \PYG{p}{[}\PYG{l+s+s1}{\PYGZsq{}}\PYG{l+s+s1}{Position}\PYG{l+s+s1}{\PYGZsq{}}\PYG{p}{]}
    \PYG{o}{.}\PYG{n}{mean}\PYG{p}{(}\PYG{p}{)}
    \PYG{o}{.}\PYG{n}{plot}\PYG{p}{(}\PYG{n}{kind}\PYG{o}{=}\PYG{l+s+s1}{\PYGZsq{}}\PYG{l+s+s1}{bar}\PYG{l+s+s1}{\PYGZsq{}}\PYG{p}{)}
\PYG{p}{)}

\PYG{n}{plt}\PYG{o}{.}\PYG{n}{xticks}\PYG{p}{(}\PYG{n}{rotation}\PYG{o}{=}\PYG{l+m+mi}{0}\PYG{p}{)}
\PYG{n}{plt}\PYG{o}{.}\PYG{n}{xlabel}\PYG{p}{(}\PYG{l+s+s1}{\PYGZsq{}}\PYG{l+s+s1}{Return Bin (1 is Lowest, 5 is Highest)}\PYG{l+s+s1}{\PYGZsq{}}\PYG{p}{)}
\PYG{n}{plt}\PYG{o}{.}\PYG{n}{ylabel}\PYG{p}{(}\PYG{l+s+s1}{\PYGZsq{}}\PYG{l+s+s1}{Fraction of Days Strategy is Long Bitcoin}\PYG{l+s+s1}{\PYGZsq{}}\PYG{p}{)}
\PYG{n}{plt}\PYG{o}{.}\PYG{n}{title}\PYG{p}{(}\PYG{l+s+s1}{\PYGZsq{}}\PYG{l+s+s1}{Mean Position by Return Bin}\PYG{l+s+s1}{\PYGZsq{}}\PYG{p}{)}
\PYG{n}{plt}\PYG{o}{.}\PYG{n}{show}\PYG{p}{(}\PYG{p}{)}
\end{sphinxVerbatim}

\end{sphinxuseclass}\end{sphinxVerbatimInput}
\begin{sphinxVerbatimOutput}

\begin{sphinxuseclass}{cell_output}
\noindent\sphinxincludegraphics{{77fcce6036e279a25a7c191e860fc4caa4bd9ab43c234b7935bfcdb356dabc6a}.png}

\end{sphinxuseclass}\end{sphinxVerbatimOutput}

\end{sphinxuseclass}
\sphinxAtStartPar
However, the SMA(20) \sphinxstyleemphasis{does} avoid the high volatility days that create \sphinxhref{https://www.kitces.com/blog/volatility-drag-variance-drain-mean-arithmetic-vs-geometric-average-investment-returns/}{volatility drag}.

\begin{sphinxuseclass}{cell}\begin{sphinxVerbatimInput}

\begin{sphinxuseclass}{cell_input}
\begin{sphinxVerbatim}[commandchars=\\\{\}]
\PYG{p}{(}
    \PYG{n}{btc\PYGZus{}sma}
    \PYG{o}{.}\PYG{n}{assign}\PYG{p}{(}
        \PYG{n}{Volatility} \PYG{o}{=} \PYG{k}{lambda} \PYG{n}{x}\PYG{p}{:} \PYG{n}{x}\PYG{p}{[}\PYG{l+s+s1}{\PYGZsq{}}\PYG{l+s+s1}{Return}\PYG{l+s+s1}{\PYGZsq{}}\PYG{p}{]}\PYG{o}{.}\PYG{n}{rolling}\PYG{p}{(}\PYG{l+m+mi}{63}\PYG{p}{)}\PYG{o}{.}\PYG{n}{std}\PYG{p}{(}\PYG{p}{)}\PYG{p}{,}
        \PYG{n}{q5\PYGZus{}volatility} \PYG{o}{=} \PYG{k}{lambda} \PYG{n}{x}\PYG{p}{:} \PYG{l+m+mi}{1} \PYG{o}{+} \PYG{n}{pd}\PYG{o}{.}\PYG{n}{qcut}\PYG{p}{(}\PYG{n}{x}\PYG{p}{[}\PYG{l+s+s1}{\PYGZsq{}}\PYG{l+s+s1}{Volatility}\PYG{l+s+s1}{\PYGZsq{}}\PYG{p}{]}\PYG{p}{,} \PYG{n}{q}\PYG{o}{=}\PYG{l+m+mi}{5}\PYG{p}{,} \PYG{n}{labels}\PYG{o}{=}\PYG{k+kc}{False}\PYG{p}{)}
    \PYG{p}{)}
    \PYG{o}{.}\PYG{n}{groupby}\PYG{p}{(}\PYG{l+s+s1}{\PYGZsq{}}\PYG{l+s+s1}{q5\PYGZus{}volatility}\PYG{l+s+s1}{\PYGZsq{}}\PYG{p}{)}
    \PYG{p}{[}\PYG{l+s+s1}{\PYGZsq{}}\PYG{l+s+s1}{Position}\PYG{l+s+s1}{\PYGZsq{}}\PYG{p}{]}
    \PYG{o}{.}\PYG{n}{mean}\PYG{p}{(}\PYG{p}{)}
    \PYG{o}{.}\PYG{n}{plot}\PYG{p}{(}\PYG{n}{kind}\PYG{o}{=}\PYG{l+s+s1}{\PYGZsq{}}\PYG{l+s+s1}{bar}\PYG{l+s+s1}{\PYGZsq{}}\PYG{p}{)}
\PYG{p}{)}

\PYG{n}{plt}\PYG{o}{.}\PYG{n}{xticks}\PYG{p}{(}\PYG{n}{rotation}\PYG{o}{=}\PYG{l+m+mi}{0}\PYG{p}{)}
\PYG{n}{plt}\PYG{o}{.}\PYG{n}{xlabel}\PYG{p}{(}\PYG{l+s+s1}{\PYGZsq{}}\PYG{l+s+s1}{63\PYGZhy{}Day Rolling Volatility Bin (1 is Lowest, 5 is Highest)}\PYG{l+s+s1}{\PYGZsq{}}\PYG{p}{)}
\PYG{n}{plt}\PYG{o}{.}\PYG{n}{ylabel}\PYG{p}{(}\PYG{l+s+s1}{\PYGZsq{}}\PYG{l+s+s1}{Fraction of Days Strategy is Long Bitcoin}\PYG{l+s+s1}{\PYGZsq{}}\PYG{p}{)}
\PYG{n}{plt}\PYG{o}{.}\PYG{n}{title}\PYG{p}{(}\PYG{l+s+s1}{\PYGZsq{}}\PYG{l+s+s1}{Mean Position by 63\PYGZhy{}Day Rolling Volatility Bin}\PYG{l+s+s1}{\PYGZsq{}}\PYG{p}{)}
\PYG{n}{plt}\PYG{o}{.}\PYG{n}{show}\PYG{p}{(}\PYG{p}{)}
\end{sphinxVerbatim}

\end{sphinxuseclass}\end{sphinxVerbatimInput}
\begin{sphinxVerbatimOutput}

\begin{sphinxuseclass}{cell_output}
\noindent\sphinxincludegraphics{{63059ddf0e219dc18f519507934fef116e57ea00861bca604dcdc2281f99d135}.png}

\end{sphinxuseclass}\end{sphinxVerbatimOutput}

\end{sphinxuseclass}
\sphinxAtStartPar
Recall that \$Arith\textbackslash{} Mean \textbackslash{}approx Geom\textbackslash{} Mean + \textbackslash{}frac\{\textbackslash{}sigma\textasciicircum{}2\}\{2\}\$, so avoiding high volatility (high variance) days, reduced the drag on the  cumulative returns that intermediate\sphinxhyphen{}term and long\sphinxhyphen{}term investors care about!

\begin{sphinxuseclass}{cell}\begin{sphinxVerbatimInput}

\begin{sphinxuseclass}{cell_input}
\begin{sphinxVerbatim}[commandchars=\\\{\}]
\PYG{p}{(}
    \PYG{n}{btc\PYGZus{}sma}
    \PYG{o}{.}\PYG{n}{groupby}\PYG{p}{(}\PYG{l+s+s1}{\PYGZsq{}}\PYG{l+s+s1}{Position}\PYG{l+s+s1}{\PYGZsq{}}\PYG{p}{)}
    \PYG{p}{[}\PYG{l+s+s1}{\PYGZsq{}}\PYG{l+s+s1}{Return}\PYG{l+s+s1}{\PYGZsq{}}\PYG{p}{]}
    \PYG{o}{.}\PYG{n}{agg}\PYG{p}{(}\PYG{p}{[}\PYG{l+s+s1}{\PYGZsq{}}\PYG{l+s+s1}{std}\PYG{l+s+s1}{\PYGZsq{}}\PYG{p}{,} \PYG{l+s+s1}{\PYGZsq{}}\PYG{l+s+s1}{mean}\PYG{l+s+s1}{\PYGZsq{}}\PYG{p}{,} \PYG{k}{lambda} \PYG{n}{x}\PYG{p}{:} \PYG{p}{(}\PYG{l+m+mi}{1} \PYG{o}{+} \PYG{n}{x}\PYG{p}{)}\PYG{o}{.}\PYG{n}{prod}\PYG{p}{(}\PYG{p}{)}\PYG{o}{*}\PYG{o}{*}\PYG{p}{(}\PYG{l+m+mi}{1} \PYG{o}{/} \PYG{n}{x}\PYG{o}{.}\PYG{n}{count}\PYG{p}{(}\PYG{p}{)}\PYG{p}{)} \PYG{o}{\PYGZhy{}} \PYG{l+m+mi}{1}\PYG{p}{]}\PYG{p}{)}
    \PYG{o}{.}\PYG{n}{mul}\PYG{p}{(}\PYG{l+m+mi}{100}\PYG{p}{)}
    \PYG{o}{.}\PYG{n}{rename}\PYG{p}{(}\PYG{n}{columns}\PYG{o}{=}\PYG{p}{\PYGZob{}}\PYG{l+s+s1}{\PYGZsq{}}\PYG{l+s+s1}{std}\PYG{l+s+s1}{\PYGZsq{}}\PYG{p}{:} \PYG{l+s+s1}{\PYGZsq{}}\PYG{l+s+s1}{Volatility}\PYG{l+s+s1}{\PYGZsq{}}\PYG{p}{,} \PYG{l+s+s1}{\PYGZsq{}}\PYG{l+s+s1}{mean}\PYG{l+s+s1}{\PYGZsq{}}\PYG{p}{:} \PYG{l+s+s1}{\PYGZsq{}}\PYG{l+s+s1}{Arith Mean}\PYG{l+s+s1}{\PYGZsq{}}\PYG{p}{,} \PYG{l+s+s1}{\PYGZsq{}}\PYG{l+s+s1}{\PYGZlt{}lambda\PYGZus{}0\PYGZgt{}}\PYG{l+s+s1}{\PYGZsq{}}\PYG{p}{:} \PYG{l+s+s1}{\PYGZsq{}}\PYG{l+s+s1}{Geom Mean}\PYG{l+s+s1}{\PYGZsq{}}\PYG{p}{\PYGZcb{}}\PYG{p}{)}
\PYG{p}{)}
\end{sphinxVerbatim}

\end{sphinxuseclass}\end{sphinxVerbatimInput}
\begin{sphinxVerbatimOutput}

\begin{sphinxuseclass}{cell_output}
\begin{sphinxVerbatim}[commandchars=\\\{\}]
          Volatility  Arith Mean  Geom Mean
Position                                   
0.0000        4.1518      0.0433    \PYGZhy{}0.0452
1.0000        3.4901      0.3553     0.2952
\end{sphinxVerbatim}

\end{sphinxuseclass}\end{sphinxVerbatimOutput}

\end{sphinxuseclass}

\subsubsection{Implement the SMA(20) strategy with the market factor from French}
\label{\detokenize{herron_02_practice_04:implement-the-sma-20-strategy-with-the-market-factor-from-french}}
\sphinxAtStartPar
We need to impute a market price before we calculate SMA(20).

\begin{sphinxuseclass}{cell}\begin{sphinxVerbatimInput}

\begin{sphinxuseclass}{cell_input}
\begin{sphinxVerbatim}[commandchars=\\\{\}]
\PYG{k+kn}{import} \PYG{n+nn}{pandas\PYGZus{}datareader} \PYG{k}{as} \PYG{n+nn}{pdr}
\PYG{k+kn}{import} \PYG{n+nn}{requests\PYGZus{}cache}
\PYG{n}{session} \PYG{o}{=} \PYG{n}{requests\PYGZus{}cache}\PYG{o}{.}\PYG{n}{CachedSession}\PYG{p}{(}\PYG{p}{)}
\end{sphinxVerbatim}

\end{sphinxuseclass}\end{sphinxVerbatimInput}

\end{sphinxuseclass}
\begin{sphinxuseclass}{cell}\begin{sphinxVerbatimInput}

\begin{sphinxuseclass}{cell_input}
\begin{sphinxVerbatim}[commandchars=\\\{\}]
\PYG{n}{ff} \PYG{o}{=} \PYG{p}{(}
    \PYG{n}{pdr}\PYG{o}{.}\PYG{n}{DataReader}\PYG{p}{(}
        \PYG{n}{name}\PYG{o}{=}\PYG{l+s+s1}{\PYGZsq{}}\PYG{l+s+s1}{F\PYGZhy{}F\PYGZus{}Research\PYGZus{}Data\PYGZus{}Factors\PYGZus{}daily}\PYG{l+s+s1}{\PYGZsq{}}\PYG{p}{,}
        \PYG{n}{data\PYGZus{}source}\PYG{o}{=}\PYG{l+s+s1}{\PYGZsq{}}\PYG{l+s+s1}{famafrench}\PYG{l+s+s1}{\PYGZsq{}}\PYG{p}{,}
        \PYG{n}{start}\PYG{o}{=}\PYG{l+s+s1}{\PYGZsq{}}\PYG{l+s+s1}{1900}\PYG{l+s+s1}{\PYGZsq{}}\PYG{p}{,}
        \PYG{n}{session}\PYG{o}{=}\PYG{n}{session}
    \PYG{p}{)}
    \PYG{p}{[}\PYG{l+m+mi}{0}\PYG{p}{]}
    \PYG{o}{.}\PYG{n}{div}\PYG{p}{(}\PYG{l+m+mi}{100}\PYG{p}{)}
    \PYG{o}{.}\PYG{n}{assign}\PYG{p}{(}
        \PYG{n}{Mkt} \PYG{o}{=} \PYG{k}{lambda} \PYG{n}{x}\PYG{p}{:} \PYG{n}{x}\PYG{p}{[}\PYG{l+s+s1}{\PYGZsq{}}\PYG{l+s+s1}{Mkt\PYGZhy{}RF}\PYG{l+s+s1}{\PYGZsq{}}\PYG{p}{]} \PYG{o}{+} \PYG{n}{x}\PYG{p}{[}\PYG{l+s+s1}{\PYGZsq{}}\PYG{l+s+s1}{RF}\PYG{l+s+s1}{\PYGZsq{}}\PYG{p}{]}\PYG{p}{,}
        \PYG{n}{Price} \PYG{o}{=} \PYG{k}{lambda} \PYG{n}{x}\PYG{p}{:} \PYG{n}{x}\PYG{p}{[}\PYG{l+s+s1}{\PYGZsq{}}\PYG{l+s+s1}{Mkt}\PYG{l+s+s1}{\PYGZsq{}}\PYG{p}{]}\PYG{o}{.}\PYG{n}{add}\PYG{p}{(}\PYG{l+m+mi}{1}\PYG{p}{)}\PYG{o}{.}\PYG{n}{cumprod}\PYG{p}{(}\PYG{p}{)}
    \PYG{p}{)}
\PYG{p}{)}
\end{sphinxVerbatim}

\end{sphinxuseclass}\end{sphinxVerbatimInput}

\end{sphinxuseclass}
\begin{sphinxuseclass}{cell}\begin{sphinxVerbatimInput}

\begin{sphinxuseclass}{cell_input}
\begin{sphinxVerbatim}[commandchars=\\\{\}]
\PYG{n}{ff\PYGZus{}sma} \PYG{o}{=} \PYG{p}{(}
    \PYG{n}{ff}
    \PYG{o}{.}\PYG{n}{rename}\PYG{p}{(}\PYG{n}{columns}\PYG{o}{=}\PYG{p}{\PYGZob{}}\PYG{l+s+s1}{\PYGZsq{}}\PYG{l+s+s1}{Price}\PYG{l+s+s1}{\PYGZsq{}}\PYG{p}{:} \PYG{l+s+s1}{\PYGZsq{}}\PYG{l+s+s1}{Adj Close}\PYG{l+s+s1}{\PYGZsq{}}\PYG{p}{\PYGZcb{}}\PYG{p}{)}
    \PYG{o}{.}\PYG{n}{pipe}\PYG{p}{(}\PYG{n}{sma}\PYG{p}{,} \PYG{n}{n}\PYG{o}{=}\PYG{l+m+mi}{20}\PYG{p}{)}
\PYG{p}{)}

\PYG{n}{ff\PYGZus{}sma}\PYG{o}{.}\PYG{n}{tail}\PYG{p}{(}\PYG{p}{)}
\end{sphinxVerbatim}

\end{sphinxuseclass}\end{sphinxVerbatimInput}
\begin{sphinxVerbatimOutput}

\begin{sphinxuseclass}{cell_output}
\begin{sphinxVerbatim}[commandchars=\\\{\}]
            Mkt\PYGZhy{}RF     SMB     HML     RF     Mkt  Adj Close  Return  \PYGZbs{}
Date                                                                   
2022\PYGZhy{}12\PYGZhy{}23  0.0051 \PYGZhy{}0.0060  0.0115 0.0002  0.0053  9266.3996  0.0053   
2022\PYGZhy{}12\PYGZhy{}27 \PYGZhy{}0.0051 \PYGZhy{}0.0073  0.0142 0.0002 \PYGZhy{}0.0049  9220.6236 \PYGZhy{}0.0049   
2022\PYGZhy{}12\PYGZhy{}28 \PYGZhy{}0.0123 \PYGZhy{}0.0025 \PYGZhy{}0.0029 0.0002 \PYGZhy{}0.0121  9108.6852 \PYGZhy{}0.0121   
2022\PYGZhy{}12\PYGZhy{}29  0.0187  0.0127 \PYGZhy{}0.0107 0.0002  0.0189  9280.4750  0.0189   
2022\PYGZhy{}12\PYGZhy{}30 \PYGZhy{}0.0022  0.0011 \PYGZhy{}0.0003 0.0002 \PYGZhy{}0.0020  9261.5429 \PYGZhy{}0.0020   

                 SMA  Position  Strategy  
Date                                      
2022\PYGZhy{}12\PYGZhy{}23 9515.5406    0.0000    0.0000  
2022\PYGZhy{}12\PYGZhy{}27 9497.8127    0.0000   \PYGZhy{}0.0000  
2022\PYGZhy{}12\PYGZhy{}28 9475.2827    0.0000   \PYGZhy{}0.0000  
2022\PYGZhy{}12\PYGZhy{}29 9446.3627    0.0000    0.0000  
2022\PYGZhy{}12\PYGZhy{}30 9416.5159    0.0000   \PYGZhy{}0.0000  
\end{sphinxVerbatim}

\end{sphinxuseclass}\end{sphinxVerbatimOutput}

\end{sphinxuseclass}
\begin{sphinxuseclass}{cell}\begin{sphinxVerbatimInput}

\begin{sphinxuseclass}{cell_input}
\begin{sphinxVerbatim}[commandchars=\\\{\}]
\PYG{n}{\PYGZus{}} \PYG{o}{=} \PYG{n}{ff\PYGZus{}sma}\PYG{p}{[}\PYG{p}{[}\PYG{l+s+s1}{\PYGZsq{}}\PYG{l+s+s1}{Return}\PYG{l+s+s1}{\PYGZsq{}}\PYG{p}{,} \PYG{l+s+s1}{\PYGZsq{}}\PYG{l+s+s1}{Strategy}\PYG{l+s+s1}{\PYGZsq{}}\PYG{p}{]}\PYG{p}{]}\PYG{o}{.}\PYG{n}{dropna}\PYG{p}{(}\PYG{p}{)}

\PYG{p}{(}
    \PYG{n}{\PYGZus{}}
    \PYG{o}{.}\PYG{n}{add}\PYG{p}{(}\PYG{l+m+mi}{1}\PYG{p}{)}
    \PYG{o}{.}\PYG{n}{cumprod}\PYG{p}{(}\PYG{p}{)}
    \PYG{o}{.}\PYG{n}{rename\PYGZus{}axis}\PYG{p}{(}\PYG{n}{columns}\PYG{o}{=}\PYG{l+s+s1}{\PYGZsq{}}\PYG{l+s+s1}{Strategy}\PYG{l+s+s1}{\PYGZsq{}}\PYG{p}{)}
    \PYG{o}{.}\PYG{n}{rename}\PYG{p}{(}\PYG{n}{columns}\PYG{o}{=}\PYG{p}{\PYGZob{}}\PYG{l+s+s1}{\PYGZsq{}}\PYG{l+s+s1}{Return}\PYG{l+s+s1}{\PYGZsq{}}\PYG{p}{:} \PYG{l+s+s1}{\PYGZsq{}}\PYG{l+s+s1}{Buy\PYGZhy{}And\PYGZhy{}Hold}\PYG{l+s+s1}{\PYGZsq{}}\PYG{p}{,} \PYG{l+s+s1}{\PYGZsq{}}\PYG{l+s+s1}{Strategy}\PYG{l+s+s1}{\PYGZsq{}}\PYG{p}{:} \PYG{l+s+s1}{\PYGZsq{}}\PYG{l+s+s1}{SMA(20)}\PYG{l+s+s1}{\PYGZsq{}}\PYG{p}{\PYGZcb{}}\PYG{p}{)}
    \PYG{o}{.}\PYG{n}{plot}\PYG{p}{(}\PYG{p}{)}
\PYG{p}{)}
\PYG{n}{plt}\PYG{o}{.}\PYG{n}{ylabel}\PYG{p}{(}\PYG{l+s+s1}{\PYGZsq{}}\PYG{l+s+s1}{Value (\PYGZdl{})}\PYG{l+s+s1}{\PYGZsq{}}\PYG{p}{)}
\PYG{n}{plt}\PYG{o}{.}\PYG{n}{title}\PYG{p}{(}\PYG{l+s+sa}{f}\PYG{l+s+s1}{\PYGZsq{}}\PYG{l+s+s1}{Value of \PYGZdl{}1 Invested in Market at Close on }\PYG{l+s+si}{\PYGZob{}}\PYG{n}{\PYGZus{}}\PYG{o}{.}\PYG{n}{index}\PYG{p}{[}\PYG{l+m+mi}{0}\PYG{p}{]} \PYG{o}{\PYGZhy{}} \PYG{n}{pd}\PYG{o}{.}\PYG{n}{offsets}\PYG{o}{.}\PYG{n}{Day}\PYG{p}{(}\PYG{l+m+mi}{1}\PYG{p}{)}\PYG{l+s+si}{:}\PYG{l+s+s1}{\PYGZpc{}B \PYGZpc{}d, \PYGZpc{}Y}\PYG{l+s+si}{\PYGZcb{}}\PYG{l+s+s1}{\PYGZsq{}}\PYG{p}{)}
\PYG{n}{plt}\PYG{o}{.}\PYG{n}{show}\PYG{p}{(}\PYG{p}{)}
\end{sphinxVerbatim}

\end{sphinxuseclass}\end{sphinxVerbatimInput}
\begin{sphinxVerbatimOutput}

\begin{sphinxuseclass}{cell_output}
\noindent\sphinxincludegraphics{{131e3565948329ffbfb65b3360199ffe069c47f7c99b79b2edd04f4debb4d99e}.png}

\end{sphinxuseclass}\end{sphinxVerbatimOutput}

\end{sphinxuseclass}

\subsubsection{How often does SMA(20) outperform buy\sphinxhyphen{}and\sphinxhyphen{}hold with 10\sphinxhyphen{}year rolling windows?}
\label{\detokenize{herron_02_practice_04:how-often-does-sma-20-outperform-buy-and-hold-with-10-year-rolling-windows}}
\begin{sphinxuseclass}{cell}\begin{sphinxVerbatimInput}

\begin{sphinxuseclass}{cell_input}
\begin{sphinxVerbatim}[commandchars=\\\{\}]
\PYG{p}{(}
    \PYG{n}{ff\PYGZus{}sma}
    \PYG{p}{[}\PYG{p}{[}\PYG{l+s+s1}{\PYGZsq{}}\PYG{l+s+s1}{Return}\PYG{l+s+s1}{\PYGZsq{}}\PYG{p}{,} \PYG{l+s+s1}{\PYGZsq{}}\PYG{l+s+s1}{Strategy}\PYG{l+s+s1}{\PYGZsq{}}\PYG{p}{]}\PYG{p}{]}
    \PYG{o}{.}\PYG{n}{rolling}\PYG{p}{(}\PYG{l+m+mi}{10} \PYG{o}{*} \PYG{l+m+mi}{252}\PYG{p}{)}
    \PYG{o}{.}\PYG{n}{apply}\PYG{p}{(}\PYG{k}{lambda} \PYG{n}{x}\PYG{p}{:} \PYG{p}{(}\PYG{l+m+mi}{1} \PYG{o}{+} \PYG{n}{x}\PYG{p}{)}\PYG{o}{.}\PYG{n}{prod}\PYG{p}{(}\PYG{p}{)}\PYG{p}{)}
    \PYG{o}{.}\PYG{n}{rename\PYGZus{}axis}\PYG{p}{(}\PYG{n}{columns}\PYG{o}{=}\PYG{l+s+s1}{\PYGZsq{}}\PYG{l+s+s1}{Strategy}\PYG{l+s+s1}{\PYGZsq{}}\PYG{p}{)}
    \PYG{o}{.}\PYG{n}{rename}\PYG{p}{(}\PYG{n}{columns}\PYG{o}{=}\PYG{p}{\PYGZob{}}\PYG{l+s+s1}{\PYGZsq{}}\PYG{l+s+s1}{Return}\PYG{l+s+s1}{\PYGZsq{}}\PYG{p}{:} \PYG{l+s+s1}{\PYGZsq{}}\PYG{l+s+s1}{Buy\PYGZhy{}And\PYGZhy{}Hold}\PYG{l+s+s1}{\PYGZsq{}}\PYG{p}{,} \PYG{l+s+s1}{\PYGZsq{}}\PYG{l+s+s1}{Strategy}\PYG{l+s+s1}{\PYGZsq{}}\PYG{p}{:} \PYG{l+s+s1}{\PYGZsq{}}\PYG{l+s+s1}{SMA(20)}\PYG{l+s+s1}{\PYGZsq{}}\PYG{p}{\PYGZcb{}}\PYG{p}{)}
    \PYG{o}{.}\PYG{n}{plot}\PYG{p}{(}\PYG{p}{)}
\PYG{p}{)}
\PYG{n}{plt}\PYG{o}{.}\PYG{n}{ylabel}\PYG{p}{(}\PYG{l+s+s1}{\PYGZsq{}}\PYG{l+s+s1}{Value (\PYGZdl{})}\PYG{l+s+s1}{\PYGZsq{}}\PYG{p}{)}
\PYG{n}{plt}\PYG{o}{.}\PYG{n}{title}\PYG{p}{(}\PYG{l+s+sa}{f}\PYG{l+s+s1}{\PYGZsq{}}\PYG{l+s+s1}{Value of \PYGZdl{}1 Investments for Rolling 10\PYGZhy{}Year Holding Periods }\PYG{l+s+s1}{\PYGZsq{}}\PYG{p}{)}
\PYG{n}{plt}\PYG{o}{.}\PYG{n}{show}\PYG{p}{(}\PYG{p}{)}
\end{sphinxVerbatim}

\end{sphinxuseclass}\end{sphinxVerbatimInput}
\begin{sphinxVerbatimOutput}

\begin{sphinxuseclass}{cell_output}
\noindent\sphinxincludegraphics{{9f8dc1a1b39e0f46b1d4fda08ef1b2a8f7ae3e2ae45f32a52dcfe96994401f08}.png}

\end{sphinxuseclass}\end{sphinxVerbatimOutput}

\end{sphinxuseclass}
\sphinxAtStartPar
In the previous example, SMA(20) looks amazing!
But over many shorter holding periods, we see the two are comparable.
This is largely because the SMA(20) strategy \sphinxstyleemphasis{by pure chance} avoids big market draw downs!

\begin{sphinxuseclass}{cell}\begin{sphinxVerbatimInput}

\begin{sphinxuseclass}{cell_input}
\begin{sphinxVerbatim}[commandchars=\\\{\}]
\PYG{n}{ff\PYGZus{}sma}\PYG{o}{.}\PYG{n}{sort\PYGZus{}values}\PYG{p}{(}\PYG{l+s+s1}{\PYGZsq{}}\PYG{l+s+s1}{Return}\PYG{l+s+s1}{\PYGZsq{}}\PYG{p}{)}\PYG{p}{[}\PYG{p}{[}\PYG{l+s+s1}{\PYGZsq{}}\PYG{l+s+s1}{Position}\PYG{l+s+s1}{\PYGZsq{}}\PYG{p}{,} \PYG{l+s+s1}{\PYGZsq{}}\PYG{l+s+s1}{Return}\PYG{l+s+s1}{\PYGZsq{}}\PYG{p}{,} \PYG{l+s+s1}{\PYGZsq{}}\PYG{l+s+s1}{Strategy}\PYG{l+s+s1}{\PYGZsq{}}\PYG{p}{]}\PYG{p}{]}\PYG{o}{.}\PYG{n}{head}\PYG{p}{(}\PYG{l+m+mi}{10}\PYG{p}{)}
\end{sphinxVerbatim}

\end{sphinxuseclass}\end{sphinxVerbatimInput}
\begin{sphinxVerbatimOutput}

\begin{sphinxuseclass}{cell_output}
\begin{sphinxVerbatim}[commandchars=\\\{\}]
            Position  Return  Strategy
Date                                  
1987\PYGZhy{}10\PYGZhy{}19    0.0000 \PYGZhy{}0.1741   \PYGZhy{}0.0000
2020\PYGZhy{}03\PYGZhy{}16    0.0000 \PYGZhy{}0.1199   \PYGZhy{}0.0000
1929\PYGZhy{}10\PYGZhy{}29    0.0000 \PYGZhy{}0.1199   \PYGZhy{}0.0000
1929\PYGZhy{}10\PYGZhy{}28    0.0000 \PYGZhy{}0.1127   \PYGZhy{}0.0000
1929\PYGZhy{}11\PYGZhy{}06    0.0000 \PYGZhy{}0.0973   \PYGZhy{}0.0000
2020\PYGZhy{}03\PYGZhy{}12    0.0000 \PYGZhy{}0.0962   \PYGZhy{}0.0000
1933\PYGZhy{}07\PYGZhy{}21    0.0000 \PYGZhy{}0.0921   \PYGZhy{}0.0000
2008\PYGZhy{}12\PYGZhy{}01    1.0000 \PYGZhy{}0.0895   \PYGZhy{}0.0895
2008\PYGZhy{}10\PYGZhy{}15    0.0000 \PYGZhy{}0.0878   \PYGZhy{}0.0000
1933\PYGZhy{}07\PYGZhy{}20    1.0000 \PYGZhy{}0.0849   \PYGZhy{}0.0849
\end{sphinxVerbatim}

\end{sphinxuseclass}\end{sphinxVerbatimOutput}

\end{sphinxuseclass}
\sphinxAtStartPar
SMA(20) also avoids the up days.
However, for this sample, missing the extreme down days helps more than missing the extreme updays hurts.

\begin{sphinxuseclass}{cell}\begin{sphinxVerbatimInput}

\begin{sphinxuseclass}{cell_input}
\begin{sphinxVerbatim}[commandchars=\\\{\}]
\PYG{n}{ff\PYGZus{}sma}\PYG{o}{.}\PYG{n}{sort\PYGZus{}values}\PYG{p}{(}\PYG{l+s+s1}{\PYGZsq{}}\PYG{l+s+s1}{Return}\PYG{l+s+s1}{\PYGZsq{}}\PYG{p}{,} \PYG{n}{ascending}\PYG{o}{=}\PYG{k+kc}{False}\PYG{p}{)}\PYG{p}{[}\PYG{p}{[}\PYG{l+s+s1}{\PYGZsq{}}\PYG{l+s+s1}{Position}\PYG{l+s+s1}{\PYGZsq{}}\PYG{p}{,} \PYG{l+s+s1}{\PYGZsq{}}\PYG{l+s+s1}{Return}\PYG{l+s+s1}{\PYGZsq{}}\PYG{p}{,} \PYG{l+s+s1}{\PYGZsq{}}\PYG{l+s+s1}{Strategy}\PYG{l+s+s1}{\PYGZsq{}}\PYG{p}{]}\PYG{p}{]}\PYG{o}{.}\PYG{n}{head}\PYG{p}{(}\PYG{l+m+mi}{10}\PYG{p}{)}
\end{sphinxVerbatim}

\end{sphinxuseclass}\end{sphinxVerbatimInput}
\begin{sphinxVerbatimOutput}

\begin{sphinxuseclass}{cell_output}
\begin{sphinxVerbatim}[commandchars=\\\{\}]
            Position  Return  Strategy
Date                                  
1933\PYGZhy{}03\PYGZhy{}15    0.0000  0.1576    0.0000
1929\PYGZhy{}10\PYGZhy{}30    0.0000  0.1218    0.0000
2008\PYGZhy{}10\PYGZhy{}13    0.0000  0.1135    0.0000
1931\PYGZhy{}10\PYGZhy{}06    0.0000  0.1116    0.0000
1932\PYGZhy{}09\PYGZhy{}21    0.0000  0.1096    0.0000
2008\PYGZhy{}10\PYGZhy{}28    0.0000  0.0977    0.0000
2020\PYGZhy{}03\PYGZhy{}24    0.0000  0.0935    0.0000
2020\PYGZhy{}03\PYGZhy{}13    0.0000  0.0897    0.0000
1939\PYGZhy{}09\PYGZhy{}05    1.0000  0.0879    0.0879
1937\PYGZhy{}10\PYGZhy{}20    0.0000  0.0867    0.0000
\end{sphinxVerbatim}

\end{sphinxuseclass}\end{sphinxVerbatimOutput}

\end{sphinxuseclass}
\sphinxAtStartPar
We can also think about this problem by decade.
If we want to get proper calendar decades (instead of 10\sphinxhyphen{}year periods that start in 1926), we combine \sphinxcode{\sphinxupquote{.groupby()}} with an anonymous function that converts the date\sphinxhyphen{}time index to a proper calendar decade.
Again, we see that SMA(20) and buy\sphinxhyphen{}and\sphinxhyphen{}hold trade wins, but SMA(20) wins bigs in the 1930s!

\begin{sphinxuseclass}{cell}\begin{sphinxVerbatimInput}

\begin{sphinxuseclass}{cell_input}
\begin{sphinxVerbatim}[commandchars=\\\{\}]
\PYG{p}{(}
    \PYG{n}{ff\PYGZus{}sma}
    \PYG{p}{[}\PYG{p}{[}\PYG{l+s+s1}{\PYGZsq{}}\PYG{l+s+s1}{Return}\PYG{l+s+s1}{\PYGZsq{}}\PYG{p}{,} \PYG{l+s+s1}{\PYGZsq{}}\PYG{l+s+s1}{Strategy}\PYG{l+s+s1}{\PYGZsq{}}\PYG{p}{]}\PYG{p}{]}
    \PYG{o}{.}\PYG{n}{groupby}\PYG{p}{(}\PYG{k}{lambda} \PYG{n}{x}\PYG{p}{:} \PYG{l+s+sa}{f}\PYG{l+s+s1}{\PYGZsq{}}\PYG{l+s+si}{\PYGZob{}}\PYG{p}{(}\PYG{n}{x}\PYG{o}{.}\PYG{n}{year} \PYG{o}{/}\PYG{o}{/} \PYG{l+m+mi}{10}\PYG{p}{)} \PYG{o}{*} \PYG{l+m+mi}{10}\PYG{l+s+si}{\PYGZcb{}}\PYG{l+s+s1}{s}\PYG{l+s+s1}{\PYGZsq{}}\PYG{p}{)}
    \PYG{o}{.}\PYG{n}{apply}\PYG{p}{(}\PYG{k}{lambda} \PYG{n}{x}\PYG{p}{:} \PYG{p}{(}\PYG{l+m+mi}{1} \PYG{o}{+} \PYG{n}{x}\PYG{p}{)}\PYG{o}{.}\PYG{n}{prod}\PYG{p}{(}\PYG{p}{)}\PYG{p}{)}
    \PYG{o}{.}\PYG{n}{rename\PYGZus{}axis}\PYG{p}{(}\PYG{n}{index}\PYG{o}{=}\PYG{l+s+s1}{\PYGZsq{}}\PYG{l+s+s1}{Decade}\PYG{l+s+s1}{\PYGZsq{}}\PYG{p}{,} \PYG{n}{columns}\PYG{o}{=}\PYG{l+s+s1}{\PYGZsq{}}\PYG{l+s+s1}{Strategy}\PYG{l+s+s1}{\PYGZsq{}}\PYG{p}{)}
    \PYG{o}{.}\PYG{n}{rename}\PYG{p}{(}\PYG{n}{columns}\PYG{o}{=}\PYG{p}{\PYGZob{}}\PYG{l+s+s1}{\PYGZsq{}}\PYG{l+s+s1}{Return}\PYG{l+s+s1}{\PYGZsq{}}\PYG{p}{:} \PYG{l+s+s1}{\PYGZsq{}}\PYG{l+s+s1}{Buy\PYGZhy{}And\PYGZhy{}Hold}\PYG{l+s+s1}{\PYGZsq{}}\PYG{p}{,} \PYG{l+s+s1}{\PYGZsq{}}\PYG{l+s+s1}{Strategy}\PYG{l+s+s1}{\PYGZsq{}}\PYG{p}{:} \PYG{l+s+s1}{\PYGZsq{}}\PYG{l+s+s1}{SMA(20)}\PYG{l+s+s1}{\PYGZsq{}}\PYG{p}{\PYGZcb{}}\PYG{p}{)}
    \PYG{o}{.}\PYG{n}{plot}\PYG{p}{(}\PYG{n}{kind}\PYG{o}{=}\PYG{l+s+s1}{\PYGZsq{}}\PYG{l+s+s1}{bar}\PYG{l+s+s1}{\PYGZsq{}}\PYG{p}{)}
\PYG{p}{)}
\PYG{n}{plt}\PYG{o}{.}\PYG{n}{xticks}\PYG{p}{(}\PYG{n}{rotation}\PYG{o}{=}\PYG{l+m+mi}{0}\PYG{p}{)}
\PYG{n}{plt}\PYG{o}{.}\PYG{n}{ylabel}\PYG{p}{(}\PYG{l+s+s1}{\PYGZsq{}}\PYG{l+s+s1}{Value (\PYGZdl{})}\PYG{l+s+s1}{\PYGZsq{}}\PYG{p}{)}
\PYG{n}{plt}\PYG{o}{.}\PYG{n}{title}\PYG{p}{(}\PYG{l+s+sa}{f}\PYG{l+s+s1}{\PYGZsq{}}\PYG{l+s+s1}{Value of \PYGZdl{}1 Investments Add End of 10\PYGZhy{}Year Holding Periods }\PYG{l+s+s1}{\PYGZsq{}}\PYG{p}{)}
\PYG{n}{plt}\PYG{o}{.}\PYG{n}{show}\PYG{p}{(}\PYG{p}{)}
\end{sphinxVerbatim}

\end{sphinxuseclass}\end{sphinxVerbatimInput}
\begin{sphinxVerbatimOutput}

\begin{sphinxuseclass}{cell_output}
\noindent\sphinxincludegraphics{{262e708684b1109abef3e7de77a33f6ae09019d5b276baf05ba5dcbda2c924e9}.png}

\end{sphinxuseclass}\end{sphinxVerbatimOutput}

\end{sphinxuseclass}
\sphinxAtStartPar
In fact, buy\sphinxhyphen{}and\sphinxhyphen{}hold outperforms SMA(20) is we start in 1950.

\begin{sphinxuseclass}{cell}\begin{sphinxVerbatimInput}

\begin{sphinxuseclass}{cell_input}
\begin{sphinxVerbatim}[commandchars=\\\{\}]
\PYG{n}{\PYGZus{}} \PYG{o}{=} \PYG{n}{ff\PYGZus{}sma}\PYG{o}{.}\PYG{n}{loc}\PYG{p}{[}\PYG{l+s+s1}{\PYGZsq{}}\PYG{l+s+s1}{1950}\PYG{l+s+s1}{\PYGZsq{}}\PYG{p}{:}\PYG{p}{,} \PYG{p}{[}\PYG{l+s+s1}{\PYGZsq{}}\PYG{l+s+s1}{Return}\PYG{l+s+s1}{\PYGZsq{}}\PYG{p}{,} \PYG{l+s+s1}{\PYGZsq{}}\PYG{l+s+s1}{Strategy}\PYG{l+s+s1}{\PYGZsq{}}\PYG{p}{]}\PYG{p}{]}\PYG{o}{.}\PYG{n}{dropna}\PYG{p}{(}\PYG{p}{)}

\PYG{p}{(}
    \PYG{n}{\PYGZus{}}
    \PYG{o}{.}\PYG{n}{add}\PYG{p}{(}\PYG{l+m+mi}{1}\PYG{p}{)}
    \PYG{o}{.}\PYG{n}{cumprod}\PYG{p}{(}\PYG{p}{)}
    \PYG{o}{.}\PYG{n}{rename\PYGZus{}axis}\PYG{p}{(}\PYG{n}{columns}\PYG{o}{=}\PYG{l+s+s1}{\PYGZsq{}}\PYG{l+s+s1}{Strategy}\PYG{l+s+s1}{\PYGZsq{}}\PYG{p}{)}
    \PYG{o}{.}\PYG{n}{rename}\PYG{p}{(}\PYG{n}{columns}\PYG{o}{=}\PYG{p}{\PYGZob{}}\PYG{l+s+s1}{\PYGZsq{}}\PYG{l+s+s1}{Return}\PYG{l+s+s1}{\PYGZsq{}}\PYG{p}{:} \PYG{l+s+s1}{\PYGZsq{}}\PYG{l+s+s1}{Buy\PYGZhy{}And\PYGZhy{}Hold}\PYG{l+s+s1}{\PYGZsq{}}\PYG{p}{,} \PYG{l+s+s1}{\PYGZsq{}}\PYG{l+s+s1}{Strategy}\PYG{l+s+s1}{\PYGZsq{}}\PYG{p}{:} \PYG{l+s+s1}{\PYGZsq{}}\PYG{l+s+s1}{SMA(20)}\PYG{l+s+s1}{\PYGZsq{}}\PYG{p}{\PYGZcb{}}\PYG{p}{)}
    \PYG{o}{.}\PYG{n}{plot}\PYG{p}{(}\PYG{p}{)}
\PYG{p}{)}
\PYG{n}{plt}\PYG{o}{.}\PYG{n}{ylabel}\PYG{p}{(}\PYG{l+s+s1}{\PYGZsq{}}\PYG{l+s+s1}{Value (\PYGZdl{})}\PYG{l+s+s1}{\PYGZsq{}}\PYG{p}{)}
\PYG{n}{plt}\PYG{o}{.}\PYG{n}{title}\PYG{p}{(}\PYG{l+s+sa}{f}\PYG{l+s+s1}{\PYGZsq{}}\PYG{l+s+s1}{Value of \PYGZdl{}1 Invested in Market at Close on }\PYG{l+s+si}{\PYGZob{}}\PYG{n}{\PYGZus{}}\PYG{o}{.}\PYG{n}{index}\PYG{p}{[}\PYG{l+m+mi}{0}\PYG{p}{]} \PYG{o}{\PYGZhy{}} \PYG{n}{pd}\PYG{o}{.}\PYG{n}{offsets}\PYG{o}{.}\PYG{n}{Day}\PYG{p}{(}\PYG{l+m+mi}{1}\PYG{p}{)}\PYG{l+s+si}{:}\PYG{l+s+s1}{\PYGZpc{}B \PYGZpc{}d, \PYGZpc{}Y}\PYG{l+s+si}{\PYGZcb{}}\PYG{l+s+s1}{\PYGZsq{}}\PYG{p}{)}
\PYG{n}{plt}\PYG{o}{.}\PYG{n}{show}\PYG{p}{(}\PYG{p}{)}
\end{sphinxVerbatim}

\end{sphinxuseclass}\end{sphinxVerbatimInput}
\begin{sphinxVerbatimOutput}

\begin{sphinxuseclass}{cell_output}
\noindent\sphinxincludegraphics{{de32325d6c5dda85fff71b12216bc00bdfbe78f65cb56ebc4cf40d6abd840d13}.png}

\end{sphinxuseclass}\end{sphinxVerbatimOutput}

\end{sphinxuseclass}

\subsubsection{Implement a long\sphinxhyphen{}only BB(20, 2) strategy with Bitcoin}
\label{\detokenize{herron_02_practice_04:implement-a-long-only-bb-20-2-strategy-with-bitcoin}}
\sphinxAtStartPar
More on Bollinger Bands \sphinxhref{https://www.bollingerbands.com/bollinger-bands}{here} and \sphinxhref{https://www.bollingerbands.com/bollinger-band-rules}{here}.
In short, Bollinger Bands are bands around a trend, typically defined in terms of simple moving averages and volatilities.
Here, long\sphinxhyphen{}only BB(20, 2) implies we have upper and lower bands at 2 standard deviations above and below SMA(20):
\begin{enumerate}
\sphinxsetlistlabels{\arabic}{enumi}{enumii}{}{.}%
\item {} 
\sphinxAtStartPar
Buy when the closing price crosses LB(20) from below, where LB(20) is SMA(20) minus 2 sigma

\item {} 
\sphinxAtStartPar
Sell when the closing price crosses UB(20) from above, where UB(20) is SMA(20) plus 2 sigma

\item {} 
\sphinxAtStartPar
No short\sphinxhyphen{}selling

\end{enumerate}

\sphinxAtStartPar
The long\sphinxhyphen{}only BB(20, 2) is more difficult to implement than the long\sphinxhyphen{}only SMA(20) because we need to track buys and sells.
For example, if the closing price is between LB(20) and BB(20), we need to know if our last trade was a buy or a sell.
Further, if the closing price is below LB(20), we can still be long because we sell when the closing price crosses UB(20) from above.

\sphinxAtStartPar
\sphinxstyleemphasis{\sphinxstylestrong{Again, write a function \sphinxcode{\sphinxupquote{bb()}} that accepts a data frame \sphinxcode{\sphinxupquote{df}}, window size \sphinxcode{\sphinxupquote{n}}, and number of standard deviations \sphinxcode{\sphinxupquote{m}}.
Use your \sphinxcode{\sphinxupquote{bb()}} function to implement the BB(20, 2) strategy and assign it to data frame \sphinxcode{\sphinxupquote{btc\_bb}}.}}

\begin{sphinxuseclass}{cell}\begin{sphinxVerbatimInput}

\begin{sphinxuseclass}{cell_input}
\begin{sphinxVerbatim}[commandchars=\\\{\}]
\PYG{k}{def} \PYG{n+nf}{bb}\PYG{p}{(}\PYG{n}{df}\PYG{p}{,} \PYG{n}{n}\PYG{o}{=}\PYG{l+m+mi}{20}\PYG{p}{,} \PYG{n}{m}\PYG{o}{=}\PYG{l+m+mi}{2}\PYG{p}{)}\PYG{p}{:}
    \PYG{k}{return} \PYG{p}{(}
        \PYG{n}{df}
        \PYG{o}{.}\PYG{n}{assign}\PYG{p}{(}
            \PYG{n}{Return} \PYG{o}{=} \PYG{k}{lambda} \PYG{n}{x}\PYG{p}{:} \PYG{n}{x}\PYG{p}{[}\PYG{l+s+s1}{\PYGZsq{}}\PYG{l+s+s1}{Adj Close}\PYG{l+s+s1}{\PYGZsq{}}\PYG{p}{]}\PYG{o}{.}\PYG{n}{pct\PYGZus{}change}\PYG{p}{(}\PYG{p}{)}\PYG{p}{,}
            \PYG{n}{SMA} \PYG{o}{=} \PYG{k}{lambda} \PYG{n}{x}\PYG{p}{:} \PYG{n}{x}\PYG{p}{[}\PYG{l+s+s1}{\PYGZsq{}}\PYG{l+s+s1}{Adj Close}\PYG{l+s+s1}{\PYGZsq{}}\PYG{p}{]}\PYG{o}{.}\PYG{n}{rolling}\PYG{p}{(}\PYG{n}{n}\PYG{p}{)}\PYG{o}{.}\PYG{n}{mean}\PYG{p}{(}\PYG{p}{)}\PYG{p}{,}
            \PYG{n}{SMV} \PYG{o}{=} \PYG{k}{lambda} \PYG{n}{x}\PYG{p}{:} \PYG{n}{x}\PYG{p}{[}\PYG{l+s+s1}{\PYGZsq{}}\PYG{l+s+s1}{Adj Close}\PYG{l+s+s1}{\PYGZsq{}}\PYG{p}{]}\PYG{o}{.}\PYG{n}{rolling}\PYG{p}{(}\PYG{n}{n}\PYG{p}{)}\PYG{o}{.}\PYG{n}{std}\PYG{p}{(}\PYG{p}{)}\PYG{p}{,}
            \PYG{n}{UB} \PYG{o}{=} \PYG{k}{lambda} \PYG{n}{x}\PYG{p}{:} \PYG{n}{x}\PYG{p}{[}\PYG{l+s+s1}{\PYGZsq{}}\PYG{l+s+s1}{SMA}\PYG{l+s+s1}{\PYGZsq{}}\PYG{p}{]} \PYG{o}{+} \PYG{n}{m}\PYG{o}{*}\PYG{n}{x}\PYG{p}{[}\PYG{l+s+s1}{\PYGZsq{}}\PYG{l+s+s1}{SMV}\PYG{l+s+s1}{\PYGZsq{}}\PYG{p}{]}\PYG{p}{,}
            \PYG{n}{LB} \PYG{o}{=} \PYG{k}{lambda} \PYG{n}{x}\PYG{p}{:} \PYG{n}{x}\PYG{p}{[}\PYG{l+s+s1}{\PYGZsq{}}\PYG{l+s+s1}{SMA}\PYG{l+s+s1}{\PYGZsq{}}\PYG{p}{]} \PYG{o}{\PYGZhy{}} \PYG{n}{m}\PYG{o}{*}\PYG{n}{x}\PYG{p}{[}\PYG{l+s+s1}{\PYGZsq{}}\PYG{l+s+s1}{SMV}\PYG{l+s+s1}{\PYGZsq{}}\PYG{p}{]}\PYG{p}{,}
            \PYG{n}{Position\PYGZus{}with\PYGZus{}nan} \PYG{o}{=} \PYG{k}{lambda} \PYG{n}{x}\PYG{p}{:} \PYG{n}{np}\PYG{o}{.}\PYG{n}{select}\PYG{p}{(}
                \PYG{n}{condlist}\PYG{o}{=}\PYG{p}{[}
                    \PYG{p}{(}\PYG{n}{x}\PYG{p}{[}\PYG{l+s+s1}{\PYGZsq{}}\PYG{l+s+s1}{Adj Close}\PYG{l+s+s1}{\PYGZsq{}}\PYG{p}{]}\PYG{o}{.}\PYG{n}{shift}\PYG{p}{(}\PYG{l+m+mi}{1}\PYG{p}{)} \PYG{o}{\PYGZgt{}}\PYG{o}{=} \PYG{n}{x}\PYG{p}{[}\PYG{l+s+s1}{\PYGZsq{}}\PYG{l+s+s1}{LB}\PYG{l+s+s1}{\PYGZsq{}}\PYG{p}{]}\PYG{o}{.}\PYG{n}{shift}\PYG{p}{(}\PYG{l+m+mi}{1}\PYG{p}{)}\PYG{p}{)} \PYG{o}{\PYGZam{}} \PYG{p}{(}\PYG{n}{x}\PYG{p}{[}\PYG{l+s+s1}{\PYGZsq{}}\PYG{l+s+s1}{Adj Close}\PYG{l+s+s1}{\PYGZsq{}}\PYG{p}{]}\PYG{o}{.}\PYG{n}{shift}\PYG{p}{(}\PYG{l+m+mi}{2}\PYG{p}{)} \PYG{o}{\PYGZlt{}} \PYG{n}{x}\PYG{p}{[}\PYG{l+s+s1}{\PYGZsq{}}\PYG{l+s+s1}{LB}\PYG{l+s+s1}{\PYGZsq{}}\PYG{p}{]}\PYG{o}{.}\PYG{n}{shift}\PYG{p}{(}\PYG{l+m+mi}{2}\PYG{p}{)}\PYG{p}{)}\PYG{p}{,} 
                    \PYG{p}{(}\PYG{n}{x}\PYG{p}{[}\PYG{l+s+s1}{\PYGZsq{}}\PYG{l+s+s1}{Adj Close}\PYG{l+s+s1}{\PYGZsq{}}\PYG{p}{]}\PYG{o}{.}\PYG{n}{shift}\PYG{p}{(}\PYG{l+m+mi}{1}\PYG{p}{)} \PYG{o}{\PYGZlt{}}\PYG{o}{=} \PYG{n}{x}\PYG{p}{[}\PYG{l+s+s1}{\PYGZsq{}}\PYG{l+s+s1}{UB}\PYG{l+s+s1}{\PYGZsq{}}\PYG{p}{]}\PYG{o}{.}\PYG{n}{shift}\PYG{p}{(}\PYG{l+m+mi}{1}\PYG{p}{)}\PYG{p}{)} \PYG{o}{\PYGZam{}} \PYG{p}{(}\PYG{n}{x}\PYG{p}{[}\PYG{l+s+s1}{\PYGZsq{}}\PYG{l+s+s1}{Adj Close}\PYG{l+s+s1}{\PYGZsq{}}\PYG{p}{]}\PYG{o}{.}\PYG{n}{shift}\PYG{p}{(}\PYG{l+m+mi}{2}\PYG{p}{)} \PYG{o}{\PYGZgt{}} \PYG{n}{x}\PYG{p}{[}\PYG{l+s+s1}{\PYGZsq{}}\PYG{l+s+s1}{UB}\PYG{l+s+s1}{\PYGZsq{}}\PYG{p}{]}\PYG{o}{.}\PYG{n}{shift}\PYG{p}{(}\PYG{l+m+mi}{2}\PYG{p}{)}\PYG{p}{)}\PYG{p}{,} 
                \PYG{p}{]}\PYG{p}{,}
                \PYG{n}{choicelist}\PYG{o}{=}\PYG{p}{[}
                    \PYG{l+m+mi}{1}\PYG{p}{,} 
                    \PYG{l+m+mi}{0}
                \PYG{p}{]}\PYG{p}{,}
                \PYG{n}{default}\PYG{o}{=}\PYG{n}{np}\PYG{o}{.}\PYG{n}{nan}
            \PYG{p}{)}\PYG{p}{,}
            \PYG{n}{Position} \PYG{o}{=} \PYG{k}{lambda} \PYG{n}{x}\PYG{p}{:} \PYG{n}{x}\PYG{p}{[}\PYG{l+s+s1}{\PYGZsq{}}\PYG{l+s+s1}{Position\PYGZus{}with\PYGZus{}nan}\PYG{l+s+s1}{\PYGZsq{}}\PYG{p}{]}\PYG{o}{.}\PYG{n}{fillna}\PYG{p}{(}\PYG{n}{method}\PYG{o}{=}\PYG{l+s+s1}{\PYGZsq{}}\PYG{l+s+s1}{ffill}\PYG{l+s+s1}{\PYGZsq{}}\PYG{p}{)}\PYG{p}{,}
            \PYG{n}{Strategy} \PYG{o}{=} \PYG{k}{lambda} \PYG{n}{x}\PYG{p}{:} \PYG{n}{x}\PYG{p}{[}\PYG{l+s+s1}{\PYGZsq{}}\PYG{l+s+s1}{Position}\PYG{l+s+s1}{\PYGZsq{}}\PYG{p}{]} \PYG{o}{*} \PYG{n}{x}\PYG{p}{[}\PYG{l+s+s1}{\PYGZsq{}}\PYG{l+s+s1}{Return}\PYG{l+s+s1}{\PYGZsq{}}\PYG{p}{]}
        \PYG{p}{)}
    \PYG{p}{)}
\end{sphinxVerbatim}

\end{sphinxuseclass}\end{sphinxVerbatimInput}

\end{sphinxuseclass}
\begin{sphinxuseclass}{cell}\begin{sphinxVerbatimInput}

\begin{sphinxuseclass}{cell_input}
\begin{sphinxVerbatim}[commandchars=\\\{\}]
\PYG{n}{btc\PYGZus{}bb} \PYG{o}{=} \PYG{n}{btc}\PYG{o}{.}\PYG{n}{pipe}\PYG{p}{(}\PYG{n}{bb}\PYG{p}{)}

\PYG{n}{btc\PYGZus{}bb}\PYG{o}{.}\PYG{n}{tail}\PYG{p}{(}\PYG{p}{)}
\end{sphinxVerbatim}

\end{sphinxuseclass}\end{sphinxVerbatimInput}
\begin{sphinxVerbatimOutput}

\begin{sphinxuseclass}{cell_output}
\begin{sphinxVerbatim}[commandchars=\\\{\}]
Variable         Open       High        Low      Close  Adj Close  \PYGZbs{}
Date                                                                
2023\PYGZhy{}03\PYGZhy{}11 20187.8770 20792.5254 20068.6602 20632.4102 20632.4102   
2023\PYGZhy{}03\PYGZhy{}12 20628.0293 22185.0312 20448.8066 22163.9492 22163.9492   
2023\PYGZhy{}03\PYGZhy{}13 22156.4062 24550.8379 21918.1992 24197.5332 24197.5332   
2023\PYGZhy{}03\PYGZhy{}14 24201.7656 26514.7168 24081.1836 24746.0742 24746.0742   
2023\PYGZhy{}03\PYGZhy{}15 24734.6523 25169.7305 24048.3516 24617.3047 24617.3047   

Variable         Volume  Return        SMA       SMV         UB         LB  \PYGZbs{}
Date                                                                         
2023\PYGZhy{}03\PYGZhy{}11  30180288176  0.0221 22791.5311 1305.1707 25401.8725 20181.1896   
2023\PYGZhy{}03\PYGZhy{}12  29279035521  0.0742 22658.2711 1219.4207 25097.1125 20219.4297   
2023\PYGZhy{}03\PYGZhy{}13  49466362688  0.0918 22646.3301 1202.1395 25050.6091 20242.0510   
2023\PYGZhy{}03\PYGZhy{}14  54622230164  0.0227 22674.1916 1245.4486 25165.0888 20183.2944   
2023\PYGZhy{}03\PYGZhy{}15  44081737728 \PYGZhy{}0.0052 22707.6822 1289.7095 25287.1013 20128.2632   

Variable    Position\PYGZus{}with\PYGZus{}nan  Position  Strategy  
Date                                               
2023\PYGZhy{}03\PYGZhy{}11                NaN    1.0000    0.0221  
2023\PYGZhy{}03\PYGZhy{}12             1.0000    1.0000    0.0742  
2023\PYGZhy{}03\PYGZhy{}13                NaN    1.0000    0.0918  
2023\PYGZhy{}03\PYGZhy{}14                NaN    1.0000    0.0227  
2023\PYGZhy{}03\PYGZhy{}15                NaN    1.0000   \PYGZhy{}0.0052  
\end{sphinxVerbatim}

\end{sphinxuseclass}\end{sphinxVerbatimOutput}

\end{sphinxuseclass}
\begin{sphinxuseclass}{cell}\begin{sphinxVerbatimInput}

\begin{sphinxuseclass}{cell_input}
\begin{sphinxVerbatim}[commandchars=\\\{\}]
\PYG{n}{\PYGZus{}} \PYG{o}{=} \PYG{n}{btc\PYGZus{}bb}\PYG{p}{[}\PYG{p}{[}\PYG{l+s+s1}{\PYGZsq{}}\PYG{l+s+s1}{Return}\PYG{l+s+s1}{\PYGZsq{}}\PYG{p}{,} \PYG{l+s+s1}{\PYGZsq{}}\PYG{l+s+s1}{Strategy}\PYG{l+s+s1}{\PYGZsq{}}\PYG{p}{]}\PYG{p}{]}\PYG{o}{.}\PYG{n}{dropna}\PYG{p}{(}\PYG{p}{)}

\PYG{p}{(}
    \PYG{n}{\PYGZus{}}
    \PYG{o}{.}\PYG{n}{add}\PYG{p}{(}\PYG{l+m+mi}{1}\PYG{p}{)}
    \PYG{o}{.}\PYG{n}{cumprod}\PYG{p}{(}\PYG{p}{)}
    \PYG{o}{.}\PYG{n}{rename\PYGZus{}axis}\PYG{p}{(}\PYG{n}{columns}\PYG{o}{=}\PYG{l+s+s1}{\PYGZsq{}}\PYG{l+s+s1}{Strategy}\PYG{l+s+s1}{\PYGZsq{}}\PYG{p}{)}
    \PYG{o}{.}\PYG{n}{rename}\PYG{p}{(}\PYG{n}{columns}\PYG{o}{=}\PYG{p}{\PYGZob{}}\PYG{l+s+s1}{\PYGZsq{}}\PYG{l+s+s1}{Return}\PYG{l+s+s1}{\PYGZsq{}}\PYG{p}{:} \PYG{l+s+s1}{\PYGZsq{}}\PYG{l+s+s1}{Buy\PYGZhy{}And\PYGZhy{}Hold}\PYG{l+s+s1}{\PYGZsq{}}\PYG{p}{,} \PYG{l+s+s1}{\PYGZsq{}}\PYG{l+s+s1}{Strategy}\PYG{l+s+s1}{\PYGZsq{}}\PYG{p}{:} \PYG{l+s+s1}{\PYGZsq{}}\PYG{l+s+s1}{BB(20, 2)}\PYG{l+s+s1}{\PYGZsq{}}\PYG{p}{\PYGZcb{}}\PYG{p}{)}
    \PYG{o}{.}\PYG{n}{plot}\PYG{p}{(}\PYG{p}{)}
\PYG{p}{)}
\PYG{n}{plt}\PYG{o}{.}\PYG{n}{ylabel}\PYG{p}{(}\PYG{l+s+s1}{\PYGZsq{}}\PYG{l+s+s1}{Value (\PYGZdl{})}\PYG{l+s+s1}{\PYGZsq{}}\PYG{p}{)}
\PYG{n}{plt}\PYG{o}{.}\PYG{n}{title}\PYG{p}{(}\PYG{l+s+sa}{f}\PYG{l+s+s1}{\PYGZsq{}}\PYG{l+s+s1}{Value of \PYGZdl{}1 Invested at Close on }\PYG{l+s+si}{\PYGZob{}}\PYG{n}{\PYGZus{}}\PYG{o}{.}\PYG{n}{index}\PYG{p}{[}\PYG{l+m+mi}{0}\PYG{p}{]} \PYG{o}{\PYGZhy{}} \PYG{n}{pd}\PYG{o}{.}\PYG{n}{offsets}\PYG{o}{.}\PYG{n}{Day}\PYG{p}{(}\PYG{l+m+mi}{1}\PYG{p}{)}\PYG{l+s+si}{:}\PYG{l+s+s1}{\PYGZpc{}B \PYGZpc{}d, \PYGZpc{}Y}\PYG{l+s+si}{\PYGZcb{}}\PYG{l+s+s1}{\PYGZsq{}}\PYG{p}{)}
\PYG{n}{plt}\PYG{o}{.}\PYG{n}{show}\PYG{p}{(}\PYG{p}{)}
\end{sphinxVerbatim}

\end{sphinxuseclass}\end{sphinxVerbatimInput}
\begin{sphinxVerbatimOutput}

\begin{sphinxuseclass}{cell_output}
\noindent\sphinxincludegraphics{{1712840168ce637c11b63aaf384533ed9deb985f91ac37aa6e4cf240669a1e8e}.png}

\end{sphinxuseclass}\end{sphinxVerbatimOutput}

\end{sphinxuseclass}
\sphinxAtStartPar
For an asset that we know has large positive returns over the sample, “time in the market” beats “timing the market”.


\subsubsection{Implement a long\sphinxhyphen{}short RSI(14) strategy with Bitcoin}
\label{\detokenize{herron_02_practice_04:implement-a-long-short-rsi-14-strategy-with-bitcoin}}
\sphinxAtStartPar
From \sphinxhref{https://www.fidelity.com/learning-center/trading-investing/technical-analysis/technical-indicator-guide/rsi}{Fidelity}:
\begin{quote}

\sphinxAtStartPar
The Relative Strength Index (RSI), developed by J. Welles Wilder, is a momentum oscillator that measures the speed and change of price movements. The RSI oscillates between zero and 100. Traditionally the RSI is considered overbought when above 70 and oversold when below 30. Signals can be generated by looking for divergences and failure swings. RSI can also be used to identify the general trend.
\end{quote}

\sphinxAtStartPar
Here is the RSI formula: \$RSI(n) = 100 \sphinxhyphen{} \textbackslash{}frac\{100\}\{1 + RS(n)\}\$, where \$RS(n) = \textbackslash{}frac\{SMA(U, n)\}\{SMA(D, n)\}\$.
For “up days”, \$U = \textbackslash{}Delta Adj\textbackslash{} Close\$ and \$D = 0\$, and, for “down days”, \$U = 0\$ and \$D = \sphinxhyphen{} \textbackslash{}Delta Adj\textbackslash{} Close\$.
Therefore, \$U\$ and \$D\$ are always non\sphinxhyphen{}negative.
We can learn more about RSI \sphinxhref{https://en.wikipedia.org/wiki/Relative\_strength\_index}{here}.

\sphinxAtStartPar
We will implement a long\sphinxhyphen{}short RSI(14) as follows:
\begin{enumerate}
\sphinxsetlistlabels{\arabic}{enumi}{enumii}{}{.}%
\item {} 
\sphinxAtStartPar
Enter a long position when  the RSI crosses 30 from below, and exit the position when the RSI crosses 50 from below

\item {} 
\sphinxAtStartPar
Enter a short position when the RSI crosses 70 from above, and exit the position when the RSI crosses 50 from above

\end{enumerate}

\sphinxAtStartPar
\sphinxstyleemphasis{\sphinxstylestrong{Again, write a function \sphinxcode{\sphinxupquote{rsi()}} that accepts a data frame \sphinxcode{\sphinxupquote{df}}, window size \sphinxcode{\sphinxupquote{n}}, and boundary percentiles \sphinxcode{\sphinxupquote{lb}}, \sphinxcode{\sphinxupquote{mb}}, and \sphinxcode{\sphinxupquote{ub}}.
Use your \sphinxcode{\sphinxupquote{rsi()}} function to implement the RSI strategy and assign it to data frame \sphinxcode{\sphinxupquote{btc\_rsi}}.}}

\begin{sphinxuseclass}{cell}\begin{sphinxVerbatimInput}

\begin{sphinxuseclass}{cell_input}
\begin{sphinxVerbatim}[commandchars=\\\{\}]
\PYG{k}{def} \PYG{n+nf}{rsi}\PYG{p}{(}\PYG{n}{df}\PYG{p}{,} \PYG{n}{n}\PYG{o}{=}\PYG{l+m+mi}{14}\PYG{p}{,} \PYG{n}{lb}\PYG{o}{=}\PYG{l+m+mi}{30}\PYG{p}{,} \PYG{n}{mb}\PYG{o}{=}\PYG{l+m+mi}{50}\PYG{p}{,} \PYG{n}{ub}\PYG{o}{=}\PYG{l+m+mi}{70}\PYG{p}{)}\PYG{p}{:}
    \PYG{k}{return} \PYG{n}{df}\PYG{o}{.}\PYG{n}{assign}\PYG{p}{(}
        \PYG{n}{Return} \PYG{o}{=} \PYG{k}{lambda} \PYG{n}{x}\PYG{p}{:} \PYG{n}{x}\PYG{p}{[}\PYG{l+s+s1}{\PYGZsq{}}\PYG{l+s+s1}{Adj Close}\PYG{l+s+s1}{\PYGZsq{}}\PYG{p}{]}\PYG{o}{.}\PYG{n}{pct\PYGZus{}change}\PYG{p}{(}\PYG{p}{)}\PYG{p}{,}
        \PYG{n}{Diff} \PYG{o}{=} \PYG{k}{lambda} \PYG{n}{x}\PYG{p}{:} \PYG{n}{x}\PYG{p}{[}\PYG{l+s+s1}{\PYGZsq{}}\PYG{l+s+s1}{Adj Close}\PYG{l+s+s1}{\PYGZsq{}}\PYG{p}{]}\PYG{o}{.}\PYG{n}{diff}\PYG{p}{(}\PYG{p}{)}\PYG{p}{,}
        \PYG{n}{U} \PYG{o}{=} \PYG{k}{lambda} \PYG{n}{x}\PYG{p}{:} \PYG{n}{np}\PYG{o}{.}\PYG{n}{select}\PYG{p}{(}
            \PYG{n}{condlist}\PYG{o}{=}\PYG{p}{[}\PYG{n}{x}\PYG{p}{[}\PYG{l+s+s1}{\PYGZsq{}}\PYG{l+s+s1}{Diff}\PYG{l+s+s1}{\PYGZsq{}}\PYG{p}{]} \PYG{o}{\PYGZgt{}}\PYG{o}{=} \PYG{l+m+mi}{0}\PYG{p}{,} \PYG{n}{x}\PYG{p}{[}\PYG{l+s+s1}{\PYGZsq{}}\PYG{l+s+s1}{Diff}\PYG{l+s+s1}{\PYGZsq{}}\PYG{p}{]} \PYG{o}{\PYGZlt{}} \PYG{l+m+mi}{0}\PYG{p}{]}\PYG{p}{,}
            \PYG{n}{choicelist}\PYG{o}{=}\PYG{p}{[}\PYG{n}{x}\PYG{p}{[}\PYG{l+s+s1}{\PYGZsq{}}\PYG{l+s+s1}{Diff}\PYG{l+s+s1}{\PYGZsq{}}\PYG{p}{]}\PYG{p}{,} \PYG{l+m+mi}{0}\PYG{p}{]}\PYG{p}{,}
            \PYG{n}{default}\PYG{o}{=}\PYG{n}{np}\PYG{o}{.}\PYG{n}{nan}
        \PYG{p}{)}\PYG{p}{,}
        \PYG{n}{D} \PYG{o}{=} \PYG{k}{lambda} \PYG{n}{x}\PYG{p}{:} \PYG{n}{np}\PYG{o}{.}\PYG{n}{select}\PYG{p}{(}
            \PYG{n}{condlist}\PYG{o}{=}\PYG{p}{[}\PYG{n}{x}\PYG{p}{[}\PYG{l+s+s1}{\PYGZsq{}}\PYG{l+s+s1}{Diff}\PYG{l+s+s1}{\PYGZsq{}}\PYG{p}{]} \PYG{o}{\PYGZlt{}}\PYG{o}{=} \PYG{l+m+mi}{0}\PYG{p}{,} \PYG{n}{x}\PYG{p}{[}\PYG{l+s+s1}{\PYGZsq{}}\PYG{l+s+s1}{Diff}\PYG{l+s+s1}{\PYGZsq{}}\PYG{p}{]} \PYG{o}{\PYGZgt{}} \PYG{l+m+mi}{0}\PYG{p}{]}\PYG{p}{,}
            \PYG{n}{choicelist}\PYG{o}{=}\PYG{p}{[}\PYG{o}{\PYGZhy{}}\PYG{l+m+mi}{1} \PYG{o}{*} \PYG{n}{x}\PYG{p}{[}\PYG{l+s+s1}{\PYGZsq{}}\PYG{l+s+s1}{Diff}\PYG{l+s+s1}{\PYGZsq{}}\PYG{p}{]}\PYG{p}{,} \PYG{l+m+mi}{0}\PYG{p}{]}\PYG{p}{,}
            \PYG{n}{default}\PYG{o}{=}\PYG{n}{np}\PYG{o}{.}\PYG{n}{nan}
        \PYG{p}{)}\PYG{p}{,}
        \PYG{n}{SMAU} \PYG{o}{=} \PYG{k}{lambda} \PYG{n}{x}\PYG{p}{:} \PYG{n}{x}\PYG{p}{[}\PYG{l+s+s1}{\PYGZsq{}}\PYG{l+s+s1}{U}\PYG{l+s+s1}{\PYGZsq{}}\PYG{p}{]}\PYG{o}{.}\PYG{n}{rolling}\PYG{p}{(}\PYG{n}{n}\PYG{p}{)}\PYG{o}{.}\PYG{n}{mean}\PYG{p}{(}\PYG{p}{)}\PYG{p}{,}
        \PYG{n}{SMAD} \PYG{o}{=} \PYG{k}{lambda} \PYG{n}{x}\PYG{p}{:} \PYG{n}{x}\PYG{p}{[}\PYG{l+s+s1}{\PYGZsq{}}\PYG{l+s+s1}{D}\PYG{l+s+s1}{\PYGZsq{}}\PYG{p}{]}\PYG{o}{.}\PYG{n}{rolling}\PYG{p}{(}\PYG{n}{n}\PYG{p}{)}\PYG{o}{.}\PYG{n}{mean}\PYG{p}{(}\PYG{p}{)}\PYG{p}{,}
        \PYG{n}{RS} \PYG{o}{=} \PYG{k}{lambda} \PYG{n}{x}\PYG{p}{:} \PYG{n}{x}\PYG{p}{[}\PYG{l+s+s1}{\PYGZsq{}}\PYG{l+s+s1}{SMAU}\PYG{l+s+s1}{\PYGZsq{}}\PYG{p}{]} \PYG{o}{/} \PYG{n}{x}\PYG{p}{[}\PYG{l+s+s1}{\PYGZsq{}}\PYG{l+s+s1}{SMAD}\PYG{l+s+s1}{\PYGZsq{}}\PYG{p}{]}\PYG{p}{,}
        \PYG{n}{RSI} \PYG{o}{=} \PYG{k}{lambda} \PYG{n}{x}\PYG{p}{:} \PYG{l+m+mi}{100} \PYG{o}{\PYGZhy{}} \PYG{l+m+mi}{100} \PYG{o}{/} \PYG{p}{(}\PYG{l+m+mi}{1} \PYG{o}{+} \PYG{n}{x}\PYG{p}{[}\PYG{l+s+s1}{\PYGZsq{}}\PYG{l+s+s1}{RS}\PYG{l+s+s1}{\PYGZsq{}}\PYG{p}{]}\PYG{p}{)}\PYG{p}{,}
        \PYG{n}{Position\PYGZus{}with\PYGZus{}nan} \PYG{o}{=} \PYG{k}{lambda} \PYG{n}{x}\PYG{p}{:} \PYG{n}{np}\PYG{o}{.}\PYG{n}{select}\PYG{p}{(}
            \PYG{n}{condlist}\PYG{o}{=}\PYG{p}{[}
                \PYG{p}{(}\PYG{n}{x}\PYG{p}{[}\PYG{l+s+s1}{\PYGZsq{}}\PYG{l+s+s1}{RSI}\PYG{l+s+s1}{\PYGZsq{}}\PYG{p}{]}\PYG{o}{.}\PYG{n}{shift}\PYG{p}{(}\PYG{l+m+mi}{1}\PYG{p}{)} \PYG{o}{\PYGZgt{}}\PYG{o}{=} \PYG{n}{lb}\PYG{p}{)} \PYG{o}{\PYGZam{}} \PYG{p}{(}\PYG{n}{x}\PYG{p}{[}\PYG{l+s+s1}{\PYGZsq{}}\PYG{l+s+s1}{RSI}\PYG{l+s+s1}{\PYGZsq{}}\PYG{p}{]}\PYG{o}{.}\PYG{n}{shift}\PYG{p}{(}\PYG{l+m+mi}{2}\PYG{p}{)} \PYG{o}{\PYGZlt{}} \PYG{n}{lb}\PYG{p}{)}\PYG{p}{,} 
                \PYG{p}{(}\PYG{n}{x}\PYG{p}{[}\PYG{l+s+s1}{\PYGZsq{}}\PYG{l+s+s1}{RSI}\PYG{l+s+s1}{\PYGZsq{}}\PYG{p}{]}\PYG{o}{.}\PYG{n}{shift}\PYG{p}{(}\PYG{l+m+mi}{1}\PYG{p}{)} \PYG{o}{\PYGZgt{}}\PYG{o}{=} \PYG{n}{mb}\PYG{p}{)} \PYG{o}{\PYGZam{}} \PYG{p}{(}\PYG{n}{x}\PYG{p}{[}\PYG{l+s+s1}{\PYGZsq{}}\PYG{l+s+s1}{RSI}\PYG{l+s+s1}{\PYGZsq{}}\PYG{p}{]}\PYG{o}{.}\PYG{n}{shift}\PYG{p}{(}\PYG{l+m+mi}{2}\PYG{p}{)} \PYG{o}{\PYGZlt{}} \PYG{n}{mb}\PYG{p}{)}\PYG{p}{,}
                \PYG{p}{(}\PYG{n}{x}\PYG{p}{[}\PYG{l+s+s1}{\PYGZsq{}}\PYG{l+s+s1}{RSI}\PYG{l+s+s1}{\PYGZsq{}}\PYG{p}{]}\PYG{o}{.}\PYG{n}{shift}\PYG{p}{(}\PYG{l+m+mi}{1}\PYG{p}{)} \PYG{o}{\PYGZlt{}}\PYG{o}{=} \PYG{n}{ub}\PYG{p}{)} \PYG{o}{\PYGZam{}} \PYG{p}{(}\PYG{n}{x}\PYG{p}{[}\PYG{l+s+s1}{\PYGZsq{}}\PYG{l+s+s1}{RSI}\PYG{l+s+s1}{\PYGZsq{}}\PYG{p}{]}\PYG{o}{.}\PYG{n}{shift}\PYG{p}{(}\PYG{l+m+mi}{2}\PYG{p}{)} \PYG{o}{\PYGZgt{}} \PYG{n}{ub}\PYG{p}{)}\PYG{p}{,} 
                \PYG{p}{(}\PYG{n}{x}\PYG{p}{[}\PYG{l+s+s1}{\PYGZsq{}}\PYG{l+s+s1}{RSI}\PYG{l+s+s1}{\PYGZsq{}}\PYG{p}{]}\PYG{o}{.}\PYG{n}{shift}\PYG{p}{(}\PYG{l+m+mi}{1}\PYG{p}{)} \PYG{o}{\PYGZlt{}}\PYG{o}{=} \PYG{n}{mb}\PYG{p}{)} \PYG{o}{\PYGZam{}} \PYG{p}{(}\PYG{n}{x}\PYG{p}{[}\PYG{l+s+s1}{\PYGZsq{}}\PYG{l+s+s1}{RSI}\PYG{l+s+s1}{\PYGZsq{}}\PYG{p}{]}\PYG{o}{.}\PYG{n}{shift}\PYG{p}{(}\PYG{l+m+mi}{2}\PYG{p}{)} \PYG{o}{\PYGZgt{}} \PYG{n}{mb}\PYG{p}{)}\PYG{p}{,}
            \PYG{p}{]}\PYG{p}{,}
            \PYG{n}{choicelist}\PYG{o}{=}\PYG{p}{[}
                \PYG{l+m+mi}{1}\PYG{p}{,} 
                \PYG{l+m+mi}{0}\PYG{p}{,}
                \PYG{o}{\PYGZhy{}}\PYG{l+m+mi}{1}\PYG{p}{,}
                \PYG{l+m+mi}{0}
            \PYG{p}{]}\PYG{p}{,}
            \PYG{n}{default}\PYG{o}{=}\PYG{n}{np}\PYG{o}{.}\PYG{n}{nan}
        \PYG{p}{)}\PYG{p}{,}
        \PYG{n}{Position} \PYG{o}{=} \PYG{k}{lambda} \PYG{n}{x}\PYG{p}{:} \PYG{n}{x}\PYG{p}{[}\PYG{l+s+s1}{\PYGZsq{}}\PYG{l+s+s1}{Position\PYGZus{}with\PYGZus{}nan}\PYG{l+s+s1}{\PYGZsq{}}\PYG{p}{]}\PYG{o}{.}\PYG{n}{fillna}\PYG{p}{(}\PYG{n}{method}\PYG{o}{=}\PYG{l+s+s1}{\PYGZsq{}}\PYG{l+s+s1}{ffill}\PYG{l+s+s1}{\PYGZsq{}}\PYG{p}{)}\PYG{p}{,}
        \PYG{n}{Strategy} \PYG{o}{=} \PYG{k}{lambda} \PYG{n}{x}\PYG{p}{:} \PYG{n}{x}\PYG{p}{[}\PYG{l+s+s1}{\PYGZsq{}}\PYG{l+s+s1}{Position}\PYG{l+s+s1}{\PYGZsq{}}\PYG{p}{]} \PYG{o}{*} \PYG{n}{x}\PYG{p}{[}\PYG{l+s+s1}{\PYGZsq{}}\PYG{l+s+s1}{Return}\PYG{l+s+s1}{\PYGZsq{}}\PYG{p}{]}
    \PYG{p}{)}
\end{sphinxVerbatim}

\end{sphinxuseclass}\end{sphinxVerbatimInput}

\end{sphinxuseclass}
\begin{sphinxuseclass}{cell}\begin{sphinxVerbatimInput}

\begin{sphinxuseclass}{cell_input}
\begin{sphinxVerbatim}[commandchars=\\\{\}]
\PYG{n}{btc\PYGZus{}rsi} \PYG{o}{=} \PYG{n}{rsi}\PYG{p}{(}\PYG{n}{btc}\PYG{p}{)}

\PYG{n}{btc\PYGZus{}rsi}\PYG{o}{.}\PYG{n}{tail}\PYG{p}{(}\PYG{p}{)}
\end{sphinxVerbatim}

\end{sphinxuseclass}\end{sphinxVerbatimInput}
\begin{sphinxVerbatimOutput}

\begin{sphinxuseclass}{cell_output}
\begin{sphinxVerbatim}[commandchars=\\\{\}]
Variable         Open       High        Low      Close  Adj Close  \PYGZbs{}
Date                                                                
2023\PYGZhy{}03\PYGZhy{}11 20187.8770 20792.5254 20068.6602 20632.4102 20632.4102   
2023\PYGZhy{}03\PYGZhy{}12 20628.0293 22185.0312 20448.8066 22163.9492 22163.9492   
2023\PYGZhy{}03\PYGZhy{}13 22156.4062 24550.8379 21918.1992 24197.5332 24197.5332   
2023\PYGZhy{}03\PYGZhy{}14 24201.7656 26514.7168 24081.1836 24746.0742 24746.0742   
2023\PYGZhy{}03\PYGZhy{}15 24734.6523 25169.7305 24048.3516 24617.3047 24617.3047   

Variable         Volume  Return      Diff         U        D     SMAU  \PYGZbs{}
Date                                                                    
2023\PYGZhy{}03\PYGZhy{}11  30180288176  0.0221  445.1660  445.1660   0.0000 100.8832   
2023\PYGZhy{}03\PYGZhy{}12  29279035521  0.0742 1531.5391 1531.5391   0.0000 182.7190   
2023\PYGZhy{}03\PYGZhy{}13  49466362688  0.0918 2033.5840 2033.5840   0.0000 327.9750   
2023\PYGZhy{}03\PYGZhy{}14  54622230164  0.0227  548.5410  548.5410   0.0000 367.1565   
2023\PYGZhy{}03\PYGZhy{}15  44081737728 \PYGZhy{}0.0052 \PYGZhy{}128.7695    0.0000 128.7695 331.4996   

Variable       SMAD     RS     RSI  Position\PYGZus{}with\PYGZus{}nan  Position  Strategy  
Date                                                                       
2023\PYGZhy{}03\PYGZhy{}11 282.5236 0.3571 26.3123                NaN    1.0000    0.0221  
2023\PYGZhy{}03\PYGZhy{}12 282.5236 0.6467 39.2739                NaN    1.0000    0.0742  
2023\PYGZhy{}03\PYGZhy{}13 279.7849 1.1722 53.9646             1.0000    1.0000    0.0918  
2023\PYGZhy{}03\PYGZhy{}14 252.9622 1.4514 59.2075             0.0000    0.0000    0.0000  
2023\PYGZhy{}03\PYGZhy{}15 262.1600 1.2645 55.8400                NaN    0.0000   \PYGZhy{}0.0000  
\end{sphinxVerbatim}

\end{sphinxuseclass}\end{sphinxVerbatimOutput}

\end{sphinxuseclass}
\begin{sphinxuseclass}{cell}\begin{sphinxVerbatimInput}

\begin{sphinxuseclass}{cell_input}
\begin{sphinxVerbatim}[commandchars=\\\{\}]
\PYG{n}{\PYGZus{}} \PYG{o}{=} \PYG{n}{btc\PYGZus{}rsi}\PYG{p}{[}\PYG{p}{[}\PYG{l+s+s1}{\PYGZsq{}}\PYG{l+s+s1}{Return}\PYG{l+s+s1}{\PYGZsq{}}\PYG{p}{,} \PYG{l+s+s1}{\PYGZsq{}}\PYG{l+s+s1}{Strategy}\PYG{l+s+s1}{\PYGZsq{}}\PYG{p}{]}\PYG{p}{]}\PYG{o}{.}\PYG{n}{dropna}\PYG{p}{(}\PYG{p}{)}

\PYG{p}{(}
    \PYG{n}{\PYGZus{}}
    \PYG{o}{.}\PYG{n}{add}\PYG{p}{(}\PYG{l+m+mi}{1}\PYG{p}{)}
    \PYG{o}{.}\PYG{n}{cumprod}\PYG{p}{(}\PYG{p}{)}
    \PYG{o}{.}\PYG{n}{rename\PYGZus{}axis}\PYG{p}{(}\PYG{n}{columns}\PYG{o}{=}\PYG{l+s+s1}{\PYGZsq{}}\PYG{l+s+s1}{Strategy}\PYG{l+s+s1}{\PYGZsq{}}\PYG{p}{)}
    \PYG{o}{.}\PYG{n}{rename}\PYG{p}{(}\PYG{n}{columns}\PYG{o}{=}\PYG{p}{\PYGZob{}}\PYG{l+s+s1}{\PYGZsq{}}\PYG{l+s+s1}{Return}\PYG{l+s+s1}{\PYGZsq{}}\PYG{p}{:} \PYG{l+s+s1}{\PYGZsq{}}\PYG{l+s+s1}{Buy\PYGZhy{}And\PYGZhy{}Hold}\PYG{l+s+s1}{\PYGZsq{}}\PYG{p}{,} \PYG{l+s+s1}{\PYGZsq{}}\PYG{l+s+s1}{Strategy}\PYG{l+s+s1}{\PYGZsq{}}\PYG{p}{:} \PYG{l+s+s1}{\PYGZsq{}}\PYG{l+s+s1}{RSI(14)}\PYG{l+s+s1}{\PYGZsq{}}\PYG{p}{\PYGZcb{}}\PYG{p}{)}
    \PYG{o}{.}\PYG{n}{plot}\PYG{p}{(}\PYG{p}{)}
\PYG{p}{)}
\PYG{n}{plt}\PYG{o}{.}\PYG{n}{ylabel}\PYG{p}{(}\PYG{l+s+s1}{\PYGZsq{}}\PYG{l+s+s1}{Value (\PYGZdl{})}\PYG{l+s+s1}{\PYGZsq{}}\PYG{p}{)}
\PYG{n}{plt}\PYG{o}{.}\PYG{n}{title}\PYG{p}{(}\PYG{l+s+sa}{f}\PYG{l+s+s1}{\PYGZsq{}}\PYG{l+s+s1}{Value of \PYGZdl{}1 Invested at Close on }\PYG{l+s+si}{\PYGZob{}}\PYG{n}{\PYGZus{}}\PYG{o}{.}\PYG{n}{index}\PYG{p}{[}\PYG{l+m+mi}{0}\PYG{p}{]} \PYG{o}{\PYGZhy{}} \PYG{n}{pd}\PYG{o}{.}\PYG{n}{offsets}\PYG{o}{.}\PYG{n}{Day}\PYG{p}{(}\PYG{l+m+mi}{1}\PYG{p}{)}\PYG{l+s+si}{:}\PYG{l+s+s1}{\PYGZpc{}B \PYGZpc{}d, \PYGZpc{}Y}\PYG{l+s+si}{\PYGZcb{}}\PYG{l+s+s1}{\PYGZsq{}}\PYG{p}{)}
\PYG{n}{plt}\PYG{o}{.}\PYG{n}{show}\PYG{p}{(}\PYG{p}{)}
\end{sphinxVerbatim}

\end{sphinxuseclass}\end{sphinxVerbatimInput}
\begin{sphinxVerbatimOutput}

\begin{sphinxuseclass}{cell_output}
\noindent\sphinxincludegraphics{{7a408772ea36fc857aee85128d3c0797cc8766924aa0789e89e26ae642d598d7}.png}

\end{sphinxuseclass}\end{sphinxVerbatimOutput}

\end{sphinxuseclass}
\sphinxAtStartPar
We can compare all three!
Shorting Bitcoin has been dangerous, as the poor returns on RSI(14) show!

\begin{sphinxuseclass}{cell}\begin{sphinxVerbatimInput}

\begin{sphinxuseclass}{cell_input}
\begin{sphinxVerbatim}[commandchars=\\\{\}]
\PYG{n}{\PYGZus{}} \PYG{o}{=} \PYG{p}{(}
    \PYG{n}{btc\PYGZus{}sma}\PYG{p}{[}\PYG{p}{[}\PYG{l+s+s1}{\PYGZsq{}}\PYG{l+s+s1}{Return}\PYG{l+s+s1}{\PYGZsq{}}\PYG{p}{,} \PYG{l+s+s1}{\PYGZsq{}}\PYG{l+s+s1}{Strategy}\PYG{l+s+s1}{\PYGZsq{}}\PYG{p}{]}\PYG{p}{]}
    \PYG{o}{.}\PYG{n}{join}\PYG{p}{(}
        \PYG{n}{btc\PYGZus{}bb}\PYG{p}{[}\PYG{p}{[}\PYG{l+s+s1}{\PYGZsq{}}\PYG{l+s+s1}{Strategy}\PYG{l+s+s1}{\PYGZsq{}}\PYG{p}{]}\PYG{p}{]}\PYG{o}{.}\PYG{n}{add\PYGZus{}suffix}\PYG{p}{(}\PYG{l+s+s1}{\PYGZsq{}}\PYG{l+s+s1}{\PYGZus{}BB}\PYG{l+s+s1}{\PYGZsq{}}\PYG{p}{)}\PYG{p}{,} 
    \PYG{p}{)}
    \PYG{o}{.}\PYG{n}{join}\PYG{p}{(}
        \PYG{n}{btc\PYGZus{}rsi}\PYG{p}{[}\PYG{p}{[}\PYG{l+s+s1}{\PYGZsq{}}\PYG{l+s+s1}{Strategy}\PYG{l+s+s1}{\PYGZsq{}}\PYG{p}{]}\PYG{p}{]}\PYG{o}{.}\PYG{n}{add\PYGZus{}suffix}\PYG{p}{(}\PYG{l+s+s1}{\PYGZsq{}}\PYG{l+s+s1}{\PYGZus{}RSI}\PYG{l+s+s1}{\PYGZsq{}}\PYG{p}{)}\PYG{p}{,} 
    \PYG{p}{)}
    \PYG{o}{.}\PYG{n}{dropna}\PYG{p}{(}\PYG{p}{)}
\PYG{p}{)}


\PYG{p}{(}
    \PYG{n}{\PYGZus{}}
    \PYG{o}{.}\PYG{n}{add}\PYG{p}{(}\PYG{l+m+mi}{1}\PYG{p}{)}
    \PYG{o}{.}\PYG{n}{cumprod}\PYG{p}{(}\PYG{p}{)}
    \PYG{o}{.}\PYG{n}{rename\PYGZus{}axis}\PYG{p}{(}\PYG{n}{columns}\PYG{o}{=}\PYG{l+s+s1}{\PYGZsq{}}\PYG{l+s+s1}{Strategy}\PYG{l+s+s1}{\PYGZsq{}}\PYG{p}{)}
    \PYG{o}{.}\PYG{n}{rename}\PYG{p}{(}\PYG{n}{columns}\PYG{o}{=}
            \PYG{p}{\PYGZob{}}
                \PYG{l+s+s1}{\PYGZsq{}}\PYG{l+s+s1}{Return}\PYG{l+s+s1}{\PYGZsq{}}\PYG{p}{:} \PYG{l+s+s1}{\PYGZsq{}}\PYG{l+s+s1}{Buy\PYGZhy{}And\PYGZhy{}Hold}\PYG{l+s+s1}{\PYGZsq{}}\PYG{p}{,} 
                \PYG{l+s+s1}{\PYGZsq{}}\PYG{l+s+s1}{Strategy}\PYG{l+s+s1}{\PYGZsq{}}\PYG{p}{:} \PYG{l+s+s1}{\PYGZsq{}}\PYG{l+s+s1}{SMA(20)}\PYG{l+s+s1}{\PYGZsq{}}\PYG{p}{,}
                \PYG{l+s+s1}{\PYGZsq{}}\PYG{l+s+s1}{Strategy\PYGZus{}BB}\PYG{l+s+s1}{\PYGZsq{}}\PYG{p}{:} \PYG{l+s+s1}{\PYGZsq{}}\PYG{l+s+s1}{BB(20, 2)}\PYG{l+s+s1}{\PYGZsq{}}\PYG{p}{,}
                \PYG{l+s+s1}{\PYGZsq{}}\PYG{l+s+s1}{Strategy\PYGZus{}RSI}\PYG{l+s+s1}{\PYGZsq{}}\PYG{p}{:} \PYG{l+s+s1}{\PYGZsq{}}\PYG{l+s+s1}{RSI(14)}\PYG{l+s+s1}{\PYGZsq{}}\PYG{p}{,}
            \PYG{p}{\PYGZcb{}}
           \PYG{p}{)}
    \PYG{o}{.}\PYG{n}{plot}\PYG{p}{(}\PYG{p}{)}
\PYG{p}{)}
\PYG{n}{plt}\PYG{o}{.}\PYG{n}{semilogy}\PYG{p}{(}\PYG{p}{)}
\PYG{n}{plt}\PYG{o}{.}\PYG{n}{ylabel}\PYG{p}{(}\PYG{l+s+s1}{\PYGZsq{}}\PYG{l+s+s1}{Value (\PYGZdl{})}\PYG{l+s+s1}{\PYGZsq{}}\PYG{p}{)}
\PYG{n}{plt}\PYG{o}{.}\PYG{n}{title}\PYG{p}{(}\PYG{l+s+sa}{f}\PYG{l+s+s1}{\PYGZsq{}}\PYG{l+s+s1}{Value of \PYGZdl{}1 Invested at Close on }\PYG{l+s+si}{\PYGZob{}}\PYG{n}{\PYGZus{}}\PYG{o}{.}\PYG{n}{index}\PYG{p}{[}\PYG{l+m+mi}{0}\PYG{p}{]} \PYG{o}{\PYGZhy{}} \PYG{n}{pd}\PYG{o}{.}\PYG{n}{offsets}\PYG{o}{.}\PYG{n}{Day}\PYG{p}{(}\PYG{l+m+mi}{1}\PYG{p}{)}\PYG{l+s+si}{:}\PYG{l+s+s1}{\PYGZpc{}B \PYGZpc{}d, \PYGZpc{}Y}\PYG{l+s+si}{\PYGZcb{}}\PYG{l+s+s1}{\PYGZsq{}}\PYG{p}{)}
\PYG{n}{plt}\PYG{o}{.}\PYG{n}{show}\PYG{p}{(}\PYG{p}{)}
\end{sphinxVerbatim}

\end{sphinxuseclass}\end{sphinxVerbatimInput}
\begin{sphinxVerbatimOutput}

\begin{sphinxuseclass}{cell_output}
\noindent\sphinxincludegraphics{{bca6fe8a2ee1156bf6d412f316c36d0be9567385beaf9e1190a825a35bc69504}.png}

\end{sphinxuseclass}\end{sphinxVerbatimOutput}

\end{sphinxuseclass}
\sphinxstepscope


\section{Herron Topic 2 \sphinxhyphen{} Practice (Wednesday 2:45 PM, Section 2)}
\label{\detokenize{herron_02_practice_02:herron-topic-2-practice-wednesday-2-45-pm-section-2}}\label{\detokenize{herron_02_practice_02::doc}}

\subsection{Announcements}
\label{\detokenize{herron_02_practice_02:announcements}}\begin{itemize}
\item {} 
\sphinxAtStartPar
I will finish grading projects this week/weekend

\item {} 
\sphinxAtStartPar
Quiz 5 due \textasciitilde{}\textasciitilde{}Friday\textasciitilde{}\textasciitilde{} Sunday at 11:59 PM
\begin{itemize}
\item {} 
\sphinxAtStartPar
A handful of students have submitted identical quizzes

\item {} 
\sphinxAtStartPar
Quizzes are individual efforts

\item {} 
\sphinxAtStartPar
Do not assume it is hard to for me to compare quiz and project submissions

\end{itemize}

\item {} 
\sphinxAtStartPar
DataCamp 20,000 XP due \sphinxstyleemphasis{next} Friday at 11:59 PM

\item {} 
\sphinxAtStartPar
Attendance and participation account for 5\% of your grade

\end{itemize}


\subsection{Practice}
\label{\detokenize{herron_02_practice_02:practice}}
\begin{sphinxuseclass}{cell}\begin{sphinxVerbatimInput}

\begin{sphinxuseclass}{cell_input}
\begin{sphinxVerbatim}[commandchars=\\\{\}]
\PYG{k+kn}{import} \PYG{n+nn}{matplotlib}\PYG{n+nn}{.}\PYG{n+nn}{pyplot} \PYG{k}{as} \PYG{n+nn}{plt}
\PYG{k+kn}{import} \PYG{n+nn}{numpy} \PYG{k}{as} \PYG{n+nn}{np}
\PYG{k+kn}{import} \PYG{n+nn}{pandas} \PYG{k}{as} \PYG{n+nn}{pd}
\end{sphinxVerbatim}

\end{sphinxuseclass}\end{sphinxVerbatimInput}

\end{sphinxuseclass}
\begin{sphinxuseclass}{cell}\begin{sphinxVerbatimInput}

\begin{sphinxuseclass}{cell_input}
\begin{sphinxVerbatim}[commandchars=\\\{\}]
\PYG{o}{\PYGZpc{}}\PYG{k}{config} InlineBackend.figure\PYGZus{}format = \PYGZsq{}retina\PYGZsq{}
\PYG{o}{\PYGZpc{}}\PYG{k}{precision} 4
\PYG{n}{pd}\PYG{o}{.}\PYG{n}{options}\PYG{o}{.}\PYG{n}{display}\PYG{o}{.}\PYG{n}{float\PYGZus{}format} \PYG{o}{=} \PYG{l+s+s1}{\PYGZsq{}}\PYG{l+s+si}{\PYGZob{}:.4f\PYGZcb{}}\PYG{l+s+s1}{\PYGZsq{}}\PYG{o}{.}\PYG{n}{format}
\end{sphinxVerbatim}

\end{sphinxuseclass}\end{sphinxVerbatimInput}

\end{sphinxuseclass}

\subsubsection{Implement the SMA(20) strategy with Bitcoin from the lecture notebook}
\label{\detokenize{herron_02_practice_02:implement-the-sma-20-strategy-with-bitcoin-from-the-lecture-notebook}}
\sphinxAtStartPar
\textasciitilde{}\textasciitilde{}Try to create the \sphinxcode{\sphinxupquote{btc}} data frame in one code cell with one assignment (i.e., one \sphinxcode{\sphinxupquote{=}}).\textasciitilde{}\textasciitilde{}
\sphinxstyleemphasis{\sphinxstylestrong{Write a function \sphinxcode{\sphinxupquote{sma()}} that accepts a data frame \sphinxcode{\sphinxupquote{df}} and window size \sphinxcode{\sphinxupquote{n}}.
Use your \sphinxcode{\sphinxupquote{sma()}} function to implement the SMA(20) strategy and assign it to data frame \sphinxcode{\sphinxupquote{btc\_sma}}.}}

\begin{sphinxuseclass}{cell}\begin{sphinxVerbatimInput}

\begin{sphinxuseclass}{cell_input}
\begin{sphinxVerbatim}[commandchars=\\\{\}]
\PYG{k+kn}{import} \PYG{n+nn}{yfinance} \PYG{k}{as} \PYG{n+nn}{yf}
\end{sphinxVerbatim}

\end{sphinxuseclass}\end{sphinxVerbatimInput}

\end{sphinxuseclass}
\begin{sphinxuseclass}{cell}\begin{sphinxVerbatimInput}

\begin{sphinxuseclass}{cell_input}
\begin{sphinxVerbatim}[commandchars=\\\{\}]
\PYG{n}{btc} \PYG{o}{=} \PYG{p}{(}
    \PYG{n}{yf}\PYG{o}{.}\PYG{n}{download}\PYG{p}{(}\PYG{n}{tickers}\PYG{o}{=}\PYG{l+s+s1}{\PYGZsq{}}\PYG{l+s+s1}{BTC\PYGZhy{}USD}\PYG{l+s+s1}{\PYGZsq{}}\PYG{p}{,} \PYG{n}{progress}\PYG{o}{=}\PYG{k+kc}{False}\PYG{p}{)}
    \PYG{o}{.}\PYG{n}{assign}\PYG{p}{(}
        \PYG{n}{Date} \PYG{o}{=} \PYG{k}{lambda} \PYG{n}{x}\PYG{p}{:} \PYG{n}{x}\PYG{o}{.}\PYG{n}{index}\PYG{o}{.}\PYG{n}{tz\PYGZus{}localize}\PYG{p}{(}\PYG{k+kc}{None}\PYG{p}{)}\PYG{p}{,}
    \PYG{p}{)}
    \PYG{o}{.}\PYG{n}{set\PYGZus{}index}\PYG{p}{(}\PYG{l+s+s1}{\PYGZsq{}}\PYG{l+s+s1}{Date}\PYG{l+s+s1}{\PYGZsq{}}\PYG{p}{)}
    \PYG{o}{.}\PYG{n}{rename\PYGZus{}axis}\PYG{p}{(}\PYG{n}{columns}\PYG{o}{=}\PYG{l+s+s1}{\PYGZsq{}}\PYG{l+s+s1}{Variable}\PYG{l+s+s1}{\PYGZsq{}}\PYG{p}{)}
\PYG{p}{)}

\PYG{n}{btc}\PYG{o}{.}\PYG{n}{head}\PYG{p}{(}\PYG{p}{)}
\end{sphinxVerbatim}

\end{sphinxuseclass}\end{sphinxVerbatimInput}
\begin{sphinxVerbatimOutput}

\begin{sphinxuseclass}{cell_output}
\begin{sphinxVerbatim}[commandchars=\\\{\}]
Variable       Open     High      Low    Close  Adj Close    Volume
Date                                                               
2014\PYGZhy{}09\PYGZhy{}17 465.8640 468.1740 452.4220 457.3340   457.3340  21056800
2014\PYGZhy{}09\PYGZhy{}18 456.8600 456.8600 413.1040 424.4400   424.4400  34483200
2014\PYGZhy{}09\PYGZhy{}19 424.1030 427.8350 384.5320 394.7960   394.7960  37919700
2014\PYGZhy{}09\PYGZhy{}20 394.6730 423.2960 389.8830 408.9040   408.9040  36863600
2014\PYGZhy{}09\PYGZhy{}21 408.0850 412.4260 393.1810 398.8210   398.8210  26580100
\end{sphinxVerbatim}

\end{sphinxuseclass}\end{sphinxVerbatimOutput}

\end{sphinxuseclass}
\begin{sphinxuseclass}{cell}\begin{sphinxVerbatimInput}

\begin{sphinxuseclass}{cell_input}
\begin{sphinxVerbatim}[commandchars=\\\{\}]
\PYG{k}{def} \PYG{n+nf}{sma}\PYG{p}{(}\PYG{n}{df}\PYG{p}{,} \PYG{n}{n}\PYG{o}{=}\PYG{l+m+mi}{20}\PYG{p}{)}\PYG{p}{:}
    \PYG{k}{return} \PYG{p}{(}
        \PYG{n}{df}
        \PYG{o}{.}\PYG{n}{assign}\PYG{p}{(}
            \PYG{n}{Return} \PYG{o}{=} \PYG{k}{lambda} \PYG{n}{x}\PYG{p}{:} \PYG{n}{x}\PYG{p}{[}\PYG{l+s+s1}{\PYGZsq{}}\PYG{l+s+s1}{Adj Close}\PYG{l+s+s1}{\PYGZsq{}}\PYG{p}{]}\PYG{o}{.}\PYG{n}{pct\PYGZus{}change}\PYG{p}{(}\PYG{p}{)}\PYG{p}{,}
            \PYG{n}{SMA} \PYG{o}{=} \PYG{k}{lambda} \PYG{n}{x}\PYG{p}{:} \PYG{n}{x}\PYG{p}{[}\PYG{l+s+s1}{\PYGZsq{}}\PYG{l+s+s1}{Adj Close}\PYG{l+s+s1}{\PYGZsq{}}\PYG{p}{]}\PYG{o}{.}\PYG{n}{rolling}\PYG{p}{(}\PYG{n}{n}\PYG{p}{)}\PYG{o}{.}\PYG{n}{mean}\PYG{p}{(}\PYG{p}{)}\PYG{p}{,}
            \PYG{n}{Position} \PYG{o}{=} \PYG{k}{lambda} \PYG{n}{x}\PYG{p}{:} \PYG{n}{np}\PYG{o}{.}\PYG{n}{select}\PYG{p}{(}
                \PYG{n}{condlist}\PYG{o}{=}\PYG{p}{[}
                    \PYG{n}{x}\PYG{p}{[}\PYG{l+s+s1}{\PYGZsq{}}\PYG{l+s+s1}{Adj Close}\PYG{l+s+s1}{\PYGZsq{}}\PYG{p}{]}\PYG{o}{.}\PYG{n}{shift}\PYG{p}{(}\PYG{p}{)} \PYG{o}{\PYGZgt{}} \PYG{n}{x}\PYG{p}{[}\PYG{l+s+s1}{\PYGZsq{}}\PYG{l+s+s1}{SMA}\PYG{l+s+s1}{\PYGZsq{}}\PYG{p}{]}\PYG{o}{.}\PYG{n}{shift}\PYG{p}{(}\PYG{p}{)}\PYG{p}{,} 
                    \PYG{n}{x}\PYG{p}{[}\PYG{l+s+s1}{\PYGZsq{}}\PYG{l+s+s1}{Adj Close}\PYG{l+s+s1}{\PYGZsq{}}\PYG{p}{]}\PYG{o}{.}\PYG{n}{shift}\PYG{p}{(}\PYG{p}{)} \PYG{o}{\PYGZlt{}}\PYG{o}{=} \PYG{n}{x}\PYG{p}{[}\PYG{l+s+s1}{\PYGZsq{}}\PYG{l+s+s1}{SMA}\PYG{l+s+s1}{\PYGZsq{}}\PYG{p}{]}\PYG{o}{.}\PYG{n}{shift}\PYG{p}{(}\PYG{p}{)}
                \PYG{p}{]}\PYG{p}{,}
                \PYG{n}{choicelist}\PYG{o}{=}\PYG{p}{[}
                    \PYG{l+m+mi}{1}\PYG{p}{,} 
                    \PYG{l+m+mi}{0}
                \PYG{p}{]}\PYG{p}{,}
                \PYG{n}{default}\PYG{o}{=}\PYG{n}{np}\PYG{o}{.}\PYG{n}{nan}
            \PYG{p}{)}\PYG{p}{,}
            \PYG{n}{Strategy} \PYG{o}{=} \PYG{k}{lambda} \PYG{n}{x}\PYG{p}{:} \PYG{n}{x}\PYG{p}{[}\PYG{l+s+s1}{\PYGZsq{}}\PYG{l+s+s1}{Position}\PYG{l+s+s1}{\PYGZsq{}}\PYG{p}{]} \PYG{o}{*} \PYG{n}{x}\PYG{p}{[}\PYG{l+s+s1}{\PYGZsq{}}\PYG{l+s+s1}{Return}\PYG{l+s+s1}{\PYGZsq{}}\PYG{p}{]}
        \PYG{p}{)}
    \PYG{p}{)}
\end{sphinxVerbatim}

\end{sphinxuseclass}\end{sphinxVerbatimInput}

\end{sphinxuseclass}
\begin{sphinxuseclass}{cell}\begin{sphinxVerbatimInput}

\begin{sphinxuseclass}{cell_input}
\begin{sphinxVerbatim}[commandchars=\\\{\}]
\PYG{n}{btc\PYGZus{}sma} \PYG{o}{=} \PYG{n}{btc}\PYG{o}{.}\PYG{n}{pipe}\PYG{p}{(}\PYG{n}{sma}\PYG{p}{,} \PYG{n}{n}\PYG{o}{=}\PYG{l+m+mi}{20}\PYG{p}{)}

\PYG{n}{btc\PYGZus{}sma}\PYG{o}{.}\PYG{n}{tail}\PYG{p}{(}\PYG{p}{)}
\end{sphinxVerbatim}

\end{sphinxuseclass}\end{sphinxVerbatimInput}
\begin{sphinxVerbatimOutput}

\begin{sphinxuseclass}{cell_output}
\begin{sphinxVerbatim}[commandchars=\\\{\}]
Variable         Open       High        Low      Close  Adj Close  \PYGZbs{}
Date                                                                
2023\PYGZhy{}03\PYGZhy{}11 20187.8770 20792.5254 20068.6602 20632.4102 20632.4102   
2023\PYGZhy{}03\PYGZhy{}12 20628.0293 22185.0312 20448.8066 22163.9492 22163.9492   
2023\PYGZhy{}03\PYGZhy{}13 22156.4062 24550.8379 21918.1992 24197.5332 24197.5332   
2023\PYGZhy{}03\PYGZhy{}14 24201.7656 26514.7168 24081.1836 24746.0742 24746.0742   
2023\PYGZhy{}03\PYGZhy{}15 24734.6523 25169.7305 24048.3516 24617.3047 24617.3047   

Variable         Volume  Return        SMA  Position  Strategy  
Date                                                            
2023\PYGZhy{}03\PYGZhy{}11  30180288176  0.0221 22791.5311    0.0000    0.0000  
2023\PYGZhy{}03\PYGZhy{}12  29279035521  0.0742 22658.2711    0.0000    0.0000  
2023\PYGZhy{}03\PYGZhy{}13  49466362688  0.0918 22646.3301    0.0000    0.0000  
2023\PYGZhy{}03\PYGZhy{}14  54622230164  0.0227 22674.1916    1.0000    0.0227  
2023\PYGZhy{}03\PYGZhy{}15  44081737728 \PYGZhy{}0.0052 22707.6822    1.0000   \PYGZhy{}0.0052  
\end{sphinxVerbatim}

\end{sphinxuseclass}\end{sphinxVerbatimOutput}

\end{sphinxuseclass}
\begin{sphinxuseclass}{cell}\begin{sphinxVerbatimInput}

\begin{sphinxuseclass}{cell_input}
\begin{sphinxVerbatim}[commandchars=\\\{\}]
\PYG{n}{\PYGZus{}} \PYG{o}{=} \PYG{n}{btc\PYGZus{}sma}\PYG{p}{[}\PYG{p}{[}\PYG{l+s+s1}{\PYGZsq{}}\PYG{l+s+s1}{Return}\PYG{l+s+s1}{\PYGZsq{}}\PYG{p}{,} \PYG{l+s+s1}{\PYGZsq{}}\PYG{l+s+s1}{Strategy}\PYG{l+s+s1}{\PYGZsq{}}\PYG{p}{]}\PYG{p}{]}\PYG{o}{.}\PYG{n}{dropna}\PYG{p}{(}\PYG{p}{)}

\PYG{p}{(}
    \PYG{n}{\PYGZus{}}
    \PYG{o}{.}\PYG{n}{add}\PYG{p}{(}\PYG{l+m+mi}{1}\PYG{p}{)}
    \PYG{o}{.}\PYG{n}{cumprod}\PYG{p}{(}\PYG{p}{)}
    \PYG{o}{.}\PYG{n}{rename\PYGZus{}axis}\PYG{p}{(}\PYG{n}{columns}\PYG{o}{=}\PYG{l+s+s1}{\PYGZsq{}}\PYG{l+s+s1}{Strategy}\PYG{l+s+s1}{\PYGZsq{}}\PYG{p}{)}
    \PYG{o}{.}\PYG{n}{rename}\PYG{p}{(}\PYG{n}{columns}\PYG{o}{=}\PYG{p}{\PYGZob{}}\PYG{l+s+s1}{\PYGZsq{}}\PYG{l+s+s1}{Return}\PYG{l+s+s1}{\PYGZsq{}}\PYG{p}{:} \PYG{l+s+s1}{\PYGZsq{}}\PYG{l+s+s1}{Buy\PYGZhy{}And\PYGZhy{}Hold}\PYG{l+s+s1}{\PYGZsq{}}\PYG{p}{,} \PYG{l+s+s1}{\PYGZsq{}}\PYG{l+s+s1}{Strategy}\PYG{l+s+s1}{\PYGZsq{}}\PYG{p}{:} \PYG{l+s+s1}{\PYGZsq{}}\PYG{l+s+s1}{SMA(20)}\PYG{l+s+s1}{\PYGZsq{}}\PYG{p}{\PYGZcb{}}\PYG{p}{)}
    \PYG{o}{.}\PYG{n}{plot}\PYG{p}{(}\PYG{p}{)}
\PYG{p}{)}
\PYG{n}{plt}\PYG{o}{.}\PYG{n}{ylabel}\PYG{p}{(}\PYG{l+s+s1}{\PYGZsq{}}\PYG{l+s+s1}{Value (\PYGZdl{})}\PYG{l+s+s1}{\PYGZsq{}}\PYG{p}{)}
\PYG{n}{plt}\PYG{o}{.}\PYG{n}{title}\PYG{p}{(}\PYG{l+s+sa}{f}\PYG{l+s+s1}{\PYGZsq{}}\PYG{l+s+s1}{Value of \PYGZdl{}1 Invested at Close on }\PYG{l+s+si}{\PYGZob{}}\PYG{n}{\PYGZus{}}\PYG{o}{.}\PYG{n}{index}\PYG{p}{[}\PYG{l+m+mi}{0}\PYG{p}{]} \PYG{o}{\PYGZhy{}} \PYG{n}{pd}\PYG{o}{.}\PYG{n}{offsets}\PYG{o}{.}\PYG{n}{Day}\PYG{p}{(}\PYG{l+m+mi}{1}\PYG{p}{)}\PYG{l+s+si}{:}\PYG{l+s+s1}{\PYGZpc{}B \PYGZpc{}d, \PYGZpc{}Y}\PYG{l+s+si}{\PYGZcb{}}\PYG{l+s+s1}{\PYGZsq{}}\PYG{p}{)}
\PYG{n}{plt}\PYG{o}{.}\PYG{n}{show}\PYG{p}{(}\PYG{p}{)}
\end{sphinxVerbatim}

\end{sphinxuseclass}\end{sphinxVerbatimInput}
\begin{sphinxVerbatimOutput}

\begin{sphinxuseclass}{cell_output}
\noindent\sphinxincludegraphics{{0465375d524760e8f7e6f5cf74e7afd4fb8b2e32201f977ff1caf134a14848b0}.png}

\end{sphinxuseclass}\end{sphinxVerbatimOutput}

\end{sphinxuseclass}

\subsubsection{How does SMA(20) outperform buy\sphinxhyphen{}and\sphinxhyphen{}hold with this sample?}
\label{\detokenize{herron_02_practice_02:how-does-sma-20-outperform-buy-and-hold-with-this-sample}}
\sphinxAtStartPar
Consider the following:
\begin{enumerate}
\sphinxsetlistlabels{\arabic}{enumi}{enumii}{}{.}%
\item {} 
\sphinxAtStartPar
Does SMA(20) avoid the worst performing days? How many of the worst 20 days does SMA(20) avoid? Try the \sphinxcode{\sphinxupquote{.sort\_values()}} or \sphinxcode{\sphinxupquote{.nlargest()}} method.

\item {} 
\sphinxAtStartPar
Does SMA(20) preferentially avoid low\sphinxhyphen{}return days? Try to combine the \sphinxcode{\sphinxupquote{.groupby()}} method and \sphinxcode{\sphinxupquote{pd.qcut()}} function.

\item {} 
\sphinxAtStartPar
Does SMA(20) preferentially avoid high\sphinxhyphen{}volatility days? Try to combine the \sphinxcode{\sphinxupquote{.groupby()}} method and \sphinxcode{\sphinxupquote{pd.qcut()}} function.

\end{enumerate}

\sphinxAtStartPar
By chance, the SMA(20) strategy avoids all but three of the worst days.

\begin{sphinxuseclass}{cell}\begin{sphinxVerbatimInput}

\begin{sphinxuseclass}{cell_input}
\begin{sphinxVerbatim}[commandchars=\\\{\}]
\PYG{n}{btc\PYGZus{}sma}\PYG{o}{.}\PYG{n}{sort\PYGZus{}values}\PYG{p}{(}\PYG{l+s+s1}{\PYGZsq{}}\PYG{l+s+s1}{Return}\PYG{l+s+s1}{\PYGZsq{}}\PYG{p}{)}\PYG{p}{[}\PYG{p}{[}\PYG{l+s+s1}{\PYGZsq{}}\PYG{l+s+s1}{Position}\PYG{l+s+s1}{\PYGZsq{}}\PYG{p}{]}\PYG{p}{]}\PYG{o}{.}\PYG{n}{head}\PYG{p}{(}\PYG{l+m+mi}{20}\PYG{p}{)}\PYG{o}{.}\PYG{n}{value\PYGZus{}counts}\PYG{p}{(}\PYG{p}{)}
\end{sphinxVerbatim}

\end{sphinxuseclass}\end{sphinxVerbatimInput}
\begin{sphinxVerbatimOutput}

\begin{sphinxuseclass}{cell_output}
\begin{sphinxVerbatim}[commandchars=\\\{\}]
Position
0.0000      17
1.0000       3
dtype: int64
\end{sphinxVerbatim}

\end{sphinxuseclass}\end{sphinxVerbatimOutput}

\end{sphinxuseclass}
\sphinxAtStartPar
However, SMA(20) does not avoid the best days, again by chance.

\begin{sphinxuseclass}{cell}\begin{sphinxVerbatimInput}

\begin{sphinxuseclass}{cell_input}
\begin{sphinxVerbatim}[commandchars=\\\{\}]
\PYG{n}{btc\PYGZus{}sma}\PYG{o}{.}\PYG{n}{sort\PYGZus{}values}\PYG{p}{(}\PYG{l+s+s1}{\PYGZsq{}}\PYG{l+s+s1}{Return}\PYG{l+s+s1}{\PYGZsq{}}\PYG{p}{,} \PYG{n}{ascending}\PYG{o}{=}\PYG{k+kc}{False}\PYG{p}{)}\PYG{p}{[}\PYG{p}{[}\PYG{l+s+s1}{\PYGZsq{}}\PYG{l+s+s1}{Position}\PYG{l+s+s1}{\PYGZsq{}}\PYG{p}{]}\PYG{p}{]}\PYG{o}{.}\PYG{n}{head}\PYG{p}{(}\PYG{l+m+mi}{20}\PYG{p}{)}\PYG{o}{.}\PYG{n}{value\PYGZus{}counts}\PYG{p}{(}\PYG{p}{)}
\end{sphinxVerbatim}

\end{sphinxuseclass}\end{sphinxVerbatimInput}
\begin{sphinxVerbatimOutput}

\begin{sphinxuseclass}{cell_output}
\begin{sphinxVerbatim}[commandchars=\\\{\}]
Position
0.0000      10
1.0000      10
dtype: int64
\end{sphinxVerbatim}

\end{sphinxuseclass}\end{sphinxVerbatimOutput}

\end{sphinxuseclass}
\begin{sphinxuseclass}{cell}\begin{sphinxVerbatimInput}

\begin{sphinxuseclass}{cell_input}
\begin{sphinxVerbatim}[commandchars=\\\{\}]
\PYG{n}{btc\PYGZus{}sma}\PYG{o}{.}\PYG{n}{groupby}\PYG{p}{(}\PYG{l+s+s1}{\PYGZsq{}}\PYG{l+s+s1}{Position}\PYG{l+s+s1}{\PYGZsq{}}\PYG{p}{)}\PYG{p}{[}\PYG{p}{[}\PYG{l+s+s1}{\PYGZsq{}}\PYG{l+s+s1}{Return}\PYG{l+s+s1}{\PYGZsq{}}\PYG{p}{,} \PYG{l+s+s1}{\PYGZsq{}}\PYG{l+s+s1}{Strategy}\PYG{l+s+s1}{\PYGZsq{}}\PYG{p}{]}\PYG{p}{]}\PYG{o}{.}\PYG{n}{agg}\PYG{p}{(}\PYG{p}{[}\PYG{l+s+s1}{\PYGZsq{}}\PYG{l+s+s1}{mean}\PYG{l+s+s1}{\PYGZsq{}}\PYG{p}{,} \PYG{l+s+s1}{\PYGZsq{}}\PYG{l+s+s1}{std}\PYG{l+s+s1}{\PYGZsq{}}\PYG{p}{]}\PYG{p}{)}
\end{sphinxVerbatim}

\end{sphinxuseclass}\end{sphinxVerbatimInput}
\begin{sphinxVerbatimOutput}

\begin{sphinxuseclass}{cell_output}
\begin{sphinxVerbatim}[commandchars=\\\{\}]
Variable Return        Strategy       
           mean    std     mean    std
Position                              
0.0000   0.0004 0.0415   0.0000 0.0000
1.0000   0.0036 0.0349   0.0036 0.0349
\end{sphinxVerbatim}

\end{sphinxuseclass}\end{sphinxVerbatimOutput}

\end{sphinxuseclass}
\sphinxAtStartPar
The SMA(20) strategy has a slight edge in picking high\sphinxhyphen{}return days, again by chance.

\begin{sphinxuseclass}{cell}\begin{sphinxVerbatimInput}

\begin{sphinxuseclass}{cell_input}
\begin{sphinxVerbatim}[commandchars=\\\{\}]
\PYG{p}{(}
    \PYG{n}{btc\PYGZus{}sma}
    \PYG{o}{.}\PYG{n}{assign}\PYG{p}{(}\PYG{n}{q5\PYGZus{}return} \PYG{o}{=} \PYG{k}{lambda} \PYG{n}{x}\PYG{p}{:} \PYG{l+m+mi}{1} \PYG{o}{+} \PYG{n}{pd}\PYG{o}{.}\PYG{n}{qcut}\PYG{p}{(}\PYG{n}{x}\PYG{p}{[}\PYG{l+s+s1}{\PYGZsq{}}\PYG{l+s+s1}{Return}\PYG{l+s+s1}{\PYGZsq{}}\PYG{p}{]}\PYG{p}{,} \PYG{n}{q}\PYG{o}{=}\PYG{l+m+mi}{5}\PYG{p}{,} \PYG{n}{labels}\PYG{o}{=}\PYG{k+kc}{False}\PYG{p}{)}\PYG{p}{)}
    \PYG{o}{.}\PYG{n}{groupby}\PYG{p}{(}\PYG{l+s+s1}{\PYGZsq{}}\PYG{l+s+s1}{q5\PYGZus{}return}\PYG{l+s+s1}{\PYGZsq{}}\PYG{p}{)}
    \PYG{p}{[}\PYG{l+s+s1}{\PYGZsq{}}\PYG{l+s+s1}{Position}\PYG{l+s+s1}{\PYGZsq{}}\PYG{p}{]}
    \PYG{o}{.}\PYG{n}{mean}\PYG{p}{(}\PYG{p}{)}
    \PYG{o}{.}\PYG{n}{plot}\PYG{p}{(}\PYG{n}{kind}\PYG{o}{=}\PYG{l+s+s1}{\PYGZsq{}}\PYG{l+s+s1}{bar}\PYG{l+s+s1}{\PYGZsq{}}\PYG{p}{)}
\PYG{p}{)}

\PYG{n}{plt}\PYG{o}{.}\PYG{n}{xticks}\PYG{p}{(}\PYG{n}{rotation}\PYG{o}{=}\PYG{l+m+mi}{0}\PYG{p}{)}
\PYG{n}{plt}\PYG{o}{.}\PYG{n}{xlabel}\PYG{p}{(}\PYG{l+s+s1}{\PYGZsq{}}\PYG{l+s+s1}{Return Bin (1 is Lowest, 5 is Highest)}\PYG{l+s+s1}{\PYGZsq{}}\PYG{p}{)}
\PYG{n}{plt}\PYG{o}{.}\PYG{n}{ylabel}\PYG{p}{(}\PYG{l+s+s1}{\PYGZsq{}}\PYG{l+s+s1}{Fraction of Days Strategy is Long Bitcoin}\PYG{l+s+s1}{\PYGZsq{}}\PYG{p}{)}
\PYG{n}{plt}\PYG{o}{.}\PYG{n}{title}\PYG{p}{(}\PYG{l+s+s1}{\PYGZsq{}}\PYG{l+s+s1}{Mean Position by Return Bin}\PYG{l+s+s1}{\PYGZsq{}}\PYG{p}{)}
\PYG{n}{plt}\PYG{o}{.}\PYG{n}{show}\PYG{p}{(}\PYG{p}{)}
\end{sphinxVerbatim}

\end{sphinxuseclass}\end{sphinxVerbatimInput}
\begin{sphinxVerbatimOutput}

\begin{sphinxuseclass}{cell_output}
\noindent\sphinxincludegraphics{{77fcce6036e279a25a7c191e860fc4caa4bd9ab43c234b7935bfcdb356dabc6a}.png}

\end{sphinxuseclass}\end{sphinxVerbatimOutput}

\end{sphinxuseclass}
\sphinxAtStartPar
However, the SMA(20) \sphinxstyleemphasis{does} avoid the high volatility days that create \sphinxhref{https://www.kitces.com/blog/volatility-drag-variance-drain-mean-arithmetic-vs-geometric-average-investment-returns/}{volatility drag}.

\begin{sphinxuseclass}{cell}\begin{sphinxVerbatimInput}

\begin{sphinxuseclass}{cell_input}
\begin{sphinxVerbatim}[commandchars=\\\{\}]
\PYG{p}{(}
    \PYG{n}{btc\PYGZus{}sma}
    \PYG{o}{.}\PYG{n}{assign}\PYG{p}{(}
        \PYG{n}{Volatility} \PYG{o}{=} \PYG{k}{lambda} \PYG{n}{x}\PYG{p}{:} \PYG{n}{x}\PYG{p}{[}\PYG{l+s+s1}{\PYGZsq{}}\PYG{l+s+s1}{Return}\PYG{l+s+s1}{\PYGZsq{}}\PYG{p}{]}\PYG{o}{.}\PYG{n}{rolling}\PYG{p}{(}\PYG{l+m+mi}{63}\PYG{p}{)}\PYG{o}{.}\PYG{n}{std}\PYG{p}{(}\PYG{p}{)}\PYG{p}{,}
        \PYG{n}{q5\PYGZus{}volatility} \PYG{o}{=} \PYG{k}{lambda} \PYG{n}{x}\PYG{p}{:} \PYG{l+m+mi}{1} \PYG{o}{+} \PYG{n}{pd}\PYG{o}{.}\PYG{n}{qcut}\PYG{p}{(}\PYG{n}{x}\PYG{p}{[}\PYG{l+s+s1}{\PYGZsq{}}\PYG{l+s+s1}{Volatility}\PYG{l+s+s1}{\PYGZsq{}}\PYG{p}{]}\PYG{p}{,} \PYG{n}{q}\PYG{o}{=}\PYG{l+m+mi}{5}\PYG{p}{,} \PYG{n}{labels}\PYG{o}{=}\PYG{k+kc}{False}\PYG{p}{)}
    \PYG{p}{)}
    \PYG{o}{.}\PYG{n}{groupby}\PYG{p}{(}\PYG{l+s+s1}{\PYGZsq{}}\PYG{l+s+s1}{q5\PYGZus{}volatility}\PYG{l+s+s1}{\PYGZsq{}}\PYG{p}{)}
    \PYG{p}{[}\PYG{l+s+s1}{\PYGZsq{}}\PYG{l+s+s1}{Position}\PYG{l+s+s1}{\PYGZsq{}}\PYG{p}{]}
    \PYG{o}{.}\PYG{n}{mean}\PYG{p}{(}\PYG{p}{)}
    \PYG{o}{.}\PYG{n}{plot}\PYG{p}{(}\PYG{n}{kind}\PYG{o}{=}\PYG{l+s+s1}{\PYGZsq{}}\PYG{l+s+s1}{bar}\PYG{l+s+s1}{\PYGZsq{}}\PYG{p}{)}
\PYG{p}{)}

\PYG{n}{plt}\PYG{o}{.}\PYG{n}{xticks}\PYG{p}{(}\PYG{n}{rotation}\PYG{o}{=}\PYG{l+m+mi}{0}\PYG{p}{)}
\PYG{n}{plt}\PYG{o}{.}\PYG{n}{xlabel}\PYG{p}{(}\PYG{l+s+s1}{\PYGZsq{}}\PYG{l+s+s1}{63\PYGZhy{}Day Rolling Volatility Bin (1 is Lowest, 5 is Highest)}\PYG{l+s+s1}{\PYGZsq{}}\PYG{p}{)}
\PYG{n}{plt}\PYG{o}{.}\PYG{n}{ylabel}\PYG{p}{(}\PYG{l+s+s1}{\PYGZsq{}}\PYG{l+s+s1}{Fraction of Days Strategy is Long Bitcoin}\PYG{l+s+s1}{\PYGZsq{}}\PYG{p}{)}
\PYG{n}{plt}\PYG{o}{.}\PYG{n}{title}\PYG{p}{(}\PYG{l+s+s1}{\PYGZsq{}}\PYG{l+s+s1}{Mean Position by 63\PYGZhy{}Day Rolling Volatility Bin}\PYG{l+s+s1}{\PYGZsq{}}\PYG{p}{)}
\PYG{n}{plt}\PYG{o}{.}\PYG{n}{show}\PYG{p}{(}\PYG{p}{)}
\end{sphinxVerbatim}

\end{sphinxuseclass}\end{sphinxVerbatimInput}
\begin{sphinxVerbatimOutput}

\begin{sphinxuseclass}{cell_output}
\noindent\sphinxincludegraphics{{63059ddf0e219dc18f519507934fef116e57ea00861bca604dcdc2281f99d135}.png}

\end{sphinxuseclass}\end{sphinxVerbatimOutput}

\end{sphinxuseclass}
\sphinxAtStartPar
Recall that \$Arith\textbackslash{} Mean \textbackslash{}approx Geom\textbackslash{} Mean + \textbackslash{}frac\{\textbackslash{}sigma\textasciicircum{}2\}\{2\}\$, so avoiding high volatility (high variance) days, reduced the drag on the  cumulative returns that intermediate\sphinxhyphen{}term and long\sphinxhyphen{}term investors care about!

\begin{sphinxuseclass}{cell}\begin{sphinxVerbatimInput}

\begin{sphinxuseclass}{cell_input}
\begin{sphinxVerbatim}[commandchars=\\\{\}]
\PYG{p}{(}
    \PYG{n}{btc\PYGZus{}sma}
    \PYG{o}{.}\PYG{n}{groupby}\PYG{p}{(}\PYG{l+s+s1}{\PYGZsq{}}\PYG{l+s+s1}{Position}\PYG{l+s+s1}{\PYGZsq{}}\PYG{p}{)}
    \PYG{p}{[}\PYG{l+s+s1}{\PYGZsq{}}\PYG{l+s+s1}{Return}\PYG{l+s+s1}{\PYGZsq{}}\PYG{p}{]}
    \PYG{o}{.}\PYG{n}{agg}\PYG{p}{(}\PYG{p}{[}\PYG{l+s+s1}{\PYGZsq{}}\PYG{l+s+s1}{std}\PYG{l+s+s1}{\PYGZsq{}}\PYG{p}{,} \PYG{l+s+s1}{\PYGZsq{}}\PYG{l+s+s1}{mean}\PYG{l+s+s1}{\PYGZsq{}}\PYG{p}{,} \PYG{k}{lambda} \PYG{n}{x}\PYG{p}{:} \PYG{p}{(}\PYG{l+m+mi}{1} \PYG{o}{+} \PYG{n}{x}\PYG{p}{)}\PYG{o}{.}\PYG{n}{prod}\PYG{p}{(}\PYG{p}{)}\PYG{o}{*}\PYG{o}{*}\PYG{p}{(}\PYG{l+m+mi}{1} \PYG{o}{/} \PYG{n}{x}\PYG{o}{.}\PYG{n}{count}\PYG{p}{(}\PYG{p}{)}\PYG{p}{)} \PYG{o}{\PYGZhy{}} \PYG{l+m+mi}{1}\PYG{p}{]}\PYG{p}{)}
    \PYG{o}{.}\PYG{n}{mul}\PYG{p}{(}\PYG{l+m+mi}{100}\PYG{p}{)}
    \PYG{o}{.}\PYG{n}{rename}\PYG{p}{(}\PYG{n}{columns}\PYG{o}{=}\PYG{p}{\PYGZob{}}\PYG{l+s+s1}{\PYGZsq{}}\PYG{l+s+s1}{std}\PYG{l+s+s1}{\PYGZsq{}}\PYG{p}{:} \PYG{l+s+s1}{\PYGZsq{}}\PYG{l+s+s1}{Volatility}\PYG{l+s+s1}{\PYGZsq{}}\PYG{p}{,} \PYG{l+s+s1}{\PYGZsq{}}\PYG{l+s+s1}{mean}\PYG{l+s+s1}{\PYGZsq{}}\PYG{p}{:} \PYG{l+s+s1}{\PYGZsq{}}\PYG{l+s+s1}{Arith Mean}\PYG{l+s+s1}{\PYGZsq{}}\PYG{p}{,} \PYG{l+s+s1}{\PYGZsq{}}\PYG{l+s+s1}{\PYGZlt{}lambda\PYGZus{}0\PYGZgt{}}\PYG{l+s+s1}{\PYGZsq{}}\PYG{p}{:} \PYG{l+s+s1}{\PYGZsq{}}\PYG{l+s+s1}{Geom Mean}\PYG{l+s+s1}{\PYGZsq{}}\PYG{p}{\PYGZcb{}}\PYG{p}{)}
\PYG{p}{)}
\end{sphinxVerbatim}

\end{sphinxuseclass}\end{sphinxVerbatimInput}
\begin{sphinxVerbatimOutput}

\begin{sphinxuseclass}{cell_output}
\begin{sphinxVerbatim}[commandchars=\\\{\}]
          Volatility  Arith Mean  Geom Mean
Position                                   
0.0000        4.1518      0.0433    \PYGZhy{}0.0452
1.0000        3.4901      0.3553     0.2952
\end{sphinxVerbatim}

\end{sphinxuseclass}\end{sphinxVerbatimOutput}

\end{sphinxuseclass}

\subsubsection{Implement the SMA(20) strategy with the market factor from French}
\label{\detokenize{herron_02_practice_02:implement-the-sma-20-strategy-with-the-market-factor-from-french}}
\sphinxAtStartPar
We need to impute a market price before we calculate SMA(20).

\begin{sphinxuseclass}{cell}\begin{sphinxVerbatimInput}

\begin{sphinxuseclass}{cell_input}
\begin{sphinxVerbatim}[commandchars=\\\{\}]
\PYG{k+kn}{import} \PYG{n+nn}{pandas\PYGZus{}datareader} \PYG{k}{as} \PYG{n+nn}{pdr}
\PYG{k+kn}{import} \PYG{n+nn}{requests\PYGZus{}cache}
\PYG{n}{session} \PYG{o}{=} \PYG{n}{requests\PYGZus{}cache}\PYG{o}{.}\PYG{n}{CachedSession}\PYG{p}{(}\PYG{p}{)}
\end{sphinxVerbatim}

\end{sphinxuseclass}\end{sphinxVerbatimInput}

\end{sphinxuseclass}
\begin{sphinxuseclass}{cell}\begin{sphinxVerbatimInput}

\begin{sphinxuseclass}{cell_input}
\begin{sphinxVerbatim}[commandchars=\\\{\}]
\PYG{n}{ff} \PYG{o}{=} \PYG{p}{(}
    \PYG{n}{pdr}\PYG{o}{.}\PYG{n}{DataReader}\PYG{p}{(}
        \PYG{n}{name}\PYG{o}{=}\PYG{l+s+s1}{\PYGZsq{}}\PYG{l+s+s1}{F\PYGZhy{}F\PYGZus{}Research\PYGZus{}Data\PYGZus{}Factors\PYGZus{}daily}\PYG{l+s+s1}{\PYGZsq{}}\PYG{p}{,}
        \PYG{n}{data\PYGZus{}source}\PYG{o}{=}\PYG{l+s+s1}{\PYGZsq{}}\PYG{l+s+s1}{famafrench}\PYG{l+s+s1}{\PYGZsq{}}\PYG{p}{,}
        \PYG{n}{start}\PYG{o}{=}\PYG{l+s+s1}{\PYGZsq{}}\PYG{l+s+s1}{1900}\PYG{l+s+s1}{\PYGZsq{}}\PYG{p}{,}
        \PYG{n}{session}\PYG{o}{=}\PYG{n}{session}
    \PYG{p}{)}
    \PYG{p}{[}\PYG{l+m+mi}{0}\PYG{p}{]}
    \PYG{o}{.}\PYG{n}{div}\PYG{p}{(}\PYG{l+m+mi}{100}\PYG{p}{)}
    \PYG{o}{.}\PYG{n}{assign}\PYG{p}{(}
        \PYG{n}{Mkt} \PYG{o}{=} \PYG{k}{lambda} \PYG{n}{x}\PYG{p}{:} \PYG{n}{x}\PYG{p}{[}\PYG{l+s+s1}{\PYGZsq{}}\PYG{l+s+s1}{Mkt\PYGZhy{}RF}\PYG{l+s+s1}{\PYGZsq{}}\PYG{p}{]} \PYG{o}{+} \PYG{n}{x}\PYG{p}{[}\PYG{l+s+s1}{\PYGZsq{}}\PYG{l+s+s1}{RF}\PYG{l+s+s1}{\PYGZsq{}}\PYG{p}{]}\PYG{p}{,}
        \PYG{n}{Price} \PYG{o}{=} \PYG{k}{lambda} \PYG{n}{x}\PYG{p}{:} \PYG{n}{x}\PYG{p}{[}\PYG{l+s+s1}{\PYGZsq{}}\PYG{l+s+s1}{Mkt}\PYG{l+s+s1}{\PYGZsq{}}\PYG{p}{]}\PYG{o}{.}\PYG{n}{add}\PYG{p}{(}\PYG{l+m+mi}{1}\PYG{p}{)}\PYG{o}{.}\PYG{n}{cumprod}\PYG{p}{(}\PYG{p}{)}
    \PYG{p}{)}
\PYG{p}{)}
\end{sphinxVerbatim}

\end{sphinxuseclass}\end{sphinxVerbatimInput}

\end{sphinxuseclass}
\begin{sphinxuseclass}{cell}\begin{sphinxVerbatimInput}

\begin{sphinxuseclass}{cell_input}
\begin{sphinxVerbatim}[commandchars=\\\{\}]
\PYG{n}{ff\PYGZus{}sma} \PYG{o}{=} \PYG{p}{(}
    \PYG{n}{ff}
    \PYG{o}{.}\PYG{n}{rename}\PYG{p}{(}\PYG{n}{columns}\PYG{o}{=}\PYG{p}{\PYGZob{}}\PYG{l+s+s1}{\PYGZsq{}}\PYG{l+s+s1}{Price}\PYG{l+s+s1}{\PYGZsq{}}\PYG{p}{:} \PYG{l+s+s1}{\PYGZsq{}}\PYG{l+s+s1}{Adj Close}\PYG{l+s+s1}{\PYGZsq{}}\PYG{p}{\PYGZcb{}}\PYG{p}{)}
    \PYG{o}{.}\PYG{n}{pipe}\PYG{p}{(}\PYG{n}{sma}\PYG{p}{,} \PYG{n}{n}\PYG{o}{=}\PYG{l+m+mi}{20}\PYG{p}{)}
\PYG{p}{)}

\PYG{n}{ff\PYGZus{}sma}\PYG{o}{.}\PYG{n}{tail}\PYG{p}{(}\PYG{p}{)}
\end{sphinxVerbatim}

\end{sphinxuseclass}\end{sphinxVerbatimInput}
\begin{sphinxVerbatimOutput}

\begin{sphinxuseclass}{cell_output}
\begin{sphinxVerbatim}[commandchars=\\\{\}]
            Mkt\PYGZhy{}RF     SMB     HML     RF     Mkt  Adj Close  Return  \PYGZbs{}
Date                                                                   
2022\PYGZhy{}12\PYGZhy{}23  0.0051 \PYGZhy{}0.0060  0.0115 0.0002  0.0053  9266.3996  0.0053   
2022\PYGZhy{}12\PYGZhy{}27 \PYGZhy{}0.0051 \PYGZhy{}0.0073  0.0142 0.0002 \PYGZhy{}0.0049  9220.6236 \PYGZhy{}0.0049   
2022\PYGZhy{}12\PYGZhy{}28 \PYGZhy{}0.0123 \PYGZhy{}0.0025 \PYGZhy{}0.0029 0.0002 \PYGZhy{}0.0121  9108.6852 \PYGZhy{}0.0121   
2022\PYGZhy{}12\PYGZhy{}29  0.0187  0.0127 \PYGZhy{}0.0107 0.0002  0.0189  9280.4750  0.0189   
2022\PYGZhy{}12\PYGZhy{}30 \PYGZhy{}0.0022  0.0011 \PYGZhy{}0.0003 0.0002 \PYGZhy{}0.0020  9261.5429 \PYGZhy{}0.0020   

                 SMA  Position  Strategy  
Date                                      
2022\PYGZhy{}12\PYGZhy{}23 9515.5406    0.0000    0.0000  
2022\PYGZhy{}12\PYGZhy{}27 9497.8127    0.0000   \PYGZhy{}0.0000  
2022\PYGZhy{}12\PYGZhy{}28 9475.2827    0.0000   \PYGZhy{}0.0000  
2022\PYGZhy{}12\PYGZhy{}29 9446.3627    0.0000    0.0000  
2022\PYGZhy{}12\PYGZhy{}30 9416.5159    0.0000   \PYGZhy{}0.0000  
\end{sphinxVerbatim}

\end{sphinxuseclass}\end{sphinxVerbatimOutput}

\end{sphinxuseclass}
\begin{sphinxuseclass}{cell}\begin{sphinxVerbatimInput}

\begin{sphinxuseclass}{cell_input}
\begin{sphinxVerbatim}[commandchars=\\\{\}]
\PYG{n}{\PYGZus{}} \PYG{o}{=} \PYG{n}{ff\PYGZus{}sma}\PYG{p}{[}\PYG{p}{[}\PYG{l+s+s1}{\PYGZsq{}}\PYG{l+s+s1}{Return}\PYG{l+s+s1}{\PYGZsq{}}\PYG{p}{,} \PYG{l+s+s1}{\PYGZsq{}}\PYG{l+s+s1}{Strategy}\PYG{l+s+s1}{\PYGZsq{}}\PYG{p}{]}\PYG{p}{]}\PYG{o}{.}\PYG{n}{dropna}\PYG{p}{(}\PYG{p}{)}

\PYG{p}{(}
    \PYG{n}{\PYGZus{}}
    \PYG{o}{.}\PYG{n}{add}\PYG{p}{(}\PYG{l+m+mi}{1}\PYG{p}{)}
    \PYG{o}{.}\PYG{n}{cumprod}\PYG{p}{(}\PYG{p}{)}
    \PYG{o}{.}\PYG{n}{rename\PYGZus{}axis}\PYG{p}{(}\PYG{n}{columns}\PYG{o}{=}\PYG{l+s+s1}{\PYGZsq{}}\PYG{l+s+s1}{Strategy}\PYG{l+s+s1}{\PYGZsq{}}\PYG{p}{)}
    \PYG{o}{.}\PYG{n}{rename}\PYG{p}{(}\PYG{n}{columns}\PYG{o}{=}\PYG{p}{\PYGZob{}}\PYG{l+s+s1}{\PYGZsq{}}\PYG{l+s+s1}{Return}\PYG{l+s+s1}{\PYGZsq{}}\PYG{p}{:} \PYG{l+s+s1}{\PYGZsq{}}\PYG{l+s+s1}{Buy\PYGZhy{}And\PYGZhy{}Hold}\PYG{l+s+s1}{\PYGZsq{}}\PYG{p}{,} \PYG{l+s+s1}{\PYGZsq{}}\PYG{l+s+s1}{Strategy}\PYG{l+s+s1}{\PYGZsq{}}\PYG{p}{:} \PYG{l+s+s1}{\PYGZsq{}}\PYG{l+s+s1}{SMA(20)}\PYG{l+s+s1}{\PYGZsq{}}\PYG{p}{\PYGZcb{}}\PYG{p}{)}
    \PYG{o}{.}\PYG{n}{plot}\PYG{p}{(}\PYG{p}{)}
\PYG{p}{)}
\PYG{n}{plt}\PYG{o}{.}\PYG{n}{ylabel}\PYG{p}{(}\PYG{l+s+s1}{\PYGZsq{}}\PYG{l+s+s1}{Value (\PYGZdl{})}\PYG{l+s+s1}{\PYGZsq{}}\PYG{p}{)}
\PYG{n}{plt}\PYG{o}{.}\PYG{n}{title}\PYG{p}{(}\PYG{l+s+sa}{f}\PYG{l+s+s1}{\PYGZsq{}}\PYG{l+s+s1}{Value of \PYGZdl{}1 Invested in Market at Close on }\PYG{l+s+si}{\PYGZob{}}\PYG{n}{\PYGZus{}}\PYG{o}{.}\PYG{n}{index}\PYG{p}{[}\PYG{l+m+mi}{0}\PYG{p}{]} \PYG{o}{\PYGZhy{}} \PYG{n}{pd}\PYG{o}{.}\PYG{n}{offsets}\PYG{o}{.}\PYG{n}{Day}\PYG{p}{(}\PYG{l+m+mi}{1}\PYG{p}{)}\PYG{l+s+si}{:}\PYG{l+s+s1}{\PYGZpc{}B \PYGZpc{}d, \PYGZpc{}Y}\PYG{l+s+si}{\PYGZcb{}}\PYG{l+s+s1}{\PYGZsq{}}\PYG{p}{)}
\PYG{n}{plt}\PYG{o}{.}\PYG{n}{show}\PYG{p}{(}\PYG{p}{)}
\end{sphinxVerbatim}

\end{sphinxuseclass}\end{sphinxVerbatimInput}
\begin{sphinxVerbatimOutput}

\begin{sphinxuseclass}{cell_output}
\noindent\sphinxincludegraphics{{131e3565948329ffbfb65b3360199ffe069c47f7c99b79b2edd04f4debb4d99e}.png}

\end{sphinxuseclass}\end{sphinxVerbatimOutput}

\end{sphinxuseclass}

\subsubsection{How often does SMA(20) outperform buy\sphinxhyphen{}and\sphinxhyphen{}hold with 10\sphinxhyphen{}year rolling windows?}
\label{\detokenize{herron_02_practice_02:how-often-does-sma-20-outperform-buy-and-hold-with-10-year-rolling-windows}}
\begin{sphinxuseclass}{cell}\begin{sphinxVerbatimInput}

\begin{sphinxuseclass}{cell_input}
\begin{sphinxVerbatim}[commandchars=\\\{\}]
\PYG{p}{(}
    \PYG{n}{ff\PYGZus{}sma}
    \PYG{p}{[}\PYG{p}{[}\PYG{l+s+s1}{\PYGZsq{}}\PYG{l+s+s1}{Return}\PYG{l+s+s1}{\PYGZsq{}}\PYG{p}{,} \PYG{l+s+s1}{\PYGZsq{}}\PYG{l+s+s1}{Strategy}\PYG{l+s+s1}{\PYGZsq{}}\PYG{p}{]}\PYG{p}{]}
    \PYG{o}{.}\PYG{n}{rolling}\PYG{p}{(}\PYG{l+m+mi}{10} \PYG{o}{*} \PYG{l+m+mi}{252}\PYG{p}{)}
    \PYG{o}{.}\PYG{n}{apply}\PYG{p}{(}\PYG{k}{lambda} \PYG{n}{x}\PYG{p}{:} \PYG{p}{(}\PYG{l+m+mi}{1} \PYG{o}{+} \PYG{n}{x}\PYG{p}{)}\PYG{o}{.}\PYG{n}{prod}\PYG{p}{(}\PYG{p}{)}\PYG{p}{)}
    \PYG{o}{.}\PYG{n}{rename\PYGZus{}axis}\PYG{p}{(}\PYG{n}{columns}\PYG{o}{=}\PYG{l+s+s1}{\PYGZsq{}}\PYG{l+s+s1}{Strategy}\PYG{l+s+s1}{\PYGZsq{}}\PYG{p}{)}
    \PYG{o}{.}\PYG{n}{rename}\PYG{p}{(}\PYG{n}{columns}\PYG{o}{=}\PYG{p}{\PYGZob{}}\PYG{l+s+s1}{\PYGZsq{}}\PYG{l+s+s1}{Return}\PYG{l+s+s1}{\PYGZsq{}}\PYG{p}{:} \PYG{l+s+s1}{\PYGZsq{}}\PYG{l+s+s1}{Buy\PYGZhy{}And\PYGZhy{}Hold}\PYG{l+s+s1}{\PYGZsq{}}\PYG{p}{,} \PYG{l+s+s1}{\PYGZsq{}}\PYG{l+s+s1}{Strategy}\PYG{l+s+s1}{\PYGZsq{}}\PYG{p}{:} \PYG{l+s+s1}{\PYGZsq{}}\PYG{l+s+s1}{SMA(20)}\PYG{l+s+s1}{\PYGZsq{}}\PYG{p}{\PYGZcb{}}\PYG{p}{)}
    \PYG{o}{.}\PYG{n}{plot}\PYG{p}{(}\PYG{p}{)}
\PYG{p}{)}
\PYG{n}{plt}\PYG{o}{.}\PYG{n}{ylabel}\PYG{p}{(}\PYG{l+s+s1}{\PYGZsq{}}\PYG{l+s+s1}{Value (\PYGZdl{})}\PYG{l+s+s1}{\PYGZsq{}}\PYG{p}{)}
\PYG{n}{plt}\PYG{o}{.}\PYG{n}{title}\PYG{p}{(}\PYG{l+s+sa}{f}\PYG{l+s+s1}{\PYGZsq{}}\PYG{l+s+s1}{Value of \PYGZdl{}1 Investments for Rolling 10\PYGZhy{}Year Holding Periods }\PYG{l+s+s1}{\PYGZsq{}}\PYG{p}{)}
\PYG{n}{plt}\PYG{o}{.}\PYG{n}{show}\PYG{p}{(}\PYG{p}{)}
\end{sphinxVerbatim}

\end{sphinxuseclass}\end{sphinxVerbatimInput}
\begin{sphinxVerbatimOutput}

\begin{sphinxuseclass}{cell_output}
\noindent\sphinxincludegraphics{{9f8dc1a1b39e0f46b1d4fda08ef1b2a8f7ae3e2ae45f32a52dcfe96994401f08}.png}

\end{sphinxuseclass}\end{sphinxVerbatimOutput}

\end{sphinxuseclass}
\sphinxAtStartPar
In the previous example, SMA(20) looks amazing!
But over many shorter holding periods, we see the two are comparable.
This is largely because the SMA(20) strategy \sphinxstyleemphasis{by pure chance} avoids big market draw downs!

\begin{sphinxuseclass}{cell}\begin{sphinxVerbatimInput}

\begin{sphinxuseclass}{cell_input}
\begin{sphinxVerbatim}[commandchars=\\\{\}]
\PYG{n}{ff\PYGZus{}sma}\PYG{o}{.}\PYG{n}{sort\PYGZus{}values}\PYG{p}{(}\PYG{l+s+s1}{\PYGZsq{}}\PYG{l+s+s1}{Return}\PYG{l+s+s1}{\PYGZsq{}}\PYG{p}{)}\PYG{p}{[}\PYG{p}{[}\PYG{l+s+s1}{\PYGZsq{}}\PYG{l+s+s1}{Position}\PYG{l+s+s1}{\PYGZsq{}}\PYG{p}{,} \PYG{l+s+s1}{\PYGZsq{}}\PYG{l+s+s1}{Return}\PYG{l+s+s1}{\PYGZsq{}}\PYG{p}{,} \PYG{l+s+s1}{\PYGZsq{}}\PYG{l+s+s1}{Strategy}\PYG{l+s+s1}{\PYGZsq{}}\PYG{p}{]}\PYG{p}{]}\PYG{o}{.}\PYG{n}{head}\PYG{p}{(}\PYG{l+m+mi}{10}\PYG{p}{)}
\end{sphinxVerbatim}

\end{sphinxuseclass}\end{sphinxVerbatimInput}
\begin{sphinxVerbatimOutput}

\begin{sphinxuseclass}{cell_output}
\begin{sphinxVerbatim}[commandchars=\\\{\}]
            Position  Return  Strategy
Date                                  
1987\PYGZhy{}10\PYGZhy{}19    0.0000 \PYGZhy{}0.1741   \PYGZhy{}0.0000
2020\PYGZhy{}03\PYGZhy{}16    0.0000 \PYGZhy{}0.1199   \PYGZhy{}0.0000
1929\PYGZhy{}10\PYGZhy{}29    0.0000 \PYGZhy{}0.1199   \PYGZhy{}0.0000
1929\PYGZhy{}10\PYGZhy{}28    0.0000 \PYGZhy{}0.1127   \PYGZhy{}0.0000
1929\PYGZhy{}11\PYGZhy{}06    0.0000 \PYGZhy{}0.0973   \PYGZhy{}0.0000
2020\PYGZhy{}03\PYGZhy{}12    0.0000 \PYGZhy{}0.0962   \PYGZhy{}0.0000
1933\PYGZhy{}07\PYGZhy{}21    0.0000 \PYGZhy{}0.0921   \PYGZhy{}0.0000
2008\PYGZhy{}12\PYGZhy{}01    1.0000 \PYGZhy{}0.0895   \PYGZhy{}0.0895
2008\PYGZhy{}10\PYGZhy{}15    0.0000 \PYGZhy{}0.0878   \PYGZhy{}0.0000
1933\PYGZhy{}07\PYGZhy{}20    1.0000 \PYGZhy{}0.0849   \PYGZhy{}0.0849
\end{sphinxVerbatim}

\end{sphinxuseclass}\end{sphinxVerbatimOutput}

\end{sphinxuseclass}
\sphinxAtStartPar
SMA(20) also avoids the up days.
However, for this sample, missing the extreme down days helps more than missing the extreme updays hurts.

\begin{sphinxuseclass}{cell}\begin{sphinxVerbatimInput}

\begin{sphinxuseclass}{cell_input}
\begin{sphinxVerbatim}[commandchars=\\\{\}]
\PYG{n}{ff\PYGZus{}sma}\PYG{o}{.}\PYG{n}{sort\PYGZus{}values}\PYG{p}{(}\PYG{l+s+s1}{\PYGZsq{}}\PYG{l+s+s1}{Return}\PYG{l+s+s1}{\PYGZsq{}}\PYG{p}{,} \PYG{n}{ascending}\PYG{o}{=}\PYG{k+kc}{False}\PYG{p}{)}\PYG{p}{[}\PYG{p}{[}\PYG{l+s+s1}{\PYGZsq{}}\PYG{l+s+s1}{Position}\PYG{l+s+s1}{\PYGZsq{}}\PYG{p}{,} \PYG{l+s+s1}{\PYGZsq{}}\PYG{l+s+s1}{Return}\PYG{l+s+s1}{\PYGZsq{}}\PYG{p}{,} \PYG{l+s+s1}{\PYGZsq{}}\PYG{l+s+s1}{Strategy}\PYG{l+s+s1}{\PYGZsq{}}\PYG{p}{]}\PYG{p}{]}\PYG{o}{.}\PYG{n}{head}\PYG{p}{(}\PYG{l+m+mi}{10}\PYG{p}{)}
\end{sphinxVerbatim}

\end{sphinxuseclass}\end{sphinxVerbatimInput}
\begin{sphinxVerbatimOutput}

\begin{sphinxuseclass}{cell_output}
\begin{sphinxVerbatim}[commandchars=\\\{\}]
            Position  Return  Strategy
Date                                  
1933\PYGZhy{}03\PYGZhy{}15    0.0000  0.1576    0.0000
1929\PYGZhy{}10\PYGZhy{}30    0.0000  0.1218    0.0000
2008\PYGZhy{}10\PYGZhy{}13    0.0000  0.1135    0.0000
1931\PYGZhy{}10\PYGZhy{}06    0.0000  0.1116    0.0000
1932\PYGZhy{}09\PYGZhy{}21    0.0000  0.1096    0.0000
2008\PYGZhy{}10\PYGZhy{}28    0.0000  0.0977    0.0000
2020\PYGZhy{}03\PYGZhy{}24    0.0000  0.0935    0.0000
2020\PYGZhy{}03\PYGZhy{}13    0.0000  0.0897    0.0000
1939\PYGZhy{}09\PYGZhy{}05    1.0000  0.0879    0.0879
1937\PYGZhy{}10\PYGZhy{}20    0.0000  0.0867    0.0000
\end{sphinxVerbatim}

\end{sphinxuseclass}\end{sphinxVerbatimOutput}

\end{sphinxuseclass}
\sphinxAtStartPar
We can also think about this problem by decade.
If we want to get proper calendar decades (instead of 10\sphinxhyphen{}year periods that start in 1926), we combine \sphinxcode{\sphinxupquote{.groupby()}} with an anonymous function that converts the date\sphinxhyphen{}time index to a proper calendar decade.
Again, we see that SMA(20) and buy\sphinxhyphen{}and\sphinxhyphen{}hold trade wins, but SMA(20) wins bigs in the 1930s!

\begin{sphinxuseclass}{cell}\begin{sphinxVerbatimInput}

\begin{sphinxuseclass}{cell_input}
\begin{sphinxVerbatim}[commandchars=\\\{\}]
\PYG{p}{(}
    \PYG{n}{ff\PYGZus{}sma}
    \PYG{p}{[}\PYG{p}{[}\PYG{l+s+s1}{\PYGZsq{}}\PYG{l+s+s1}{Return}\PYG{l+s+s1}{\PYGZsq{}}\PYG{p}{,} \PYG{l+s+s1}{\PYGZsq{}}\PYG{l+s+s1}{Strategy}\PYG{l+s+s1}{\PYGZsq{}}\PYG{p}{]}\PYG{p}{]}
    \PYG{o}{.}\PYG{n}{groupby}\PYG{p}{(}\PYG{k}{lambda} \PYG{n}{x}\PYG{p}{:} \PYG{l+s+sa}{f}\PYG{l+s+s1}{\PYGZsq{}}\PYG{l+s+si}{\PYGZob{}}\PYG{p}{(}\PYG{n}{x}\PYG{o}{.}\PYG{n}{year} \PYG{o}{/}\PYG{o}{/} \PYG{l+m+mi}{10}\PYG{p}{)} \PYG{o}{*} \PYG{l+m+mi}{10}\PYG{l+s+si}{\PYGZcb{}}\PYG{l+s+s1}{s}\PYG{l+s+s1}{\PYGZsq{}}\PYG{p}{)}
    \PYG{o}{.}\PYG{n}{apply}\PYG{p}{(}\PYG{k}{lambda} \PYG{n}{x}\PYG{p}{:} \PYG{p}{(}\PYG{l+m+mi}{1} \PYG{o}{+} \PYG{n}{x}\PYG{p}{)}\PYG{o}{.}\PYG{n}{prod}\PYG{p}{(}\PYG{p}{)}\PYG{p}{)}
    \PYG{o}{.}\PYG{n}{rename\PYGZus{}axis}\PYG{p}{(}\PYG{n}{index}\PYG{o}{=}\PYG{l+s+s1}{\PYGZsq{}}\PYG{l+s+s1}{Decade}\PYG{l+s+s1}{\PYGZsq{}}\PYG{p}{,} \PYG{n}{columns}\PYG{o}{=}\PYG{l+s+s1}{\PYGZsq{}}\PYG{l+s+s1}{Strategy}\PYG{l+s+s1}{\PYGZsq{}}\PYG{p}{)}
    \PYG{o}{.}\PYG{n}{rename}\PYG{p}{(}\PYG{n}{columns}\PYG{o}{=}\PYG{p}{\PYGZob{}}\PYG{l+s+s1}{\PYGZsq{}}\PYG{l+s+s1}{Return}\PYG{l+s+s1}{\PYGZsq{}}\PYG{p}{:} \PYG{l+s+s1}{\PYGZsq{}}\PYG{l+s+s1}{Buy\PYGZhy{}And\PYGZhy{}Hold}\PYG{l+s+s1}{\PYGZsq{}}\PYG{p}{,} \PYG{l+s+s1}{\PYGZsq{}}\PYG{l+s+s1}{Strategy}\PYG{l+s+s1}{\PYGZsq{}}\PYG{p}{:} \PYG{l+s+s1}{\PYGZsq{}}\PYG{l+s+s1}{SMA(20)}\PYG{l+s+s1}{\PYGZsq{}}\PYG{p}{\PYGZcb{}}\PYG{p}{)}
    \PYG{o}{.}\PYG{n}{plot}\PYG{p}{(}\PYG{n}{kind}\PYG{o}{=}\PYG{l+s+s1}{\PYGZsq{}}\PYG{l+s+s1}{bar}\PYG{l+s+s1}{\PYGZsq{}}\PYG{p}{)}
\PYG{p}{)}
\PYG{n}{plt}\PYG{o}{.}\PYG{n}{xticks}\PYG{p}{(}\PYG{n}{rotation}\PYG{o}{=}\PYG{l+m+mi}{0}\PYG{p}{)}
\PYG{n}{plt}\PYG{o}{.}\PYG{n}{ylabel}\PYG{p}{(}\PYG{l+s+s1}{\PYGZsq{}}\PYG{l+s+s1}{Value (\PYGZdl{})}\PYG{l+s+s1}{\PYGZsq{}}\PYG{p}{)}
\PYG{n}{plt}\PYG{o}{.}\PYG{n}{title}\PYG{p}{(}\PYG{l+s+sa}{f}\PYG{l+s+s1}{\PYGZsq{}}\PYG{l+s+s1}{Value of \PYGZdl{}1 Investments Add End of 10\PYGZhy{}Year Holding Periods }\PYG{l+s+s1}{\PYGZsq{}}\PYG{p}{)}
\PYG{n}{plt}\PYG{o}{.}\PYG{n}{show}\PYG{p}{(}\PYG{p}{)}
\end{sphinxVerbatim}

\end{sphinxuseclass}\end{sphinxVerbatimInput}
\begin{sphinxVerbatimOutput}

\begin{sphinxuseclass}{cell_output}
\noindent\sphinxincludegraphics{{262e708684b1109abef3e7de77a33f6ae09019d5b276baf05ba5dcbda2c924e9}.png}

\end{sphinxuseclass}\end{sphinxVerbatimOutput}

\end{sphinxuseclass}
\sphinxAtStartPar
In fact, buy\sphinxhyphen{}and\sphinxhyphen{}hold outperforms SMA(20) is we start in 1950.

\begin{sphinxuseclass}{cell}\begin{sphinxVerbatimInput}

\begin{sphinxuseclass}{cell_input}
\begin{sphinxVerbatim}[commandchars=\\\{\}]
\PYG{n}{\PYGZus{}} \PYG{o}{=} \PYG{n}{ff\PYGZus{}sma}\PYG{o}{.}\PYG{n}{loc}\PYG{p}{[}\PYG{l+s+s1}{\PYGZsq{}}\PYG{l+s+s1}{1950}\PYG{l+s+s1}{\PYGZsq{}}\PYG{p}{:}\PYG{p}{,} \PYG{p}{[}\PYG{l+s+s1}{\PYGZsq{}}\PYG{l+s+s1}{Return}\PYG{l+s+s1}{\PYGZsq{}}\PYG{p}{,} \PYG{l+s+s1}{\PYGZsq{}}\PYG{l+s+s1}{Strategy}\PYG{l+s+s1}{\PYGZsq{}}\PYG{p}{]}\PYG{p}{]}\PYG{o}{.}\PYG{n}{dropna}\PYG{p}{(}\PYG{p}{)}

\PYG{p}{(}
    \PYG{n}{\PYGZus{}}
    \PYG{o}{.}\PYG{n}{add}\PYG{p}{(}\PYG{l+m+mi}{1}\PYG{p}{)}
    \PYG{o}{.}\PYG{n}{cumprod}\PYG{p}{(}\PYG{p}{)}
    \PYG{o}{.}\PYG{n}{rename\PYGZus{}axis}\PYG{p}{(}\PYG{n}{columns}\PYG{o}{=}\PYG{l+s+s1}{\PYGZsq{}}\PYG{l+s+s1}{Strategy}\PYG{l+s+s1}{\PYGZsq{}}\PYG{p}{)}
    \PYG{o}{.}\PYG{n}{rename}\PYG{p}{(}\PYG{n}{columns}\PYG{o}{=}\PYG{p}{\PYGZob{}}\PYG{l+s+s1}{\PYGZsq{}}\PYG{l+s+s1}{Return}\PYG{l+s+s1}{\PYGZsq{}}\PYG{p}{:} \PYG{l+s+s1}{\PYGZsq{}}\PYG{l+s+s1}{Buy\PYGZhy{}And\PYGZhy{}Hold}\PYG{l+s+s1}{\PYGZsq{}}\PYG{p}{,} \PYG{l+s+s1}{\PYGZsq{}}\PYG{l+s+s1}{Strategy}\PYG{l+s+s1}{\PYGZsq{}}\PYG{p}{:} \PYG{l+s+s1}{\PYGZsq{}}\PYG{l+s+s1}{SMA(20)}\PYG{l+s+s1}{\PYGZsq{}}\PYG{p}{\PYGZcb{}}\PYG{p}{)}
    \PYG{o}{.}\PYG{n}{plot}\PYG{p}{(}\PYG{p}{)}
\PYG{p}{)}
\PYG{n}{plt}\PYG{o}{.}\PYG{n}{ylabel}\PYG{p}{(}\PYG{l+s+s1}{\PYGZsq{}}\PYG{l+s+s1}{Value (\PYGZdl{})}\PYG{l+s+s1}{\PYGZsq{}}\PYG{p}{)}
\PYG{n}{plt}\PYG{o}{.}\PYG{n}{title}\PYG{p}{(}\PYG{l+s+sa}{f}\PYG{l+s+s1}{\PYGZsq{}}\PYG{l+s+s1}{Value of \PYGZdl{}1 Invested in Market at Close on }\PYG{l+s+si}{\PYGZob{}}\PYG{n}{\PYGZus{}}\PYG{o}{.}\PYG{n}{index}\PYG{p}{[}\PYG{l+m+mi}{0}\PYG{p}{]} \PYG{o}{\PYGZhy{}} \PYG{n}{pd}\PYG{o}{.}\PYG{n}{offsets}\PYG{o}{.}\PYG{n}{Day}\PYG{p}{(}\PYG{l+m+mi}{1}\PYG{p}{)}\PYG{l+s+si}{:}\PYG{l+s+s1}{\PYGZpc{}B \PYGZpc{}d, \PYGZpc{}Y}\PYG{l+s+si}{\PYGZcb{}}\PYG{l+s+s1}{\PYGZsq{}}\PYG{p}{)}
\PYG{n}{plt}\PYG{o}{.}\PYG{n}{show}\PYG{p}{(}\PYG{p}{)}
\end{sphinxVerbatim}

\end{sphinxuseclass}\end{sphinxVerbatimInput}
\begin{sphinxVerbatimOutput}

\begin{sphinxuseclass}{cell_output}
\noindent\sphinxincludegraphics{{de32325d6c5dda85fff71b12216bc00bdfbe78f65cb56ebc4cf40d6abd840d13}.png}

\end{sphinxuseclass}\end{sphinxVerbatimOutput}

\end{sphinxuseclass}

\subsubsection{Implement a long\sphinxhyphen{}only BB(20, 2) strategy with Bitcoin}
\label{\detokenize{herron_02_practice_02:implement-a-long-only-bb-20-2-strategy-with-bitcoin}}
\sphinxAtStartPar
More on Bollinger Bands \sphinxhref{https://www.bollingerbands.com/bollinger-bands}{here} and \sphinxhref{https://www.bollingerbands.com/bollinger-band-rules}{here}.
In short, Bollinger Bands are bands around a trend, typically defined in terms of simple moving averages and volatilities.
Here, long\sphinxhyphen{}only BB(20, 2) implies we have upper and lower bands at 2 standard deviations above and below SMA(20):
\begin{enumerate}
\sphinxsetlistlabels{\arabic}{enumi}{enumii}{}{.}%
\item {} 
\sphinxAtStartPar
Buy when the closing price crosses LB(20) from below, where LB(20) is SMA(20) minus 2 sigma

\item {} 
\sphinxAtStartPar
Sell when the closing price crosses UB(20) from above, where UB(20) is SMA(20) plus 2 sigma

\item {} 
\sphinxAtStartPar
No short\sphinxhyphen{}selling

\end{enumerate}

\sphinxAtStartPar
The long\sphinxhyphen{}only BB(20, 2) is more difficult to implement than the long\sphinxhyphen{}only SMA(20) because we need to track buys and sells.
For example, if the closing price is between LB(20) and BB(20), we need to know if our last trade was a buy or a sell.
Further, if the closing price is below LB(20), we can still be long because we sell when the closing price crosses UB(20) from above.

\sphinxAtStartPar
\sphinxstyleemphasis{\sphinxstylestrong{Again, write a function \sphinxcode{\sphinxupquote{bb()}} that accepts a data frame \sphinxcode{\sphinxupquote{df}}, window size \sphinxcode{\sphinxupquote{n}}, and number of standard deviations \sphinxcode{\sphinxupquote{m}}.
Use your \sphinxcode{\sphinxupquote{bb()}} function to implement the BB(20, 2) strategy and assign it to data frame \sphinxcode{\sphinxupquote{btc\_bb}}.}}

\begin{sphinxuseclass}{cell}\begin{sphinxVerbatimInput}

\begin{sphinxuseclass}{cell_input}
\begin{sphinxVerbatim}[commandchars=\\\{\}]
\PYG{k}{def} \PYG{n+nf}{bb}\PYG{p}{(}\PYG{n}{df}\PYG{p}{,} \PYG{n}{n}\PYG{o}{=}\PYG{l+m+mi}{20}\PYG{p}{,} \PYG{n}{m}\PYG{o}{=}\PYG{l+m+mi}{2}\PYG{p}{)}\PYG{p}{:}
    \PYG{k}{return} \PYG{p}{(}
        \PYG{n}{df}
        \PYG{o}{.}\PYG{n}{assign}\PYG{p}{(}
            \PYG{n}{Return} \PYG{o}{=} \PYG{k}{lambda} \PYG{n}{x}\PYG{p}{:} \PYG{n}{x}\PYG{p}{[}\PYG{l+s+s1}{\PYGZsq{}}\PYG{l+s+s1}{Adj Close}\PYG{l+s+s1}{\PYGZsq{}}\PYG{p}{]}\PYG{o}{.}\PYG{n}{pct\PYGZus{}change}\PYG{p}{(}\PYG{p}{)}\PYG{p}{,}
            \PYG{n}{SMA} \PYG{o}{=} \PYG{k}{lambda} \PYG{n}{x}\PYG{p}{:} \PYG{n}{x}\PYG{p}{[}\PYG{l+s+s1}{\PYGZsq{}}\PYG{l+s+s1}{Adj Close}\PYG{l+s+s1}{\PYGZsq{}}\PYG{p}{]}\PYG{o}{.}\PYG{n}{rolling}\PYG{p}{(}\PYG{n}{n}\PYG{p}{)}\PYG{o}{.}\PYG{n}{mean}\PYG{p}{(}\PYG{p}{)}\PYG{p}{,}
            \PYG{n}{SMV} \PYG{o}{=} \PYG{k}{lambda} \PYG{n}{x}\PYG{p}{:} \PYG{n}{x}\PYG{p}{[}\PYG{l+s+s1}{\PYGZsq{}}\PYG{l+s+s1}{Adj Close}\PYG{l+s+s1}{\PYGZsq{}}\PYG{p}{]}\PYG{o}{.}\PYG{n}{rolling}\PYG{p}{(}\PYG{n}{n}\PYG{p}{)}\PYG{o}{.}\PYG{n}{std}\PYG{p}{(}\PYG{p}{)}\PYG{p}{,}
            \PYG{n}{UB} \PYG{o}{=} \PYG{k}{lambda} \PYG{n}{x}\PYG{p}{:} \PYG{n}{x}\PYG{p}{[}\PYG{l+s+s1}{\PYGZsq{}}\PYG{l+s+s1}{SMA}\PYG{l+s+s1}{\PYGZsq{}}\PYG{p}{]} \PYG{o}{+} \PYG{n}{m}\PYG{o}{*}\PYG{n}{x}\PYG{p}{[}\PYG{l+s+s1}{\PYGZsq{}}\PYG{l+s+s1}{SMV}\PYG{l+s+s1}{\PYGZsq{}}\PYG{p}{]}\PYG{p}{,}
            \PYG{n}{LB} \PYG{o}{=} \PYG{k}{lambda} \PYG{n}{x}\PYG{p}{:} \PYG{n}{x}\PYG{p}{[}\PYG{l+s+s1}{\PYGZsq{}}\PYG{l+s+s1}{SMA}\PYG{l+s+s1}{\PYGZsq{}}\PYG{p}{]} \PYG{o}{\PYGZhy{}} \PYG{n}{m}\PYG{o}{*}\PYG{n}{x}\PYG{p}{[}\PYG{l+s+s1}{\PYGZsq{}}\PYG{l+s+s1}{SMV}\PYG{l+s+s1}{\PYGZsq{}}\PYG{p}{]}\PYG{p}{,}
            \PYG{n}{Position\PYGZus{}with\PYGZus{}nan} \PYG{o}{=} \PYG{k}{lambda} \PYG{n}{x}\PYG{p}{:} \PYG{n}{np}\PYG{o}{.}\PYG{n}{select}\PYG{p}{(}
                \PYG{n}{condlist}\PYG{o}{=}\PYG{p}{[}
                    \PYG{p}{(}\PYG{n}{x}\PYG{p}{[}\PYG{l+s+s1}{\PYGZsq{}}\PYG{l+s+s1}{Adj Close}\PYG{l+s+s1}{\PYGZsq{}}\PYG{p}{]}\PYG{o}{.}\PYG{n}{shift}\PYG{p}{(}\PYG{l+m+mi}{1}\PYG{p}{)} \PYG{o}{\PYGZgt{}} \PYG{n}{x}\PYG{p}{[}\PYG{l+s+s1}{\PYGZsq{}}\PYG{l+s+s1}{LB}\PYG{l+s+s1}{\PYGZsq{}}\PYG{p}{]}\PYG{o}{.}\PYG{n}{shift}\PYG{p}{(}\PYG{l+m+mi}{1}\PYG{p}{)}\PYG{p}{)} \PYG{o}{\PYGZam{}} \PYG{p}{(}\PYG{n}{x}\PYG{p}{[}\PYG{l+s+s1}{\PYGZsq{}}\PYG{l+s+s1}{Adj Close}\PYG{l+s+s1}{\PYGZsq{}}\PYG{p}{]}\PYG{o}{.}\PYG{n}{shift}\PYG{p}{(}\PYG{l+m+mi}{2}\PYG{p}{)} \PYG{o}{\PYGZlt{}}\PYG{o}{=} \PYG{n}{x}\PYG{p}{[}\PYG{l+s+s1}{\PYGZsq{}}\PYG{l+s+s1}{LB}\PYG{l+s+s1}{\PYGZsq{}}\PYG{p}{]}\PYG{o}{.}\PYG{n}{shift}\PYG{p}{(}\PYG{l+m+mi}{2}\PYG{p}{)}\PYG{p}{)}\PYG{p}{,} 
                    \PYG{p}{(}\PYG{n}{x}\PYG{p}{[}\PYG{l+s+s1}{\PYGZsq{}}\PYG{l+s+s1}{Adj Close}\PYG{l+s+s1}{\PYGZsq{}}\PYG{p}{]}\PYG{o}{.}\PYG{n}{shift}\PYG{p}{(}\PYG{l+m+mi}{1}\PYG{p}{)} \PYG{o}{\PYGZlt{}} \PYG{n}{x}\PYG{p}{[}\PYG{l+s+s1}{\PYGZsq{}}\PYG{l+s+s1}{UB}\PYG{l+s+s1}{\PYGZsq{}}\PYG{p}{]}\PYG{o}{.}\PYG{n}{shift}\PYG{p}{(}\PYG{l+m+mi}{1}\PYG{p}{)}\PYG{p}{)} \PYG{o}{\PYGZam{}} \PYG{p}{(}\PYG{n}{x}\PYG{p}{[}\PYG{l+s+s1}{\PYGZsq{}}\PYG{l+s+s1}{Adj Close}\PYG{l+s+s1}{\PYGZsq{}}\PYG{p}{]}\PYG{o}{.}\PYG{n}{shift}\PYG{p}{(}\PYG{l+m+mi}{2}\PYG{p}{)} \PYG{o}{\PYGZgt{}}\PYG{o}{=} \PYG{n}{x}\PYG{p}{[}\PYG{l+s+s1}{\PYGZsq{}}\PYG{l+s+s1}{UB}\PYG{l+s+s1}{\PYGZsq{}}\PYG{p}{]}\PYG{o}{.}\PYG{n}{shift}\PYG{p}{(}\PYG{l+m+mi}{2}\PYG{p}{)}\PYG{p}{)}
                \PYG{p}{]}\PYG{p}{,}
                \PYG{n}{choicelist}\PYG{o}{=}\PYG{p}{[}
                    \PYG{l+m+mi}{1}\PYG{p}{,} 
                    \PYG{l+m+mi}{0}
                \PYG{p}{]}\PYG{p}{,}
                \PYG{n}{default}\PYG{o}{=}\PYG{n}{np}\PYG{o}{.}\PYG{n}{nan}
            \PYG{p}{)}\PYG{p}{,}
            \PYG{n}{Position} \PYG{o}{=} \PYG{k}{lambda} \PYG{n}{x}\PYG{p}{:} \PYG{n}{x}\PYG{p}{[}\PYG{l+s+s1}{\PYGZsq{}}\PYG{l+s+s1}{Position\PYGZus{}with\PYGZus{}nan}\PYG{l+s+s1}{\PYGZsq{}}\PYG{p}{]}\PYG{o}{.}\PYG{n}{fillna}\PYG{p}{(}\PYG{n}{method}\PYG{o}{=}\PYG{l+s+s1}{\PYGZsq{}}\PYG{l+s+s1}{ffill}\PYG{l+s+s1}{\PYGZsq{}}\PYG{p}{)}\PYG{p}{,}
            \PYG{n}{Strategy} \PYG{o}{=} \PYG{k}{lambda} \PYG{n}{x}\PYG{p}{:} \PYG{n}{x}\PYG{p}{[}\PYG{l+s+s1}{\PYGZsq{}}\PYG{l+s+s1}{Position}\PYG{l+s+s1}{\PYGZsq{}}\PYG{p}{]} \PYG{o}{*} \PYG{n}{x}\PYG{p}{[}\PYG{l+s+s1}{\PYGZsq{}}\PYG{l+s+s1}{Return}\PYG{l+s+s1}{\PYGZsq{}}\PYG{p}{]}
        \PYG{p}{)}
    \PYG{p}{)}
\end{sphinxVerbatim}

\end{sphinxuseclass}\end{sphinxVerbatimInput}

\end{sphinxuseclass}
\begin{sphinxuseclass}{cell}\begin{sphinxVerbatimInput}

\begin{sphinxuseclass}{cell_input}
\begin{sphinxVerbatim}[commandchars=\\\{\}]
\PYG{n}{btc\PYGZus{}bb} \PYG{o}{=} \PYG{n}{btc}\PYG{o}{.}\PYG{n}{pipe}\PYG{p}{(}\PYG{n}{bb}\PYG{p}{,} \PYG{n}{n}\PYG{o}{=}\PYG{l+m+mi}{20}\PYG{p}{,} \PYG{n}{m}\PYG{o}{=}\PYG{l+m+mi}{2}\PYG{p}{)}

\PYG{n}{btc\PYGZus{}bb}\PYG{o}{.}\PYG{n}{tail}\PYG{p}{(}\PYG{p}{)}
\end{sphinxVerbatim}

\end{sphinxuseclass}\end{sphinxVerbatimInput}
\begin{sphinxVerbatimOutput}

\begin{sphinxuseclass}{cell_output}
\begin{sphinxVerbatim}[commandchars=\\\{\}]
Variable         Open       High        Low      Close  Adj Close  \PYGZbs{}
Date                                                                
2023\PYGZhy{}03\PYGZhy{}11 20187.8770 20792.5254 20068.6602 20632.4102 20632.4102   
2023\PYGZhy{}03\PYGZhy{}12 20628.0293 22185.0312 20448.8066 22163.9492 22163.9492   
2023\PYGZhy{}03\PYGZhy{}13 22156.4062 24550.8379 21918.1992 24197.5332 24197.5332   
2023\PYGZhy{}03\PYGZhy{}14 24201.7656 26514.7168 24081.1836 24746.0742 24746.0742   
2023\PYGZhy{}03\PYGZhy{}15 24734.6523 25169.7305 24048.3516 24617.3047 24617.3047   

Variable         Volume  Return        SMA       SMV         UB         LB  \PYGZbs{}
Date                                                                         
2023\PYGZhy{}03\PYGZhy{}11  30180288176  0.0221 22791.5311 1305.1707 25401.8725 20181.1896   
2023\PYGZhy{}03\PYGZhy{}12  29279035521  0.0742 22658.2711 1219.4207 25097.1125 20219.4297   
2023\PYGZhy{}03\PYGZhy{}13  49466362688  0.0918 22646.3301 1202.1395 25050.6091 20242.0510   
2023\PYGZhy{}03\PYGZhy{}14  54622230164  0.0227 22674.1916 1245.4486 25165.0888 20183.2944   
2023\PYGZhy{}03\PYGZhy{}15  44081737728 \PYGZhy{}0.0052 22707.6822 1289.7095 25287.1013 20128.2632   

Variable    Position\PYGZus{}with\PYGZus{}nan  Position  Strategy  
Date                                               
2023\PYGZhy{}03\PYGZhy{}11                NaN    1.0000    0.0221  
2023\PYGZhy{}03\PYGZhy{}12             1.0000    1.0000    0.0742  
2023\PYGZhy{}03\PYGZhy{}13                NaN    1.0000    0.0918  
2023\PYGZhy{}03\PYGZhy{}14                NaN    1.0000    0.0227  
2023\PYGZhy{}03\PYGZhy{}15                NaN    1.0000   \PYGZhy{}0.0052  
\end{sphinxVerbatim}

\end{sphinxuseclass}\end{sphinxVerbatimOutput}

\end{sphinxuseclass}
\begin{sphinxuseclass}{cell}\begin{sphinxVerbatimInput}

\begin{sphinxuseclass}{cell_input}
\begin{sphinxVerbatim}[commandchars=\\\{\}]
\PYG{n}{\PYGZus{}} \PYG{o}{=} \PYG{n}{btc\PYGZus{}bb}\PYG{p}{[}\PYG{p}{[}\PYG{l+s+s1}{\PYGZsq{}}\PYG{l+s+s1}{Return}\PYG{l+s+s1}{\PYGZsq{}}\PYG{p}{,} \PYG{l+s+s1}{\PYGZsq{}}\PYG{l+s+s1}{Strategy}\PYG{l+s+s1}{\PYGZsq{}}\PYG{p}{]}\PYG{p}{]}\PYG{o}{.}\PYG{n}{dropna}\PYG{p}{(}\PYG{p}{)}

\PYG{p}{(}
    \PYG{n}{\PYGZus{}}
    \PYG{o}{.}\PYG{n}{add}\PYG{p}{(}\PYG{l+m+mi}{1}\PYG{p}{)}
    \PYG{o}{.}\PYG{n}{cumprod}\PYG{p}{(}\PYG{p}{)}
    \PYG{o}{.}\PYG{n}{rename\PYGZus{}axis}\PYG{p}{(}\PYG{n}{columns}\PYG{o}{=}\PYG{l+s+s1}{\PYGZsq{}}\PYG{l+s+s1}{Strategy}\PYG{l+s+s1}{\PYGZsq{}}\PYG{p}{)}
    \PYG{o}{.}\PYG{n}{rename}\PYG{p}{(}\PYG{n}{columns}\PYG{o}{=}\PYG{p}{\PYGZob{}}\PYG{l+s+s1}{\PYGZsq{}}\PYG{l+s+s1}{Return}\PYG{l+s+s1}{\PYGZsq{}}\PYG{p}{:} \PYG{l+s+s1}{\PYGZsq{}}\PYG{l+s+s1}{Buy\PYGZhy{}And\PYGZhy{}Hold}\PYG{l+s+s1}{\PYGZsq{}}\PYG{p}{,} \PYG{l+s+s1}{\PYGZsq{}}\PYG{l+s+s1}{Strategy}\PYG{l+s+s1}{\PYGZsq{}}\PYG{p}{:} \PYG{l+s+s1}{\PYGZsq{}}\PYG{l+s+s1}{BB(20, 2)}\PYG{l+s+s1}{\PYGZsq{}}\PYG{p}{\PYGZcb{}}\PYG{p}{)}
    \PYG{o}{.}\PYG{n}{plot}\PYG{p}{(}\PYG{p}{)}
\PYG{p}{)}
\PYG{n}{plt}\PYG{o}{.}\PYG{n}{ylabel}\PYG{p}{(}\PYG{l+s+s1}{\PYGZsq{}}\PYG{l+s+s1}{Value (\PYGZdl{})}\PYG{l+s+s1}{\PYGZsq{}}\PYG{p}{)}
\PYG{n}{plt}\PYG{o}{.}\PYG{n}{title}\PYG{p}{(}\PYG{l+s+sa}{f}\PYG{l+s+s1}{\PYGZsq{}}\PYG{l+s+s1}{Value of \PYGZdl{}1 Invested at Close on }\PYG{l+s+si}{\PYGZob{}}\PYG{n}{\PYGZus{}}\PYG{o}{.}\PYG{n}{index}\PYG{p}{[}\PYG{l+m+mi}{0}\PYG{p}{]} \PYG{o}{\PYGZhy{}} \PYG{n}{pd}\PYG{o}{.}\PYG{n}{offsets}\PYG{o}{.}\PYG{n}{Day}\PYG{p}{(}\PYG{l+m+mi}{1}\PYG{p}{)}\PYG{l+s+si}{:}\PYG{l+s+s1}{\PYGZpc{}B \PYGZpc{}d, \PYGZpc{}Y}\PYG{l+s+si}{\PYGZcb{}}\PYG{l+s+s1}{\PYGZsq{}}\PYG{p}{)}
\PYG{n}{plt}\PYG{o}{.}\PYG{n}{show}\PYG{p}{(}\PYG{p}{)}
\end{sphinxVerbatim}

\end{sphinxuseclass}\end{sphinxVerbatimInput}
\begin{sphinxVerbatimOutput}

\begin{sphinxuseclass}{cell_output}
\noindent\sphinxincludegraphics{{1712840168ce637c11b63aaf384533ed9deb985f91ac37aa6e4cf240669a1e8e}.png}

\end{sphinxuseclass}\end{sphinxVerbatimOutput}

\end{sphinxuseclass}
\sphinxAtStartPar
For an asset that we know has large positive returns over the sample, “time in the market” beats “timing the market”.


\subsubsection{Implement a long\sphinxhyphen{}short RSI(14) strategy with Bitcoin}
\label{\detokenize{herron_02_practice_02:implement-a-long-short-rsi-14-strategy-with-bitcoin}}
\sphinxAtStartPar
From \sphinxhref{https://www.fidelity.com/learning-center/trading-investing/technical-analysis/technical-indicator-guide/rsi}{Fidelity}:
\begin{quote}

\sphinxAtStartPar
The Relative Strength Index (RSI), developed by J. Welles Wilder, is a momentum oscillator that measures the speed and change of price movements. The RSI oscillates between zero and 100. Traditionally the RSI is considered overbought when above 70 and oversold when below 30. Signals can be generated by looking for divergences and failure swings. RSI can also be used to identify the general trend.
\end{quote}

\sphinxAtStartPar
Here is the RSI formula: \$RSI(n) = 100 \sphinxhyphen{} \textbackslash{}frac\{100\}\{1 + RS(n)\}\$, where \$RS(n) = \textbackslash{}frac\{SMA(U, n)\}\{SMA(D, n)\}\$.
For “up days”, \$U = \textbackslash{}Delta Adj\textbackslash{} Close\$ and \$D = 0\$, and, for “down days”, \$U = 0\$ and \$D = \sphinxhyphen{} \textbackslash{}Delta Adj\textbackslash{} Close\$.
Therefore, \$U\$ and \$D\$ are always non\sphinxhyphen{}negative.
We can learn more about RSI \sphinxhref{https://en.wikipedia.org/wiki/Relative\_strength\_index}{here}.

\sphinxAtStartPar
We will implement a long\sphinxhyphen{}short RSI(14) as follows:
\begin{enumerate}
\sphinxsetlistlabels{\arabic}{enumi}{enumii}{}{.}%
\item {} 
\sphinxAtStartPar
Enter a long position when  the RSI crosses 30 from below, and exit the position when the RSI crosses 50 from below

\item {} 
\sphinxAtStartPar
Enter a short position when the RSI crosses 70 from above, and exit the position when the RSI crosses 50 from above

\end{enumerate}

\sphinxAtStartPar
\sphinxstyleemphasis{\sphinxstylestrong{Again, write a function \sphinxcode{\sphinxupquote{rsi()}} that accepts a data frame \sphinxcode{\sphinxupquote{df}}, window size \sphinxcode{\sphinxupquote{n}}, and boundary percentiles \sphinxcode{\sphinxupquote{lb}}, \sphinxcode{\sphinxupquote{mb}}, and \sphinxcode{\sphinxupquote{ub}}.
Use your \sphinxcode{\sphinxupquote{rsi()}} function to implement the RSI strategy and assign it to data frame \sphinxcode{\sphinxupquote{btc\_rsi}}.}}

\begin{sphinxuseclass}{cell}\begin{sphinxVerbatimInput}

\begin{sphinxuseclass}{cell_input}
\begin{sphinxVerbatim}[commandchars=\\\{\}]
\PYG{k}{def} \PYG{n+nf}{rsi}\PYG{p}{(}\PYG{n}{df}\PYG{p}{,} \PYG{n}{n}\PYG{o}{=}\PYG{l+m+mi}{14}\PYG{p}{,} \PYG{n}{lb}\PYG{o}{=}\PYG{l+m+mi}{30}\PYG{p}{,} \PYG{n}{mb}\PYG{o}{=}\PYG{l+m+mi}{50}\PYG{p}{,} \PYG{n}{ub}\PYG{o}{=}\PYG{l+m+mi}{70}\PYG{p}{)}\PYG{p}{:}
    \PYG{k}{return} \PYG{n}{df}\PYG{o}{.}\PYG{n}{assign}\PYG{p}{(}
        \PYG{n}{Return} \PYG{o}{=} \PYG{k}{lambda} \PYG{n}{x}\PYG{p}{:} \PYG{n}{x}\PYG{p}{[}\PYG{l+s+s1}{\PYGZsq{}}\PYG{l+s+s1}{Adj Close}\PYG{l+s+s1}{\PYGZsq{}}\PYG{p}{]}\PYG{o}{.}\PYG{n}{pct\PYGZus{}change}\PYG{p}{(}\PYG{p}{)}\PYG{p}{,}
        \PYG{n}{Diff} \PYG{o}{=} \PYG{k}{lambda} \PYG{n}{x}\PYG{p}{:} \PYG{n}{x}\PYG{p}{[}\PYG{l+s+s1}{\PYGZsq{}}\PYG{l+s+s1}{Adj Close}\PYG{l+s+s1}{\PYGZsq{}}\PYG{p}{]}\PYG{o}{.}\PYG{n}{diff}\PYG{p}{(}\PYG{p}{)}\PYG{p}{,}
        \PYG{n}{U} \PYG{o}{=} \PYG{k}{lambda} \PYG{n}{x}\PYG{p}{:} \PYG{n}{np}\PYG{o}{.}\PYG{n}{select}\PYG{p}{(}
            \PYG{n}{condlist}\PYG{o}{=}\PYG{p}{[}\PYG{n}{x}\PYG{p}{[}\PYG{l+s+s1}{\PYGZsq{}}\PYG{l+s+s1}{Diff}\PYG{l+s+s1}{\PYGZsq{}}\PYG{p}{]} \PYG{o}{\PYGZgt{}}\PYG{o}{=} \PYG{l+m+mi}{0}\PYG{p}{,} \PYG{n}{x}\PYG{p}{[}\PYG{l+s+s1}{\PYGZsq{}}\PYG{l+s+s1}{Diff}\PYG{l+s+s1}{\PYGZsq{}}\PYG{p}{]} \PYG{o}{\PYGZlt{}} \PYG{l+m+mi}{0}\PYG{p}{]}\PYG{p}{,}
            \PYG{n}{choicelist}\PYG{o}{=}\PYG{p}{[}\PYG{n}{x}\PYG{p}{[}\PYG{l+s+s1}{\PYGZsq{}}\PYG{l+s+s1}{Diff}\PYG{l+s+s1}{\PYGZsq{}}\PYG{p}{]}\PYG{p}{,} \PYG{l+m+mi}{0}\PYG{p}{]}\PYG{p}{,}
            \PYG{n}{default}\PYG{o}{=}\PYG{n}{np}\PYG{o}{.}\PYG{n}{nan}
        \PYG{p}{)}\PYG{p}{,}
        \PYG{n}{D} \PYG{o}{=} \PYG{k}{lambda} \PYG{n}{x}\PYG{p}{:} \PYG{n}{np}\PYG{o}{.}\PYG{n}{select}\PYG{p}{(}
            \PYG{n}{condlist}\PYG{o}{=}\PYG{p}{[}\PYG{n}{x}\PYG{p}{[}\PYG{l+s+s1}{\PYGZsq{}}\PYG{l+s+s1}{Diff}\PYG{l+s+s1}{\PYGZsq{}}\PYG{p}{]} \PYG{o}{\PYGZlt{}}\PYG{o}{=} \PYG{l+m+mi}{0}\PYG{p}{,} \PYG{n}{x}\PYG{p}{[}\PYG{l+s+s1}{\PYGZsq{}}\PYG{l+s+s1}{Diff}\PYG{l+s+s1}{\PYGZsq{}}\PYG{p}{]} \PYG{o}{\PYGZgt{}} \PYG{l+m+mi}{0}\PYG{p}{]}\PYG{p}{,}
            \PYG{n}{choicelist}\PYG{o}{=}\PYG{p}{[}\PYG{o}{\PYGZhy{}}\PYG{l+m+mi}{1} \PYG{o}{*} \PYG{n}{x}\PYG{p}{[}\PYG{l+s+s1}{\PYGZsq{}}\PYG{l+s+s1}{Diff}\PYG{l+s+s1}{\PYGZsq{}}\PYG{p}{]}\PYG{p}{,} \PYG{l+m+mi}{0}\PYG{p}{]}\PYG{p}{,}
            \PYG{n}{default}\PYG{o}{=}\PYG{n}{np}\PYG{o}{.}\PYG{n}{nan}
        \PYG{p}{)}\PYG{p}{,}
        \PYG{n}{SMAU} \PYG{o}{=} \PYG{k}{lambda} \PYG{n}{x}\PYG{p}{:} \PYG{n}{x}\PYG{p}{[}\PYG{l+s+s1}{\PYGZsq{}}\PYG{l+s+s1}{U}\PYG{l+s+s1}{\PYGZsq{}}\PYG{p}{]}\PYG{o}{.}\PYG{n}{rolling}\PYG{p}{(}\PYG{n}{n}\PYG{p}{)}\PYG{o}{.}\PYG{n}{mean}\PYG{p}{(}\PYG{p}{)}\PYG{p}{,}
        \PYG{n}{SMAD} \PYG{o}{=} \PYG{k}{lambda} \PYG{n}{x}\PYG{p}{:} \PYG{n}{x}\PYG{p}{[}\PYG{l+s+s1}{\PYGZsq{}}\PYG{l+s+s1}{D}\PYG{l+s+s1}{\PYGZsq{}}\PYG{p}{]}\PYG{o}{.}\PYG{n}{rolling}\PYG{p}{(}\PYG{n}{n}\PYG{p}{)}\PYG{o}{.}\PYG{n}{mean}\PYG{p}{(}\PYG{p}{)}\PYG{p}{,}
        \PYG{n}{RS} \PYG{o}{=} \PYG{k}{lambda} \PYG{n}{x}\PYG{p}{:} \PYG{n}{x}\PYG{p}{[}\PYG{l+s+s1}{\PYGZsq{}}\PYG{l+s+s1}{SMAU}\PYG{l+s+s1}{\PYGZsq{}}\PYG{p}{]} \PYG{o}{/} \PYG{n}{x}\PYG{p}{[}\PYG{l+s+s1}{\PYGZsq{}}\PYG{l+s+s1}{SMAD}\PYG{l+s+s1}{\PYGZsq{}}\PYG{p}{]}\PYG{p}{,}
        \PYG{n}{RSI} \PYG{o}{=} \PYG{k}{lambda} \PYG{n}{x}\PYG{p}{:} \PYG{l+m+mi}{100} \PYG{o}{\PYGZhy{}} \PYG{l+m+mi}{100} \PYG{o}{/} \PYG{p}{(}\PYG{l+m+mi}{1} \PYG{o}{+} \PYG{n}{x}\PYG{p}{[}\PYG{l+s+s1}{\PYGZsq{}}\PYG{l+s+s1}{RS}\PYG{l+s+s1}{\PYGZsq{}}\PYG{p}{]}\PYG{p}{)}\PYG{p}{,}
        \PYG{n}{Position\PYGZus{}with\PYGZus{}nan} \PYG{o}{=} \PYG{k}{lambda} \PYG{n}{x}\PYG{p}{:} \PYG{n}{np}\PYG{o}{.}\PYG{n}{select}\PYG{p}{(}
            \PYG{n}{condlist}\PYG{o}{=}\PYG{p}{[}
                \PYG{p}{(}\PYG{n}{x}\PYG{p}{[}\PYG{l+s+s1}{\PYGZsq{}}\PYG{l+s+s1}{RSI}\PYG{l+s+s1}{\PYGZsq{}}\PYG{p}{]}\PYG{o}{.}\PYG{n}{shift}\PYG{p}{(}\PYG{l+m+mi}{1}\PYG{p}{)} \PYG{o}{\PYGZgt{}}\PYG{o}{=} \PYG{n}{lb}\PYG{p}{)} \PYG{o}{\PYGZam{}} \PYG{p}{(}\PYG{n}{x}\PYG{p}{[}\PYG{l+s+s1}{\PYGZsq{}}\PYG{l+s+s1}{RSI}\PYG{l+s+s1}{\PYGZsq{}}\PYG{p}{]}\PYG{o}{.}\PYG{n}{shift}\PYG{p}{(}\PYG{l+m+mi}{2}\PYG{p}{)} \PYG{o}{\PYGZlt{}} \PYG{n}{lb}\PYG{p}{)}\PYG{p}{,} 
                \PYG{p}{(}\PYG{n}{x}\PYG{p}{[}\PYG{l+s+s1}{\PYGZsq{}}\PYG{l+s+s1}{RSI}\PYG{l+s+s1}{\PYGZsq{}}\PYG{p}{]}\PYG{o}{.}\PYG{n}{shift}\PYG{p}{(}\PYG{l+m+mi}{1}\PYG{p}{)} \PYG{o}{\PYGZgt{}}\PYG{o}{=} \PYG{n}{mb}\PYG{p}{)} \PYG{o}{\PYGZam{}} \PYG{p}{(}\PYG{n}{x}\PYG{p}{[}\PYG{l+s+s1}{\PYGZsq{}}\PYG{l+s+s1}{RSI}\PYG{l+s+s1}{\PYGZsq{}}\PYG{p}{]}\PYG{o}{.}\PYG{n}{shift}\PYG{p}{(}\PYG{l+m+mi}{2}\PYG{p}{)} \PYG{o}{\PYGZlt{}} \PYG{n}{mb}\PYG{p}{)}\PYG{p}{,}
                \PYG{p}{(}\PYG{n}{x}\PYG{p}{[}\PYG{l+s+s1}{\PYGZsq{}}\PYG{l+s+s1}{RSI}\PYG{l+s+s1}{\PYGZsq{}}\PYG{p}{]}\PYG{o}{.}\PYG{n}{shift}\PYG{p}{(}\PYG{l+m+mi}{1}\PYG{p}{)} \PYG{o}{\PYGZlt{}}\PYG{o}{=} \PYG{n}{ub}\PYG{p}{)} \PYG{o}{\PYGZam{}} \PYG{p}{(}\PYG{n}{x}\PYG{p}{[}\PYG{l+s+s1}{\PYGZsq{}}\PYG{l+s+s1}{RSI}\PYG{l+s+s1}{\PYGZsq{}}\PYG{p}{]}\PYG{o}{.}\PYG{n}{shift}\PYG{p}{(}\PYG{l+m+mi}{2}\PYG{p}{)} \PYG{o}{\PYGZgt{}} \PYG{n}{ub}\PYG{p}{)}\PYG{p}{,} 
                \PYG{p}{(}\PYG{n}{x}\PYG{p}{[}\PYG{l+s+s1}{\PYGZsq{}}\PYG{l+s+s1}{RSI}\PYG{l+s+s1}{\PYGZsq{}}\PYG{p}{]}\PYG{o}{.}\PYG{n}{shift}\PYG{p}{(}\PYG{l+m+mi}{1}\PYG{p}{)} \PYG{o}{\PYGZlt{}}\PYG{o}{=} \PYG{n}{mb}\PYG{p}{)} \PYG{o}{\PYGZam{}} \PYG{p}{(}\PYG{n}{x}\PYG{p}{[}\PYG{l+s+s1}{\PYGZsq{}}\PYG{l+s+s1}{RSI}\PYG{l+s+s1}{\PYGZsq{}}\PYG{p}{]}\PYG{o}{.}\PYG{n}{shift}\PYG{p}{(}\PYG{l+m+mi}{2}\PYG{p}{)} \PYG{o}{\PYGZgt{}} \PYG{n}{mb}\PYG{p}{)}\PYG{p}{,}
            \PYG{p}{]}\PYG{p}{,}
            \PYG{n}{choicelist}\PYG{o}{=}\PYG{p}{[}
                \PYG{l+m+mi}{1}\PYG{p}{,} 
                \PYG{l+m+mi}{0}\PYG{p}{,}
                \PYG{o}{\PYGZhy{}}\PYG{l+m+mi}{1}\PYG{p}{,}
                \PYG{l+m+mi}{0}
            \PYG{p}{]}\PYG{p}{,}
            \PYG{n}{default}\PYG{o}{=}\PYG{n}{np}\PYG{o}{.}\PYG{n}{nan}
        \PYG{p}{)}\PYG{p}{,}
        \PYG{n}{Position} \PYG{o}{=} \PYG{k}{lambda} \PYG{n}{x}\PYG{p}{:} \PYG{n}{x}\PYG{p}{[}\PYG{l+s+s1}{\PYGZsq{}}\PYG{l+s+s1}{Position\PYGZus{}with\PYGZus{}nan}\PYG{l+s+s1}{\PYGZsq{}}\PYG{p}{]}\PYG{o}{.}\PYG{n}{fillna}\PYG{p}{(}\PYG{n}{method}\PYG{o}{=}\PYG{l+s+s1}{\PYGZsq{}}\PYG{l+s+s1}{ffill}\PYG{l+s+s1}{\PYGZsq{}}\PYG{p}{)}\PYG{p}{,}
        \PYG{n}{Strategy} \PYG{o}{=} \PYG{k}{lambda} \PYG{n}{x}\PYG{p}{:} \PYG{n}{x}\PYG{p}{[}\PYG{l+s+s1}{\PYGZsq{}}\PYG{l+s+s1}{Position}\PYG{l+s+s1}{\PYGZsq{}}\PYG{p}{]} \PYG{o}{*} \PYG{n}{x}\PYG{p}{[}\PYG{l+s+s1}{\PYGZsq{}}\PYG{l+s+s1}{Return}\PYG{l+s+s1}{\PYGZsq{}}\PYG{p}{]}
    \PYG{p}{)}
\end{sphinxVerbatim}

\end{sphinxuseclass}\end{sphinxVerbatimInput}

\end{sphinxuseclass}
\begin{sphinxuseclass}{cell}\begin{sphinxVerbatimInput}

\begin{sphinxuseclass}{cell_input}
\begin{sphinxVerbatim}[commandchars=\\\{\}]
\PYG{n}{btc\PYGZus{}rsi} \PYG{o}{=} \PYG{n}{rsi}\PYG{p}{(}\PYG{n}{btc}\PYG{p}{)}

\PYG{n}{btc\PYGZus{}rsi}\PYG{o}{.}\PYG{n}{tail}\PYG{p}{(}\PYG{p}{)}
\end{sphinxVerbatim}

\end{sphinxuseclass}\end{sphinxVerbatimInput}
\begin{sphinxVerbatimOutput}

\begin{sphinxuseclass}{cell_output}
\begin{sphinxVerbatim}[commandchars=\\\{\}]
Variable         Open       High        Low      Close  Adj Close  \PYGZbs{}
Date                                                                
2023\PYGZhy{}03\PYGZhy{}11 20187.8770 20792.5254 20068.6602 20632.4102 20632.4102   
2023\PYGZhy{}03\PYGZhy{}12 20628.0293 22185.0312 20448.8066 22163.9492 22163.9492   
2023\PYGZhy{}03\PYGZhy{}13 22156.4062 24550.8379 21918.1992 24197.5332 24197.5332   
2023\PYGZhy{}03\PYGZhy{}14 24201.7656 26514.7168 24081.1836 24746.0742 24746.0742   
2023\PYGZhy{}03\PYGZhy{}15 24734.6523 25169.7305 24048.3516 24617.3047 24617.3047   

Variable         Volume  Return      Diff         U        D     SMAU  \PYGZbs{}
Date                                                                    
2023\PYGZhy{}03\PYGZhy{}11  30180288176  0.0221  445.1660  445.1660   0.0000 100.8832   
2023\PYGZhy{}03\PYGZhy{}12  29279035521  0.0742 1531.5391 1531.5391   0.0000 182.7190   
2023\PYGZhy{}03\PYGZhy{}13  49466362688  0.0918 2033.5840 2033.5840   0.0000 327.9750   
2023\PYGZhy{}03\PYGZhy{}14  54622230164  0.0227  548.5410  548.5410   0.0000 367.1565   
2023\PYGZhy{}03\PYGZhy{}15  44081737728 \PYGZhy{}0.0052 \PYGZhy{}128.7695    0.0000 128.7695 331.4996   

Variable       SMAD     RS     RSI  Position\PYGZus{}with\PYGZus{}nan  Position  Strategy  
Date                                                                       
2023\PYGZhy{}03\PYGZhy{}11 282.5236 0.3571 26.3123                NaN    1.0000    0.0221  
2023\PYGZhy{}03\PYGZhy{}12 282.5236 0.6467 39.2739                NaN    1.0000    0.0742  
2023\PYGZhy{}03\PYGZhy{}13 279.7849 1.1722 53.9646             1.0000    1.0000    0.0918  
2023\PYGZhy{}03\PYGZhy{}14 252.9622 1.4514 59.2075             0.0000    0.0000    0.0000  
2023\PYGZhy{}03\PYGZhy{}15 262.1600 1.2645 55.8400                NaN    0.0000   \PYGZhy{}0.0000  
\end{sphinxVerbatim}

\end{sphinxuseclass}\end{sphinxVerbatimOutput}

\end{sphinxuseclass}
\begin{sphinxuseclass}{cell}\begin{sphinxVerbatimInput}

\begin{sphinxuseclass}{cell_input}
\begin{sphinxVerbatim}[commandchars=\\\{\}]
\PYG{n}{\PYGZus{}} \PYG{o}{=} \PYG{n}{btc\PYGZus{}rsi}\PYG{p}{[}\PYG{p}{[}\PYG{l+s+s1}{\PYGZsq{}}\PYG{l+s+s1}{Return}\PYG{l+s+s1}{\PYGZsq{}}\PYG{p}{,} \PYG{l+s+s1}{\PYGZsq{}}\PYG{l+s+s1}{Strategy}\PYG{l+s+s1}{\PYGZsq{}}\PYG{p}{]}\PYG{p}{]}\PYG{o}{.}\PYG{n}{dropna}\PYG{p}{(}\PYG{p}{)}

\PYG{p}{(}
    \PYG{n}{\PYGZus{}}
    \PYG{o}{.}\PYG{n}{add}\PYG{p}{(}\PYG{l+m+mi}{1}\PYG{p}{)}
    \PYG{o}{.}\PYG{n}{cumprod}\PYG{p}{(}\PYG{p}{)}
    \PYG{o}{.}\PYG{n}{rename\PYGZus{}axis}\PYG{p}{(}\PYG{n}{columns}\PYG{o}{=}\PYG{l+s+s1}{\PYGZsq{}}\PYG{l+s+s1}{Strategy}\PYG{l+s+s1}{\PYGZsq{}}\PYG{p}{)}
    \PYG{o}{.}\PYG{n}{rename}\PYG{p}{(}\PYG{n}{columns}\PYG{o}{=}\PYG{p}{\PYGZob{}}\PYG{l+s+s1}{\PYGZsq{}}\PYG{l+s+s1}{Return}\PYG{l+s+s1}{\PYGZsq{}}\PYG{p}{:} \PYG{l+s+s1}{\PYGZsq{}}\PYG{l+s+s1}{Buy\PYGZhy{}And\PYGZhy{}Hold}\PYG{l+s+s1}{\PYGZsq{}}\PYG{p}{,} \PYG{l+s+s1}{\PYGZsq{}}\PYG{l+s+s1}{Strategy}\PYG{l+s+s1}{\PYGZsq{}}\PYG{p}{:} \PYG{l+s+s1}{\PYGZsq{}}\PYG{l+s+s1}{RSI(14)}\PYG{l+s+s1}{\PYGZsq{}}\PYG{p}{\PYGZcb{}}\PYG{p}{)}
    \PYG{o}{.}\PYG{n}{plot}\PYG{p}{(}\PYG{p}{)}
\PYG{p}{)}
\PYG{n}{plt}\PYG{o}{.}\PYG{n}{ylabel}\PYG{p}{(}\PYG{l+s+s1}{\PYGZsq{}}\PYG{l+s+s1}{Value (\PYGZdl{})}\PYG{l+s+s1}{\PYGZsq{}}\PYG{p}{)}
\PYG{n}{plt}\PYG{o}{.}\PYG{n}{title}\PYG{p}{(}\PYG{l+s+sa}{f}\PYG{l+s+s1}{\PYGZsq{}}\PYG{l+s+s1}{Value of \PYGZdl{}1 Invested at Close on }\PYG{l+s+si}{\PYGZob{}}\PYG{n}{\PYGZus{}}\PYG{o}{.}\PYG{n}{index}\PYG{p}{[}\PYG{l+m+mi}{0}\PYG{p}{]} \PYG{o}{\PYGZhy{}} \PYG{n}{pd}\PYG{o}{.}\PYG{n}{offsets}\PYG{o}{.}\PYG{n}{Day}\PYG{p}{(}\PYG{l+m+mi}{1}\PYG{p}{)}\PYG{l+s+si}{:}\PYG{l+s+s1}{\PYGZpc{}B \PYGZpc{}d, \PYGZpc{}Y}\PYG{l+s+si}{\PYGZcb{}}\PYG{l+s+s1}{\PYGZsq{}}\PYG{p}{)}
\PYG{n}{plt}\PYG{o}{.}\PYG{n}{show}\PYG{p}{(}\PYG{p}{)}
\end{sphinxVerbatim}

\end{sphinxuseclass}\end{sphinxVerbatimInput}
\begin{sphinxVerbatimOutput}

\begin{sphinxuseclass}{cell_output}
\noindent\sphinxincludegraphics{{7a408772ea36fc857aee85128d3c0797cc8766924aa0789e89e26ae642d598d7}.png}

\end{sphinxuseclass}\end{sphinxVerbatimOutput}

\end{sphinxuseclass}
\sphinxAtStartPar
We can compare all three!
Shorting Bitcoin has been dangerous, as the poor returns on RSI(14) show!

\begin{sphinxuseclass}{cell}\begin{sphinxVerbatimInput}

\begin{sphinxuseclass}{cell_input}
\begin{sphinxVerbatim}[commandchars=\\\{\}]
\PYG{n}{\PYGZus{}} \PYG{o}{=} \PYG{p}{(}
    \PYG{n}{btc\PYGZus{}sma}\PYG{p}{[}\PYG{p}{[}\PYG{l+s+s1}{\PYGZsq{}}\PYG{l+s+s1}{Return}\PYG{l+s+s1}{\PYGZsq{}}\PYG{p}{,} \PYG{l+s+s1}{\PYGZsq{}}\PYG{l+s+s1}{Strategy}\PYG{l+s+s1}{\PYGZsq{}}\PYG{p}{]}\PYG{p}{]}
    \PYG{o}{.}\PYG{n}{join}\PYG{p}{(}
        \PYG{n}{btc\PYGZus{}bb}\PYG{p}{[}\PYG{p}{[}\PYG{l+s+s1}{\PYGZsq{}}\PYG{l+s+s1}{Strategy}\PYG{l+s+s1}{\PYGZsq{}}\PYG{p}{]}\PYG{p}{]}\PYG{o}{.}\PYG{n}{add\PYGZus{}suffix}\PYG{p}{(}\PYG{l+s+s1}{\PYGZsq{}}\PYG{l+s+s1}{\PYGZus{}BB}\PYG{l+s+s1}{\PYGZsq{}}\PYG{p}{)}\PYG{p}{,} 
    \PYG{p}{)}
    \PYG{o}{.}\PYG{n}{join}\PYG{p}{(}
        \PYG{n}{btc\PYGZus{}rsi}\PYG{p}{[}\PYG{p}{[}\PYG{l+s+s1}{\PYGZsq{}}\PYG{l+s+s1}{Strategy}\PYG{l+s+s1}{\PYGZsq{}}\PYG{p}{]}\PYG{p}{]}\PYG{o}{.}\PYG{n}{add\PYGZus{}suffix}\PYG{p}{(}\PYG{l+s+s1}{\PYGZsq{}}\PYG{l+s+s1}{\PYGZus{}RSI}\PYG{l+s+s1}{\PYGZsq{}}\PYG{p}{)}\PYG{p}{,} 
    \PYG{p}{)}
    \PYG{o}{.}\PYG{n}{dropna}\PYG{p}{(}\PYG{p}{)}
\PYG{p}{)}


\PYG{p}{(}
    \PYG{n}{\PYGZus{}}
    \PYG{o}{.}\PYG{n}{add}\PYG{p}{(}\PYG{l+m+mi}{1}\PYG{p}{)}
    \PYG{o}{.}\PYG{n}{cumprod}\PYG{p}{(}\PYG{p}{)}
    \PYG{o}{.}\PYG{n}{rename\PYGZus{}axis}\PYG{p}{(}\PYG{n}{columns}\PYG{o}{=}\PYG{l+s+s1}{\PYGZsq{}}\PYG{l+s+s1}{Strategy}\PYG{l+s+s1}{\PYGZsq{}}\PYG{p}{)}
    \PYG{o}{.}\PYG{n}{rename}\PYG{p}{(}\PYG{n}{columns}\PYG{o}{=}
            \PYG{p}{\PYGZob{}}
                \PYG{l+s+s1}{\PYGZsq{}}\PYG{l+s+s1}{Return}\PYG{l+s+s1}{\PYGZsq{}}\PYG{p}{:} \PYG{l+s+s1}{\PYGZsq{}}\PYG{l+s+s1}{Buy\PYGZhy{}And\PYGZhy{}Hold}\PYG{l+s+s1}{\PYGZsq{}}\PYG{p}{,} 
                \PYG{l+s+s1}{\PYGZsq{}}\PYG{l+s+s1}{Strategy}\PYG{l+s+s1}{\PYGZsq{}}\PYG{p}{:} \PYG{l+s+s1}{\PYGZsq{}}\PYG{l+s+s1}{SMA(20)}\PYG{l+s+s1}{\PYGZsq{}}\PYG{p}{,}
                \PYG{l+s+s1}{\PYGZsq{}}\PYG{l+s+s1}{Strategy\PYGZus{}BB}\PYG{l+s+s1}{\PYGZsq{}}\PYG{p}{:} \PYG{l+s+s1}{\PYGZsq{}}\PYG{l+s+s1}{BB(20, 2)}\PYG{l+s+s1}{\PYGZsq{}}\PYG{p}{,}
                \PYG{l+s+s1}{\PYGZsq{}}\PYG{l+s+s1}{Strategy\PYGZus{}RSI}\PYG{l+s+s1}{\PYGZsq{}}\PYG{p}{:} \PYG{l+s+s1}{\PYGZsq{}}\PYG{l+s+s1}{RSI(14)}\PYG{l+s+s1}{\PYGZsq{}}\PYG{p}{,}
            \PYG{p}{\PYGZcb{}}
           \PYG{p}{)}
    \PYG{o}{.}\PYG{n}{plot}\PYG{p}{(}\PYG{p}{)}
\PYG{p}{)}
\PYG{n}{plt}\PYG{o}{.}\PYG{n}{semilogy}\PYG{p}{(}\PYG{p}{)}
\PYG{n}{plt}\PYG{o}{.}\PYG{n}{ylabel}\PYG{p}{(}\PYG{l+s+s1}{\PYGZsq{}}\PYG{l+s+s1}{Value (\PYGZdl{})}\PYG{l+s+s1}{\PYGZsq{}}\PYG{p}{)}
\PYG{n}{plt}\PYG{o}{.}\PYG{n}{title}\PYG{p}{(}\PYG{l+s+sa}{f}\PYG{l+s+s1}{\PYGZsq{}}\PYG{l+s+s1}{Value of \PYGZdl{}1 Invested at Close on }\PYG{l+s+si}{\PYGZob{}}\PYG{n}{\PYGZus{}}\PYG{o}{.}\PYG{n}{index}\PYG{p}{[}\PYG{l+m+mi}{0}\PYG{p}{]} \PYG{o}{\PYGZhy{}} \PYG{n}{pd}\PYG{o}{.}\PYG{n}{offsets}\PYG{o}{.}\PYG{n}{Day}\PYG{p}{(}\PYG{l+m+mi}{1}\PYG{p}{)}\PYG{l+s+si}{:}\PYG{l+s+s1}{\PYGZpc{}B \PYGZpc{}d, \PYGZpc{}Y}\PYG{l+s+si}{\PYGZcb{}}\PYG{l+s+s1}{\PYGZsq{}}\PYG{p}{)}
\PYG{n}{plt}\PYG{o}{.}\PYG{n}{show}\PYG{p}{(}\PYG{p}{)}
\end{sphinxVerbatim}

\end{sphinxuseclass}\end{sphinxVerbatimInput}
\begin{sphinxVerbatimOutput}

\begin{sphinxuseclass}{cell_output}
\noindent\sphinxincludegraphics{{bca6fe8a2ee1156bf6d412f316c36d0be9567385beaf9e1190a825a35bc69504}.png}

\end{sphinxuseclass}\end{sphinxVerbatimOutput}

\end{sphinxuseclass}
\sphinxstepscope


\chapter{Herron Topic 3 \sphinxhyphen{} Multifactor Models}
\label{\detokenize{herron_03_lecture:herron-topic-3-multifactor-models}}\label{\detokenize{herron_03_lecture::doc}}
\sphinxAtStartPar
This notebook covers multifactor models, emphasizing the capital asset pricing model (CAPM) and the Fama\sphinxhyphen{}French three\sphinxhyphen{}factor model (FF3).
Ivo Welch provides a high\sphinxhyphen{}level review of the CAPM and multifactor models in \sphinxhref{https://book.ivo-welch.info/read/source5.mba/10-capm.pdf}{Chapter 10 of his free \sphinxstyleemphasis{Corporate Finance} textbook}.
The \sphinxhref{https://en.wikipedia.org/wiki/Capital\_asset\_pricing\_model}{Wikipedia page for the CAPM} is surprisingly good and includes its assumptions and shortcomings.

\begin{sphinxuseclass}{cell}\begin{sphinxVerbatimInput}

\begin{sphinxuseclass}{cell_input}
\begin{sphinxVerbatim}[commandchars=\\\{\}]
\PYG{k+kn}{import} \PYG{n+nn}{matplotlib}\PYG{n+nn}{.}\PYG{n+nn}{pyplot} \PYG{k}{as} \PYG{n+nn}{plt}
\PYG{k+kn}{import} \PYG{n+nn}{numpy} \PYG{k}{as} \PYG{n+nn}{np}
\PYG{k+kn}{import} \PYG{n+nn}{pandas} \PYG{k}{as} \PYG{n+nn}{pd}
\end{sphinxVerbatim}

\end{sphinxuseclass}\end{sphinxVerbatimInput}

\end{sphinxuseclass}
\begin{sphinxuseclass}{cell}\begin{sphinxVerbatimInput}

\begin{sphinxuseclass}{cell_input}
\begin{sphinxVerbatim}[commandchars=\\\{\}]
\PYG{o}{\PYGZpc{}}\PYG{k}{config} InlineBackend.figure\PYGZus{}format = \PYGZsq{}retina\PYGZsq{}
\PYG{o}{\PYGZpc{}}\PYG{k}{precision} 4
\PYG{n}{pd}\PYG{o}{.}\PYG{n}{options}\PYG{o}{.}\PYG{n}{display}\PYG{o}{.}\PYG{n}{float\PYGZus{}format} \PYG{o}{=} \PYG{l+s+s1}{\PYGZsq{}}\PYG{l+s+si}{\PYGZob{}:.4f\PYGZcb{}}\PYG{l+s+s1}{\PYGZsq{}}\PYG{o}{.}\PYG{n}{format}
\end{sphinxVerbatim}

\end{sphinxuseclass}\end{sphinxVerbatimInput}

\end{sphinxuseclass}

\section{The Capital Asset Pricing Model (CAPM)}
\label{\detokenize{herron_03_lecture:the-capital-asset-pricing-model-capm}}
\sphinxAtStartPar
The CAPM explains the relation between non\sphinxhyphen{}diversifiable risk and expected return, and it has applications throughout finance.
We use the CAPM to estimate costs of capital in corporate finance, assemble portfolios with a given risk\sphinxhyphen{}return tradeoff in portfolio management, and estimate expected rates of return in investment analysis.
The formula for the CAPM is \$E(R\_i) = R\_F + \textbackslash{}beta\_i {[}E(R\_M) \sphinxhyphen{} R\_F{]}\$, where:
\begin{enumerate}
\sphinxsetlistlabels{\arabic}{enumi}{enumii}{}{.}%
\item {} 
\sphinxAtStartPar
\$R\_F\$ is the risk\sphinxhyphen{}free rate of return,

\item {} 
\sphinxAtStartPar
\$\textbackslash{}beta\_i\$ is the measure of non\sphinxhyphen{}diversifiable risk for asset \$i\$, and

\item {} 
\sphinxAtStartPar
\$E(R\_M)\$ is the expected rate of return on the market.

\end{enumerate}

\sphinxAtStartPar
Here, \$\textbackslash{}beta\_i\$ measures asset \$i\$’s risk exposure or sensitivity to market returns.
The value\sphinxhyphen{}weighted mean of \$\textbackslash{}beta\_i\$’s is 1 by construction, but a range of values is possible:
\begin{enumerate}
\sphinxsetlistlabels{\arabic}{enumi}{enumii}{}{.}%
\item {} 
\sphinxAtStartPar
\$\textbackslash{}beta\_i < \sphinxhyphen{}1\$: Asset \$i\$ moves in the opposite direction as the market, in larger magnitudes

\item {} 
\sphinxAtStartPar
\$\sphinxhyphen{}1 \textbackslash{}leq \textbackslash{}beta\_i < 0\$: Asset \$i\$ moves in the opposite direction as the market

\item {} 
\sphinxAtStartPar
\$\textbackslash{}beta\_i = 0\$: Asset \$i\$ has no correlation between with the market

\item {} 
\sphinxAtStartPar
\$0 < \textbackslash{}beta\_i \textbackslash{}leq 1\$: Asset \$i\$ moves in the same direction as the market, in smaller magnitudes

\item {} 
\sphinxAtStartPar
\$\textbackslash{}beta\_i = 1\$: Asset \$i\$ moves in the same direction with the same magnitude as the market

\item {} 
\sphinxAtStartPar
\$\textbackslash{}beta\_i > 1\$: Asset \$i\$ moves in the same direction as the market, in larger magnitudes

\end{enumerate}


\subsection{\$\textbackslash{}beta\$ Estimation}
\label{\detokenize{herron_03_lecture:beta-estimation}}
\sphinxAtStartPar
We can use three (equivalent) approaches to estimate \$\textbackslash{}beta\_i\$:
\begin{enumerate}
\sphinxsetlistlabels{\arabic}{enumi}{enumii}{}{.}%
\item {} 
\sphinxAtStartPar
From covariance and variance as \$\textbackslash{}beta\_i = \textbackslash{}frac\{Cov(R\_i \sphinxhyphen{} R\_F, R\_M \sphinxhyphen{} R\_F)\}\{Var(R\_M \sphinxhyphen{} R\_F)\}\$

\item {} 
\sphinxAtStartPar
From correlation and standard deviations as \$\textbackslash{}beta\_i = \textbackslash{}rho\_\{i, M\} \textbackslash{}cdot \textbackslash{}frac\{\textbackslash{}sigma\_i\}\{\textbackslash{}sigma\_M\}\$, where all statistics use \sphinxstyleemphasis{excess} returns (i.e., \$R\_i\sphinxhyphen{}R\_F\$ and \$R\_M\sphinxhyphen{}R\_F\$)

\item {} 
\sphinxAtStartPar
From a linear regression of \$R\_i\sphinxhyphen{}R\_F\$ on \$R\_M\sphinxhyphen{}R\_F\$

\end{enumerate}

\sphinxAtStartPar
The first two approaches are computationally simpler.
However, the third approach also estimates the intercept \$\textbackslash{}alpha\$ and goodness\sphinxhyphen{}of\sphinxhyphen{}fit statistics.
We can use Apple (AAPL) to convince ourselves these three approaches are equivalent.

\begin{sphinxuseclass}{cell}\begin{sphinxVerbatimInput}

\begin{sphinxuseclass}{cell_input}
\begin{sphinxVerbatim}[commandchars=\\\{\}]
\PYG{k+kn}{import} \PYG{n+nn}{yfinance} \PYG{k}{as} \PYG{n+nn}{yf}
\PYG{k+kn}{import} \PYG{n+nn}{pandas\PYGZus{}datareader} \PYG{k}{as} \PYG{n+nn}{pdr}
\PYG{k+kn}{import} \PYG{n+nn}{requests\PYGZus{}cache}
\PYG{n}{session} \PYG{o}{=} \PYG{n}{requests\PYGZus{}cache}\PYG{o}{.}\PYG{n}{CachedSession}\PYG{p}{(}\PYG{p}{)}
\end{sphinxVerbatim}

\end{sphinxuseclass}\end{sphinxVerbatimInput}

\end{sphinxuseclass}
\sphinxAtStartPar
\sphinxstyleemphasis{\sphinxstylestrong{Note, we will leave returns in percent to make it easier to interpret our regression output!}}
Leaving returns in percent does not affect the \$\textbackslash{}beta\$s (slopes) but makes the \$\textbackslash{}alpha\$ (intercept) easier to interpret by removing two leading zeros.

\begin{sphinxuseclass}{cell}\begin{sphinxVerbatimInput}

\begin{sphinxuseclass}{cell_input}
\begin{sphinxVerbatim}[commandchars=\\\{\}]
\PYG{n}{aapl} \PYG{o}{=} \PYG{p}{(}
    \PYG{n}{yf}\PYG{o}{.}\PYG{n}{download}\PYG{p}{(}\PYG{n}{tickers}\PYG{o}{=}\PYG{l+s+s1}{\PYGZsq{}}\PYG{l+s+s1}{AAPL}\PYG{l+s+s1}{\PYGZsq{}}\PYG{p}{,} \PYG{n}{progress}\PYG{o}{=}\PYG{k+kc}{False}\PYG{p}{)}
    \PYG{o}{.}\PYG{n}{assign}\PYG{p}{(}
        \PYG{n}{Date}\PYG{o}{=}\PYG{k}{lambda} \PYG{n}{x}\PYG{p}{:} \PYG{n}{x}\PYG{o}{.}\PYG{n}{index}\PYG{o}{.}\PYG{n}{tz\PYGZus{}localize}\PYG{p}{(}\PYG{k+kc}{None}\PYG{p}{)}\PYG{p}{,}
        \PYG{n}{Ri}\PYG{o}{=}\PYG{k}{lambda} \PYG{n}{x}\PYG{p}{:} \PYG{n}{x}\PYG{p}{[}\PYG{l+s+s1}{\PYGZsq{}}\PYG{l+s+s1}{Adj Close}\PYG{l+s+s1}{\PYGZsq{}}\PYG{p}{]}\PYG{o}{.}\PYG{n}{pct\PYGZus{}change}\PYG{p}{(}\PYG{p}{)}\PYG{o}{.}\PYG{n}{mul}\PYG{p}{(}\PYG{l+m+mi}{100}\PYG{p}{)}
    \PYG{p}{)}
    \PYG{o}{.}\PYG{n}{set\PYGZus{}index}\PYG{p}{(}\PYG{l+s+s1}{\PYGZsq{}}\PYG{l+s+s1}{Date}\PYG{l+s+s1}{\PYGZsq{}}\PYG{p}{)}
    \PYG{o}{.}\PYG{n}{rename\PYGZus{}axis}\PYG{p}{(}\PYG{n}{columns}\PYG{o}{=}\PYG{l+s+s1}{\PYGZsq{}}\PYG{l+s+s1}{Variable}\PYG{l+s+s1}{\PYGZsq{}}\PYG{p}{)}
\PYG{p}{)}

\PYG{n}{aapl}\PYG{o}{.}\PYG{n}{tail}\PYG{p}{(}\PYG{p}{)}
\end{sphinxVerbatim}

\end{sphinxuseclass}\end{sphinxVerbatimInput}
\begin{sphinxVerbatimOutput}

\begin{sphinxuseclass}{cell_output}
\begin{sphinxVerbatim}[commandchars=\\\{\}]
Variable       Open     High      Low    Close  Adj Close    Volume      Ri
Date                                                                       
2023\PYGZhy{}03\PYGZhy{}08 152.8100 153.4700 151.8300 152.8700   152.8700  47204800  0.8377
2023\PYGZhy{}03\PYGZhy{}09 153.5600 154.5400 150.2300 150.5900   150.5900  53833600 \PYGZhy{}1.4915
2023\PYGZhy{}03\PYGZhy{}10 150.2100 150.9400 147.6100 148.5000   148.5000  68524400 \PYGZhy{}1.3879
2023\PYGZhy{}03\PYGZhy{}13 147.8100 153.1400 147.7000 150.4700   150.4700  84457100  1.3266
2023\PYGZhy{}03\PYGZhy{}14 151.2800 153.4000 150.1000 152.5900   152.5900  73628400  1.4089
\end{sphinxVerbatim}

\end{sphinxuseclass}\end{sphinxVerbatimOutput}

\end{sphinxuseclass}
\begin{sphinxuseclass}{cell}\begin{sphinxVerbatimInput}

\begin{sphinxuseclass}{cell_input}
\begin{sphinxVerbatim}[commandchars=\\\{\}]
\PYG{n}{ff} \PYG{o}{=} \PYG{p}{(}
    \PYG{n}{pdr}\PYG{o}{.}\PYG{n}{DataReader}\PYG{p}{(}
        \PYG{n}{name}\PYG{o}{=}\PYG{l+s+s1}{\PYGZsq{}}\PYG{l+s+s1}{F\PYGZhy{}F\PYGZus{}Research\PYGZus{}Data\PYGZus{}Factors\PYGZus{}daily}\PYG{l+s+s1}{\PYGZsq{}}\PYG{p}{,}
        \PYG{n}{data\PYGZus{}source}\PYG{o}{=}\PYG{l+s+s1}{\PYGZsq{}}\PYG{l+s+s1}{famafrench}\PYG{l+s+s1}{\PYGZsq{}}\PYG{p}{,}
        \PYG{n}{start}\PYG{o}{=}\PYG{l+s+s1}{\PYGZsq{}}\PYG{l+s+s1}{1900}\PYG{l+s+s1}{\PYGZsq{}}\PYG{p}{,}
        \PYG{n}{session}\PYG{o}{=}\PYG{n}{session}
    \PYG{p}{)}
\PYG{p}{)}
\end{sphinxVerbatim}

\end{sphinxuseclass}\end{sphinxVerbatimInput}

\end{sphinxuseclass}
\begin{sphinxuseclass}{cell}\begin{sphinxVerbatimInput}

\begin{sphinxuseclass}{cell_input}
\begin{sphinxVerbatim}[commandchars=\\\{\}]
\PYG{n}{aapl} \PYG{o}{=} \PYG{p}{(}
    \PYG{n}{aapl}
    \PYG{o}{.}\PYG{n}{join}\PYG{p}{(}\PYG{n}{ff}\PYG{p}{[}\PYG{l+m+mi}{0}\PYG{p}{]}\PYG{p}{)}
    \PYG{o}{.}\PYG{n}{assign}\PYG{p}{(}\PYG{n}{RiRF} \PYG{o}{=} \PYG{k}{lambda} \PYG{n}{x}\PYG{p}{:} \PYG{n}{x}\PYG{p}{[}\PYG{l+s+s1}{\PYGZsq{}}\PYG{l+s+s1}{Ri}\PYG{l+s+s1}{\PYGZsq{}}\PYG{p}{]} \PYG{o}{\PYGZhy{}} \PYG{n}{x}\PYG{p}{[}\PYG{l+s+s1}{\PYGZsq{}}\PYG{l+s+s1}{RF}\PYG{l+s+s1}{\PYGZsq{}}\PYG{p}{]}\PYG{p}{)}
    \PYG{o}{.}\PYG{n}{rename}\PYG{p}{(}\PYG{n}{columns}\PYG{o}{=}\PYG{p}{\PYGZob{}}\PYG{l+s+s1}{\PYGZsq{}}\PYG{l+s+s1}{Mkt\PYGZhy{}RF}\PYG{l+s+s1}{\PYGZsq{}}\PYG{p}{:} \PYG{l+s+s1}{\PYGZsq{}}\PYG{l+s+s1}{MktRF}\PYG{l+s+s1}{\PYGZsq{}}\PYG{p}{\PYGZcb{}}\PYG{p}{)}
\PYG{p}{)}
\end{sphinxVerbatim}

\end{sphinxuseclass}\end{sphinxVerbatimInput}

\end{sphinxuseclass}
\begin{sphinxuseclass}{cell}\begin{sphinxVerbatimInput}

\begin{sphinxuseclass}{cell_input}
\begin{sphinxVerbatim}[commandchars=\\\{\}]
\PYG{n}{aapl}\PYG{o}{.}\PYG{n}{head}\PYG{p}{(}\PYG{p}{)}
\end{sphinxVerbatim}

\end{sphinxuseclass}\end{sphinxVerbatimInput}
\begin{sphinxVerbatimOutput}

\begin{sphinxuseclass}{cell_output}
\begin{sphinxVerbatim}[commandchars=\\\{\}]
             Open   High    Low  Close  Adj Close     Volume      Ri  MktRF  \PYGZbs{}
Date                                                                          
1980\PYGZhy{}12\PYGZhy{}12 0.1283 0.1289 0.1283 0.1283     0.0997  469033600     NaN 1.3800   
1980\PYGZhy{}12\PYGZhy{}15 0.1222 0.1222 0.1217 0.1217     0.0945  175884800 \PYGZhy{}5.2170 0.1100   
1980\PYGZhy{}12\PYGZhy{}16 0.1133 0.1133 0.1127 0.1127     0.0876  105728000 \PYGZhy{}7.3398 0.7100   
1980\PYGZhy{}12\PYGZhy{}17 0.1155 0.1161 0.1155 0.1155     0.0897   86441600  2.4751 1.5200   
1980\PYGZhy{}12\PYGZhy{}18 0.1189 0.1194 0.1189 0.1189     0.0924   73449600  2.8993 0.4100   

               SMB     HML     RF    RiRF  
Date                                       
1980\PYGZhy{}12\PYGZhy{}12 \PYGZhy{}0.0100 \PYGZhy{}1.0500 0.0590     NaN  
1980\PYGZhy{}12\PYGZhy{}15  0.2500 \PYGZhy{}0.4600 0.0590 \PYGZhy{}5.2760  
1980\PYGZhy{}12\PYGZhy{}16 \PYGZhy{}0.7500 \PYGZhy{}0.4700 0.0590 \PYGZhy{}7.3988  
1980\PYGZhy{}12\PYGZhy{}17 \PYGZhy{}0.8600 \PYGZhy{}0.3400 0.0590  2.4161  
1980\PYGZhy{}12\PYGZhy{}18  0.2200  1.2600 0.0590  2.8403  
\end{sphinxVerbatim}

\end{sphinxuseclass}\end{sphinxVerbatimOutput}

\end{sphinxuseclass}

\subsubsection{Covariance and Variance}
\label{\detokenize{herron_03_lecture:covariance-and-variance}}
\begin{sphinxuseclass}{cell}\begin{sphinxVerbatimInput}

\begin{sphinxuseclass}{cell_input}
\begin{sphinxVerbatim}[commandchars=\\\{\}]
\PYG{n}{vcv} \PYG{o}{=} \PYG{n}{aapl}\PYG{p}{[}\PYG{p}{[}\PYG{l+s+s1}{\PYGZsq{}}\PYG{l+s+s1}{MktRF}\PYG{l+s+s1}{\PYGZsq{}}\PYG{p}{,} \PYG{l+s+s1}{\PYGZsq{}}\PYG{l+s+s1}{RiRF}\PYG{l+s+s1}{\PYGZsq{}}\PYG{p}{]}\PYG{p}{]}\PYG{o}{.}\PYG{n}{dropna}\PYG{p}{(}\PYG{p}{)}\PYG{o}{.}\PYG{n}{cov}\PYG{p}{(}\PYG{p}{)}
\PYG{n}{vcv}
\end{sphinxVerbatim}

\end{sphinxuseclass}\end{sphinxVerbatimInput}
\begin{sphinxVerbatimOutput}

\begin{sphinxuseclass}{cell_output}
\begin{sphinxVerbatim}[commandchars=\\\{\}]
       MktRF   RiRF
MktRF 1.2441 1.5706
RiRF  1.5706 7.9996
\end{sphinxVerbatim}

\end{sphinxuseclass}\end{sphinxVerbatimOutput}

\end{sphinxuseclass}
\begin{sphinxuseclass}{cell}\begin{sphinxVerbatimInput}

\begin{sphinxuseclass}{cell_input}
\begin{sphinxVerbatim}[commandchars=\\\{\}]
\PYG{n+nb}{print}\PYG{p}{(}\PYG{l+s+sa}{f}\PYG{l+s+s2}{\PYGZdq{}}\PYG{l+s+s2}{Apple beta from cov/var: }\PYG{l+s+si}{\PYGZob{}}\PYG{n}{vcv}\PYG{o}{.}\PYG{n}{loc}\PYG{p}{[}\PYG{l+s+s1}{\PYGZsq{}}\PYG{l+s+s1}{MktRF}\PYG{l+s+s1}{\PYGZsq{}}\PYG{p}{,} \PYG{l+s+s1}{\PYGZsq{}}\PYG{l+s+s1}{RiRF}\PYG{l+s+s1}{\PYGZsq{}}\PYG{p}{]} \PYG{o}{/} \PYG{n}{vcv}\PYG{o}{.}\PYG{n}{loc}\PYG{p}{[}\PYG{l+s+s1}{\PYGZsq{}}\PYG{l+s+s1}{MktRF}\PYG{l+s+s1}{\PYGZsq{}}\PYG{p}{,} \PYG{l+s+s1}{\PYGZsq{}}\PYG{l+s+s1}{MktRF}\PYG{l+s+s1}{\PYGZsq{}}\PYG{p}{]}\PYG{l+s+si}{:}\PYG{l+s+s2}{0.4f}\PYG{l+s+si}{\PYGZcb{}}\PYG{l+s+s2}{\PYGZdq{}}\PYG{p}{)}
\end{sphinxVerbatim}

\end{sphinxuseclass}\end{sphinxVerbatimInput}
\begin{sphinxVerbatimOutput}

\begin{sphinxuseclass}{cell_output}
\begin{sphinxVerbatim}[commandchars=\\\{\}]
Apple beta from cov/var: 1.2625
\end{sphinxVerbatim}

\end{sphinxuseclass}\end{sphinxVerbatimOutput}

\end{sphinxuseclass}

\subsubsection{Correlation and Standard Deviations}
\label{\detokenize{herron_03_lecture:correlation-and-standard-deviations}}
\begin{sphinxuseclass}{cell}\begin{sphinxVerbatimInput}

\begin{sphinxuseclass}{cell_input}
\begin{sphinxVerbatim}[commandchars=\\\{\}]
\PYG{n}{\PYGZus{}} \PYG{o}{=} \PYG{n}{aapl}\PYG{p}{[}\PYG{p}{[}\PYG{l+s+s1}{\PYGZsq{}}\PYG{l+s+s1}{MktRF}\PYG{l+s+s1}{\PYGZsq{}}\PYG{p}{,} \PYG{l+s+s1}{\PYGZsq{}}\PYG{l+s+s1}{RiRF}\PYG{l+s+s1}{\PYGZsq{}}\PYG{p}{]}\PYG{p}{]}\PYG{o}{.}\PYG{n}{dropna}\PYG{p}{(}\PYG{p}{)}
\PYG{n}{rho} \PYG{o}{=} \PYG{n}{\PYGZus{}}\PYG{o}{.}\PYG{n}{corr}\PYG{p}{(}\PYG{p}{)}
\PYG{n}{sigma} \PYG{o}{=} \PYG{n}{\PYGZus{}}\PYG{o}{.}\PYG{n}{std}\PYG{p}{(}\PYG{p}{)}
\PYG{n+nb}{print}\PYG{p}{(}\PYG{l+s+sa}{f}\PYG{l+s+s1}{\PYGZsq{}}\PYG{l+s+s1}{rho:}\PYG{l+s+se}{\PYGZbs{}n}\PYG{l+s+si}{\PYGZob{}}\PYG{n}{rho}\PYG{l+s+si}{\PYGZcb{}}\PYG{l+s+se}{\PYGZbs{}n}\PYG{l+s+se}{\PYGZbs{}n}\PYG{l+s+s1}{sigma:}\PYG{l+s+se}{\PYGZbs{}n}\PYG{l+s+si}{\PYGZob{}}\PYG{n}{sigma}\PYG{l+s+si}{\PYGZcb{}}\PYG{l+s+s1}{\PYGZsq{}}\PYG{p}{)}
\end{sphinxVerbatim}

\end{sphinxuseclass}\end{sphinxVerbatimInput}
\begin{sphinxVerbatimOutput}

\begin{sphinxuseclass}{cell_output}
\begin{sphinxVerbatim}[commandchars=\\\{\}]
rho:
       MktRF   RiRF
MktRF 1.0000 0.4979
RiRF  0.4979 1.0000

sigma:
MktRF   1.1154
RiRF    2.8284
dtype: float64
\end{sphinxVerbatim}

\end{sphinxuseclass}\end{sphinxVerbatimOutput}

\end{sphinxuseclass}
\begin{sphinxuseclass}{cell}\begin{sphinxVerbatimInput}

\begin{sphinxuseclass}{cell_input}
\begin{sphinxVerbatim}[commandchars=\\\{\}]
\PYG{n+nb}{print}\PYG{p}{(}\PYG{l+s+sa}{f}\PYG{l+s+s2}{\PYGZdq{}}\PYG{l+s+s2}{Apple beta from rho and sigmas: }\PYG{l+s+si}{\PYGZob{}}\PYG{n}{rho}\PYG{o}{.}\PYG{n}{loc}\PYG{p}{[}\PYG{l+s+s1}{\PYGZsq{}}\PYG{l+s+s1}{MktRF}\PYG{l+s+s1}{\PYGZsq{}}\PYG{p}{,} \PYG{l+s+s1}{\PYGZsq{}}\PYG{l+s+s1}{RiRF}\PYG{l+s+s1}{\PYGZsq{}}\PYG{p}{]} \PYG{o}{*} \PYG{n}{sigma}\PYG{o}{.}\PYG{n}{loc}\PYG{p}{[}\PYG{l+s+s1}{\PYGZsq{}}\PYG{l+s+s1}{RiRF}\PYG{l+s+s1}{\PYGZsq{}}\PYG{p}{]} \PYG{o}{/} \PYG{n}{sigma}\PYG{o}{.}\PYG{n}{loc}\PYG{p}{[}\PYG{l+s+s1}{\PYGZsq{}}\PYG{l+s+s1}{MktRF}\PYG{l+s+s1}{\PYGZsq{}}\PYG{p}{]}\PYG{l+s+si}{:}\PYG{l+s+s2}{0.4f}\PYG{l+s+si}{\PYGZcb{}}\PYG{l+s+s2}{\PYGZdq{}}\PYG{p}{)}
\end{sphinxVerbatim}

\end{sphinxuseclass}\end{sphinxVerbatimInput}
\begin{sphinxVerbatimOutput}

\begin{sphinxuseclass}{cell_output}
\begin{sphinxVerbatim}[commandchars=\\\{\}]
Apple beta from rho and sigmas: 1.2625
\end{sphinxVerbatim}

\end{sphinxuseclass}\end{sphinxVerbatimOutput}

\end{sphinxuseclass}

\subsubsection{Linear Regression}
\label{\detokenize{herron_03_lecture:linear-regression}}
\sphinxAtStartPar
We will use the statsmodels package to estimate linear.
I typically use the formula application programming interface (API).

\begin{sphinxuseclass}{cell}\begin{sphinxVerbatimInput}

\begin{sphinxuseclass}{cell_input}
\begin{sphinxVerbatim}[commandchars=\\\{\}]
\PYG{k+kn}{import} \PYG{n+nn}{statsmodels}\PYG{n+nn}{.}\PYG{n+nn}{formula}\PYG{n+nn}{.}\PYG{n+nn}{api} \PYG{k}{as} \PYG{n+nn}{smf}
\end{sphinxVerbatim}

\end{sphinxuseclass}\end{sphinxVerbatimInput}

\end{sphinxuseclass}
\sphinxAtStartPar
With statsmodels (and most Python model estimation packages), we have three steps:
\begin{enumerate}
\sphinxsetlistlabels{\arabic}{enumi}{enumii}{}{.}%
\item {} 
\sphinxAtStartPar
Specify the model

\item {} 
\sphinxAtStartPar
Fit the model

\item {} 
\sphinxAtStartPar
Summarize the model

\end{enumerate}

\begin{sphinxuseclass}{cell}\begin{sphinxVerbatimInput}

\begin{sphinxuseclass}{cell_input}
\begin{sphinxVerbatim}[commandchars=\\\{\}]
\PYG{n}{model} \PYG{o}{=} \PYG{n}{smf}\PYG{o}{.}\PYG{n}{ols}\PYG{p}{(}\PYG{l+s+s1}{\PYGZsq{}}\PYG{l+s+s1}{RiRF \PYGZti{} MktRF}\PYG{l+s+s1}{\PYGZsq{}}\PYG{p}{,} \PYG{n}{aapl}\PYG{p}{)}
\PYG{n}{fit} \PYG{o}{=} \PYG{n}{model}\PYG{o}{.}\PYG{n}{fit}\PYG{p}{(}\PYG{p}{)}
\PYG{n}{summary} \PYG{o}{=} \PYG{n}{fit}\PYG{o}{.}\PYG{n}{summary}\PYG{p}{(}\PYG{p}{)}
\PYG{n}{summary}
\end{sphinxVerbatim}

\end{sphinxuseclass}\end{sphinxVerbatimInput}
\begin{sphinxVerbatimOutput}

\begin{sphinxuseclass}{cell_output}
\begin{sphinxVerbatim}[commandchars=\\\{\}]
\PYGZlt{}class \PYGZsq{}statsmodels.iolib.summary.Summary\PYGZsq{}\PYGZgt{}
\PYGZdq{}\PYGZdq{}\PYGZdq{}
                            OLS Regression Results                            
==============================================================================
Dep. Variable:                   RiRF   R\PYGZhy{}squared:                       0.248
Model:                            OLS   Adj. R\PYGZhy{}squared:                  0.248
Method:                 Least Squares   F\PYGZhy{}statistic:                     3493.
Date:                Wed, 15 Mar 2023   Prob (F\PYGZhy{}statistic):               0.00
Time:                        11:20:50   Log\PYGZhy{}Likelihood:                \PYGZhy{}24556.
No. Observations:               10602   AIC:                         4.912e+04
Df Residuals:                   10600   BIC:                         4.913e+04
Df Model:                           1                                         
Covariance Type:            nonrobust                                         
==============================================================================
                 coef    std err          t      P\PYGZgt{}|t|      [0.025      0.975]
\PYGZhy{}\PYGZhy{}\PYGZhy{}\PYGZhy{}\PYGZhy{}\PYGZhy{}\PYGZhy{}\PYGZhy{}\PYGZhy{}\PYGZhy{}\PYGZhy{}\PYGZhy{}\PYGZhy{}\PYGZhy{}\PYGZhy{}\PYGZhy{}\PYGZhy{}\PYGZhy{}\PYGZhy{}\PYGZhy{}\PYGZhy{}\PYGZhy{}\PYGZhy{}\PYGZhy{}\PYGZhy{}\PYGZhy{}\PYGZhy{}\PYGZhy{}\PYGZhy{}\PYGZhy{}\PYGZhy{}\PYGZhy{}\PYGZhy{}\PYGZhy{}\PYGZhy{}\PYGZhy{}\PYGZhy{}\PYGZhy{}\PYGZhy{}\PYGZhy{}\PYGZhy{}\PYGZhy{}\PYGZhy{}\PYGZhy{}\PYGZhy{}\PYGZhy{}\PYGZhy{}\PYGZhy{}\PYGZhy{}\PYGZhy{}\PYGZhy{}\PYGZhy{}\PYGZhy{}\PYGZhy{}\PYGZhy{}\PYGZhy{}\PYGZhy{}\PYGZhy{}\PYGZhy{}\PYGZhy{}\PYGZhy{}\PYGZhy{}\PYGZhy{}\PYGZhy{}\PYGZhy{}\PYGZhy{}\PYGZhy{}\PYGZhy{}\PYGZhy{}\PYGZhy{}\PYGZhy{}\PYGZhy{}\PYGZhy{}\PYGZhy{}\PYGZhy{}\PYGZhy{}\PYGZhy{}\PYGZhy{}
Intercept      0.0520      0.024      2.180      0.029       0.005       0.099
MktRF          1.2625      0.021     59.105      0.000       1.221       1.304
==============================================================================
Omnibus:                     3126.527   Durbin\PYGZhy{}Watson:                   1.917
Prob(Omnibus):                  0.000   Jarque\PYGZhy{}Bera (JB):           313356.303
Skew:                          \PYGZhy{}0.376   Prob(JB):                         0.00
Kurtosis:                      29.623   Cond. No.                         1.12
==============================================================================

Notes:
[1] Standard Errors assume that the covariance matrix of the errors is correctly specified.
\PYGZdq{}\PYGZdq{}\PYGZdq{}
\end{sphinxVerbatim}

\end{sphinxuseclass}\end{sphinxVerbatimOutput}

\end{sphinxuseclass}
\begin{sphinxuseclass}{cell}\begin{sphinxVerbatimInput}

\begin{sphinxuseclass}{cell_input}
\begin{sphinxVerbatim}[commandchars=\\\{\}]
\PYG{n}{fit}\PYG{o}{.}\PYG{n}{params}
\end{sphinxVerbatim}

\end{sphinxuseclass}\end{sphinxVerbatimInput}
\begin{sphinxVerbatimOutput}

\begin{sphinxuseclass}{cell_output}
\begin{sphinxVerbatim}[commandchars=\\\{\}]
Intercept   0.0520
MktRF       1.2625
dtype: float64
\end{sphinxVerbatim}

\end{sphinxuseclass}\end{sphinxVerbatimOutput}

\end{sphinxuseclass}
\begin{sphinxuseclass}{cell}\begin{sphinxVerbatimInput}

\begin{sphinxuseclass}{cell_input}
\begin{sphinxVerbatim}[commandchars=\\\{\}]
\PYG{n+nb}{print}\PYG{p}{(}\PYG{l+s+sa}{f}\PYG{l+s+s2}{\PYGZdq{}}\PYG{l+s+s2}{Apple beta from linear regression: }\PYG{l+s+si}{\PYGZob{}}\PYG{n}{fit}\PYG{o}{.}\PYG{n}{params}\PYG{p}{[}\PYG{l+s+s1}{\PYGZsq{}}\PYG{l+s+s1}{MktRF}\PYG{l+s+s1}{\PYGZsq{}}\PYG{p}{]}\PYG{l+s+si}{:}\PYG{l+s+s2}{0.4f}\PYG{l+s+si}{\PYGZcb{}}\PYG{l+s+s2}{\PYGZdq{}}\PYG{p}{)}
\end{sphinxVerbatim}

\end{sphinxuseclass}\end{sphinxVerbatimInput}
\begin{sphinxVerbatimOutput}

\begin{sphinxuseclass}{cell_output}
\begin{sphinxVerbatim}[commandchars=\\\{\}]
Apple beta from linear regression: 1.2625
\end{sphinxVerbatim}

\end{sphinxuseclass}\end{sphinxVerbatimOutput}

\end{sphinxuseclass}
\sphinxAtStartPar
We can chain these operations, but it often makes sense to save the intermediate results (i.e., \sphinxcode{\sphinxupquote{model}} and \sphinxcode{\sphinxupquote{fit}}).

\begin{sphinxuseclass}{cell}\begin{sphinxVerbatimInput}

\begin{sphinxuseclass}{cell_input}
\begin{sphinxVerbatim}[commandchars=\\\{\}]
\PYG{n}{smf}\PYG{o}{.}\PYG{n}{ols}\PYG{p}{(}\PYG{l+s+s1}{\PYGZsq{}}\PYG{l+s+s1}{RiRF \PYGZti{} MktRF}\PYG{l+s+s1}{\PYGZsq{}}\PYG{p}{,} \PYG{n}{aapl}\PYG{p}{)}\PYG{o}{.}\PYG{n}{fit}\PYG{p}{(}\PYG{p}{)}\PYG{o}{.}\PYG{n}{summary}\PYG{p}{(}\PYG{p}{)}
\end{sphinxVerbatim}

\end{sphinxuseclass}\end{sphinxVerbatimInput}
\begin{sphinxVerbatimOutput}

\begin{sphinxuseclass}{cell_output}
\begin{sphinxVerbatim}[commandchars=\\\{\}]
\PYGZlt{}class \PYGZsq{}statsmodels.iolib.summary.Summary\PYGZsq{}\PYGZgt{}
\PYGZdq{}\PYGZdq{}\PYGZdq{}
                            OLS Regression Results                            
==============================================================================
Dep. Variable:                   RiRF   R\PYGZhy{}squared:                       0.248
Model:                            OLS   Adj. R\PYGZhy{}squared:                  0.248
Method:                 Least Squares   F\PYGZhy{}statistic:                     3493.
Date:                Wed, 15 Mar 2023   Prob (F\PYGZhy{}statistic):               0.00
Time:                        11:22:03   Log\PYGZhy{}Likelihood:                \PYGZhy{}24556.
No. Observations:               10602   AIC:                         4.912e+04
Df Residuals:                   10600   BIC:                         4.913e+04
Df Model:                           1                                         
Covariance Type:            nonrobust                                         
==============================================================================
                 coef    std err          t      P\PYGZgt{}|t|      [0.025      0.975]
\PYGZhy{}\PYGZhy{}\PYGZhy{}\PYGZhy{}\PYGZhy{}\PYGZhy{}\PYGZhy{}\PYGZhy{}\PYGZhy{}\PYGZhy{}\PYGZhy{}\PYGZhy{}\PYGZhy{}\PYGZhy{}\PYGZhy{}\PYGZhy{}\PYGZhy{}\PYGZhy{}\PYGZhy{}\PYGZhy{}\PYGZhy{}\PYGZhy{}\PYGZhy{}\PYGZhy{}\PYGZhy{}\PYGZhy{}\PYGZhy{}\PYGZhy{}\PYGZhy{}\PYGZhy{}\PYGZhy{}\PYGZhy{}\PYGZhy{}\PYGZhy{}\PYGZhy{}\PYGZhy{}\PYGZhy{}\PYGZhy{}\PYGZhy{}\PYGZhy{}\PYGZhy{}\PYGZhy{}\PYGZhy{}\PYGZhy{}\PYGZhy{}\PYGZhy{}\PYGZhy{}\PYGZhy{}\PYGZhy{}\PYGZhy{}\PYGZhy{}\PYGZhy{}\PYGZhy{}\PYGZhy{}\PYGZhy{}\PYGZhy{}\PYGZhy{}\PYGZhy{}\PYGZhy{}\PYGZhy{}\PYGZhy{}\PYGZhy{}\PYGZhy{}\PYGZhy{}\PYGZhy{}\PYGZhy{}\PYGZhy{}\PYGZhy{}\PYGZhy{}\PYGZhy{}\PYGZhy{}\PYGZhy{}\PYGZhy{}\PYGZhy{}\PYGZhy{}\PYGZhy{}\PYGZhy{}\PYGZhy{}
Intercept      0.0520      0.024      2.180      0.029       0.005       0.099
MktRF          1.2625      0.021     59.105      0.000       1.221       1.304
==============================================================================
Omnibus:                     3126.527   Durbin\PYGZhy{}Watson:                   1.917
Prob(Omnibus):                  0.000   Jarque\PYGZhy{}Bera (JB):           313356.303
Skew:                          \PYGZhy{}0.376   Prob(JB):                         0.00
Kurtosis:                      29.623   Cond. No.                         1.12
==============================================================================

Notes:
[1] Standard Errors assume that the covariance matrix of the errors is correctly specified.
\PYGZdq{}\PYGZdq{}\PYGZdq{}
\end{sphinxVerbatim}

\end{sphinxuseclass}\end{sphinxVerbatimOutput}

\end{sphinxuseclass}

\subsection{\$\textbackslash{}beta\$ Visualization}
\label{\detokenize{herron_03_lecture:beta-visualization}}
\sphinxAtStartPar
We can visualize Apple’s \$\textbackslash{}beta\$, using seaborn’s \sphinxcode{\sphinxupquote{regplot()}} to add a best\sphinxhyphen{}fit line.

\begin{sphinxuseclass}{cell}\begin{sphinxVerbatimInput}

\begin{sphinxuseclass}{cell_input}
\begin{sphinxVerbatim}[commandchars=\\\{\}]
\PYG{k+kn}{import} \PYG{n+nn}{seaborn} \PYG{k}{as} \PYG{n+nn}{sns}
\end{sphinxVerbatim}

\end{sphinxuseclass}\end{sphinxVerbatimInput}

\end{sphinxuseclass}
\sphinxAtStartPar
We can write a couple of function to more easily make prettier plots.

\begin{sphinxuseclass}{cell}\begin{sphinxVerbatimInput}

\begin{sphinxuseclass}{cell_input}
\begin{sphinxVerbatim}[commandchars=\\\{\}]
\PYG{k}{def} \PYG{n+nf}{label\PYGZus{}beta}\PYG{p}{(}\PYG{n}{x}\PYG{p}{)}\PYG{p}{:}
    \PYG{n}{vcv} \PYG{o}{=} \PYG{n}{x}\PYG{o}{.}\PYG{n}{dropna}\PYG{p}{(}\PYG{p}{)}\PYG{o}{.}\PYG{n}{cov}\PYG{p}{(}\PYG{p}{)}
    \PYG{n}{beta} \PYG{o}{=} \PYG{n}{vcv}\PYG{o}{.}\PYG{n}{loc}\PYG{p}{[}\PYG{l+s+s1}{\PYGZsq{}}\PYG{l+s+s1}{RiRF}\PYG{l+s+s1}{\PYGZsq{}}\PYG{p}{,} \PYG{l+s+s1}{\PYGZsq{}}\PYG{l+s+s1}{MktRF}\PYG{l+s+s1}{\PYGZsq{}}\PYG{p}{]} \PYG{o}{/} \PYG{n}{vcv}\PYG{o}{.}\PYG{n}{loc}\PYG{p}{[}\PYG{l+s+s1}{\PYGZsq{}}\PYG{l+s+s1}{MktRF}\PYG{l+s+s1}{\PYGZsq{}}\PYG{p}{,} \PYG{l+s+s1}{\PYGZsq{}}\PYG{l+s+s1}{MktRF}\PYG{l+s+s1}{\PYGZsq{}}\PYG{p}{]}
    \PYG{k}{return} \PYG{l+s+sa}{r}\PYG{l+s+s1}{\PYGZsq{}}\PYG{l+s+s1}{\PYGZdl{}}\PYG{l+s+s1}{\PYGZbs{}}\PYG{l+s+s1}{beta=\PYGZdl{}}\PYG{l+s+s1}{\PYGZsq{}} \PYG{o}{+} \PYG{l+s+sa}{f}\PYG{l+s+s1}{\PYGZsq{}}\PYG{l+s+si}{\PYGZob{}}\PYG{n}{beta}\PYG{l+s+si}{:}\PYG{l+s+s1}{ 0.4f}\PYG{l+s+si}{\PYGZcb{}}\PYG{l+s+s1}{\PYGZsq{}}
\end{sphinxVerbatim}

\end{sphinxuseclass}\end{sphinxVerbatimInput}

\end{sphinxuseclass}
\begin{sphinxuseclass}{cell}\begin{sphinxVerbatimInput}

\begin{sphinxuseclass}{cell_input}
\begin{sphinxVerbatim}[commandchars=\\\{\}]
\PYG{k}{def} \PYG{n+nf}{label\PYGZus{}dates}\PYG{p}{(}\PYG{n}{x}\PYG{p}{)}\PYG{p}{:}
    \PYG{n}{y} \PYG{o}{=} \PYG{n}{x}\PYG{o}{.}\PYG{n}{dropna}\PYG{p}{(}\PYG{p}{)}
    \PYG{k}{return} \PYG{l+s+sa}{f}\PYG{l+s+s1}{\PYGZsq{}}\PYG{l+s+s1}{from }\PYG{l+s+si}{\PYGZob{}}\PYG{n}{y}\PYG{o}{.}\PYG{n}{index}\PYG{p}{[}\PYG{l+m+mi}{0}\PYG{p}{]}\PYG{l+s+si}{:}\PYG{l+s+s1}{\PYGZpc{}b \PYGZpc{}d, \PYGZpc{}Y}\PYG{l+s+si}{\PYGZcb{}}\PYG{l+s+s1}{ to }\PYG{l+s+si}{\PYGZob{}}\PYG{n}{y}\PYG{o}{.}\PYG{n}{index}\PYG{p}{[}\PYG{o}{\PYGZhy{}}\PYG{l+m+mi}{1}\PYG{p}{]}\PYG{l+s+si}{:}\PYG{l+s+s1}{\PYGZpc{}b \PYGZpc{}d, \PYGZpc{}Y}\PYG{l+s+si}{\PYGZcb{}}\PYG{l+s+s1}{\PYGZsq{}}
\end{sphinxVerbatim}

\end{sphinxuseclass}\end{sphinxVerbatimInput}

\end{sphinxuseclass}
\begin{sphinxuseclass}{cell}\begin{sphinxVerbatimInput}

\begin{sphinxuseclass}{cell_input}
\begin{sphinxVerbatim}[commandchars=\\\{\}]
\PYG{n}{\PYGZus{}} \PYG{o}{=} \PYG{n}{aapl}\PYG{p}{[}\PYG{p}{[}\PYG{l+s+s1}{\PYGZsq{}}\PYG{l+s+s1}{MktRF}\PYG{l+s+s1}{\PYGZsq{}}\PYG{p}{,} \PYG{l+s+s1}{\PYGZsq{}}\PYG{l+s+s1}{RiRF}\PYG{l+s+s1}{\PYGZsq{}}\PYG{p}{]}\PYG{p}{]}\PYG{o}{.}\PYG{n}{dropna}\PYG{p}{(}\PYG{p}{)}

\PYG{n}{sns}\PYG{o}{.}\PYG{n}{regplot}\PYG{p}{(}
    \PYG{n}{x}\PYG{o}{=}\PYG{l+s+s1}{\PYGZsq{}}\PYG{l+s+s1}{MktRF}\PYG{l+s+s1}{\PYGZsq{}}\PYG{p}{,}
    \PYG{n}{y}\PYG{o}{=}\PYG{l+s+s1}{\PYGZsq{}}\PYG{l+s+s1}{RiRF}\PYG{l+s+s1}{\PYGZsq{}}\PYG{p}{,}
    \PYG{n}{data}\PYG{o}{=}\PYG{n}{\PYGZus{}}\PYG{p}{,}
    \PYG{n}{scatter\PYGZus{}kws}\PYG{o}{=}\PYG{p}{\PYGZob{}}\PYG{l+s+s1}{\PYGZsq{}}\PYG{l+s+s1}{alpha}\PYG{l+s+s1}{\PYGZsq{}}\PYG{p}{:} \PYG{l+m+mf}{0.1}\PYG{p}{\PYGZcb{}}\PYG{p}{,}
    \PYG{n}{line\PYGZus{}kws}\PYG{o}{=}\PYG{p}{\PYGZob{}}\PYG{l+s+s1}{\PYGZsq{}}\PYG{l+s+s1}{label}\PYG{l+s+s1}{\PYGZsq{}}\PYG{p}{:} \PYG{n}{\PYGZus{}}\PYG{o}{.}\PYG{n}{pipe}\PYG{p}{(}\PYG{n}{label\PYGZus{}beta}\PYG{p}{)}\PYG{p}{\PYGZcb{}}
\PYG{p}{)}
\PYG{n}{plt}\PYG{o}{.}\PYG{n}{legend}\PYG{p}{(}\PYG{p}{)}
\PYG{n}{plt}\PYG{o}{.}\PYG{n}{xlabel}\PYG{p}{(}\PYG{l+s+s1}{\PYGZsq{}}\PYG{l+s+s1}{Market Excess Return (}\PYG{l+s+s1}{\PYGZpc{}}\PYG{l+s+s1}{)}\PYG{l+s+s1}{\PYGZsq{}}\PYG{p}{)}
\PYG{n}{plt}\PYG{o}{.}\PYG{n}{ylabel}\PYG{p}{(}\PYG{l+s+s1}{\PYGZsq{}}\PYG{l+s+s1}{Stock Excess Return (}\PYG{l+s+s1}{\PYGZpc{}}\PYG{l+s+s1}{)}\PYG{l+s+s1}{\PYGZsq{}}\PYG{p}{)}
\PYG{n}{plt}\PYG{o}{.}\PYG{n}{title}\PYG{p}{(}\PYG{l+s+sa}{r}\PYG{l+s+s1}{\PYGZsq{}}\PYG{l+s+s1}{\PYGZdl{}}\PYG{l+s+s1}{\PYGZbs{}}\PYG{l+s+s1}{beta\PYGZdl{} Plot with Daily Returns for Apple}\PYG{l+s+s1}{\PYGZsq{}} \PYG{o}{+} \PYG{l+s+s1}{\PYGZsq{}}\PYG{l+s+se}{\PYGZbs{}n}\PYG{l+s+s1}{\PYGZsq{}} \PYG{o}{+} \PYG{n}{\PYGZus{}}\PYG{o}{.}\PYG{n}{pipe}\PYG{p}{(}\PYG{n}{label\PYGZus{}dates}\PYG{p}{)}\PYG{p}{)}
\PYG{n}{plt}\PYG{o}{.}\PYG{n}{show}\PYG{p}{(}\PYG{p}{)}
\end{sphinxVerbatim}

\end{sphinxuseclass}\end{sphinxVerbatimInput}
\begin{sphinxVerbatimOutput}

\begin{sphinxuseclass}{cell_output}
\noindent\sphinxincludegraphics{{70f54012e0434ce15f5a2f9de33f71bd8037e968d7632c4980609b77d5aeda32}.png}

\end{sphinxuseclass}\end{sphinxVerbatimOutput}

\end{sphinxuseclass}
\sphinxAtStartPar
We see a strong relation between Apple and market (excess) returns, but there is a lot of unexplained variation in Apple (excess) returns.
The best practice is to estimate \$\textbackslash{}beta\$ with one to three years of daily data.

\begin{sphinxuseclass}{cell}\begin{sphinxVerbatimInput}

\begin{sphinxuseclass}{cell_input}
\begin{sphinxVerbatim}[commandchars=\\\{\}]
\PYG{n}{\PYGZus{}} \PYG{o}{=} \PYG{n}{aapl}\PYG{p}{[}\PYG{p}{[}\PYG{l+s+s1}{\PYGZsq{}}\PYG{l+s+s1}{MktRF}\PYG{l+s+s1}{\PYGZsq{}}\PYG{p}{,} \PYG{l+s+s1}{\PYGZsq{}}\PYG{l+s+s1}{RiRF}\PYG{l+s+s1}{\PYGZsq{}}\PYG{p}{]}\PYG{p}{]}\PYG{o}{.}\PYG{n}{dropna}\PYG{p}{(}\PYG{p}{)}\PYG{o}{.}\PYG{n}{iloc}\PYG{p}{[}\PYG{o}{\PYGZhy{}}\PYG{l+m+mi}{504}\PYG{p}{:}\PYG{p}{]}

\PYG{n}{sns}\PYG{o}{.}\PYG{n}{regplot}\PYG{p}{(}
    \PYG{n}{x}\PYG{o}{=}\PYG{l+s+s1}{\PYGZsq{}}\PYG{l+s+s1}{MktRF}\PYG{l+s+s1}{\PYGZsq{}}\PYG{p}{,}
    \PYG{n}{y}\PYG{o}{=}\PYG{l+s+s1}{\PYGZsq{}}\PYG{l+s+s1}{RiRF}\PYG{l+s+s1}{\PYGZsq{}}\PYG{p}{,}
    \PYG{n}{data}\PYG{o}{=}\PYG{n}{\PYGZus{}}\PYG{p}{,}
    \PYG{n}{scatter\PYGZus{}kws}\PYG{o}{=}\PYG{p}{\PYGZob{}}\PYG{l+s+s1}{\PYGZsq{}}\PYG{l+s+s1}{alpha}\PYG{l+s+s1}{\PYGZsq{}}\PYG{p}{:} \PYG{l+m+mf}{0.1}\PYG{p}{\PYGZcb{}}\PYG{p}{,}
    \PYG{n}{line\PYGZus{}kws}\PYG{o}{=}\PYG{p}{\PYGZob{}}\PYG{l+s+s1}{\PYGZsq{}}\PYG{l+s+s1}{label}\PYG{l+s+s1}{\PYGZsq{}}\PYG{p}{:} \PYG{n}{\PYGZus{}}\PYG{o}{.}\PYG{n}{pipe}\PYG{p}{(}\PYG{n}{label\PYGZus{}beta}\PYG{p}{)}\PYG{p}{\PYGZcb{}}
\PYG{p}{)}
\PYG{n}{plt}\PYG{o}{.}\PYG{n}{legend}\PYG{p}{(}\PYG{p}{)}
\PYG{n}{plt}\PYG{o}{.}\PYG{n}{xlabel}\PYG{p}{(}\PYG{l+s+s1}{\PYGZsq{}}\PYG{l+s+s1}{Market Excess Return (}\PYG{l+s+s1}{\PYGZpc{}}\PYG{l+s+s1}{)}\PYG{l+s+s1}{\PYGZsq{}}\PYG{p}{)}
\PYG{n}{plt}\PYG{o}{.}\PYG{n}{ylabel}\PYG{p}{(}\PYG{l+s+s1}{\PYGZsq{}}\PYG{l+s+s1}{Stock Excess Return (}\PYG{l+s+s1}{\PYGZpc{}}\PYG{l+s+s1}{)}\PYG{l+s+s1}{\PYGZsq{}}\PYG{p}{)}
\PYG{n}{plt}\PYG{o}{.}\PYG{n}{title}\PYG{p}{(}\PYG{l+s+sa}{r}\PYG{l+s+s1}{\PYGZsq{}}\PYG{l+s+s1}{\PYGZdl{}}\PYG{l+s+s1}{\PYGZbs{}}\PYG{l+s+s1}{beta\PYGZdl{} Plot with Daily Returns for Apple}\PYG{l+s+s1}{\PYGZsq{}} \PYG{o}{+} \PYG{l+s+s1}{\PYGZsq{}}\PYG{l+s+se}{\PYGZbs{}n}\PYG{l+s+s1}{\PYGZsq{}} \PYG{o}{+} \PYG{n}{\PYGZus{}}\PYG{o}{.}\PYG{n}{pipe}\PYG{p}{(}\PYG{n}{label\PYGZus{}dates}\PYG{p}{)}\PYG{p}{)}
\PYG{n}{plt}\PYG{o}{.}\PYG{n}{show}\PYG{p}{(}\PYG{p}{)}
\end{sphinxVerbatim}

\end{sphinxuseclass}\end{sphinxVerbatimInput}
\begin{sphinxVerbatimOutput}

\begin{sphinxuseclass}{cell_output}
\noindent\sphinxincludegraphics{{88fd45dd288580e7e60f902f88cd4886ff3a0a5248c45b61a95baab10b7672da}.png}

\end{sphinxuseclass}\end{sphinxVerbatimOutput}

\end{sphinxuseclass}

\subsection{The Security Market Line (SML)}
\label{\detokenize{herron_03_lecture:the-security-market-line-sml}}
\sphinxAtStartPar
The SML is a visualization of the CAPM.
We can think of \$E(R\_i) = R\_F +  \textbackslash{}beta\_i {[}E(R\_M) \sphinxhyphen{} R\_F{]}\$ as \$y = b + mx\$, where:
\begin{enumerate}
\sphinxsetlistlabels{\arabic}{enumi}{enumii}{}{.}%
\item {} 
\sphinxAtStartPar
The equity premium \$E(R\_M) \sphinxhyphen{} R\_F\$ is the slope \$m\$ of the SML, and

\item {} 
\sphinxAtStartPar
The risk\sphinxhyphen{}free rate of return \$R\_F\$ is its intercept \$b\$.

\end{enumerate}

\sphinxAtStartPar
We will explore the SML more in the practice notebook.


\subsection{How well does the CAPM work?}
\label{\detokenize{herron_03_lecture:how-well-does-the-capm-work}}
\sphinxAtStartPar
The CAPM \sphinxstyleemphasis{appears} to work well as a single\sphinxhyphen{}period model.
We can see this with portfolios formed on \$\textbackslash{}beta\$ from Ken French.

\begin{sphinxuseclass}{cell}\begin{sphinxVerbatimInput}

\begin{sphinxuseclass}{cell_input}
\begin{sphinxVerbatim}[commandchars=\\\{\}]
\PYG{n}{ff\PYGZus{}beta} \PYG{o}{=} \PYG{n}{pdr}\PYG{o}{.}\PYG{n}{DataReader}\PYG{p}{(}
    \PYG{n}{name}\PYG{o}{=}\PYG{l+s+s1}{\PYGZsq{}}\PYG{l+s+s1}{Portfolios\PYGZus{}Formed\PYGZus{}on\PYGZus{}BETA}\PYG{l+s+s1}{\PYGZsq{}}\PYG{p}{,}
    \PYG{n}{data\PYGZus{}source}\PYG{o}{=}\PYG{l+s+s1}{\PYGZsq{}}\PYG{l+s+s1}{famafrench}\PYG{l+s+s1}{\PYGZsq{}}\PYG{p}{,}
    \PYG{n}{start}\PYG{o}{=}\PYG{l+s+s1}{\PYGZsq{}}\PYG{l+s+s1}{1900}\PYG{l+s+s1}{\PYGZsq{}}\PYG{p}{,}
    \PYG{n}{session}\PYG{o}{=}\PYG{n}{session}
\PYG{p}{)}
\end{sphinxVerbatim}

\end{sphinxuseclass}\end{sphinxVerbatimInput}

\end{sphinxuseclass}
\begin{sphinxuseclass}{cell}\begin{sphinxVerbatimInput}

\begin{sphinxuseclass}{cell_input}
\begin{sphinxVerbatim}[commandchars=\\\{\}]
\PYG{n+nb}{print}\PYG{p}{(}\PYG{n}{ff\PYGZus{}beta}\PYG{p}{[}\PYG{l+s+s1}{\PYGZsq{}}\PYG{l+s+s1}{DESCR}\PYG{l+s+s1}{\PYGZsq{}}\PYG{p}{]}\PYG{p}{)}
\end{sphinxVerbatim}

\end{sphinxuseclass}\end{sphinxVerbatimInput}
\begin{sphinxVerbatimOutput}

\begin{sphinxuseclass}{cell_output}
\begin{sphinxVerbatim}[commandchars=\\\{\}]
Portfolios Formed on BETA
\PYGZhy{}\PYGZhy{}\PYGZhy{}\PYGZhy{}\PYGZhy{}\PYGZhy{}\PYGZhy{}\PYGZhy{}\PYGZhy{}\PYGZhy{}\PYGZhy{}\PYGZhy{}\PYGZhy{}\PYGZhy{}\PYGZhy{}\PYGZhy{}\PYGZhy{}\PYGZhy{}\PYGZhy{}\PYGZhy{}\PYGZhy{}\PYGZhy{}\PYGZhy{}\PYGZhy{}\PYGZhy{}

This file was created by CMPT\PYGZus{}BETA\PYGZus{}RETS using the 202301 CRSP database. It contains value\PYGZhy{} and equal\PYGZhy{}weighted returns for portfolios formed on BETA. The portfolios are constructed at the end of June. Beta is estimated using monthly returns for the past 60 months (requiring at least 24 months with non\PYGZhy{}missing returns). Beta is estimated using the Scholes\PYGZhy{}Williams method. Annual returns are from January to December. Missing data are indicated by \PYGZhy{}99.99 or \PYGZhy{}999. The break points include utilities and include financials. The portfolios include utilities and include financials. Copyright 2023 Kenneth R. French

  0 : Value Weighted Returns \PYGZhy{}\PYGZhy{} Monthly (715 rows x 15 cols)
  1 : Equal Weighted Returns \PYGZhy{}\PYGZhy{} Monthly (715 rows x 15 cols)
  2 : Value Weighted Returns \PYGZhy{}\PYGZhy{} Annual from January to December (59 rows x 15 cols)
  3 : Equal Weighted Returns \PYGZhy{}\PYGZhy{} Annual from January to December (59 rows x 15 cols)
  4 : Number of Firms in Portfolios (715 rows x 15 cols)
  5 : Average Firm Size (715 rows x 15 cols)
  6 : Value\PYGZhy{}Weighted Average of Prior Beta (60 rows x 15 cols)
\end{sphinxVerbatim}

\end{sphinxuseclass}\end{sphinxVerbatimOutput}

\end{sphinxuseclass}
\sphinxAtStartPar
This file contains seven data frames.
We want the data frame at \sphinxcode{\sphinxupquote{{[}2{]}}}, which contains the annual returns on two sets of portfolios formed on the previous year’s \$\textbackslash{}beta\$s.

\begin{sphinxuseclass}{cell}\begin{sphinxVerbatimInput}

\begin{sphinxuseclass}{cell_input}
\begin{sphinxVerbatim}[commandchars=\\\{\}]
\PYG{n}{ff\PYGZus{}beta}\PYG{p}{[}\PYG{l+m+mi}{2}\PYG{p}{]}\PYG{o}{.}\PYG{n}{head}\PYG{p}{(}\PYG{p}{)}
\end{sphinxVerbatim}

\end{sphinxuseclass}\end{sphinxVerbatimInput}
\begin{sphinxVerbatimOutput}

\begin{sphinxuseclass}{cell_output}
\begin{sphinxVerbatim}[commandchars=\\\{\}]
       Lo 20    Qnt 2   Qnt 3   Qnt 4   Hi 20   Lo 10   Dec 2    Dec 3  \PYGZbs{}
Date                                                                     
1964 16.8700  19.7500 17.6600  8.7300 12.5500 24.7000 13.5900  20.2700   
1965  8.7200   6.9800 15.4000 24.8800 49.6700 12.4400  6.7600  10.0200   
1966 \PYGZhy{}9.2500 \PYGZhy{}12.1700 \PYGZhy{}9.1000 \PYGZhy{}2.7300 \PYGZhy{}2.2000 \PYGZhy{}9.4300 \PYGZhy{}9.4400 \PYGZhy{}12.5700   
1967 13.4300  22.0600 31.9500 42.7900 51.1500  8.9200 18.6400  22.7900   
1968 15.8100   9.1200 13.6400 14.4300 24.2400 18.4800 12.8400  14.5100   

        Dec 4   Dec 5    Dec 6   Dec 7   Dec 8   Dec 9   Hi 10  
Date                                                            
1964  19.0100 19.7900  15.6800  9.7900  8.2300 13.6000 11.6600  
1965   5.2500 13.7100  17.1100 20.7000 28.2800 42.0700 57.7300  
1966 \PYGZhy{}12.1600 \PYGZhy{}7.5000 \PYGZhy{}10.5600 \PYGZhy{}6.2200  0.1800 \PYGZhy{}3.3100 \PYGZhy{}0.9400  
1967  21.6500 31.1100  34.1600 40.5700 44.4000 41.6100 59.7800  
1968   5.8400 12.6600  14.7500 19.0800 10.5400 23.2500 25.2900  
\end{sphinxVerbatim}

\end{sphinxuseclass}\end{sphinxVerbatimOutput}

\end{sphinxuseclass}
\sphinxAtStartPar
We can plot the mean annual return on each of the five portfolios in the first set of portflios.
We do not need to annualize these numbers because they are the means of annual returns.

\begin{sphinxuseclass}{cell}\begin{sphinxVerbatimInput}

\begin{sphinxuseclass}{cell_input}
\begin{sphinxVerbatim}[commandchars=\\\{\}]
\PYG{n}{\PYGZus{}} \PYG{o}{=} \PYG{n}{ff\PYGZus{}beta}\PYG{p}{[}\PYG{l+m+mi}{2}\PYG{p}{]}\PYG{o}{.}\PYG{n}{iloc}\PYG{p}{[}\PYG{p}{:}\PYG{p}{,} \PYG{p}{:}\PYG{l+m+mi}{5}\PYG{p}{]}
\PYG{n}{\PYGZus{}}\PYG{o}{.}\PYG{n}{mean}\PYG{p}{(}\PYG{p}{)}\PYG{o}{.}\PYG{n}{plot}\PYG{p}{(}\PYG{n}{kind}\PYG{o}{=}\PYG{l+s+s1}{\PYGZsq{}}\PYG{l+s+s1}{bar}\PYG{l+s+s1}{\PYGZsq{}}\PYG{p}{)}
\PYG{n}{plt}\PYG{o}{.}\PYG{n}{ylabel}\PYG{p}{(}\PYG{l+s+s1}{\PYGZsq{}}\PYG{l+s+s1}{Mean Annual Return (}\PYG{l+s+s1}{\PYGZpc{}}\PYG{l+s+s1}{)}\PYG{l+s+s1}{\PYGZsq{}}\PYG{p}{)}
\PYG{n}{plt}\PYG{o}{.}\PYG{n}{xlabel}\PYG{p}{(}\PYG{l+s+s1}{\PYGZsq{}}\PYG{l+s+s1}{Portfolio}\PYG{l+s+s1}{\PYGZsq{}}\PYG{p}{)}
\PYG{n}{plt}\PYG{o}{.}\PYG{n}{xticks}\PYG{p}{(}\PYG{n}{rotation}\PYG{o}{=}\PYG{l+m+mi}{0}\PYG{p}{)}
\PYG{n}{plt}\PYG{o}{.}\PYG{n}{title}\PYG{p}{(}\PYG{l+s+sa}{r}\PYG{l+s+s1}{\PYGZsq{}}\PYG{l+s+s1}{Mean Returns on Portfolios Formed on \PYGZdl{}}\PYG{l+s+s1}{\PYGZbs{}}\PYG{l+s+s1}{beta\PYGZdl{}}\PYG{l+s+s1}{\PYGZsq{}} \PYG{o}{+} \PYG{l+s+s1}{\PYGZsq{}}\PYG{l+s+se}{\PYGZbs{}n}\PYG{l+s+s1}{\PYGZsq{}} \PYG{o}{+} \PYG{l+s+sa}{f}\PYG{l+s+s1}{\PYGZsq{}}\PYG{l+s+s1}{from }\PYG{l+s+si}{\PYGZob{}}\PYG{n}{\PYGZus{}}\PYG{o}{.}\PYG{n}{index}\PYG{p}{[}\PYG{l+m+mi}{0}\PYG{p}{]}\PYG{l+s+si}{\PYGZcb{}}\PYG{l+s+s1}{ to }\PYG{l+s+si}{\PYGZob{}}\PYG{n}{\PYGZus{}}\PYG{o}{.}\PYG{n}{index}\PYG{p}{[}\PYG{o}{\PYGZhy{}}\PYG{l+m+mi}{1}\PYG{p}{]}\PYG{l+s+si}{\PYGZcb{}}\PYG{l+s+s1}{\PYGZsq{}}\PYG{p}{)}
\PYG{n}{plt}\PYG{o}{.}\PYG{n}{show}\PYG{p}{(}\PYG{p}{)}
\end{sphinxVerbatim}

\end{sphinxuseclass}\end{sphinxVerbatimInput}
\begin{sphinxVerbatimOutput}

\begin{sphinxuseclass}{cell_output}
\noindent\sphinxincludegraphics{{a4dfd01df876026fdb8bd64889b4b3fc7115a9e32b9d9dc60827524aac22a6d3}.png}

\end{sphinxuseclass}\end{sphinxVerbatimOutput}

\end{sphinxuseclass}
\sphinxAtStartPar
We can think of the plot above as a binned plot of the SML.
The x axis above is an ordinal measure of \$\textbackslash{}beta\$, and the y axis above is the mean return.
Recall the slope of the SML is the market risk premium.
If the market risk premium is too low, then high \$\textbackslash{}beta\$ stocks do not have high enough returns.
We can see this failure of the CAPM by plotting long\sphinxhyphen{}term or cumulative returns on these five portfolios.

\begin{sphinxuseclass}{cell}\begin{sphinxVerbatimInput}

\begin{sphinxuseclass}{cell_input}
\begin{sphinxVerbatim}[commandchars=\\\{\}]
\PYG{n}{\PYGZus{}} \PYG{o}{=} \PYG{n}{ff\PYGZus{}beta}\PYG{p}{[}\PYG{l+m+mi}{2}\PYG{p}{]}\PYG{o}{.}\PYG{n}{iloc}\PYG{p}{[}\PYG{p}{:}\PYG{p}{,} \PYG{p}{:}\PYG{l+m+mi}{5}\PYG{p}{]}
\PYG{n}{\PYGZus{}}\PYG{o}{.}\PYG{n}{div}\PYG{p}{(}\PYG{l+m+mi}{100}\PYG{p}{)}\PYG{o}{.}\PYG{n}{add}\PYG{p}{(}\PYG{l+m+mi}{1}\PYG{p}{)}\PYG{o}{.}\PYG{n}{cumprod}\PYG{p}{(}\PYG{p}{)}\PYG{o}{.}\PYG{n}{plot}\PYG{p}{(}\PYG{p}{)}
\PYG{n}{plt}\PYG{o}{.}\PYG{n}{semilogy}\PYG{p}{(}\PYG{p}{)}
\PYG{n}{plt}\PYG{o}{.}\PYG{n}{ylabel}\PYG{p}{(}\PYG{l+s+s1}{\PYGZsq{}}\PYG{l+s+s1}{Value of }\PYG{l+s+s1}{\PYGZbs{}}\PYG{l+s+s1}{\PYGZdl{}1 Investment (}\PYG{l+s+s1}{\PYGZbs{}}\PYG{l+s+s1}{\PYGZdl{})}\PYG{l+s+s1}{\PYGZsq{}}\PYG{p}{)}
\PYG{n}{plt}\PYG{o}{.}\PYG{n}{title}\PYG{p}{(}\PYG{l+s+sa}{r}\PYG{l+s+s1}{\PYGZsq{}}\PYG{l+s+s1}{Value of }\PYG{l+s+s1}{\PYGZbs{}}\PYG{l+s+s1}{\PYGZdl{}1 Investments in Portfolios Formed on \PYGZdl{}}\PYG{l+s+s1}{\PYGZbs{}}\PYG{l+s+s1}{beta\PYGZdl{}}\PYG{l+s+s1}{\PYGZsq{}} \PYG{o}{+} \PYG{l+s+s1}{\PYGZsq{}}\PYG{l+s+se}{\PYGZbs{}n}\PYG{l+s+s1}{\PYGZsq{}} \PYG{o}{+} \PYG{l+s+sa}{f}\PYG{l+s+s1}{\PYGZsq{}}\PYG{l+s+s1}{from }\PYG{l+s+si}{\PYGZob{}}\PYG{n}{\PYGZus{}}\PYG{o}{.}\PYG{n}{index}\PYG{p}{[}\PYG{l+m+mi}{0}\PYG{p}{]}\PYG{l+s+si}{\PYGZcb{}}\PYG{l+s+s1}{ to }\PYG{l+s+si}{\PYGZob{}}\PYG{n}{\PYGZus{}}\PYG{o}{.}\PYG{n}{index}\PYG{p}{[}\PYG{o}{\PYGZhy{}}\PYG{l+m+mi}{1}\PYG{p}{]}\PYG{l+s+si}{\PYGZcb{}}\PYG{l+s+s1}{\PYGZsq{}}\PYG{p}{)}
\PYG{n}{plt}\PYG{o}{.}\PYG{n}{show}\PYG{p}{(}\PYG{p}{)}
\end{sphinxVerbatim}

\end{sphinxuseclass}\end{sphinxVerbatimInput}
\begin{sphinxVerbatimOutput}

\begin{sphinxuseclass}{cell_output}
\noindent\sphinxincludegraphics{{6743a28600fcf5b73b79da81775497415ced999087ed3744371c9492089e4799}.png}

\end{sphinxuseclass}\end{sphinxVerbatimOutput}

\end{sphinxuseclass}
\sphinxAtStartPar
In the plot above, the highest\sphinxhyphen{}\$\textbackslash{}beta\$ portfolio has the lowest cumulative returns!
The log scale masks a lot of variation, too!

\begin{sphinxuseclass}{cell}\begin{sphinxVerbatimInput}

\begin{sphinxuseclass}{cell_input}
\begin{sphinxVerbatim}[commandchars=\\\{\}]
\PYG{n}{\PYGZus{}} \PYG{o}{=} \PYG{n}{ff\PYGZus{}beta}\PYG{p}{[}\PYG{l+m+mi}{2}\PYG{p}{]}\PYG{o}{.}\PYG{n}{iloc}\PYG{p}{[}\PYG{p}{:}\PYG{p}{,} \PYG{p}{:}\PYG{l+m+mi}{5}\PYG{p}{]}
\PYG{n}{\PYGZus{}}\PYG{o}{.}\PYG{n}{div}\PYG{p}{(}\PYG{l+m+mi}{100}\PYG{p}{)}\PYG{o}{.}\PYG{n}{add}\PYG{p}{(}\PYG{l+m+mi}{1}\PYG{p}{)}\PYG{o}{.}\PYG{n}{prod}\PYG{p}{(}\PYG{p}{)}\PYG{o}{.}\PYG{n}{plot}\PYG{p}{(}\PYG{n}{kind}\PYG{o}{=}\PYG{l+s+s1}{\PYGZsq{}}\PYG{l+s+s1}{bar}\PYG{l+s+s1}{\PYGZsq{}}\PYG{p}{)}
\PYG{n}{plt}\PYG{o}{.}\PYG{n}{ylabel}\PYG{p}{(}\PYG{l+s+s1}{\PYGZsq{}}\PYG{l+s+s1}{Value of }\PYG{l+s+s1}{\PYGZbs{}}\PYG{l+s+s1}{\PYGZdl{}1 Investment (}\PYG{l+s+s1}{\PYGZbs{}}\PYG{l+s+s1}{\PYGZdl{})}\PYG{l+s+s1}{\PYGZsq{}}\PYG{p}{)}
\PYG{n}{plt}\PYG{o}{.}\PYG{n}{title}\PYG{p}{(}\PYG{l+s+sa}{r}\PYG{l+s+s1}{\PYGZsq{}}\PYG{l+s+s1}{Value of }\PYG{l+s+s1}{\PYGZbs{}}\PYG{l+s+s1}{\PYGZdl{}1 Investments in Portfolios Formed on \PYGZdl{}}\PYG{l+s+s1}{\PYGZbs{}}\PYG{l+s+s1}{beta\PYGZdl{}}\PYG{l+s+s1}{\PYGZsq{}} \PYG{o}{+} \PYG{l+s+s1}{\PYGZsq{}}\PYG{l+s+se}{\PYGZbs{}n}\PYG{l+s+s1}{\PYGZsq{}} \PYG{o}{+} \PYG{l+s+sa}{f}\PYG{l+s+s1}{\PYGZsq{}}\PYG{l+s+s1}{from }\PYG{l+s+si}{\PYGZob{}}\PYG{n}{\PYGZus{}}\PYG{o}{.}\PYG{n}{index}\PYG{p}{[}\PYG{l+m+mi}{0}\PYG{p}{]}\PYG{l+s+si}{\PYGZcb{}}\PYG{l+s+s1}{ to }\PYG{l+s+si}{\PYGZob{}}\PYG{n}{\PYGZus{}}\PYG{o}{.}\PYG{n}{index}\PYG{p}{[}\PYG{o}{\PYGZhy{}}\PYG{l+m+mi}{1}\PYG{p}{]}\PYG{l+s+si}{\PYGZcb{}}\PYG{l+s+s1}{\PYGZsq{}}\PYG{p}{)}
\PYG{n}{plt}\PYG{o}{.}\PYG{n}{show}\PYG{p}{(}\PYG{p}{)}
\end{sphinxVerbatim}

\end{sphinxuseclass}\end{sphinxVerbatimInput}
\begin{sphinxVerbatimOutput}

\begin{sphinxuseclass}{cell_output}
\noindent\sphinxincludegraphics{{52874944cf95ce6bddd2862f4104c2453ad24608fa7bc643b9c626c455e224dc}.png}

\end{sphinxuseclass}\end{sphinxVerbatimOutput}

\end{sphinxuseclass}
\sphinxAtStartPar
If the CAPM does not work well, especially over the horizons we use it for (e.g., capital budgeting), why do we continue to learn it?
\begin{enumerate}
\sphinxsetlistlabels{\arabic}{enumi}{enumii}{}{.}%
\item {} 
\sphinxAtStartPar
The CAPM works well \sphinxstyleemphasis{across} asset classes. We will explore this more in the practice notebook.

\item {} 
\sphinxAtStartPar
The CAPM intuition that diversification matters is correct and important

\item {} 
\sphinxAtStartPar
The CAPM assigns high costs of capital to high\sphinxhyphen{}\$\textbackslash{}beta\$ projects (i.e., high\sphinxhyphen{}risk projects), which is a hidden benefit

\item {} 
\sphinxAtStartPar
In practice, everyone uses the CAPM

\end{enumerate}

\sphinxAtStartPar
Ivo Welch provides a more complete discussion in section 10.5 of \sphinxhref{https://book.ivo-welch.info/read/source5.mba/10-capm.pdf}{chapter 10 of this his free \sphinxstyleemphasis{Corporate Finance} textbook}.


\section{Multifactor Models}
\label{\detokenize{herron_03_lecture:multifactor-models}}
\sphinxAtStartPar
Another shortcoming of the CAPM is that it fails to explain the returns on portfolios formed on size (market capitalization) and value (book\sphinxhyphen{}to\sphinxhyphen{}market equity ratio), which we will explore in the practice notebook.
These shortcomings have led to an explosion in multifactor models, which we will explore here.


\subsection{The Fama\sphinxhyphen{}French Three\sphinxhyphen{}Factor Model}
\label{\detokenize{herron_03_lecture:the-fama-french-three-factor-model}}
\sphinxAtStartPar
Fama and French (1993) expand the CAPM by adding two additional factors to explain the excess returns on size and value.
The size factor, SMB or small\sphinxhyphen{}minus\sphinxhyphen{}big, is a diversified portfolio that measures the excess returns of  small market cap. stocks over large market cap. stocks.
The value factor, HML of high\sphinxhyphen{}minus\sphinxhyphen{}low, is a diversified portfolio that measures the excess returns of high book\sphinxhyphen{}to\sphinxhyphen{}market stocks over low  high book\sphinxhyphen{}to\sphinxhyphen{}market stocks.
We typically call this model the “Fama\sphinxhyphen{}French three\sphinxhyphen{}factor model” and express it as: \$E(R\_i) = R\_F + \textbackslash{}alpha + \textbackslash{}beta\_\{M\} {[}E(R\_M) \sphinxhyphen{} R\_M){]} + \textbackslash{}beta\_\{SMB\} SMB + \textbackslash{}beta\_\{HML\} HML\$.
There are two common uses for the three\sphinxhyphen{}factor model:
\begin{enumerate}
\sphinxsetlistlabels{\arabic}{enumi}{enumii}{}{.}%
\item {} 
\sphinxAtStartPar
Use the coefficient estimate on the intercept (i.e., \$\textbackslash{}alpha\$,  often called “Jensen’s \$\textbackslash{}alpha\$”) as a risk\sphinxhyphen{}adjusted performance measure. If \$\textbackslash{}alpha\$ is positive and statistically significant, we may attribute fund performance to manager skill.

\item {} 
\sphinxAtStartPar
Use the remaining coefficient estimates to evaluate how the fund manager generates returns. If the regression \$R\textasciicircum{}2\$ is high, we may replace the fund manager with the factor itself.

\end{enumerate}

\sphinxAtStartPar
We can use the Fama\sphinxhyphen{}French three\sphinxhyphen{}factor model to evaluate Warren Buffett at Berkshire Hathaway (BRK\sphinxhyphen{}A).
We will focus on the first three\sphinxhyphen{}years of easily available returns because Buffett had a larger edge when BRK was much smaller.

\begin{sphinxuseclass}{cell}\begin{sphinxVerbatimInput}

\begin{sphinxuseclass}{cell_input}
\begin{sphinxVerbatim}[commandchars=\\\{\}]
\PYG{n}{brk} \PYG{o}{=} \PYG{p}{(}
    \PYG{n}{yf}\PYG{o}{.}\PYG{n}{download}\PYG{p}{(}\PYG{n}{tickers}\PYG{o}{=}\PYG{l+s+s1}{\PYGZsq{}}\PYG{l+s+s1}{BRK\PYGZhy{}A}\PYG{l+s+s1}{\PYGZsq{}}\PYG{p}{,} \PYG{n}{progress}\PYG{o}{=}\PYG{k+kc}{False}\PYG{p}{)}
    \PYG{o}{.}\PYG{n}{assign}\PYG{p}{(}
        \PYG{n}{Date}\PYG{o}{=}\PYG{k}{lambda} \PYG{n}{x}\PYG{p}{:} \PYG{n}{x}\PYG{o}{.}\PYG{n}{index}\PYG{o}{.}\PYG{n}{tz\PYGZus{}localize}\PYG{p}{(}\PYG{k+kc}{None}\PYG{p}{)}\PYG{p}{,}
        \PYG{n}{Ri}\PYG{o}{=}\PYG{k}{lambda} \PYG{n}{x}\PYG{p}{:} \PYG{n}{x}\PYG{p}{[}\PYG{l+s+s1}{\PYGZsq{}}\PYG{l+s+s1}{Adj Close}\PYG{l+s+s1}{\PYGZsq{}}\PYG{p}{]}\PYG{o}{.}\PYG{n}{pct\PYGZus{}change}\PYG{p}{(}\PYG{p}{)}\PYG{o}{.}\PYG{n}{mul}\PYG{p}{(}\PYG{l+m+mi}{100}\PYG{p}{)}
    \PYG{p}{)}
    \PYG{o}{.}\PYG{n}{set\PYGZus{}index}\PYG{p}{(}\PYG{l+s+s1}{\PYGZsq{}}\PYG{l+s+s1}{Date}\PYG{l+s+s1}{\PYGZsq{}}\PYG{p}{)}
    \PYG{o}{.}\PYG{n}{join}\PYG{p}{(}\PYG{n}{ff}\PYG{p}{[}\PYG{l+m+mi}{0}\PYG{p}{]}\PYG{p}{)}
    \PYG{o}{.}\PYG{n}{assign}\PYG{p}{(}\PYG{n}{RiRF} \PYG{o}{=} \PYG{k}{lambda} \PYG{n}{x}\PYG{p}{:} \PYG{n}{x}\PYG{p}{[}\PYG{l+s+s1}{\PYGZsq{}}\PYG{l+s+s1}{Ri}\PYG{l+s+s1}{\PYGZsq{}}\PYG{p}{]} \PYG{o}{\PYGZhy{}} \PYG{n}{x}\PYG{p}{[}\PYG{l+s+s1}{\PYGZsq{}}\PYG{l+s+s1}{RF}\PYG{l+s+s1}{\PYGZsq{}}\PYG{p}{]}\PYG{p}{)}
    \PYG{o}{.}\PYG{n}{rename}\PYG{p}{(}\PYG{n}{columns}\PYG{o}{=}\PYG{p}{\PYGZob{}}\PYG{l+s+s1}{\PYGZsq{}}\PYG{l+s+s1}{Mkt\PYGZhy{}RF}\PYG{l+s+s1}{\PYGZsq{}}\PYG{p}{:} \PYG{l+s+s1}{\PYGZsq{}}\PYG{l+s+s1}{MktRF}\PYG{l+s+s1}{\PYGZsq{}}\PYG{p}{\PYGZcb{}}\PYG{p}{)}
    \PYG{o}{.}\PYG{n}{rename\PYGZus{}axis}\PYG{p}{(}\PYG{n}{columns}\PYG{o}{=}\PYG{l+s+s1}{\PYGZsq{}}\PYG{l+s+s1}{Variable}\PYG{l+s+s1}{\PYGZsq{}}\PYG{p}{)}
\PYG{p}{)}
\end{sphinxVerbatim}

\end{sphinxuseclass}\end{sphinxVerbatimInput}

\end{sphinxuseclass}
\begin{sphinxuseclass}{cell}\begin{sphinxVerbatimInput}

\begin{sphinxuseclass}{cell_input}
\begin{sphinxVerbatim}[commandchars=\\\{\}]
\PYG{n}{model} \PYG{o}{=} \PYG{n}{smf}\PYG{o}{.}\PYG{n}{ols}\PYG{p}{(}\PYG{n}{formula}\PYG{o}{=}\PYG{l+s+s1}{\PYGZsq{}}\PYG{l+s+s1}{RiRF \PYGZti{} MktRF + SMB + HML}\PYG{l+s+s1}{\PYGZsq{}}\PYG{p}{,} \PYG{n}{data}\PYG{o}{=}\PYG{n}{brk}\PYG{o}{.}\PYG{n}{iloc}\PYG{p}{[}\PYG{p}{:}\PYG{l+m+mi}{756}\PYG{p}{]}\PYG{p}{)}
\PYG{n}{fit} \PYG{o}{=} \PYG{n}{model}\PYG{o}{.}\PYG{n}{fit}\PYG{p}{(}\PYG{p}{)}
\PYG{n}{summary} \PYG{o}{=} \PYG{n}{fit}\PYG{o}{.}\PYG{n}{summary}\PYG{p}{(}\PYG{p}{)}
\PYG{n}{summary}
\end{sphinxVerbatim}

\end{sphinxuseclass}\end{sphinxVerbatimInput}
\begin{sphinxVerbatimOutput}

\begin{sphinxuseclass}{cell_output}
\begin{sphinxVerbatim}[commandchars=\\\{\}]
\PYGZlt{}class \PYGZsq{}statsmodels.iolib.summary.Summary\PYGZsq{}\PYGZgt{}
\PYGZdq{}\PYGZdq{}\PYGZdq{}
                            OLS Regression Results                            
==============================================================================
Dep. Variable:                   RiRF   R\PYGZhy{}squared:                       0.053
Model:                            OLS   Adj. R\PYGZhy{}squared:                  0.049
Method:                 Least Squares   F\PYGZhy{}statistic:                     13.91
Date:                Wed, 15 Mar 2023   Prob (F\PYGZhy{}statistic):           7.81e\PYGZhy{}09
Time:                        11:35:00   Log\PYGZhy{}Likelihood:                \PYGZhy{}1208.9
No. Observations:                 755   AIC:                             2426.
Df Residuals:                     751   BIC:                             2444.
Df Model:                           3                                         
Covariance Type:            nonrobust                                         
==============================================================================
                 coef    std err          t      P\PYGZgt{}|t|      [0.025      0.975]
\PYGZhy{}\PYGZhy{}\PYGZhy{}\PYGZhy{}\PYGZhy{}\PYGZhy{}\PYGZhy{}\PYGZhy{}\PYGZhy{}\PYGZhy{}\PYGZhy{}\PYGZhy{}\PYGZhy{}\PYGZhy{}\PYGZhy{}\PYGZhy{}\PYGZhy{}\PYGZhy{}\PYGZhy{}\PYGZhy{}\PYGZhy{}\PYGZhy{}\PYGZhy{}\PYGZhy{}\PYGZhy{}\PYGZhy{}\PYGZhy{}\PYGZhy{}\PYGZhy{}\PYGZhy{}\PYGZhy{}\PYGZhy{}\PYGZhy{}\PYGZhy{}\PYGZhy{}\PYGZhy{}\PYGZhy{}\PYGZhy{}\PYGZhy{}\PYGZhy{}\PYGZhy{}\PYGZhy{}\PYGZhy{}\PYGZhy{}\PYGZhy{}\PYGZhy{}\PYGZhy{}\PYGZhy{}\PYGZhy{}\PYGZhy{}\PYGZhy{}\PYGZhy{}\PYGZhy{}\PYGZhy{}\PYGZhy{}\PYGZhy{}\PYGZhy{}\PYGZhy{}\PYGZhy{}\PYGZhy{}\PYGZhy{}\PYGZhy{}\PYGZhy{}\PYGZhy{}\PYGZhy{}\PYGZhy{}\PYGZhy{}\PYGZhy{}\PYGZhy{}\PYGZhy{}\PYGZhy{}\PYGZhy{}\PYGZhy{}\PYGZhy{}\PYGZhy{}\PYGZhy{}\PYGZhy{}\PYGZhy{}
Intercept      0.0801      0.044      1.809      0.071      \PYGZhy{}0.007       0.167
MktRF          0.3484      0.075      4.643      0.000       0.201       0.496
SMB            0.4021      0.093      4.302      0.000       0.219       0.586
HML            0.0907      0.125      0.724      0.469      \PYGZhy{}0.155       0.336
==============================================================================
Omnibus:                      118.864   Durbin\PYGZhy{}Watson:                   1.797
Prob(Omnibus):                  0.000   Jarque\PYGZhy{}Bera (JB):             1308.034
Skew:                           0.284   Prob(JB):                    9.20e\PYGZhy{}285
Kurtosis:                       9.423   Cond. No.                         3.51
==============================================================================

Notes:
[1] Standard Errors assume that the covariance matrix of the errors is correctly specified.
\PYGZdq{}\PYGZdq{}\PYGZdq{}
\end{sphinxVerbatim}

\end{sphinxuseclass}\end{sphinxVerbatimOutput}

\end{sphinxuseclass}
\sphinxAtStartPar
The \$\textbackslash{}alpha\$ above seems small, but this is a \sphinxstyleemphasis{daily} value.
We can multiple \$\textbackslash{}alpha\$ by 252 to annualize it.

\begin{sphinxuseclass}{cell}\begin{sphinxVerbatimInput}

\begin{sphinxuseclass}{cell_input}
\begin{sphinxVerbatim}[commandchars=\\\{\}]
\PYG{n+nb}{print}\PYG{p}{(}\PYG{l+s+sa}{f}\PYG{l+s+s2}{\PYGZdq{}}\PYG{l+s+s2}{Buffet}\PYG{l+s+s2}{\PYGZsq{}}\PYG{l+s+s2}{s annualized alpha in the early 1980s: }\PYG{l+s+si}{\PYGZob{}}\PYG{n}{fit}\PYG{o}{.}\PYG{n}{params}\PYG{p}{[}\PYG{l+s+s1}{\PYGZsq{}}\PYG{l+s+s1}{Intercept}\PYG{l+s+s1}{\PYGZsq{}}\PYG{p}{]} \PYG{o}{*} \PYG{l+m+mi}{252}\PYG{l+s+si}{:}\PYG{l+s+s2}{0.4f}\PYG{l+s+si}{\PYGZcb{}}\PYG{l+s+s2}{\PYGZdq{}}\PYG{p}{)}
\end{sphinxVerbatim}

\end{sphinxuseclass}\end{sphinxVerbatimInput}
\begin{sphinxVerbatimOutput}

\begin{sphinxuseclass}{cell_output}
\begin{sphinxVerbatim}[commandchars=\\\{\}]
Buffet\PYGZsq{}s annualized alpha in the early 1980s: 20.1836
\end{sphinxVerbatim}

\end{sphinxuseclass}\end{sphinxVerbatimOutput}

\end{sphinxuseclass}
\sphinxAtStartPar
We will explore rolling \$\textbackslash{}alpha\$s and \$\textbackslash{}beta\$s in the practice notebook using \sphinxcode{\sphinxupquote{RollingOLS()}} from \sphinxcode{\sphinxupquote{statsmodels.regression.rolling}}.


\subsection{The Four\sphinxhyphen{}Factor and Five\sphinxhyphen{}Factor Models}
\label{\detokenize{herron_03_lecture:the-four-factor-and-five-factor-models}}
\sphinxAtStartPar
There are literally hundreds of published factors!
However, many of them have little explanatory power, in or out of sample.
Two more factor models that have explanatory power, economic intuition, and widespread adoption are the four\sphinxhyphen{}factor model and five\sphinxhyphen{}factor model.

\sphinxAtStartPar
The four\sphinxhyphen{}factor model adds a momentum factor to the Fama\sphinxhyphen{}French three\sphinxhyphen{}factor model.
The momentum factor is a diversified portfolio that measures the excess returns of winner stocks over the loser stocks over the past 12 months.
The momentum factor is often called UMD for up\sphinxhyphen{}minus\sphinxhyphen{}down or WML for winners\sphinxhyphen{}minus\sphinxhyphen{}losers.
French stores the momentum factor in a different file because Fama and French are skeptical of momentum as a foundational risk factor.

\sphinxAtStartPar
The five\sphinxhyphen{}factor model adds profitability and investment policy factors.
The profitability factor, RMW or robust\sphinxhyphen{}minus\sphinxhyphen{}weak, measures the excess returns of stocks with high profits over those with low profits.
The investment policy factor, CMA or conservative\sphinxhyphen{}minus\sphinxhyphen{}aggressive, measures the excess returns of stocks with low corporate investment (conservative) over those with high corporate investment (aggressive).

\sphinxAtStartPar
We will explore the four\sphinxhyphen{}factor and five\sphinxhyphen{}factor models in the practice notebook.

\sphinxstepscope


\section{Herron Topic 3 \sphinxhyphen{} Practice (Blank)}
\label{\detokenize{herron_03_practice:herron-topic-3-practice-blank}}\label{\detokenize{herron_03_practice::doc}}

\subsection{Announcements}
\label{\detokenize{herron_03_practice:announcements}}

\subsection{Practice}
\label{\detokenize{herron_03_practice:practice}}

\subsubsection{Plot the security market line (SML) for a variety of asset classes}
\label{\detokenize{herron_03_practice:plot-the-security-market-line-sml-for-a-variety-of-asset-classes}}
\sphinxAtStartPar
Use the past three years of daily data for the following exhange traded funds (ETFs):
\begin{enumerate}
\sphinxsetlistlabels{\arabic}{enumi}{enumii}{}{.}%
\item {} 
\sphinxAtStartPar
SPY (SPDR—Standard and Poor’s Depository Receipts—ETF for the S\&P 500 index)

\item {} 
\sphinxAtStartPar
BIL (SPDR ETF for 1\sphinxhyphen{}3 month Treasury bills)

\item {} 
\sphinxAtStartPar
GLD (SPDR ETF for gold)

\item {} 
\sphinxAtStartPar
JNK (SPDR ETF for high\sphinxhyphen{}yield debt)

\item {} 
\sphinxAtStartPar
MDY (SPDR ETF for S\&P 400 mid\sphinxhyphen{}cap index)

\item {} 
\sphinxAtStartPar
SLY (SPDR ETF for S\&P 600 small\sphinxhyphen{}cap index)

\item {} 
\sphinxAtStartPar
SPBO (SPDR ETF for corporate bonds)

\item {} 
\sphinxAtStartPar
SPMB (SPDR ETF for mortgage\sphinxhyphen{}backed securities)

\item {} 
\sphinxAtStartPar
SPTL (SPDR ETF for long\sphinxhyphen{}term Treasury bonds)

\end{enumerate}


\subsubsection{Plot the SML for the Dow Jones Industrial Average (DJIA) stocks}
\label{\detokenize{herron_03_practice:plot-the-sml-for-the-dow-jones-industrial-average-djia-stocks}}
\sphinxAtStartPar
Use the past three years of daily returns data for the stocks listed on the \sphinxhref{https://en.wikipedia.org/wiki/Dow\_Jones\_Industrial\_Average}{DJIA Wikipedia page}.
Compare the DJIA SML to the asset class SML above.


\subsubsection{Plot the SML for the five portfolios formed on beta}
\label{\detokenize{herron_03_practice:plot-the-sml-for-the-five-portfolios-formed-on-beta}}
\sphinxAtStartPar
Download data for portfolios formed on \$\textbackslash{}beta\$ (\sphinxcode{\sphinxupquote{Portfolios\_Formed\_on\_BETA}}) from Ken French.
For the value\sphinxhyphen{}weighted portfolios, plot realized returns versus \$\textbackslash{}beta\$.
These data should elements \sphinxcode{\sphinxupquote{{[}2{]}}} and \sphinxcode{\sphinxupquote{{[}6{]}}}, respectively.


\subsubsection{Estimate the CAPM \$\textbackslash{}beta\$s on several levered and inverse exchange traded funds (ETFs)}
\label{\detokenize{herron_03_practice:estimate-the-capm-beta-s-on-several-levered-and-inverse-exchange-traded-funds-etfs}}
\sphinxAtStartPar
Try the following ETFs:
\begin{enumerate}
\sphinxsetlistlabels{\arabic}{enumi}{enumii}{}{.}%
\item {} 
\sphinxAtStartPar
SPY

\item {} 
\sphinxAtStartPar
UPRO

\item {} 
\sphinxAtStartPar
SPXU

\end{enumerate}

\sphinxAtStartPar
Can you determine what these products do from the data alone?
Estimate \$\textbackslash{}beta\$s and plot cumulative returns.
You may want to pick short periods of time with large market swings.


\subsubsection{Explore the size factor}
\label{\detokenize{herron_03_practice:explore-the-size-factor}}

\paragraph{Estimate \$\textbackslash{}alpha\$s for the ten portfolios formed on size}
\label{\detokenize{herron_03_practice:estimate-alpha-s-for-the-ten-portfolios-formed-on-size}}
\sphinxAtStartPar
Academics started researching size\sphinxhyphen{}based portfolios in the early 1980s, so you may want to focus on the pre\sphinxhyphen{}1980 sample.


\paragraph{Are the returns on these ten portfolios formed on size concentrated in a specific month?}
\label{\detokenize{herron_03_practice:are-the-returns-on-these-ten-portfolios-formed-on-size-concentrated-in-a-specific-month}}

\paragraph{Compare the size factor to the market factor}
\label{\detokenize{herron_03_practice:compare-the-size-factor-to-the-market-factor}}
\sphinxAtStartPar
You may want to consider mean excess returns by decade.


\subsubsection{Repeat the exercises above with the value factor}
\label{\detokenize{herron_03_practice:repeat-the-exercises-above-with-the-value-factor}}

\subsubsection{Repeat the exercises above with the momentum factor}
\label{\detokenize{herron_03_practice:repeat-the-exercises-above-with-the-momentum-factor}}
\sphinxAtStartPar
You may find it helpful to consider the worst months and years for the momentum factor.


\subsubsection{Plot the coefficient estimates from a rolling Fama\sphinxhyphen{}French three\sphinxhyphen{}factor model for Berkshire Hathaway}
\label{\detokenize{herron_03_practice:plot-the-coefficient-estimates-from-a-rolling-fama-french-three-factor-model-for-berkshire-hathaway}}
\sphinxAtStartPar
Use a three\sphinxhyphen{}year window with daily returns.
How has Buffett’s \$\textbackslash{}alpha\$ and \$\textbackslash{}beta\$s changed over the past four decades?


\subsubsection{Use the three\sphinxhyphen{}, four\sphinxhyphen{}, and five\sphinxhyphen{}factor models to determine how the ARKK Innovation ETF generates returns}
\label{\detokenize{herron_03_practice:use-the-three-four-and-five-factor-models-to-determine-how-the-arkk-innovation-etf-generates-returns}}
\sphinxstepscope


\section{Herron Topic 3 \sphinxhyphen{} Practice (Monday 2:45 PM, Section 3)}
\label{\detokenize{herron_03_practice_03:herron-topic-3-practice-monday-2-45-pm-section-3}}\label{\detokenize{herron_03_practice_03::doc}}

\subsection{Announcements}
\label{\detokenize{herron_03_practice_03:announcements}}\begin{itemize}
\item {} 
\sphinxAtStartPar
Team Project 1 grades posted
\begin{itemize}
\item {} 
\sphinxAtStartPar
Mean of 81\% and median of 84\%

\item {} 
\sphinxAtStartPar
100\% on Teammates reviews pushes mean to 84\%, which is about the target

\item {} 
\sphinxAtStartPar
Overall course mean is 86\%, so I would curve up overall course grades about 4\% if posted grades today, much more at the low end

\item {} 
\sphinxAtStartPar
However, I cannot commit to a curve today

\end{itemize}

\item {} 
\sphinxAtStartPar
20,000 XP on DataCamp due by 11:59 PM on Friday

\end{itemize}


\subsection{Practice}
\label{\detokenize{herron_03_practice_03:practice}}
\begin{sphinxuseclass}{cell}\begin{sphinxVerbatimInput}

\begin{sphinxuseclass}{cell_input}
\begin{sphinxVerbatim}[commandchars=\\\{\}]
\PYG{k+kn}{import} \PYG{n+nn}{matplotlib}\PYG{n+nn}{.}\PYG{n+nn}{pyplot} \PYG{k}{as} \PYG{n+nn}{plt}
\PYG{k+kn}{import} \PYG{n+nn}{numpy} \PYG{k}{as} \PYG{n+nn}{np}
\PYG{k+kn}{import} \PYG{n+nn}{pandas} \PYG{k}{as} \PYG{n+nn}{pd}
\end{sphinxVerbatim}

\end{sphinxuseclass}\end{sphinxVerbatimInput}

\end{sphinxuseclass}
\begin{sphinxuseclass}{cell}\begin{sphinxVerbatimInput}

\begin{sphinxuseclass}{cell_input}
\begin{sphinxVerbatim}[commandchars=\\\{\}]
\PYG{o}{\PYGZpc{}}\PYG{k}{config} InlineBackend.figure\PYGZus{}format = \PYGZsq{}retina\PYGZsq{}
\PYG{o}{\PYGZpc{}}\PYG{k}{precision} 4
\PYG{n}{pd}\PYG{o}{.}\PYG{n}{options}\PYG{o}{.}\PYG{n}{display}\PYG{o}{.}\PYG{n}{float\PYGZus{}format} \PYG{o}{=} \PYG{l+s+s1}{\PYGZsq{}}\PYG{l+s+si}{\PYGZob{}:.4f\PYGZcb{}}\PYG{l+s+s1}{\PYGZsq{}}\PYG{o}{.}\PYG{n}{format}
\end{sphinxVerbatim}

\end{sphinxuseclass}\end{sphinxVerbatimInput}

\end{sphinxuseclass}
\begin{sphinxuseclass}{cell}\begin{sphinxVerbatimInput}

\begin{sphinxuseclass}{cell_input}
\begin{sphinxVerbatim}[commandchars=\\\{\}]
\PYG{k+kn}{import} \PYG{n+nn}{seaborn} \PYG{k}{as} \PYG{n+nn}{sns}
\PYG{k+kn}{import} \PYG{n+nn}{statsmodels}\PYG{n+nn}{.}\PYG{n+nn}{formula}\PYG{n+nn}{.}\PYG{n+nn}{api} \PYG{k}{as} \PYG{n+nn}{smf}
\end{sphinxVerbatim}

\end{sphinxuseclass}\end{sphinxVerbatimInput}

\end{sphinxuseclass}
\begin{sphinxuseclass}{cell}\begin{sphinxVerbatimInput}

\begin{sphinxuseclass}{cell_input}
\begin{sphinxVerbatim}[commandchars=\\\{\}]
\PYG{k+kn}{import} \PYG{n+nn}{yfinance} \PYG{k}{as} \PYG{n+nn}{yf}
\PYG{k+kn}{import} \PYG{n+nn}{pandas\PYGZus{}datareader} \PYG{k}{as} \PYG{n+nn}{pdr}
\PYG{k+kn}{import} \PYG{n+nn}{requests\PYGZus{}cache}
\PYG{n}{session} \PYG{o}{=} \PYG{n}{requests\PYGZus{}cache}\PYG{o}{.}\PYG{n}{CachedSession}\PYG{p}{(}\PYG{p}{)}
\end{sphinxVerbatim}

\end{sphinxuseclass}\end{sphinxVerbatimInput}

\end{sphinxuseclass}

\subsubsection{Plot the security market line (SML) for a variety of asset classes}
\label{\detokenize{herron_03_practice_03:plot-the-security-market-line-sml-for-a-variety-of-asset-classes}}
\sphinxAtStartPar
Use the past three years of daily data for the following exhange traded funds (ETFs):
\begin{enumerate}
\sphinxsetlistlabels{\arabic}{enumi}{enumii}{}{.}%
\item {} 
\sphinxAtStartPar
SPY (SPDR—Standard and Poor’s Depository Receipts—ETF for the S\&P 500 index)

\item {} 
\sphinxAtStartPar
BIL (SPDR ETF for 1\sphinxhyphen{}3 month Treasury bills)

\item {} 
\sphinxAtStartPar
GLD (SPDR ETF for gold)

\item {} 
\sphinxAtStartPar
JNK (SPDR ETF for high\sphinxhyphen{}yield debt)

\item {} 
\sphinxAtStartPar
MDY (SPDR ETF for S\&P 400 mid\sphinxhyphen{}cap index)

\item {} 
\sphinxAtStartPar
SLY (SPDR ETF for S\&P 600 small\sphinxhyphen{}cap index)

\item {} 
\sphinxAtStartPar
SPBO (SPDR ETF for corporate bonds)

\item {} 
\sphinxAtStartPar
SPMB (SPDR ETF for mortgage\sphinxhyphen{}backed securities)

\item {} 
\sphinxAtStartPar
SPTL (SPDR ETF for long\sphinxhyphen{}term Treasury bonds)

\end{enumerate}

\begin{sphinxuseclass}{cell}\begin{sphinxVerbatimInput}

\begin{sphinxuseclass}{cell_input}
\begin{sphinxVerbatim}[commandchars=\\\{\}]
\PYG{n}{etf} \PYG{o}{=} \PYG{p}{(}
    \PYG{n}{yf}\PYG{o}{.}\PYG{n}{download}\PYG{p}{(}
        \PYG{n}{tickers}\PYG{o}{=}\PYG{l+s+s1}{\PYGZsq{}}\PYG{l+s+s1}{SPY BIL GLD JNK MDY SLY SPBO SPMB SPTL}\PYG{l+s+s1}{\PYGZsq{}}\PYG{p}{,}
        \PYG{n}{progress}\PYG{o}{=}\PYG{k+kc}{False}
    \PYG{p}{)}
    \PYG{o}{.}\PYG{n}{assign}\PYG{p}{(}\PYG{n}{Date} \PYG{o}{=} \PYG{k}{lambda} \PYG{n}{x}\PYG{p}{:} \PYG{n}{x}\PYG{o}{.}\PYG{n}{index}\PYG{o}{.}\PYG{n}{tz\PYGZus{}localize}\PYG{p}{(}\PYG{k+kc}{None}\PYG{p}{)}\PYG{p}{)}
    \PYG{o}{.}\PYG{n}{set\PYGZus{}index}\PYG{p}{(}\PYG{l+s+s1}{\PYGZsq{}}\PYG{l+s+s1}{Date}\PYG{l+s+s1}{\PYGZsq{}}\PYG{p}{)}
    \PYG{o}{.}\PYG{n}{rename\PYGZus{}axis}\PYG{p}{(}\PYG{n}{columns}\PYG{o}{=}\PYG{p}{[}\PYG{l+s+s1}{\PYGZsq{}}\PYG{l+s+s1}{Variable}\PYG{l+s+s1}{\PYGZsq{}}\PYG{p}{,} \PYG{l+s+s1}{\PYGZsq{}}\PYG{l+s+s1}{Ticker}\PYG{l+s+s1}{\PYGZsq{}}\PYG{p}{]}\PYG{p}{)}
    \PYG{p}{[}\PYG{l+s+s1}{\PYGZsq{}}\PYG{l+s+s1}{Adj Close}\PYG{l+s+s1}{\PYGZsq{}}\PYG{p}{]}
    \PYG{o}{.}\PYG{n}{pct\PYGZus{}change}\PYG{p}{(}\PYG{p}{)}
    \PYG{o}{.}\PYG{n}{dropna}\PYG{p}{(}\PYG{p}{)}
\PYG{p}{)}
\end{sphinxVerbatim}

\end{sphinxuseclass}\end{sphinxVerbatimInput}

\end{sphinxuseclass}
\begin{sphinxuseclass}{cell}\begin{sphinxVerbatimInput}

\begin{sphinxuseclass}{cell_input}
\begin{sphinxVerbatim}[commandchars=\\\{\}]
\PYG{n}{ff} \PYG{o}{=} \PYG{p}{(}
    \PYG{n}{pdr}\PYG{o}{.}\PYG{n}{DataReader}\PYG{p}{(}
        \PYG{n}{name}\PYG{o}{=}\PYG{l+s+s1}{\PYGZsq{}}\PYG{l+s+s1}{F\PYGZhy{}F\PYGZus{}Research\PYGZus{}Data\PYGZus{}Factors\PYGZus{}daily}\PYG{l+s+s1}{\PYGZsq{}}\PYG{p}{,}
        \PYG{n}{data\PYGZus{}source}\PYG{o}{=}\PYG{l+s+s1}{\PYGZsq{}}\PYG{l+s+s1}{famafrench}\PYG{l+s+s1}{\PYGZsq{}}\PYG{p}{,}
        \PYG{n}{start}\PYG{o}{=}\PYG{l+s+s1}{\PYGZsq{}}\PYG{l+s+s1}{1900}\PYG{l+s+s1}{\PYGZsq{}}\PYG{p}{,}
        \PYG{n}{session}\PYG{o}{=}\PYG{n}{session}
    \PYG{p}{)}
    \PYG{p}{[}\PYG{l+m+mi}{0}\PYG{p}{]}
    \PYG{o}{.}\PYG{n}{rename\PYGZus{}axis}\PYG{p}{(}\PYG{n}{columns}\PYG{o}{=}\PYG{l+s+s1}{\PYGZsq{}}\PYG{l+s+s1}{Variable}\PYG{l+s+s1}{\PYGZsq{}}\PYG{p}{)}
    \PYG{o}{.}\PYG{n}{div}\PYG{p}{(}\PYG{l+m+mi}{100}\PYG{p}{)}
\PYG{p}{)}
\end{sphinxVerbatim}

\end{sphinxuseclass}\end{sphinxVerbatimInput}

\end{sphinxuseclass}
\begin{sphinxuseclass}{cell}\begin{sphinxVerbatimInput}

\begin{sphinxuseclass}{cell_input}
\begin{sphinxVerbatim}[commandchars=\\\{\}]
\PYG{k}{def} \PYG{n+nf}{mean}\PYG{p}{(}\PYG{n}{ri}\PYG{p}{,} \PYG{n}{ann}\PYG{o}{=}\PYG{l+m+mi}{252}\PYG{p}{,} \PYG{n}{mul}\PYG{o}{=}\PYG{l+m+mi}{100}\PYG{p}{)}\PYG{p}{:}
    \PYG{k}{return} \PYG{n}{mul} \PYG{o}{*} \PYG{n}{ann} \PYG{o}{*} \PYG{n}{ri}\PYG{o}{.}\PYG{n}{mean}\PYG{p}{(}\PYG{p}{)}
\end{sphinxVerbatim}

\end{sphinxuseclass}\end{sphinxVerbatimInput}

\end{sphinxuseclass}
\begin{sphinxuseclass}{cell}\begin{sphinxVerbatimInput}

\begin{sphinxuseclass}{cell_input}
\begin{sphinxVerbatim}[commandchars=\\\{\}]
\PYG{k}{def} \PYG{n+nf}{beta}\PYG{p}{(}\PYG{n}{ri}\PYG{p}{,} \PYG{n}{rf}\PYG{o}{=}\PYG{n}{ff}\PYG{p}{[}\PYG{l+s+s1}{\PYGZsq{}}\PYG{l+s+s1}{RF}\PYG{l+s+s1}{\PYGZsq{}}\PYG{p}{]}\PYG{p}{,} \PYG{n}{rm\PYGZus{}rf}\PYG{o}{=}\PYG{n}{ff}\PYG{p}{[}\PYG{l+s+s1}{\PYGZsq{}}\PYG{l+s+s1}{Mkt\PYGZhy{}RF}\PYG{l+s+s1}{\PYGZsq{}}\PYG{p}{]}\PYG{p}{)}\PYG{p}{:}
    \PYG{n}{ri\PYGZus{}rf} \PYG{o}{=} \PYG{n}{ri}\PYG{o}{.}\PYG{n}{sub}\PYG{p}{(}\PYG{n}{rf}\PYG{p}{)}\PYG{o}{.}\PYG{n}{dropna}\PYG{p}{(}\PYG{p}{)}
    \PYG{k}{return} \PYG{n}{ri\PYGZus{}rf}\PYG{o}{.}\PYG{n}{cov}\PYG{p}{(}\PYG{n}{rm\PYGZus{}rf}\PYG{p}{)} \PYG{o}{/} \PYG{n}{rm\PYGZus{}rf}\PYG{o}{.}\PYG{n}{loc}\PYG{p}{[}\PYG{n}{ri\PYGZus{}rf}\PYG{o}{.}\PYG{n}{index}\PYG{p}{]}\PYG{o}{.}\PYG{n}{var}\PYG{p}{(}\PYG{p}{)}
\end{sphinxVerbatim}

\end{sphinxuseclass}\end{sphinxVerbatimInput}

\end{sphinxuseclass}
\begin{sphinxuseclass}{cell}\begin{sphinxVerbatimInput}

\begin{sphinxuseclass}{cell_input}
\begin{sphinxVerbatim}[commandchars=\\\{\}]
\PYG{k}{def} \PYG{n+nf}{date\PYGZus{}range}\PYG{p}{(}\PYG{n}{x}\PYG{p}{)}\PYG{p}{:}
    \PYG{k}{return} \PYG{l+s+sa}{f}\PYG{l+s+s1}{\PYGZsq{}}\PYG{l+s+si}{\PYGZob{}}\PYG{n}{x}\PYG{o}{.}\PYG{n}{index}\PYG{p}{[}\PYG{l+m+mi}{0}\PYG{p}{]}\PYG{l+s+si}{:}\PYG{l+s+s1}{\PYGZpc{}b \PYGZpc{}d, \PYGZpc{}Y}\PYG{l+s+si}{\PYGZcb{}}\PYG{l+s+s1}{ to }\PYG{l+s+si}{\PYGZob{}}\PYG{n}{x}\PYG{o}{.}\PYG{n}{index}\PYG{p}{[}\PYG{o}{\PYGZhy{}}\PYG{l+m+mi}{1}\PYG{p}{]}\PYG{l+s+si}{:}\PYG{l+s+s1}{\PYGZpc{}b \PYGZpc{}d, \PYGZpc{}Y}\PYG{l+s+si}{\PYGZcb{}}\PYG{l+s+s1}{\PYGZsq{}}
\end{sphinxVerbatim}

\end{sphinxuseclass}\end{sphinxVerbatimInput}

\end{sphinxuseclass}
\begin{sphinxuseclass}{cell}\begin{sphinxVerbatimInput}

\begin{sphinxuseclass}{cell_input}
\begin{sphinxVerbatim}[commandchars=\\\{\}]
\PYG{n}{\PYGZus{}1} \PYG{o}{=} \PYG{n}{etf}\PYG{o}{.}\PYG{n}{iloc}\PYG{p}{[}\PYG{o}{\PYGZhy{}}\PYG{l+m+mi}{756}\PYG{p}{:}\PYG{p}{]}
\PYG{n}{\PYGZus{}2} \PYG{o}{=} \PYG{n}{\PYGZus{}1}\PYG{o}{.}\PYG{n}{agg}\PYG{p}{(}\PYG{p}{[}\PYG{n}{mean}\PYG{p}{,} \PYG{n}{beta}\PYG{p}{]}\PYG{p}{)}\PYG{o}{.}\PYG{n}{rename\PYGZus{}axis}\PYG{p}{(}\PYG{n}{index}\PYG{o}{=}\PYG{l+s+s1}{\PYGZsq{}}\PYG{l+s+s1}{Statistic}\PYG{l+s+s1}{\PYGZsq{}}\PYG{p}{)}\PYG{o}{.}\PYG{n}{transpose}\PYG{p}{(}\PYG{p}{)}

\PYG{n}{sns}\PYG{o}{.}\PYG{n}{regplot}\PYG{p}{(}\PYG{n}{x}\PYG{o}{=}\PYG{l+s+s1}{\PYGZsq{}}\PYG{l+s+s1}{beta}\PYG{l+s+s1}{\PYGZsq{}}\PYG{p}{,} \PYG{n}{y}\PYG{o}{=}\PYG{l+s+s1}{\PYGZsq{}}\PYG{l+s+s1}{mean}\PYG{l+s+s1}{\PYGZsq{}}\PYG{p}{,} \PYG{n}{data}\PYG{o}{=}\PYG{n}{\PYGZus{}2}\PYG{p}{)}
\PYG{k}{for} \PYG{n}{t}\PYG{p}{,} \PYG{p}{(}\PYG{n}{x}\PYG{p}{,} \PYG{n}{y}\PYG{p}{)} \PYG{o+ow}{in} \PYG{n}{\PYGZus{}2}\PYG{p}{[}\PYG{p}{[}\PYG{l+s+s1}{\PYGZsq{}}\PYG{l+s+s1}{beta}\PYG{l+s+s1}{\PYGZsq{}}\PYG{p}{,} \PYG{l+s+s1}{\PYGZsq{}}\PYG{l+s+s1}{mean}\PYG{l+s+s1}{\PYGZsq{}}\PYG{p}{]}\PYG{p}{]}\PYG{o}{.}\PYG{n}{iterrows}\PYG{p}{(}\PYG{p}{)}\PYG{p}{:}
    \PYG{n}{plt}\PYG{o}{.}\PYG{n}{annotate}\PYG{p}{(}\PYG{n}{text}\PYG{o}{=}\PYG{n}{t}\PYG{p}{,} \PYG{n}{xy}\PYG{o}{=}\PYG{p}{(}\PYG{n}{x}\PYG{p}{,} \PYG{n}{y}\PYG{p}{)}\PYG{p}{)} \PYG{c+c1}{\PYGZsh{} use a for\PYGZhy{}loop to add tickers}

\PYG{n}{plt}\PYG{o}{.}\PYG{n}{ylabel}\PYG{p}{(}\PYG{l+s+s1}{\PYGZsq{}}\PYG{l+s+s1}{Annualized Mean of Daily Returns (}\PYG{l+s+s1}{\PYGZpc{}}\PYG{l+s+s1}{)}\PYG{l+s+s1}{\PYGZsq{}}\PYG{p}{)}
\PYG{n}{plt}\PYG{o}{.}\PYG{n}{xlabel}\PYG{p}{(}\PYG{l+s+sa}{r}\PYG{l+s+s1}{\PYGZsq{}}\PYG{l+s+s1}{Capital Asset Pricing Model (CAPM) \PYGZdl{}}\PYG{l+s+s1}{\PYGZbs{}}\PYG{l+s+s1}{beta\PYGZdl{}}\PYG{l+s+s1}{\PYGZsq{}}\PYG{p}{)}
\PYG{n}{plt}\PYG{o}{.}\PYG{n}{title}\PYG{p}{(}
    \PYG{l+s+s1}{\PYGZsq{}}\PYG{l+s+s1}{Security Market Line (SML) for Exchange Traded Funds (ETFs)}\PYG{l+s+se}{\PYGZbs{}n}\PYG{l+s+s1}{\PYGZsq{}} \PYG{o}{+}
    \PYG{l+s+s1}{\PYGZsq{}}\PYG{l+s+s1}{for Daily Returns from }\PYG{l+s+s1}{\PYGZsq{}} \PYG{o}{+} \PYG{n}{\PYGZus{}1}\PYG{o}{.}\PYG{n}{pipe}\PYG{p}{(}\PYG{n}{date\PYGZus{}range}\PYG{p}{)}\PYG{p}{)}
\PYG{n}{plt}\PYG{o}{.}\PYG{n}{show}\PYG{p}{(}\PYG{p}{)}
\end{sphinxVerbatim}

\end{sphinxuseclass}\end{sphinxVerbatimInput}
\begin{sphinxVerbatimOutput}

\begin{sphinxuseclass}{cell_output}
\noindent\sphinxincludegraphics{{65593fe6e057815dbfbaafe719da3e10322b83eb93691fa4d88b93840bf40803}.png}

\end{sphinxuseclass}\end{sphinxVerbatimOutput}

\end{sphinxuseclass}
\sphinxAtStartPar
\sphinxstyleemphasis{\sphinxstylestrong{Note the asset\sphinxhyphen{}class ETFs fit the SML very well!}}


\subsubsection{Plot the SML for the Dow Jones Industrial Average (DJIA) stocks}
\label{\detokenize{herron_03_practice_03:plot-the-sml-for-the-dow-jones-industrial-average-djia-stocks}}
\sphinxAtStartPar
Use the past three years of daily returns data for the stocks listed on the \sphinxhref{https://en.wikipedia.org/wiki/Dow\_Jones\_Industrial\_Average}{DJIA Wikipedia page}.
Compare the DJIA SML to the asset class SML above.

\sphinxAtStartPar
We can re\sphinxhyphen{}use out code from above!

\begin{sphinxuseclass}{cell}\begin{sphinxVerbatimInput}

\begin{sphinxuseclass}{cell_input}
\begin{sphinxVerbatim}[commandchars=\\\{\}]
\PYG{n}{wiki} \PYG{o}{=} \PYG{n}{pd}\PYG{o}{.}\PYG{n}{read\PYGZus{}html}\PYG{p}{(}\PYG{l+s+s1}{\PYGZsq{}}\PYG{l+s+s1}{https://en.wikipedia.org/wiki/Dow\PYGZus{}Jones\PYGZus{}Industrial\PYGZus{}Average}\PYG{l+s+s1}{\PYGZsq{}}\PYG{p}{)}
\end{sphinxVerbatim}

\end{sphinxuseclass}\end{sphinxVerbatimInput}

\end{sphinxuseclass}
\begin{sphinxuseclass}{cell}\begin{sphinxVerbatimInput}

\begin{sphinxuseclass}{cell_input}
\begin{sphinxVerbatim}[commandchars=\\\{\}]
\PYG{n}{djia} \PYG{o}{=} \PYG{p}{(}
    \PYG{n}{yf}\PYG{o}{.}\PYG{n}{download}\PYG{p}{(}
        \PYG{n}{tickers}\PYG{o}{=}\PYG{n}{wiki}\PYG{p}{[}\PYG{l+m+mi}{1}\PYG{p}{]}\PYG{p}{[}\PYG{l+s+s1}{\PYGZsq{}}\PYG{l+s+s1}{Symbol}\PYG{l+s+s1}{\PYGZsq{}}\PYG{p}{]}\PYG{o}{.}\PYG{n}{to\PYGZus{}list}\PYG{p}{(}\PYG{p}{)}\PYG{p}{,}
        \PYG{n}{progress}\PYG{o}{=}\PYG{k+kc}{False}
    \PYG{p}{)}
    \PYG{o}{.}\PYG{n}{assign}\PYG{p}{(}\PYG{n}{Date} \PYG{o}{=} \PYG{k}{lambda} \PYG{n}{x}\PYG{p}{:} \PYG{n}{x}\PYG{o}{.}\PYG{n}{index}\PYG{o}{.}\PYG{n}{tz\PYGZus{}localize}\PYG{p}{(}\PYG{k+kc}{None}\PYG{p}{)}\PYG{p}{)}
    \PYG{o}{.}\PYG{n}{set\PYGZus{}index}\PYG{p}{(}\PYG{l+s+s1}{\PYGZsq{}}\PYG{l+s+s1}{Date}\PYG{l+s+s1}{\PYGZsq{}}\PYG{p}{)}
    \PYG{o}{.}\PYG{n}{rename\PYGZus{}axis}\PYG{p}{(}\PYG{n}{columns}\PYG{o}{=}\PYG{p}{[}\PYG{l+s+s1}{\PYGZsq{}}\PYG{l+s+s1}{Variable}\PYG{l+s+s1}{\PYGZsq{}}\PYG{p}{,} \PYG{l+s+s1}{\PYGZsq{}}\PYG{l+s+s1}{Ticker}\PYG{l+s+s1}{\PYGZsq{}}\PYG{p}{]}\PYG{p}{)}
    \PYG{p}{[}\PYG{l+s+s1}{\PYGZsq{}}\PYG{l+s+s1}{Adj Close}\PYG{l+s+s1}{\PYGZsq{}}\PYG{p}{]}
    \PYG{o}{.}\PYG{n}{pct\PYGZus{}change}\PYG{p}{(}\PYG{p}{)}
    \PYG{o}{.}\PYG{n}{dropna}\PYG{p}{(}\PYG{p}{)}
\PYG{p}{)}
\end{sphinxVerbatim}

\end{sphinxuseclass}\end{sphinxVerbatimInput}

\end{sphinxuseclass}
\begin{sphinxuseclass}{cell}\begin{sphinxVerbatimInput}

\begin{sphinxuseclass}{cell_input}
\begin{sphinxVerbatim}[commandchars=\\\{\}]
\PYG{n}{\PYGZus{}1} \PYG{o}{=} \PYG{n}{djia}\PYG{o}{.}\PYG{n}{iloc}\PYG{p}{[}\PYG{o}{\PYGZhy{}}\PYG{l+m+mi}{756}\PYG{p}{:}\PYG{p}{]}
\PYG{n}{\PYGZus{}2} \PYG{o}{=} \PYG{n}{\PYGZus{}1}\PYG{o}{.}\PYG{n}{agg}\PYG{p}{(}\PYG{p}{[}\PYG{n}{mean}\PYG{p}{,} \PYG{n}{beta}\PYG{p}{]}\PYG{p}{)}\PYG{o}{.}\PYG{n}{rename\PYGZus{}axis}\PYG{p}{(}\PYG{n}{index}\PYG{o}{=}\PYG{l+s+s1}{\PYGZsq{}}\PYG{l+s+s1}{Statistic}\PYG{l+s+s1}{\PYGZsq{}}\PYG{p}{)}\PYG{o}{.}\PYG{n}{transpose}\PYG{p}{(}\PYG{p}{)}

\PYG{n}{sns}\PYG{o}{.}\PYG{n}{regplot}\PYG{p}{(}\PYG{n}{x}\PYG{o}{=}\PYG{l+s+s1}{\PYGZsq{}}\PYG{l+s+s1}{beta}\PYG{l+s+s1}{\PYGZsq{}}\PYG{p}{,} \PYG{n}{y}\PYG{o}{=}\PYG{l+s+s1}{\PYGZsq{}}\PYG{l+s+s1}{mean}\PYG{l+s+s1}{\PYGZsq{}}\PYG{p}{,} \PYG{n}{data}\PYG{o}{=}\PYG{n}{\PYGZus{}2}\PYG{p}{)}
\PYG{k}{for} \PYG{n}{t}\PYG{p}{,} \PYG{p}{(}\PYG{n}{x}\PYG{p}{,} \PYG{n}{y}\PYG{p}{)} \PYG{o+ow}{in} \PYG{n}{\PYGZus{}2}\PYG{p}{[}\PYG{p}{[}\PYG{l+s+s1}{\PYGZsq{}}\PYG{l+s+s1}{beta}\PYG{l+s+s1}{\PYGZsq{}}\PYG{p}{,} \PYG{l+s+s1}{\PYGZsq{}}\PYG{l+s+s1}{mean}\PYG{l+s+s1}{\PYGZsq{}}\PYG{p}{]}\PYG{p}{]}\PYG{o}{.}\PYG{n}{iterrows}\PYG{p}{(}\PYG{p}{)}\PYG{p}{:}
    \PYG{n}{plt}\PYG{o}{.}\PYG{n}{annotate}\PYG{p}{(}\PYG{n}{text}\PYG{o}{=}\PYG{n}{t}\PYG{p}{,} \PYG{n}{xy}\PYG{o}{=}\PYG{p}{(}\PYG{n}{x}\PYG{p}{,} \PYG{n}{y}\PYG{p}{)}\PYG{p}{)} \PYG{c+c1}{\PYGZsh{} use a for\PYGZhy{}loop to add tickers}

\PYG{n}{plt}\PYG{o}{.}\PYG{n}{ylabel}\PYG{p}{(}\PYG{l+s+s1}{\PYGZsq{}}\PYG{l+s+s1}{Annualized Mean of Daily Returns (}\PYG{l+s+s1}{\PYGZpc{}}\PYG{l+s+s1}{)}\PYG{l+s+s1}{\PYGZsq{}}\PYG{p}{)}
\PYG{n}{plt}\PYG{o}{.}\PYG{n}{xlabel}\PYG{p}{(}\PYG{l+s+sa}{r}\PYG{l+s+s1}{\PYGZsq{}}\PYG{l+s+s1}{Capital Asset Pricing Model (CAPM) \PYGZdl{}}\PYG{l+s+s1}{\PYGZbs{}}\PYG{l+s+s1}{beta\PYGZdl{}}\PYG{l+s+s1}{\PYGZsq{}}\PYG{p}{)}
\PYG{n}{plt}\PYG{o}{.}\PYG{n}{title}\PYG{p}{(}
    \PYG{l+s+s1}{\PYGZsq{}}\PYG{l+s+s1}{Security Market Line (SML) for Dow\PYGZhy{}Jones Stocks}\PYG{l+s+se}{\PYGZbs{}n}\PYG{l+s+s1}{\PYGZsq{}} \PYG{o}{+}
    \PYG{l+s+s1}{\PYGZsq{}}\PYG{l+s+s1}{for Daily Returns from }\PYG{l+s+s1}{\PYGZsq{}} \PYG{o}{+} \PYG{n}{\PYGZus{}1}\PYG{o}{.}\PYG{n}{pipe}\PYG{p}{(}\PYG{n}{date\PYGZus{}range}\PYG{p}{)}\PYG{p}{)}
\PYG{n}{plt}\PYG{o}{.}\PYG{n}{show}\PYG{p}{(}\PYG{p}{)}
\end{sphinxVerbatim}

\end{sphinxuseclass}\end{sphinxVerbatimInput}
\begin{sphinxVerbatimOutput}

\begin{sphinxuseclass}{cell_output}
\noindent\sphinxincludegraphics{{06e3ad3ac2dec671848acc3550782bfdbdff169ebe62e70f0356fb4b6f862195}.png}

\end{sphinxuseclass}\end{sphinxVerbatimOutput}

\end{sphinxuseclass}
\sphinxAtStartPar
\sphinxstyleemphasis{\sphinxstylestrong{Note the SML is flatter and the fit is poorer for single stocks (even large ones) relative to ETFs.}}


\subsubsection{Plot the SML for the five portfolios formed on beta}
\label{\detokenize{herron_03_practice_03:plot-the-sml-for-the-five-portfolios-formed-on-beta}}
\sphinxAtStartPar
Download data for portfolios formed on \$\textbackslash{}beta\$ (\sphinxcode{\sphinxupquote{Portfolios\_Formed\_on\_BETA}}) from Ken French.
For the value\sphinxhyphen{}weighted portfolios, plot realized returns versus \$\textbackslash{}beta\$.
These data should elements \sphinxcode{\sphinxupquote{{[}2{]}}} and \sphinxcode{\sphinxupquote{{[}6{]}}}, respectively.

\begin{sphinxuseclass}{cell}\begin{sphinxVerbatimInput}

\begin{sphinxuseclass}{cell_input}
\begin{sphinxVerbatim}[commandchars=\\\{\}]
\PYG{n}{beta\PYGZus{}0} \PYG{o}{=} \PYG{n}{pdr}\PYG{o}{.}\PYG{n}{DataReader}\PYG{p}{(}
    \PYG{n}{name}\PYG{o}{=}\PYG{l+s+s1}{\PYGZsq{}}\PYG{l+s+s1}{Portfolios\PYGZus{}Formed\PYGZus{}on\PYGZus{}BETA}\PYG{l+s+s1}{\PYGZsq{}}\PYG{p}{,}
    \PYG{n}{data\PYGZus{}source}\PYG{o}{=}\PYG{l+s+s1}{\PYGZsq{}}\PYG{l+s+s1}{famafrench}\PYG{l+s+s1}{\PYGZsq{}}\PYG{p}{,}
    \PYG{n}{start}\PYG{o}{=}\PYG{l+s+s1}{\PYGZsq{}}\PYG{l+s+s1}{1900}\PYG{l+s+s1}{\PYGZsq{}}\PYG{p}{,}
    \PYG{n}{session}\PYG{o}{=}\PYG{n}{session}
\PYG{p}{)}

\PYG{n+nb}{print}\PYG{p}{(}\PYG{n}{beta\PYGZus{}0}\PYG{p}{[}\PYG{l+s+s1}{\PYGZsq{}}\PYG{l+s+s1}{DESCR}\PYG{l+s+s1}{\PYGZsq{}}\PYG{p}{]}\PYG{p}{)}
\end{sphinxVerbatim}

\end{sphinxuseclass}\end{sphinxVerbatimInput}
\begin{sphinxVerbatimOutput}

\begin{sphinxuseclass}{cell_output}
\begin{sphinxVerbatim}[commandchars=\\\{\}]
Portfolios Formed on BETA
\PYGZhy{}\PYGZhy{}\PYGZhy{}\PYGZhy{}\PYGZhy{}\PYGZhy{}\PYGZhy{}\PYGZhy{}\PYGZhy{}\PYGZhy{}\PYGZhy{}\PYGZhy{}\PYGZhy{}\PYGZhy{}\PYGZhy{}\PYGZhy{}\PYGZhy{}\PYGZhy{}\PYGZhy{}\PYGZhy{}\PYGZhy{}\PYGZhy{}\PYGZhy{}\PYGZhy{}\PYGZhy{}

This file was created by CMPT\PYGZus{}BETA\PYGZus{}RETS using the 202301 CRSP database. It contains value\PYGZhy{} and equal\PYGZhy{}weighted returns for portfolios formed on BETA. The portfolios are constructed at the end of June. Beta is estimated using monthly returns for the past 60 months (requiring at least 24 months with non\PYGZhy{}missing returns). Beta is estimated using the Scholes\PYGZhy{}Williams method. Annual returns are from January to December. Missing data are indicated by \PYGZhy{}99.99 or \PYGZhy{}999. The break points include utilities and include financials. The portfolios include utilities and include financials. Copyright 2023 Kenneth R. French

  0 : Value Weighted Returns \PYGZhy{}\PYGZhy{} Monthly (715 rows x 15 cols)
  1 : Equal Weighted Returns \PYGZhy{}\PYGZhy{} Monthly (715 rows x 15 cols)
  2 : Value Weighted Returns \PYGZhy{}\PYGZhy{} Annual from January to December (59 rows x 15 cols)
  3 : Equal Weighted Returns \PYGZhy{}\PYGZhy{} Annual from January to December (59 rows x 15 cols)
  4 : Number of Firms in Portfolios (715 rows x 15 cols)
  5 : Average Firm Size (715 rows x 15 cols)
  6 : Value\PYGZhy{}Weighted Average of Prior Beta (60 rows x 15 cols)
\end{sphinxVerbatim}

\end{sphinxuseclass}\end{sphinxVerbatimOutput}

\end{sphinxuseclass}
\begin{sphinxuseclass}{cell}\begin{sphinxVerbatimInput}

\begin{sphinxuseclass}{cell_input}
\begin{sphinxVerbatim}[commandchars=\\\{\}]
\PYG{n}{ports} \PYG{o}{=} \PYG{p}{[}\PYG{l+s+s1}{\PYGZsq{}}\PYG{l+s+s1}{Lo 20}\PYG{l+s+s1}{\PYGZsq{}}\PYG{p}{,} \PYG{l+s+s1}{\PYGZsq{}}\PYG{l+s+s1}{Qnt 2}\PYG{l+s+s1}{\PYGZsq{}}\PYG{p}{,} \PYG{l+s+s1}{\PYGZsq{}}\PYG{l+s+s1}{Qnt 3}\PYG{l+s+s1}{\PYGZsq{}}\PYG{p}{,} \PYG{l+s+s1}{\PYGZsq{}}\PYG{l+s+s1}{Qnt 4}\PYG{l+s+s1}{\PYGZsq{}}\PYG{p}{,} \PYG{l+s+s1}{\PYGZsq{}}\PYG{l+s+s1}{Hi 20}\PYG{l+s+s1}{\PYGZsq{}}\PYG{p}{]}

\PYG{n}{\PYGZus{}} \PYG{o}{=} \PYG{p}{(}
    \PYG{n}{pd}\PYG{o}{.}\PYG{n}{concat}\PYG{p}{(}
        \PYG{p}{[}
            \PYG{n}{beta\PYGZus{}0}\PYG{p}{[}\PYG{l+m+mi}{6}\PYG{p}{]}\PYG{p}{[}\PYG{n}{ports}\PYG{p}{]}\PYG{o}{.}\PYG{n}{shift}\PYG{p}{(}\PYG{p}{)}\PYG{p}{,} \PYG{c+c1}{\PYGZsh{} betas}
            \PYG{n}{beta\PYGZus{}0}\PYG{p}{[}\PYG{l+m+mi}{2}\PYG{p}{]}\PYG{p}{[}\PYG{n}{ports}\PYG{p}{]} \PYG{c+c1}{\PYGZsh{} value\PYGZhy{}weighted returns}
        \PYG{p}{]}\PYG{p}{,}
        \PYG{n}{axis}\PYG{o}{=}\PYG{l+m+mi}{1}\PYG{p}{,}
        \PYG{n}{keys}\PYG{o}{=}\PYG{p}{[}\PYG{l+s+s1}{\PYGZsq{}}\PYG{l+s+s1}{Beta}\PYG{l+s+s1}{\PYGZsq{}}\PYG{p}{,} \PYG{l+s+s1}{\PYGZsq{}}\PYG{l+s+s1}{Return}\PYG{l+s+s1}{\PYGZsq{}}\PYG{p}{]}\PYG{p}{,}
        \PYG{n}{names}\PYG{o}{=}\PYG{p}{[}\PYG{l+s+s1}{\PYGZsq{}}\PYG{l+s+s1}{Statistic}\PYG{l+s+s1}{\PYGZsq{}}\PYG{p}{,} \PYG{l+s+sa}{r}\PYG{l+s+s1}{\PYGZsq{}}\PYG{l+s+s1}{\PYGZdl{}}\PYG{l+s+s1}{\PYGZbs{}}\PYG{l+s+s1}{beta\PYGZdl{} Portfolio}\PYG{l+s+s1}{\PYGZsq{}}\PYG{p}{]}
    \PYG{p}{)}
    \PYG{o}{.}\PYG{n}{stack}\PYG{p}{(}\PYG{p}{)}
\PYG{p}{)}

\PYG{n}{sns}\PYG{o}{.}\PYG{n}{regplot}\PYG{p}{(}\PYG{n}{x}\PYG{o}{=}\PYG{l+s+s1}{\PYGZsq{}}\PYG{l+s+s1}{Beta}\PYG{l+s+s1}{\PYGZsq{}}\PYG{p}{,} \PYG{n}{y}\PYG{o}{=}\PYG{l+s+s1}{\PYGZsq{}}\PYG{l+s+s1}{Return}\PYG{l+s+s1}{\PYGZsq{}}\PYG{p}{,} \PYG{n}{data}\PYG{o}{=}\PYG{n}{\PYGZus{}}\PYG{p}{)}
\PYG{n}{plt}\PYG{o}{.}\PYG{n}{ylabel}\PYG{p}{(}\PYG{l+s+s1}{\PYGZsq{}}\PYG{l+s+s1}{Mean Annual Return (}\PYG{l+s+s1}{\PYGZpc{}}\PYG{l+s+s1}{)}\PYG{l+s+s1}{\PYGZsq{}}\PYG{p}{)}
\PYG{n}{plt}\PYG{o}{.}\PYG{n}{xlabel}\PYG{p}{(}\PYG{l+s+sa}{r}\PYG{l+s+s1}{\PYGZsq{}}\PYG{l+s+s1}{Capital Asset Pricing Model (CAPM) \PYGZdl{}}\PYG{l+s+s1}{\PYGZbs{}}\PYG{l+s+s1}{beta\PYGZdl{}}\PYG{l+s+s1}{\PYGZsq{}}\PYG{p}{)}
\PYG{n}{plt}\PYG{o}{.}\PYG{n}{title}\PYG{p}{(}
    \PYG{l+s+sa}{r}\PYG{l+s+s1}{\PYGZsq{}}\PYG{l+s+s1}{Security Market Line (SML) for \PYGZdl{}}\PYG{l+s+s1}{\PYGZbs{}}\PYG{l+s+s1}{beta\PYGZdl{} Portfolios}\PYG{l+s+s1}{\PYGZsq{}} \PYG{o}{+}
    \PYG{l+s+sa}{f}\PYG{l+s+s1}{\PYGZsq{}}\PYG{l+s+se}{\PYGZbs{}n}\PYG{l+s+s1}{for Annual Returns from }\PYG{l+s+si}{\PYGZob{}}\PYG{n}{\PYGZus{}}\PYG{o}{.}\PYG{n}{index}\PYG{o}{.}\PYG{n}{get\PYGZus{}level\PYGZus{}values}\PYG{p}{(}\PYG{l+m+mi}{0}\PYG{p}{)}\PYG{o}{.}\PYG{n}{year}\PYG{p}{[}\PYG{l+m+mi}{0}\PYG{p}{]}\PYG{l+s+si}{\PYGZcb{}}\PYG{l+s+s1}{ to }\PYG{l+s+si}{\PYGZob{}}\PYG{n}{\PYGZus{}}\PYG{o}{.}\PYG{n}{index}\PYG{o}{.}\PYG{n}{get\PYGZus{}level\PYGZus{}values}\PYG{p}{(}\PYG{l+m+mi}{0}\PYG{p}{)}\PYG{o}{.}\PYG{n}{year}\PYG{p}{[}\PYG{o}{\PYGZhy{}}\PYG{l+m+mi}{1}\PYG{p}{]}\PYG{l+s+si}{\PYGZcb{}}\PYG{l+s+s1}{\PYGZsq{}}
\PYG{p}{)}
\PYG{n}{plt}\PYG{o}{.}\PYG{n}{show}\PYG{p}{(}\PYG{p}{)}
\end{sphinxVerbatim}

\end{sphinxuseclass}\end{sphinxVerbatimInput}
\begin{sphinxVerbatimOutput}

\begin{sphinxuseclass}{cell_output}
\noindent\sphinxincludegraphics{{3cc2f6d6c6d11051afe30a1f1b241f1730895599f868e9aa8ac68a643b101066}.png}

\end{sphinxuseclass}\end{sphinxVerbatimOutput}

\end{sphinxuseclass}
\sphinxAtStartPar
With a wider range of stock \$\textbackslash{}beta\$s, we see the SML is \sphinxstyleemphasis{flat}!
Or at least flatter than we expect.


\subsubsection{Estimate the CAPM \$\textbackslash{}beta\$s on several levered and inverse exchange traded funds (ETFs)}
\label{\detokenize{herron_03_practice_03:estimate-the-capm-beta-s-on-several-levered-and-inverse-exchange-traded-funds-etfs}}
\sphinxAtStartPar
Try the following ETFs:
\begin{enumerate}
\sphinxsetlistlabels{\arabic}{enumi}{enumii}{}{.}%
\item {} 
\sphinxAtStartPar
SPY

\item {} 
\sphinxAtStartPar
UPRO

\item {} 
\sphinxAtStartPar
SPXU

\end{enumerate}

\sphinxAtStartPar
Can you determine what these products do from the data alone?
Estimate \$\textbackslash{}beta\$s and plot cumulative returns.
You may want to pick short periods of time with large market swings.

\begin{sphinxuseclass}{cell}\begin{sphinxVerbatimInput}

\begin{sphinxuseclass}{cell_input}
\begin{sphinxVerbatim}[commandchars=\\\{\}]
\PYG{n}{etf\PYGZus{}2} \PYG{o}{=} \PYG{p}{(}
    \PYG{n}{yf}\PYG{o}{.}\PYG{n}{download}\PYG{p}{(}
        \PYG{n}{tickers}\PYG{o}{=}\PYG{l+s+s1}{\PYGZsq{}}\PYG{l+s+s1}{SPY UPRO SPXU}\PYG{l+s+s1}{\PYGZsq{}}\PYG{p}{,}
        \PYG{n}{progress}\PYG{o}{=}\PYG{k+kc}{False}
    \PYG{p}{)}
    \PYG{o}{.}\PYG{n}{assign}\PYG{p}{(}\PYG{n}{Date} \PYG{o}{=} \PYG{k}{lambda} \PYG{n}{x}\PYG{p}{:} \PYG{n}{x}\PYG{o}{.}\PYG{n}{index}\PYG{o}{.}\PYG{n}{tz\PYGZus{}localize}\PYG{p}{(}\PYG{k+kc}{None}\PYG{p}{)}\PYG{p}{)}
    \PYG{o}{.}\PYG{n}{set\PYGZus{}index}\PYG{p}{(}\PYG{l+s+s1}{\PYGZsq{}}\PYG{l+s+s1}{Date}\PYG{l+s+s1}{\PYGZsq{}}\PYG{p}{)}
    \PYG{o}{.}\PYG{n}{rename\PYGZus{}axis}\PYG{p}{(}\PYG{n}{columns}\PYG{o}{=}\PYG{p}{[}\PYG{l+s+s1}{\PYGZsq{}}\PYG{l+s+s1}{Variable}\PYG{l+s+s1}{\PYGZsq{}}\PYG{p}{,} \PYG{l+s+s1}{\PYGZsq{}}\PYG{l+s+s1}{Ticker}\PYG{l+s+s1}{\PYGZsq{}}\PYG{p}{]}\PYG{p}{)}
    \PYG{p}{[}\PYG{l+s+s1}{\PYGZsq{}}\PYG{l+s+s1}{Adj Close}\PYG{l+s+s1}{\PYGZsq{}}\PYG{p}{]}
    \PYG{o}{.}\PYG{n}{pct\PYGZus{}change}\PYG{p}{(}\PYG{p}{)}
    \PYG{o}{.}\PYG{n}{dropna}\PYG{p}{(}\PYG{p}{)}
\PYG{p}{)}
\end{sphinxVerbatim}

\end{sphinxuseclass}\end{sphinxVerbatimInput}

\end{sphinxuseclass}
\begin{sphinxuseclass}{cell}\begin{sphinxVerbatimInput}

\begin{sphinxuseclass}{cell_input}
\begin{sphinxVerbatim}[commandchars=\\\{\}]
\PYG{n}{etf\PYGZus{}2}\PYG{o}{.}\PYG{n}{apply}\PYG{p}{(}\PYG{n}{beta}\PYG{p}{)}\PYG{o}{.}\PYG{n}{rename}\PYG{p}{(}\PYG{l+s+sa}{r}\PYG{l+s+s1}{\PYGZsq{}}\PYG{l+s+s1}{\PYGZdl{}}\PYG{l+s+s1}{\PYGZbs{}}\PYG{l+s+s1}{beta\PYGZdl{}}\PYG{l+s+s1}{\PYGZsq{}}\PYG{p}{)}
\end{sphinxVerbatim}

\end{sphinxuseclass}\end{sphinxVerbatimInput}
\begin{sphinxVerbatimOutput}

\begin{sphinxuseclass}{cell_output}
\begin{sphinxVerbatim}[commandchars=\\\{\}]
Ticker
SPXU   \PYGZhy{}2.8660
SPY     0.9586
UPRO    2.8814
Name: \PYGZdl{}\PYGZbs{}beta\PYGZdl{}, dtype: float64
\end{sphinxVerbatim}

\end{sphinxuseclass}\end{sphinxVerbatimOutput}

\end{sphinxuseclass}
\begin{sphinxuseclass}{cell}\begin{sphinxVerbatimInput}

\begin{sphinxuseclass}{cell_input}
\begin{sphinxVerbatim}[commandchars=\\\{\}]
\PYG{n}{etf\PYGZus{}2}\PYG{o}{.}\PYG{n}{apply}\PYG{p}{(}\PYG{n}{beta}\PYG{p}{)}\PYG{o}{.}\PYG{n}{plot}\PYG{p}{(}\PYG{n}{kind}\PYG{o}{=}\PYG{l+s+s1}{\PYGZsq{}}\PYG{l+s+s1}{bar}\PYG{l+s+s1}{\PYGZsq{}}\PYG{p}{)}
\PYG{n}{plt}\PYG{o}{.}\PYG{n}{xticks}\PYG{p}{(}\PYG{n}{rotation}\PYG{o}{=}\PYG{l+m+mi}{0}\PYG{p}{)}
\PYG{n}{plt}\PYG{o}{.}\PYG{n}{ylabel}\PYG{p}{(}\PYG{l+s+sa}{r}\PYG{l+s+s1}{\PYGZsq{}}\PYG{l+s+s1}{Capital Asset Pricing Model (CAPM) \PYGZdl{}}\PYG{l+s+s1}{\PYGZbs{}}\PYG{l+s+s1}{beta\PYGZdl{}}\PYG{l+s+s1}{\PYGZsq{}}\PYG{p}{)}
\PYG{n}{plt}\PYG{o}{.}\PYG{n}{title}\PYG{p}{(}
    \PYG{l+s+sa}{r}\PYG{l+s+s1}{\PYGZsq{}}\PYG{l+s+s1}{Capital Asset Pricing Model (CAPM) \PYGZdl{}}\PYG{l+s+s1}{\PYGZbs{}}\PYG{l+s+s1}{beta\PYGZdl{}s}\PYG{l+s+s1}{\PYGZsq{}} \PYG{o}{+} 
    \PYG{l+s+sa}{f}\PYG{l+s+s1}{\PYGZsq{}}\PYG{l+s+se}{\PYGZbs{}n}\PYG{l+s+s1}{from Daily Returns from }\PYG{l+s+si}{\PYGZob{}}\PYG{n}{etf\PYGZus{}2}\PYG{o}{.}\PYG{n}{pipe}\PYG{p}{(}\PYG{n}{date\PYGZus{}range}\PYG{p}{)}\PYG{l+s+si}{\PYGZcb{}}\PYG{l+s+s1}{\PYGZsq{}}
\PYG{p}{)}
\PYG{n}{plt}\PYG{o}{.}\PYG{n}{show}\PYG{p}{(}\PYG{p}{)}
\end{sphinxVerbatim}

\end{sphinxuseclass}\end{sphinxVerbatimInput}
\begin{sphinxVerbatimOutput}

\begin{sphinxuseclass}{cell_output}
\noindent\sphinxincludegraphics{{e704518cf36499e9d04b29af9257d6cb57f3a78cdd8fbda4550483071f8896d4}.png}

\end{sphinxuseclass}\end{sphinxVerbatimOutput}

\end{sphinxuseclass}
\sphinxAtStartPar
UPRO is a tripled\sphinxhyphen{}levered S\&P 500 ETF,
SPY is an S\&P 500 ETF,
and
SPXU is an \sphinxstyleemphasis{inverse} triple\sphinxhyphen{}levered S\&P 500 ETF.
More on levered and inverse ETFs \sphinxhref{https://www.schwab.com/learn/story/what-are-leveraged-inverse-etfs-etns-how-do-they-work}{here}.


\subsubsection{Explore the size factor}
\label{\detokenize{herron_03_practice_03:explore-the-size-factor}}

\paragraph{Estimate \$\textbackslash{}alpha\$s for the ten portfolios formed on size}
\label{\detokenize{herron_03_practice_03:estimate-alpha-s-for-the-ten-portfolios-formed-on-size}}
\sphinxAtStartPar
Academics started researching size\sphinxhyphen{}based portfolios in the early 1980s, so you may want to focus on the pre\sphinxhyphen{}1980 sample.

\begin{sphinxuseclass}{cell}\begin{sphinxVerbatimInput}

\begin{sphinxuseclass}{cell_input}
\begin{sphinxVerbatim}[commandchars=\\\{\}]
\PYG{n}{size\PYGZus{}0} \PYG{o}{=} \PYG{n}{pdr}\PYG{o}{.}\PYG{n}{DataReader}\PYG{p}{(}
    \PYG{n}{name}\PYG{o}{=}\PYG{l+s+s1}{\PYGZsq{}}\PYG{l+s+s1}{Portfolios\PYGZus{}Formed\PYGZus{}on\PYGZus{}ME}\PYG{l+s+s1}{\PYGZsq{}}\PYG{p}{,}
    \PYG{n}{data\PYGZus{}source}\PYG{o}{=}\PYG{l+s+s1}{\PYGZsq{}}\PYG{l+s+s1}{famafrench}\PYG{l+s+s1}{\PYGZsq{}}\PYG{p}{,}
    \PYG{n}{start}\PYG{o}{=}\PYG{l+s+s1}{\PYGZsq{}}\PYG{l+s+s1}{1900}\PYG{l+s+s1}{\PYGZsq{}}\PYG{p}{,}
    \PYG{n}{session}\PYG{o}{=}\PYG{n}{session}
\PYG{p}{)}

\PYG{n+nb}{print}\PYG{p}{(}\PYG{n}{size\PYGZus{}0}\PYG{p}{[}\PYG{l+s+s1}{\PYGZsq{}}\PYG{l+s+s1}{DESCR}\PYG{l+s+s1}{\PYGZsq{}}\PYG{p}{]}\PYG{p}{)}
\end{sphinxVerbatim}

\end{sphinxuseclass}\end{sphinxVerbatimInput}
\begin{sphinxVerbatimOutput}

\begin{sphinxuseclass}{cell_output}
\begin{sphinxVerbatim}[commandchars=\\\{\}]
Portfolios Formed on ME
\PYGZhy{}\PYGZhy{}\PYGZhy{}\PYGZhy{}\PYGZhy{}\PYGZhy{}\PYGZhy{}\PYGZhy{}\PYGZhy{}\PYGZhy{}\PYGZhy{}\PYGZhy{}\PYGZhy{}\PYGZhy{}\PYGZhy{}\PYGZhy{}\PYGZhy{}\PYGZhy{}\PYGZhy{}\PYGZhy{}\PYGZhy{}\PYGZhy{}\PYGZhy{}

This file was created by CMPT\PYGZus{}ME\PYGZus{}RETS using the 202301 CRSP database. It contains value\PYGZhy{} and equal\PYGZhy{}weighted returns for size portfolios. Each record contains returns for: Negative (not used) 30\PYGZpc{} 40\PYGZpc{} 30\PYGZpc{}   5 Quintiles  10 Deciles The portfolios are constructed at the end of Jun. The annual returns are from January to December. Missing data are indicated by \PYGZhy{}99.99 or \PYGZhy{}999. Copyright 2023 Kenneth R. French

  0 : Value Weight Returns \PYGZhy{}\PYGZhy{} Monthly (1159 rows x 19 cols)
  1 : Equal Weight Returns \PYGZhy{}\PYGZhy{} Monthly (1159 rows x 19 cols)
  2 : Value Weight Returns \PYGZhy{}\PYGZhy{} Annual from January to December (96 rows x 19 cols)
  3 : Equal Weight Returns \PYGZhy{}\PYGZhy{} Annual from January to December (96 rows x 19 cols)
  4 : Number of Firms in Portfolios (1159 rows x 19 cols)
  5 : Average Firm Size (1159 rows x 19 cols)
\end{sphinxVerbatim}

\end{sphinxuseclass}\end{sphinxVerbatimOutput}

\end{sphinxuseclass}
\begin{sphinxuseclass}{cell}\begin{sphinxVerbatimInput}

\begin{sphinxuseclass}{cell_input}
\begin{sphinxVerbatim}[commandchars=\\\{\}]
\PYG{n}{ff\PYGZus{}m} \PYG{o}{=} \PYG{n}{pdr}\PYG{o}{.}\PYG{n}{DataReader}\PYG{p}{(}
    \PYG{n}{name}\PYG{o}{=}\PYG{l+s+s1}{\PYGZsq{}}\PYG{l+s+s1}{F\PYGZhy{}F\PYGZus{}Research\PYGZus{}Data\PYGZus{}Factors}\PYG{l+s+s1}{\PYGZsq{}}\PYG{p}{,}
    \PYG{n}{data\PYGZus{}source}\PYG{o}{=}\PYG{l+s+s1}{\PYGZsq{}}\PYG{l+s+s1}{famafrench}\PYG{l+s+s1}{\PYGZsq{}}\PYG{p}{,}
    \PYG{n}{start}\PYG{o}{=}\PYG{l+s+s1}{\PYGZsq{}}\PYG{l+s+s1}{1900}\PYG{l+s+s1}{\PYGZsq{}}\PYG{p}{,}
    \PYG{n}{session}\PYG{o}{=}\PYG{n}{session}
\PYG{p}{)}
\end{sphinxVerbatim}

\end{sphinxuseclass}\end{sphinxVerbatimInput}

\end{sphinxuseclass}
\sphinxAtStartPar
\sphinxstyleemphasis{\sphinxstylestrong{In class, I used \sphinxcode{\sphinxupquote{size\_0{[}0{]}}} for value\sphinxhyphen{}weighted returns. To exaggerate the \$\textbackslash{}alpha\$ of small stock strategies, I will use equal\sphinxhyphen{}weighted returns here.}}
Equal\sphinxhyphen{}weighted portfolio assign higher weights to small stocks, and lower weights to large stocks.

\begin{sphinxuseclass}{cell}\begin{sphinxVerbatimInput}

\begin{sphinxuseclass}{cell_input}
\begin{sphinxVerbatim}[commandchars=\\\{\}]
\PYG{n}{ports} \PYG{o}{=} \PYG{p}{[}\PYG{l+s+s1}{\PYGZsq{}}\PYG{l+s+s1}{Lo 10}\PYG{l+s+s1}{\PYGZsq{}}\PYG{p}{,} \PYG{l+s+s1}{\PYGZsq{}}\PYG{l+s+s1}{Dec 2}\PYG{l+s+s1}{\PYGZsq{}}\PYG{p}{,} \PYG{l+s+s1}{\PYGZsq{}}\PYG{l+s+s1}{Dec 3}\PYG{l+s+s1}{\PYGZsq{}}\PYG{p}{,} \PYG{l+s+s1}{\PYGZsq{}}\PYG{l+s+s1}{Dec 4}\PYG{l+s+s1}{\PYGZsq{}}\PYG{p}{,} \PYG{l+s+s1}{\PYGZsq{}}\PYG{l+s+s1}{Dec 5}\PYG{l+s+s1}{\PYGZsq{}}\PYG{p}{,} \PYG{l+s+s1}{\PYGZsq{}}\PYG{l+s+s1}{Dec 6}\PYG{l+s+s1}{\PYGZsq{}}\PYG{p}{,} \PYG{l+s+s1}{\PYGZsq{}}\PYG{l+s+s1}{Dec 7}\PYG{l+s+s1}{\PYGZsq{}}\PYG{p}{,} \PYG{l+s+s1}{\PYGZsq{}}\PYG{l+s+s1}{Dec 8}\PYG{l+s+s1}{\PYGZsq{}}\PYG{p}{,} \PYG{l+s+s1}{\PYGZsq{}}\PYG{l+s+s1}{Dec 9}\PYG{l+s+s1}{\PYGZsq{}}\PYG{p}{,} \PYG{l+s+s1}{\PYGZsq{}}\PYG{l+s+s1}{Hi 10}\PYG{l+s+s1}{\PYGZsq{}}\PYG{p}{]}
\PYG{n}{joined} \PYG{o}{=} \PYG{n}{size\PYGZus{}0}\PYG{p}{[}\PYG{l+m+mi}{1}\PYG{p}{]}\PYG{p}{[}\PYG{n}{ports}\PYG{p}{]}\PYG{o}{.}\PYG{n}{join}\PYG{p}{(}\PYG{n}{ff\PYGZus{}m}\PYG{p}{[}\PYG{l+m+mi}{0}\PYG{p}{]}\PYG{p}{)}
\PYG{n}{joined}\PYG{o}{.}\PYG{n}{head}\PYG{p}{(}\PYG{p}{)}
\end{sphinxVerbatim}

\end{sphinxuseclass}\end{sphinxVerbatimInput}
\begin{sphinxVerbatimOutput}

\begin{sphinxuseclass}{cell_output}
\begin{sphinxVerbatim}[commandchars=\\\{\}]
          Lo 10   Dec 2   Dec 3   Dec 4   Dec 5   Dec 6   Dec 7   Dec 8  \PYGZbs{}
Date                                                                      
1926\PYGZhy{}07 \PYGZhy{}1.4200  0.2900 \PYGZhy{}0.1500  0.8800  1.4500  1.8500  1.6300  1.3800   
1926\PYGZhy{}08  4.6100  2.5900  4.0300  3.2400  2.6600  4.6700  1.5400  1.6300   
1926\PYGZhy{}09  0.9100 \PYGZhy{}1.8700 \PYGZhy{}2.2700 \PYGZhy{}0.8400  0.1200 \PYGZhy{}0.0700 \PYGZhy{}1.5800  0.6400   
1926\PYGZhy{}10 \PYGZhy{}4.7200 \PYGZhy{}1.7700 \PYGZhy{}3.3600 \PYGZhy{}5.0100 \PYGZhy{}3.0900 \PYGZhy{}2.7100 \PYGZhy{}3.4500 \PYGZhy{}3.2700   
1926\PYGZhy{}11 \PYGZhy{}0.7700 \PYGZhy{}0.3200 \PYGZhy{}0.2900  4.7900  3.1700  3.5800  3.8000  2.9500   

          Dec 9   Hi 10  Mkt\PYGZhy{}RF     SMB     HML     RF  
Date                                                    
1926\PYGZhy{}07  3.3800  3.2900  2.9600 \PYGZhy{}2.5600 \PYGZhy{}2.4300 0.2200  
1926\PYGZhy{}08  0.9800  3.7000  2.6400 \PYGZhy{}1.1700  3.8200 0.2500  
1926\PYGZhy{}09 \PYGZhy{}0.8600  0.6700  0.3600 \PYGZhy{}1.4000  0.1300 0.2300  
1926\PYGZhy{}10 \PYGZhy{}3.4700 \PYGZhy{}2.4300 \PYGZhy{}3.2400 \PYGZhy{}0.0900  0.7000 0.3200  
1926\PYGZhy{}11  3.6100  2.7000  2.5300 \PYGZhy{}0.1000 \PYGZhy{}0.5100 0.3100  
\end{sphinxVerbatim}

\end{sphinxuseclass}\end{sphinxVerbatimOutput}

\end{sphinxuseclass}
\begin{sphinxuseclass}{cell}\begin{sphinxVerbatimInput}

\begin{sphinxuseclass}{cell_input}
\begin{sphinxVerbatim}[commandchars=\\\{\}]
\PYG{n}{models} \PYG{o}{=} \PYG{p}{[}\PYG{n}{smf}\PYG{o}{.}\PYG{n}{ols}\PYG{p}{(}\PYG{l+s+sa}{f}\PYG{l+s+s1}{\PYGZsq{}}\PYG{l+s+s1}{I(Q(}\PYG{l+s+s1}{\PYGZdq{}}\PYG{l+s+si}{\PYGZob{}}\PYG{n}{p}\PYG{l+s+si}{\PYGZcb{}}\PYG{l+s+s1}{\PYGZdq{}}\PYG{l+s+s1}{) \PYGZhy{} RF) \PYGZti{} Q(}\PYG{l+s+s1}{\PYGZdq{}}\PYG{l+s+s1}{Mkt\PYGZhy{}RF}\PYG{l+s+s1}{\PYGZdq{}}\PYG{l+s+s1}{)}\PYG{l+s+s1}{\PYGZsq{}}\PYG{p}{,} \PYG{n}{data}\PYG{o}{=}\PYG{n}{joined}\PYG{o}{.}\PYG{n}{loc}\PYG{p}{[}\PYG{p}{:}\PYG{l+s+s1}{\PYGZsq{}}\PYG{l+s+s1}{1979}\PYG{l+s+s1}{\PYGZsq{}}\PYG{p}{]}\PYG{p}{)} \PYG{k}{for} \PYG{n}{p} \PYG{o+ow}{in} \PYG{n}{ports}\PYG{p}{]}
\PYG{n}{fits} \PYG{o}{=} \PYG{p}{[}\PYG{n}{m}\PYG{o}{.}\PYG{n}{fit}\PYG{p}{(}\PYG{p}{)} \PYG{k}{for} \PYG{n}{m} \PYG{o+ow}{in} \PYG{n}{models}\PYG{p}{]}

\PYG{n}{coefs} \PYG{o}{=} \PYG{p}{(}
    \PYG{n}{pd}\PYG{o}{.}\PYG{n}{concat}\PYG{p}{(}
        \PYG{n}{objs}\PYG{o}{=}\PYG{p}{[}\PYG{n}{f}\PYG{o}{.}\PYG{n}{params} \PYG{k}{for} \PYG{n}{f} \PYG{o+ow}{in} \PYG{n}{fits}\PYG{p}{]}\PYG{p}{,}
        \PYG{n}{axis}\PYG{o}{=}\PYG{l+m+mi}{1}\PYG{p}{,}
        \PYG{n}{keys}\PYG{o}{=}\PYG{n}{ports}
    \PYG{p}{)}
    \PYG{o}{.}\PYG{n}{rename\PYGZus{}axis}\PYG{p}{(}\PYG{n}{index}\PYG{o}{=}\PYG{l+s+s1}{\PYGZsq{}}\PYG{l+s+s1}{Coefficient}\PYG{l+s+s1}{\PYGZsq{}}\PYG{p}{,} \PYG{n}{columns}\PYG{o}{=}\PYG{l+s+s1}{\PYGZsq{}}\PYG{l+s+s1}{Size Portfolio}\PYG{l+s+s1}{\PYGZsq{}}\PYG{p}{)}
    \PYG{o}{.}\PYG{n}{transpose}\PYG{p}{(}\PYG{p}{)}
\PYG{p}{)}

\PYG{n}{ses} \PYG{o}{=} \PYG{p}{[}\PYG{n}{f}\PYG{o}{.}\PYG{n}{bse}\PYG{p}{[}\PYG{l+m+mi}{0}\PYG{p}{]} \PYG{k}{for} \PYG{n}{f} \PYG{o+ow}{in} \PYG{n}{fits}\PYG{p}{]}
\end{sphinxVerbatim}

\end{sphinxuseclass}\end{sphinxVerbatimInput}

\end{sphinxuseclass}
\sphinxAtStartPar
We can get the standard errors, too.
The standard errors are in the \sphinxcode{\sphinxupquote{.params}} attribute of our model fits.

\begin{sphinxuseclass}{cell}\begin{sphinxVerbatimInput}

\begin{sphinxuseclass}{cell_input}
\begin{sphinxVerbatim}[commandchars=\\\{\}]
\PYG{n}{ses} \PYG{o}{=} \PYG{p}{[}\PYG{n}{f}\PYG{o}{.}\PYG{n}{bse}\PYG{p}{[}\PYG{l+m+mi}{0}\PYG{p}{]} \PYG{k}{for} \PYG{n}{f} \PYG{o+ow}{in} \PYG{n}{fits}\PYG{p}{]}
\end{sphinxVerbatim}

\end{sphinxuseclass}\end{sphinxVerbatimInput}

\end{sphinxuseclass}
\begin{sphinxuseclass}{cell}\begin{sphinxVerbatimInput}

\begin{sphinxuseclass}{cell_input}
\begin{sphinxVerbatim}[commandchars=\\\{\}]
\PYG{n}{coefs}\PYG{p}{[}\PYG{l+s+s1}{\PYGZsq{}}\PYG{l+s+s1}{Intercept}\PYG{l+s+s1}{\PYGZsq{}}\PYG{p}{]}\PYG{o}{.}\PYG{n}{plot}\PYG{p}{(}\PYG{n}{kind}\PYG{o}{=}\PYG{l+s+s1}{\PYGZsq{}}\PYG{l+s+s1}{bar}\PYG{l+s+s1}{\PYGZsq{}}\PYG{p}{,} \PYG{n}{yerr}\PYG{o}{=}\PYG{n}{ses}\PYG{p}{)}
\PYG{n}{plt}\PYG{o}{.}\PYG{n}{ylabel}\PYG{p}{(}\PYG{l+s+sa}{r}\PYG{l+s+s1}{\PYGZsq{}}\PYG{l+s+s1}{Monthly \PYGZdl{}}\PYG{l+s+s1}{\PYGZbs{}}\PYG{l+s+s1}{alpha\PYGZdl{} (}\PYG{l+s+s1}{\PYGZpc{}}\PYG{l+s+s1}{) from CAPM}\PYG{l+s+s1}{\PYGZsq{}}\PYG{p}{)}
\PYG{n}{plt}\PYG{o}{.}\PYG{n}{xticks}\PYG{p}{(}\PYG{n}{rotation}\PYG{o}{=}\PYG{l+m+mi}{0}\PYG{p}{)}
\PYG{n}{plt}\PYG{o}{.}\PYG{n}{title}\PYG{p}{(}
    \PYG{l+s+sa}{r}\PYG{l+s+s1}{\PYGZsq{}}\PYG{l+s+s1}{Size Portfolio CAPM \PYGZdl{}}\PYG{l+s+s1}{\PYGZbs{}}\PYG{l+s+s1}{alpha\PYGZdl{}s for Monthly Returns}\PYG{l+s+s1}{\PYGZsq{}} \PYG{o}{+}
    \PYG{l+s+s1}{\PYGZsq{}}\PYG{l+s+se}{\PYGZbs{}n}\PYG{l+s+s1}{from July 1926 through December 1979}\PYG{l+s+s1}{\PYGZsq{}}
\PYG{p}{)}
\PYG{n}{plt}\PYG{o}{.}\PYG{n}{show}\PYG{p}{(}\PYG{p}{)}
\end{sphinxVerbatim}

\end{sphinxuseclass}\end{sphinxVerbatimInput}
\begin{sphinxVerbatimOutput}

\begin{sphinxuseclass}{cell_output}
\noindent\sphinxincludegraphics{{589340b5d177c04f34cbd2ca1bb456a1d2dc60aa82c69e38a6d75ab46b5c3e06}.png}

\end{sphinxuseclass}\end{sphinxVerbatimOutput}

\end{sphinxuseclass}
\sphinxAtStartPar
The size effect (i.e., the CAPM \$\textbackslash{}alpha\$ for small stock portfolios) appears large!
We will dig a little deeper!


\paragraph{Are the returns on these ten portfolios formed on size concentrated in a specific month?}
\label{\detokenize{herron_03_practice_03:are-the-returns-on-these-ten-portfolios-formed-on-size-concentrated-in-a-specific-month}}
\begin{sphinxuseclass}{cell}\begin{sphinxVerbatimInput}

\begin{sphinxuseclass}{cell_input}
\begin{sphinxVerbatim}[commandchars=\\\{\}]
\PYG{p}{(}
    \PYG{n}{size\PYGZus{}0}\PYG{p}{[}\PYG{l+m+mi}{0}\PYG{p}{]}\PYG{p}{[}\PYG{n}{ports}\PYG{p}{]}
    \PYG{o}{.}\PYG{n}{groupby}\PYG{p}{(}\PYG{k}{lambda} \PYG{n}{x}\PYG{p}{:} \PYG{n}{np}\PYG{o}{.}\PYG{n}{where}\PYG{p}{(}\PYG{n}{x}\PYG{o}{.}\PYG{n}{month}\PYG{o}{==}\PYG{l+m+mi}{1}\PYG{p}{,} \PYG{l+s+s1}{\PYGZsq{}}\PYG{l+s+s1}{January}\PYG{l+s+s1}{\PYGZsq{}}\PYG{p}{,} \PYG{l+s+s1}{\PYGZsq{}}\PYG{l+s+s1}{Not January}\PYG{l+s+s1}{\PYGZsq{}}\PYG{p}{)}\PYG{p}{)}
    \PYG{o}{.}\PYG{n}{mean}\PYG{p}{(}\PYG{p}{)}
    \PYG{o}{.}\PYG{n}{rename\PYGZus{}axis}\PYG{p}{(}\PYG{n}{index}\PYG{o}{=}\PYG{l+s+s1}{\PYGZsq{}}\PYG{l+s+s1}{Month}\PYG{l+s+s1}{\PYGZsq{}}\PYG{p}{,} \PYG{n}{columns}\PYG{o}{=}\PYG{l+s+s1}{\PYGZsq{}}\PYG{l+s+s1}{Equal\PYGZhy{}Weighted Size Portfolio}\PYG{l+s+s1}{\PYGZsq{}}\PYG{p}{)}
    \PYG{o}{.}\PYG{n}{plot}\PYG{p}{(}\PYG{n}{kind}\PYG{o}{=}\PYG{l+s+s1}{\PYGZsq{}}\PYG{l+s+s1}{bar}\PYG{l+s+s1}{\PYGZsq{}}\PYG{p}{)}
\PYG{p}{)}

\PYG{n}{plt}\PYG{o}{.}\PYG{n}{xticks}\PYG{p}{(}\PYG{n}{rotation}\PYG{o}{=}\PYG{l+m+mi}{0}\PYG{p}{)}
\PYG{n}{plt}\PYG{o}{.}\PYG{n}{ylabel}\PYG{p}{(}\PYG{l+s+s1}{\PYGZsq{}}\PYG{l+s+s1}{Mean Monthly Return (}\PYG{l+s+s1}{\PYGZpc{}}\PYG{l+s+s1}{)}\PYG{l+s+s1}{\PYGZsq{}}\PYG{p}{)}
\PYG{n}{plt}\PYG{o}{.}\PYG{n}{title}\PYG{p}{(}\PYG{l+s+s1}{\PYGZsq{}}\PYG{l+s+s1}{When Do We Earn Size\PYGZhy{}Effect Returns?}\PYG{l+s+s1}{\PYGZsq{}}\PYG{p}{)}
\PYG{n}{plt}\PYG{o}{.}\PYG{n}{show}\PYG{p}{(}\PYG{p}{)}
\end{sphinxVerbatim}

\end{sphinxuseclass}\end{sphinxVerbatimInput}
\begin{sphinxVerbatimOutput}

\begin{sphinxuseclass}{cell_output}
\noindent\sphinxincludegraphics{{58f7dd3f2070232a24cd5991e72f272cbbc4fcd99b2f1ffac6a982c4837dd184}.png}

\end{sphinxuseclass}\end{sphinxVerbatimOutput}

\end{sphinxuseclass}
\sphinxAtStartPar
We earn size effect returns in January!
The size effect is likely due to tax\sphinxhyphen{}loss harvesting in small stocks.


\paragraph{Compare the size factor to the market factor}
\label{\detokenize{herron_03_practice_03:compare-the-size-factor-to-the-market-factor}}
\sphinxAtStartPar
You may want to consider mean excess returns by decade.

\begin{sphinxuseclass}{cell}\begin{sphinxVerbatimInput}

\begin{sphinxuseclass}{cell_input}
\begin{sphinxVerbatim}[commandchars=\\\{\}]
\PYG{p}{(}
    \PYG{n}{ff\PYGZus{}m}\PYG{p}{[}\PYG{l+m+mi}{0}\PYG{p}{]}\PYG{p}{[}\PYG{p}{[}\PYG{l+s+s1}{\PYGZsq{}}\PYG{l+s+s1}{Mkt\PYGZhy{}RF}\PYG{l+s+s1}{\PYGZsq{}}\PYG{p}{,} \PYG{l+s+s1}{\PYGZsq{}}\PYG{l+s+s1}{SMB}\PYG{l+s+s1}{\PYGZsq{}}\PYG{p}{]}\PYG{p}{]}
    \PYG{o}{.}\PYG{n}{resample}\PYG{p}{(}\PYG{l+s+s1}{\PYGZsq{}}\PYG{l+s+s1}{10Y}\PYG{l+s+s1}{\PYGZsq{}}\PYG{p}{)}
    \PYG{o}{.}\PYG{n}{mean}\PYG{p}{(}\PYG{p}{)}
    \PYG{o}{.}\PYG{n}{mul}\PYG{p}{(}\PYG{l+m+mi}{12}\PYG{p}{)}
    \PYG{o}{.}\PYG{n}{rename\PYGZus{}axis}\PYG{p}{(}\PYG{n}{index}\PYG{o}{=}\PYG{l+s+s1}{\PYGZsq{}}\PYG{l+s+s1}{10\PYGZhy{}Year Period}\PYG{l+s+s1}{\PYGZsq{}}\PYG{p}{,} \PYG{n}{columns}\PYG{o}{=}\PYG{l+s+s1}{\PYGZsq{}}\PYG{l+s+s1}{Factor}\PYG{l+s+s1}{\PYGZsq{}}\PYG{p}{)}
    \PYG{o}{.}\PYG{n}{plot}\PYG{p}{(}\PYG{n}{kind}\PYG{o}{=}\PYG{l+s+s1}{\PYGZsq{}}\PYG{l+s+s1}{bar}\PYG{l+s+s1}{\PYGZsq{}}\PYG{p}{)}
\PYG{p}{)}

\PYG{n}{plt}\PYG{o}{.}\PYG{n}{xticks}\PYG{p}{(}\PYG{n}{rotation}\PYG{o}{=}\PYG{l+m+mi}{0}\PYG{p}{)}
\PYG{n}{plt}\PYG{o}{.}\PYG{n}{ylabel}\PYG{p}{(}\PYG{l+s+s1}{\PYGZsq{}}\PYG{l+s+s1}{Annualize Mean of Monthly Returns (}\PYG{l+s+s1}{\PYGZpc{}}\PYG{l+s+s1}{)}\PYG{l+s+s1}{\PYGZsq{}}\PYG{p}{)}
\PYG{n}{plt}\PYG{o}{.}\PYG{n}{title}\PYG{p}{(}\PYG{l+s+s1}{\PYGZsq{}}\PYG{l+s+s1}{Comparison on Market Risk and Small Stock Premia}\PYG{l+s+s1}{\PYGZsq{}}\PYG{p}{)}
\PYG{n}{plt}\PYG{o}{.}\PYG{n}{show}\PYG{p}{(}\PYG{p}{)}
\end{sphinxVerbatim}

\end{sphinxuseclass}\end{sphinxVerbatimInput}
\begin{sphinxVerbatimOutput}

\begin{sphinxuseclass}{cell_output}
\noindent\sphinxincludegraphics{{5f3edf7217ff43bbb287d6bd9935dc56919398a42681701ac07da0dcc08a4d8a}.png}

\end{sphinxuseclass}\end{sphinxVerbatimOutput}

\end{sphinxuseclass}
\sphinxAtStartPar
The size factor (SMB) may have \$\textbackslash{}alpha\$ early in the sample, but it rarely generates outsize returns.
Plus, the size factor has returned (effectively) zero the last two decades of the sample (2006\sphinxhyphen{}2015 and 2016\sphinxhyphen{}today).


\subsubsection{Repeat the exercises above with the value factor}
\label{\detokenize{herron_03_practice_03:repeat-the-exercises-above-with-the-value-factor}}
\begin{sphinxuseclass}{cell}\begin{sphinxVerbatimInput}

\begin{sphinxuseclass}{cell_input}
\begin{sphinxVerbatim}[commandchars=\\\{\}]
\PYG{n}{hml\PYGZus{}0} \PYG{o}{=} \PYG{n}{pdr}\PYG{o}{.}\PYG{n}{DataReader}\PYG{p}{(}
    \PYG{n}{name}\PYG{o}{=}\PYG{l+s+s1}{\PYGZsq{}}\PYG{l+s+s1}{Portfolios\PYGZus{}Formed\PYGZus{}on\PYGZus{}BE\PYGZhy{}ME}\PYG{l+s+s1}{\PYGZsq{}}\PYG{p}{,}
    \PYG{n}{data\PYGZus{}source}\PYG{o}{=}\PYG{l+s+s1}{\PYGZsq{}}\PYG{l+s+s1}{famafrench}\PYG{l+s+s1}{\PYGZsq{}}\PYG{p}{,}
    \PYG{n}{start}\PYG{o}{=}\PYG{l+s+s1}{\PYGZsq{}}\PYG{l+s+s1}{1900}\PYG{l+s+s1}{\PYGZsq{}}\PYG{p}{,}
    \PYG{n}{session}\PYG{o}{=}\PYG{n}{session}
\PYG{p}{)}

\PYG{n+nb}{print}\PYG{p}{(}\PYG{n}{hml\PYGZus{}0}\PYG{p}{[}\PYG{l+s+s1}{\PYGZsq{}}\PYG{l+s+s1}{DESCR}\PYG{l+s+s1}{\PYGZsq{}}\PYG{p}{]}\PYG{p}{)}
\end{sphinxVerbatim}

\end{sphinxuseclass}\end{sphinxVerbatimInput}
\begin{sphinxVerbatimOutput}

\begin{sphinxuseclass}{cell_output}
\begin{sphinxVerbatim}[commandchars=\\\{\}]
Portfolios Formed on BE\PYGZhy{}ME
\PYGZhy{}\PYGZhy{}\PYGZhy{}\PYGZhy{}\PYGZhy{}\PYGZhy{}\PYGZhy{}\PYGZhy{}\PYGZhy{}\PYGZhy{}\PYGZhy{}\PYGZhy{}\PYGZhy{}\PYGZhy{}\PYGZhy{}\PYGZhy{}\PYGZhy{}\PYGZhy{}\PYGZhy{}\PYGZhy{}\PYGZhy{}\PYGZhy{}\PYGZhy{}\PYGZhy{}\PYGZhy{}\PYGZhy{}

This file was created by CMPT\PYGZus{}BEME\PYGZus{}RETS using the 202301 CRSP database. It contains value\PYGZhy{} and equal\PYGZhy{}weighted returns for portfolios formed on BE/ME. The portfolios are constructed at the end of June. BE/ME is book equity at the last fiscal year end of the prior calendar year divided by ME at the end of December of the prior year. The annual returns are from January to December. Missing data are indicated by \PYGZhy{}99.99 or \PYGZhy{}999. The break points use Compustat firms plus the firms hand\PYGZhy{}collected from the Moodys Industrial, Utilities, Transportation, and Financial Manuals. The portfolios use Compustat firms plus the firms hand\PYGZhy{}collected from the Moodys Industrial, Utilities, Transportation, and Financial Manuals. The break points include utilities and include financials. The portfolios include utilities and include financials. Copyright 2023 Kenneth R. French

  0 : Value Weight Returns \PYGZhy{}\PYGZhy{} Monthly (1159 rows x 19 cols)
  1 : Equal Weight Returns \PYGZhy{}\PYGZhy{} Monthly (1159 rows x 19 cols)
  2 : Value Weight Returns \PYGZhy{}\PYGZhy{} Annual from January to December (96 rows x 19 cols)
  3 : Equal Weight Returns \PYGZhy{}\PYGZhy{} Annual from January to December (96 rows x 19 cols)
  4 : Number of Firms in Portfolios (1159 rows x 19 cols)
  5 : Average Firm Size (1159 rows x 19 cols)
  6 : Sum of BE / Sum of ME (97 rows x 19 cols)
  7 : Value Weight Average of BE / ME (97 rows x 19 cols)
\end{sphinxVerbatim}

\end{sphinxuseclass}\end{sphinxVerbatimOutput}

\end{sphinxuseclass}
\begin{sphinxuseclass}{cell}\begin{sphinxVerbatimInput}

\begin{sphinxuseclass}{cell_input}
\begin{sphinxVerbatim}[commandchars=\\\{\}]
\PYG{n}{ports} \PYG{o}{=} \PYG{p}{[}\PYG{l+s+s1}{\PYGZsq{}}\PYG{l+s+s1}{Lo 10}\PYG{l+s+s1}{\PYGZsq{}}\PYG{p}{,} \PYG{l+s+s1}{\PYGZsq{}}\PYG{l+s+s1}{Dec 2}\PYG{l+s+s1}{\PYGZsq{}}\PYG{p}{,} \PYG{l+s+s1}{\PYGZsq{}}\PYG{l+s+s1}{Dec 3}\PYG{l+s+s1}{\PYGZsq{}}\PYG{p}{,} \PYG{l+s+s1}{\PYGZsq{}}\PYG{l+s+s1}{Dec 4}\PYG{l+s+s1}{\PYGZsq{}}\PYG{p}{,} \PYG{l+s+s1}{\PYGZsq{}}\PYG{l+s+s1}{Dec 5}\PYG{l+s+s1}{\PYGZsq{}}\PYG{p}{,} \PYG{l+s+s1}{\PYGZsq{}}\PYG{l+s+s1}{Dec 6}\PYG{l+s+s1}{\PYGZsq{}}\PYG{p}{,} \PYG{l+s+s1}{\PYGZsq{}}\PYG{l+s+s1}{Dec 7}\PYG{l+s+s1}{\PYGZsq{}}\PYG{p}{,} \PYG{l+s+s1}{\PYGZsq{}}\PYG{l+s+s1}{Dec 8}\PYG{l+s+s1}{\PYGZsq{}}\PYG{p}{,} \PYG{l+s+s1}{\PYGZsq{}}\PYG{l+s+s1}{Dec 9}\PYG{l+s+s1}{\PYGZsq{}}\PYG{p}{,} \PYG{l+s+s1}{\PYGZsq{}}\PYG{l+s+s1}{Hi 10}\PYG{l+s+s1}{\PYGZsq{}}\PYG{p}{]}
\PYG{n}{joined} \PYG{o}{=} \PYG{n}{hml\PYGZus{}0}\PYG{p}{[}\PYG{l+m+mi}{1}\PYG{p}{]}\PYG{p}{[}\PYG{n}{ports}\PYG{p}{]}\PYG{o}{.}\PYG{n}{join}\PYG{p}{(}\PYG{n}{ff\PYGZus{}m}\PYG{p}{[}\PYG{l+m+mi}{0}\PYG{p}{]}\PYG{p}{)}
\PYG{n}{joined}\PYG{o}{.}\PYG{n}{head}\PYG{p}{(}\PYG{p}{)}
\end{sphinxVerbatim}

\end{sphinxuseclass}\end{sphinxVerbatimInput}
\begin{sphinxVerbatimOutput}

\begin{sphinxuseclass}{cell_output}
\begin{sphinxVerbatim}[commandchars=\\\{\}]
          Lo 10   Dec 2   Dec 3   Dec 4   Dec 5   Dec 6   Dec 7   Dec 8  \PYGZbs{}
Date                                                                      
1926\PYGZhy{}07  4.5200  1.5300  1.9900  2.0700  0.0700  1.9600  0.2300  2.8800   
1926\PYGZhy{}08  0.3400  0.5100  3.0000  0.8100  2.7900  1.6600  0.9600  4.4900   
1926\PYGZhy{}09  0.3600 \PYGZhy{}4.2500 \PYGZhy{}1.1600  0.8800 \PYGZhy{}0.7400 \PYGZhy{}1.1000 \PYGZhy{}0.5000 \PYGZhy{}1.4500   
1926\PYGZhy{}10 \PYGZhy{}4.1200 \PYGZhy{}3.8300 \PYGZhy{}2.6700 \PYGZhy{}2.7800 \PYGZhy{}2.3100 \PYGZhy{}3.9800 \PYGZhy{}4.1800 \PYGZhy{}2.0100   
1926\PYGZhy{}11  4.4400  0.4700  1.3500  4.4500  3.5600  1.4700  3.5100  3.5100   

          Dec 9   Hi 10  Mkt\PYGZhy{}RF     SMB     HML     RF  
Date                                                    
1926\PYGZhy{}07 \PYGZhy{}1.4400 \PYGZhy{}0.1700  2.9600 \PYGZhy{}2.5600 \PYGZhy{}2.4300 0.2200  
1926\PYGZhy{}08  5.8100  6.4800  2.6400 \PYGZhy{}1.1700  3.8200 0.2500  
1926\PYGZhy{}09 \PYGZhy{}2.1700  3.9300  0.3600 \PYGZhy{}1.4000  0.1300 0.2300  
1926\PYGZhy{}10 \PYGZhy{}4.5800 \PYGZhy{}1.8800 \PYGZhy{}3.2400 \PYGZhy{}0.0900  0.7000 0.3200  
1926\PYGZhy{}11  2.2300  2.6600  2.5300 \PYGZhy{}0.1000 \PYGZhy{}0.5100 0.3100  
\end{sphinxVerbatim}

\end{sphinxuseclass}\end{sphinxVerbatimOutput}

\end{sphinxuseclass}
\begin{sphinxuseclass}{cell}\begin{sphinxVerbatimInput}

\begin{sphinxuseclass}{cell_input}
\begin{sphinxVerbatim}[commandchars=\\\{\}]
\PYG{n}{models} \PYG{o}{=} \PYG{p}{[}\PYG{n}{smf}\PYG{o}{.}\PYG{n}{ols}\PYG{p}{(}\PYG{l+s+sa}{f}\PYG{l+s+s1}{\PYGZsq{}}\PYG{l+s+s1}{I(Q(}\PYG{l+s+s1}{\PYGZdq{}}\PYG{l+s+si}{\PYGZob{}}\PYG{n}{p}\PYG{l+s+si}{\PYGZcb{}}\PYG{l+s+s1}{\PYGZdq{}}\PYG{l+s+s1}{) \PYGZhy{} RF) \PYGZti{} Q(}\PYG{l+s+s1}{\PYGZdq{}}\PYG{l+s+s1}{Mkt\PYGZhy{}RF}\PYG{l+s+s1}{\PYGZdq{}}\PYG{l+s+s1}{)}\PYG{l+s+s1}{\PYGZsq{}}\PYG{p}{,} \PYG{n}{data}\PYG{o}{=}\PYG{n}{joined}\PYG{p}{)} \PYG{k}{for} \PYG{n}{p} \PYG{o+ow}{in} \PYG{n}{ports}\PYG{p}{]}

\PYG{n}{fits} \PYG{o}{=} \PYG{p}{[}\PYG{n}{m}\PYG{o}{.}\PYG{n}{fit}\PYG{p}{(}\PYG{p}{)} \PYG{k}{for} \PYG{n}{m} \PYG{o+ow}{in} \PYG{n}{models}\PYG{p}{]}

\PYG{n}{coefs} \PYG{o}{=} \PYG{p}{(}
    \PYG{n}{pd}\PYG{o}{.}\PYG{n}{concat}\PYG{p}{(}
        \PYG{n}{objs}\PYG{o}{=}\PYG{p}{[}\PYG{n}{f}\PYG{o}{.}\PYG{n}{params} \PYG{k}{for} \PYG{n}{f} \PYG{o+ow}{in} \PYG{n}{fits}\PYG{p}{]}\PYG{p}{,}
        \PYG{n}{axis}\PYG{o}{=}\PYG{l+m+mi}{1}\PYG{p}{,}
        \PYG{n}{keys}\PYG{o}{=}\PYG{n}{ports}
    \PYG{p}{)}
    \PYG{o}{.}\PYG{n}{rename\PYGZus{}axis}\PYG{p}{(}\PYG{n}{index}\PYG{o}{=}\PYG{l+s+s1}{\PYGZsq{}}\PYG{l+s+s1}{Coefficient}\PYG{l+s+s1}{\PYGZsq{}}\PYG{p}{,} \PYG{n}{columns}\PYG{o}{=}\PYG{l+s+s1}{\PYGZsq{}}\PYG{l+s+s1}{Equal\PYGZhy{}Weighted Value Portfolio}\PYG{l+s+s1}{\PYGZsq{}}\PYG{p}{)}
    \PYG{o}{.}\PYG{n}{transpose}\PYG{p}{(}\PYG{p}{)}
\PYG{p}{)}
\end{sphinxVerbatim}

\end{sphinxuseclass}\end{sphinxVerbatimInput}

\end{sphinxuseclass}
\sphinxAtStartPar
We can get the standard errors, too.
The standard errors are in the \sphinxcode{\sphinxupquote{.params}} attribute of our model fits.

\begin{sphinxuseclass}{cell}\begin{sphinxVerbatimInput}

\begin{sphinxuseclass}{cell_input}
\begin{sphinxVerbatim}[commandchars=\\\{\}]
\PYG{n}{ses} \PYG{o}{=} \PYG{p}{[}\PYG{n}{f}\PYG{o}{.}\PYG{n}{bse}\PYG{p}{[}\PYG{l+m+mi}{0}\PYG{p}{]} \PYG{k}{for} \PYG{n}{f} \PYG{o+ow}{in} \PYG{n}{fits}\PYG{p}{]}
\end{sphinxVerbatim}

\end{sphinxuseclass}\end{sphinxVerbatimInput}

\end{sphinxuseclass}
\begin{sphinxuseclass}{cell}\begin{sphinxVerbatimInput}

\begin{sphinxuseclass}{cell_input}
\begin{sphinxVerbatim}[commandchars=\\\{\}]
\PYG{n}{coefs}\PYG{p}{[}\PYG{l+s+s1}{\PYGZsq{}}\PYG{l+s+s1}{Intercept}\PYG{l+s+s1}{\PYGZsq{}}\PYG{p}{]}\PYG{o}{.}\PYG{n}{plot}\PYG{p}{(}\PYG{n}{kind}\PYG{o}{=}\PYG{l+s+s1}{\PYGZsq{}}\PYG{l+s+s1}{bar}\PYG{l+s+s1}{\PYGZsq{}}\PYG{p}{,} \PYG{n}{yerr}\PYG{o}{=}\PYG{n}{ses}\PYG{p}{)}
\PYG{n}{plt}\PYG{o}{.}\PYG{n}{ylabel}\PYG{p}{(}\PYG{l+s+sa}{r}\PYG{l+s+s1}{\PYGZsq{}}\PYG{l+s+s1}{Monthly \PYGZdl{}}\PYG{l+s+s1}{\PYGZbs{}}\PYG{l+s+s1}{alpha\PYGZdl{} (}\PYG{l+s+s1}{\PYGZpc{}}\PYG{l+s+s1}{) from CAPM}\PYG{l+s+s1}{\PYGZsq{}}\PYG{p}{)}
\PYG{c+c1}{\PYGZsh{} plt.xticks(rotation=0)}
\PYG{n}{plt}\PYG{o}{.}\PYG{n}{title}\PYG{p}{(}
    \PYG{l+s+sa}{r}\PYG{l+s+s1}{\PYGZsq{}}\PYG{l+s+s1}{Value Portfolio CAPM \PYGZdl{}}\PYG{l+s+s1}{\PYGZbs{}}\PYG{l+s+s1}{alpha\PYGZdl{}s for Monthly Returns}\PYG{l+s+s1}{\PYGZsq{}} \PYG{o}{+}
    \PYG{l+s+s1}{\PYGZsq{}}\PYG{l+s+se}{\PYGZbs{}n}\PYG{l+s+s1}{from July 1926 through january 1979}\PYG{l+s+s1}{\PYGZsq{}}
\PYG{p}{)}
\PYG{n}{plt}\PYG{o}{.}\PYG{n}{show}\PYG{p}{(}\PYG{p}{)}
\end{sphinxVerbatim}

\end{sphinxuseclass}\end{sphinxVerbatimInput}
\begin{sphinxVerbatimOutput}

\begin{sphinxuseclass}{cell_output}
\noindent\sphinxincludegraphics{{8fad74c42a93b4cea2c88078ade4b80a065ea5ec0510ecdb7206c2d785a22602}.png}

\end{sphinxuseclass}\end{sphinxVerbatimOutput}

\end{sphinxuseclass}

\subsubsection{Repeat the exercises above with the momentum factor}
\label{\detokenize{herron_03_practice_03:repeat-the-exercises-above-with-the-momentum-factor}}
\sphinxAtStartPar
You may find it helpful to consider the worst months and years for the momentum factor.

\begin{sphinxuseclass}{cell}\begin{sphinxVerbatimInput}

\begin{sphinxuseclass}{cell_input}
\begin{sphinxVerbatim}[commandchars=\\\{\}]
\PYG{n}{mom\PYGZus{}0} \PYG{o}{=} \PYG{n}{pdr}\PYG{o}{.}\PYG{n}{DataReader}\PYG{p}{(}
    \PYG{n}{name}\PYG{o}{=}\PYG{l+s+s1}{\PYGZsq{}}\PYG{l+s+s1}{10\PYGZus{}Portfolios\PYGZus{}Prior\PYGZus{}12\PYGZus{}2}\PYG{l+s+s1}{\PYGZsq{}}\PYG{p}{,}
    \PYG{n}{data\PYGZus{}source}\PYG{o}{=}\PYG{l+s+s1}{\PYGZsq{}}\PYG{l+s+s1}{famafrench}\PYG{l+s+s1}{\PYGZsq{}}\PYG{p}{,}
    \PYG{n}{start}\PYG{o}{=}\PYG{l+s+s1}{\PYGZsq{}}\PYG{l+s+s1}{1900}\PYG{l+s+s1}{\PYGZsq{}}\PYG{p}{,}
    \PYG{n}{session}\PYG{o}{=}\PYG{n}{session}
\PYG{p}{)}

\PYG{n+nb}{print}\PYG{p}{(}\PYG{n}{mom\PYGZus{}0}\PYG{p}{[}\PYG{l+s+s1}{\PYGZsq{}}\PYG{l+s+s1}{DESCR}\PYG{l+s+s1}{\PYGZsq{}}\PYG{p}{]}\PYG{p}{)}
\end{sphinxVerbatim}

\end{sphinxuseclass}\end{sphinxVerbatimInput}
\begin{sphinxVerbatimOutput}

\begin{sphinxuseclass}{cell_output}
\begin{sphinxVerbatim}[commandchars=\\\{\}]
10 Portfolios Prior 12 2
\PYGZhy{}\PYGZhy{}\PYGZhy{}\PYGZhy{}\PYGZhy{}\PYGZhy{}\PYGZhy{}\PYGZhy{}\PYGZhy{}\PYGZhy{}\PYGZhy{}\PYGZhy{}\PYGZhy{}\PYGZhy{}\PYGZhy{}\PYGZhy{}\PYGZhy{}\PYGZhy{}\PYGZhy{}\PYGZhy{}\PYGZhy{}\PYGZhy{}\PYGZhy{}\PYGZhy{}

This file was created by CMPT\PYGZus{}PRIOR\PYGZus{}RETS using the 202301 CRSP database. It contains value\PYGZhy{} and equal\PYGZhy{}weighted returns for 10 prior\PYGZhy{}return portfolios. The portfolios are constructed monthly. PRIOR\PYGZus{}RET is from \PYGZhy{}12 to \PYGZhy{} 2. The annual returns are from January to December. Missing data are indicated by \PYGZhy{}99.99 or \PYGZhy{}999.

  0 : Average Value Weighted Returns \PYGZhy{}\PYGZhy{} Monthly (1153 rows x 10 cols)
  1 : Average Equal Weighted Returns \PYGZhy{}\PYGZhy{} Monthly (1153 rows x 10 cols)
  2 : Average Value Weighted Returns \PYGZhy{}\PYGZhy{} Annual (96 rows x 10 cols)
  3 : Average Equal Weighted Returns \PYGZhy{}\PYGZhy{} Annual (96 rows x 10 cols)
  4 : Number of Firms in Portfolios (1153 rows x 10 cols)
  5 : Average Firm Size (1153 rows x 10 cols)
  6 : Value\PYGZhy{}Weighted Average of Prior Returns (96 rows x 10 cols)
\end{sphinxVerbatim}

\end{sphinxuseclass}\end{sphinxVerbatimOutput}

\end{sphinxuseclass}
\begin{sphinxuseclass}{cell}\begin{sphinxVerbatimInput}

\begin{sphinxuseclass}{cell_input}
\begin{sphinxVerbatim}[commandchars=\\\{\}]
\PYG{n}{ports} \PYG{o}{=} \PYG{p}{[}\PYG{l+s+s1}{\PYGZsq{}}\PYG{l+s+s1}{Lo PRIOR}\PYG{l+s+s1}{\PYGZsq{}}\PYG{p}{,} \PYG{l+s+s1}{\PYGZsq{}}\PYG{l+s+s1}{PRIOR 2}\PYG{l+s+s1}{\PYGZsq{}}\PYG{p}{,} \PYG{l+s+s1}{\PYGZsq{}}\PYG{l+s+s1}{PRIOR 3}\PYG{l+s+s1}{\PYGZsq{}}\PYG{p}{,} \PYG{l+s+s1}{\PYGZsq{}}\PYG{l+s+s1}{PRIOR 4}\PYG{l+s+s1}{\PYGZsq{}}\PYG{p}{,} \PYG{l+s+s1}{\PYGZsq{}}\PYG{l+s+s1}{PRIOR 5}\PYG{l+s+s1}{\PYGZsq{}}\PYG{p}{,} \PYG{l+s+s1}{\PYGZsq{}}\PYG{l+s+s1}{PRIOR 6}\PYG{l+s+s1}{\PYGZsq{}}\PYG{p}{,} \PYG{l+s+s1}{\PYGZsq{}}\PYG{l+s+s1}{PRIOR 7}\PYG{l+s+s1}{\PYGZsq{}}\PYG{p}{,} \PYG{l+s+s1}{\PYGZsq{}}\PYG{l+s+s1}{PRIOR 8}\PYG{l+s+s1}{\PYGZsq{}}\PYG{p}{,} \PYG{l+s+s1}{\PYGZsq{}}\PYG{l+s+s1}{PRIOR 9}\PYG{l+s+s1}{\PYGZsq{}}\PYG{p}{,} \PYG{l+s+s1}{\PYGZsq{}}\PYG{l+s+s1}{Hi PRIOR}\PYG{l+s+s1}{\PYGZsq{}}\PYG{p}{]}
\PYG{n}{joined} \PYG{o}{=} \PYG{n}{mom\PYGZus{}0}\PYG{p}{[}\PYG{l+m+mi}{1}\PYG{p}{]}\PYG{p}{[}\PYG{n}{ports}\PYG{p}{]}\PYG{o}{.}\PYG{n}{join}\PYG{p}{(}\PYG{n}{ff\PYGZus{}m}\PYG{p}{[}\PYG{l+m+mi}{0}\PYG{p}{]}\PYG{p}{)}
\PYG{n}{joined}\PYG{o}{.}\PYG{n}{head}\PYG{p}{(}\PYG{p}{)}
\end{sphinxVerbatim}

\end{sphinxuseclass}\end{sphinxVerbatimInput}
\begin{sphinxVerbatimOutput}

\begin{sphinxuseclass}{cell_output}
\begin{sphinxVerbatim}[commandchars=\\\{\}]
         Lo PRIOR  PRIOR 2  PRIOR 3  PRIOR 4  PRIOR 5  PRIOR 6  PRIOR 7  \PYGZbs{}
Date                                                                      
1927\PYGZhy{}01   \PYGZhy{}0.4700   1.0400   4.1700   2.4000  \PYGZhy{}0.3100   3.0800   1.4800   
1927\PYGZhy{}02    8.4900   3.8800   8.3300   7.5900   4.7900   5.4700   5.4000   
1927\PYGZhy{}03   \PYGZhy{}6.1500  \PYGZhy{}5.1100  \PYGZhy{}2.3800  \PYGZhy{}1.6000  \PYGZhy{}1.6300  \PYGZhy{}1.7000   0.4900   
1927\PYGZhy{}04    2.6100  \PYGZhy{}0.9300   0.1000  \PYGZhy{}1.2000   0.0900   0.4000   1.2700   
1927\PYGZhy{}05    1.1200   3.7300   6.0100   5.1700   7.5500   9.2700   6.6100   

         PRIOR 8  PRIOR 9  Hi PRIOR  Mkt\PYGZhy{}RF     SMB     HML     RF  
Date                                                                
1927\PYGZhy{}01   0.6400  \PYGZhy{}0.5900    1.1800 \PYGZhy{}0.0600 \PYGZhy{}0.3700  4.5400 0.2500  
1927\PYGZhy{}02   4.3000   5.7600    5.5800  4.1800  0.0400  2.9400 0.2600  
1927\PYGZhy{}03  \PYGZhy{}1.5300   0.3700   \PYGZhy{}0.7800  0.1300 \PYGZhy{}1.6500 \PYGZhy{}2.6100 0.3000  
1927\PYGZhy{}04  \PYGZhy{}0.5700   2.2200    4.4200  0.4600  0.3000  0.8100 0.2500  
1927\PYGZhy{}05   8.2100   6.9100   10.0000  5.4400  1.5300  4.7300 0.3000  
\end{sphinxVerbatim}

\end{sphinxuseclass}\end{sphinxVerbatimOutput}

\end{sphinxuseclass}
\begin{sphinxuseclass}{cell}\begin{sphinxVerbatimInput}

\begin{sphinxuseclass}{cell_input}
\begin{sphinxVerbatim}[commandchars=\\\{\}]
\PYG{n}{models} \PYG{o}{=} \PYG{p}{[}\PYG{n}{smf}\PYG{o}{.}\PYG{n}{ols}\PYG{p}{(}\PYG{l+s+sa}{f}\PYG{l+s+s1}{\PYGZsq{}}\PYG{l+s+s1}{I(Q(}\PYG{l+s+s1}{\PYGZdq{}}\PYG{l+s+si}{\PYGZob{}}\PYG{n}{p}\PYG{l+s+si}{\PYGZcb{}}\PYG{l+s+s1}{\PYGZdq{}}\PYG{l+s+s1}{) \PYGZhy{} RF) \PYGZti{} Q(}\PYG{l+s+s1}{\PYGZdq{}}\PYG{l+s+s1}{Mkt\PYGZhy{}RF}\PYG{l+s+s1}{\PYGZdq{}}\PYG{l+s+s1}{)}\PYG{l+s+s1}{\PYGZsq{}}\PYG{p}{,} \PYG{n}{data}\PYG{o}{=}\PYG{n}{joined}\PYG{p}{)} \PYG{k}{for} \PYG{n}{p} \PYG{o+ow}{in} \PYG{n}{ports}\PYG{p}{]}
\end{sphinxVerbatim}

\end{sphinxuseclass}\end{sphinxVerbatimInput}

\end{sphinxuseclass}
\begin{sphinxuseclass}{cell}\begin{sphinxVerbatimInput}

\begin{sphinxuseclass}{cell_input}
\begin{sphinxVerbatim}[commandchars=\\\{\}]
\PYG{n}{fits} \PYG{o}{=} \PYG{p}{[}\PYG{n}{m}\PYG{o}{.}\PYG{n}{fit}\PYG{p}{(}\PYG{p}{)} \PYG{k}{for} \PYG{n}{m} \PYG{o+ow}{in} \PYG{n}{models}\PYG{p}{]}
\end{sphinxVerbatim}

\end{sphinxuseclass}\end{sphinxVerbatimInput}

\end{sphinxuseclass}
\begin{sphinxuseclass}{cell}\begin{sphinxVerbatimInput}

\begin{sphinxuseclass}{cell_input}
\begin{sphinxVerbatim}[commandchars=\\\{\}]
\PYG{n}{coefs} \PYG{o}{=} \PYG{p}{(}
    \PYG{n}{pd}\PYG{o}{.}\PYG{n}{concat}\PYG{p}{(}
        \PYG{n}{objs}\PYG{o}{=}\PYG{p}{[}\PYG{n}{f}\PYG{o}{.}\PYG{n}{params} \PYG{k}{for} \PYG{n}{f} \PYG{o+ow}{in} \PYG{n}{fits}\PYG{p}{]}\PYG{p}{,}
        \PYG{n}{axis}\PYG{o}{=}\PYG{l+m+mi}{1}\PYG{p}{,}
        \PYG{n}{keys}\PYG{o}{=}\PYG{n}{ports}
    \PYG{p}{)}
    \PYG{o}{.}\PYG{n}{rename\PYGZus{}axis}\PYG{p}{(}\PYG{n}{index}\PYG{o}{=}\PYG{l+s+s1}{\PYGZsq{}}\PYG{l+s+s1}{Coefficient}\PYG{l+s+s1}{\PYGZsq{}}\PYG{p}{,} \PYG{n}{columns}\PYG{o}{=}\PYG{l+s+s1}{\PYGZsq{}}\PYG{l+s+s1}{Equal\PYGZhy{}Weighted Momentum Portfolio}\PYG{l+s+s1}{\PYGZsq{}}\PYG{p}{)}
    \PYG{o}{.}\PYG{n}{transpose}\PYG{p}{(}\PYG{p}{)}
\PYG{p}{)}
\end{sphinxVerbatim}

\end{sphinxuseclass}\end{sphinxVerbatimInput}

\end{sphinxuseclass}
\sphinxAtStartPar
We can get the standard errors, too.
The standard errors are in the \sphinxcode{\sphinxupquote{.params}} attribute of our model fits.

\begin{sphinxuseclass}{cell}\begin{sphinxVerbatimInput}

\begin{sphinxuseclass}{cell_input}
\begin{sphinxVerbatim}[commandchars=\\\{\}]
\PYG{n}{ses} \PYG{o}{=} \PYG{p}{[}\PYG{n}{f}\PYG{o}{.}\PYG{n}{bse}\PYG{p}{[}\PYG{l+m+mi}{0}\PYG{p}{]} \PYG{k}{for} \PYG{n}{f} \PYG{o+ow}{in} \PYG{n}{fits}\PYG{p}{]}
\end{sphinxVerbatim}

\end{sphinxuseclass}\end{sphinxVerbatimInput}

\end{sphinxuseclass}
\begin{sphinxuseclass}{cell}\begin{sphinxVerbatimInput}

\begin{sphinxuseclass}{cell_input}
\begin{sphinxVerbatim}[commandchars=\\\{\}]
\PYG{n}{coefs}\PYG{p}{[}\PYG{l+s+s1}{\PYGZsq{}}\PYG{l+s+s1}{Intercept}\PYG{l+s+s1}{\PYGZsq{}}\PYG{p}{]}\PYG{o}{.}\PYG{n}{plot}\PYG{p}{(}\PYG{n}{kind}\PYG{o}{=}\PYG{l+s+s1}{\PYGZsq{}}\PYG{l+s+s1}{bar}\PYG{l+s+s1}{\PYGZsq{}}\PYG{p}{,} \PYG{n}{yerr}\PYG{o}{=}\PYG{n}{ses}\PYG{p}{)}
\PYG{n}{plt}\PYG{o}{.}\PYG{n}{ylabel}\PYG{p}{(}\PYG{l+s+sa}{r}\PYG{l+s+s1}{\PYGZsq{}}\PYG{l+s+s1}{Monthly \PYGZdl{}}\PYG{l+s+s1}{\PYGZbs{}}\PYG{l+s+s1}{alpha\PYGZdl{} (}\PYG{l+s+s1}{\PYGZpc{}}\PYG{l+s+s1}{) from CAPM}\PYG{l+s+s1}{\PYGZsq{}}\PYG{p}{)}
\PYG{c+c1}{\PYGZsh{} plt.xticks(rotation=0)}
\PYG{n}{plt}\PYG{o}{.}\PYG{n}{title}\PYG{p}{(}
    \PYG{l+s+sa}{r}\PYG{l+s+s1}{\PYGZsq{}}\PYG{l+s+s1}{Momentum Portfolio CAPM \PYGZdl{}}\PYG{l+s+s1}{\PYGZbs{}}\PYG{l+s+s1}{alpha\PYGZdl{}s for Monthly Returns}\PYG{l+s+s1}{\PYGZsq{}} \PYG{o}{+}
    \PYG{l+s+s1}{\PYGZsq{}}\PYG{l+s+se}{\PYGZbs{}n}\PYG{l+s+s1}{from July 1926 through january 1979}\PYG{l+s+s1}{\PYGZsq{}}
\PYG{p}{)}
\PYG{n}{plt}\PYG{o}{.}\PYG{n}{show}\PYG{p}{(}\PYG{p}{)}
\end{sphinxVerbatim}

\end{sphinxuseclass}\end{sphinxVerbatimInput}
\begin{sphinxVerbatimOutput}

\begin{sphinxuseclass}{cell_output}
\noindent\sphinxincludegraphics{{561fb57ef0e2da2f6371ef7bcf748c3c03051c3f465b8cead6b1dd2eb3e19424}.png}

\end{sphinxuseclass}\end{sphinxVerbatimOutput}

\end{sphinxuseclass}

\subsubsection{Plot the coefficient estimates from a rolling Fama\sphinxhyphen{}French three\sphinxhyphen{}factor model for Berkshire Hathaway}
\label{\detokenize{herron_03_practice_03:plot-the-coefficient-estimates-from-a-rolling-fama-french-three-factor-model-for-berkshire-hathaway}}
\sphinxAtStartPar
Use a three\sphinxhyphen{}year window with daily returns.
How has Buffett’s \$\textbackslash{}alpha\$ and \$\textbackslash{}beta\$s changed over the past four decades?

\begin{sphinxuseclass}{cell}\begin{sphinxVerbatimInput}

\begin{sphinxuseclass}{cell_input}
\begin{sphinxVerbatim}[commandchars=\\\{\}]
\PYG{n}{brk} \PYG{o}{=} \PYG{p}{(}
    \PYG{n}{yf}\PYG{o}{.}\PYG{n}{download}\PYG{p}{(}\PYG{n}{tickers}\PYG{o}{=}\PYG{l+s+s1}{\PYGZsq{}}\PYG{l+s+s1}{BRK\PYGZhy{}A}\PYG{l+s+s1}{\PYGZsq{}}\PYG{p}{,} \PYG{n}{progress}\PYG{o}{=}\PYG{k+kc}{False}\PYG{p}{)}
    \PYG{o}{.}\PYG{n}{assign}\PYG{p}{(}\PYG{n}{Date} \PYG{o}{=} \PYG{k}{lambda} \PYG{n}{x}\PYG{p}{:} \PYG{n}{x}\PYG{o}{.}\PYG{n}{index}\PYG{o}{.}\PYG{n}{tz\PYGZus{}localize}\PYG{p}{(}\PYG{k+kc}{None}\PYG{p}{)}\PYG{p}{)}
    \PYG{o}{.}\PYG{n}{set\PYGZus{}index}\PYG{p}{(}\PYG{l+s+s1}{\PYGZsq{}}\PYG{l+s+s1}{Date}\PYG{l+s+s1}{\PYGZsq{}}\PYG{p}{)}
    \PYG{o}{.}\PYG{n}{rename\PYGZus{}axis}\PYG{p}{(}\PYG{n}{columns}\PYG{o}{=}\PYG{p}{[}\PYG{l+s+s1}{\PYGZsq{}}\PYG{l+s+s1}{Variable}\PYG{l+s+s1}{\PYGZsq{}}\PYG{p}{]}\PYG{p}{)}
    \PYG{o}{.}\PYG{n}{assign}\PYG{p}{(}\PYG{n}{R}\PYG{o}{=}\PYG{k}{lambda} \PYG{n}{x}\PYG{p}{:} \PYG{n}{x}\PYG{p}{[}\PYG{l+s+s1}{\PYGZsq{}}\PYG{l+s+s1}{Adj Close}\PYG{l+s+s1}{\PYGZsq{}}\PYG{p}{]}\PYG{o}{.}\PYG{n}{pct\PYGZus{}change}\PYG{p}{(}\PYG{p}{)}\PYG{o}{.}\PYG{n}{mul}\PYG{p}{(}\PYG{l+m+mi}{100}\PYG{p}{)}\PYG{p}{)}
    \PYG{o}{.}\PYG{n}{dropna}\PYG{p}{(}\PYG{p}{)}
    \PYG{o}{.}\PYG{n}{join}\PYG{p}{(}
        \PYG{n}{pdr}\PYG{o}{.}\PYG{n}{DataReader}\PYG{p}{(}
            \PYG{n}{name}\PYG{o}{=}\PYG{l+s+s1}{\PYGZsq{}}\PYG{l+s+s1}{F\PYGZhy{}F\PYGZus{}Research\PYGZus{}Data\PYGZus{}Factors\PYGZus{}daily}\PYG{l+s+s1}{\PYGZsq{}}\PYG{p}{,}
            \PYG{n}{data\PYGZus{}source}\PYG{o}{=}\PYG{l+s+s1}{\PYGZsq{}}\PYG{l+s+s1}{famafrench}\PYG{l+s+s1}{\PYGZsq{}}\PYG{p}{,}
            \PYG{n}{start}\PYG{o}{=}\PYG{l+s+s1}{\PYGZsq{}}\PYG{l+s+s1}{1900}\PYG{l+s+s1}{\PYGZsq{}}\PYG{p}{,}
            \PYG{n}{session}\PYG{o}{=}\PYG{n}{session}
        \PYG{p}{)}\PYG{p}{[}\PYG{l+m+mi}{0}\PYG{p}{]}\PYG{p}{,}
        \PYG{n}{how}\PYG{o}{=}\PYG{l+s+s1}{\PYGZsq{}}\PYG{l+s+s1}{inner}\PYG{l+s+s1}{\PYGZsq{}}
    \PYG{p}{)}
\PYG{p}{)}
\end{sphinxVerbatim}

\end{sphinxuseclass}\end{sphinxVerbatimInput}

\end{sphinxuseclass}
\begin{sphinxuseclass}{cell}\begin{sphinxVerbatimInput}

\begin{sphinxuseclass}{cell_input}
\begin{sphinxVerbatim}[commandchars=\\\{\}]
\PYG{k+kn}{from} \PYG{n+nn}{statsmodels}\PYG{n+nn}{.}\PYG{n+nn}{regression}\PYG{n+nn}{.}\PYG{n+nn}{rolling} \PYG{k+kn}{import} \PYG{n}{RollingOLS}
\end{sphinxVerbatim}

\end{sphinxuseclass}\end{sphinxVerbatimInput}

\end{sphinxuseclass}
\begin{sphinxuseclass}{cell}\begin{sphinxVerbatimInput}

\begin{sphinxuseclass}{cell_input}
\begin{sphinxVerbatim}[commandchars=\\\{\}]
\PYG{n}{coefs} \PYG{o}{=} \PYG{p}{(}
    \PYG{n}{RollingOLS}\PYG{o}{.}\PYG{n}{from\PYGZus{}formula}\PYG{p}{(}
        \PYG{n}{formula}\PYG{o}{=}\PYG{l+s+s1}{\PYGZsq{}}\PYG{l+s+s1}{I(R\PYGZhy{}RF) \PYGZti{} Q(}\PYG{l+s+s1}{\PYGZdq{}}\PYG{l+s+s1}{Mkt\PYGZhy{}RF}\PYG{l+s+s1}{\PYGZdq{}}\PYG{l+s+s1}{) + SMB + HML}\PYG{l+s+s1}{\PYGZsq{}}\PYG{p}{,}
        \PYG{n}{data}\PYG{o}{=}\PYG{n}{brk}\PYG{p}{,}
        \PYG{n}{window}\PYG{o}{=}\PYG{l+m+mi}{3}\PYG{o}{*}\PYG{l+m+mi}{252}
    \PYG{p}{)}
    \PYG{o}{.}\PYG{n}{fit}\PYG{p}{(}\PYG{p}{)}
    \PYG{o}{.}\PYG{n}{params}
    \PYG{o}{.}\PYG{n}{rename\PYGZus{}axis}\PYG{p}{(}\PYG{n}{columns}\PYG{o}{=}\PYG{l+s+s1}{\PYGZsq{}}\PYG{l+s+s1}{Coefficient}\PYG{l+s+s1}{\PYGZsq{}}\PYG{p}{)}
    \PYG{o}{.}\PYG{n}{rename}\PYG{p}{(}\PYG{n}{columns}\PYG{o}{=}\PYG{p}{\PYGZob{}}\PYG{l+s+s1}{\PYGZsq{}}\PYG{l+s+s1}{Q(}\PYG{l+s+s1}{\PYGZdq{}}\PYG{l+s+s1}{Mkt\PYGZhy{}RF}\PYG{l+s+s1}{\PYGZdq{}}\PYG{l+s+s1}{)}\PYG{l+s+s1}{\PYGZsq{}}\PYG{p}{:} \PYG{l+s+s1}{\PYGZsq{}}\PYG{l+s+s1}{Mkt\PYGZhy{}RF}\PYG{l+s+s1}{\PYGZsq{}}\PYG{p}{\PYGZcb{}}\PYG{p}{)}
\PYG{p}{)}
\end{sphinxVerbatim}

\end{sphinxuseclass}\end{sphinxVerbatimInput}

\end{sphinxuseclass}
\begin{sphinxuseclass}{cell}\begin{sphinxVerbatimInput}

\begin{sphinxuseclass}{cell_input}
\begin{sphinxVerbatim}[commandchars=\\\{\}]
\PYG{n}{fig}\PYG{p}{,} \PYG{n}{ax} \PYG{o}{=} \PYG{n}{plt}\PYG{o}{.}\PYG{n}{subplots}\PYG{p}{(}\PYG{l+m+mi}{2}\PYG{p}{,} \PYG{l+m+mi}{1}\PYG{p}{,} \PYG{n}{sharex}\PYG{o}{=}\PYG{k+kc}{True}\PYG{p}{)}
\PYG{n}{coefs}\PYG{p}{[}\PYG{l+s+s1}{\PYGZsq{}}\PYG{l+s+s1}{Intercept}\PYG{l+s+s1}{\PYGZsq{}}\PYG{p}{]}\PYG{o}{.}\PYG{n}{plot}\PYG{p}{(}\PYG{n}{ax}\PYG{o}{=}\PYG{n}{ax}\PYG{p}{[}\PYG{l+m+mi}{0}\PYG{p}{]}\PYG{p}{,} \PYG{n}{legend}\PYG{o}{=}\PYG{k+kc}{True}\PYG{p}{)}
\PYG{n}{coefs}\PYG{o}{.}\PYG{n}{drop}\PYG{p}{(}\PYG{l+s+s1}{\PYGZsq{}}\PYG{l+s+s1}{Intercept}\PYG{l+s+s1}{\PYGZsq{}}\PYG{p}{,} \PYG{n}{axis}\PYG{o}{=}\PYG{l+m+mi}{1}\PYG{p}{)}\PYG{o}{.}\PYG{n}{plot}\PYG{p}{(}\PYG{n}{ax}\PYG{o}{=}\PYG{n}{ax}\PYG{p}{[}\PYG{l+m+mi}{1}\PYG{p}{]}\PYG{p}{)}
\PYG{n}{plt}\PYG{o}{.}\PYG{n}{suptitle}\PYG{p}{(}
    \PYG{l+s+s1}{\PYGZsq{}}\PYG{l+s+s1}{Rolling Three\PYGZhy{}Factor Regressions}\PYG{l+s+s1}{\PYGZsq{}} \PYG{o}{+}
    \PYG{l+s+s1}{\PYGZsq{}}\PYG{l+s+se}{\PYGZbs{}n}\PYG{l+s+s1}{Three\PYGZhy{}Year Windows with Daily Returns in Percent}\PYG{l+s+s1}{\PYGZsq{}}
\PYG{p}{)}
\PYG{n}{plt}\PYG{o}{.}\PYG{n}{show}\PYG{p}{(}\PYG{p}{)}
\end{sphinxVerbatim}

\end{sphinxuseclass}\end{sphinxVerbatimInput}
\begin{sphinxVerbatimOutput}

\begin{sphinxuseclass}{cell_output}
\noindent\sphinxincludegraphics{{f3f7c57e9982c23fc26048b9555cbf9edabb9a944fc6dc1da29461abc007d976}.png}

\end{sphinxuseclass}\end{sphinxVerbatimOutput}

\end{sphinxuseclass}
\sphinxAtStartPar
Buffett’s \$\textbackslash{}alpha\$ was large, but has declined to zero.
Also, his loading on SMB (size factor) has gone from positive to negative, indicating that he has moved from small stocks to large stocks as Berkshire Hathaway has grown.


\subsubsection{Use the three\sphinxhyphen{}, four\sphinxhyphen{}, and five\sphinxhyphen{}factor models to determine how the ARKK Innovation ETF generates returns}
\label{\detokenize{herron_03_practice_03:use-the-three-four-and-five-factor-models-to-determine-how-the-arkk-innovation-etf-generates-returns}}
\sphinxstepscope


\section{Herron Topic 3 \sphinxhyphen{} Practice (Monday 11:45 AM, Section 4)}
\label{\detokenize{herron_03_practice_04:herron-topic-3-practice-monday-11-45-am-section-4}}\label{\detokenize{herron_03_practice_04::doc}}

\subsection{Announcements}
\label{\detokenize{herron_03_practice_04:announcements}}\begin{itemize}
\item {} 
\sphinxAtStartPar
Team Project 1 grades posted
\begin{itemize}
\item {} 
\sphinxAtStartPar
Mean of 81\% and median of 84\%

\item {} 
\sphinxAtStartPar
100\% on Teammates reviews pushes mean to 84\%, which is about the target

\item {} 
\sphinxAtStartPar
Overall course mean is 86\%, so I would curve up overall course grades about 4\% if posted grades today, much more at the low end

\item {} 
\sphinxAtStartPar
However, I cannot commit to a curve today

\end{itemize}

\item {} 
\sphinxAtStartPar
20,000 XP on DataCamp due by 11:59 PM on Friday

\end{itemize}


\subsection{Practice}
\label{\detokenize{herron_03_practice_04:practice}}
\begin{sphinxuseclass}{cell}\begin{sphinxVerbatimInput}

\begin{sphinxuseclass}{cell_input}
\begin{sphinxVerbatim}[commandchars=\\\{\}]
\PYG{k+kn}{import} \PYG{n+nn}{matplotlib}\PYG{n+nn}{.}\PYG{n+nn}{pyplot} \PYG{k}{as} \PYG{n+nn}{plt}
\PYG{k+kn}{import} \PYG{n+nn}{numpy} \PYG{k}{as} \PYG{n+nn}{np}
\PYG{k+kn}{import} \PYG{n+nn}{pandas} \PYG{k}{as} \PYG{n+nn}{pd}
\end{sphinxVerbatim}

\end{sphinxuseclass}\end{sphinxVerbatimInput}

\end{sphinxuseclass}
\begin{sphinxuseclass}{cell}\begin{sphinxVerbatimInput}

\begin{sphinxuseclass}{cell_input}
\begin{sphinxVerbatim}[commandchars=\\\{\}]
\PYG{o}{\PYGZpc{}}\PYG{k}{config} InlineBackend.figure\PYGZus{}format = \PYGZsq{}retina\PYGZsq{}
\PYG{o}{\PYGZpc{}}\PYG{k}{precision} 4
\PYG{n}{pd}\PYG{o}{.}\PYG{n}{options}\PYG{o}{.}\PYG{n}{display}\PYG{o}{.}\PYG{n}{float\PYGZus{}format} \PYG{o}{=} \PYG{l+s+s1}{\PYGZsq{}}\PYG{l+s+si}{\PYGZob{}:.4f\PYGZcb{}}\PYG{l+s+s1}{\PYGZsq{}}\PYG{o}{.}\PYG{n}{format}
\end{sphinxVerbatim}

\end{sphinxuseclass}\end{sphinxVerbatimInput}

\end{sphinxuseclass}
\begin{sphinxuseclass}{cell}\begin{sphinxVerbatimInput}

\begin{sphinxuseclass}{cell_input}
\begin{sphinxVerbatim}[commandchars=\\\{\}]
\PYG{k+kn}{import} \PYG{n+nn}{yfinance} \PYG{k}{as} \PYG{n+nn}{yf}
\PYG{k+kn}{import} \PYG{n+nn}{pandas\PYGZus{}datareader} \PYG{k}{as} \PYG{n+nn}{pdr}
\PYG{k+kn}{import} \PYG{n+nn}{requests\PYGZus{}cache}
\PYG{n}{session} \PYG{o}{=} \PYG{n}{requests\PYGZus{}cache}\PYG{o}{.}\PYG{n}{CachedSession}\PYG{p}{(}\PYG{p}{)}
\end{sphinxVerbatim}

\end{sphinxuseclass}\end{sphinxVerbatimInput}

\end{sphinxuseclass}

\subsubsection{Plot the security market line (SML) for a variety of asset classes}
\label{\detokenize{herron_03_practice_04:plot-the-security-market-line-sml-for-a-variety-of-asset-classes}}
\sphinxAtStartPar
Use the past three years of daily data for the following exhange traded funds (ETFs):
\begin{enumerate}
\sphinxsetlistlabels{\arabic}{enumi}{enumii}{}{.}%
\item {} 
\sphinxAtStartPar
SPY (SPDR—Standard and Poor’s Depository Receipts—ETF for the S\&P 500 index)

\item {} 
\sphinxAtStartPar
BIL (SPDR ETF for 1\sphinxhyphen{}3 month Treasury bills)

\item {} 
\sphinxAtStartPar
GLD (SPDR ETF for gold)

\item {} 
\sphinxAtStartPar
JNK (SPDR ETF for high\sphinxhyphen{}yield debt)

\item {} 
\sphinxAtStartPar
MDY (SPDR ETF for S\&P 400 mid\sphinxhyphen{}cap index)

\item {} 
\sphinxAtStartPar
SLY (SPDR ETF for S\&P 600 small\sphinxhyphen{}cap index)

\item {} 
\sphinxAtStartPar
SPBO (SPDR ETF for corporate bonds)

\item {} 
\sphinxAtStartPar
SPMB (SPDR ETF for mortgage\sphinxhyphen{}backed securities)

\item {} 
\sphinxAtStartPar
SPTL (SPDR ETF for long\sphinxhyphen{}term Treasury bonds)

\end{enumerate}

\begin{sphinxuseclass}{cell}\begin{sphinxVerbatimInput}

\begin{sphinxuseclass}{cell_input}
\begin{sphinxVerbatim}[commandchars=\\\{\}]
\PYG{n}{etf} \PYG{o}{=} \PYG{p}{(}
    \PYG{n}{yf}\PYG{o}{.}\PYG{n}{download}\PYG{p}{(}
        \PYG{n}{tickers}\PYG{o}{=}\PYG{l+s+s1}{\PYGZsq{}}\PYG{l+s+s1}{SPY BIL GLD JNK MDY SLY SPBO SPMB SPTL}\PYG{l+s+s1}{\PYGZsq{}}\PYG{p}{,}
        \PYG{n}{progress}\PYG{o}{=}\PYG{k+kc}{False}
    \PYG{p}{)}
    \PYG{o}{.}\PYG{n}{assign}\PYG{p}{(}\PYG{n}{Date} \PYG{o}{=} \PYG{k}{lambda} \PYG{n}{x}\PYG{p}{:} \PYG{n}{x}\PYG{o}{.}\PYG{n}{index}\PYG{o}{.}\PYG{n}{tz\PYGZus{}localize}\PYG{p}{(}\PYG{k+kc}{None}\PYG{p}{)}\PYG{p}{)}
    \PYG{o}{.}\PYG{n}{set\PYGZus{}index}\PYG{p}{(}\PYG{l+s+s1}{\PYGZsq{}}\PYG{l+s+s1}{Date}\PYG{l+s+s1}{\PYGZsq{}}\PYG{p}{)}
    \PYG{o}{.}\PYG{n}{rename\PYGZus{}axis}\PYG{p}{(}\PYG{n}{columns}\PYG{o}{=}\PYG{p}{[}\PYG{l+s+s1}{\PYGZsq{}}\PYG{l+s+s1}{Variable}\PYG{l+s+s1}{\PYGZsq{}}\PYG{p}{,} \PYG{l+s+s1}{\PYGZsq{}}\PYG{l+s+s1}{Ticker}\PYG{l+s+s1}{\PYGZsq{}}\PYG{p}{]}\PYG{p}{)}
    \PYG{p}{[}\PYG{l+s+s1}{\PYGZsq{}}\PYG{l+s+s1}{Adj Close}\PYG{l+s+s1}{\PYGZsq{}}\PYG{p}{]}
    \PYG{o}{.}\PYG{n}{pct\PYGZus{}change}\PYG{p}{(}\PYG{p}{)}
\PYG{p}{)}

\PYG{n}{etf}\PYG{o}{.}\PYG{n}{tail}\PYG{p}{(}\PYG{p}{)}
\end{sphinxVerbatim}

\end{sphinxuseclass}\end{sphinxVerbatimInput}
\begin{sphinxVerbatimOutput}

\begin{sphinxuseclass}{cell_output}
\begin{sphinxVerbatim}[commandchars=\\\{\}]
Ticker        BIL     GLD     JNK     MDY     SLY    SPBO    SPMB    SPTL  \PYGZbs{}
Date                                                                        
2023\PYGZhy{}03\PYGZhy{}16 0.0004  0.0020  0.0066  0.0138  0.0159 \PYGZhy{}0.0031 \PYGZhy{}0.0036 \PYGZhy{}0.0088   
2023\PYGZhy{}03\PYGZhy{}17 0.0002  0.0291 \PYGZhy{}0.0060 \PYGZhy{}0.0228 \PYGZhy{}0.0278  0.0038  0.0086  0.0134   
2023\PYGZhy{}03\PYGZhy{}20 0.0001  0.0004 \PYGZhy{}0.0027  0.0166  0.0155 \PYGZhy{}0.0010 \PYGZhy{}0.0067 \PYGZhy{}0.0081   
2023\PYGZhy{}03\PYGZhy{}21 0.0000 \PYGZhy{}0.0189  0.0114  0.0182  0.0165  0.0031 \PYGZhy{}0.0036 \PYGZhy{}0.0088   
2023\PYGZhy{}03\PYGZhy{}22 0.0002  0.0170  0.0011 \PYGZhy{}0.0254 \PYGZhy{}0.0268  0.0055  0.0141  0.0131   

Ticker         SPY  
Date                
2023\PYGZhy{}03\PYGZhy{}16  0.0175  
2023\PYGZhy{}03\PYGZhy{}17 \PYGZhy{}0.0117  
2023\PYGZhy{}03\PYGZhy{}20  0.0096  
2023\PYGZhy{}03\PYGZhy{}21  0.0131  
2023\PYGZhy{}03\PYGZhy{}22 \PYGZhy{}0.0170  
\end{sphinxVerbatim}

\end{sphinxuseclass}\end{sphinxVerbatimOutput}

\end{sphinxuseclass}
\begin{sphinxuseclass}{cell}\begin{sphinxVerbatimInput}

\begin{sphinxuseclass}{cell_input}
\begin{sphinxVerbatim}[commandchars=\\\{\}]
\PYG{n}{ff} \PYG{o}{=} \PYG{p}{(}
    \PYG{n}{pdr}\PYG{o}{.}\PYG{n}{DataReader}\PYG{p}{(}
        \PYG{n}{name}\PYG{o}{=}\PYG{l+s+s1}{\PYGZsq{}}\PYG{l+s+s1}{F\PYGZhy{}F\PYGZus{}Research\PYGZus{}Data\PYGZus{}Factors\PYGZus{}daily}\PYG{l+s+s1}{\PYGZsq{}}\PYG{p}{,}
        \PYG{n}{data\PYGZus{}source}\PYG{o}{=}\PYG{l+s+s1}{\PYGZsq{}}\PYG{l+s+s1}{famafrench}\PYG{l+s+s1}{\PYGZsq{}}\PYG{p}{,}
        \PYG{n}{start}\PYG{o}{=}\PYG{l+s+s1}{\PYGZsq{}}\PYG{l+s+s1}{1900}\PYG{l+s+s1}{\PYGZsq{}}\PYG{p}{,}
        \PYG{n}{session}\PYG{o}{=}\PYG{n}{session}
    \PYG{p}{)}
    \PYG{p}{[}\PYG{l+m+mi}{0}\PYG{p}{]}
    \PYG{o}{.}\PYG{n}{rename\PYGZus{}axis}\PYG{p}{(}\PYG{n}{columns}\PYG{o}{=}\PYG{l+s+s1}{\PYGZsq{}}\PYG{l+s+s1}{Variable}\PYG{l+s+s1}{\PYGZsq{}}\PYG{p}{)}
    \PYG{o}{.}\PYG{n}{div}\PYG{p}{(}\PYG{l+m+mi}{100}\PYG{p}{)}
\PYG{p}{)}

\PYG{n}{ff}\PYG{o}{.}\PYG{n}{tail}\PYG{p}{(}\PYG{p}{)}
\end{sphinxVerbatim}

\end{sphinxuseclass}\end{sphinxVerbatimInput}
\begin{sphinxVerbatimOutput}

\begin{sphinxuseclass}{cell_output}
\begin{sphinxVerbatim}[commandchars=\\\{\}]
Variable    Mkt\PYGZhy{}RF     SMB     HML     RF
Date                                     
2022\PYGZhy{}12\PYGZhy{}23  0.0051 \PYGZhy{}0.0060  0.0115 0.0002
2022\PYGZhy{}12\PYGZhy{}27 \PYGZhy{}0.0051 \PYGZhy{}0.0073  0.0142 0.0002
2022\PYGZhy{}12\PYGZhy{}28 \PYGZhy{}0.0123 \PYGZhy{}0.0025 \PYGZhy{}0.0029 0.0002
2022\PYGZhy{}12\PYGZhy{}29  0.0187  0.0127 \PYGZhy{}0.0107 0.0002
2022\PYGZhy{}12\PYGZhy{}30 \PYGZhy{}0.0022  0.0011 \PYGZhy{}0.0003 0.0002
\end{sphinxVerbatim}

\end{sphinxuseclass}\end{sphinxVerbatimOutput}

\end{sphinxuseclass}
\begin{sphinxuseclass}{cell}\begin{sphinxVerbatimInput}

\begin{sphinxuseclass}{cell_input}
\begin{sphinxVerbatim}[commandchars=\\\{\}]
\PYG{k}{def} \PYG{n+nf}{mean}\PYG{p}{(}\PYG{n}{ri}\PYG{p}{,} \PYG{n}{ann}\PYG{o}{=}\PYG{l+m+mi}{252}\PYG{p}{,} \PYG{n}{mul}\PYG{o}{=}\PYG{l+m+mi}{100}\PYG{p}{)}\PYG{p}{:}
    \PYG{k}{return} \PYG{n}{mul} \PYG{o}{*} \PYG{n}{ann} \PYG{o}{*} \PYG{n}{ri}\PYG{o}{.}\PYG{n}{mean}\PYG{p}{(}\PYG{p}{)}
\end{sphinxVerbatim}

\end{sphinxuseclass}\end{sphinxVerbatimInput}

\end{sphinxuseclass}
\begin{sphinxuseclass}{cell}\begin{sphinxVerbatimInput}

\begin{sphinxuseclass}{cell_input}
\begin{sphinxVerbatim}[commandchars=\\\{\}]
\PYG{k}{def} \PYG{n+nf}{beta}\PYG{p}{(}\PYG{n}{ri}\PYG{p}{,} \PYG{n}{rf}\PYG{o}{=}\PYG{n}{ff}\PYG{p}{[}\PYG{l+s+s1}{\PYGZsq{}}\PYG{l+s+s1}{RF}\PYG{l+s+s1}{\PYGZsq{}}\PYG{p}{]}\PYG{p}{,} \PYG{n}{rm\PYGZus{}rf}\PYG{o}{=}\PYG{n}{ff}\PYG{p}{[}\PYG{l+s+s1}{\PYGZsq{}}\PYG{l+s+s1}{Mkt\PYGZhy{}RF}\PYG{l+s+s1}{\PYGZsq{}}\PYG{p}{]}\PYG{p}{)}\PYG{p}{:}
    \PYG{n}{ri\PYGZus{}rf} \PYG{o}{=} \PYG{n}{ri}\PYG{o}{.}\PYG{n}{sub}\PYG{p}{(}\PYG{n}{rf}\PYG{p}{)}\PYG{o}{.}\PYG{n}{dropna}\PYG{p}{(}\PYG{p}{)}
    \PYG{k}{return} \PYG{n}{ri\PYGZus{}rf}\PYG{o}{.}\PYG{n}{cov}\PYG{p}{(}\PYG{n}{rm\PYGZus{}rf}\PYG{p}{)} \PYG{o}{/} \PYG{n}{rm\PYGZus{}rf}\PYG{o}{.}\PYG{n}{loc}\PYG{p}{[}\PYG{n}{ri\PYGZus{}rf}\PYG{o}{.}\PYG{n}{index}\PYG{p}{]}\PYG{o}{.}\PYG{n}{var}\PYG{p}{(}\PYG{p}{)}
\end{sphinxVerbatim}

\end{sphinxuseclass}\end{sphinxVerbatimInput}

\end{sphinxuseclass}
\begin{sphinxuseclass}{cell}\begin{sphinxVerbatimInput}

\begin{sphinxuseclass}{cell_input}
\begin{sphinxVerbatim}[commandchars=\\\{\}]
\PYG{k}{def} \PYG{n+nf}{date\PYGZus{}range}\PYG{p}{(}\PYG{n}{x}\PYG{p}{)}\PYG{p}{:}
    \PYG{k}{return} \PYG{l+s+sa}{f}\PYG{l+s+s1}{\PYGZsq{}}\PYG{l+s+si}{\PYGZob{}}\PYG{n}{x}\PYG{o}{.}\PYG{n}{index}\PYG{p}{[}\PYG{l+m+mi}{0}\PYG{p}{]}\PYG{l+s+si}{:}\PYG{l+s+s1}{\PYGZpc{}b \PYGZpc{}d, \PYGZpc{}Y}\PYG{l+s+si}{\PYGZcb{}}\PYG{l+s+s1}{ to }\PYG{l+s+si}{\PYGZob{}}\PYG{n}{x}\PYG{o}{.}\PYG{n}{index}\PYG{p}{[}\PYG{o}{\PYGZhy{}}\PYG{l+m+mi}{1}\PYG{p}{]}\PYG{l+s+si}{:}\PYG{l+s+s1}{\PYGZpc{}b \PYGZpc{}d, \PYGZpc{}Y}\PYG{l+s+si}{\PYGZcb{}}\PYG{l+s+s1}{\PYGZsq{}}
\end{sphinxVerbatim}

\end{sphinxuseclass}\end{sphinxVerbatimInput}

\end{sphinxuseclass}
\begin{sphinxuseclass}{cell}\begin{sphinxVerbatimInput}

\begin{sphinxuseclass}{cell_input}
\begin{sphinxVerbatim}[commandchars=\\\{\}]
\PYG{k+kn}{import} \PYG{n+nn}{seaborn} \PYG{k}{as} \PYG{n+nn}{sns}
\end{sphinxVerbatim}

\end{sphinxuseclass}\end{sphinxVerbatimInput}

\end{sphinxuseclass}
\begin{sphinxuseclass}{cell}\begin{sphinxVerbatimInput}

\begin{sphinxuseclass}{cell_input}
\begin{sphinxVerbatim}[commandchars=\\\{\}]
\PYG{n}{\PYGZus{}} \PYG{o}{=} \PYG{p}{(}
    \PYG{n}{etf}
    \PYG{o}{.}\PYG{n}{iloc}\PYG{p}{[}\PYG{o}{\PYGZhy{}}\PYG{l+m+mi}{756}\PYG{p}{:}\PYG{p}{]} \PYG{c+c1}{\PYGZsh{} I forget to slice to 3 years in class}
    \PYG{o}{.}\PYG{n}{agg}\PYG{p}{(}\PYG{p}{[}\PYG{n}{mean}\PYG{p}{,} \PYG{n}{beta}\PYG{p}{]}\PYG{p}{)}
    \PYG{o}{.}\PYG{n}{rename\PYGZus{}axis}\PYG{p}{(}\PYG{n}{index}\PYG{o}{=}\PYG{l+s+s1}{\PYGZsq{}}\PYG{l+s+s1}{Statistic}\PYG{l+s+s1}{\PYGZsq{}}\PYG{p}{)}
    \PYG{o}{.}\PYG{n}{transpose}\PYG{p}{(}\PYG{p}{)}
\PYG{p}{)}
 
\PYG{n}{sns}\PYG{o}{.}\PYG{n}{regplot}\PYG{p}{(}\PYG{n}{x}\PYG{o}{=}\PYG{l+s+s1}{\PYGZsq{}}\PYG{l+s+s1}{beta}\PYG{l+s+s1}{\PYGZsq{}}\PYG{p}{,} \PYG{n}{y}\PYG{o}{=}\PYG{l+s+s1}{\PYGZsq{}}\PYG{l+s+s1}{mean}\PYG{l+s+s1}{\PYGZsq{}}\PYG{p}{,} \PYG{n}{data}\PYG{o}{=}\PYG{n}{\PYGZus{}}\PYG{p}{)}
\PYG{k}{for} \PYG{n}{t}\PYG{p}{,} \PYG{p}{(}\PYG{n}{x}\PYG{p}{,} \PYG{n}{y}\PYG{p}{)} \PYG{o+ow}{in} \PYG{n}{\PYGZus{}}\PYG{p}{[}\PYG{p}{[}\PYG{l+s+s1}{\PYGZsq{}}\PYG{l+s+s1}{beta}\PYG{l+s+s1}{\PYGZsq{}}\PYG{p}{,} \PYG{l+s+s1}{\PYGZsq{}}\PYG{l+s+s1}{mean}\PYG{l+s+s1}{\PYGZsq{}}\PYG{p}{]}\PYG{p}{]}\PYG{o}{.}\PYG{n}{iterrows}\PYG{p}{(}\PYG{p}{)}\PYG{p}{:}
    \PYG{n}{plt}\PYG{o}{.}\PYG{n}{annotate}\PYG{p}{(}\PYG{n}{text}\PYG{o}{=}\PYG{n}{t}\PYG{p}{,} \PYG{n}{xy}\PYG{o}{=}\PYG{p}{(}\PYG{n}{x}\PYG{p}{,} \PYG{n}{y}\PYG{p}{)}\PYG{p}{)} \PYG{c+c1}{\PYGZsh{} use a for\PYGZhy{}loop to add tickers}

\PYG{n}{plt}\PYG{o}{.}\PYG{n}{ylabel}\PYG{p}{(}\PYG{l+s+s1}{\PYGZsq{}}\PYG{l+s+s1}{Annualized Mean of Daily Returns (}\PYG{l+s+s1}{\PYGZpc{}}\PYG{l+s+s1}{)}\PYG{l+s+s1}{\PYGZsq{}}\PYG{p}{)}
\PYG{n}{plt}\PYG{o}{.}\PYG{n}{xlabel}\PYG{p}{(}\PYG{l+s+sa}{r}\PYG{l+s+s1}{\PYGZsq{}}\PYG{l+s+s1}{Capital Asset Pricing Model (CAPM) \PYGZdl{}}\PYG{l+s+s1}{\PYGZbs{}}\PYG{l+s+s1}{beta\PYGZdl{}}\PYG{l+s+s1}{\PYGZsq{}}\PYG{p}{)}
\PYG{n}{plt}\PYG{o}{.}\PYG{n}{title}\PYG{p}{(}
    \PYG{l+s+s1}{\PYGZsq{}}\PYG{l+s+s1}{Security Market Line (SML) for Exchange Traded Funds (ETFs)}\PYG{l+s+se}{\PYGZbs{}n}\PYG{l+s+s1}{\PYGZsq{}} \PYG{o}{+}
    \PYG{l+s+s1}{\PYGZsq{}}\PYG{l+s+s1}{for Daily Returns from }\PYG{l+s+s1}{\PYGZsq{}} \PYG{o}{+} \PYG{n}{etf}\PYG{o}{.}\PYG{n}{iloc}\PYG{p}{[}\PYG{o}{\PYGZhy{}}\PYG{l+m+mi}{756}\PYG{p}{:}\PYG{p}{]}\PYG{o}{.}\PYG{n}{pipe}\PYG{p}{(}\PYG{n}{date\PYGZus{}range}\PYG{p}{)}\PYG{p}{)}
\PYG{n}{plt}\PYG{o}{.}\PYG{n}{show}\PYG{p}{(}\PYG{p}{)}
\end{sphinxVerbatim}

\end{sphinxuseclass}\end{sphinxVerbatimInput}
\begin{sphinxVerbatimOutput}

\begin{sphinxuseclass}{cell_output}
\noindent\sphinxincludegraphics{{3b6dd06cf96dae0129f622e33c9115bdd267739d4c089478c34f7a149757ffce}.png}

\end{sphinxuseclass}\end{sphinxVerbatimOutput}

\end{sphinxuseclass}

\subsubsection{Plot the SML for the Dow Jones Industrial Average (DJIA) stocks}
\label{\detokenize{herron_03_practice_04:plot-the-sml-for-the-dow-jones-industrial-average-djia-stocks}}
\sphinxAtStartPar
Use the past three years of daily returns data for the stocks listed on the \sphinxhref{https://en.wikipedia.org/wiki/Dow\_Jones\_Industrial\_Average}{DJIA Wikipedia page}.
Compare the DJIA SML to the asset class SML above.

\begin{sphinxuseclass}{cell}\begin{sphinxVerbatimInput}

\begin{sphinxuseclass}{cell_input}
\begin{sphinxVerbatim}[commandchars=\\\{\}]
\PYG{n}{wiki} \PYG{o}{=} \PYG{n}{pd}\PYG{o}{.}\PYG{n}{read\PYGZus{}html}\PYG{p}{(}\PYG{l+s+s1}{\PYGZsq{}}\PYG{l+s+s1}{https://en.wikipedia.org/wiki/Dow\PYGZus{}Jones\PYGZus{}Industrial\PYGZus{}Average}\PYG{l+s+s1}{\PYGZsq{}}\PYG{p}{)}
\end{sphinxVerbatim}

\end{sphinxuseclass}\end{sphinxVerbatimInput}

\end{sphinxuseclass}
\begin{sphinxuseclass}{cell}\begin{sphinxVerbatimInput}

\begin{sphinxuseclass}{cell_input}
\begin{sphinxVerbatim}[commandchars=\\\{\}]
\PYG{n}{djia} \PYG{o}{=} \PYG{p}{(}
    \PYG{n}{yf}\PYG{o}{.}\PYG{n}{download}\PYG{p}{(}
        \PYG{n}{tickers}\PYG{o}{=}\PYG{n}{wiki}\PYG{p}{[}\PYG{l+m+mi}{1}\PYG{p}{]}\PYG{p}{[}\PYG{l+s+s1}{\PYGZsq{}}\PYG{l+s+s1}{Symbol}\PYG{l+s+s1}{\PYGZsq{}}\PYG{p}{]}\PYG{o}{.}\PYG{n}{to\PYGZus{}list}\PYG{p}{(}\PYG{p}{)}\PYG{p}{,}
        \PYG{n}{progress}\PYG{o}{=}\PYG{k+kc}{False}
    \PYG{p}{)}
    \PYG{o}{.}\PYG{n}{assign}\PYG{p}{(}\PYG{n}{Date} \PYG{o}{=} \PYG{k}{lambda} \PYG{n}{x}\PYG{p}{:} \PYG{n}{x}\PYG{o}{.}\PYG{n}{index}\PYG{o}{.}\PYG{n}{tz\PYGZus{}localize}\PYG{p}{(}\PYG{k+kc}{None}\PYG{p}{)}\PYG{p}{)}
    \PYG{o}{.}\PYG{n}{set\PYGZus{}index}\PYG{p}{(}\PYG{l+s+s1}{\PYGZsq{}}\PYG{l+s+s1}{Date}\PYG{l+s+s1}{\PYGZsq{}}\PYG{p}{)}
    \PYG{o}{.}\PYG{n}{rename\PYGZus{}axis}\PYG{p}{(}\PYG{n}{columns}\PYG{o}{=}\PYG{p}{[}\PYG{l+s+s1}{\PYGZsq{}}\PYG{l+s+s1}{Variable}\PYG{l+s+s1}{\PYGZsq{}}\PYG{p}{,} \PYG{l+s+s1}{\PYGZsq{}}\PYG{l+s+s1}{Ticker}\PYG{l+s+s1}{\PYGZsq{}}\PYG{p}{]}\PYG{p}{)}
    \PYG{p}{[}\PYG{l+s+s1}{\PYGZsq{}}\PYG{l+s+s1}{Adj Close}\PYG{l+s+s1}{\PYGZsq{}}\PYG{p}{]}
    \PYG{o}{.}\PYG{n}{pct\PYGZus{}change}\PYG{p}{(}\PYG{p}{)}
    \PYG{o}{.}\PYG{n}{dropna}\PYG{p}{(}\PYG{p}{)}
\PYG{p}{)}
\end{sphinxVerbatim}

\end{sphinxuseclass}\end{sphinxVerbatimInput}

\end{sphinxuseclass}
\begin{sphinxuseclass}{cell}\begin{sphinxVerbatimInput}

\begin{sphinxuseclass}{cell_input}
\begin{sphinxVerbatim}[commandchars=\\\{\}]
\PYG{n}{\PYGZus{}} \PYG{o}{=} \PYG{p}{(}
    \PYG{n}{djia}
    \PYG{o}{.}\PYG{n}{iloc}\PYG{p}{[}\PYG{o}{\PYGZhy{}}\PYG{l+m+mi}{756}\PYG{p}{:}\PYG{p}{]} \PYG{c+c1}{\PYGZsh{} I forget to slice to 3 years in class}
    \PYG{o}{.}\PYG{n}{agg}\PYG{p}{(}\PYG{p}{[}\PYG{n}{mean}\PYG{p}{,} \PYG{n}{beta}\PYG{p}{]}\PYG{p}{)}
    \PYG{o}{.}\PYG{n}{rename\PYGZus{}axis}\PYG{p}{(}\PYG{n}{index}\PYG{o}{=}\PYG{l+s+s1}{\PYGZsq{}}\PYG{l+s+s1}{Statistic}\PYG{l+s+s1}{\PYGZsq{}}\PYG{p}{)}
    \PYG{o}{.}\PYG{n}{transpose}\PYG{p}{(}\PYG{p}{)}
\PYG{p}{)}
 
\PYG{n}{sns}\PYG{o}{.}\PYG{n}{regplot}\PYG{p}{(}\PYG{n}{x}\PYG{o}{=}\PYG{l+s+s1}{\PYGZsq{}}\PYG{l+s+s1}{beta}\PYG{l+s+s1}{\PYGZsq{}}\PYG{p}{,} \PYG{n}{y}\PYG{o}{=}\PYG{l+s+s1}{\PYGZsq{}}\PYG{l+s+s1}{mean}\PYG{l+s+s1}{\PYGZsq{}}\PYG{p}{,} \PYG{n}{data}\PYG{o}{=}\PYG{n}{\PYGZus{}}\PYG{p}{)}
\PYG{k}{for} \PYG{n}{t}\PYG{p}{,} \PYG{p}{(}\PYG{n}{x}\PYG{p}{,} \PYG{n}{y}\PYG{p}{)} \PYG{o+ow}{in} \PYG{n}{\PYGZus{}}\PYG{p}{[}\PYG{p}{[}\PYG{l+s+s1}{\PYGZsq{}}\PYG{l+s+s1}{beta}\PYG{l+s+s1}{\PYGZsq{}}\PYG{p}{,} \PYG{l+s+s1}{\PYGZsq{}}\PYG{l+s+s1}{mean}\PYG{l+s+s1}{\PYGZsq{}}\PYG{p}{]}\PYG{p}{]}\PYG{o}{.}\PYG{n}{iterrows}\PYG{p}{(}\PYG{p}{)}\PYG{p}{:}
    \PYG{n}{plt}\PYG{o}{.}\PYG{n}{annotate}\PYG{p}{(}\PYG{n}{text}\PYG{o}{=}\PYG{n}{t}\PYG{p}{,} \PYG{n}{xy}\PYG{o}{=}\PYG{p}{(}\PYG{n}{x}\PYG{p}{,} \PYG{n}{y}\PYG{p}{)}\PYG{p}{)} \PYG{c+c1}{\PYGZsh{} use a for\PYGZhy{}loop to add tickers}

\PYG{n}{plt}\PYG{o}{.}\PYG{n}{ylabel}\PYG{p}{(}\PYG{l+s+s1}{\PYGZsq{}}\PYG{l+s+s1}{Annualized Mean of Daily Returns (}\PYG{l+s+s1}{\PYGZpc{}}\PYG{l+s+s1}{)}\PYG{l+s+s1}{\PYGZsq{}}\PYG{p}{)}
\PYG{n}{plt}\PYG{o}{.}\PYG{n}{xlabel}\PYG{p}{(}\PYG{l+s+sa}{r}\PYG{l+s+s1}{\PYGZsq{}}\PYG{l+s+s1}{Capital Asset Pricing Model (CAPM) \PYGZdl{}}\PYG{l+s+s1}{\PYGZbs{}}\PYG{l+s+s1}{beta\PYGZdl{}}\PYG{l+s+s1}{\PYGZsq{}}\PYG{p}{)}
\PYG{n}{plt}\PYG{o}{.}\PYG{n}{title}\PYG{p}{(}
    \PYG{l+s+s1}{\PYGZsq{}}\PYG{l+s+s1}{Security Market Line (SML) for Dow\PYGZhy{}Jones Stocks}\PYG{l+s+se}{\PYGZbs{}n}\PYG{l+s+s1}{\PYGZsq{}} \PYG{o}{+}
    \PYG{l+s+s1}{\PYGZsq{}}\PYG{l+s+s1}{for Daily Returns from }\PYG{l+s+s1}{\PYGZsq{}} \PYG{o}{+} \PYG{n}{djia}\PYG{o}{.}\PYG{n}{iloc}\PYG{p}{[}\PYG{o}{\PYGZhy{}}\PYG{l+m+mi}{756}\PYG{p}{:}\PYG{p}{]}\PYG{o}{.}\PYG{n}{pipe}\PYG{p}{(}\PYG{n}{date\PYGZus{}range}\PYG{p}{)}\PYG{p}{)}
\PYG{n}{plt}\PYG{o}{.}\PYG{n}{show}\PYG{p}{(}\PYG{p}{)}
\end{sphinxVerbatim}

\end{sphinxuseclass}\end{sphinxVerbatimInput}
\begin{sphinxVerbatimOutput}

\begin{sphinxuseclass}{cell_output}
\noindent\sphinxincludegraphics{{c4cc55f421fd3b9df7b122f1985ba02aa080ba364eb574acf84d1f940906d8de}.png}

\end{sphinxuseclass}\end{sphinxVerbatimOutput}

\end{sphinxuseclass}

\subsubsection{Plot the SML for the five portfolios formed on beta}
\label{\detokenize{herron_03_practice_04:plot-the-sml-for-the-five-portfolios-formed-on-beta}}
\sphinxAtStartPar
Download data for portfolios formed on \$\textbackslash{}beta\$ (\sphinxcode{\sphinxupquote{Portfolios\_Formed\_on\_BETA}}) from Ken French.
For the value\sphinxhyphen{}weighted portfolios, plot realized returns versus \$\textbackslash{}beta\$.
These data should elements \sphinxcode{\sphinxupquote{{[}2{]}}} and \sphinxcode{\sphinxupquote{{[}6{]}}}, respectively.

\begin{sphinxuseclass}{cell}\begin{sphinxVerbatimInput}

\begin{sphinxuseclass}{cell_input}
\begin{sphinxVerbatim}[commandchars=\\\{\}]
\PYG{n}{beta\PYGZus{}0} \PYG{o}{=} \PYG{p}{(}
    \PYG{n}{pdr}\PYG{o}{.}\PYG{n}{DataReader}\PYG{p}{(}
        \PYG{n}{name}\PYG{o}{=}\PYG{l+s+s1}{\PYGZsq{}}\PYG{l+s+s1}{Portfolios\PYGZus{}Formed\PYGZus{}on\PYGZus{}BETA}\PYG{l+s+s1}{\PYGZsq{}}\PYG{p}{,}
        \PYG{n}{data\PYGZus{}source}\PYG{o}{=}\PYG{l+s+s1}{\PYGZsq{}}\PYG{l+s+s1}{famafrench}\PYG{l+s+s1}{\PYGZsq{}}\PYG{p}{,}
        \PYG{n}{start}\PYG{o}{=}\PYG{l+s+s1}{\PYGZsq{}}\PYG{l+s+s1}{1900}\PYG{l+s+s1}{\PYGZsq{}}\PYG{p}{,}
        \PYG{n}{session}\PYG{o}{=}\PYG{n}{session}
    \PYG{p}{)}
\PYG{p}{)}

\PYG{n+nb}{print}\PYG{p}{(}\PYG{n}{beta\PYGZus{}0}\PYG{p}{[}\PYG{l+s+s1}{\PYGZsq{}}\PYG{l+s+s1}{DESCR}\PYG{l+s+s1}{\PYGZsq{}}\PYG{p}{]}\PYG{p}{)}
\end{sphinxVerbatim}

\end{sphinxuseclass}\end{sphinxVerbatimInput}
\begin{sphinxVerbatimOutput}

\begin{sphinxuseclass}{cell_output}
\begin{sphinxVerbatim}[commandchars=\\\{\}]
Portfolios Formed on BETA
\PYGZhy{}\PYGZhy{}\PYGZhy{}\PYGZhy{}\PYGZhy{}\PYGZhy{}\PYGZhy{}\PYGZhy{}\PYGZhy{}\PYGZhy{}\PYGZhy{}\PYGZhy{}\PYGZhy{}\PYGZhy{}\PYGZhy{}\PYGZhy{}\PYGZhy{}\PYGZhy{}\PYGZhy{}\PYGZhy{}\PYGZhy{}\PYGZhy{}\PYGZhy{}\PYGZhy{}\PYGZhy{}

This file was created by CMPT\PYGZus{}BETA\PYGZus{}RETS using the 202301 CRSP database. It contains value\PYGZhy{} and equal\PYGZhy{}weighted returns for portfolios formed on BETA. The portfolios are constructed at the end of June. Beta is estimated using monthly returns for the past 60 months (requiring at least 24 months with non\PYGZhy{}missing returns). Beta is estimated using the Scholes\PYGZhy{}Williams method. Annual returns are from January to December. Missing data are indicated by \PYGZhy{}99.99 or \PYGZhy{}999. The break points include utilities and include financials. The portfolios include utilities and include financials. Copyright 2023 Kenneth R. French

  0 : Value Weighted Returns \PYGZhy{}\PYGZhy{} Monthly (715 rows x 15 cols)
  1 : Equal Weighted Returns \PYGZhy{}\PYGZhy{} Monthly (715 rows x 15 cols)
  2 : Value Weighted Returns \PYGZhy{}\PYGZhy{} Annual from January to December (59 rows x 15 cols)
  3 : Equal Weighted Returns \PYGZhy{}\PYGZhy{} Annual from January to December (59 rows x 15 cols)
  4 : Number of Firms in Portfolios (715 rows x 15 cols)
  5 : Average Firm Size (715 rows x 15 cols)
  6 : Value\PYGZhy{}Weighted Average of Prior Beta (60 rows x 15 cols)
\end{sphinxVerbatim}

\end{sphinxuseclass}\end{sphinxVerbatimOutput}

\end{sphinxuseclass}
\begin{sphinxuseclass}{cell}\begin{sphinxVerbatimInput}

\begin{sphinxuseclass}{cell_input}
\begin{sphinxVerbatim}[commandchars=\\\{\}]
\PYG{n}{ports} \PYG{o}{=} \PYG{p}{[}\PYG{l+s+s1}{\PYGZsq{}}\PYG{l+s+s1}{Lo 20}\PYG{l+s+s1}{\PYGZsq{}}\PYG{p}{,} \PYG{l+s+s1}{\PYGZsq{}}\PYG{l+s+s1}{Qnt 2}\PYG{l+s+s1}{\PYGZsq{}}\PYG{p}{,} \PYG{l+s+s1}{\PYGZsq{}}\PYG{l+s+s1}{Qnt 3}\PYG{l+s+s1}{\PYGZsq{}}\PYG{p}{,} \PYG{l+s+s1}{\PYGZsq{}}\PYG{l+s+s1}{Qnt 4}\PYG{l+s+s1}{\PYGZsq{}}\PYG{p}{,} \PYG{l+s+s1}{\PYGZsq{}}\PYG{l+s+s1}{Hi 20}\PYG{l+s+s1}{\PYGZsq{}}\PYG{p}{]}

\PYG{n}{\PYGZus{}} \PYG{o}{=} \PYG{p}{(}
    \PYG{n}{pd}\PYG{o}{.}\PYG{n}{concat}\PYG{p}{(}
        \PYG{n}{objs}\PYG{o}{=}\PYG{p}{[}
            \PYG{n}{beta\PYGZus{}0}\PYG{p}{[}\PYG{l+m+mi}{2}\PYG{p}{]}\PYG{p}{[}\PYG{n}{ports}\PYG{p}{]}\PYG{p}{,}
            \PYG{n}{beta\PYGZus{}0}\PYG{p}{[}\PYG{l+m+mi}{6}\PYG{p}{]}\PYG{p}{[}\PYG{n}{ports}\PYG{p}{]}\PYG{o}{.}\PYG{n}{shift}\PYG{p}{(}\PYG{p}{)}
        \PYG{p}{]}\PYG{p}{,}
        \PYG{n}{axis}\PYG{o}{=}\PYG{l+m+mi}{1}\PYG{p}{,}
        \PYG{n}{keys}\PYG{o}{=}\PYG{p}{[}\PYG{l+s+s1}{\PYGZsq{}}\PYG{l+s+s1}{Return}\PYG{l+s+s1}{\PYGZsq{}}\PYG{p}{,} \PYG{l+s+s1}{\PYGZsq{}}\PYG{l+s+s1}{Beta}\PYG{l+s+s1}{\PYGZsq{}}\PYG{p}{]}\PYG{p}{,}
        \PYG{n}{names}\PYG{o}{=}\PYG{p}{[}\PYG{l+s+s1}{\PYGZsq{}}\PYG{l+s+s1}{Statistic}\PYG{l+s+s1}{\PYGZsq{}}\PYG{p}{,} \PYG{l+s+s1}{\PYGZsq{}}\PYG{l+s+s1}{Beta Portfolio}\PYG{l+s+s1}{\PYGZsq{}}\PYG{p}{]}
    \PYG{p}{)}
    \PYG{o}{.}\PYG{n}{stack}\PYG{p}{(}\PYG{l+s+s1}{\PYGZsq{}}\PYG{l+s+s1}{Beta Portfolio}\PYG{l+s+s1}{\PYGZsq{}}\PYG{p}{)}
\PYG{p}{)}

\PYG{n}{sns}\PYG{o}{.}\PYG{n}{regplot}\PYG{p}{(}\PYG{n}{x}\PYG{o}{=}\PYG{l+s+s1}{\PYGZsq{}}\PYG{l+s+s1}{Beta}\PYG{l+s+s1}{\PYGZsq{}}\PYG{p}{,} \PYG{n}{y}\PYG{o}{=}\PYG{l+s+s1}{\PYGZsq{}}\PYG{l+s+s1}{Return}\PYG{l+s+s1}{\PYGZsq{}}\PYG{p}{,} \PYG{n}{data}\PYG{o}{=}\PYG{n}{\PYGZus{}}\PYG{p}{)}
\PYG{n}{plt}\PYG{o}{.}\PYG{n}{ylabel}\PYG{p}{(}\PYG{l+s+s1}{\PYGZsq{}}\PYG{l+s+s1}{Mean Annual Return (}\PYG{l+s+s1}{\PYGZpc{}}\PYG{l+s+s1}{)}\PYG{l+s+s1}{\PYGZsq{}}\PYG{p}{)}
\PYG{n}{plt}\PYG{o}{.}\PYG{n}{xlabel}\PYG{p}{(}\PYG{l+s+sa}{r}\PYG{l+s+s1}{\PYGZsq{}}\PYG{l+s+s1}{Capital Asset Pricing Model (CAPM) \PYGZdl{}}\PYG{l+s+s1}{\PYGZbs{}}\PYG{l+s+s1}{beta\PYGZdl{}}\PYG{l+s+s1}{\PYGZsq{}}\PYG{p}{)}
\PYG{n}{plt}\PYG{o}{.}\PYG{n}{title}\PYG{p}{(}
    \PYG{l+s+sa}{r}\PYG{l+s+s1}{\PYGZsq{}}\PYG{l+s+s1}{Security Market Line (SML) for \PYGZdl{}}\PYG{l+s+s1}{\PYGZbs{}}\PYG{l+s+s1}{beta\PYGZdl{} Portfolios}\PYG{l+s+s1}{\PYGZsq{}} \PYG{o}{+}
    \PYG{l+s+sa}{f}\PYG{l+s+s1}{\PYGZsq{}}\PYG{l+s+se}{\PYGZbs{}n}\PYG{l+s+s1}{for Annual Returns from }\PYG{l+s+si}{\PYGZob{}}\PYG{n}{\PYGZus{}}\PYG{o}{.}\PYG{n}{index}\PYG{o}{.}\PYG{n}{get\PYGZus{}level\PYGZus{}values}\PYG{p}{(}\PYG{l+m+mi}{0}\PYG{p}{)}\PYG{o}{.}\PYG{n}{year}\PYG{p}{[}\PYG{l+m+mi}{0}\PYG{p}{]}\PYG{l+s+si}{\PYGZcb{}}\PYG{l+s+s1}{ to }\PYG{l+s+si}{\PYGZob{}}\PYG{n}{\PYGZus{}}\PYG{o}{.}\PYG{n}{index}\PYG{o}{.}\PYG{n}{get\PYGZus{}level\PYGZus{}values}\PYG{p}{(}\PYG{l+m+mi}{0}\PYG{p}{)}\PYG{o}{.}\PYG{n}{year}\PYG{p}{[}\PYG{o}{\PYGZhy{}}\PYG{l+m+mi}{1}\PYG{p}{]}\PYG{l+s+si}{\PYGZcb{}}\PYG{l+s+s1}{\PYGZsq{}}
\PYG{p}{)}
\PYG{n}{plt}\PYG{o}{.}\PYG{n}{show}\PYG{p}{(}\PYG{p}{)}
\end{sphinxVerbatim}

\end{sphinxuseclass}\end{sphinxVerbatimInput}
\begin{sphinxVerbatimOutput}

\begin{sphinxuseclass}{cell_output}
\noindent\sphinxincludegraphics{{4e73d9e17f908c1760bb04097c69c5cc6ead8ccaeff577bfbca9066bc7a3666d}.png}

\end{sphinxuseclass}\end{sphinxVerbatimOutput}

\end{sphinxuseclass}

\subsubsection{Estimate the CAPM \$\textbackslash{}beta\$s on several levered and inverse exchange traded funds (ETFs)}
\label{\detokenize{herron_03_practice_04:estimate-the-capm-beta-s-on-several-levered-and-inverse-exchange-traded-funds-etfs}}
\sphinxAtStartPar
Try the following ETFs:
\begin{enumerate}
\sphinxsetlistlabels{\arabic}{enumi}{enumii}{}{.}%
\item {} 
\sphinxAtStartPar
SPY

\item {} 
\sphinxAtStartPar
UPRO

\item {} 
\sphinxAtStartPar
SPXU

\end{enumerate}

\sphinxAtStartPar
Can you determine what these products do from the data alone?
Estimate \$\textbackslash{}beta\$s and plot cumulative returns.
You may want to pick short periods of time with large market swings.

\begin{sphinxuseclass}{cell}\begin{sphinxVerbatimInput}

\begin{sphinxuseclass}{cell_input}
\begin{sphinxVerbatim}[commandchars=\\\{\}]
\PYG{n}{etf\PYGZus{}2} \PYG{o}{=} \PYG{p}{(}
    \PYG{n}{yf}\PYG{o}{.}\PYG{n}{download}\PYG{p}{(}
        \PYG{n}{tickers}\PYG{o}{=}\PYG{l+s+s1}{\PYGZsq{}}\PYG{l+s+s1}{SPY UPRO SPXU}\PYG{l+s+s1}{\PYGZsq{}}\PYG{p}{,}
        \PYG{n}{progress}\PYG{o}{=}\PYG{k+kc}{False}
    \PYG{p}{)}
    \PYG{o}{.}\PYG{n}{assign}\PYG{p}{(}\PYG{n}{Date} \PYG{o}{=} \PYG{k}{lambda} \PYG{n}{x}\PYG{p}{:} \PYG{n}{x}\PYG{o}{.}\PYG{n}{index}\PYG{o}{.}\PYG{n}{tz\PYGZus{}localize}\PYG{p}{(}\PYG{k+kc}{None}\PYG{p}{)}\PYG{p}{)}
    \PYG{o}{.}\PYG{n}{set\PYGZus{}index}\PYG{p}{(}\PYG{l+s+s1}{\PYGZsq{}}\PYG{l+s+s1}{Date}\PYG{l+s+s1}{\PYGZsq{}}\PYG{p}{)}
    \PYG{o}{.}\PYG{n}{rename\PYGZus{}axis}\PYG{p}{(}\PYG{n}{columns}\PYG{o}{=}\PYG{p}{[}\PYG{l+s+s1}{\PYGZsq{}}\PYG{l+s+s1}{Variable}\PYG{l+s+s1}{\PYGZsq{}}\PYG{p}{,} \PYG{l+s+s1}{\PYGZsq{}}\PYG{l+s+s1}{Ticker}\PYG{l+s+s1}{\PYGZsq{}}\PYG{p}{]}\PYG{p}{)}
    \PYG{p}{[}\PYG{l+s+s1}{\PYGZsq{}}\PYG{l+s+s1}{Adj Close}\PYG{l+s+s1}{\PYGZsq{}}\PYG{p}{]}
    \PYG{o}{.}\PYG{n}{pct\PYGZus{}change}\PYG{p}{(}\PYG{p}{)}
    \PYG{o}{.}\PYG{n}{dropna}\PYG{p}{(}\PYG{p}{)}
\PYG{p}{)}
\end{sphinxVerbatim}

\end{sphinxuseclass}\end{sphinxVerbatimInput}

\end{sphinxuseclass}
\begin{sphinxuseclass}{cell}\begin{sphinxVerbatimInput}

\begin{sphinxuseclass}{cell_input}
\begin{sphinxVerbatim}[commandchars=\\\{\}]
\PYG{n}{etf\PYGZus{}2}\PYG{o}{.}\PYG{n}{apply}\PYG{p}{(}\PYG{n}{beta}\PYG{p}{)}\PYG{o}{.}\PYG{n}{rename}\PYG{p}{(}\PYG{l+s+sa}{r}\PYG{l+s+s1}{\PYGZsq{}}\PYG{l+s+s1}{\PYGZdl{}}\PYG{l+s+s1}{\PYGZbs{}}\PYG{l+s+s1}{beta\PYGZdl{}}\PYG{l+s+s1}{\PYGZsq{}}\PYG{p}{)}
\end{sphinxVerbatim}

\end{sphinxuseclass}\end{sphinxVerbatimInput}
\begin{sphinxVerbatimOutput}

\begin{sphinxuseclass}{cell_output}
\begin{sphinxVerbatim}[commandchars=\\\{\}]
Ticker
SPXU   \PYGZhy{}2.8660
SPY     0.9586
UPRO    2.8814
Name: \PYGZdl{}\PYGZbs{}beta\PYGZdl{}, dtype: float64
\end{sphinxVerbatim}

\end{sphinxuseclass}\end{sphinxVerbatimOutput}

\end{sphinxuseclass}
\begin{sphinxuseclass}{cell}\begin{sphinxVerbatimInput}

\begin{sphinxuseclass}{cell_input}
\begin{sphinxVerbatim}[commandchars=\\\{\}]
\PYG{n}{etf\PYGZus{}2}\PYG{o}{.}\PYG{n}{apply}\PYG{p}{(}\PYG{n}{beta}\PYG{p}{)}\PYG{o}{.}\PYG{n}{plot}\PYG{p}{(}\PYG{n}{kind}\PYG{o}{=}\PYG{l+s+s1}{\PYGZsq{}}\PYG{l+s+s1}{bar}\PYG{l+s+s1}{\PYGZsq{}}\PYG{p}{)}
\PYG{n}{plt}\PYG{o}{.}\PYG{n}{xticks}\PYG{p}{(}\PYG{n}{rotation}\PYG{o}{=}\PYG{l+m+mi}{0}\PYG{p}{)}
\PYG{n}{plt}\PYG{o}{.}\PYG{n}{ylabel}\PYG{p}{(}\PYG{l+s+sa}{r}\PYG{l+s+s1}{\PYGZsq{}}\PYG{l+s+s1}{Capital Asset Pricing Model (CAPM) \PYGZdl{}}\PYG{l+s+s1}{\PYGZbs{}}\PYG{l+s+s1}{beta\PYGZdl{}}\PYG{l+s+s1}{\PYGZsq{}}\PYG{p}{)}
\PYG{n}{plt}\PYG{o}{.}\PYG{n}{title}\PYG{p}{(}
    \PYG{l+s+sa}{r}\PYG{l+s+s1}{\PYGZsq{}}\PYG{l+s+s1}{Capital Asset Pricing Model (CAPM) \PYGZdl{}}\PYG{l+s+s1}{\PYGZbs{}}\PYG{l+s+s1}{beta\PYGZdl{}s}\PYG{l+s+s1}{\PYGZsq{}} \PYG{o}{+} 
    \PYG{l+s+sa}{f}\PYG{l+s+s1}{\PYGZsq{}}\PYG{l+s+se}{\PYGZbs{}n}\PYG{l+s+s1}{from Daily Returns from }\PYG{l+s+si}{\PYGZob{}}\PYG{n}{etf\PYGZus{}2}\PYG{o}{.}\PYG{n}{pipe}\PYG{p}{(}\PYG{n}{date\PYGZus{}range}\PYG{p}{)}\PYG{l+s+si}{\PYGZcb{}}\PYG{l+s+s1}{\PYGZsq{}}
\PYG{p}{)}
\PYG{n}{plt}\PYG{o}{.}\PYG{n}{show}\PYG{p}{(}\PYG{p}{)}
\end{sphinxVerbatim}

\end{sphinxuseclass}\end{sphinxVerbatimInput}
\begin{sphinxVerbatimOutput}

\begin{sphinxuseclass}{cell_output}
\noindent\sphinxincludegraphics{{e704518cf36499e9d04b29af9257d6cb57f3a78cdd8fbda4550483071f8896d4}.png}

\end{sphinxuseclass}\end{sphinxVerbatimOutput}

\end{sphinxuseclass}

\subsubsection{Explore the size factor}
\label{\detokenize{herron_03_practice_04:explore-the-size-factor}}

\paragraph{Estimate \$\textbackslash{}alpha\$s for the ten portfolios formed on size}
\label{\detokenize{herron_03_practice_04:estimate-alpha-s-for-the-ten-portfolios-formed-on-size}}
\sphinxAtStartPar
Academics started researching size\sphinxhyphen{}based portfolios in the early 1980s, so you may want to focus on the pre\sphinxhyphen{}1980 sample.

\begin{sphinxuseclass}{cell}\begin{sphinxVerbatimInput}

\begin{sphinxuseclass}{cell_input}
\begin{sphinxVerbatim}[commandchars=\\\{\}]
\PYG{n}{size\PYGZus{}0} \PYG{o}{=} \PYG{n}{pdr}\PYG{o}{.}\PYG{n}{DataReader}\PYG{p}{(}
    \PYG{n}{name}\PYG{o}{=}\PYG{l+s+s1}{\PYGZsq{}}\PYG{l+s+s1}{Portfolios\PYGZus{}Formed\PYGZus{}on\PYGZus{}ME}\PYG{l+s+s1}{\PYGZsq{}}\PYG{p}{,}
    \PYG{n}{data\PYGZus{}source}\PYG{o}{=}\PYG{l+s+s1}{\PYGZsq{}}\PYG{l+s+s1}{famafrench}\PYG{l+s+s1}{\PYGZsq{}}\PYG{p}{,}
    \PYG{n}{start}\PYG{o}{=}\PYG{l+s+s1}{\PYGZsq{}}\PYG{l+s+s1}{1900}\PYG{l+s+s1}{\PYGZsq{}}\PYG{p}{,}
    \PYG{n}{session}\PYG{o}{=}\PYG{n}{session}
\PYG{p}{)}

\PYG{n+nb}{print}\PYG{p}{(}\PYG{n}{size\PYGZus{}0}\PYG{p}{[}\PYG{l+s+s1}{\PYGZsq{}}\PYG{l+s+s1}{DESCR}\PYG{l+s+s1}{\PYGZsq{}}\PYG{p}{]}\PYG{p}{)}
\end{sphinxVerbatim}

\end{sphinxuseclass}\end{sphinxVerbatimInput}
\begin{sphinxVerbatimOutput}

\begin{sphinxuseclass}{cell_output}
\begin{sphinxVerbatim}[commandchars=\\\{\}]
Portfolios Formed on ME
\PYGZhy{}\PYGZhy{}\PYGZhy{}\PYGZhy{}\PYGZhy{}\PYGZhy{}\PYGZhy{}\PYGZhy{}\PYGZhy{}\PYGZhy{}\PYGZhy{}\PYGZhy{}\PYGZhy{}\PYGZhy{}\PYGZhy{}\PYGZhy{}\PYGZhy{}\PYGZhy{}\PYGZhy{}\PYGZhy{}\PYGZhy{}\PYGZhy{}\PYGZhy{}

This file was created by CMPT\PYGZus{}ME\PYGZus{}RETS using the 202301 CRSP database. It contains value\PYGZhy{} and equal\PYGZhy{}weighted returns for size portfolios. Each record contains returns for: Negative (not used) 30\PYGZpc{} 40\PYGZpc{} 30\PYGZpc{}   5 Quintiles  10 Deciles The portfolios are constructed at the end of Jun. The annual returns are from January to December. Missing data are indicated by \PYGZhy{}99.99 or \PYGZhy{}999. Copyright 2023 Kenneth R. French

  0 : Value Weight Returns \PYGZhy{}\PYGZhy{} Monthly (1159 rows x 19 cols)
  1 : Equal Weight Returns \PYGZhy{}\PYGZhy{} Monthly (1159 rows x 19 cols)
  2 : Value Weight Returns \PYGZhy{}\PYGZhy{} Annual from January to December (96 rows x 19 cols)
  3 : Equal Weight Returns \PYGZhy{}\PYGZhy{} Annual from January to December (96 rows x 19 cols)
  4 : Number of Firms in Portfolios (1159 rows x 19 cols)
  5 : Average Firm Size (1159 rows x 19 cols)
\end{sphinxVerbatim}

\end{sphinxuseclass}\end{sphinxVerbatimOutput}

\end{sphinxuseclass}
\begin{sphinxuseclass}{cell}\begin{sphinxVerbatimInput}

\begin{sphinxuseclass}{cell_input}
\begin{sphinxVerbatim}[commandchars=\\\{\}]
\PYG{n}{ff\PYGZus{}m} \PYG{o}{=} \PYG{n}{pdr}\PYG{o}{.}\PYG{n}{DataReader}\PYG{p}{(}
    \PYG{n}{name}\PYG{o}{=}\PYG{l+s+s1}{\PYGZsq{}}\PYG{l+s+s1}{F\PYGZhy{}F\PYGZus{}Research\PYGZus{}Data\PYGZus{}Factors}\PYG{l+s+s1}{\PYGZsq{}}\PYG{p}{,}
    \PYG{n}{data\PYGZus{}source}\PYG{o}{=}\PYG{l+s+s1}{\PYGZsq{}}\PYG{l+s+s1}{famafrench}\PYG{l+s+s1}{\PYGZsq{}}\PYG{p}{,}
    \PYG{n}{start}\PYG{o}{=}\PYG{l+s+s1}{\PYGZsq{}}\PYG{l+s+s1}{1900}\PYG{l+s+s1}{\PYGZsq{}}\PYG{p}{,}
    \PYG{n}{session}\PYG{o}{=}\PYG{n}{session}
\PYG{p}{)}
\end{sphinxVerbatim}

\end{sphinxuseclass}\end{sphinxVerbatimInput}

\end{sphinxuseclass}
\begin{sphinxuseclass}{cell}\begin{sphinxVerbatimInput}

\begin{sphinxuseclass}{cell_input}
\begin{sphinxVerbatim}[commandchars=\\\{\}]
\PYG{n}{ports} \PYG{o}{=} \PYG{p}{[}\PYG{l+s+s1}{\PYGZsq{}}\PYG{l+s+s1}{Lo 10}\PYG{l+s+s1}{\PYGZsq{}}\PYG{p}{,} \PYG{l+s+s1}{\PYGZsq{}}\PYG{l+s+s1}{Dec 2}\PYG{l+s+s1}{\PYGZsq{}}\PYG{p}{,} \PYG{l+s+s1}{\PYGZsq{}}\PYG{l+s+s1}{Dec 3}\PYG{l+s+s1}{\PYGZsq{}}\PYG{p}{,} \PYG{l+s+s1}{\PYGZsq{}}\PYG{l+s+s1}{Dec 4}\PYG{l+s+s1}{\PYGZsq{}}\PYG{p}{,} \PYG{l+s+s1}{\PYGZsq{}}\PYG{l+s+s1}{Dec 5}\PYG{l+s+s1}{\PYGZsq{}}\PYG{p}{,} \PYG{l+s+s1}{\PYGZsq{}}\PYG{l+s+s1}{Dec 6}\PYG{l+s+s1}{\PYGZsq{}}\PYG{p}{,} \PYG{l+s+s1}{\PYGZsq{}}\PYG{l+s+s1}{Dec 7}\PYG{l+s+s1}{\PYGZsq{}}\PYG{p}{,} \PYG{l+s+s1}{\PYGZsq{}}\PYG{l+s+s1}{Dec 8}\PYG{l+s+s1}{\PYGZsq{}}\PYG{p}{,} \PYG{l+s+s1}{\PYGZsq{}}\PYG{l+s+s1}{Dec 9}\PYG{l+s+s1}{\PYGZsq{}}\PYG{p}{,} \PYG{l+s+s1}{\PYGZsq{}}\PYG{l+s+s1}{Hi 10}\PYG{l+s+s1}{\PYGZsq{}}\PYG{p}{]}
\PYG{n}{joined} \PYG{o}{=} \PYG{n}{size\PYGZus{}0}\PYG{p}{[}\PYG{l+m+mi}{1}\PYG{p}{]}\PYG{p}{[}\PYG{n}{ports}\PYG{p}{]}\PYG{o}{.}\PYG{n}{join}\PYG{p}{(}\PYG{n}{ff\PYGZus{}m}\PYG{p}{[}\PYG{l+m+mi}{0}\PYG{p}{]}\PYG{p}{)}
\PYG{n}{joined}\PYG{o}{.}\PYG{n}{head}\PYG{p}{(}\PYG{p}{)}
\end{sphinxVerbatim}

\end{sphinxuseclass}\end{sphinxVerbatimInput}
\begin{sphinxVerbatimOutput}

\begin{sphinxuseclass}{cell_output}
\begin{sphinxVerbatim}[commandchars=\\\{\}]
          Lo 10   Dec 2   Dec 3   Dec 4   Dec 5   Dec 6   Dec 7   Dec 8  \PYGZbs{}
Date                                                                      
1926\PYGZhy{}07 \PYGZhy{}1.4200  0.2900 \PYGZhy{}0.1500  0.8800  1.4500  1.8500  1.6300  1.3800   
1926\PYGZhy{}08  4.6100  2.5900  4.0300  3.2400  2.6600  4.6700  1.5400  1.6300   
1926\PYGZhy{}09  0.9100 \PYGZhy{}1.8700 \PYGZhy{}2.2700 \PYGZhy{}0.8400  0.1200 \PYGZhy{}0.0700 \PYGZhy{}1.5800  0.6400   
1926\PYGZhy{}10 \PYGZhy{}4.7200 \PYGZhy{}1.7700 \PYGZhy{}3.3600 \PYGZhy{}5.0100 \PYGZhy{}3.0900 \PYGZhy{}2.7100 \PYGZhy{}3.4500 \PYGZhy{}3.2700   
1926\PYGZhy{}11 \PYGZhy{}0.7700 \PYGZhy{}0.3200 \PYGZhy{}0.2900  4.7900  3.1700  3.5800  3.8000  2.9500   

          Dec 9   Hi 10  Mkt\PYGZhy{}RF     SMB     HML     RF  
Date                                                    
1926\PYGZhy{}07  3.3800  3.2900  2.9600 \PYGZhy{}2.5600 \PYGZhy{}2.4300 0.2200  
1926\PYGZhy{}08  0.9800  3.7000  2.6400 \PYGZhy{}1.1700  3.8200 0.2500  
1926\PYGZhy{}09 \PYGZhy{}0.8600  0.6700  0.3600 \PYGZhy{}1.4000  0.1300 0.2300  
1926\PYGZhy{}10 \PYGZhy{}3.4700 \PYGZhy{}2.4300 \PYGZhy{}3.2400 \PYGZhy{}0.0900  0.7000 0.3200  
1926\PYGZhy{}11  3.6100  2.7000  2.5300 \PYGZhy{}0.1000 \PYGZhy{}0.5100 0.3100  
\end{sphinxVerbatim}

\end{sphinxuseclass}\end{sphinxVerbatimOutput}

\end{sphinxuseclass}
\begin{sphinxuseclass}{cell}\begin{sphinxVerbatimInput}

\begin{sphinxuseclass}{cell_input}
\begin{sphinxVerbatim}[commandchars=\\\{\}]
\PYG{k+kn}{import} \PYG{n+nn}{statsmodels}\PYG{n+nn}{.}\PYG{n+nn}{formula}\PYG{n+nn}{.}\PYG{n+nn}{api} \PYG{k}{as} \PYG{n+nn}{smf}
\end{sphinxVerbatim}

\end{sphinxuseclass}\end{sphinxVerbatimInput}

\end{sphinxuseclass}
\begin{sphinxuseclass}{cell}\begin{sphinxVerbatimInput}

\begin{sphinxuseclass}{cell_input}
\begin{sphinxVerbatim}[commandchars=\\\{\}]
\PYG{n}{smf}\PYG{o}{.}\PYG{n}{ols}\PYG{p}{(}\PYG{l+s+s1}{\PYGZsq{}}\PYG{l+s+s1}{I(Q(}\PYG{l+s+s1}{\PYGZdq{}}\PYG{l+s+s1}{Lo 10}\PYG{l+s+s1}{\PYGZdq{}}\PYG{l+s+s1}{) \PYGZhy{} RF) \PYGZti{} Q(}\PYG{l+s+s1}{\PYGZdq{}}\PYG{l+s+s1}{Mkt\PYGZhy{}RF}\PYG{l+s+s1}{\PYGZdq{}}\PYG{l+s+s1}{)}\PYG{l+s+s1}{\PYGZsq{}}\PYG{p}{,} \PYG{n}{data}\PYG{o}{=}\PYG{n}{joined}\PYG{p}{)}\PYG{o}{.}\PYG{n}{fit}\PYG{p}{(}\PYG{p}{)}\PYG{o}{.}\PYG{n}{summary}\PYG{p}{(}\PYG{p}{)}
\end{sphinxVerbatim}

\end{sphinxuseclass}\end{sphinxVerbatimInput}
\begin{sphinxVerbatimOutput}

\begin{sphinxuseclass}{cell_output}
\begin{sphinxVerbatim}[commandchars=\\\{\}]
\PYGZlt{}class \PYGZsq{}statsmodels.iolib.summary.Summary\PYGZsq{}\PYGZgt{}
\PYGZdq{}\PYGZdq{}\PYGZdq{}
                            OLS Regression Results                            
==============================================================================
Dep. Variable:     I(Q(\PYGZdq{}Lo 10\PYGZdq{}) \PYGZhy{} RF)   R\PYGZhy{}squared:                       0.516
Model:                            OLS   Adj. R\PYGZhy{}squared:                  0.515
Method:                 Least Squares   F\PYGZhy{}statistic:                     1232.
Date:                Wed, 22 Mar 2023   Prob (F\PYGZhy{}statistic):          2.06e\PYGZhy{}184
Time:                        18:33:05   Log\PYGZhy{}Likelihood:                \PYGZhy{}3956.4
No. Observations:                1159   AIC:                             7917.
Df Residuals:                    1157   BIC:                             7927.
Df Model:                           1                                         
Covariance Type:            nonrobust                                         
===============================================================================
                  coef    std err          t      P\PYGZgt{}|t|      [0.025      0.975]
\PYGZhy{}\PYGZhy{}\PYGZhy{}\PYGZhy{}\PYGZhy{}\PYGZhy{}\PYGZhy{}\PYGZhy{}\PYGZhy{}\PYGZhy{}\PYGZhy{}\PYGZhy{}\PYGZhy{}\PYGZhy{}\PYGZhy{}\PYGZhy{}\PYGZhy{}\PYGZhy{}\PYGZhy{}\PYGZhy{}\PYGZhy{}\PYGZhy{}\PYGZhy{}\PYGZhy{}\PYGZhy{}\PYGZhy{}\PYGZhy{}\PYGZhy{}\PYGZhy{}\PYGZhy{}\PYGZhy{}\PYGZhy{}\PYGZhy{}\PYGZhy{}\PYGZhy{}\PYGZhy{}\PYGZhy{}\PYGZhy{}\PYGZhy{}\PYGZhy{}\PYGZhy{}\PYGZhy{}\PYGZhy{}\PYGZhy{}\PYGZhy{}\PYGZhy{}\PYGZhy{}\PYGZhy{}\PYGZhy{}\PYGZhy{}\PYGZhy{}\PYGZhy{}\PYGZhy{}\PYGZhy{}\PYGZhy{}\PYGZhy{}\PYGZhy{}\PYGZhy{}\PYGZhy{}\PYGZhy{}\PYGZhy{}\PYGZhy{}\PYGZhy{}\PYGZhy{}\PYGZhy{}\PYGZhy{}\PYGZhy{}\PYGZhy{}\PYGZhy{}\PYGZhy{}\PYGZhy{}\PYGZhy{}\PYGZhy{}\PYGZhy{}\PYGZhy{}\PYGZhy{}\PYGZhy{}\PYGZhy{}\PYGZhy{}
Intercept       0.5309      0.218      2.438      0.015       0.104       0.958
Q(\PYGZdq{}Mkt\PYGZhy{}RF\PYGZdq{})     1.4175      0.040     35.105      0.000       1.338       1.497
==============================================================================
Omnibus:                     1202.137   Durbin\PYGZhy{}Watson:                   1.894
Prob(Omnibus):                  0.000   Jarque\PYGZhy{}Bera (JB):           102452.221
Skew:                           4.798   Prob(JB):                         0.00
Kurtosis:                      48.050   Cond. No.                         5.44
==============================================================================

Notes:
[1] Standard Errors assume that the covariance matrix of the errors is correctly specified.
\PYGZdq{}\PYGZdq{}\PYGZdq{}
\end{sphinxVerbatim}

\end{sphinxuseclass}\end{sphinxVerbatimOutput}

\end{sphinxuseclass}
\begin{sphinxuseclass}{cell}\begin{sphinxVerbatimInput}

\begin{sphinxuseclass}{cell_input}
\begin{sphinxVerbatim}[commandchars=\\\{\}]
\PYG{n}{models} \PYG{o}{=} \PYG{p}{[}\PYG{n}{smf}\PYG{o}{.}\PYG{n}{ols}\PYG{p}{(}\PYG{l+s+sa}{f}\PYG{l+s+s1}{\PYGZsq{}}\PYG{l+s+s1}{I(Q(}\PYG{l+s+s1}{\PYGZdq{}}\PYG{l+s+si}{\PYGZob{}}\PYG{n}{p}\PYG{l+s+si}{\PYGZcb{}}\PYG{l+s+s1}{\PYGZdq{}}\PYG{l+s+s1}{) \PYGZhy{} RF) \PYGZti{} Q(}\PYG{l+s+s1}{\PYGZdq{}}\PYG{l+s+s1}{Mkt\PYGZhy{}RF}\PYG{l+s+s1}{\PYGZdq{}}\PYG{l+s+s1}{)}\PYG{l+s+s1}{\PYGZsq{}}\PYG{p}{,} \PYG{n}{data}\PYG{o}{=}\PYG{n}{joined}\PYG{o}{.}\PYG{n}{loc}\PYG{p}{[}\PYG{p}{:}\PYG{l+s+s1}{\PYGZsq{}}\PYG{l+s+s1}{1979}\PYG{l+s+s1}{\PYGZsq{}}\PYG{p}{]}\PYG{p}{)} \PYG{k}{for} \PYG{n}{p} \PYG{o+ow}{in} \PYG{n}{ports}\PYG{p}{]}

\PYG{n}{fits} \PYG{o}{=} \PYG{p}{[}\PYG{n}{m}\PYG{o}{.}\PYG{n}{fit}\PYG{p}{(}\PYG{p}{)} \PYG{k}{for} \PYG{n}{m} \PYG{o+ow}{in} \PYG{n}{models}\PYG{p}{]}

\PYG{n}{coefs} \PYG{o}{=} \PYG{p}{(}
    \PYG{n}{pd}\PYG{o}{.}\PYG{n}{concat}\PYG{p}{(}
        \PYG{n}{objs}\PYG{o}{=}\PYG{p}{[}\PYG{n}{f}\PYG{o}{.}\PYG{n}{params} \PYG{k}{for} \PYG{n}{f} \PYG{o+ow}{in} \PYG{n}{fits}\PYG{p}{]}\PYG{p}{,} 
        \PYG{n}{axis}\PYG{o}{=}\PYG{l+m+mi}{1}\PYG{p}{,} 
        \PYG{n}{keys}\PYG{o}{=}\PYG{n}{ports}\PYG{p}{,} 
        \PYG{n}{names}\PYG{o}{=}\PYG{l+s+s1}{\PYGZsq{}}\PYG{l+s+s1}{Size Portfolio}\PYG{l+s+s1}{\PYGZsq{}}
    \PYG{p}{)}
    \PYG{o}{.}\PYG{n}{rename\PYGZus{}axis}\PYG{p}{(}\PYG{n}{index}\PYG{o}{=}\PYG{l+s+s1}{\PYGZsq{}}\PYG{l+s+s1}{Coefficient}\PYG{l+s+s1}{\PYGZsq{}}\PYG{p}{,} \PYG{n}{columns}\PYG{o}{=}\PYG{l+s+s1}{\PYGZsq{}}\PYG{l+s+s1}{Size Portfolio}\PYG{l+s+s1}{\PYGZsq{}}\PYG{p}{)}
    \PYG{o}{.}\PYG{n}{transpose}\PYG{p}{(}\PYG{p}{)}
\PYG{p}{)}
\end{sphinxVerbatim}

\end{sphinxuseclass}\end{sphinxVerbatimInput}

\end{sphinxuseclass}
\sphinxAtStartPar
We can get the standard errors, too.
The standard errors are in the \sphinxcode{\sphinxupquote{.params}} attribute of our model fits.

\begin{sphinxuseclass}{cell}\begin{sphinxVerbatimInput}

\begin{sphinxuseclass}{cell_input}
\begin{sphinxVerbatim}[commandchars=\\\{\}]
\PYG{n}{ses} \PYG{o}{=} \PYG{p}{[}\PYG{n}{f}\PYG{o}{.}\PYG{n}{bse}\PYG{p}{[}\PYG{l+m+mi}{0}\PYG{p}{]} \PYG{k}{for} \PYG{n}{f} \PYG{o+ow}{in} \PYG{n}{fits}\PYG{p}{]}
\end{sphinxVerbatim}

\end{sphinxuseclass}\end{sphinxVerbatimInput}

\end{sphinxuseclass}
\begin{sphinxuseclass}{cell}\begin{sphinxVerbatimInput}

\begin{sphinxuseclass}{cell_input}
\begin{sphinxVerbatim}[commandchars=\\\{\}]
\PYG{n}{coefs}\PYG{p}{[}\PYG{l+s+s1}{\PYGZsq{}}\PYG{l+s+s1}{Intercept}\PYG{l+s+s1}{\PYGZsq{}}\PYG{p}{]}\PYG{o}{.}\PYG{n}{plot}\PYG{p}{(}\PYG{n}{kind}\PYG{o}{=}\PYG{l+s+s1}{\PYGZsq{}}\PYG{l+s+s1}{bar}\PYG{l+s+s1}{\PYGZsq{}}\PYG{p}{,} \PYG{n}{yerr}\PYG{o}{=}\PYG{n}{ses}\PYG{p}{)}
\PYG{n}{plt}\PYG{o}{.}\PYG{n}{ylabel}\PYG{p}{(}\PYG{l+s+sa}{r}\PYG{l+s+s1}{\PYGZsq{}}\PYG{l+s+s1}{Monthly \PYGZdl{}}\PYG{l+s+s1}{\PYGZbs{}}\PYG{l+s+s1}{alpha\PYGZdl{} (}\PYG{l+s+s1}{\PYGZpc{}}\PYG{l+s+s1}{) from CAPM}\PYG{l+s+s1}{\PYGZsq{}}\PYG{p}{)}
\PYG{n}{plt}\PYG{o}{.}\PYG{n}{xticks}\PYG{p}{(}\PYG{n}{rotation}\PYG{o}{=}\PYG{l+m+mi}{0}\PYG{p}{)}
\PYG{n}{plt}\PYG{o}{.}\PYG{n}{title}\PYG{p}{(}
    \PYG{l+s+sa}{r}\PYG{l+s+s1}{\PYGZsq{}}\PYG{l+s+s1}{Size Portfolio CAPM \PYGZdl{}}\PYG{l+s+s1}{\PYGZbs{}}\PYG{l+s+s1}{alpha\PYGZdl{}s for Monthly Returns}\PYG{l+s+s1}{\PYGZsq{}} \PYG{o}{+}
    \PYG{l+s+s1}{\PYGZsq{}}\PYG{l+s+se}{\PYGZbs{}n}\PYG{l+s+s1}{from July 1926 through December 1979}\PYG{l+s+s1}{\PYGZsq{}}
\PYG{p}{)}
\PYG{n}{plt}\PYG{o}{.}\PYG{n}{show}\PYG{p}{(}\PYG{p}{)}
\end{sphinxVerbatim}

\end{sphinxuseclass}\end{sphinxVerbatimInput}
\begin{sphinxVerbatimOutput}

\begin{sphinxuseclass}{cell_output}
\noindent\sphinxincludegraphics{{589340b5d177c04f34cbd2ca1bb456a1d2dc60aa82c69e38a6d75ab46b5c3e06}.png}

\end{sphinxuseclass}\end{sphinxVerbatimOutput}

\end{sphinxuseclass}
\sphinxAtStartPar
The size effect (i.e., the CAPM \$\textbackslash{}alpha\$ for small stock portfolios) appears large!
We will dig a little deeper!


\paragraph{Are the returns on these ten portfolios formed on size concentrated in a specific month?}
\label{\detokenize{herron_03_practice_04:are-the-returns-on-these-ten-portfolios-formed-on-size-concentrated-in-a-specific-month}}
\begin{sphinxuseclass}{cell}\begin{sphinxVerbatimInput}

\begin{sphinxuseclass}{cell_input}
\begin{sphinxVerbatim}[commandchars=\\\{\}]
\PYG{p}{(}
    \PYG{n}{size\PYGZus{}0}\PYG{p}{[}\PYG{l+m+mi}{0}\PYG{p}{]}\PYG{p}{[}\PYG{n}{ports}\PYG{p}{]}
    \PYG{o}{.}\PYG{n}{groupby}\PYG{p}{(}\PYG{k}{lambda} \PYG{n}{x}\PYG{p}{:} \PYG{n}{np}\PYG{o}{.}\PYG{n}{where}\PYG{p}{(}\PYG{n}{x}\PYG{o}{.}\PYG{n}{month}\PYG{o}{==}\PYG{l+m+mi}{1}\PYG{p}{,} \PYG{l+s+s1}{\PYGZsq{}}\PYG{l+s+s1}{January}\PYG{l+s+s1}{\PYGZsq{}}\PYG{p}{,} \PYG{l+s+s1}{\PYGZsq{}}\PYG{l+s+s1}{Not January}\PYG{l+s+s1}{\PYGZsq{}}\PYG{p}{)}\PYG{p}{)}
    \PYG{o}{.}\PYG{n}{mean}\PYG{p}{(}\PYG{p}{)}
    \PYG{o}{.}\PYG{n}{rename\PYGZus{}axis}\PYG{p}{(}\PYG{n}{index}\PYG{o}{=}\PYG{l+s+s1}{\PYGZsq{}}\PYG{l+s+s1}{Month}\PYG{l+s+s1}{\PYGZsq{}}\PYG{p}{,} \PYG{n}{columns}\PYG{o}{=}\PYG{l+s+s1}{\PYGZsq{}}\PYG{l+s+s1}{Equal\PYGZhy{}Weighted Size Portfolio}\PYG{l+s+s1}{\PYGZsq{}}\PYG{p}{)}
    \PYG{o}{.}\PYG{n}{plot}\PYG{p}{(}\PYG{n}{kind}\PYG{o}{=}\PYG{l+s+s1}{\PYGZsq{}}\PYG{l+s+s1}{bar}\PYG{l+s+s1}{\PYGZsq{}}\PYG{p}{)}
\PYG{p}{)}

\PYG{n}{plt}\PYG{o}{.}\PYG{n}{xticks}\PYG{p}{(}\PYG{n}{rotation}\PYG{o}{=}\PYG{l+m+mi}{0}\PYG{p}{)}
\PYG{n}{plt}\PYG{o}{.}\PYG{n}{ylabel}\PYG{p}{(}\PYG{l+s+s1}{\PYGZsq{}}\PYG{l+s+s1}{Mean Monthly Return (}\PYG{l+s+s1}{\PYGZpc{}}\PYG{l+s+s1}{)}\PYG{l+s+s1}{\PYGZsq{}}\PYG{p}{)}
\PYG{n}{plt}\PYG{o}{.}\PYG{n}{title}\PYG{p}{(}\PYG{l+s+s1}{\PYGZsq{}}\PYG{l+s+s1}{When Do We Earn Size\PYGZhy{}Effect Returns?}\PYG{l+s+s1}{\PYGZsq{}}\PYG{p}{)}
\PYG{n}{plt}\PYG{o}{.}\PYG{n}{show}\PYG{p}{(}\PYG{p}{)}
\end{sphinxVerbatim}

\end{sphinxuseclass}\end{sphinxVerbatimInput}
\begin{sphinxVerbatimOutput}

\begin{sphinxuseclass}{cell_output}
\noindent\sphinxincludegraphics{{58f7dd3f2070232a24cd5991e72f272cbbc4fcd99b2f1ffac6a982c4837dd184}.png}

\end{sphinxuseclass}\end{sphinxVerbatimOutput}

\end{sphinxuseclass}
\sphinxAtStartPar
We earn size effect returns in January!
The size effect is likely due to tax\sphinxhyphen{}loss harvesting in small stocks.


\paragraph{Compare the size factor to the market factor}
\label{\detokenize{herron_03_practice_04:compare-the-size-factor-to-the-market-factor}}
\sphinxAtStartPar
You may want to consider mean excess returns by decade.

\begin{sphinxuseclass}{cell}\begin{sphinxVerbatimInput}

\begin{sphinxuseclass}{cell_input}
\begin{sphinxVerbatim}[commandchars=\\\{\}]
\PYG{p}{(}
    \PYG{n}{ff\PYGZus{}m}\PYG{p}{[}\PYG{l+m+mi}{0}\PYG{p}{]}\PYG{p}{[}\PYG{p}{[}\PYG{l+s+s1}{\PYGZsq{}}\PYG{l+s+s1}{Mkt\PYGZhy{}RF}\PYG{l+s+s1}{\PYGZsq{}}\PYG{p}{,} \PYG{l+s+s1}{\PYGZsq{}}\PYG{l+s+s1}{SMB}\PYG{l+s+s1}{\PYGZsq{}}\PYG{p}{]}\PYG{p}{]}
    \PYG{o}{.}\PYG{n}{resample}\PYG{p}{(}\PYG{l+s+s1}{\PYGZsq{}}\PYG{l+s+s1}{10Y}\PYG{l+s+s1}{\PYGZsq{}}\PYG{p}{)}
    \PYG{o}{.}\PYG{n}{mean}\PYG{p}{(}\PYG{p}{)}
    \PYG{o}{.}\PYG{n}{mul}\PYG{p}{(}\PYG{l+m+mi}{12}\PYG{p}{)}
    \PYG{o}{.}\PYG{n}{rename\PYGZus{}axis}\PYG{p}{(}\PYG{n}{index}\PYG{o}{=}\PYG{l+s+s1}{\PYGZsq{}}\PYG{l+s+s1}{10\PYGZhy{}Year Period}\PYG{l+s+s1}{\PYGZsq{}}\PYG{p}{,} \PYG{n}{columns}\PYG{o}{=}\PYG{l+s+s1}{\PYGZsq{}}\PYG{l+s+s1}{Factor}\PYG{l+s+s1}{\PYGZsq{}}\PYG{p}{)}
    \PYG{o}{.}\PYG{n}{plot}\PYG{p}{(}\PYG{n}{kind}\PYG{o}{=}\PYG{l+s+s1}{\PYGZsq{}}\PYG{l+s+s1}{bar}\PYG{l+s+s1}{\PYGZsq{}}\PYG{p}{)}
\PYG{p}{)}

\PYG{n}{plt}\PYG{o}{.}\PYG{n}{xticks}\PYG{p}{(}\PYG{n}{rotation}\PYG{o}{=}\PYG{l+m+mi}{0}\PYG{p}{)}
\PYG{n}{plt}\PYG{o}{.}\PYG{n}{ylabel}\PYG{p}{(}\PYG{l+s+s1}{\PYGZsq{}}\PYG{l+s+s1}{Annualize Mean of Monthly Returns (}\PYG{l+s+s1}{\PYGZpc{}}\PYG{l+s+s1}{)}\PYG{l+s+s1}{\PYGZsq{}}\PYG{p}{)}
\PYG{n}{plt}\PYG{o}{.}\PYG{n}{title}\PYG{p}{(}\PYG{l+s+s1}{\PYGZsq{}}\PYG{l+s+s1}{Comparison on Market Risk and Small Stock Premia}\PYG{l+s+s1}{\PYGZsq{}}\PYG{p}{)}
\PYG{n}{plt}\PYG{o}{.}\PYG{n}{show}\PYG{p}{(}\PYG{p}{)}
\end{sphinxVerbatim}

\end{sphinxuseclass}\end{sphinxVerbatimInput}
\begin{sphinxVerbatimOutput}

\begin{sphinxuseclass}{cell_output}
\noindent\sphinxincludegraphics{{5f3edf7217ff43bbb287d6bd9935dc56919398a42681701ac07da0dcc08a4d8a}.png}

\end{sphinxuseclass}\end{sphinxVerbatimOutput}

\end{sphinxuseclass}
\sphinxAtStartPar
The size factor (SMB) may have \$\textbackslash{}alpha\$ early in the sample, but it rarely generates outsize returns.
Plus, the size factor has returned (effectively) zero the last two decades of the sample (2006\sphinxhyphen{}2015 and 2016\sphinxhyphen{}today).


\subsubsection{Repeat the exercises above with the value factor}
\label{\detokenize{herron_03_practice_04:repeat-the-exercises-above-with-the-value-factor}}
\begin{sphinxuseclass}{cell}\begin{sphinxVerbatimInput}

\begin{sphinxuseclass}{cell_input}
\begin{sphinxVerbatim}[commandchars=\\\{\}]
\PYG{n}{hml\PYGZus{}0} \PYG{o}{=} \PYG{n}{pdr}\PYG{o}{.}\PYG{n}{DataReader}\PYG{p}{(}
    \PYG{n}{name}\PYG{o}{=}\PYG{l+s+s1}{\PYGZsq{}}\PYG{l+s+s1}{Portfolios\PYGZus{}Formed\PYGZus{}on\PYGZus{}BE\PYGZhy{}ME}\PYG{l+s+s1}{\PYGZsq{}}\PYG{p}{,}
    \PYG{n}{data\PYGZus{}source}\PYG{o}{=}\PYG{l+s+s1}{\PYGZsq{}}\PYG{l+s+s1}{famafrench}\PYG{l+s+s1}{\PYGZsq{}}\PYG{p}{,}
    \PYG{n}{start}\PYG{o}{=}\PYG{l+s+s1}{\PYGZsq{}}\PYG{l+s+s1}{1900}\PYG{l+s+s1}{\PYGZsq{}}\PYG{p}{,}
    \PYG{n}{session}\PYG{o}{=}\PYG{n}{session}
\PYG{p}{)}

\PYG{n+nb}{print}\PYG{p}{(}\PYG{n}{hml\PYGZus{}0}\PYG{p}{[}\PYG{l+s+s1}{\PYGZsq{}}\PYG{l+s+s1}{DESCR}\PYG{l+s+s1}{\PYGZsq{}}\PYG{p}{]}\PYG{p}{)}
\end{sphinxVerbatim}

\end{sphinxuseclass}\end{sphinxVerbatimInput}
\begin{sphinxVerbatimOutput}

\begin{sphinxuseclass}{cell_output}
\begin{sphinxVerbatim}[commandchars=\\\{\}]
Portfolios Formed on BE\PYGZhy{}ME
\PYGZhy{}\PYGZhy{}\PYGZhy{}\PYGZhy{}\PYGZhy{}\PYGZhy{}\PYGZhy{}\PYGZhy{}\PYGZhy{}\PYGZhy{}\PYGZhy{}\PYGZhy{}\PYGZhy{}\PYGZhy{}\PYGZhy{}\PYGZhy{}\PYGZhy{}\PYGZhy{}\PYGZhy{}\PYGZhy{}\PYGZhy{}\PYGZhy{}\PYGZhy{}\PYGZhy{}\PYGZhy{}\PYGZhy{}

This file was created by CMPT\PYGZus{}BEME\PYGZus{}RETS using the 202301 CRSP database. It contains value\PYGZhy{} and equal\PYGZhy{}weighted returns for portfolios formed on BE/ME. The portfolios are constructed at the end of June. BE/ME is book equity at the last fiscal year end of the prior calendar year divided by ME at the end of December of the prior year. The annual returns are from January to December. Missing data are indicated by \PYGZhy{}99.99 or \PYGZhy{}999. The break points use Compustat firms plus the firms hand\PYGZhy{}collected from the Moodys Industrial, Utilities, Transportation, and Financial Manuals. The portfolios use Compustat firms plus the firms hand\PYGZhy{}collected from the Moodys Industrial, Utilities, Transportation, and Financial Manuals. The break points include utilities and include financials. The portfolios include utilities and include financials. Copyright 2023 Kenneth R. French

  0 : Value Weight Returns \PYGZhy{}\PYGZhy{} Monthly (1159 rows x 19 cols)
  1 : Equal Weight Returns \PYGZhy{}\PYGZhy{} Monthly (1159 rows x 19 cols)
  2 : Value Weight Returns \PYGZhy{}\PYGZhy{} Annual from January to December (96 rows x 19 cols)
  3 : Equal Weight Returns \PYGZhy{}\PYGZhy{} Annual from January to December (96 rows x 19 cols)
  4 : Number of Firms in Portfolios (1159 rows x 19 cols)
  5 : Average Firm Size (1159 rows x 19 cols)
  6 : Sum of BE / Sum of ME (97 rows x 19 cols)
  7 : Value Weight Average of BE / ME (97 rows x 19 cols)
\end{sphinxVerbatim}

\end{sphinxuseclass}\end{sphinxVerbatimOutput}

\end{sphinxuseclass}
\begin{sphinxuseclass}{cell}\begin{sphinxVerbatimInput}

\begin{sphinxuseclass}{cell_input}
\begin{sphinxVerbatim}[commandchars=\\\{\}]
\PYG{n}{ports} \PYG{o}{=} \PYG{p}{[}\PYG{l+s+s1}{\PYGZsq{}}\PYG{l+s+s1}{Lo 10}\PYG{l+s+s1}{\PYGZsq{}}\PYG{p}{,} \PYG{l+s+s1}{\PYGZsq{}}\PYG{l+s+s1}{Dec 2}\PYG{l+s+s1}{\PYGZsq{}}\PYG{p}{,} \PYG{l+s+s1}{\PYGZsq{}}\PYG{l+s+s1}{Dec 3}\PYG{l+s+s1}{\PYGZsq{}}\PYG{p}{,} \PYG{l+s+s1}{\PYGZsq{}}\PYG{l+s+s1}{Dec 4}\PYG{l+s+s1}{\PYGZsq{}}\PYG{p}{,} \PYG{l+s+s1}{\PYGZsq{}}\PYG{l+s+s1}{Dec 5}\PYG{l+s+s1}{\PYGZsq{}}\PYG{p}{,} \PYG{l+s+s1}{\PYGZsq{}}\PYG{l+s+s1}{Dec 6}\PYG{l+s+s1}{\PYGZsq{}}\PYG{p}{,} \PYG{l+s+s1}{\PYGZsq{}}\PYG{l+s+s1}{Dec 7}\PYG{l+s+s1}{\PYGZsq{}}\PYG{p}{,} \PYG{l+s+s1}{\PYGZsq{}}\PYG{l+s+s1}{Dec 8}\PYG{l+s+s1}{\PYGZsq{}}\PYG{p}{,} \PYG{l+s+s1}{\PYGZsq{}}\PYG{l+s+s1}{Dec 9}\PYG{l+s+s1}{\PYGZsq{}}\PYG{p}{,} \PYG{l+s+s1}{\PYGZsq{}}\PYG{l+s+s1}{Hi 10}\PYG{l+s+s1}{\PYGZsq{}}\PYG{p}{]}
\PYG{n}{joined} \PYG{o}{=} \PYG{n}{hml\PYGZus{}0}\PYG{p}{[}\PYG{l+m+mi}{1}\PYG{p}{]}\PYG{p}{[}\PYG{n}{ports}\PYG{p}{]}\PYG{o}{.}\PYG{n}{join}\PYG{p}{(}\PYG{n}{ff\PYGZus{}m}\PYG{p}{[}\PYG{l+m+mi}{0}\PYG{p}{]}\PYG{p}{)}
\PYG{n}{joined}\PYG{o}{.}\PYG{n}{head}\PYG{p}{(}\PYG{p}{)}
\end{sphinxVerbatim}

\end{sphinxuseclass}\end{sphinxVerbatimInput}
\begin{sphinxVerbatimOutput}

\begin{sphinxuseclass}{cell_output}
\begin{sphinxVerbatim}[commandchars=\\\{\}]
          Lo 10   Dec 2   Dec 3   Dec 4   Dec 5   Dec 6   Dec 7   Dec 8  \PYGZbs{}
Date                                                                      
1926\PYGZhy{}07  4.5200  1.5300  1.9900  2.0700  0.0700  1.9600  0.2300  2.8800   
1926\PYGZhy{}08  0.3400  0.5100  3.0000  0.8100  2.7900  1.6600  0.9600  4.4900   
1926\PYGZhy{}09  0.3600 \PYGZhy{}4.2500 \PYGZhy{}1.1600  0.8800 \PYGZhy{}0.7400 \PYGZhy{}1.1000 \PYGZhy{}0.5000 \PYGZhy{}1.4500   
1926\PYGZhy{}10 \PYGZhy{}4.1200 \PYGZhy{}3.8300 \PYGZhy{}2.6700 \PYGZhy{}2.7800 \PYGZhy{}2.3100 \PYGZhy{}3.9800 \PYGZhy{}4.1800 \PYGZhy{}2.0100   
1926\PYGZhy{}11  4.4400  0.4700  1.3500  4.4500  3.5600  1.4700  3.5100  3.5100   

          Dec 9   Hi 10  Mkt\PYGZhy{}RF     SMB     HML     RF  
Date                                                    
1926\PYGZhy{}07 \PYGZhy{}1.4400 \PYGZhy{}0.1700  2.9600 \PYGZhy{}2.5600 \PYGZhy{}2.4300 0.2200  
1926\PYGZhy{}08  5.8100  6.4800  2.6400 \PYGZhy{}1.1700  3.8200 0.2500  
1926\PYGZhy{}09 \PYGZhy{}2.1700  3.9300  0.3600 \PYGZhy{}1.4000  0.1300 0.2300  
1926\PYGZhy{}10 \PYGZhy{}4.5800 \PYGZhy{}1.8800 \PYGZhy{}3.2400 \PYGZhy{}0.0900  0.7000 0.3200  
1926\PYGZhy{}11  2.2300  2.6600  2.5300 \PYGZhy{}0.1000 \PYGZhy{}0.5100 0.3100  
\end{sphinxVerbatim}

\end{sphinxuseclass}\end{sphinxVerbatimOutput}

\end{sphinxuseclass}
\begin{sphinxuseclass}{cell}\begin{sphinxVerbatimInput}

\begin{sphinxuseclass}{cell_input}
\begin{sphinxVerbatim}[commandchars=\\\{\}]
\PYG{n}{models} \PYG{o}{=} \PYG{p}{[}\PYG{n}{smf}\PYG{o}{.}\PYG{n}{ols}\PYG{p}{(}\PYG{l+s+sa}{f}\PYG{l+s+s1}{\PYGZsq{}}\PYG{l+s+s1}{I(Q(}\PYG{l+s+s1}{\PYGZdq{}}\PYG{l+s+si}{\PYGZob{}}\PYG{n}{p}\PYG{l+s+si}{\PYGZcb{}}\PYG{l+s+s1}{\PYGZdq{}}\PYG{l+s+s1}{) \PYGZhy{} RF) \PYGZti{} Q(}\PYG{l+s+s1}{\PYGZdq{}}\PYG{l+s+s1}{Mkt\PYGZhy{}RF}\PYG{l+s+s1}{\PYGZdq{}}\PYG{l+s+s1}{)}\PYG{l+s+s1}{\PYGZsq{}}\PYG{p}{,} \PYG{n}{data}\PYG{o}{=}\PYG{n}{joined}\PYG{p}{)} \PYG{k}{for} \PYG{n}{p} \PYG{o+ow}{in} \PYG{n}{ports}\PYG{p}{]}

\PYG{n}{fits} \PYG{o}{=} \PYG{p}{[}\PYG{n}{m}\PYG{o}{.}\PYG{n}{fit}\PYG{p}{(}\PYG{p}{)} \PYG{k}{for} \PYG{n}{m} \PYG{o+ow}{in} \PYG{n}{models}\PYG{p}{]}

\PYG{n}{coefs} \PYG{o}{=} \PYG{p}{(}
    \PYG{n}{pd}\PYG{o}{.}\PYG{n}{concat}\PYG{p}{(}
        \PYG{n}{objs}\PYG{o}{=}\PYG{p}{[}\PYG{n}{f}\PYG{o}{.}\PYG{n}{params} \PYG{k}{for} \PYG{n}{f} \PYG{o+ow}{in} \PYG{n}{fits}\PYG{p}{]}\PYG{p}{,}
        \PYG{n}{axis}\PYG{o}{=}\PYG{l+m+mi}{1}\PYG{p}{,}
        \PYG{n}{keys}\PYG{o}{=}\PYG{n}{ports}
    \PYG{p}{)}
    \PYG{o}{.}\PYG{n}{rename\PYGZus{}axis}\PYG{p}{(}\PYG{n}{index}\PYG{o}{=}\PYG{l+s+s1}{\PYGZsq{}}\PYG{l+s+s1}{Coefficient}\PYG{l+s+s1}{\PYGZsq{}}\PYG{p}{,} \PYG{n}{columns}\PYG{o}{=}\PYG{l+s+s1}{\PYGZsq{}}\PYG{l+s+s1}{Equal\PYGZhy{}Weighted Value Portfolio}\PYG{l+s+s1}{\PYGZsq{}}\PYG{p}{)}
    \PYG{o}{.}\PYG{n}{transpose}\PYG{p}{(}\PYG{p}{)}
\PYG{p}{)}
\end{sphinxVerbatim}

\end{sphinxuseclass}\end{sphinxVerbatimInput}

\end{sphinxuseclass}
\sphinxAtStartPar
We can get the standard errors, too.
The standard errors are in the \sphinxcode{\sphinxupquote{.params}} attribute of our model fits.

\begin{sphinxuseclass}{cell}\begin{sphinxVerbatimInput}

\begin{sphinxuseclass}{cell_input}
\begin{sphinxVerbatim}[commandchars=\\\{\}]
\PYG{n}{ses} \PYG{o}{=} \PYG{p}{[}\PYG{n}{f}\PYG{o}{.}\PYG{n}{bse}\PYG{p}{[}\PYG{l+m+mi}{0}\PYG{p}{]} \PYG{k}{for} \PYG{n}{f} \PYG{o+ow}{in} \PYG{n}{fits}\PYG{p}{]}
\end{sphinxVerbatim}

\end{sphinxuseclass}\end{sphinxVerbatimInput}

\end{sphinxuseclass}
\begin{sphinxuseclass}{cell}\begin{sphinxVerbatimInput}

\begin{sphinxuseclass}{cell_input}
\begin{sphinxVerbatim}[commandchars=\\\{\}]
\PYG{n}{coefs}\PYG{p}{[}\PYG{l+s+s1}{\PYGZsq{}}\PYG{l+s+s1}{Intercept}\PYG{l+s+s1}{\PYGZsq{}}\PYG{p}{]}\PYG{o}{.}\PYG{n}{plot}\PYG{p}{(}\PYG{n}{kind}\PYG{o}{=}\PYG{l+s+s1}{\PYGZsq{}}\PYG{l+s+s1}{bar}\PYG{l+s+s1}{\PYGZsq{}}\PYG{p}{,} \PYG{n}{yerr}\PYG{o}{=}\PYG{n}{ses}\PYG{p}{)}
\PYG{n}{plt}\PYG{o}{.}\PYG{n}{ylabel}\PYG{p}{(}\PYG{l+s+sa}{r}\PYG{l+s+s1}{\PYGZsq{}}\PYG{l+s+s1}{Monthly \PYGZdl{}}\PYG{l+s+s1}{\PYGZbs{}}\PYG{l+s+s1}{alpha\PYGZdl{} (}\PYG{l+s+s1}{\PYGZpc{}}\PYG{l+s+s1}{) from CAPM}\PYG{l+s+s1}{\PYGZsq{}}\PYG{p}{)}
\PYG{c+c1}{\PYGZsh{} plt.xticks(rotation=0)}
\PYG{n}{plt}\PYG{o}{.}\PYG{n}{title}\PYG{p}{(}
    \PYG{l+s+sa}{r}\PYG{l+s+s1}{\PYGZsq{}}\PYG{l+s+s1}{Value Portfolio CAPM \PYGZdl{}}\PYG{l+s+s1}{\PYGZbs{}}\PYG{l+s+s1}{alpha\PYGZdl{}s for Monthly Returns}\PYG{l+s+s1}{\PYGZsq{}} \PYG{o}{+}
    \PYG{l+s+s1}{\PYGZsq{}}\PYG{l+s+se}{\PYGZbs{}n}\PYG{l+s+s1}{from July 1926 through january 1979}\PYG{l+s+s1}{\PYGZsq{}}
\PYG{p}{)}
\PYG{n}{plt}\PYG{o}{.}\PYG{n}{show}\PYG{p}{(}\PYG{p}{)}
\end{sphinxVerbatim}

\end{sphinxuseclass}\end{sphinxVerbatimInput}
\begin{sphinxVerbatimOutput}

\begin{sphinxuseclass}{cell_output}
\noindent\sphinxincludegraphics{{8fad74c42a93b4cea2c88078ade4b80a065ea5ec0510ecdb7206c2d785a22602}.png}

\end{sphinxuseclass}\end{sphinxVerbatimOutput}

\end{sphinxuseclass}

\subsubsection{Repeat the exercises above with the momentum factor}
\label{\detokenize{herron_03_practice_04:repeat-the-exercises-above-with-the-momentum-factor}}
\sphinxAtStartPar
You may find it helpful to consider the worst months and years for the momentum factor.

\begin{sphinxuseclass}{cell}\begin{sphinxVerbatimInput}

\begin{sphinxuseclass}{cell_input}
\begin{sphinxVerbatim}[commandchars=\\\{\}]
\PYG{n}{mom\PYGZus{}0} \PYG{o}{=} \PYG{n}{pdr}\PYG{o}{.}\PYG{n}{DataReader}\PYG{p}{(}
    \PYG{n}{name}\PYG{o}{=}\PYG{l+s+s1}{\PYGZsq{}}\PYG{l+s+s1}{10\PYGZus{}Portfolios\PYGZus{}Prior\PYGZus{}12\PYGZus{}2}\PYG{l+s+s1}{\PYGZsq{}}\PYG{p}{,}
    \PYG{n}{data\PYGZus{}source}\PYG{o}{=}\PYG{l+s+s1}{\PYGZsq{}}\PYG{l+s+s1}{famafrench}\PYG{l+s+s1}{\PYGZsq{}}\PYG{p}{,}
    \PYG{n}{start}\PYG{o}{=}\PYG{l+s+s1}{\PYGZsq{}}\PYG{l+s+s1}{1900}\PYG{l+s+s1}{\PYGZsq{}}\PYG{p}{,}
    \PYG{n}{session}\PYG{o}{=}\PYG{n}{session}
\PYG{p}{)}

\PYG{n+nb}{print}\PYG{p}{(}\PYG{n}{mom\PYGZus{}0}\PYG{p}{[}\PYG{l+s+s1}{\PYGZsq{}}\PYG{l+s+s1}{DESCR}\PYG{l+s+s1}{\PYGZsq{}}\PYG{p}{]}\PYG{p}{)}
\end{sphinxVerbatim}

\end{sphinxuseclass}\end{sphinxVerbatimInput}
\begin{sphinxVerbatimOutput}

\begin{sphinxuseclass}{cell_output}
\begin{sphinxVerbatim}[commandchars=\\\{\}]
10 Portfolios Prior 12 2
\PYGZhy{}\PYGZhy{}\PYGZhy{}\PYGZhy{}\PYGZhy{}\PYGZhy{}\PYGZhy{}\PYGZhy{}\PYGZhy{}\PYGZhy{}\PYGZhy{}\PYGZhy{}\PYGZhy{}\PYGZhy{}\PYGZhy{}\PYGZhy{}\PYGZhy{}\PYGZhy{}\PYGZhy{}\PYGZhy{}\PYGZhy{}\PYGZhy{}\PYGZhy{}\PYGZhy{}

This file was created by CMPT\PYGZus{}PRIOR\PYGZus{}RETS using the 202301 CRSP database. It contains value\PYGZhy{} and equal\PYGZhy{}weighted returns for 10 prior\PYGZhy{}return portfolios. The portfolios are constructed monthly. PRIOR\PYGZus{}RET is from \PYGZhy{}12 to \PYGZhy{} 2. The annual returns are from January to December. Missing data are indicated by \PYGZhy{}99.99 or \PYGZhy{}999.

  0 : Average Value Weighted Returns \PYGZhy{}\PYGZhy{} Monthly (1153 rows x 10 cols)
  1 : Average Equal Weighted Returns \PYGZhy{}\PYGZhy{} Monthly (1153 rows x 10 cols)
  2 : Average Value Weighted Returns \PYGZhy{}\PYGZhy{} Annual (96 rows x 10 cols)
  3 : Average Equal Weighted Returns \PYGZhy{}\PYGZhy{} Annual (96 rows x 10 cols)
  4 : Number of Firms in Portfolios (1153 rows x 10 cols)
  5 : Average Firm Size (1153 rows x 10 cols)
  6 : Value\PYGZhy{}Weighted Average of Prior Returns (96 rows x 10 cols)
\end{sphinxVerbatim}

\end{sphinxuseclass}\end{sphinxVerbatimOutput}

\end{sphinxuseclass}
\begin{sphinxuseclass}{cell}\begin{sphinxVerbatimInput}

\begin{sphinxuseclass}{cell_input}
\begin{sphinxVerbatim}[commandchars=\\\{\}]
\PYG{n}{ports} \PYG{o}{=} \PYG{p}{[}\PYG{l+s+s1}{\PYGZsq{}}\PYG{l+s+s1}{Lo PRIOR}\PYG{l+s+s1}{\PYGZsq{}}\PYG{p}{,} \PYG{l+s+s1}{\PYGZsq{}}\PYG{l+s+s1}{PRIOR 2}\PYG{l+s+s1}{\PYGZsq{}}\PYG{p}{,} \PYG{l+s+s1}{\PYGZsq{}}\PYG{l+s+s1}{PRIOR 3}\PYG{l+s+s1}{\PYGZsq{}}\PYG{p}{,} \PYG{l+s+s1}{\PYGZsq{}}\PYG{l+s+s1}{PRIOR 4}\PYG{l+s+s1}{\PYGZsq{}}\PYG{p}{,} \PYG{l+s+s1}{\PYGZsq{}}\PYG{l+s+s1}{PRIOR 5}\PYG{l+s+s1}{\PYGZsq{}}\PYG{p}{,} \PYG{l+s+s1}{\PYGZsq{}}\PYG{l+s+s1}{PRIOR 6}\PYG{l+s+s1}{\PYGZsq{}}\PYG{p}{,} \PYG{l+s+s1}{\PYGZsq{}}\PYG{l+s+s1}{PRIOR 7}\PYG{l+s+s1}{\PYGZsq{}}\PYG{p}{,} \PYG{l+s+s1}{\PYGZsq{}}\PYG{l+s+s1}{PRIOR 8}\PYG{l+s+s1}{\PYGZsq{}}\PYG{p}{,} \PYG{l+s+s1}{\PYGZsq{}}\PYG{l+s+s1}{PRIOR 9}\PYG{l+s+s1}{\PYGZsq{}}\PYG{p}{,} \PYG{l+s+s1}{\PYGZsq{}}\PYG{l+s+s1}{Hi PRIOR}\PYG{l+s+s1}{\PYGZsq{}}\PYG{p}{]}
\PYG{n}{joined} \PYG{o}{=} \PYG{n}{mom\PYGZus{}0}\PYG{p}{[}\PYG{l+m+mi}{1}\PYG{p}{]}\PYG{p}{[}\PYG{n}{ports}\PYG{p}{]}\PYG{o}{.}\PYG{n}{join}\PYG{p}{(}\PYG{n}{ff\PYGZus{}m}\PYG{p}{[}\PYG{l+m+mi}{0}\PYG{p}{]}\PYG{p}{)}
\PYG{n}{joined}\PYG{o}{.}\PYG{n}{head}\PYG{p}{(}\PYG{p}{)}
\end{sphinxVerbatim}

\end{sphinxuseclass}\end{sphinxVerbatimInput}
\begin{sphinxVerbatimOutput}

\begin{sphinxuseclass}{cell_output}
\begin{sphinxVerbatim}[commandchars=\\\{\}]
         Lo PRIOR  PRIOR 2  PRIOR 3  PRIOR 4  PRIOR 5  PRIOR 6  PRIOR 7  \PYGZbs{}
Date                                                                      
1927\PYGZhy{}01   \PYGZhy{}0.4700   1.0400   4.1700   2.4000  \PYGZhy{}0.3100   3.0800   1.4800   
1927\PYGZhy{}02    8.4900   3.8800   8.3300   7.5900   4.7900   5.4700   5.4000   
1927\PYGZhy{}03   \PYGZhy{}6.1500  \PYGZhy{}5.1100  \PYGZhy{}2.3800  \PYGZhy{}1.6000  \PYGZhy{}1.6300  \PYGZhy{}1.7000   0.4900   
1927\PYGZhy{}04    2.6100  \PYGZhy{}0.9300   0.1000  \PYGZhy{}1.2000   0.0900   0.4000   1.2700   
1927\PYGZhy{}05    1.1200   3.7300   6.0100   5.1700   7.5500   9.2700   6.6100   

         PRIOR 8  PRIOR 9  Hi PRIOR  Mkt\PYGZhy{}RF     SMB     HML     RF  
Date                                                                
1927\PYGZhy{}01   0.6400  \PYGZhy{}0.5900    1.1800 \PYGZhy{}0.0600 \PYGZhy{}0.3700  4.5400 0.2500  
1927\PYGZhy{}02   4.3000   5.7600    5.5800  4.1800  0.0400  2.9400 0.2600  
1927\PYGZhy{}03  \PYGZhy{}1.5300   0.3700   \PYGZhy{}0.7800  0.1300 \PYGZhy{}1.6500 \PYGZhy{}2.6100 0.3000  
1927\PYGZhy{}04  \PYGZhy{}0.5700   2.2200    4.4200  0.4600  0.3000  0.8100 0.2500  
1927\PYGZhy{}05   8.2100   6.9100   10.0000  5.4400  1.5300  4.7300 0.3000  
\end{sphinxVerbatim}

\end{sphinxuseclass}\end{sphinxVerbatimOutput}

\end{sphinxuseclass}
\begin{sphinxuseclass}{cell}\begin{sphinxVerbatimInput}

\begin{sphinxuseclass}{cell_input}
\begin{sphinxVerbatim}[commandchars=\\\{\}]
\PYG{n}{models} \PYG{o}{=} \PYG{p}{[}\PYG{n}{smf}\PYG{o}{.}\PYG{n}{ols}\PYG{p}{(}\PYG{l+s+sa}{f}\PYG{l+s+s1}{\PYGZsq{}}\PYG{l+s+s1}{I(Q(}\PYG{l+s+s1}{\PYGZdq{}}\PYG{l+s+si}{\PYGZob{}}\PYG{n}{p}\PYG{l+s+si}{\PYGZcb{}}\PYG{l+s+s1}{\PYGZdq{}}\PYG{l+s+s1}{) \PYGZhy{} RF) \PYGZti{} Q(}\PYG{l+s+s1}{\PYGZdq{}}\PYG{l+s+s1}{Mkt\PYGZhy{}RF}\PYG{l+s+s1}{\PYGZdq{}}\PYG{l+s+s1}{)}\PYG{l+s+s1}{\PYGZsq{}}\PYG{p}{,} \PYG{n}{data}\PYG{o}{=}\PYG{n}{joined}\PYG{p}{)} \PYG{k}{for} \PYG{n}{p} \PYG{o+ow}{in} \PYG{n}{ports}\PYG{p}{]}
\end{sphinxVerbatim}

\end{sphinxuseclass}\end{sphinxVerbatimInput}

\end{sphinxuseclass}
\begin{sphinxuseclass}{cell}\begin{sphinxVerbatimInput}

\begin{sphinxuseclass}{cell_input}
\begin{sphinxVerbatim}[commandchars=\\\{\}]
\PYG{n}{fits} \PYG{o}{=} \PYG{p}{[}\PYG{n}{m}\PYG{o}{.}\PYG{n}{fit}\PYG{p}{(}\PYG{p}{)} \PYG{k}{for} \PYG{n}{m} \PYG{o+ow}{in} \PYG{n}{models}\PYG{p}{]}
\end{sphinxVerbatim}

\end{sphinxuseclass}\end{sphinxVerbatimInput}

\end{sphinxuseclass}
\begin{sphinxuseclass}{cell}\begin{sphinxVerbatimInput}

\begin{sphinxuseclass}{cell_input}
\begin{sphinxVerbatim}[commandchars=\\\{\}]
\PYG{n}{coefs} \PYG{o}{=} \PYG{p}{(}
    \PYG{n}{pd}\PYG{o}{.}\PYG{n}{concat}\PYG{p}{(}
        \PYG{n}{objs}\PYG{o}{=}\PYG{p}{[}\PYG{n}{f}\PYG{o}{.}\PYG{n}{params} \PYG{k}{for} \PYG{n}{f} \PYG{o+ow}{in} \PYG{n}{fits}\PYG{p}{]}\PYG{p}{,}
        \PYG{n}{axis}\PYG{o}{=}\PYG{l+m+mi}{1}\PYG{p}{,}
        \PYG{n}{keys}\PYG{o}{=}\PYG{n}{ports}
    \PYG{p}{)}
    \PYG{o}{.}\PYG{n}{rename\PYGZus{}axis}\PYG{p}{(}\PYG{n}{index}\PYG{o}{=}\PYG{l+s+s1}{\PYGZsq{}}\PYG{l+s+s1}{Coefficient}\PYG{l+s+s1}{\PYGZsq{}}\PYG{p}{,} \PYG{n}{columns}\PYG{o}{=}\PYG{l+s+s1}{\PYGZsq{}}\PYG{l+s+s1}{Equal\PYGZhy{}Weighted Momentum Portfolio}\PYG{l+s+s1}{\PYGZsq{}}\PYG{p}{)}
    \PYG{o}{.}\PYG{n}{transpose}\PYG{p}{(}\PYG{p}{)}
\PYG{p}{)}
\end{sphinxVerbatim}

\end{sphinxuseclass}\end{sphinxVerbatimInput}

\end{sphinxuseclass}
\sphinxAtStartPar
We can get the standard errors, too.
The standard errors are in the \sphinxcode{\sphinxupquote{.params}} attribute of our model fits.

\begin{sphinxuseclass}{cell}\begin{sphinxVerbatimInput}

\begin{sphinxuseclass}{cell_input}
\begin{sphinxVerbatim}[commandchars=\\\{\}]
\PYG{n}{ses} \PYG{o}{=} \PYG{p}{[}\PYG{n}{f}\PYG{o}{.}\PYG{n}{bse}\PYG{p}{[}\PYG{l+m+mi}{0}\PYG{p}{]} \PYG{k}{for} \PYG{n}{f} \PYG{o+ow}{in} \PYG{n}{fits}\PYG{p}{]}
\end{sphinxVerbatim}

\end{sphinxuseclass}\end{sphinxVerbatimInput}

\end{sphinxuseclass}
\begin{sphinxuseclass}{cell}\begin{sphinxVerbatimInput}

\begin{sphinxuseclass}{cell_input}
\begin{sphinxVerbatim}[commandchars=\\\{\}]
\PYG{n}{coefs}\PYG{p}{[}\PYG{l+s+s1}{\PYGZsq{}}\PYG{l+s+s1}{Intercept}\PYG{l+s+s1}{\PYGZsq{}}\PYG{p}{]}\PYG{o}{.}\PYG{n}{plot}\PYG{p}{(}\PYG{n}{kind}\PYG{o}{=}\PYG{l+s+s1}{\PYGZsq{}}\PYG{l+s+s1}{bar}\PYG{l+s+s1}{\PYGZsq{}}\PYG{p}{,} \PYG{n}{yerr}\PYG{o}{=}\PYG{n}{ses}\PYG{p}{)}
\PYG{n}{plt}\PYG{o}{.}\PYG{n}{ylabel}\PYG{p}{(}\PYG{l+s+sa}{r}\PYG{l+s+s1}{\PYGZsq{}}\PYG{l+s+s1}{Monthly \PYGZdl{}}\PYG{l+s+s1}{\PYGZbs{}}\PYG{l+s+s1}{alpha\PYGZdl{} (}\PYG{l+s+s1}{\PYGZpc{}}\PYG{l+s+s1}{) from CAPM}\PYG{l+s+s1}{\PYGZsq{}}\PYG{p}{)}
\PYG{c+c1}{\PYGZsh{} plt.xticks(rotation=0)}
\PYG{n}{plt}\PYG{o}{.}\PYG{n}{title}\PYG{p}{(}
    \PYG{l+s+sa}{r}\PYG{l+s+s1}{\PYGZsq{}}\PYG{l+s+s1}{Momentum Portfolio CAPM \PYGZdl{}}\PYG{l+s+s1}{\PYGZbs{}}\PYG{l+s+s1}{alpha\PYGZdl{}s for Monthly Returns}\PYG{l+s+s1}{\PYGZsq{}} \PYG{o}{+}
    \PYG{l+s+s1}{\PYGZsq{}}\PYG{l+s+se}{\PYGZbs{}n}\PYG{l+s+s1}{from July 1926 through january 1979}\PYG{l+s+s1}{\PYGZsq{}}
\PYG{p}{)}
\PYG{n}{plt}\PYG{o}{.}\PYG{n}{show}\PYG{p}{(}\PYG{p}{)}
\end{sphinxVerbatim}

\end{sphinxuseclass}\end{sphinxVerbatimInput}
\begin{sphinxVerbatimOutput}

\begin{sphinxuseclass}{cell_output}
\noindent\sphinxincludegraphics{{561fb57ef0e2da2f6371ef7bcf748c3c03051c3f465b8cead6b1dd2eb3e19424}.png}

\end{sphinxuseclass}\end{sphinxVerbatimOutput}

\end{sphinxuseclass}

\subsubsection{Plot the coefficient estimates from a rolling Fama\sphinxhyphen{}French three\sphinxhyphen{}factor model for Berkshire Hathaway}
\label{\detokenize{herron_03_practice_04:plot-the-coefficient-estimates-from-a-rolling-fama-french-three-factor-model-for-berkshire-hathaway}}
\sphinxAtStartPar
Use a three\sphinxhyphen{}year window with daily returns.
How has Buffett’s \$\textbackslash{}alpha\$ and \$\textbackslash{}beta\$s changed over the past four decades?

\begin{sphinxuseclass}{cell}\begin{sphinxVerbatimInput}

\begin{sphinxuseclass}{cell_input}
\begin{sphinxVerbatim}[commandchars=\\\{\}]
\PYG{n}{brk} \PYG{o}{=} \PYG{p}{(}
    \PYG{n}{yf}\PYG{o}{.}\PYG{n}{download}\PYG{p}{(}\PYG{n}{tickers}\PYG{o}{=}\PYG{l+s+s1}{\PYGZsq{}}\PYG{l+s+s1}{BRK\PYGZhy{}A}\PYG{l+s+s1}{\PYGZsq{}}\PYG{p}{,} \PYG{n}{progress}\PYG{o}{=}\PYG{k+kc}{False}\PYG{p}{)}
    \PYG{o}{.}\PYG{n}{assign}\PYG{p}{(}\PYG{n}{Date} \PYG{o}{=} \PYG{k}{lambda} \PYG{n}{x}\PYG{p}{:} \PYG{n}{x}\PYG{o}{.}\PYG{n}{index}\PYG{o}{.}\PYG{n}{tz\PYGZus{}localize}\PYG{p}{(}\PYG{k+kc}{None}\PYG{p}{)}\PYG{p}{)}
    \PYG{o}{.}\PYG{n}{set\PYGZus{}index}\PYG{p}{(}\PYG{l+s+s1}{\PYGZsq{}}\PYG{l+s+s1}{Date}\PYG{l+s+s1}{\PYGZsq{}}\PYG{p}{)}
    \PYG{o}{.}\PYG{n}{rename\PYGZus{}axis}\PYG{p}{(}\PYG{n}{columns}\PYG{o}{=}\PYG{p}{[}\PYG{l+s+s1}{\PYGZsq{}}\PYG{l+s+s1}{Variable}\PYG{l+s+s1}{\PYGZsq{}}\PYG{p}{]}\PYG{p}{)}
    \PYG{o}{.}\PYG{n}{assign}\PYG{p}{(}\PYG{n}{R}\PYG{o}{=}\PYG{k}{lambda} \PYG{n}{x}\PYG{p}{:} \PYG{n}{x}\PYG{p}{[}\PYG{l+s+s1}{\PYGZsq{}}\PYG{l+s+s1}{Adj Close}\PYG{l+s+s1}{\PYGZsq{}}\PYG{p}{]}\PYG{o}{.}\PYG{n}{pct\PYGZus{}change}\PYG{p}{(}\PYG{p}{)}\PYG{o}{.}\PYG{n}{mul}\PYG{p}{(}\PYG{l+m+mi}{100}\PYG{p}{)}\PYG{p}{)}
    \PYG{o}{.}\PYG{n}{dropna}\PYG{p}{(}\PYG{p}{)}
    \PYG{o}{.}\PYG{n}{join}\PYG{p}{(}
        \PYG{n}{pdr}\PYG{o}{.}\PYG{n}{DataReader}\PYG{p}{(}
            \PYG{n}{name}\PYG{o}{=}\PYG{l+s+s1}{\PYGZsq{}}\PYG{l+s+s1}{F\PYGZhy{}F\PYGZus{}Research\PYGZus{}Data\PYGZus{}Factors\PYGZus{}daily}\PYG{l+s+s1}{\PYGZsq{}}\PYG{p}{,}
            \PYG{n}{data\PYGZus{}source}\PYG{o}{=}\PYG{l+s+s1}{\PYGZsq{}}\PYG{l+s+s1}{famafrench}\PYG{l+s+s1}{\PYGZsq{}}\PYG{p}{,}
            \PYG{n}{start}\PYG{o}{=}\PYG{l+s+s1}{\PYGZsq{}}\PYG{l+s+s1}{1900}\PYG{l+s+s1}{\PYGZsq{}}\PYG{p}{,}
            \PYG{n}{session}\PYG{o}{=}\PYG{n}{session}
        \PYG{p}{)}\PYG{p}{[}\PYG{l+m+mi}{0}\PYG{p}{]}\PYG{p}{,}
        \PYG{n}{how}\PYG{o}{=}\PYG{l+s+s1}{\PYGZsq{}}\PYG{l+s+s1}{inner}\PYG{l+s+s1}{\PYGZsq{}}
    \PYG{p}{)}
\PYG{p}{)}
\end{sphinxVerbatim}

\end{sphinxuseclass}\end{sphinxVerbatimInput}

\end{sphinxuseclass}
\begin{sphinxuseclass}{cell}\begin{sphinxVerbatimInput}

\begin{sphinxuseclass}{cell_input}
\begin{sphinxVerbatim}[commandchars=\\\{\}]
\PYG{k+kn}{from} \PYG{n+nn}{statsmodels}\PYG{n+nn}{.}\PYG{n+nn}{regression}\PYG{n+nn}{.}\PYG{n+nn}{rolling} \PYG{k+kn}{import} \PYG{n}{RollingOLS}
\end{sphinxVerbatim}

\end{sphinxuseclass}\end{sphinxVerbatimInput}

\end{sphinxuseclass}
\begin{sphinxuseclass}{cell}\begin{sphinxVerbatimInput}

\begin{sphinxuseclass}{cell_input}
\begin{sphinxVerbatim}[commandchars=\\\{\}]
\PYG{n}{coefs} \PYG{o}{=} \PYG{p}{(}
    \PYG{n}{RollingOLS}\PYG{o}{.}\PYG{n}{from\PYGZus{}formula}\PYG{p}{(}
        \PYG{n}{formula}\PYG{o}{=}\PYG{l+s+s1}{\PYGZsq{}}\PYG{l+s+s1}{I(R\PYGZhy{}RF) \PYGZti{} Q(}\PYG{l+s+s1}{\PYGZdq{}}\PYG{l+s+s1}{Mkt\PYGZhy{}RF}\PYG{l+s+s1}{\PYGZdq{}}\PYG{l+s+s1}{) + SMB + HML}\PYG{l+s+s1}{\PYGZsq{}}\PYG{p}{,}
        \PYG{n}{data}\PYG{o}{=}\PYG{n}{brk}\PYG{p}{,}
        \PYG{n}{window}\PYG{o}{=}\PYG{l+m+mi}{3}\PYG{o}{*}\PYG{l+m+mi}{252}
    \PYG{p}{)}
    \PYG{o}{.}\PYG{n}{fit}\PYG{p}{(}\PYG{p}{)}
    \PYG{o}{.}\PYG{n}{params}
    \PYG{o}{.}\PYG{n}{rename\PYGZus{}axis}\PYG{p}{(}\PYG{n}{columns}\PYG{o}{=}\PYG{l+s+s1}{\PYGZsq{}}\PYG{l+s+s1}{Coefficient}\PYG{l+s+s1}{\PYGZsq{}}\PYG{p}{)}
    \PYG{o}{.}\PYG{n}{rename}\PYG{p}{(}\PYG{n}{columns}\PYG{o}{=}\PYG{p}{\PYGZob{}}\PYG{l+s+s1}{\PYGZsq{}}\PYG{l+s+s1}{Q(}\PYG{l+s+s1}{\PYGZdq{}}\PYG{l+s+s1}{Mkt\PYGZhy{}RF}\PYG{l+s+s1}{\PYGZdq{}}\PYG{l+s+s1}{)}\PYG{l+s+s1}{\PYGZsq{}}\PYG{p}{:} \PYG{l+s+s1}{\PYGZsq{}}\PYG{l+s+s1}{Mkt\PYGZhy{}RF}\PYG{l+s+s1}{\PYGZsq{}}\PYG{p}{\PYGZcb{}}\PYG{p}{)}
\PYG{p}{)}
\end{sphinxVerbatim}

\end{sphinxuseclass}\end{sphinxVerbatimInput}

\end{sphinxuseclass}
\begin{sphinxuseclass}{cell}\begin{sphinxVerbatimInput}

\begin{sphinxuseclass}{cell_input}
\begin{sphinxVerbatim}[commandchars=\\\{\}]
\PYG{n}{fig}\PYG{p}{,} \PYG{n}{ax} \PYG{o}{=} \PYG{n}{plt}\PYG{o}{.}\PYG{n}{subplots}\PYG{p}{(}\PYG{l+m+mi}{2}\PYG{p}{,} \PYG{l+m+mi}{1}\PYG{p}{,} \PYG{n}{sharex}\PYG{o}{=}\PYG{k+kc}{True}\PYG{p}{)}
\PYG{n}{coefs}\PYG{p}{[}\PYG{l+s+s1}{\PYGZsq{}}\PYG{l+s+s1}{Intercept}\PYG{l+s+s1}{\PYGZsq{}}\PYG{p}{]}\PYG{o}{.}\PYG{n}{plot}\PYG{p}{(}\PYG{n}{ax}\PYG{o}{=}\PYG{n}{ax}\PYG{p}{[}\PYG{l+m+mi}{0}\PYG{p}{]}\PYG{p}{,} \PYG{n}{legend}\PYG{o}{=}\PYG{k+kc}{True}\PYG{p}{)}
\PYG{n}{coefs}\PYG{o}{.}\PYG{n}{drop}\PYG{p}{(}\PYG{l+s+s1}{\PYGZsq{}}\PYG{l+s+s1}{Intercept}\PYG{l+s+s1}{\PYGZsq{}}\PYG{p}{,} \PYG{n}{axis}\PYG{o}{=}\PYG{l+m+mi}{1}\PYG{p}{)}\PYG{o}{.}\PYG{n}{plot}\PYG{p}{(}\PYG{n}{ax}\PYG{o}{=}\PYG{n}{ax}\PYG{p}{[}\PYG{l+m+mi}{1}\PYG{p}{]}\PYG{p}{)}
\PYG{n}{plt}\PYG{o}{.}\PYG{n}{suptitle}\PYG{p}{(}
    \PYG{l+s+s1}{\PYGZsq{}}\PYG{l+s+s1}{Rolling Three\PYGZhy{}Factor Regressions}\PYG{l+s+s1}{\PYGZsq{}} \PYG{o}{+}
    \PYG{l+s+s1}{\PYGZsq{}}\PYG{l+s+se}{\PYGZbs{}n}\PYG{l+s+s1}{Three\PYGZhy{}Year Windows with Daily Returns in Percent}\PYG{l+s+s1}{\PYGZsq{}}
\PYG{p}{)}
\PYG{n}{plt}\PYG{o}{.}\PYG{n}{show}\PYG{p}{(}\PYG{p}{)}
\end{sphinxVerbatim}

\end{sphinxuseclass}\end{sphinxVerbatimInput}
\begin{sphinxVerbatimOutput}

\begin{sphinxuseclass}{cell_output}
\noindent\sphinxincludegraphics{{f3f7c57e9982c23fc26048b9555cbf9edabb9a944fc6dc1da29461abc007d976}.png}

\end{sphinxuseclass}\end{sphinxVerbatimOutput}

\end{sphinxuseclass}
\sphinxAtStartPar
Buffett’s \$\textbackslash{}alpha\$ was large, but has declined to zero.
Also, his loading on SMB (size factor) has gone from positive to negative, indicating that he has moved from small stocks to large stocks as Berkshire Hathaway has grown.


\subsubsection{Use the three\sphinxhyphen{}, four\sphinxhyphen{}, and five\sphinxhyphen{}factor models to determine how the ARKK Innovation ETF generates returns}
\label{\detokenize{herron_03_practice_04:use-the-three-four-and-five-factor-models-to-determine-how-the-arkk-innovation-etf-generates-returns}}
\sphinxstepscope


\section{Herron Topic 3 \sphinxhyphen{} Practice (Wednesday 2:45 PM, Section 2)}
\label{\detokenize{herron_03_practice_02:herron-topic-3-practice-wednesday-2-45-pm-section-2}}\label{\detokenize{herron_03_practice_02::doc}}

\subsection{Announcements}
\label{\detokenize{herron_03_practice_02:announcements}}\begin{itemize}
\item {} 
\sphinxAtStartPar
Team Project 1 grades posted
\begin{itemize}
\item {} 
\sphinxAtStartPar
Mean of 81\% and median of 84\%

\item {} 
\sphinxAtStartPar
100\% on Teammates reviews pushes mean to 84\%, which is about the target

\item {} 
\sphinxAtStartPar
Overall course mean is 86\%, so I would curve up overall course grades about 4\% if posted grades today, much more at the low end

\item {} 
\sphinxAtStartPar
However, I cannot commit to a curve today

\end{itemize}

\item {} 
\sphinxAtStartPar
20,000 XP on DataCamp due by 11:59 PM on Friday

\end{itemize}


\subsection{Practice}
\label{\detokenize{herron_03_practice_02:practice}}
\begin{sphinxuseclass}{cell}\begin{sphinxVerbatimInput}

\begin{sphinxuseclass}{cell_input}
\begin{sphinxVerbatim}[commandchars=\\\{\}]
\PYG{k+kn}{import} \PYG{n+nn}{matplotlib}\PYG{n+nn}{.}\PYG{n+nn}{pyplot} \PYG{k}{as} \PYG{n+nn}{plt}
\PYG{k+kn}{import} \PYG{n+nn}{numpy} \PYG{k}{as} \PYG{n+nn}{np}
\PYG{k+kn}{import} \PYG{n+nn}{pandas} \PYG{k}{as} \PYG{n+nn}{pd}
\end{sphinxVerbatim}

\end{sphinxuseclass}\end{sphinxVerbatimInput}

\end{sphinxuseclass}
\begin{sphinxuseclass}{cell}\begin{sphinxVerbatimInput}

\begin{sphinxuseclass}{cell_input}
\begin{sphinxVerbatim}[commandchars=\\\{\}]
\PYG{o}{\PYGZpc{}}\PYG{k}{config} InlineBackend.figure\PYGZus{}format = \PYGZsq{}retina\PYGZsq{}
\PYG{o}{\PYGZpc{}}\PYG{k}{precision} 4
\PYG{n}{pd}\PYG{o}{.}\PYG{n}{options}\PYG{o}{.}\PYG{n}{display}\PYG{o}{.}\PYG{n}{float\PYGZus{}format} \PYG{o}{=} \PYG{l+s+s1}{\PYGZsq{}}\PYG{l+s+si}{\PYGZob{}:.4f\PYGZcb{}}\PYG{l+s+s1}{\PYGZsq{}}\PYG{o}{.}\PYG{n}{format}
\end{sphinxVerbatim}

\end{sphinxuseclass}\end{sphinxVerbatimInput}

\end{sphinxuseclass}
\begin{sphinxuseclass}{cell}\begin{sphinxVerbatimInput}

\begin{sphinxuseclass}{cell_input}
\begin{sphinxVerbatim}[commandchars=\\\{\}]
\PYG{k+kn}{import} \PYG{n+nn}{yfinance} \PYG{k}{as} \PYG{n+nn}{yf}
\PYG{k+kn}{import} \PYG{n+nn}{pandas\PYGZus{}datareader} \PYG{k}{as} \PYG{n+nn}{pdr}
\PYG{k+kn}{import} \PYG{n+nn}{requests\PYGZus{}cache}
\PYG{n}{session} \PYG{o}{=} \PYG{n}{requests\PYGZus{}cache}\PYG{o}{.}\PYG{n}{CachedSession}\PYG{p}{(}\PYG{n}{expire\PYGZus{}after}\PYG{o}{=}\PYG{l+m+mi}{1}\PYG{p}{)}
\end{sphinxVerbatim}

\end{sphinxuseclass}\end{sphinxVerbatimInput}

\end{sphinxuseclass}

\subsubsection{Plot the security market line (SML) for a variety of asset classes}
\label{\detokenize{herron_03_practice_02:plot-the-security-market-line-sml-for-a-variety-of-asset-classes}}
\sphinxAtStartPar
Use the past three years of daily data for the following exhange traded funds (ETFs):
\begin{enumerate}
\sphinxsetlistlabels{\arabic}{enumi}{enumii}{}{.}%
\item {} 
\sphinxAtStartPar
SPY (SPDR—Standard and Poor’s Depository Receipts—ETF for the S\&P 500 index)

\item {} 
\sphinxAtStartPar
BIL (SPDR ETF for 1\sphinxhyphen{}3 month Treasury bills)

\item {} 
\sphinxAtStartPar
GLD (SPDR ETF for gold)

\item {} 
\sphinxAtStartPar
JNK (SPDR ETF for high\sphinxhyphen{}yield debt)

\item {} 
\sphinxAtStartPar
MDY (SPDR ETF for S\&P 400 mid\sphinxhyphen{}cap index)

\item {} 
\sphinxAtStartPar
SLY (SPDR ETF for S\&P 600 small\sphinxhyphen{}cap index)

\item {} 
\sphinxAtStartPar
SPBO (SPDR ETF for corporate bonds)

\item {} 
\sphinxAtStartPar
SPMB (SPDR ETF for mortgage\sphinxhyphen{}backed securities)

\item {} 
\sphinxAtStartPar
SPTL (SPDR ETF for long\sphinxhyphen{}term Treasury bonds)

\end{enumerate}

\begin{sphinxuseclass}{cell}\begin{sphinxVerbatimInput}

\begin{sphinxuseclass}{cell_input}
\begin{sphinxVerbatim}[commandchars=\\\{\}]
\PYG{n}{tickers} \PYG{o}{=} \PYG{p}{[}\PYG{l+s+s1}{\PYGZsq{}}\PYG{l+s+s1}{SPY}\PYG{l+s+s1}{\PYGZsq{}}\PYG{p}{,} \PYG{l+s+s1}{\PYGZsq{}}\PYG{l+s+s1}{BIL}\PYG{l+s+s1}{\PYGZsq{}}\PYG{p}{,} \PYG{l+s+s1}{\PYGZsq{}}\PYG{l+s+s1}{GLD}\PYG{l+s+s1}{\PYGZsq{}}\PYG{p}{,} \PYG{l+s+s1}{\PYGZsq{}}\PYG{l+s+s1}{JNK}\PYG{l+s+s1}{\PYGZsq{}}\PYG{p}{,} \PYG{l+s+s1}{\PYGZsq{}}\PYG{l+s+s1}{MDY}\PYG{l+s+s1}{\PYGZsq{}}\PYG{p}{,} \PYG{l+s+s1}{\PYGZsq{}}\PYG{l+s+s1}{SLY}\PYG{l+s+s1}{\PYGZsq{}}\PYG{p}{,} \PYG{l+s+s1}{\PYGZsq{}}\PYG{l+s+s1}{SPBO}\PYG{l+s+s1}{\PYGZsq{}}\PYG{p}{,} \PYG{l+s+s1}{\PYGZsq{}}\PYG{l+s+s1}{SPMB}\PYG{l+s+s1}{\PYGZsq{}}\PYG{p}{,} \PYG{l+s+s1}{\PYGZsq{}}\PYG{l+s+s1}{SPTL}\PYG{l+s+s1}{\PYGZsq{}}\PYG{p}{]}

\PYG{n}{etf} \PYG{o}{=} \PYG{p}{(}
    \PYG{n}{yf}\PYG{o}{.}\PYG{n}{download}\PYG{p}{(}
        \PYG{n}{tickers}\PYG{o}{=}\PYG{n}{tickers}\PYG{p}{,}
        \PYG{n}{progress}\PYG{o}{=}\PYG{k+kc}{False}
    \PYG{p}{)}
    \PYG{o}{.}\PYG{n}{assign}\PYG{p}{(}\PYG{n}{Date} \PYG{o}{=} \PYG{k}{lambda} \PYG{n}{x}\PYG{p}{:} \PYG{n}{x}\PYG{o}{.}\PYG{n}{index}\PYG{o}{.}\PYG{n}{tz\PYGZus{}localize}\PYG{p}{(}\PYG{k+kc}{None}\PYG{p}{)}\PYG{p}{)}
    \PYG{o}{.}\PYG{n}{set\PYGZus{}index}\PYG{p}{(}\PYG{l+s+s1}{\PYGZsq{}}\PYG{l+s+s1}{Date}\PYG{l+s+s1}{\PYGZsq{}}\PYG{p}{)}
    \PYG{o}{.}\PYG{n}{rename\PYGZus{}axis}\PYG{p}{(}\PYG{n}{columns}\PYG{o}{=}\PYG{p}{[}\PYG{l+s+s1}{\PYGZsq{}}\PYG{l+s+s1}{Variable}\PYG{l+s+s1}{\PYGZsq{}}\PYG{p}{,} \PYG{l+s+s1}{\PYGZsq{}}\PYG{l+s+s1}{Ticker}\PYG{l+s+s1}{\PYGZsq{}}\PYG{p}{]}\PYG{p}{)}
    \PYG{p}{[}\PYG{l+s+s1}{\PYGZsq{}}\PYG{l+s+s1}{Adj Close}\PYG{l+s+s1}{\PYGZsq{}}\PYG{p}{]}
    \PYG{o}{.}\PYG{n}{pct\PYGZus{}change}\PYG{p}{(}\PYG{p}{)}
\PYG{p}{)}

\PYG{n}{etf}\PYG{o}{.}\PYG{n}{tail}\PYG{p}{(}\PYG{p}{)}
\end{sphinxVerbatim}

\end{sphinxuseclass}\end{sphinxVerbatimInput}
\begin{sphinxVerbatimOutput}

\begin{sphinxuseclass}{cell_output}
\begin{sphinxVerbatim}[commandchars=\\\{\}]
Ticker        BIL     GLD     JNK     MDY     SLY    SPBO    SPMB    SPTL  \PYGZbs{}
Date                                                                        
2023\PYGZhy{}03\PYGZhy{}16 0.0004  0.0020  0.0066  0.0138  0.0159 \PYGZhy{}0.0031 \PYGZhy{}0.0036 \PYGZhy{}0.0088   
2023\PYGZhy{}03\PYGZhy{}17 0.0002  0.0291 \PYGZhy{}0.0060 \PYGZhy{}0.0228 \PYGZhy{}0.0278  0.0038  0.0086  0.0134   
2023\PYGZhy{}03\PYGZhy{}20 0.0001  0.0004 \PYGZhy{}0.0027  0.0166  0.0155 \PYGZhy{}0.0010 \PYGZhy{}0.0067 \PYGZhy{}0.0081   
2023\PYGZhy{}03\PYGZhy{}21 0.0000 \PYGZhy{}0.0189  0.0114  0.0182  0.0165  0.0031 \PYGZhy{}0.0036 \PYGZhy{}0.0088   
2023\PYGZhy{}03\PYGZhy{}22 0.0002  0.0170  0.0011 \PYGZhy{}0.0254 \PYGZhy{}0.0268  0.0055  0.0141  0.0131   

Ticker         SPY  
Date                
2023\PYGZhy{}03\PYGZhy{}16  0.0175  
2023\PYGZhy{}03\PYGZhy{}17 \PYGZhy{}0.0117  
2023\PYGZhy{}03\PYGZhy{}20  0.0096  
2023\PYGZhy{}03\PYGZhy{}21  0.0131  
2023\PYGZhy{}03\PYGZhy{}22 \PYGZhy{}0.0170  
\end{sphinxVerbatim}

\end{sphinxuseclass}\end{sphinxVerbatimOutput}

\end{sphinxuseclass}
\begin{sphinxuseclass}{cell}\begin{sphinxVerbatimInput}

\begin{sphinxuseclass}{cell_input}
\begin{sphinxVerbatim}[commandchars=\\\{\}]
\PYG{n}{ff} \PYG{o}{=} \PYG{p}{(}
    \PYG{n}{pdr}\PYG{o}{.}\PYG{n}{DataReader}\PYG{p}{(}
        \PYG{n}{name}\PYG{o}{=}\PYG{l+s+s1}{\PYGZsq{}}\PYG{l+s+s1}{F\PYGZhy{}F\PYGZus{}Research\PYGZus{}Data\PYGZus{}Factors\PYGZus{}daily}\PYG{l+s+s1}{\PYGZsq{}}\PYG{p}{,}
        \PYG{n}{data\PYGZus{}source}\PYG{o}{=}\PYG{l+s+s1}{\PYGZsq{}}\PYG{l+s+s1}{famafrench}\PYG{l+s+s1}{\PYGZsq{}}\PYG{p}{,}
        \PYG{n}{start}\PYG{o}{=}\PYG{l+s+s1}{\PYGZsq{}}\PYG{l+s+s1}{1900}\PYG{l+s+s1}{\PYGZsq{}}\PYG{p}{,}
        \PYG{n}{session}\PYG{o}{=}\PYG{n}{session}
    \PYG{p}{)}
    \PYG{p}{[}\PYG{l+m+mi}{0}\PYG{p}{]}
    \PYG{o}{.}\PYG{n}{rename\PYGZus{}axis}\PYG{p}{(}\PYG{n}{columns}\PYG{o}{=}\PYG{l+s+s1}{\PYGZsq{}}\PYG{l+s+s1}{Variable}\PYG{l+s+s1}{\PYGZsq{}}\PYG{p}{)}
    \PYG{o}{.}\PYG{n}{div}\PYG{p}{(}\PYG{l+m+mi}{100}\PYG{p}{)}
\PYG{p}{)}

\PYG{n}{ff}\PYG{o}{.}\PYG{n}{tail}\PYG{p}{(}\PYG{p}{)}
\end{sphinxVerbatim}

\end{sphinxuseclass}\end{sphinxVerbatimInput}
\begin{sphinxVerbatimOutput}

\begin{sphinxuseclass}{cell_output}
\begin{sphinxVerbatim}[commandchars=\\\{\}]
Variable    Mkt\PYGZhy{}RF     SMB     HML     RF
Date                                     
2022\PYGZhy{}12\PYGZhy{}23  0.0051 \PYGZhy{}0.0060  0.0115 0.0002
2022\PYGZhy{}12\PYGZhy{}27 \PYGZhy{}0.0051 \PYGZhy{}0.0073  0.0142 0.0002
2022\PYGZhy{}12\PYGZhy{}28 \PYGZhy{}0.0123 \PYGZhy{}0.0025 \PYGZhy{}0.0029 0.0002
2022\PYGZhy{}12\PYGZhy{}29  0.0187  0.0127 \PYGZhy{}0.0107 0.0002
2022\PYGZhy{}12\PYGZhy{}30 \PYGZhy{}0.0022  0.0011 \PYGZhy{}0.0003 0.0002
\end{sphinxVerbatim}

\end{sphinxuseclass}\end{sphinxVerbatimOutput}

\end{sphinxuseclass}
\begin{sphinxuseclass}{cell}\begin{sphinxVerbatimInput}

\begin{sphinxuseclass}{cell_input}
\begin{sphinxVerbatim}[commandchars=\\\{\}]
\PYG{k}{def} \PYG{n+nf}{mean}\PYG{p}{(}\PYG{n}{ri}\PYG{p}{,} \PYG{n}{ann}\PYG{o}{=}\PYG{l+m+mi}{252}\PYG{p}{,} \PYG{n}{mul}\PYG{o}{=}\PYG{l+m+mi}{100}\PYG{p}{)}\PYG{p}{:}
    \PYG{k}{return} \PYG{n}{mul} \PYG{o}{*} \PYG{n}{ann} \PYG{o}{*} \PYG{n}{ri}\PYG{o}{.}\PYG{n}{mean}\PYG{p}{(}\PYG{p}{)}
\end{sphinxVerbatim}

\end{sphinxuseclass}\end{sphinxVerbatimInput}

\end{sphinxuseclass}
\begin{sphinxuseclass}{cell}\begin{sphinxVerbatimInput}

\begin{sphinxuseclass}{cell_input}
\begin{sphinxVerbatim}[commandchars=\\\{\}]
\PYG{k}{def} \PYG{n+nf}{beta}\PYG{p}{(}\PYG{n}{ri}\PYG{p}{,} \PYG{n}{rf}\PYG{o}{=}\PYG{n}{ff}\PYG{p}{[}\PYG{l+s+s1}{\PYGZsq{}}\PYG{l+s+s1}{RF}\PYG{l+s+s1}{\PYGZsq{}}\PYG{p}{]}\PYG{p}{,} \PYG{n}{rm\PYGZus{}rf}\PYG{o}{=}\PYG{n}{ff}\PYG{p}{[}\PYG{l+s+s1}{\PYGZsq{}}\PYG{l+s+s1}{Mkt\PYGZhy{}RF}\PYG{l+s+s1}{\PYGZsq{}}\PYG{p}{]}\PYG{p}{)}\PYG{p}{:}
    \PYG{n}{ri\PYGZus{}rf} \PYG{o}{=} \PYG{n}{ri}\PYG{o}{.}\PYG{n}{sub}\PYG{p}{(}\PYG{n}{rf}\PYG{p}{)}\PYG{o}{.}\PYG{n}{dropna}\PYG{p}{(}\PYG{p}{)}
    \PYG{k}{return} \PYG{n}{ri\PYGZus{}rf}\PYG{o}{.}\PYG{n}{cov}\PYG{p}{(}\PYG{n}{rm\PYGZus{}rf}\PYG{p}{)} \PYG{o}{/} \PYG{n}{rm\PYGZus{}rf}\PYG{o}{.}\PYG{n}{loc}\PYG{p}{[}\PYG{n}{ri\PYGZus{}rf}\PYG{o}{.}\PYG{n}{index}\PYG{p}{]}\PYG{o}{.}\PYG{n}{var}\PYG{p}{(}\PYG{p}{)}
\end{sphinxVerbatim}

\end{sphinxuseclass}\end{sphinxVerbatimInput}

\end{sphinxuseclass}
\begin{sphinxuseclass}{cell}\begin{sphinxVerbatimInput}

\begin{sphinxuseclass}{cell_input}
\begin{sphinxVerbatim}[commandchars=\\\{\}]
\PYG{k}{def} \PYG{n+nf}{date\PYGZus{}range}\PYG{p}{(}\PYG{n}{x}\PYG{p}{)}\PYG{p}{:}
    \PYG{k}{return} \PYG{l+s+sa}{f}\PYG{l+s+s1}{\PYGZsq{}}\PYG{l+s+si}{\PYGZob{}}\PYG{n}{x}\PYG{o}{.}\PYG{n}{index}\PYG{p}{[}\PYG{l+m+mi}{0}\PYG{p}{]}\PYG{l+s+si}{:}\PYG{l+s+s1}{\PYGZpc{}b \PYGZpc{}d, \PYGZpc{}Y}\PYG{l+s+si}{\PYGZcb{}}\PYG{l+s+s1}{ to }\PYG{l+s+si}{\PYGZob{}}\PYG{n}{x}\PYG{o}{.}\PYG{n}{index}\PYG{p}{[}\PYG{o}{\PYGZhy{}}\PYG{l+m+mi}{1}\PYG{p}{]}\PYG{l+s+si}{:}\PYG{l+s+s1}{\PYGZpc{}b \PYGZpc{}d, \PYGZpc{}Y}\PYG{l+s+si}{\PYGZcb{}}\PYG{l+s+s1}{\PYGZsq{}}
\end{sphinxVerbatim}

\end{sphinxuseclass}\end{sphinxVerbatimInput}

\end{sphinxuseclass}
\begin{sphinxuseclass}{cell}\begin{sphinxVerbatimInput}

\begin{sphinxuseclass}{cell_input}
\begin{sphinxVerbatim}[commandchars=\\\{\}]
\PYG{k+kn}{import} \PYG{n+nn}{seaborn} \PYG{k}{as} \PYG{n+nn}{sns}
\end{sphinxVerbatim}

\end{sphinxuseclass}\end{sphinxVerbatimInput}

\end{sphinxuseclass}
\begin{sphinxuseclass}{cell}\begin{sphinxVerbatimInput}

\begin{sphinxuseclass}{cell_input}
\begin{sphinxVerbatim}[commandchars=\\\{\}]
\PYG{n}{\PYGZus{}} \PYG{o}{=} \PYG{p}{(}
    \PYG{n}{etf}
    \PYG{o}{.}\PYG{n}{iloc}\PYG{p}{[}\PYG{o}{\PYGZhy{}}\PYG{l+m+mi}{756}\PYG{p}{:}\PYG{p}{]} \PYG{c+c1}{\PYGZsh{} I forget to slice to 3 years in class}
    \PYG{o}{.}\PYG{n}{agg}\PYG{p}{(}\PYG{p}{[}\PYG{n}{mean}\PYG{p}{,} \PYG{n}{beta}\PYG{p}{]}\PYG{p}{)}
    \PYG{o}{.}\PYG{n}{rename\PYGZus{}axis}\PYG{p}{(}\PYG{n}{index}\PYG{o}{=}\PYG{l+s+s1}{\PYGZsq{}}\PYG{l+s+s1}{Statistic}\PYG{l+s+s1}{\PYGZsq{}}\PYG{p}{)}
    \PYG{o}{.}\PYG{n}{transpose}\PYG{p}{(}\PYG{p}{)}
\PYG{p}{)}
 
\PYG{n}{sns}\PYG{o}{.}\PYG{n}{regplot}\PYG{p}{(}\PYG{n}{x}\PYG{o}{=}\PYG{l+s+s1}{\PYGZsq{}}\PYG{l+s+s1}{beta}\PYG{l+s+s1}{\PYGZsq{}}\PYG{p}{,} \PYG{n}{y}\PYG{o}{=}\PYG{l+s+s1}{\PYGZsq{}}\PYG{l+s+s1}{mean}\PYG{l+s+s1}{\PYGZsq{}}\PYG{p}{,} \PYG{n}{data}\PYG{o}{=}\PYG{n}{\PYGZus{}}\PYG{p}{)}
\PYG{k}{for} \PYG{n}{t}\PYG{p}{,} \PYG{p}{(}\PYG{n}{x}\PYG{p}{,} \PYG{n}{y}\PYG{p}{)} \PYG{o+ow}{in} \PYG{n}{\PYGZus{}}\PYG{p}{[}\PYG{p}{[}\PYG{l+s+s1}{\PYGZsq{}}\PYG{l+s+s1}{beta}\PYG{l+s+s1}{\PYGZsq{}}\PYG{p}{,} \PYG{l+s+s1}{\PYGZsq{}}\PYG{l+s+s1}{mean}\PYG{l+s+s1}{\PYGZsq{}}\PYG{p}{]}\PYG{p}{]}\PYG{o}{.}\PYG{n}{iterrows}\PYG{p}{(}\PYG{p}{)}\PYG{p}{:}
    \PYG{n}{plt}\PYG{o}{.}\PYG{n}{annotate}\PYG{p}{(}\PYG{n}{text}\PYG{o}{=}\PYG{n}{t}\PYG{p}{,} \PYG{n}{xy}\PYG{o}{=}\PYG{p}{(}\PYG{n}{x}\PYG{p}{,} \PYG{n}{y}\PYG{p}{)}\PYG{p}{)} \PYG{c+c1}{\PYGZsh{} use a for\PYGZhy{}loop to add tickers}

\PYG{n}{plt}\PYG{o}{.}\PYG{n}{ylabel}\PYG{p}{(}\PYG{l+s+s1}{\PYGZsq{}}\PYG{l+s+s1}{Annualized Mean of Daily Returns (}\PYG{l+s+s1}{\PYGZpc{}}\PYG{l+s+s1}{)}\PYG{l+s+s1}{\PYGZsq{}}\PYG{p}{)}
\PYG{n}{plt}\PYG{o}{.}\PYG{n}{xlabel}\PYG{p}{(}\PYG{l+s+sa}{r}\PYG{l+s+s1}{\PYGZsq{}}\PYG{l+s+s1}{Capital Asset Pricing Model (CAPM) \PYGZdl{}}\PYG{l+s+s1}{\PYGZbs{}}\PYG{l+s+s1}{beta\PYGZdl{}}\PYG{l+s+s1}{\PYGZsq{}}\PYG{p}{)}
\PYG{n}{plt}\PYG{o}{.}\PYG{n}{title}\PYG{p}{(}
    \PYG{l+s+s1}{\PYGZsq{}}\PYG{l+s+s1}{Security Market Line (SML) for Exchange Traded Funds (ETFs)}\PYG{l+s+se}{\PYGZbs{}n}\PYG{l+s+s1}{\PYGZsq{}} \PYG{o}{+}
    \PYG{l+s+s1}{\PYGZsq{}}\PYG{l+s+s1}{for Daily Returns from }\PYG{l+s+s1}{\PYGZsq{}} \PYG{o}{+} \PYG{n}{etf}\PYG{o}{.}\PYG{n}{iloc}\PYG{p}{[}\PYG{o}{\PYGZhy{}}\PYG{l+m+mi}{756}\PYG{p}{:}\PYG{p}{]}\PYG{o}{.}\PYG{n}{pipe}\PYG{p}{(}\PYG{n}{date\PYGZus{}range}\PYG{p}{)}\PYG{p}{)}
\PYG{n}{plt}\PYG{o}{.}\PYG{n}{show}\PYG{p}{(}\PYG{p}{)}
\end{sphinxVerbatim}

\end{sphinxuseclass}\end{sphinxVerbatimInput}
\begin{sphinxVerbatimOutput}

\begin{sphinxuseclass}{cell_output}
\noindent\sphinxincludegraphics{{82d4e71dc5728c1deb211e0155d73a42aad3ff04065c726109c6621637cd9c44}.png}

\end{sphinxuseclass}\end{sphinxVerbatimOutput}

\end{sphinxuseclass}

\subsubsection{Plot the SML for the Dow Jones Industrial Average (DJIA) stocks}
\label{\detokenize{herron_03_practice_02:plot-the-sml-for-the-dow-jones-industrial-average-djia-stocks}}
\sphinxAtStartPar
Use the past three years of daily returns data for the stocks listed on the \sphinxhref{https://en.wikipedia.org/wiki/Dow\_Jones\_Industrial\_Average}{DJIA Wikipedia page}.
Compare the DJIA SML to the asset class SML above.

\begin{sphinxuseclass}{cell}\begin{sphinxVerbatimInput}

\begin{sphinxuseclass}{cell_input}
\begin{sphinxVerbatim}[commandchars=\\\{\}]
\PYG{n}{wiki} \PYG{o}{=} \PYG{n}{pd}\PYG{o}{.}\PYG{n}{read\PYGZus{}html}\PYG{p}{(}\PYG{l+s+s1}{\PYGZsq{}}\PYG{l+s+s1}{https://en.wikipedia.org/wiki/Dow\PYGZus{}Jones\PYGZus{}Industrial\PYGZus{}Average}\PYG{l+s+s1}{\PYGZsq{}}\PYG{p}{)}
\end{sphinxVerbatim}

\end{sphinxuseclass}\end{sphinxVerbatimInput}

\end{sphinxuseclass}
\begin{sphinxuseclass}{cell}\begin{sphinxVerbatimInput}

\begin{sphinxuseclass}{cell_input}
\begin{sphinxVerbatim}[commandchars=\\\{\}]
\PYG{n}{djia} \PYG{o}{=} \PYG{p}{(}
    \PYG{n}{yf}\PYG{o}{.}\PYG{n}{download}\PYG{p}{(}
        \PYG{n}{tickers}\PYG{o}{=}\PYG{n}{wiki}\PYG{p}{[}\PYG{l+m+mi}{1}\PYG{p}{]}\PYG{p}{[}\PYG{l+s+s1}{\PYGZsq{}}\PYG{l+s+s1}{Symbol}\PYG{l+s+s1}{\PYGZsq{}}\PYG{p}{]}\PYG{o}{.}\PYG{n}{to\PYGZus{}list}\PYG{p}{(}\PYG{p}{)}\PYG{p}{,}
        \PYG{n}{progress}\PYG{o}{=}\PYG{k+kc}{False}
    \PYG{p}{)}
    \PYG{o}{.}\PYG{n}{assign}\PYG{p}{(}\PYG{n}{Date} \PYG{o}{=} \PYG{k}{lambda} \PYG{n}{x}\PYG{p}{:} \PYG{n}{x}\PYG{o}{.}\PYG{n}{index}\PYG{o}{.}\PYG{n}{tz\PYGZus{}localize}\PYG{p}{(}\PYG{k+kc}{None}\PYG{p}{)}\PYG{p}{)}
    \PYG{o}{.}\PYG{n}{set\PYGZus{}index}\PYG{p}{(}\PYG{l+s+s1}{\PYGZsq{}}\PYG{l+s+s1}{Date}\PYG{l+s+s1}{\PYGZsq{}}\PYG{p}{)}
    \PYG{o}{.}\PYG{n}{rename\PYGZus{}axis}\PYG{p}{(}\PYG{n}{columns}\PYG{o}{=}\PYG{p}{[}\PYG{l+s+s1}{\PYGZsq{}}\PYG{l+s+s1}{Variable}\PYG{l+s+s1}{\PYGZsq{}}\PYG{p}{,} \PYG{l+s+s1}{\PYGZsq{}}\PYG{l+s+s1}{Ticker}\PYG{l+s+s1}{\PYGZsq{}}\PYG{p}{]}\PYG{p}{)}
    \PYG{p}{[}\PYG{l+s+s1}{\PYGZsq{}}\PYG{l+s+s1}{Adj Close}\PYG{l+s+s1}{\PYGZsq{}}\PYG{p}{]}
    \PYG{o}{.}\PYG{n}{pct\PYGZus{}change}\PYG{p}{(}\PYG{p}{)}
    \PYG{o}{.}\PYG{n}{dropna}\PYG{p}{(}\PYG{p}{)}
\PYG{p}{)}
\end{sphinxVerbatim}

\end{sphinxuseclass}\end{sphinxVerbatimInput}

\end{sphinxuseclass}
\begin{sphinxuseclass}{cell}\begin{sphinxVerbatimInput}

\begin{sphinxuseclass}{cell_input}
\begin{sphinxVerbatim}[commandchars=\\\{\}]
\PYG{n}{\PYGZus{}} \PYG{o}{=} \PYG{p}{(}
    \PYG{n}{djia}
    \PYG{o}{.}\PYG{n}{iloc}\PYG{p}{[}\PYG{o}{\PYGZhy{}}\PYG{l+m+mi}{756}\PYG{p}{:}\PYG{p}{]} \PYG{c+c1}{\PYGZsh{} I forget to slice to 3 years in class}
    \PYG{o}{.}\PYG{n}{agg}\PYG{p}{(}\PYG{p}{[}\PYG{n}{mean}\PYG{p}{,} \PYG{n}{beta}\PYG{p}{]}\PYG{p}{)}
    \PYG{o}{.}\PYG{n}{rename\PYGZus{}axis}\PYG{p}{(}\PYG{n}{index}\PYG{o}{=}\PYG{l+s+s1}{\PYGZsq{}}\PYG{l+s+s1}{Statistic}\PYG{l+s+s1}{\PYGZsq{}}\PYG{p}{)}
    \PYG{o}{.}\PYG{n}{transpose}\PYG{p}{(}\PYG{p}{)}
\PYG{p}{)}
 
\PYG{n}{sns}\PYG{o}{.}\PYG{n}{regplot}\PYG{p}{(}\PYG{n}{x}\PYG{o}{=}\PYG{l+s+s1}{\PYGZsq{}}\PYG{l+s+s1}{beta}\PYG{l+s+s1}{\PYGZsq{}}\PYG{p}{,} \PYG{n}{y}\PYG{o}{=}\PYG{l+s+s1}{\PYGZsq{}}\PYG{l+s+s1}{mean}\PYG{l+s+s1}{\PYGZsq{}}\PYG{p}{,} \PYG{n}{data}\PYG{o}{=}\PYG{n}{\PYGZus{}}\PYG{p}{)}
\PYG{k}{for} \PYG{n}{t}\PYG{p}{,} \PYG{p}{(}\PYG{n}{x}\PYG{p}{,} \PYG{n}{y}\PYG{p}{)} \PYG{o+ow}{in} \PYG{n}{\PYGZus{}}\PYG{p}{[}\PYG{p}{[}\PYG{l+s+s1}{\PYGZsq{}}\PYG{l+s+s1}{beta}\PYG{l+s+s1}{\PYGZsq{}}\PYG{p}{,} \PYG{l+s+s1}{\PYGZsq{}}\PYG{l+s+s1}{mean}\PYG{l+s+s1}{\PYGZsq{}}\PYG{p}{]}\PYG{p}{]}\PYG{o}{.}\PYG{n}{iterrows}\PYG{p}{(}\PYG{p}{)}\PYG{p}{:}
    \PYG{n}{plt}\PYG{o}{.}\PYG{n}{annotate}\PYG{p}{(}\PYG{n}{text}\PYG{o}{=}\PYG{n}{t}\PYG{p}{,} \PYG{n}{xy}\PYG{o}{=}\PYG{p}{(}\PYG{n}{x}\PYG{p}{,} \PYG{n}{y}\PYG{p}{)}\PYG{p}{)} \PYG{c+c1}{\PYGZsh{} use a for\PYGZhy{}loop to add tickers}

\PYG{n}{plt}\PYG{o}{.}\PYG{n}{ylabel}\PYG{p}{(}\PYG{l+s+s1}{\PYGZsq{}}\PYG{l+s+s1}{Annualized Mean of Daily Returns (}\PYG{l+s+s1}{\PYGZpc{}}\PYG{l+s+s1}{)}\PYG{l+s+s1}{\PYGZsq{}}\PYG{p}{)}
\PYG{n}{plt}\PYG{o}{.}\PYG{n}{xlabel}\PYG{p}{(}\PYG{l+s+sa}{r}\PYG{l+s+s1}{\PYGZsq{}}\PYG{l+s+s1}{Capital Asset Pricing Model (CAPM) \PYGZdl{}}\PYG{l+s+s1}{\PYGZbs{}}\PYG{l+s+s1}{beta\PYGZdl{}}\PYG{l+s+s1}{\PYGZsq{}}\PYG{p}{)}
\PYG{n}{plt}\PYG{o}{.}\PYG{n}{title}\PYG{p}{(}
    \PYG{l+s+s1}{\PYGZsq{}}\PYG{l+s+s1}{Security Market Line (SML) for Dow\PYGZhy{}Jones Stocks}\PYG{l+s+se}{\PYGZbs{}n}\PYG{l+s+s1}{\PYGZsq{}} \PYG{o}{+}
    \PYG{l+s+s1}{\PYGZsq{}}\PYG{l+s+s1}{for Daily Returns from }\PYG{l+s+s1}{\PYGZsq{}} \PYG{o}{+} \PYG{n}{djia}\PYG{o}{.}\PYG{n}{iloc}\PYG{p}{[}\PYG{o}{\PYGZhy{}}\PYG{l+m+mi}{756}\PYG{p}{:}\PYG{p}{]}\PYG{o}{.}\PYG{n}{pipe}\PYG{p}{(}\PYG{n}{date\PYGZus{}range}\PYG{p}{)}\PYG{p}{)}
\PYG{n}{plt}\PYG{o}{.}\PYG{n}{show}\PYG{p}{(}\PYG{p}{)}
\end{sphinxVerbatim}

\end{sphinxuseclass}\end{sphinxVerbatimInput}
\begin{sphinxVerbatimOutput}

\begin{sphinxuseclass}{cell_output}
\noindent\sphinxincludegraphics{{ed5f8cc5ec3bfb78503eb0e92e9a8258d466e02a363bb897cf79bc4aab5eae85}.png}

\end{sphinxuseclass}\end{sphinxVerbatimOutput}

\end{sphinxuseclass}

\subsubsection{Plot the SML for the five portfolios formed on beta}
\label{\detokenize{herron_03_practice_02:plot-the-sml-for-the-five-portfolios-formed-on-beta}}
\sphinxAtStartPar
Download data for portfolios formed on \$\textbackslash{}beta\$ (\sphinxcode{\sphinxupquote{Portfolios\_Formed\_on\_BETA}}) from Ken French.
For the value\sphinxhyphen{}weighted portfolios, plot realized returns versus \$\textbackslash{}beta\$.
These data should elements \sphinxcode{\sphinxupquote{{[}2{]}}} and \sphinxcode{\sphinxupquote{{[}6{]}}}, respectively.

\begin{sphinxuseclass}{cell}\begin{sphinxVerbatimInput}

\begin{sphinxuseclass}{cell_input}
\begin{sphinxVerbatim}[commandchars=\\\{\}]
\PYG{n}{beta\PYGZus{}0} \PYG{o}{=} \PYG{p}{(}
    \PYG{n}{pdr}\PYG{o}{.}\PYG{n}{DataReader}\PYG{p}{(}
        \PYG{n}{name}\PYG{o}{=}\PYG{l+s+s1}{\PYGZsq{}}\PYG{l+s+s1}{Portfolios\PYGZus{}Formed\PYGZus{}on\PYGZus{}BETA}\PYG{l+s+s1}{\PYGZsq{}}\PYG{p}{,}
        \PYG{n}{data\PYGZus{}source}\PYG{o}{=}\PYG{l+s+s1}{\PYGZsq{}}\PYG{l+s+s1}{famafrench}\PYG{l+s+s1}{\PYGZsq{}}\PYG{p}{,}
        \PYG{n}{start}\PYG{o}{=}\PYG{l+s+s1}{\PYGZsq{}}\PYG{l+s+s1}{1900}\PYG{l+s+s1}{\PYGZsq{}}\PYG{p}{,}
        \PYG{n}{session}\PYG{o}{=}\PYG{n}{session}
    \PYG{p}{)}
\PYG{p}{)}

\PYG{n+nb}{print}\PYG{p}{(}\PYG{n}{beta\PYGZus{}0}\PYG{p}{[}\PYG{l+s+s1}{\PYGZsq{}}\PYG{l+s+s1}{DESCR}\PYG{l+s+s1}{\PYGZsq{}}\PYG{p}{]}\PYG{p}{)}
\end{sphinxVerbatim}

\end{sphinxuseclass}\end{sphinxVerbatimInput}
\begin{sphinxVerbatimOutput}

\begin{sphinxuseclass}{cell_output}
\begin{sphinxVerbatim}[commandchars=\\\{\}]
Portfolios Formed on BETA
\PYGZhy{}\PYGZhy{}\PYGZhy{}\PYGZhy{}\PYGZhy{}\PYGZhy{}\PYGZhy{}\PYGZhy{}\PYGZhy{}\PYGZhy{}\PYGZhy{}\PYGZhy{}\PYGZhy{}\PYGZhy{}\PYGZhy{}\PYGZhy{}\PYGZhy{}\PYGZhy{}\PYGZhy{}\PYGZhy{}\PYGZhy{}\PYGZhy{}\PYGZhy{}\PYGZhy{}\PYGZhy{}

This file was created by CMPT\PYGZus{}BETA\PYGZus{}RETS using the 202301 CRSP database. It contains value\PYGZhy{} and equal\PYGZhy{}weighted returns for portfolios formed on BETA. The portfolios are constructed at the end of June. Beta is estimated using monthly returns for the past 60 months (requiring at least 24 months with non\PYGZhy{}missing returns). Beta is estimated using the Scholes\PYGZhy{}Williams method. Annual returns are from January to December. Missing data are indicated by \PYGZhy{}99.99 or \PYGZhy{}999. The break points include utilities and include financials. The portfolios include utilities and include financials. Copyright 2023 Kenneth R. French

  0 : Value Weighted Returns \PYGZhy{}\PYGZhy{} Monthly (715 rows x 15 cols)
  1 : Equal Weighted Returns \PYGZhy{}\PYGZhy{} Monthly (715 rows x 15 cols)
  2 : Value Weighted Returns \PYGZhy{}\PYGZhy{} Annual from January to December (59 rows x 15 cols)
  3 : Equal Weighted Returns \PYGZhy{}\PYGZhy{} Annual from January to December (59 rows x 15 cols)
  4 : Number of Firms in Portfolios (715 rows x 15 cols)
  5 : Average Firm Size (715 rows x 15 cols)
  6 : Value\PYGZhy{}Weighted Average of Prior Beta (60 rows x 15 cols)
\end{sphinxVerbatim}

\end{sphinxuseclass}\end{sphinxVerbatimOutput}

\end{sphinxuseclass}
\begin{sphinxuseclass}{cell}\begin{sphinxVerbatimInput}

\begin{sphinxuseclass}{cell_input}
\begin{sphinxVerbatim}[commandchars=\\\{\}]
\PYG{n}{ports} \PYG{o}{=} \PYG{p}{[}\PYG{l+s+s1}{\PYGZsq{}}\PYG{l+s+s1}{Lo 20}\PYG{l+s+s1}{\PYGZsq{}}\PYG{p}{,} \PYG{l+s+s1}{\PYGZsq{}}\PYG{l+s+s1}{Qnt 2}\PYG{l+s+s1}{\PYGZsq{}}\PYG{p}{,} \PYG{l+s+s1}{\PYGZsq{}}\PYG{l+s+s1}{Qnt 3}\PYG{l+s+s1}{\PYGZsq{}}\PYG{p}{,} \PYG{l+s+s1}{\PYGZsq{}}\PYG{l+s+s1}{Qnt 4}\PYG{l+s+s1}{\PYGZsq{}}\PYG{p}{,} \PYG{l+s+s1}{\PYGZsq{}}\PYG{l+s+s1}{Hi 20}\PYG{l+s+s1}{\PYGZsq{}}\PYG{p}{]}

\PYG{n}{\PYGZus{}} \PYG{o}{=} \PYG{p}{(}
    \PYG{n}{pd}\PYG{o}{.}\PYG{n}{concat}\PYG{p}{(}
        \PYG{n}{objs}\PYG{o}{=}\PYG{p}{[}
            \PYG{n}{beta\PYGZus{}0}\PYG{p}{[}\PYG{l+m+mi}{2}\PYG{p}{]}\PYG{p}{[}\PYG{n}{ports}\PYG{p}{]}\PYG{p}{,}
            \PYG{n}{beta\PYGZus{}0}\PYG{p}{[}\PYG{l+m+mi}{6}\PYG{p}{]}\PYG{p}{[}\PYG{n}{ports}\PYG{p}{]}\PYG{o}{.}\PYG{n}{shift}\PYG{p}{(}\PYG{p}{)}
        \PYG{p}{]}\PYG{p}{,}
        \PYG{n}{keys}\PYG{o}{=}\PYG{p}{[}\PYG{l+s+s1}{\PYGZsq{}}\PYG{l+s+s1}{Return}\PYG{l+s+s1}{\PYGZsq{}}\PYG{p}{,} \PYG{l+s+s1}{\PYGZsq{}}\PYG{l+s+s1}{Beta}\PYG{l+s+s1}{\PYGZsq{}}\PYG{p}{]}\PYG{p}{,}
        \PYG{n}{names}\PYG{o}{=}\PYG{p}{[}\PYG{l+s+s1}{\PYGZsq{}}\PYG{l+s+s1}{Statistic}\PYG{l+s+s1}{\PYGZsq{}}\PYG{p}{,} \PYG{l+s+s1}{\PYGZsq{}}\PYG{l+s+s1}{Beta Portfolio}\PYG{l+s+s1}{\PYGZsq{}}\PYG{p}{]}\PYG{p}{,}
        \PYG{n}{axis}\PYG{o}{=}\PYG{l+m+mi}{1}
    \PYG{p}{)}
    \PYG{o}{.}\PYG{n}{stack}\PYG{p}{(}\PYG{l+s+s1}{\PYGZsq{}}\PYG{l+s+s1}{Beta Portfolio}\PYG{l+s+s1}{\PYGZsq{}}\PYG{p}{)}
\PYG{p}{)}

\PYG{n}{sns}\PYG{o}{.}\PYG{n}{regplot}\PYG{p}{(}\PYG{n}{x}\PYG{o}{=}\PYG{l+s+s1}{\PYGZsq{}}\PYG{l+s+s1}{Beta}\PYG{l+s+s1}{\PYGZsq{}}\PYG{p}{,} \PYG{n}{y}\PYG{o}{=}\PYG{l+s+s1}{\PYGZsq{}}\PYG{l+s+s1}{Return}\PYG{l+s+s1}{\PYGZsq{}}\PYG{p}{,} \PYG{n}{data}\PYG{o}{=}\PYG{n}{\PYGZus{}}\PYG{p}{)}
\PYG{n}{plt}\PYG{o}{.}\PYG{n}{ylabel}\PYG{p}{(}\PYG{l+s+s1}{\PYGZsq{}}\PYG{l+s+s1}{Mean Annual Return (}\PYG{l+s+s1}{\PYGZpc{}}\PYG{l+s+s1}{)}\PYG{l+s+s1}{\PYGZsq{}}\PYG{p}{)}
\PYG{n}{plt}\PYG{o}{.}\PYG{n}{xlabel}\PYG{p}{(}\PYG{l+s+sa}{r}\PYG{l+s+s1}{\PYGZsq{}}\PYG{l+s+s1}{Capital Asset Pricing Model (CAPM) \PYGZdl{}}\PYG{l+s+s1}{\PYGZbs{}}\PYG{l+s+s1}{beta\PYGZdl{}}\PYG{l+s+s1}{\PYGZsq{}}\PYG{p}{)}
\PYG{n}{plt}\PYG{o}{.}\PYG{n}{title}\PYG{p}{(}
    \PYG{l+s+sa}{r}\PYG{l+s+s1}{\PYGZsq{}}\PYG{l+s+s1}{Security Market Line (SML) for \PYGZdl{}}\PYG{l+s+s1}{\PYGZbs{}}\PYG{l+s+s1}{beta\PYGZdl{} Portfolios}\PYG{l+s+s1}{\PYGZsq{}} \PYG{o}{+}
    \PYG{l+s+sa}{f}\PYG{l+s+s1}{\PYGZsq{}}\PYG{l+s+se}{\PYGZbs{}n}\PYG{l+s+s1}{for Annual Returns from }\PYG{l+s+si}{\PYGZob{}}\PYG{n}{\PYGZus{}}\PYG{o}{.}\PYG{n}{index}\PYG{o}{.}\PYG{n}{get\PYGZus{}level\PYGZus{}values}\PYG{p}{(}\PYG{l+m+mi}{0}\PYG{p}{)}\PYG{o}{.}\PYG{n}{year}\PYG{p}{[}\PYG{l+m+mi}{0}\PYG{p}{]}\PYG{l+s+si}{\PYGZcb{}}\PYG{l+s+s1}{ to }\PYG{l+s+si}{\PYGZob{}}\PYG{n}{\PYGZus{}}\PYG{o}{.}\PYG{n}{index}\PYG{o}{.}\PYG{n}{get\PYGZus{}level\PYGZus{}values}\PYG{p}{(}\PYG{l+m+mi}{0}\PYG{p}{)}\PYG{o}{.}\PYG{n}{year}\PYG{p}{[}\PYG{o}{\PYGZhy{}}\PYG{l+m+mi}{1}\PYG{p}{]}\PYG{l+s+si}{\PYGZcb{}}\PYG{l+s+s1}{\PYGZsq{}}
\PYG{p}{)}
\PYG{n}{plt}\PYG{o}{.}\PYG{n}{show}\PYG{p}{(}\PYG{p}{)}
\end{sphinxVerbatim}

\end{sphinxuseclass}\end{sphinxVerbatimInput}
\begin{sphinxVerbatimOutput}

\begin{sphinxuseclass}{cell_output}
\noindent\sphinxincludegraphics{{53cce4d26126ae5a7a6b04889446920d598e05e8c0b14f8795b3323c0f9c98a9}.png}

\end{sphinxuseclass}\end{sphinxVerbatimOutput}

\end{sphinxuseclass}

\subsubsection{Estimate the CAPM \$\textbackslash{}beta\$s on several levered and inverse exchange traded funds (ETFs)}
\label{\detokenize{herron_03_practice_02:estimate-the-capm-beta-s-on-several-levered-and-inverse-exchange-traded-funds-etfs}}
\sphinxAtStartPar
Try the following ETFs:
\begin{enumerate}
\sphinxsetlistlabels{\arabic}{enumi}{enumii}{}{.}%
\item {} 
\sphinxAtStartPar
SPY

\item {} 
\sphinxAtStartPar
UPRO

\item {} 
\sphinxAtStartPar
SPXU

\end{enumerate}

\sphinxAtStartPar
Can you determine what these products do from the data alone?
Estimate \$\textbackslash{}beta\$s and plot cumulative returns.
You may want to pick short periods of time with large market swings.

\begin{sphinxuseclass}{cell}\begin{sphinxVerbatimInput}

\begin{sphinxuseclass}{cell_input}
\begin{sphinxVerbatim}[commandchars=\\\{\}]
\PYG{n}{etf\PYGZus{}2} \PYG{o}{=} \PYG{p}{(}
    \PYG{n}{yf}\PYG{o}{.}\PYG{n}{download}\PYG{p}{(}
        \PYG{n}{tickers}\PYG{o}{=}\PYG{l+s+s1}{\PYGZsq{}}\PYG{l+s+s1}{SPY UPRO SPXU}\PYG{l+s+s1}{\PYGZsq{}}\PYG{p}{,}
        \PYG{n}{progress}\PYG{o}{=}\PYG{k+kc}{False}
    \PYG{p}{)}
    \PYG{o}{.}\PYG{n}{assign}\PYG{p}{(}\PYG{n}{Date} \PYG{o}{=} \PYG{k}{lambda} \PYG{n}{x}\PYG{p}{:} \PYG{n}{x}\PYG{o}{.}\PYG{n}{index}\PYG{o}{.}\PYG{n}{tz\PYGZus{}localize}\PYG{p}{(}\PYG{k+kc}{None}\PYG{p}{)}\PYG{p}{)}
    \PYG{o}{.}\PYG{n}{set\PYGZus{}index}\PYG{p}{(}\PYG{l+s+s1}{\PYGZsq{}}\PYG{l+s+s1}{Date}\PYG{l+s+s1}{\PYGZsq{}}\PYG{p}{)}
    \PYG{o}{.}\PYG{n}{rename\PYGZus{}axis}\PYG{p}{(}\PYG{n}{columns}\PYG{o}{=}\PYG{p}{[}\PYG{l+s+s1}{\PYGZsq{}}\PYG{l+s+s1}{Variable}\PYG{l+s+s1}{\PYGZsq{}}\PYG{p}{,} \PYG{l+s+s1}{\PYGZsq{}}\PYG{l+s+s1}{Ticker}\PYG{l+s+s1}{\PYGZsq{}}\PYG{p}{]}\PYG{p}{)}
    \PYG{p}{[}\PYG{l+s+s1}{\PYGZsq{}}\PYG{l+s+s1}{Adj Close}\PYG{l+s+s1}{\PYGZsq{}}\PYG{p}{]}
    \PYG{o}{.}\PYG{n}{pct\PYGZus{}change}\PYG{p}{(}\PYG{p}{)}
    \PYG{o}{.}\PYG{n}{dropna}\PYG{p}{(}\PYG{p}{)}
\PYG{p}{)}
\end{sphinxVerbatim}

\end{sphinxuseclass}\end{sphinxVerbatimInput}

\end{sphinxuseclass}
\begin{sphinxuseclass}{cell}\begin{sphinxVerbatimInput}

\begin{sphinxuseclass}{cell_input}
\begin{sphinxVerbatim}[commandchars=\\\{\}]
\PYG{n}{etf\PYGZus{}2}\PYG{o}{.}\PYG{n}{apply}\PYG{p}{(}\PYG{n}{beta}\PYG{p}{)}\PYG{o}{.}\PYG{n}{rename}\PYG{p}{(}\PYG{l+s+sa}{r}\PYG{l+s+s1}{\PYGZsq{}}\PYG{l+s+s1}{\PYGZdl{}}\PYG{l+s+s1}{\PYGZbs{}}\PYG{l+s+s1}{beta\PYGZdl{}}\PYG{l+s+s1}{\PYGZsq{}}\PYG{p}{)}
\end{sphinxVerbatim}

\end{sphinxuseclass}\end{sphinxVerbatimInput}
\begin{sphinxVerbatimOutput}

\begin{sphinxuseclass}{cell_output}
\begin{sphinxVerbatim}[commandchars=\\\{\}]
Ticker
SPXU   \PYGZhy{}2.8660
SPY     0.9586
UPRO    2.8814
Name: \PYGZdl{}\PYGZbs{}beta\PYGZdl{}, dtype: float64
\end{sphinxVerbatim}

\end{sphinxuseclass}\end{sphinxVerbatimOutput}

\end{sphinxuseclass}
\begin{sphinxuseclass}{cell}\begin{sphinxVerbatimInput}

\begin{sphinxuseclass}{cell_input}
\begin{sphinxVerbatim}[commandchars=\\\{\}]
\PYG{n}{etf\PYGZus{}2}\PYG{o}{.}\PYG{n}{apply}\PYG{p}{(}\PYG{n}{beta}\PYG{p}{)}\PYG{o}{.}\PYG{n}{plot}\PYG{p}{(}\PYG{n}{kind}\PYG{o}{=}\PYG{l+s+s1}{\PYGZsq{}}\PYG{l+s+s1}{bar}\PYG{l+s+s1}{\PYGZsq{}}\PYG{p}{)}
\PYG{n}{plt}\PYG{o}{.}\PYG{n}{xticks}\PYG{p}{(}\PYG{n}{rotation}\PYG{o}{=}\PYG{l+m+mi}{0}\PYG{p}{)}
\PYG{n}{plt}\PYG{o}{.}\PYG{n}{ylabel}\PYG{p}{(}\PYG{l+s+sa}{r}\PYG{l+s+s1}{\PYGZsq{}}\PYG{l+s+s1}{Capital Asset Pricing Model (CAPM) \PYGZdl{}}\PYG{l+s+s1}{\PYGZbs{}}\PYG{l+s+s1}{beta\PYGZdl{}}\PYG{l+s+s1}{\PYGZsq{}}\PYG{p}{)}
\PYG{n}{plt}\PYG{o}{.}\PYG{n}{title}\PYG{p}{(}
    \PYG{l+s+sa}{r}\PYG{l+s+s1}{\PYGZsq{}}\PYG{l+s+s1}{Capital Asset Pricing Model (CAPM) \PYGZdl{}}\PYG{l+s+s1}{\PYGZbs{}}\PYG{l+s+s1}{beta\PYGZdl{}s}\PYG{l+s+s1}{\PYGZsq{}} \PYG{o}{+} 
    \PYG{l+s+sa}{f}\PYG{l+s+s1}{\PYGZsq{}}\PYG{l+s+se}{\PYGZbs{}n}\PYG{l+s+s1}{from Daily Returns from }\PYG{l+s+si}{\PYGZob{}}\PYG{n}{etf\PYGZus{}2}\PYG{o}{.}\PYG{n}{pipe}\PYG{p}{(}\PYG{n}{date\PYGZus{}range}\PYG{p}{)}\PYG{l+s+si}{\PYGZcb{}}\PYG{l+s+s1}{\PYGZsq{}}
\PYG{p}{)}
\PYG{n}{plt}\PYG{o}{.}\PYG{n}{show}\PYG{p}{(}\PYG{p}{)}
\end{sphinxVerbatim}

\end{sphinxuseclass}\end{sphinxVerbatimInput}
\begin{sphinxVerbatimOutput}

\begin{sphinxuseclass}{cell_output}
\noindent\sphinxincludegraphics{{e704518cf36499e9d04b29af9257d6cb57f3a78cdd8fbda4550483071f8896d4}.png}

\end{sphinxuseclass}\end{sphinxVerbatimOutput}

\end{sphinxuseclass}

\subsubsection{Explore the size factor}
\label{\detokenize{herron_03_practice_02:explore-the-size-factor}}

\paragraph{Estimate \$\textbackslash{}alpha\$s for the ten portfolios formed on size}
\label{\detokenize{herron_03_practice_02:estimate-alpha-s-for-the-ten-portfolios-formed-on-size}}
\sphinxAtStartPar
Academics started researching size\sphinxhyphen{}based portfolios in the early 1980s, so you may want to focus on the pre\sphinxhyphen{}1980 sample.

\begin{sphinxuseclass}{cell}\begin{sphinxVerbatimInput}

\begin{sphinxuseclass}{cell_input}
\begin{sphinxVerbatim}[commandchars=\\\{\}]
\PYG{n}{size\PYGZus{}0} \PYG{o}{=} \PYG{p}{(}
    \PYG{n}{pdr}\PYG{o}{.}\PYG{n}{DataReader}\PYG{p}{(}
        \PYG{n}{name}\PYG{o}{=}\PYG{l+s+s1}{\PYGZsq{}}\PYG{l+s+s1}{Portfolios\PYGZus{}Formed\PYGZus{}on\PYGZus{}ME}\PYG{l+s+s1}{\PYGZsq{}}\PYG{p}{,}
        \PYG{n}{data\PYGZus{}source}\PYG{o}{=}\PYG{l+s+s1}{\PYGZsq{}}\PYG{l+s+s1}{famafrench}\PYG{l+s+s1}{\PYGZsq{}}\PYG{p}{,}
        \PYG{n}{start}\PYG{o}{=}\PYG{l+s+s1}{\PYGZsq{}}\PYG{l+s+s1}{1900}\PYG{l+s+s1}{\PYGZsq{}}\PYG{p}{,}
        \PYG{n}{session}\PYG{o}{=}\PYG{n}{session}
    \PYG{p}{)}
\PYG{p}{)}

\PYG{n+nb}{print}\PYG{p}{(}\PYG{n}{size\PYGZus{}0}\PYG{p}{[}\PYG{l+s+s1}{\PYGZsq{}}\PYG{l+s+s1}{DESCR}\PYG{l+s+s1}{\PYGZsq{}}\PYG{p}{]}\PYG{p}{)}
\end{sphinxVerbatim}

\end{sphinxuseclass}\end{sphinxVerbatimInput}
\begin{sphinxVerbatimOutput}

\begin{sphinxuseclass}{cell_output}
\begin{sphinxVerbatim}[commandchars=\\\{\}]
Portfolios Formed on ME
\PYGZhy{}\PYGZhy{}\PYGZhy{}\PYGZhy{}\PYGZhy{}\PYGZhy{}\PYGZhy{}\PYGZhy{}\PYGZhy{}\PYGZhy{}\PYGZhy{}\PYGZhy{}\PYGZhy{}\PYGZhy{}\PYGZhy{}\PYGZhy{}\PYGZhy{}\PYGZhy{}\PYGZhy{}\PYGZhy{}\PYGZhy{}\PYGZhy{}\PYGZhy{}

This file was created by CMPT\PYGZus{}ME\PYGZus{}RETS using the 202301 CRSP database. It contains value\PYGZhy{} and equal\PYGZhy{}weighted returns for size portfolios. Each record contains returns for: Negative (not used) 30\PYGZpc{} 40\PYGZpc{} 30\PYGZpc{}   5 Quintiles  10 Deciles The portfolios are constructed at the end of Jun. The annual returns are from January to December. Missing data are indicated by \PYGZhy{}99.99 or \PYGZhy{}999. Copyright 2023 Kenneth R. French

  0 : Value Weight Returns \PYGZhy{}\PYGZhy{} Monthly (1159 rows x 19 cols)
  1 : Equal Weight Returns \PYGZhy{}\PYGZhy{} Monthly (1159 rows x 19 cols)
  2 : Value Weight Returns \PYGZhy{}\PYGZhy{} Annual from January to December (96 rows x 19 cols)
  3 : Equal Weight Returns \PYGZhy{}\PYGZhy{} Annual from January to December (96 rows x 19 cols)
  4 : Number of Firms in Portfolios (1159 rows x 19 cols)
  5 : Average Firm Size (1159 rows x 19 cols)
\end{sphinxVerbatim}

\end{sphinxuseclass}\end{sphinxVerbatimOutput}

\end{sphinxuseclass}
\begin{sphinxuseclass}{cell}\begin{sphinxVerbatimInput}

\begin{sphinxuseclass}{cell_input}
\begin{sphinxVerbatim}[commandchars=\\\{\}]
\PYG{n}{ff\PYGZus{}m} \PYG{o}{=} \PYG{p}{(}
    \PYG{n}{pdr}\PYG{o}{.}\PYG{n}{DataReader}\PYG{p}{(}
        \PYG{n}{name}\PYG{o}{=}\PYG{l+s+s1}{\PYGZsq{}}\PYG{l+s+s1}{F\PYGZhy{}F\PYGZus{}Research\PYGZus{}Data\PYGZus{}Factors}\PYG{l+s+s1}{\PYGZsq{}}\PYG{p}{,}
        \PYG{n}{data\PYGZus{}source}\PYG{o}{=}\PYG{l+s+s1}{\PYGZsq{}}\PYG{l+s+s1}{famafrench}\PYG{l+s+s1}{\PYGZsq{}}\PYG{p}{,}
        \PYG{n}{start}\PYG{o}{=}\PYG{l+s+s1}{\PYGZsq{}}\PYG{l+s+s1}{1900}\PYG{l+s+s1}{\PYGZsq{}}\PYG{p}{,}
        \PYG{n}{session}\PYG{o}{=}\PYG{n}{session}
    \PYG{p}{)}
\PYG{p}{)}

\PYG{n+nb}{print}\PYG{p}{(}\PYG{n}{ff\PYGZus{}m}\PYG{p}{[}\PYG{l+s+s1}{\PYGZsq{}}\PYG{l+s+s1}{DESCR}\PYG{l+s+s1}{\PYGZsq{}}\PYG{p}{]}\PYG{p}{)}
\end{sphinxVerbatim}

\end{sphinxuseclass}\end{sphinxVerbatimInput}
\begin{sphinxVerbatimOutput}

\begin{sphinxuseclass}{cell_output}
\begin{sphinxVerbatim}[commandchars=\\\{\}]
F\PYGZhy{}F Research Data Factors
\PYGZhy{}\PYGZhy{}\PYGZhy{}\PYGZhy{}\PYGZhy{}\PYGZhy{}\PYGZhy{}\PYGZhy{}\PYGZhy{}\PYGZhy{}\PYGZhy{}\PYGZhy{}\PYGZhy{}\PYGZhy{}\PYGZhy{}\PYGZhy{}\PYGZhy{}\PYGZhy{}\PYGZhy{}\PYGZhy{}\PYGZhy{}\PYGZhy{}\PYGZhy{}\PYGZhy{}\PYGZhy{}

This file was created by CMPT\PYGZus{}ME\PYGZus{}BEME\PYGZus{}RETS using the 202301 CRSP database. The 1\PYGZhy{}month TBill return is from Ibbotson and Associates, Inc. Copyright 2023 Kenneth R. French

  0 : (1159 rows x 4 cols)
  1 : Annual Factors: January\PYGZhy{}December (96 rows x 4 cols)
\end{sphinxVerbatim}

\end{sphinxuseclass}\end{sphinxVerbatimOutput}

\end{sphinxuseclass}
\begin{sphinxuseclass}{cell}\begin{sphinxVerbatimInput}

\begin{sphinxuseclass}{cell_input}
\begin{sphinxVerbatim}[commandchars=\\\{\}]
\PYG{n}{ports} \PYG{o}{=} \PYG{p}{[}\PYG{l+s+s1}{\PYGZsq{}}\PYG{l+s+s1}{Lo 10}\PYG{l+s+s1}{\PYGZsq{}}\PYG{p}{,} \PYG{l+s+s1}{\PYGZsq{}}\PYG{l+s+s1}{Dec 2}\PYG{l+s+s1}{\PYGZsq{}}\PYG{p}{,} \PYG{l+s+s1}{\PYGZsq{}}\PYG{l+s+s1}{Dec 3}\PYG{l+s+s1}{\PYGZsq{}}\PYG{p}{,} \PYG{l+s+s1}{\PYGZsq{}}\PYG{l+s+s1}{Dec 4}\PYG{l+s+s1}{\PYGZsq{}}\PYG{p}{,} \PYG{l+s+s1}{\PYGZsq{}}\PYG{l+s+s1}{Dec 5}\PYG{l+s+s1}{\PYGZsq{}}\PYG{p}{,} \PYG{l+s+s1}{\PYGZsq{}}\PYG{l+s+s1}{Dec 6}\PYG{l+s+s1}{\PYGZsq{}}\PYG{p}{,} \PYG{l+s+s1}{\PYGZsq{}}\PYG{l+s+s1}{Dec 7}\PYG{l+s+s1}{\PYGZsq{}}\PYG{p}{,} \PYG{l+s+s1}{\PYGZsq{}}\PYG{l+s+s1}{Dec 8}\PYG{l+s+s1}{\PYGZsq{}}\PYG{p}{,} \PYG{l+s+s1}{\PYGZsq{}}\PYG{l+s+s1}{Dec 9}\PYG{l+s+s1}{\PYGZsq{}}\PYG{p}{,} \PYG{l+s+s1}{\PYGZsq{}}\PYG{l+s+s1}{Hi 10}\PYG{l+s+s1}{\PYGZsq{}}\PYG{p}{]}
\PYG{n}{joined} \PYG{o}{=} \PYG{n}{size\PYGZus{}0}\PYG{p}{[}\PYG{l+m+mi}{1}\PYG{p}{]}\PYG{p}{[}\PYG{n}{ports}\PYG{p}{]}\PYG{o}{.}\PYG{n}{join}\PYG{p}{(}\PYG{n}{ff\PYGZus{}m}\PYG{p}{[}\PYG{l+m+mi}{0}\PYG{p}{]}\PYG{p}{)}
\PYG{n}{joined}\PYG{o}{.}\PYG{n}{head}\PYG{p}{(}\PYG{p}{)}
\end{sphinxVerbatim}

\end{sphinxuseclass}\end{sphinxVerbatimInput}
\begin{sphinxVerbatimOutput}

\begin{sphinxuseclass}{cell_output}
\begin{sphinxVerbatim}[commandchars=\\\{\}]
          Lo 10   Dec 2   Dec 3   Dec 4   Dec 5   Dec 6   Dec 7   Dec 8  \PYGZbs{}
Date                                                                      
1926\PYGZhy{}07 \PYGZhy{}1.4200  0.2900 \PYGZhy{}0.1500  0.8800  1.4500  1.8500  1.6300  1.3800   
1926\PYGZhy{}08  4.6100  2.5900  4.0300  3.2400  2.6600  4.6700  1.5400  1.6300   
1926\PYGZhy{}09  0.9100 \PYGZhy{}1.8700 \PYGZhy{}2.2700 \PYGZhy{}0.8400  0.1200 \PYGZhy{}0.0700 \PYGZhy{}1.5800  0.6400   
1926\PYGZhy{}10 \PYGZhy{}4.7200 \PYGZhy{}1.7700 \PYGZhy{}3.3600 \PYGZhy{}5.0100 \PYGZhy{}3.0900 \PYGZhy{}2.7100 \PYGZhy{}3.4500 \PYGZhy{}3.2700   
1926\PYGZhy{}11 \PYGZhy{}0.7700 \PYGZhy{}0.3200 \PYGZhy{}0.2900  4.7900  3.1700  3.5800  3.8000  2.9500   

          Dec 9   Hi 10  Mkt\PYGZhy{}RF     SMB     HML     RF  
Date                                                    
1926\PYGZhy{}07  3.3800  3.2900  2.9600 \PYGZhy{}2.5600 \PYGZhy{}2.4300 0.2200  
1926\PYGZhy{}08  0.9800  3.7000  2.6400 \PYGZhy{}1.1700  3.8200 0.2500  
1926\PYGZhy{}09 \PYGZhy{}0.8600  0.6700  0.3600 \PYGZhy{}1.4000  0.1300 0.2300  
1926\PYGZhy{}10 \PYGZhy{}3.4700 \PYGZhy{}2.4300 \PYGZhy{}3.2400 \PYGZhy{}0.0900  0.7000 0.3200  
1926\PYGZhy{}11  3.6100  2.7000  2.5300 \PYGZhy{}0.1000 \PYGZhy{}0.5100 0.3100  
\end{sphinxVerbatim}

\end{sphinxuseclass}\end{sphinxVerbatimOutput}

\end{sphinxuseclass}
\begin{sphinxuseclass}{cell}\begin{sphinxVerbatimInput}

\begin{sphinxuseclass}{cell_input}
\begin{sphinxVerbatim}[commandchars=\\\{\}]
\PYG{k+kn}{import} \PYG{n+nn}{statsmodels}\PYG{n+nn}{.}\PYG{n+nn}{formula}\PYG{n+nn}{.}\PYG{n+nn}{api} \PYG{k}{as} \PYG{n+nn}{smf}
\end{sphinxVerbatim}

\end{sphinxuseclass}\end{sphinxVerbatimInput}

\end{sphinxuseclass}
\begin{sphinxuseclass}{cell}\begin{sphinxVerbatimInput}

\begin{sphinxuseclass}{cell_input}
\begin{sphinxVerbatim}[commandchars=\\\{\}]
\PYG{n}{smf}\PYG{o}{.}\PYG{n}{ols}\PYG{p}{(}\PYG{l+s+s1}{\PYGZsq{}}\PYG{l+s+s1}{I(Q(}\PYG{l+s+s1}{\PYGZdq{}}\PYG{l+s+s1}{Lo 10}\PYG{l+s+s1}{\PYGZdq{}}\PYG{l+s+s1}{) \PYGZhy{} RF) \PYGZti{} Q(}\PYG{l+s+s1}{\PYGZdq{}}\PYG{l+s+s1}{Mkt\PYGZhy{}RF}\PYG{l+s+s1}{\PYGZdq{}}\PYG{l+s+s1}{)}\PYG{l+s+s1}{\PYGZsq{}}\PYG{p}{,} \PYG{n}{data}\PYG{o}{=}\PYG{n}{joined}\PYG{o}{.}\PYG{n}{loc}\PYG{p}{[}\PYG{p}{:}\PYG{l+s+s1}{\PYGZsq{}}\PYG{l+s+s1}{1979}\PYG{l+s+s1}{\PYGZsq{}}\PYG{p}{]}\PYG{p}{)}\PYG{o}{.}\PYG{n}{fit}\PYG{p}{(}\PYG{p}{)}\PYG{o}{.}\PYG{n}{summary}\PYG{p}{(}\PYG{p}{)}
\end{sphinxVerbatim}

\end{sphinxuseclass}\end{sphinxVerbatimInput}
\begin{sphinxVerbatimOutput}

\begin{sphinxuseclass}{cell_output}
\begin{sphinxVerbatim}[commandchars=\\\{\}]
\PYGZlt{}class \PYGZsq{}statsmodels.iolib.summary.Summary\PYGZsq{}\PYGZgt{}
\PYGZdq{}\PYGZdq{}\PYGZdq{}
                            OLS Regression Results                            
==============================================================================
Dep. Variable:     I(Q(\PYGZdq{}Lo 10\PYGZdq{}) \PYGZhy{} RF)   R\PYGZhy{}squared:                       0.539
Model:                            OLS   Adj. R\PYGZhy{}squared:                  0.539
Method:                 Least Squares   F\PYGZhy{}statistic:                     749.7
Date:                Wed, 22 Mar 2023   Prob (F\PYGZhy{}statistic):          7.47e\PYGZhy{}110
Time:                        18:33:07   Log\PYGZhy{}Likelihood:                \PYGZhy{}2305.1
No. Observations:                 642   AIC:                             4614.
Df Residuals:                     640   BIC:                             4623.
Df Model:                           1                                         
Covariance Type:            nonrobust                                         
===============================================================================
                  coef    std err          t      P\PYGZgt{}|t|      [0.025      0.975]
\PYGZhy{}\PYGZhy{}\PYGZhy{}\PYGZhy{}\PYGZhy{}\PYGZhy{}\PYGZhy{}\PYGZhy{}\PYGZhy{}\PYGZhy{}\PYGZhy{}\PYGZhy{}\PYGZhy{}\PYGZhy{}\PYGZhy{}\PYGZhy{}\PYGZhy{}\PYGZhy{}\PYGZhy{}\PYGZhy{}\PYGZhy{}\PYGZhy{}\PYGZhy{}\PYGZhy{}\PYGZhy{}\PYGZhy{}\PYGZhy{}\PYGZhy{}\PYGZhy{}\PYGZhy{}\PYGZhy{}\PYGZhy{}\PYGZhy{}\PYGZhy{}\PYGZhy{}\PYGZhy{}\PYGZhy{}\PYGZhy{}\PYGZhy{}\PYGZhy{}\PYGZhy{}\PYGZhy{}\PYGZhy{}\PYGZhy{}\PYGZhy{}\PYGZhy{}\PYGZhy{}\PYGZhy{}\PYGZhy{}\PYGZhy{}\PYGZhy{}\PYGZhy{}\PYGZhy{}\PYGZhy{}\PYGZhy{}\PYGZhy{}\PYGZhy{}\PYGZhy{}\PYGZhy{}\PYGZhy{}\PYGZhy{}\PYGZhy{}\PYGZhy{}\PYGZhy{}\PYGZhy{}\PYGZhy{}\PYGZhy{}\PYGZhy{}\PYGZhy{}\PYGZhy{}\PYGZhy{}\PYGZhy{}\PYGZhy{}\PYGZhy{}\PYGZhy{}\PYGZhy{}\PYGZhy{}\PYGZhy{}\PYGZhy{}
Intercept       0.9283      0.349      2.661      0.008       0.243       1.613
Q(\PYGZdq{}Mkt\PYGZhy{}RF\PYGZdq{})     1.6043      0.059     27.381      0.000       1.489       1.719
==============================================================================
Omnibus:                      611.504   Durbin\PYGZhy{}Watson:                   1.972
Prob(Omnibus):                  0.000   Jarque\PYGZhy{}Bera (JB):            27321.130
Skew:                           4.189   Prob(JB):                         0.00
Kurtosis:                      33.841   Cond. No.                         5.99
==============================================================================

Notes:
[1] Standard Errors assume that the covariance matrix of the errors is correctly specified.
\PYGZdq{}\PYGZdq{}\PYGZdq{}
\end{sphinxVerbatim}

\end{sphinxuseclass}\end{sphinxVerbatimOutput}

\end{sphinxuseclass}
\begin{sphinxuseclass}{cell}\begin{sphinxVerbatimInput}

\begin{sphinxuseclass}{cell_input}
\begin{sphinxVerbatim}[commandchars=\\\{\}]
\PYG{n}{models} \PYG{o}{=} \PYG{p}{[}\PYG{n}{smf}\PYG{o}{.}\PYG{n}{ols}\PYG{p}{(}\PYG{l+s+sa}{f}\PYG{l+s+s1}{\PYGZsq{}}\PYG{l+s+s1}{I(Q(}\PYG{l+s+s1}{\PYGZdq{}}\PYG{l+s+si}{\PYGZob{}}\PYG{n}{p}\PYG{l+s+si}{\PYGZcb{}}\PYG{l+s+s1}{\PYGZdq{}}\PYG{l+s+s1}{) \PYGZhy{} RF) \PYGZti{} Q(}\PYG{l+s+s1}{\PYGZdq{}}\PYG{l+s+s1}{Mkt\PYGZhy{}RF}\PYG{l+s+s1}{\PYGZdq{}}\PYG{l+s+s1}{)}\PYG{l+s+s1}{\PYGZsq{}}\PYG{p}{,} \PYG{n}{data}\PYG{o}{=}\PYG{n}{joined}\PYG{o}{.}\PYG{n}{loc}\PYG{p}{[}\PYG{p}{:}\PYG{l+s+s1}{\PYGZsq{}}\PYG{l+s+s1}{1979}\PYG{l+s+s1}{\PYGZsq{}}\PYG{p}{]}\PYG{p}{)} \PYG{k}{for} \PYG{n}{p} \PYG{o+ow}{in} \PYG{n}{ports}\PYG{p}{]}

\PYG{n}{fits} \PYG{o}{=} \PYG{p}{[}\PYG{n}{m}\PYG{o}{.}\PYG{n}{fit}\PYG{p}{(}\PYG{p}{)} \PYG{k}{for} \PYG{n}{m} \PYG{o+ow}{in} \PYG{n}{models}\PYG{p}{]}

\PYG{n}{coefs} \PYG{o}{=} \PYG{p}{(}
    \PYG{n}{pd}\PYG{o}{.}\PYG{n}{concat}\PYG{p}{(}
        \PYG{n}{objs}\PYG{o}{=}\PYG{p}{[}\PYG{n}{f}\PYG{o}{.}\PYG{n}{params} \PYG{k}{for} \PYG{n}{f} \PYG{o+ow}{in} \PYG{n}{fits}\PYG{p}{]}\PYG{p}{,} 
        \PYG{n}{axis}\PYG{o}{=}\PYG{l+m+mi}{1}\PYG{p}{,} 
        \PYG{n}{keys}\PYG{o}{=}\PYG{n}{ports}\PYG{p}{,} 
        \PYG{n}{names}\PYG{o}{=}\PYG{l+s+s1}{\PYGZsq{}}\PYG{l+s+s1}{Size Portfolio}\PYG{l+s+s1}{\PYGZsq{}}
    \PYG{p}{)}
    \PYG{o}{.}\PYG{n}{rename\PYGZus{}axis}\PYG{p}{(}\PYG{n}{index}\PYG{o}{=}\PYG{l+s+s1}{\PYGZsq{}}\PYG{l+s+s1}{Coefficient}\PYG{l+s+s1}{\PYGZsq{}}\PYG{p}{,} \PYG{n}{columns}\PYG{o}{=}\PYG{l+s+s1}{\PYGZsq{}}\PYG{l+s+s1}{Size Portfolio}\PYG{l+s+s1}{\PYGZsq{}}\PYG{p}{)}
    \PYG{o}{.}\PYG{n}{transpose}\PYG{p}{(}\PYG{p}{)}
\PYG{p}{)}
\end{sphinxVerbatim}

\end{sphinxuseclass}\end{sphinxVerbatimInput}

\end{sphinxuseclass}
\sphinxAtStartPar
We can get the standard errors, too.
The standard errors are in the \sphinxcode{\sphinxupquote{.params}} attribute of our model fits.

\begin{sphinxuseclass}{cell}\begin{sphinxVerbatimInput}

\begin{sphinxuseclass}{cell_input}
\begin{sphinxVerbatim}[commandchars=\\\{\}]
\PYG{n}{ses} \PYG{o}{=} \PYG{p}{[}\PYG{n}{f}\PYG{o}{.}\PYG{n}{bse}\PYG{p}{[}\PYG{l+m+mi}{0}\PYG{p}{]} \PYG{k}{for} \PYG{n}{f} \PYG{o+ow}{in} \PYG{n}{fits}\PYG{p}{]}
\end{sphinxVerbatim}

\end{sphinxuseclass}\end{sphinxVerbatimInput}

\end{sphinxuseclass}
\begin{sphinxuseclass}{cell}\begin{sphinxVerbatimInput}

\begin{sphinxuseclass}{cell_input}
\begin{sphinxVerbatim}[commandchars=\\\{\}]
\PYG{n}{coefs}\PYG{p}{[}\PYG{l+s+s1}{\PYGZsq{}}\PYG{l+s+s1}{Intercept}\PYG{l+s+s1}{\PYGZsq{}}\PYG{p}{]}\PYG{o}{.}\PYG{n}{plot}\PYG{p}{(}\PYG{n}{kind}\PYG{o}{=}\PYG{l+s+s1}{\PYGZsq{}}\PYG{l+s+s1}{bar}\PYG{l+s+s1}{\PYGZsq{}}\PYG{p}{,} \PYG{n}{yerr}\PYG{o}{=}\PYG{n}{ses}\PYG{p}{)}
\PYG{n}{plt}\PYG{o}{.}\PYG{n}{ylabel}\PYG{p}{(}\PYG{l+s+sa}{r}\PYG{l+s+s1}{\PYGZsq{}}\PYG{l+s+s1}{Monthly \PYGZdl{}}\PYG{l+s+s1}{\PYGZbs{}}\PYG{l+s+s1}{alpha\PYGZdl{} (}\PYG{l+s+s1}{\PYGZpc{}}\PYG{l+s+s1}{) from CAPM}\PYG{l+s+s1}{\PYGZsq{}}\PYG{p}{)}
\PYG{n}{plt}\PYG{o}{.}\PYG{n}{xticks}\PYG{p}{(}\PYG{n}{rotation}\PYG{o}{=}\PYG{l+m+mi}{0}\PYG{p}{)}
\PYG{n}{plt}\PYG{o}{.}\PYG{n}{title}\PYG{p}{(}
    \PYG{l+s+sa}{r}\PYG{l+s+s1}{\PYGZsq{}}\PYG{l+s+s1}{Size Portfolio CAPM \PYGZdl{}}\PYG{l+s+s1}{\PYGZbs{}}\PYG{l+s+s1}{alpha\PYGZdl{}s for Monthly Returns}\PYG{l+s+s1}{\PYGZsq{}} \PYG{o}{+}
    \PYG{l+s+s1}{\PYGZsq{}}\PYG{l+s+se}{\PYGZbs{}n}\PYG{l+s+s1}{from July 1926 through December 1979}\PYG{l+s+s1}{\PYGZsq{}}
\PYG{p}{)}
\PYG{n}{plt}\PYG{o}{.}\PYG{n}{show}\PYG{p}{(}\PYG{p}{)}
\end{sphinxVerbatim}

\end{sphinxuseclass}\end{sphinxVerbatimInput}
\begin{sphinxVerbatimOutput}

\begin{sphinxuseclass}{cell_output}
\noindent\sphinxincludegraphics{{589340b5d177c04f34cbd2ca1bb456a1d2dc60aa82c69e38a6d75ab46b5c3e06}.png}

\end{sphinxuseclass}\end{sphinxVerbatimOutput}

\end{sphinxuseclass}
\sphinxAtStartPar
The size effect (i.e., the CAPM \$\textbackslash{}alpha\$ for small stock portfolios) appears large!
We will dig a little deeper!


\paragraph{Are the returns on these ten portfolios formed on size concentrated in a specific month?}
\label{\detokenize{herron_03_practice_02:are-the-returns-on-these-ten-portfolios-formed-on-size-concentrated-in-a-specific-month}}
\begin{sphinxuseclass}{cell}\begin{sphinxVerbatimInput}

\begin{sphinxuseclass}{cell_input}
\begin{sphinxVerbatim}[commandchars=\\\{\}]
\PYG{p}{(}
    \PYG{n}{size\PYGZus{}0}\PYG{p}{[}\PYG{l+m+mi}{0}\PYG{p}{]}\PYG{p}{[}\PYG{n}{ports}\PYG{p}{]}
    \PYG{o}{.}\PYG{n}{groupby}\PYG{p}{(}\PYG{k}{lambda} \PYG{n}{x}\PYG{p}{:} \PYG{n}{np}\PYG{o}{.}\PYG{n}{where}\PYG{p}{(}\PYG{n}{x}\PYG{o}{.}\PYG{n}{month}\PYG{o}{==}\PYG{l+m+mi}{1}\PYG{p}{,} \PYG{l+s+s1}{\PYGZsq{}}\PYG{l+s+s1}{January}\PYG{l+s+s1}{\PYGZsq{}}\PYG{p}{,} \PYG{l+s+s1}{\PYGZsq{}}\PYG{l+s+s1}{Not January}\PYG{l+s+s1}{\PYGZsq{}}\PYG{p}{)}\PYG{p}{)}
    \PYG{o}{.}\PYG{n}{mean}\PYG{p}{(}\PYG{p}{)}
    \PYG{o}{.}\PYG{n}{rename\PYGZus{}axis}\PYG{p}{(}\PYG{n}{index}\PYG{o}{=}\PYG{l+s+s1}{\PYGZsq{}}\PYG{l+s+s1}{Month}\PYG{l+s+s1}{\PYGZsq{}}\PYG{p}{,} \PYG{n}{columns}\PYG{o}{=}\PYG{l+s+s1}{\PYGZsq{}}\PYG{l+s+s1}{Equal\PYGZhy{}Weighted Size Portfolio}\PYG{l+s+s1}{\PYGZsq{}}\PYG{p}{)}
    \PYG{o}{.}\PYG{n}{plot}\PYG{p}{(}\PYG{n}{kind}\PYG{o}{=}\PYG{l+s+s1}{\PYGZsq{}}\PYG{l+s+s1}{bar}\PYG{l+s+s1}{\PYGZsq{}}\PYG{p}{)}
\PYG{p}{)}

\PYG{n}{plt}\PYG{o}{.}\PYG{n}{xticks}\PYG{p}{(}\PYG{n}{rotation}\PYG{o}{=}\PYG{l+m+mi}{0}\PYG{p}{)}
\PYG{n}{plt}\PYG{o}{.}\PYG{n}{ylabel}\PYG{p}{(}\PYG{l+s+s1}{\PYGZsq{}}\PYG{l+s+s1}{Mean Monthly Return (}\PYG{l+s+s1}{\PYGZpc{}}\PYG{l+s+s1}{)}\PYG{l+s+s1}{\PYGZsq{}}\PYG{p}{)}
\PYG{n}{plt}\PYG{o}{.}\PYG{n}{title}\PYG{p}{(}\PYG{l+s+s1}{\PYGZsq{}}\PYG{l+s+s1}{When Do We Earn Size\PYGZhy{}Effect Returns?}\PYG{l+s+s1}{\PYGZsq{}}\PYG{p}{)}
\PYG{n}{plt}\PYG{o}{.}\PYG{n}{show}\PYG{p}{(}\PYG{p}{)}
\end{sphinxVerbatim}

\end{sphinxuseclass}\end{sphinxVerbatimInput}
\begin{sphinxVerbatimOutput}

\begin{sphinxuseclass}{cell_output}
\noindent\sphinxincludegraphics{{58f7dd3f2070232a24cd5991e72f272cbbc4fcd99b2f1ffac6a982c4837dd184}.png}

\end{sphinxuseclass}\end{sphinxVerbatimOutput}

\end{sphinxuseclass}
\sphinxAtStartPar
We earn size effect returns in January!
The size effect is likely due to tax\sphinxhyphen{}loss harvesting in small stocks.


\paragraph{Compare the size factor to the market factor}
\label{\detokenize{herron_03_practice_02:compare-the-size-factor-to-the-market-factor}}
\sphinxAtStartPar
You may want to consider mean excess returns by decade.

\sphinxAtStartPar
You may want to consider mean excess returns by decade.

\begin{sphinxuseclass}{cell}\begin{sphinxVerbatimInput}

\begin{sphinxuseclass}{cell_input}
\begin{sphinxVerbatim}[commandchars=\\\{\}]
\PYG{p}{(}
    \PYG{n}{ff\PYGZus{}m}\PYG{p}{[}\PYG{l+m+mi}{0}\PYG{p}{]}\PYG{p}{[}\PYG{p}{[}\PYG{l+s+s1}{\PYGZsq{}}\PYG{l+s+s1}{Mkt\PYGZhy{}RF}\PYG{l+s+s1}{\PYGZsq{}}\PYG{p}{,} \PYG{l+s+s1}{\PYGZsq{}}\PYG{l+s+s1}{SMB}\PYG{l+s+s1}{\PYGZsq{}}\PYG{p}{]}\PYG{p}{]}
    \PYG{o}{.}\PYG{n}{resample}\PYG{p}{(}\PYG{l+s+s1}{\PYGZsq{}}\PYG{l+s+s1}{10Y}\PYG{l+s+s1}{\PYGZsq{}}\PYG{p}{)}
    \PYG{o}{.}\PYG{n}{mean}\PYG{p}{(}\PYG{p}{)}
    \PYG{o}{.}\PYG{n}{mul}\PYG{p}{(}\PYG{l+m+mi}{12}\PYG{p}{)}
    \PYG{o}{.}\PYG{n}{rename\PYGZus{}axis}\PYG{p}{(}\PYG{n}{index}\PYG{o}{=}\PYG{l+s+s1}{\PYGZsq{}}\PYG{l+s+s1}{10\PYGZhy{}Year Period}\PYG{l+s+s1}{\PYGZsq{}}\PYG{p}{,} \PYG{n}{columns}\PYG{o}{=}\PYG{l+s+s1}{\PYGZsq{}}\PYG{l+s+s1}{Factor}\PYG{l+s+s1}{\PYGZsq{}}\PYG{p}{)}
    \PYG{o}{.}\PYG{n}{plot}\PYG{p}{(}\PYG{n}{kind}\PYG{o}{=}\PYG{l+s+s1}{\PYGZsq{}}\PYG{l+s+s1}{bar}\PYG{l+s+s1}{\PYGZsq{}}\PYG{p}{)}
\PYG{p}{)}

\PYG{n}{plt}\PYG{o}{.}\PYG{n}{xticks}\PYG{p}{(}\PYG{n}{rotation}\PYG{o}{=}\PYG{l+m+mi}{0}\PYG{p}{)}
\PYG{n}{plt}\PYG{o}{.}\PYG{n}{ylabel}\PYG{p}{(}\PYG{l+s+s1}{\PYGZsq{}}\PYG{l+s+s1}{Annualize Mean of Monthly Returns (}\PYG{l+s+s1}{\PYGZpc{}}\PYG{l+s+s1}{)}\PYG{l+s+s1}{\PYGZsq{}}\PYG{p}{)}
\PYG{n}{plt}\PYG{o}{.}\PYG{n}{title}\PYG{p}{(}\PYG{l+s+s1}{\PYGZsq{}}\PYG{l+s+s1}{Comparison on Market Risk and Small Stock Premia}\PYG{l+s+s1}{\PYGZsq{}}\PYG{p}{)}
\PYG{n}{plt}\PYG{o}{.}\PYG{n}{show}\PYG{p}{(}\PYG{p}{)}
\end{sphinxVerbatim}

\end{sphinxuseclass}\end{sphinxVerbatimInput}
\begin{sphinxVerbatimOutput}

\begin{sphinxuseclass}{cell_output}
\noindent\sphinxincludegraphics{{5f3edf7217ff43bbb287d6bd9935dc56919398a42681701ac07da0dcc08a4d8a}.png}

\end{sphinxuseclass}\end{sphinxVerbatimOutput}

\end{sphinxuseclass}
\sphinxAtStartPar
The size factor (SMB) may have \$\textbackslash{}alpha\$ early in the sample, but it rarely generates outsize returns.
Plus, the size factor has returned (effectively) zero the last two decades of the sample (2006\sphinxhyphen{}2015 and 2016\sphinxhyphen{}today).


\subsubsection{Repeat the exercises above with the value factor}
\label{\detokenize{herron_03_practice_02:repeat-the-exercises-above-with-the-value-factor}}
\begin{sphinxuseclass}{cell}\begin{sphinxVerbatimInput}

\begin{sphinxuseclass}{cell_input}
\begin{sphinxVerbatim}[commandchars=\\\{\}]
\PYG{n}{hml\PYGZus{}0} \PYG{o}{=} \PYG{n}{pdr}\PYG{o}{.}\PYG{n}{DataReader}\PYG{p}{(}
    \PYG{n}{name}\PYG{o}{=}\PYG{l+s+s1}{\PYGZsq{}}\PYG{l+s+s1}{Portfolios\PYGZus{}Formed\PYGZus{}on\PYGZus{}BE\PYGZhy{}ME}\PYG{l+s+s1}{\PYGZsq{}}\PYG{p}{,}
    \PYG{n}{data\PYGZus{}source}\PYG{o}{=}\PYG{l+s+s1}{\PYGZsq{}}\PYG{l+s+s1}{famafrench}\PYG{l+s+s1}{\PYGZsq{}}\PYG{p}{,}
    \PYG{n}{start}\PYG{o}{=}\PYG{l+s+s1}{\PYGZsq{}}\PYG{l+s+s1}{1900}\PYG{l+s+s1}{\PYGZsq{}}\PYG{p}{,}
    \PYG{n}{session}\PYG{o}{=}\PYG{n}{session}
\PYG{p}{)}

\PYG{n+nb}{print}\PYG{p}{(}\PYG{n}{hml\PYGZus{}0}\PYG{p}{[}\PYG{l+s+s1}{\PYGZsq{}}\PYG{l+s+s1}{DESCR}\PYG{l+s+s1}{\PYGZsq{}}\PYG{p}{]}\PYG{p}{)}
\end{sphinxVerbatim}

\end{sphinxuseclass}\end{sphinxVerbatimInput}
\begin{sphinxVerbatimOutput}

\begin{sphinxuseclass}{cell_output}
\begin{sphinxVerbatim}[commandchars=\\\{\}]
Portfolios Formed on BE\PYGZhy{}ME
\PYGZhy{}\PYGZhy{}\PYGZhy{}\PYGZhy{}\PYGZhy{}\PYGZhy{}\PYGZhy{}\PYGZhy{}\PYGZhy{}\PYGZhy{}\PYGZhy{}\PYGZhy{}\PYGZhy{}\PYGZhy{}\PYGZhy{}\PYGZhy{}\PYGZhy{}\PYGZhy{}\PYGZhy{}\PYGZhy{}\PYGZhy{}\PYGZhy{}\PYGZhy{}\PYGZhy{}\PYGZhy{}\PYGZhy{}

This file was created by CMPT\PYGZus{}BEME\PYGZus{}RETS using the 202301 CRSP database. It contains value\PYGZhy{} and equal\PYGZhy{}weighted returns for portfolios formed on BE/ME. The portfolios are constructed at the end of June. BE/ME is book equity at the last fiscal year end of the prior calendar year divided by ME at the end of December of the prior year. The annual returns are from January to December. Missing data are indicated by \PYGZhy{}99.99 or \PYGZhy{}999. The break points use Compustat firms plus the firms hand\PYGZhy{}collected from the Moodys Industrial, Utilities, Transportation, and Financial Manuals. The portfolios use Compustat firms plus the firms hand\PYGZhy{}collected from the Moodys Industrial, Utilities, Transportation, and Financial Manuals. The break points include utilities and include financials. The portfolios include utilities and include financials. Copyright 2023 Kenneth R. French

  0 : Value Weight Returns \PYGZhy{}\PYGZhy{} Monthly (1159 rows x 19 cols)
  1 : Equal Weight Returns \PYGZhy{}\PYGZhy{} Monthly (1159 rows x 19 cols)
  2 : Value Weight Returns \PYGZhy{}\PYGZhy{} Annual from January to December (96 rows x 19 cols)
  3 : Equal Weight Returns \PYGZhy{}\PYGZhy{} Annual from January to December (96 rows x 19 cols)
  4 : Number of Firms in Portfolios (1159 rows x 19 cols)
  5 : Average Firm Size (1159 rows x 19 cols)
  6 : Sum of BE / Sum of ME (97 rows x 19 cols)
  7 : Value Weight Average of BE / ME (97 rows x 19 cols)
\end{sphinxVerbatim}

\end{sphinxuseclass}\end{sphinxVerbatimOutput}

\end{sphinxuseclass}
\begin{sphinxuseclass}{cell}\begin{sphinxVerbatimInput}

\begin{sphinxuseclass}{cell_input}
\begin{sphinxVerbatim}[commandchars=\\\{\}]
\PYG{n}{ports} \PYG{o}{=} \PYG{p}{[}\PYG{l+s+s1}{\PYGZsq{}}\PYG{l+s+s1}{Lo 10}\PYG{l+s+s1}{\PYGZsq{}}\PYG{p}{,} \PYG{l+s+s1}{\PYGZsq{}}\PYG{l+s+s1}{Dec 2}\PYG{l+s+s1}{\PYGZsq{}}\PYG{p}{,} \PYG{l+s+s1}{\PYGZsq{}}\PYG{l+s+s1}{Dec 3}\PYG{l+s+s1}{\PYGZsq{}}\PYG{p}{,} \PYG{l+s+s1}{\PYGZsq{}}\PYG{l+s+s1}{Dec 4}\PYG{l+s+s1}{\PYGZsq{}}\PYG{p}{,} \PYG{l+s+s1}{\PYGZsq{}}\PYG{l+s+s1}{Dec 5}\PYG{l+s+s1}{\PYGZsq{}}\PYG{p}{,} \PYG{l+s+s1}{\PYGZsq{}}\PYG{l+s+s1}{Dec 6}\PYG{l+s+s1}{\PYGZsq{}}\PYG{p}{,} \PYG{l+s+s1}{\PYGZsq{}}\PYG{l+s+s1}{Dec 7}\PYG{l+s+s1}{\PYGZsq{}}\PYG{p}{,} \PYG{l+s+s1}{\PYGZsq{}}\PYG{l+s+s1}{Dec 8}\PYG{l+s+s1}{\PYGZsq{}}\PYG{p}{,} \PYG{l+s+s1}{\PYGZsq{}}\PYG{l+s+s1}{Dec 9}\PYG{l+s+s1}{\PYGZsq{}}\PYG{p}{,} \PYG{l+s+s1}{\PYGZsq{}}\PYG{l+s+s1}{Hi 10}\PYG{l+s+s1}{\PYGZsq{}}\PYG{p}{]}
\PYG{n}{joined} \PYG{o}{=} \PYG{n}{hml\PYGZus{}0}\PYG{p}{[}\PYG{l+m+mi}{1}\PYG{p}{]}\PYG{p}{[}\PYG{n}{ports}\PYG{p}{]}\PYG{o}{.}\PYG{n}{join}\PYG{p}{(}\PYG{n}{ff\PYGZus{}m}\PYG{p}{[}\PYG{l+m+mi}{0}\PYG{p}{]}\PYG{p}{)}
\PYG{n}{joined}\PYG{o}{.}\PYG{n}{head}\PYG{p}{(}\PYG{p}{)}
\end{sphinxVerbatim}

\end{sphinxuseclass}\end{sphinxVerbatimInput}
\begin{sphinxVerbatimOutput}

\begin{sphinxuseclass}{cell_output}
\begin{sphinxVerbatim}[commandchars=\\\{\}]
          Lo 10   Dec 2   Dec 3   Dec 4   Dec 5   Dec 6   Dec 7   Dec 8  \PYGZbs{}
Date                                                                      
1926\PYGZhy{}07  4.5200  1.5300  1.9900  2.0700  0.0700  1.9600  0.2300  2.8800   
1926\PYGZhy{}08  0.3400  0.5100  3.0000  0.8100  2.7900  1.6600  0.9600  4.4900   
1926\PYGZhy{}09  0.3600 \PYGZhy{}4.2500 \PYGZhy{}1.1600  0.8800 \PYGZhy{}0.7400 \PYGZhy{}1.1000 \PYGZhy{}0.5000 \PYGZhy{}1.4500   
1926\PYGZhy{}10 \PYGZhy{}4.1200 \PYGZhy{}3.8300 \PYGZhy{}2.6700 \PYGZhy{}2.7800 \PYGZhy{}2.3100 \PYGZhy{}3.9800 \PYGZhy{}4.1800 \PYGZhy{}2.0100   
1926\PYGZhy{}11  4.4400  0.4700  1.3500  4.4500  3.5600  1.4700  3.5100  3.5100   

          Dec 9   Hi 10  Mkt\PYGZhy{}RF     SMB     HML     RF  
Date                                                    
1926\PYGZhy{}07 \PYGZhy{}1.4400 \PYGZhy{}0.1700  2.9600 \PYGZhy{}2.5600 \PYGZhy{}2.4300 0.2200  
1926\PYGZhy{}08  5.8100  6.4800  2.6400 \PYGZhy{}1.1700  3.8200 0.2500  
1926\PYGZhy{}09 \PYGZhy{}2.1700  3.9300  0.3600 \PYGZhy{}1.4000  0.1300 0.2300  
1926\PYGZhy{}10 \PYGZhy{}4.5800 \PYGZhy{}1.8800 \PYGZhy{}3.2400 \PYGZhy{}0.0900  0.7000 0.3200  
1926\PYGZhy{}11  2.2300  2.6600  2.5300 \PYGZhy{}0.1000 \PYGZhy{}0.5100 0.3100  
\end{sphinxVerbatim}

\end{sphinxuseclass}\end{sphinxVerbatimOutput}

\end{sphinxuseclass}
\begin{sphinxuseclass}{cell}\begin{sphinxVerbatimInput}

\begin{sphinxuseclass}{cell_input}
\begin{sphinxVerbatim}[commandchars=\\\{\}]
\PYG{n}{models} \PYG{o}{=} \PYG{p}{[}\PYG{n}{smf}\PYG{o}{.}\PYG{n}{ols}\PYG{p}{(}\PYG{l+s+sa}{f}\PYG{l+s+s1}{\PYGZsq{}}\PYG{l+s+s1}{I(Q(}\PYG{l+s+s1}{\PYGZdq{}}\PYG{l+s+si}{\PYGZob{}}\PYG{n}{p}\PYG{l+s+si}{\PYGZcb{}}\PYG{l+s+s1}{\PYGZdq{}}\PYG{l+s+s1}{) \PYGZhy{} RF) \PYGZti{} Q(}\PYG{l+s+s1}{\PYGZdq{}}\PYG{l+s+s1}{Mkt\PYGZhy{}RF}\PYG{l+s+s1}{\PYGZdq{}}\PYG{l+s+s1}{)}\PYG{l+s+s1}{\PYGZsq{}}\PYG{p}{,} \PYG{n}{data}\PYG{o}{=}\PYG{n}{joined}\PYG{p}{)} \PYG{k}{for} \PYG{n}{p} \PYG{o+ow}{in} \PYG{n}{ports}\PYG{p}{]}

\PYG{n}{fits} \PYG{o}{=} \PYG{p}{[}\PYG{n}{m}\PYG{o}{.}\PYG{n}{fit}\PYG{p}{(}\PYG{p}{)} \PYG{k}{for} \PYG{n}{m} \PYG{o+ow}{in} \PYG{n}{models}\PYG{p}{]}

\PYG{n}{coefs} \PYG{o}{=} \PYG{p}{(}
    \PYG{n}{pd}\PYG{o}{.}\PYG{n}{concat}\PYG{p}{(}
        \PYG{n}{objs}\PYG{o}{=}\PYG{p}{[}\PYG{n}{f}\PYG{o}{.}\PYG{n}{params} \PYG{k}{for} \PYG{n}{f} \PYG{o+ow}{in} \PYG{n}{fits}\PYG{p}{]}\PYG{p}{,}
        \PYG{n}{axis}\PYG{o}{=}\PYG{l+m+mi}{1}\PYG{p}{,}
        \PYG{n}{keys}\PYG{o}{=}\PYG{n}{ports}
    \PYG{p}{)}
    \PYG{o}{.}\PYG{n}{rename\PYGZus{}axis}\PYG{p}{(}\PYG{n}{index}\PYG{o}{=}\PYG{l+s+s1}{\PYGZsq{}}\PYG{l+s+s1}{Coefficient}\PYG{l+s+s1}{\PYGZsq{}}\PYG{p}{,} \PYG{n}{columns}\PYG{o}{=}\PYG{l+s+s1}{\PYGZsq{}}\PYG{l+s+s1}{Equal\PYGZhy{}Weighted Value Portfolio}\PYG{l+s+s1}{\PYGZsq{}}\PYG{p}{)}
    \PYG{o}{.}\PYG{n}{transpose}\PYG{p}{(}\PYG{p}{)}
\PYG{p}{)}
\end{sphinxVerbatim}

\end{sphinxuseclass}\end{sphinxVerbatimInput}

\end{sphinxuseclass}
\sphinxAtStartPar
We can get the standard errors, too.
The standard errors are in the \sphinxcode{\sphinxupquote{.params}} attribute of our model fits.

\begin{sphinxuseclass}{cell}\begin{sphinxVerbatimInput}

\begin{sphinxuseclass}{cell_input}
\begin{sphinxVerbatim}[commandchars=\\\{\}]
\PYG{n}{ses} \PYG{o}{=} \PYG{p}{[}\PYG{n}{f}\PYG{o}{.}\PYG{n}{bse}\PYG{p}{[}\PYG{l+m+mi}{0}\PYG{p}{]} \PYG{k}{for} \PYG{n}{f} \PYG{o+ow}{in} \PYG{n}{fits}\PYG{p}{]}
\end{sphinxVerbatim}

\end{sphinxuseclass}\end{sphinxVerbatimInput}

\end{sphinxuseclass}
\begin{sphinxuseclass}{cell}\begin{sphinxVerbatimInput}

\begin{sphinxuseclass}{cell_input}
\begin{sphinxVerbatim}[commandchars=\\\{\}]
\PYG{n}{coefs}\PYG{p}{[}\PYG{l+s+s1}{\PYGZsq{}}\PYG{l+s+s1}{Intercept}\PYG{l+s+s1}{\PYGZsq{}}\PYG{p}{]}\PYG{o}{.}\PYG{n}{plot}\PYG{p}{(}\PYG{n}{kind}\PYG{o}{=}\PYG{l+s+s1}{\PYGZsq{}}\PYG{l+s+s1}{bar}\PYG{l+s+s1}{\PYGZsq{}}\PYG{p}{,} \PYG{n}{yerr}\PYG{o}{=}\PYG{n}{ses}\PYG{p}{)}
\PYG{n}{plt}\PYG{o}{.}\PYG{n}{ylabel}\PYG{p}{(}\PYG{l+s+sa}{r}\PYG{l+s+s1}{\PYGZsq{}}\PYG{l+s+s1}{Monthly \PYGZdl{}}\PYG{l+s+s1}{\PYGZbs{}}\PYG{l+s+s1}{alpha\PYGZdl{} (}\PYG{l+s+s1}{\PYGZpc{}}\PYG{l+s+s1}{) from CAPM}\PYG{l+s+s1}{\PYGZsq{}}\PYG{p}{)}
\PYG{c+c1}{\PYGZsh{} plt.xticks(rotation=0)}
\PYG{n}{plt}\PYG{o}{.}\PYG{n}{title}\PYG{p}{(}
    \PYG{l+s+sa}{r}\PYG{l+s+s1}{\PYGZsq{}}\PYG{l+s+s1}{Value Portfolio CAPM \PYGZdl{}}\PYG{l+s+s1}{\PYGZbs{}}\PYG{l+s+s1}{alpha\PYGZdl{}s for Monthly Returns}\PYG{l+s+s1}{\PYGZsq{}} \PYG{o}{+}
    \PYG{l+s+s1}{\PYGZsq{}}\PYG{l+s+se}{\PYGZbs{}n}\PYG{l+s+s1}{from July 1926 through january 1979}\PYG{l+s+s1}{\PYGZsq{}}
\PYG{p}{)}
\PYG{n}{plt}\PYG{o}{.}\PYG{n}{show}\PYG{p}{(}\PYG{p}{)}
\end{sphinxVerbatim}

\end{sphinxuseclass}\end{sphinxVerbatimInput}
\begin{sphinxVerbatimOutput}

\begin{sphinxuseclass}{cell_output}
\noindent\sphinxincludegraphics{{8fad74c42a93b4cea2c88078ade4b80a065ea5ec0510ecdb7206c2d785a22602}.png}

\end{sphinxuseclass}\end{sphinxVerbatimOutput}

\end{sphinxuseclass}

\subsubsection{Repeat the exercises above with the momentum factor}
\label{\detokenize{herron_03_practice_02:repeat-the-exercises-above-with-the-momentum-factor}}
\sphinxAtStartPar
You may find it helpful to consider the worst months and years for the momentum factor.

\begin{sphinxuseclass}{cell}\begin{sphinxVerbatimInput}

\begin{sphinxuseclass}{cell_input}
\begin{sphinxVerbatim}[commandchars=\\\{\}]
\PYG{n}{mom\PYGZus{}0} \PYG{o}{=} \PYG{n}{pdr}\PYG{o}{.}\PYG{n}{DataReader}\PYG{p}{(}
    \PYG{n}{name}\PYG{o}{=}\PYG{l+s+s1}{\PYGZsq{}}\PYG{l+s+s1}{10\PYGZus{}Portfolios\PYGZus{}Prior\PYGZus{}12\PYGZus{}2}\PYG{l+s+s1}{\PYGZsq{}}\PYG{p}{,}
    \PYG{n}{data\PYGZus{}source}\PYG{o}{=}\PYG{l+s+s1}{\PYGZsq{}}\PYG{l+s+s1}{famafrench}\PYG{l+s+s1}{\PYGZsq{}}\PYG{p}{,}
    \PYG{n}{start}\PYG{o}{=}\PYG{l+s+s1}{\PYGZsq{}}\PYG{l+s+s1}{1900}\PYG{l+s+s1}{\PYGZsq{}}\PYG{p}{,}
    \PYG{n}{session}\PYG{o}{=}\PYG{n}{session}
\PYG{p}{)}

\PYG{n+nb}{print}\PYG{p}{(}\PYG{n}{mom\PYGZus{}0}\PYG{p}{[}\PYG{l+s+s1}{\PYGZsq{}}\PYG{l+s+s1}{DESCR}\PYG{l+s+s1}{\PYGZsq{}}\PYG{p}{]}\PYG{p}{)}
\end{sphinxVerbatim}

\end{sphinxuseclass}\end{sphinxVerbatimInput}
\begin{sphinxVerbatimOutput}

\begin{sphinxuseclass}{cell_output}
\begin{sphinxVerbatim}[commandchars=\\\{\}]
10 Portfolios Prior 12 2
\PYGZhy{}\PYGZhy{}\PYGZhy{}\PYGZhy{}\PYGZhy{}\PYGZhy{}\PYGZhy{}\PYGZhy{}\PYGZhy{}\PYGZhy{}\PYGZhy{}\PYGZhy{}\PYGZhy{}\PYGZhy{}\PYGZhy{}\PYGZhy{}\PYGZhy{}\PYGZhy{}\PYGZhy{}\PYGZhy{}\PYGZhy{}\PYGZhy{}\PYGZhy{}\PYGZhy{}

This file was created by CMPT\PYGZus{}PRIOR\PYGZus{}RETS using the 202301 CRSP database. It contains value\PYGZhy{} and equal\PYGZhy{}weighted returns for 10 prior\PYGZhy{}return portfolios. The portfolios are constructed monthly. PRIOR\PYGZus{}RET is from \PYGZhy{}12 to \PYGZhy{} 2. The annual returns are from January to December. Missing data are indicated by \PYGZhy{}99.99 or \PYGZhy{}999.

  0 : Average Value Weighted Returns \PYGZhy{}\PYGZhy{} Monthly (1153 rows x 10 cols)
  1 : Average Equal Weighted Returns \PYGZhy{}\PYGZhy{} Monthly (1153 rows x 10 cols)
  2 : Average Value Weighted Returns \PYGZhy{}\PYGZhy{} Annual (96 rows x 10 cols)
  3 : Average Equal Weighted Returns \PYGZhy{}\PYGZhy{} Annual (96 rows x 10 cols)
  4 : Number of Firms in Portfolios (1153 rows x 10 cols)
  5 : Average Firm Size (1153 rows x 10 cols)
  6 : Value\PYGZhy{}Weighted Average of Prior Returns (96 rows x 10 cols)
\end{sphinxVerbatim}

\end{sphinxuseclass}\end{sphinxVerbatimOutput}

\end{sphinxuseclass}
\begin{sphinxuseclass}{cell}\begin{sphinxVerbatimInput}

\begin{sphinxuseclass}{cell_input}
\begin{sphinxVerbatim}[commandchars=\\\{\}]
\PYG{n}{ports} \PYG{o}{=} \PYG{p}{[}\PYG{l+s+s1}{\PYGZsq{}}\PYG{l+s+s1}{Lo PRIOR}\PYG{l+s+s1}{\PYGZsq{}}\PYG{p}{,} \PYG{l+s+s1}{\PYGZsq{}}\PYG{l+s+s1}{PRIOR 2}\PYG{l+s+s1}{\PYGZsq{}}\PYG{p}{,} \PYG{l+s+s1}{\PYGZsq{}}\PYG{l+s+s1}{PRIOR 3}\PYG{l+s+s1}{\PYGZsq{}}\PYG{p}{,} \PYG{l+s+s1}{\PYGZsq{}}\PYG{l+s+s1}{PRIOR 4}\PYG{l+s+s1}{\PYGZsq{}}\PYG{p}{,} \PYG{l+s+s1}{\PYGZsq{}}\PYG{l+s+s1}{PRIOR 5}\PYG{l+s+s1}{\PYGZsq{}}\PYG{p}{,} \PYG{l+s+s1}{\PYGZsq{}}\PYG{l+s+s1}{PRIOR 6}\PYG{l+s+s1}{\PYGZsq{}}\PYG{p}{,} \PYG{l+s+s1}{\PYGZsq{}}\PYG{l+s+s1}{PRIOR 7}\PYG{l+s+s1}{\PYGZsq{}}\PYG{p}{,} \PYG{l+s+s1}{\PYGZsq{}}\PYG{l+s+s1}{PRIOR 8}\PYG{l+s+s1}{\PYGZsq{}}\PYG{p}{,} \PYG{l+s+s1}{\PYGZsq{}}\PYG{l+s+s1}{PRIOR 9}\PYG{l+s+s1}{\PYGZsq{}}\PYG{p}{,} \PYG{l+s+s1}{\PYGZsq{}}\PYG{l+s+s1}{Hi PRIOR}\PYG{l+s+s1}{\PYGZsq{}}\PYG{p}{]}
\PYG{n}{joined} \PYG{o}{=} \PYG{n}{mom\PYGZus{}0}\PYG{p}{[}\PYG{l+m+mi}{1}\PYG{p}{]}\PYG{p}{[}\PYG{n}{ports}\PYG{p}{]}\PYG{o}{.}\PYG{n}{join}\PYG{p}{(}\PYG{n}{ff\PYGZus{}m}\PYG{p}{[}\PYG{l+m+mi}{0}\PYG{p}{]}\PYG{p}{)}
\PYG{n}{joined}\PYG{o}{.}\PYG{n}{head}\PYG{p}{(}\PYG{p}{)}
\end{sphinxVerbatim}

\end{sphinxuseclass}\end{sphinxVerbatimInput}
\begin{sphinxVerbatimOutput}

\begin{sphinxuseclass}{cell_output}
\begin{sphinxVerbatim}[commandchars=\\\{\}]
         Lo PRIOR  PRIOR 2  PRIOR 3  PRIOR 4  PRIOR 5  PRIOR 6  PRIOR 7  \PYGZbs{}
Date                                                                      
1927\PYGZhy{}01   \PYGZhy{}0.4700   1.0400   4.1700   2.4000  \PYGZhy{}0.3100   3.0800   1.4800   
1927\PYGZhy{}02    8.4900   3.8800   8.3300   7.5900   4.7900   5.4700   5.4000   
1927\PYGZhy{}03   \PYGZhy{}6.1500  \PYGZhy{}5.1100  \PYGZhy{}2.3800  \PYGZhy{}1.6000  \PYGZhy{}1.6300  \PYGZhy{}1.7000   0.4900   
1927\PYGZhy{}04    2.6100  \PYGZhy{}0.9300   0.1000  \PYGZhy{}1.2000   0.0900   0.4000   1.2700   
1927\PYGZhy{}05    1.1200   3.7300   6.0100   5.1700   7.5500   9.2700   6.6100   

         PRIOR 8  PRIOR 9  Hi PRIOR  Mkt\PYGZhy{}RF     SMB     HML     RF  
Date                                                                
1927\PYGZhy{}01   0.6400  \PYGZhy{}0.5900    1.1800 \PYGZhy{}0.0600 \PYGZhy{}0.3700  4.5400 0.2500  
1927\PYGZhy{}02   4.3000   5.7600    5.5800  4.1800  0.0400  2.9400 0.2600  
1927\PYGZhy{}03  \PYGZhy{}1.5300   0.3700   \PYGZhy{}0.7800  0.1300 \PYGZhy{}1.6500 \PYGZhy{}2.6100 0.3000  
1927\PYGZhy{}04  \PYGZhy{}0.5700   2.2200    4.4200  0.4600  0.3000  0.8100 0.2500  
1927\PYGZhy{}05   8.2100   6.9100   10.0000  5.4400  1.5300  4.7300 0.3000  
\end{sphinxVerbatim}

\end{sphinxuseclass}\end{sphinxVerbatimOutput}

\end{sphinxuseclass}
\begin{sphinxuseclass}{cell}\begin{sphinxVerbatimInput}

\begin{sphinxuseclass}{cell_input}
\begin{sphinxVerbatim}[commandchars=\\\{\}]
\PYG{n}{models} \PYG{o}{=} \PYG{p}{[}\PYG{n}{smf}\PYG{o}{.}\PYG{n}{ols}\PYG{p}{(}\PYG{l+s+sa}{f}\PYG{l+s+s1}{\PYGZsq{}}\PYG{l+s+s1}{I(Q(}\PYG{l+s+s1}{\PYGZdq{}}\PYG{l+s+si}{\PYGZob{}}\PYG{n}{p}\PYG{l+s+si}{\PYGZcb{}}\PYG{l+s+s1}{\PYGZdq{}}\PYG{l+s+s1}{) \PYGZhy{} RF) \PYGZti{} Q(}\PYG{l+s+s1}{\PYGZdq{}}\PYG{l+s+s1}{Mkt\PYGZhy{}RF}\PYG{l+s+s1}{\PYGZdq{}}\PYG{l+s+s1}{)}\PYG{l+s+s1}{\PYGZsq{}}\PYG{p}{,} \PYG{n}{data}\PYG{o}{=}\PYG{n}{joined}\PYG{p}{)} \PYG{k}{for} \PYG{n}{p} \PYG{o+ow}{in} \PYG{n}{ports}\PYG{p}{]}
\end{sphinxVerbatim}

\end{sphinxuseclass}\end{sphinxVerbatimInput}

\end{sphinxuseclass}
\begin{sphinxuseclass}{cell}\begin{sphinxVerbatimInput}

\begin{sphinxuseclass}{cell_input}
\begin{sphinxVerbatim}[commandchars=\\\{\}]
\PYG{n}{fits} \PYG{o}{=} \PYG{p}{[}\PYG{n}{m}\PYG{o}{.}\PYG{n}{fit}\PYG{p}{(}\PYG{p}{)} \PYG{k}{for} \PYG{n}{m} \PYG{o+ow}{in} \PYG{n}{models}\PYG{p}{]}
\end{sphinxVerbatim}

\end{sphinxuseclass}\end{sphinxVerbatimInput}

\end{sphinxuseclass}
\begin{sphinxuseclass}{cell}\begin{sphinxVerbatimInput}

\begin{sphinxuseclass}{cell_input}
\begin{sphinxVerbatim}[commandchars=\\\{\}]
\PYG{n}{coefs} \PYG{o}{=} \PYG{p}{(}
    \PYG{n}{pd}\PYG{o}{.}\PYG{n}{concat}\PYG{p}{(}
        \PYG{n}{objs}\PYG{o}{=}\PYG{p}{[}\PYG{n}{f}\PYG{o}{.}\PYG{n}{params} \PYG{k}{for} \PYG{n}{f} \PYG{o+ow}{in} \PYG{n}{fits}\PYG{p}{]}\PYG{p}{,}
        \PYG{n}{axis}\PYG{o}{=}\PYG{l+m+mi}{1}\PYG{p}{,}
        \PYG{n}{keys}\PYG{o}{=}\PYG{n}{ports}
    \PYG{p}{)}
    \PYG{o}{.}\PYG{n}{rename\PYGZus{}axis}\PYG{p}{(}\PYG{n}{index}\PYG{o}{=}\PYG{l+s+s1}{\PYGZsq{}}\PYG{l+s+s1}{Coefficient}\PYG{l+s+s1}{\PYGZsq{}}\PYG{p}{,} \PYG{n}{columns}\PYG{o}{=}\PYG{l+s+s1}{\PYGZsq{}}\PYG{l+s+s1}{Equal\PYGZhy{}Weighted Momentum Portfolio}\PYG{l+s+s1}{\PYGZsq{}}\PYG{p}{)}
    \PYG{o}{.}\PYG{n}{transpose}\PYG{p}{(}\PYG{p}{)}
\PYG{p}{)}
\end{sphinxVerbatim}

\end{sphinxuseclass}\end{sphinxVerbatimInput}

\end{sphinxuseclass}
\sphinxAtStartPar
We can get the standard errors, too.
The standard errors are in the \sphinxcode{\sphinxupquote{.params}} attribute of our model fits.

\begin{sphinxuseclass}{cell}\begin{sphinxVerbatimInput}

\begin{sphinxuseclass}{cell_input}
\begin{sphinxVerbatim}[commandchars=\\\{\}]
\PYG{n}{ses} \PYG{o}{=} \PYG{p}{[}\PYG{n}{f}\PYG{o}{.}\PYG{n}{bse}\PYG{p}{[}\PYG{l+m+mi}{0}\PYG{p}{]} \PYG{k}{for} \PYG{n}{f} \PYG{o+ow}{in} \PYG{n}{fits}\PYG{p}{]}
\end{sphinxVerbatim}

\end{sphinxuseclass}\end{sphinxVerbatimInput}

\end{sphinxuseclass}
\begin{sphinxuseclass}{cell}\begin{sphinxVerbatimInput}

\begin{sphinxuseclass}{cell_input}
\begin{sphinxVerbatim}[commandchars=\\\{\}]
\PYG{n}{coefs}\PYG{p}{[}\PYG{l+s+s1}{\PYGZsq{}}\PYG{l+s+s1}{Intercept}\PYG{l+s+s1}{\PYGZsq{}}\PYG{p}{]}\PYG{o}{.}\PYG{n}{plot}\PYG{p}{(}\PYG{n}{kind}\PYG{o}{=}\PYG{l+s+s1}{\PYGZsq{}}\PYG{l+s+s1}{bar}\PYG{l+s+s1}{\PYGZsq{}}\PYG{p}{,} \PYG{n}{yerr}\PYG{o}{=}\PYG{n}{ses}\PYG{p}{)}
\PYG{n}{plt}\PYG{o}{.}\PYG{n}{ylabel}\PYG{p}{(}\PYG{l+s+sa}{r}\PYG{l+s+s1}{\PYGZsq{}}\PYG{l+s+s1}{Monthly \PYGZdl{}}\PYG{l+s+s1}{\PYGZbs{}}\PYG{l+s+s1}{alpha\PYGZdl{} (}\PYG{l+s+s1}{\PYGZpc{}}\PYG{l+s+s1}{) from CAPM}\PYG{l+s+s1}{\PYGZsq{}}\PYG{p}{)}
\PYG{c+c1}{\PYGZsh{} plt.xticks(rotation=0)}
\PYG{n}{plt}\PYG{o}{.}\PYG{n}{title}\PYG{p}{(}
    \PYG{l+s+sa}{r}\PYG{l+s+s1}{\PYGZsq{}}\PYG{l+s+s1}{Momentum Portfolio CAPM \PYGZdl{}}\PYG{l+s+s1}{\PYGZbs{}}\PYG{l+s+s1}{alpha\PYGZdl{}s for Monthly Returns}\PYG{l+s+s1}{\PYGZsq{}} \PYG{o}{+}
    \PYG{l+s+s1}{\PYGZsq{}}\PYG{l+s+se}{\PYGZbs{}n}\PYG{l+s+s1}{from July 1926 through january 1979}\PYG{l+s+s1}{\PYGZsq{}}
\PYG{p}{)}
\PYG{n}{plt}\PYG{o}{.}\PYG{n}{show}\PYG{p}{(}\PYG{p}{)}
\end{sphinxVerbatim}

\end{sphinxuseclass}\end{sphinxVerbatimInput}
\begin{sphinxVerbatimOutput}

\begin{sphinxuseclass}{cell_output}
\noindent\sphinxincludegraphics{{561fb57ef0e2da2f6371ef7bcf748c3c03051c3f465b8cead6b1dd2eb3e19424}.png}

\end{sphinxuseclass}\end{sphinxVerbatimOutput}

\end{sphinxuseclass}

\subsubsection{Plot the coefficient estimates from a rolling Fama\sphinxhyphen{}French three\sphinxhyphen{}factor model for Berkshire Hathaway}
\label{\detokenize{herron_03_practice_02:plot-the-coefficient-estimates-from-a-rolling-fama-french-three-factor-model-for-berkshire-hathaway}}
\sphinxAtStartPar
Use a three\sphinxhyphen{}year window with daily returns.
How has Buffett’s \$\textbackslash{}alpha\$ and \$\textbackslash{}beta\$s changed over the past four decades?

\begin{sphinxuseclass}{cell}\begin{sphinxVerbatimInput}

\begin{sphinxuseclass}{cell_input}
\begin{sphinxVerbatim}[commandchars=\\\{\}]
\PYG{n}{brk} \PYG{o}{=} \PYG{p}{(}
    \PYG{n}{yf}\PYG{o}{.}\PYG{n}{download}\PYG{p}{(}\PYG{n}{tickers}\PYG{o}{=}\PYG{l+s+s1}{\PYGZsq{}}\PYG{l+s+s1}{BRK\PYGZhy{}A}\PYG{l+s+s1}{\PYGZsq{}}\PYG{p}{,} \PYG{n}{progress}\PYG{o}{=}\PYG{k+kc}{False}\PYG{p}{)}
    \PYG{o}{.}\PYG{n}{assign}\PYG{p}{(}\PYG{n}{Date} \PYG{o}{=} \PYG{k}{lambda} \PYG{n}{x}\PYG{p}{:} \PYG{n}{x}\PYG{o}{.}\PYG{n}{index}\PYG{o}{.}\PYG{n}{tz\PYGZus{}localize}\PYG{p}{(}\PYG{k+kc}{None}\PYG{p}{)}\PYG{p}{)}
    \PYG{o}{.}\PYG{n}{set\PYGZus{}index}\PYG{p}{(}\PYG{l+s+s1}{\PYGZsq{}}\PYG{l+s+s1}{Date}\PYG{l+s+s1}{\PYGZsq{}}\PYG{p}{)}
    \PYG{o}{.}\PYG{n}{rename\PYGZus{}axis}\PYG{p}{(}\PYG{n}{columns}\PYG{o}{=}\PYG{p}{[}\PYG{l+s+s1}{\PYGZsq{}}\PYG{l+s+s1}{Variable}\PYG{l+s+s1}{\PYGZsq{}}\PYG{p}{]}\PYG{p}{)}
    \PYG{o}{.}\PYG{n}{assign}\PYG{p}{(}\PYG{n}{R}\PYG{o}{=}\PYG{k}{lambda} \PYG{n}{x}\PYG{p}{:} \PYG{n}{x}\PYG{p}{[}\PYG{l+s+s1}{\PYGZsq{}}\PYG{l+s+s1}{Adj Close}\PYG{l+s+s1}{\PYGZsq{}}\PYG{p}{]}\PYG{o}{.}\PYG{n}{pct\PYGZus{}change}\PYG{p}{(}\PYG{p}{)}\PYG{o}{.}\PYG{n}{mul}\PYG{p}{(}\PYG{l+m+mi}{100}\PYG{p}{)}\PYG{p}{)}
    \PYG{o}{.}\PYG{n}{dropna}\PYG{p}{(}\PYG{p}{)}
    \PYG{o}{.}\PYG{n}{join}\PYG{p}{(}
        \PYG{n}{pdr}\PYG{o}{.}\PYG{n}{DataReader}\PYG{p}{(}
            \PYG{n}{name}\PYG{o}{=}\PYG{l+s+s1}{\PYGZsq{}}\PYG{l+s+s1}{F\PYGZhy{}F\PYGZus{}Research\PYGZus{}Data\PYGZus{}Factors\PYGZus{}daily}\PYG{l+s+s1}{\PYGZsq{}}\PYG{p}{,}
            \PYG{n}{data\PYGZus{}source}\PYG{o}{=}\PYG{l+s+s1}{\PYGZsq{}}\PYG{l+s+s1}{famafrench}\PYG{l+s+s1}{\PYGZsq{}}\PYG{p}{,}
            \PYG{n}{start}\PYG{o}{=}\PYG{l+s+s1}{\PYGZsq{}}\PYG{l+s+s1}{1900}\PYG{l+s+s1}{\PYGZsq{}}\PYG{p}{,}
            \PYG{n}{session}\PYG{o}{=}\PYG{n}{session}
        \PYG{p}{)}\PYG{p}{[}\PYG{l+m+mi}{0}\PYG{p}{]}\PYG{p}{,}
        \PYG{n}{how}\PYG{o}{=}\PYG{l+s+s1}{\PYGZsq{}}\PYG{l+s+s1}{inner}\PYG{l+s+s1}{\PYGZsq{}}
    \PYG{p}{)}
\PYG{p}{)}
\end{sphinxVerbatim}

\end{sphinxuseclass}\end{sphinxVerbatimInput}

\end{sphinxuseclass}
\begin{sphinxuseclass}{cell}\begin{sphinxVerbatimInput}

\begin{sphinxuseclass}{cell_input}
\begin{sphinxVerbatim}[commandchars=\\\{\}]
\PYG{k+kn}{from} \PYG{n+nn}{statsmodels}\PYG{n+nn}{.}\PYG{n+nn}{regression}\PYG{n+nn}{.}\PYG{n+nn}{rolling} \PYG{k+kn}{import} \PYG{n}{RollingOLS}
\end{sphinxVerbatim}

\end{sphinxuseclass}\end{sphinxVerbatimInput}

\end{sphinxuseclass}
\begin{sphinxuseclass}{cell}\begin{sphinxVerbatimInput}

\begin{sphinxuseclass}{cell_input}
\begin{sphinxVerbatim}[commandchars=\\\{\}]
\PYG{n}{coefs} \PYG{o}{=} \PYG{p}{(}
    \PYG{n}{RollingOLS}\PYG{o}{.}\PYG{n}{from\PYGZus{}formula}\PYG{p}{(}
        \PYG{n}{formula}\PYG{o}{=}\PYG{l+s+s1}{\PYGZsq{}}\PYG{l+s+s1}{I(R\PYGZhy{}RF) \PYGZti{} Q(}\PYG{l+s+s1}{\PYGZdq{}}\PYG{l+s+s1}{Mkt\PYGZhy{}RF}\PYG{l+s+s1}{\PYGZdq{}}\PYG{l+s+s1}{) + SMB + HML}\PYG{l+s+s1}{\PYGZsq{}}\PYG{p}{,}
        \PYG{n}{data}\PYG{o}{=}\PYG{n}{brk}\PYG{p}{,}
        \PYG{n}{window}\PYG{o}{=}\PYG{l+m+mi}{3}\PYG{o}{*}\PYG{l+m+mi}{252}
    \PYG{p}{)}
    \PYG{o}{.}\PYG{n}{fit}\PYG{p}{(}\PYG{p}{)}
    \PYG{o}{.}\PYG{n}{params}
    \PYG{o}{.}\PYG{n}{rename\PYGZus{}axis}\PYG{p}{(}\PYG{n}{columns}\PYG{o}{=}\PYG{l+s+s1}{\PYGZsq{}}\PYG{l+s+s1}{Coefficient}\PYG{l+s+s1}{\PYGZsq{}}\PYG{p}{)}
    \PYG{o}{.}\PYG{n}{rename}\PYG{p}{(}\PYG{n}{columns}\PYG{o}{=}\PYG{p}{\PYGZob{}}\PYG{l+s+s1}{\PYGZsq{}}\PYG{l+s+s1}{Q(}\PYG{l+s+s1}{\PYGZdq{}}\PYG{l+s+s1}{Mkt\PYGZhy{}RF}\PYG{l+s+s1}{\PYGZdq{}}\PYG{l+s+s1}{)}\PYG{l+s+s1}{\PYGZsq{}}\PYG{p}{:} \PYG{l+s+s1}{\PYGZsq{}}\PYG{l+s+s1}{Mkt\PYGZhy{}RF}\PYG{l+s+s1}{\PYGZsq{}}\PYG{p}{\PYGZcb{}}\PYG{p}{)}
\PYG{p}{)}
\end{sphinxVerbatim}

\end{sphinxuseclass}\end{sphinxVerbatimInput}

\end{sphinxuseclass}
\begin{sphinxuseclass}{cell}\begin{sphinxVerbatimInput}

\begin{sphinxuseclass}{cell_input}
\begin{sphinxVerbatim}[commandchars=\\\{\}]
\PYG{n}{fig}\PYG{p}{,} \PYG{n}{ax} \PYG{o}{=} \PYG{n}{plt}\PYG{o}{.}\PYG{n}{subplots}\PYG{p}{(}\PYG{l+m+mi}{2}\PYG{p}{,} \PYG{l+m+mi}{1}\PYG{p}{,} \PYG{n}{sharex}\PYG{o}{=}\PYG{k+kc}{True}\PYG{p}{)}
\PYG{n}{coefs}\PYG{p}{[}\PYG{l+s+s1}{\PYGZsq{}}\PYG{l+s+s1}{Intercept}\PYG{l+s+s1}{\PYGZsq{}}\PYG{p}{]}\PYG{o}{.}\PYG{n}{plot}\PYG{p}{(}\PYG{n}{ax}\PYG{o}{=}\PYG{n}{ax}\PYG{p}{[}\PYG{l+m+mi}{0}\PYG{p}{]}\PYG{p}{,} \PYG{n}{legend}\PYG{o}{=}\PYG{k+kc}{True}\PYG{p}{)}
\PYG{n}{coefs}\PYG{o}{.}\PYG{n}{drop}\PYG{p}{(}\PYG{l+s+s1}{\PYGZsq{}}\PYG{l+s+s1}{Intercept}\PYG{l+s+s1}{\PYGZsq{}}\PYG{p}{,} \PYG{n}{axis}\PYG{o}{=}\PYG{l+m+mi}{1}\PYG{p}{)}\PYG{o}{.}\PYG{n}{plot}\PYG{p}{(}\PYG{n}{ax}\PYG{o}{=}\PYG{n}{ax}\PYG{p}{[}\PYG{l+m+mi}{1}\PYG{p}{]}\PYG{p}{)}
\PYG{n}{plt}\PYG{o}{.}\PYG{n}{suptitle}\PYG{p}{(}
    \PYG{l+s+s1}{\PYGZsq{}}\PYG{l+s+s1}{Rolling Three\PYGZhy{}Factor Regressions}\PYG{l+s+s1}{\PYGZsq{}} \PYG{o}{+}
    \PYG{l+s+s1}{\PYGZsq{}}\PYG{l+s+se}{\PYGZbs{}n}\PYG{l+s+s1}{Three\PYGZhy{}Year Windows with Daily Returns in Percent}\PYG{l+s+s1}{\PYGZsq{}}
\PYG{p}{)}
\PYG{n}{plt}\PYG{o}{.}\PYG{n}{show}\PYG{p}{(}\PYG{p}{)}
\end{sphinxVerbatim}

\end{sphinxuseclass}\end{sphinxVerbatimInput}
\begin{sphinxVerbatimOutput}

\begin{sphinxuseclass}{cell_output}
\noindent\sphinxincludegraphics{{f3f7c57e9982c23fc26048b9555cbf9edabb9a944fc6dc1da29461abc007d976}.png}

\end{sphinxuseclass}\end{sphinxVerbatimOutput}

\end{sphinxuseclass}
\sphinxAtStartPar
Buffett’s \$\textbackslash{}alpha\$ was large, but has declined to zero.
Also, his loading on SMB (size factor) has gone from positive to negative, indicating that he has moved from small stocks to large stocks as Berkshire Hathaway has grown.


\subsubsection{Use the three\sphinxhyphen{}, four\sphinxhyphen{}, and five\sphinxhyphen{}factor models to determine how the ARKK Innovation ETF generates returns}
\label{\detokenize{herron_03_practice_02:use-the-three-four-and-five-factor-models-to-determine-how-the-arkk-innovation-etf-generates-returns}}
\sphinxstepscope


\chapter{Herron Topic 4 \sphinxhyphen{} Portfolio Optimization}
\label{\detokenize{herron_04_lecture:herron-topic-4-portfolio-optimization}}\label{\detokenize{herron_04_lecture::doc}}
\sphinxAtStartPar
This notebook covers portfolio optimization.
I have not found a perfect reference that combines portfolio optimization and Python, but here are two references that I find useful:
\begin{enumerate}
\sphinxsetlistlabels{\arabic}{enumi}{enumii}{}{.}%
\item {} 
\sphinxAtStartPar
Ivo Welch discusses the mathematics and finance of portfolio optimization in \sphinxhref{https://book.ivo-welch.info/bookg.pdf\#chapter.12}{Chapter 12 of his draft textbook on investments}.

\item {} 
\sphinxAtStartPar
Eryk Lewinson provides Python code for portfolio optimization in chapter 7 of his \sphinxhref{https://onesearch.library.northeastern.edu/permalink/01NEU\_INST/i2gqis/alma9952082522901401}{\sphinxstyleemphasis{Python for Finance Cookbook}}, but he uses several packages that are either non\sphinxhyphen{}free or abandoned.

\end{enumerate}

\sphinxAtStartPar
In this notebook, we will:
\begin{enumerate}
\sphinxsetlistlabels{\arabic}{enumi}{enumii}{}{.}%
\item {} 
\sphinxAtStartPar
Review the \$\textbackslash{}frac\{1\}\{n\}\$ portfolio (or equal\sphinxhyphen{}weighted portfolio) from {\hyperref[\detokenize{herron_01_lecture::doc}]{\sphinxcrossref{\DUrole{doc,std,std-doc}{Herron Topic 1}}}}

\item {} 
\sphinxAtStartPar
Use SciPy’s \sphinxcode{\sphinxupquote{minimize()}} function to:
\begin{enumerate}
\sphinxsetlistlabels{\arabic}{enumii}{enumiii}{}{.}%
\item {} 
\sphinxAtStartPar
Find the minimum variance portfolio

\item {} 
\sphinxAtStartPar
Find the (mean\sphinxhyphen{}variance) efficient frontier

\end{enumerate}

\end{enumerate}

\sphinxAtStartPar
In the practice notebook, we will use SciPy’s \sphinxcode{\sphinxupquote{minimize()}} function to achieve any objective.

\begin{sphinxuseclass}{cell}\begin{sphinxVerbatimInput}

\begin{sphinxuseclass}{cell_input}
\begin{sphinxVerbatim}[commandchars=\\\{\}]
\PYG{k+kn}{import} \PYG{n+nn}{matplotlib}\PYG{n+nn}{.}\PYG{n+nn}{pyplot} \PYG{k}{as} \PYG{n+nn}{plt}
\PYG{k+kn}{import} \PYG{n+nn}{numpy} \PYG{k}{as} \PYG{n+nn}{np}
\PYG{k+kn}{import} \PYG{n+nn}{pandas} \PYG{k}{as} \PYG{n+nn}{pd}
\end{sphinxVerbatim}

\end{sphinxuseclass}\end{sphinxVerbatimInput}

\end{sphinxuseclass}
\begin{sphinxuseclass}{cell}\begin{sphinxVerbatimInput}

\begin{sphinxuseclass}{cell_input}
\begin{sphinxVerbatim}[commandchars=\\\{\}]
\PYG{o}{\PYGZpc{}}\PYG{k}{config} InlineBackend.figure\PYGZus{}format = \PYGZsq{}retina\PYGZsq{}
\PYG{o}{\PYGZpc{}}\PYG{k}{precision} 4
\PYG{n}{pd}\PYG{o}{.}\PYG{n}{options}\PYG{o}{.}\PYG{n}{display}\PYG{o}{.}\PYG{n}{float\PYGZus{}format} \PYG{o}{=} \PYG{l+s+s1}{\PYGZsq{}}\PYG{l+s+si}{\PYGZob{}:.4f\PYGZcb{}}\PYG{l+s+s1}{\PYGZsq{}}\PYG{o}{.}\PYG{n}{format}
\end{sphinxVerbatim}

\end{sphinxuseclass}\end{sphinxVerbatimInput}

\end{sphinxuseclass}
\begin{sphinxuseclass}{cell}\begin{sphinxVerbatimInput}

\begin{sphinxuseclass}{cell_input}
\begin{sphinxVerbatim}[commandchars=\\\{\}]
\PYG{k+kn}{import} \PYG{n+nn}{yfinance} \PYG{k}{as} \PYG{n+nn}{yf}
\PYG{k+kn}{import} \PYG{n+nn}{pandas\PYGZus{}datareader} \PYG{k}{as} \PYG{n+nn}{pdr}
\PYG{k+kn}{import} \PYG{n+nn}{requests\PYGZus{}cache}
\PYG{n}{session} \PYG{o}{=} \PYG{n}{requests\PYGZus{}cache}\PYG{o}{.}\PYG{n}{CachedSession}\PYG{p}{(}\PYG{p}{)}
\end{sphinxVerbatim}

\end{sphinxuseclass}\end{sphinxVerbatimInput}

\end{sphinxuseclass}

\section{The \$\textbackslash{}frac\{1\}\{n\}\$ Portfolio}
\label{\detokenize{herron_04_lecture:the-frac-1-n-portfolio}}
\sphinxAtStartPar
We first saw the \$\textbackslash{}frac\{1\}\{n\}\$ portfolio (or equal\sphinxhyphen{}weighted portfolio) in {\hyperref[\detokenize{herron_01_lecture::doc}]{\sphinxcrossref{\DUrole{doc,std,std-doc}{Herron Topic 1}}}}.
In the \$\textbackslash{}frac\{1\}\{n\}\$ portfolio, each of \$n\$ assets receives an equal portfolio weight \$w\_i = \textbackslash{}frac\{1\}\{n\}\$.
While the \$\textbackslash{}frac\{1\}\{n\}\$ strategy seems too simple to be useful, DeMiguel, Garlappi, and Uppal (2007) show that it is difficult to beat \$\textbackslash{}frac\{1\}\{n\}\$ strategy, even with more advanced strategies.

\begin{sphinxuseclass}{cell}\begin{sphinxVerbatimInput}

\begin{sphinxuseclass}{cell_input}
\begin{sphinxVerbatim}[commandchars=\\\{\}]
\PYG{n}{tickers} \PYG{o}{=} \PYG{l+s+s1}{\PYGZsq{}}\PYG{l+s+s1}{MSFT AAPL TSLA AMZN NVDA GOOG}\PYG{l+s+s1}{\PYGZsq{}}

\PYG{n}{matana} \PYG{o}{=} \PYG{p}{(}
    \PYG{n}{yf}\PYG{o}{.}\PYG{n}{download}\PYG{p}{(}\PYG{n}{tickers}\PYG{o}{=}\PYG{n}{tickers}\PYG{p}{,} \PYG{n}{progress}\PYG{o}{=}\PYG{k+kc}{False}\PYG{p}{)}
    \PYG{o}{.}\PYG{n}{assign}\PYG{p}{(}\PYG{n}{Date}\PYG{o}{=}\PYG{k}{lambda} \PYG{n}{x}\PYG{p}{:} \PYG{n}{x}\PYG{o}{.}\PYG{n}{index}\PYG{o}{.}\PYG{n}{tz\PYGZus{}localize}\PYG{p}{(}\PYG{k+kc}{None}\PYG{p}{)}\PYG{p}{)}
    \PYG{o}{.}\PYG{n}{set\PYGZus{}index}\PYG{p}{(}\PYG{l+s+s1}{\PYGZsq{}}\PYG{l+s+s1}{Date}\PYG{l+s+s1}{\PYGZsq{}}\PYG{p}{)}
    \PYG{o}{.}\PYG{n}{rename\PYGZus{}axis}\PYG{p}{(}\PYG{n}{columns}\PYG{o}{=}\PYG{p}{[}\PYG{l+s+s1}{\PYGZsq{}}\PYG{l+s+s1}{Variable}\PYG{l+s+s1}{\PYGZsq{}}\PYG{p}{,} \PYG{l+s+s1}{\PYGZsq{}}\PYG{l+s+s1}{Ticker}\PYG{l+s+s1}{\PYGZsq{}}\PYG{p}{]}\PYG{p}{)}
\PYG{p}{)}

\PYG{n}{matana}\PYG{o}{.}\PYG{n}{tail}\PYG{p}{(}\PYG{p}{)}
\end{sphinxVerbatim}

\end{sphinxuseclass}\end{sphinxVerbatimInput}
\begin{sphinxVerbatimOutput}

\begin{sphinxuseclass}{cell_output}
\begin{sphinxVerbatim}[commandchars=\\\{\}]
Variable   Adj Close                                                 Close  \PYGZbs{}
Ticker          AAPL     AMZN     GOOG     MSFT     NVDA     TSLA     AAPL   
Date                                                                         
2023\PYGZhy{}03\PYGZhy{}20  157.4000  97.7100 101.9300 272.2300 259.0000 183.2500 157.4000   
2023\PYGZhy{}03\PYGZhy{}21  159.2800 100.6100 105.8400 273.7800 261.9900 197.5800 159.2800   
2023\PYGZhy{}03\PYGZhy{}22  157.8300  98.7000 104.2200 272.2900 264.6800 191.1500 157.8300   
2023\PYGZhy{}03\PYGZhy{}23  158.9300  98.7100 106.2600 277.6600 271.9100 192.2200 158.9300   
2023\PYGZhy{}03\PYGZhy{}24  158.9300  97.4100 105.6050 278.5400 266.2550 189.4000 158.9300   

Variable                               ...     Open                    \PYGZbs{}
Ticker         AMZN     GOOG     MSFT  ...     GOOG     MSFT     NVDA   
Date                                   ...                              
2023\PYGZhy{}03\PYGZhy{}20  97.7100 101.9300 272.2300  ... 101.0600 276.9800 256.1500   
2023\PYGZhy{}03\PYGZhy{}21 100.6100 105.8400 273.7800  ... 101.9800 274.8800 261.8000   
2023\PYGZhy{}03\PYGZhy{}22  98.7000 104.2200 272.2900  ... 105.1400 273.4000 264.2500   
2023\PYGZhy{}03\PYGZhy{}23  98.7100 106.2600 277.6600  ... 105.8900 277.9400 271.1500   
2023\PYGZhy{}03\PYGZhy{}24  97.4100 105.6050 278.5400  ... 105.7400 277.2400 270.3100   

Variable               Volume                                            \PYGZbs{}
Ticker         TSLA      AAPL          AMZN          GOOG          MSFT   
Date                                                                      
2023\PYGZhy{}03\PYGZhy{}20 178.0800  73641400 62388900.0000 26033900.0000 43466600.0000   
2023\PYGZhy{}03\PYGZhy{}21 188.2800  73938300 58597300.0000 33122800.0000 34558700.0000   
2023\PYGZhy{}03\PYGZhy{}22 199.3000  75701800 57475400.0000 32336900.0000 34873300.0000   
2023\PYGZhy{}03\PYGZhy{}23 195.2600  67572500 57491000.0000 31369800.0000 36590300.0000   
2023\PYGZhy{}03\PYGZhy{}24 191.6500  26073408 26008374.0000 12005487.0000 12500555.0000   

Variable                                 
Ticker              NVDA           TSLA  
Date                                     
2023\PYGZhy{}03\PYGZhy{}20 43274700.0000 129684400.0000  
2023\PYGZhy{}03\PYGZhy{}21 54740800.0000 153391400.0000  
2023\PYGZhy{}03\PYGZhy{}22 79729500.0000 150376400.0000  
2023\PYGZhy{}03\PYGZhy{}23 56377300.0000 143865300.0000  
2023\PYGZhy{}03\PYGZhy{}24 24673779.0000  63328036.0000  

[5 rows x 36 columns]
\end{sphinxVerbatim}

\end{sphinxuseclass}\end{sphinxVerbatimOutput}

\end{sphinxuseclass}
\begin{sphinxuseclass}{cell}\begin{sphinxVerbatimInput}

\begin{sphinxuseclass}{cell_input}
\begin{sphinxVerbatim}[commandchars=\\\{\}]
\PYG{n}{returns} \PYG{o}{=} \PYG{n}{matana}\PYG{p}{[}\PYG{l+s+s1}{\PYGZsq{}}\PYG{l+s+s1}{Adj Close}\PYG{l+s+s1}{\PYGZsq{}}\PYG{p}{]}\PYG{o}{.}\PYG{n}{pct\PYGZus{}change}\PYG{p}{(}\PYG{p}{)}\PYG{o}{.}\PYG{n}{iloc}\PYG{p}{[}\PYG{p}{(}\PYG{o}{\PYGZhy{}}\PYG{l+m+mi}{3} \PYG{o}{*} \PYG{l+m+mi}{252}\PYG{p}{)}\PYG{p}{:}\PYG{p}{]}

\PYG{n}{returns}\PYG{o}{.}\PYG{n}{describe}\PYG{p}{(}\PYG{p}{)}
\end{sphinxVerbatim}

\end{sphinxuseclass}\end{sphinxVerbatimInput}
\begin{sphinxVerbatimOutput}

\begin{sphinxuseclass}{cell_output}
\begin{sphinxVerbatim}[commandchars=\\\{\}]
Ticker     AAPL     AMZN     GOOG     MSFT     NVDA     TSLA
count  756.0000 756.0000 756.0000 756.0000 756.0000 756.0000
mean     0.0015   0.0003   0.0010   0.0011   0.0025   0.0032
std      0.0206   0.0243   0.0207   0.0191   0.0334   0.0421
min     \PYGZhy{}0.0801  \PYGZhy{}0.1405  \PYGZhy{}0.0963  \PYGZhy{}0.0772  \PYGZhy{}0.0947  \PYGZhy{}0.2106
25\PYGZpc{}     \PYGZhy{}0.0101  \PYGZhy{}0.0132  \PYGZhy{}0.0097  \PYGZhy{}0.0096  \PYGZhy{}0.0172  \PYGZhy{}0.0208
50\PYGZpc{}      0.0012   0.0005   0.0015   0.0007   0.0032   0.0021
75\PYGZpc{}      0.0141   0.0127   0.0117   0.0120   0.0230   0.0251
max      0.1047   0.1354   0.0874   0.0823   0.1433   0.1964
\end{sphinxVerbatim}

\end{sphinxuseclass}\end{sphinxVerbatimOutput}

\end{sphinxuseclass}
\sphinxAtStartPar
Before we revisit the advanced techniques from {\hyperref[\detokenize{herron_01_lecture::doc}]{\sphinxcrossref{\DUrole{doc,std,std-doc}{Herron Topic 1}}}}, we can calculate \$\textbackslash{}frac\{1\}\{n\}\$ portfolio returns manually, where \$R\_P = \textbackslash{}frac\{\textbackslash{}sum\_\{i\}\textasciicircum{}\{n\} R\_i\}\{n\}\$
Since our weights are constant (i.e., do not change over time), we rebalance our portfolio every return period.
If we have daily data, rebalance daily.
If we have monthly data, we rebalance monthly, and so on.

\begin{sphinxuseclass}{cell}\begin{sphinxVerbatimInput}

\begin{sphinxuseclass}{cell_input}
\begin{sphinxVerbatim}[commandchars=\\\{\}]
\PYG{n}{n} \PYG{o}{=} \PYG{n}{returns}\PYG{o}{.}\PYG{n}{shape}\PYG{p}{[}\PYG{l+m+mi}{1}\PYG{p}{]}
\PYG{n}{p1} \PYG{o}{=} \PYG{n}{returns}\PYG{o}{.}\PYG{n}{sum}\PYG{p}{(}\PYG{n}{axis}\PYG{o}{=}\PYG{l+m+mi}{1}\PYG{p}{)}\PYG{o}{.}\PYG{n}{div}\PYG{p}{(}\PYG{n}{n}\PYG{p}{)}

\PYG{n}{p1}\PYG{o}{.}\PYG{n}{describe}\PYG{p}{(}\PYG{p}{)}
\end{sphinxVerbatim}

\end{sphinxuseclass}\end{sphinxVerbatimInput}
\begin{sphinxVerbatimOutput}

\begin{sphinxuseclass}{cell_output}
\begin{sphinxVerbatim}[commandchars=\\\{\}]
count   756.0000
mean      0.0016
std       0.0218
min      \PYGZhy{}0.0782
25\PYGZpc{}      \PYGZhy{}0.0113
50\PYGZpc{}       0.0021
75\PYGZpc{}       0.0149
max       0.0980
dtype: float64
\end{sphinxVerbatim}

\end{sphinxuseclass}\end{sphinxVerbatimOutput}

\end{sphinxuseclass}
\sphinxAtStartPar
Recall from {\hyperref[\detokenize{herron_01_lecture::doc}]{\sphinxcrossref{\DUrole{doc,std,std-doc}{Herron Topic 1}}}} we have two better options:
\begin{enumerate}
\sphinxsetlistlabels{\arabic}{enumi}{enumii}{}{.}%
\item {} 
\sphinxAtStartPar
The \sphinxcode{\sphinxupquote{.mean(axis=1)}} method for the \$\textbackslash{}frac\{1\}\{n\}\$ portfolio

\item {} 
\sphinxAtStartPar
The \sphinxcode{\sphinxupquote{.dot(weights)}} method where \sphinxcode{\sphinxupquote{weights}} is a pandas series or NumPy array of portfolio weights, allowing different weights for each asset

\end{enumerate}

\begin{sphinxuseclass}{cell}\begin{sphinxVerbatimInput}

\begin{sphinxuseclass}{cell_input}
\begin{sphinxVerbatim}[commandchars=\\\{\}]
\PYG{n}{p2} \PYG{o}{=} \PYG{n}{returns}\PYG{o}{.}\PYG{n}{mean}\PYG{p}{(}\PYG{n}{axis}\PYG{o}{=}\PYG{l+m+mi}{1}\PYG{p}{)}

\PYG{n}{p2}\PYG{o}{.}\PYG{n}{describe}\PYG{p}{(}\PYG{p}{)}
\end{sphinxVerbatim}

\end{sphinxuseclass}\end{sphinxVerbatimInput}
\begin{sphinxVerbatimOutput}

\begin{sphinxuseclass}{cell_output}
\begin{sphinxVerbatim}[commandchars=\\\{\}]
count   756.0000
mean      0.0016
std       0.0218
min      \PYGZhy{}0.0782
25\PYGZpc{}      \PYGZhy{}0.0113
50\PYGZpc{}       0.0021
75\PYGZpc{}       0.0149
max       0.0980
dtype: float64
\end{sphinxVerbatim}

\end{sphinxuseclass}\end{sphinxVerbatimOutput}

\end{sphinxuseclass}
\begin{sphinxuseclass}{cell}\begin{sphinxVerbatimInput}

\begin{sphinxuseclass}{cell_input}
\begin{sphinxVerbatim}[commandchars=\\\{\}]
\PYG{n}{weights} \PYG{o}{=} \PYG{n}{np}\PYG{o}{.}\PYG{n}{ones}\PYG{p}{(}\PYG{n}{n}\PYG{p}{)} \PYG{o}{/} \PYG{n}{n}
\PYG{n}{p3} \PYG{o}{=} \PYG{n}{returns}\PYG{o}{.}\PYG{n}{dot}\PYG{p}{(}\PYG{n}{weights}\PYG{p}{)}

\PYG{n}{p3}\PYG{o}{.}\PYG{n}{describe}\PYG{p}{(}\PYG{p}{)}
\end{sphinxVerbatim}

\end{sphinxuseclass}\end{sphinxVerbatimInput}
\begin{sphinxVerbatimOutput}

\begin{sphinxuseclass}{cell_output}
\begin{sphinxVerbatim}[commandchars=\\\{\}]
count   756.0000
mean      0.0016
std       0.0218
min      \PYGZhy{}0.0782
25\PYGZpc{}      \PYGZhy{}0.0113
50\PYGZpc{}       0.0021
75\PYGZpc{}       0.0149
max       0.0980
dtype: float64
\end{sphinxVerbatim}

\end{sphinxuseclass}\end{sphinxVerbatimOutput}

\end{sphinxuseclass}
\sphinxAtStartPar
The \sphinxcode{\sphinxupquote{.describe()}} method provides summary statistics for data, letting us make quick comparisons.
However, we should use \sphinxcode{\sphinxupquote{np.allclose()}} if we want to be sure that \sphinxcode{\sphinxupquote{p1}}, \sphinxcode{\sphinxupquote{p2}}, and \sphinxcode{\sphinxupquote{p3}} are similar.

\begin{sphinxuseclass}{cell}\begin{sphinxVerbatimInput}

\begin{sphinxuseclass}{cell_input}
\begin{sphinxVerbatim}[commandchars=\\\{\}]
\PYG{n}{np}\PYG{o}{.}\PYG{n}{allclose}\PYG{p}{(}\PYG{n}{p1}\PYG{p}{,} \PYG{n}{p2}\PYG{p}{)}
\end{sphinxVerbatim}

\end{sphinxuseclass}\end{sphinxVerbatimInput}
\begin{sphinxVerbatimOutput}

\begin{sphinxuseclass}{cell_output}
\begin{sphinxVerbatim}[commandchars=\\\{\}]
True
\end{sphinxVerbatim}

\end{sphinxuseclass}\end{sphinxVerbatimOutput}

\end{sphinxuseclass}
\begin{sphinxuseclass}{cell}\begin{sphinxVerbatimInput}

\begin{sphinxuseclass}{cell_input}
\begin{sphinxVerbatim}[commandchars=\\\{\}]
\PYG{n}{np}\PYG{o}{.}\PYG{n}{allclose}\PYG{p}{(}\PYG{n}{p1}\PYG{p}{,} \PYG{n}{p3}\PYG{p}{)}
\end{sphinxVerbatim}

\end{sphinxuseclass}\end{sphinxVerbatimInput}
\begin{sphinxVerbatimOutput}

\begin{sphinxuseclass}{cell_output}
\begin{sphinxVerbatim}[commandchars=\\\{\}]
True
\end{sphinxVerbatim}

\end{sphinxuseclass}\end{sphinxVerbatimOutput}

\end{sphinxuseclass}

\bigskip\hrule\bigskip


\sphinxAtStartPar
Here is a simple example to help understand the \sphinxcode{\sphinxupquote{.dot()}} method.

\begin{sphinxuseclass}{cell}\begin{sphinxVerbatimInput}

\begin{sphinxuseclass}{cell_input}
\begin{sphinxVerbatim}[commandchars=\\\{\}]
\PYG{n}{silly\PYGZus{}n} \PYG{o}{=} \PYG{l+m+mi}{3}
\PYG{n}{silly\PYGZus{}R} \PYG{o}{=} \PYG{n}{pd}\PYG{o}{.}\PYG{n}{DataFrame}\PYG{p}{(}\PYG{n}{np}\PYG{o}{.}\PYG{n}{arange}\PYG{p}{(}\PYG{l+m+mi}{2}\PYG{o}{*}\PYG{n}{silly\PYGZus{}n}\PYG{p}{)}\PYG{o}{.}\PYG{n}{reshape}\PYG{p}{(}\PYG{l+m+mi}{2}\PYG{p}{,} \PYG{n}{silly\PYGZus{}n}\PYG{p}{)}\PYG{p}{)}
\PYG{n}{silly\PYGZus{}w} \PYG{o}{=} \PYG{n}{np}\PYG{o}{.}\PYG{n}{ones}\PYG{p}{(}\PYG{l+m+mi}{3}\PYG{p}{)} \PYG{o}{/} \PYG{l+m+mi}{3}
\end{sphinxVerbatim}

\end{sphinxuseclass}\end{sphinxVerbatimInput}

\end{sphinxuseclass}
\begin{sphinxuseclass}{cell}\begin{sphinxVerbatimInput}

\begin{sphinxuseclass}{cell_input}
\begin{sphinxVerbatim}[commandchars=\\\{\}]
\PYG{n+nb}{print}\PYG{p}{(}
    \PYG{l+s+sa}{f}\PYG{l+s+s1}{\PYGZsq{}}\PYG{l+s+s1}{silly\PYGZus{}n:}\PYG{l+s+se}{\PYGZbs{}n}\PYG{l+s+si}{\PYGZob{}}\PYG{n}{silly\PYGZus{}n}\PYG{l+s+si}{\PYGZcb{}}\PYG{l+s+s1}{\PYGZsq{}}\PYG{p}{,}
    \PYG{l+s+sa}{f}\PYG{l+s+s1}{\PYGZsq{}}\PYG{l+s+s1}{silly\PYGZus{}R:}\PYG{l+s+se}{\PYGZbs{}n}\PYG{l+s+si}{\PYGZob{}}\PYG{n}{silly\PYGZus{}R}\PYG{l+s+si}{\PYGZcb{}}\PYG{l+s+s1}{\PYGZsq{}}\PYG{p}{,}
    \PYG{l+s+sa}{f}\PYG{l+s+s1}{\PYGZsq{}}\PYG{l+s+s1}{silly\PYGZus{}w:}\PYG{l+s+se}{\PYGZbs{}n}\PYG{l+s+si}{\PYGZob{}}\PYG{n}{silly\PYGZus{}w}\PYG{l+s+si}{\PYGZcb{}}\PYG{l+s+s1}{\PYGZsq{}}\PYG{p}{,}
    \PYG{n}{sep}\PYG{o}{=}\PYG{l+s+s1}{\PYGZsq{}}\PYG{l+s+se}{\PYGZbs{}n}\PYG{l+s+se}{\PYGZbs{}n}\PYG{l+s+s1}{\PYGZsq{}}
\PYG{p}{)}
\end{sphinxVerbatim}

\end{sphinxuseclass}\end{sphinxVerbatimInput}
\begin{sphinxVerbatimOutput}

\begin{sphinxuseclass}{cell_output}
\begin{sphinxVerbatim}[commandchars=\\\{\}]
silly\PYGZus{}n:
3

silly\PYGZus{}R:
   0  1  2
0  0  1  2
1  3  4  5

silly\PYGZus{}w:
[0.3333 0.3333 0.3333]
\end{sphinxVerbatim}

\end{sphinxuseclass}\end{sphinxVerbatimOutput}

\end{sphinxuseclass}
\begin{sphinxuseclass}{cell}\begin{sphinxVerbatimInput}

\begin{sphinxuseclass}{cell_input}
\begin{sphinxVerbatim}[commandchars=\\\{\}]
\PYG{n}{silly\PYGZus{}R}\PYG{o}{.}\PYG{n}{dot}\PYG{p}{(}\PYG{n}{silly\PYGZus{}w}\PYG{p}{)}
\end{sphinxVerbatim}

\end{sphinxuseclass}\end{sphinxVerbatimInput}
\begin{sphinxVerbatimOutput}

\begin{sphinxuseclass}{cell_output}
\begin{sphinxVerbatim}[commandchars=\\\{\}]
0   1.0000
1   4.0000
dtype: float64
\end{sphinxVerbatim}

\end{sphinxuseclass}\end{sphinxVerbatimOutput}

\end{sphinxuseclass}
\sphinxAtStartPar
Under the hood, Python and the \sphinxcode{\sphinxupquote{.dot()}} method (effectively) do the following calculation:

\begin{sphinxuseclass}{cell}\begin{sphinxVerbatimInput}

\begin{sphinxuseclass}{cell_input}
\begin{sphinxVerbatim}[commandchars=\\\{\}]
\PYG{k}{for} \PYG{n}{i}\PYG{p}{,} \PYG{n}{row} \PYG{o+ow}{in} \PYG{n}{silly\PYGZus{}R}\PYG{o}{.}\PYG{n}{iterrows}\PYG{p}{(}\PYG{p}{)}\PYG{p}{:}
    \PYG{n+nb}{print}\PYG{p}{(}
        \PYG{l+s+sa}{f}\PYG{l+s+s1}{\PYGZsq{}}\PYG{l+s+s1}{Row }\PYG{l+s+si}{\PYGZob{}}\PYG{n}{i}\PYG{l+s+si}{\PYGZcb{}}\PYG{l+s+s1}{: }\PYG{l+s+s1}{\PYGZsq{}}\PYG{p}{,}
        \PYG{l+s+s1}{\PYGZsq{}}\PYG{l+s+s1}{ + }\PYG{l+s+s1}{\PYGZsq{}}\PYG{o}{.}\PYG{n}{join}\PYG{p}{(}\PYG{p}{[}\PYG{l+s+sa}{f}\PYG{l+s+s1}{\PYGZsq{}}\PYG{l+s+si}{\PYGZob{}}\PYG{n}{w}\PYG{l+s+si}{:}\PYG{l+s+s1}{0.2f}\PYG{l+s+si}{\PYGZcb{}}\PYG{l+s+s1}{ * }\PYG{l+s+si}{\PYGZob{}}\PYG{n}{y}\PYG{l+s+si}{\PYGZcb{}}\PYG{l+s+s1}{\PYGZsq{}} \PYG{k}{for} \PYG{n}{w}\PYG{p}{,} \PYG{n}{y} \PYG{o+ow}{in} \PYG{n+nb}{zip}\PYG{p}{(}\PYG{n}{silly\PYGZus{}w}\PYG{p}{,} \PYG{n}{row}\PYG{p}{)}\PYG{p}{]}\PYG{p}{)}\PYG{p}{,}
        \PYG{l+s+s1}{\PYGZsq{}}\PYG{l+s+s1}{ = }\PYG{l+s+s1}{\PYGZsq{}}\PYG{p}{,}
        \PYG{l+s+sa}{f}\PYG{l+s+s1}{\PYGZsq{}}\PYG{l+s+si}{\PYGZob{}}\PYG{n}{silly\PYGZus{}R}\PYG{o}{.}\PYG{n}{dot}\PYG{p}{(}\PYG{n}{silly\PYGZus{}w}\PYG{p}{)}\PYG{o}{.}\PYG{n}{iloc}\PYG{p}{[}\PYG{n}{i}\PYG{p}{]}\PYG{l+s+si}{:}\PYG{l+s+s1}{0.2f}\PYG{l+s+si}{\PYGZcb{}}\PYG{l+s+s1}{\PYGZsq{}}
    \PYG{p}{)}
\end{sphinxVerbatim}

\end{sphinxuseclass}\end{sphinxVerbatimInput}
\begin{sphinxVerbatimOutput}

\begin{sphinxuseclass}{cell_output}
\begin{sphinxVerbatim}[commandchars=\\\{\}]
Row 0:  0.33 * 0 + 0.33 * 1 + 0.33 * 2  =  1.00
Row 1:  0.33 * 3 + 0.33 * 4 + 0.33 * 5  =  4.00
\end{sphinxVerbatim}

\end{sphinxuseclass}\end{sphinxVerbatimOutput}

\end{sphinxuseclass}

\bigskip\hrule\bigskip



\section{SciPy’s \sphinxstyleliteralintitle{\sphinxupquote{minimize()}} Function}
\label{\detokenize{herron_04_lecture:scipy-s-minimize-function}}

\subsection{A Crash Course in SciPy’s \sphinxstyleliteralintitle{\sphinxupquote{minimize()}} Function}
\label{\detokenize{herron_04_lecture:a-crash-course-in-scipy-s-minimize-function}}
\sphinxAtStartPar
The \sphinxcode{\sphinxupquote{minimize()}} function from SciPy’s \sphinxcode{\sphinxupquote{optimize}} module finds the input array \sphinxcode{\sphinxupquote{x}} that minimizes the output of the function \sphinxcode{\sphinxupquote{fun}}.
The \sphinxcode{\sphinxupquote{minimize()}} function uses optimization techniques that are outside this course, but you can consider these optimization techniques to be sophisticated trial and error.

\sphinxAtStartPar
Here are the most common arguments we will pass to the \sphinxcode{\sphinxupquote{minimize()}} function:
\begin{enumerate}
\sphinxsetlistlabels{\arabic}{enumi}{enumii}{}{.}%
\item {} 
\sphinxAtStartPar
We pass our first guess for input array \sphinxcode{\sphinxupquote{x}} to argument \sphinxcode{\sphinxupquote{x0=}}.

\item {} 
\sphinxAtStartPar
We pass additional arguments for function \sphinxcode{\sphinxupquote{fun}} as a tuple to argument \sphinxcode{\sphinxupquote{args=}}.

\item {} 
\sphinxAtStartPar
We pass lower and upper bounds on \sphinxcode{\sphinxupquote{x}} as a tuple of tuples to argument \sphinxcode{\sphinxupquote{bounds=}}.

\item {} 
\sphinxAtStartPar
We constrain our results with a tuple of dictionaries of functions to argument \sphinxcode{\sphinxupquote{contraints=}}.

\end{enumerate}

\sphinxAtStartPar
Here is a simple example that minimizes the function \sphinxcode{\sphinxupquote{quadratic()}} that accepts arguments \sphinxcode{\sphinxupquote{x}} and \sphinxcode{\sphinxupquote{a}} and returns \$y = (x \sphinxhyphen{} a)\textasciicircum{}2\$.

\begin{sphinxuseclass}{cell}\begin{sphinxVerbatimInput}

\begin{sphinxuseclass}{cell_input}
\begin{sphinxVerbatim}[commandchars=\\\{\}]
\PYG{k+kn}{import} \PYG{n+nn}{scipy}\PYG{n+nn}{.}\PYG{n+nn}{optimize} \PYG{k}{as} \PYG{n+nn}{sco}
\end{sphinxVerbatim}

\end{sphinxuseclass}\end{sphinxVerbatimInput}

\end{sphinxuseclass}
\begin{sphinxuseclass}{cell}\begin{sphinxVerbatimInput}

\begin{sphinxuseclass}{cell_input}
\begin{sphinxVerbatim}[commandchars=\\\{\}]
\PYG{k}{def} \PYG{n+nf}{quadratic}\PYG{p}{(}\PYG{n}{x}\PYG{p}{,} \PYG{n}{a}\PYG{o}{=}\PYG{l+m+mi}{5}\PYG{p}{)}\PYG{p}{:}
    \PYG{k}{return} \PYG{p}{(}\PYG{n}{x} \PYG{o}{\PYGZhy{}} \PYG{n}{a}\PYG{p}{)} \PYG{o}{*}\PYG{o}{*} \PYG{l+m+mi}{2}
\end{sphinxVerbatim}

\end{sphinxuseclass}\end{sphinxVerbatimInput}

\end{sphinxuseclass}
\begin{sphinxuseclass}{cell}\begin{sphinxVerbatimInput}

\begin{sphinxuseclass}{cell_input}
\begin{sphinxVerbatim}[commandchars=\\\{\}]
\PYG{n}{quadratic}\PYG{p}{(}\PYG{n}{x}\PYG{o}{=}\PYG{l+m+mi}{5}\PYG{p}{,} \PYG{n}{a}\PYG{o}{=}\PYG{l+m+mi}{5}\PYG{p}{)}
\end{sphinxVerbatim}

\end{sphinxuseclass}\end{sphinxVerbatimInput}
\begin{sphinxVerbatimOutput}

\begin{sphinxuseclass}{cell_output}
\begin{sphinxVerbatim}[commandchars=\\\{\}]
0
\end{sphinxVerbatim}

\end{sphinxuseclass}\end{sphinxVerbatimOutput}

\end{sphinxuseclass}
\begin{sphinxuseclass}{cell}\begin{sphinxVerbatimInput}

\begin{sphinxuseclass}{cell_input}
\begin{sphinxVerbatim}[commandchars=\\\{\}]
\PYG{n}{quadratic}\PYG{p}{(}\PYG{n}{x}\PYG{o}{=}\PYG{l+m+mi}{10}\PYG{p}{,} \PYG{n}{a}\PYG{o}{=}\PYG{l+m+mi}{5}\PYG{p}{)}
\end{sphinxVerbatim}

\end{sphinxuseclass}\end{sphinxVerbatimInput}
\begin{sphinxVerbatimOutput}

\begin{sphinxuseclass}{cell_output}
\begin{sphinxVerbatim}[commandchars=\\\{\}]
25
\end{sphinxVerbatim}

\end{sphinxuseclass}\end{sphinxVerbatimOutput}

\end{sphinxuseclass}
\sphinxAtStartPar
It is helpful to plot \$y = (x \sphinxhyphen{} a)\$ first.

\begin{sphinxuseclass}{cell}\begin{sphinxVerbatimInput}

\begin{sphinxuseclass}{cell_input}
\begin{sphinxVerbatim}[commandchars=\\\{\}]
\PYG{n}{x} \PYG{o}{=} \PYG{n}{np}\PYG{o}{.}\PYG{n}{linspace}\PYG{p}{(}\PYG{o}{\PYGZhy{}}\PYG{l+m+mi}{5}\PYG{p}{,} \PYG{l+m+mi}{15}\PYG{p}{,} \PYG{l+m+mi}{101}\PYG{p}{)}
\PYG{n}{y} \PYG{o}{=} \PYG{n}{quadratic}\PYG{p}{(}\PYG{n}{x}\PYG{o}{=}\PYG{n}{x}\PYG{p}{)}
\PYG{n}{plt}\PYG{o}{.}\PYG{n}{plot}\PYG{p}{(}\PYG{n}{x}\PYG{p}{,} \PYG{n}{y}\PYG{p}{)}
\PYG{n}{plt}\PYG{o}{.}\PYG{n}{xlabel}\PYG{p}{(}\PYG{l+s+sa}{r}\PYG{l+s+s1}{\PYGZsq{}}\PYG{l+s+s1}{\PYGZdl{}x\PYGZdl{}}\PYG{l+s+s1}{\PYGZsq{}}\PYG{p}{)}
\PYG{n}{plt}\PYG{o}{.}\PYG{n}{ylabel}\PYG{p}{(}\PYG{l+s+sa}{r}\PYG{l+s+s1}{\PYGZsq{}}\PYG{l+s+s1}{\PYGZdl{}y\PYGZdl{}}\PYG{l+s+s1}{\PYGZsq{}}\PYG{p}{)}
\PYG{n}{plt}\PYG{o}{.}\PYG{n}{title}\PYG{p}{(}\PYG{l+s+sa}{r}\PYG{l+s+s1}{\PYGZsq{}}\PYG{l+s+s1}{\PYGZdl{}y = (x \PYGZhy{} 5)\PYGZca{}2\PYGZdl{}}\PYG{l+s+s1}{\PYGZsq{}}\PYG{p}{)}
\PYG{n}{plt}\PYG{o}{.}\PYG{n}{show}\PYG{p}{(}\PYG{p}{)}
\end{sphinxVerbatim}

\end{sphinxuseclass}\end{sphinxVerbatimInput}
\begin{sphinxVerbatimOutput}

\begin{sphinxuseclass}{cell_output}
\noindent\sphinxincludegraphics{{e5601187c4449bd165c85037d0df5deb0354cd30b9879456bbf596624e883c93}.png}

\end{sphinxuseclass}\end{sphinxVerbatimOutput}

\end{sphinxuseclass}
\sphinxAtStartPar
The minimum output of \sphinxcode{\sphinxupquote{quadratic()}} occurs at \$x=5\$ if we do not use bounds or constraints, even if we start far away from \$x=5\$.

\begin{sphinxuseclass}{cell}\begin{sphinxVerbatimInput}

\begin{sphinxuseclass}{cell_input}
\begin{sphinxVerbatim}[commandchars=\\\{\}]
\PYG{n}{sco}\PYG{o}{.}\PYG{n}{minimize}\PYG{p}{(}
    \PYG{n}{fun}\PYG{o}{=}\PYG{n}{quadratic}\PYG{p}{,}
    \PYG{n}{x0}\PYG{o}{=}\PYG{n}{np}\PYG{o}{.}\PYG{n}{array}\PYG{p}{(}\PYG{p}{[}\PYG{l+m+mi}{2001}\PYG{p}{]}\PYG{p}{)}
\PYG{p}{)}
\end{sphinxVerbatim}

\end{sphinxuseclass}\end{sphinxVerbatimInput}
\begin{sphinxVerbatimOutput}

\begin{sphinxuseclass}{cell_output}
\begin{sphinxVerbatim}[commandchars=\\\{\}]
      fun: 2.0392713450495178e\PYGZhy{}16
 hess\PYGZus{}inv: array([[0.5]])
      jac: array([\PYGZhy{}1.3659e\PYGZhy{}08])
  message: \PYGZsq{}Optimization terminated successfully.\PYGZsq{}
     nfev: 18
      nit: 4
     njev: 9
   status: 0
  success: True
        x: array([5.])
\end{sphinxVerbatim}

\end{sphinxuseclass}\end{sphinxVerbatimOutput}

\end{sphinxuseclass}
\sphinxAtStartPar
The minimum output of \sphinxcode{\sphinxupquote{quadratic()}} occurs at \$x=6\$ if we bound \sphinxcode{\sphinxupquote{x}} between 6 and 10 (i.e., \$6 \textbackslash{}leq x \textbackslash{}leq 10\$).

\begin{sphinxuseclass}{cell}\begin{sphinxVerbatimInput}

\begin{sphinxuseclass}{cell_input}
\begin{sphinxVerbatim}[commandchars=\\\{\}]
\PYG{n}{sco}\PYG{o}{.}\PYG{n}{minimize}\PYG{p}{(}
    \PYG{n}{fun}\PYG{o}{=}\PYG{n}{quadratic}\PYG{p}{,}
    \PYG{n}{x0}\PYG{o}{=}\PYG{n}{np}\PYG{o}{.}\PYG{n}{array}\PYG{p}{(}\PYG{p}{[}\PYG{l+m+mi}{2001}\PYG{p}{]}\PYG{p}{)}\PYG{p}{,}
    \PYG{n}{bounds}\PYG{o}{=}\PYG{p}{(}\PYG{p}{(}\PYG{l+m+mi}{6}\PYG{p}{,} \PYG{l+m+mi}{10}\PYG{p}{)}\PYG{p}{,}\PYG{p}{)}
\PYG{p}{)}
\end{sphinxVerbatim}

\end{sphinxuseclass}\end{sphinxVerbatimInput}
\begin{sphinxVerbatimOutput}

\begin{sphinxuseclass}{cell_output}
\begin{sphinxVerbatim}[commandchars=\\\{\}]
      fun: 1.0
 hess\PYGZus{}inv: \PYGZlt{}1x1 LbfgsInvHessProduct with dtype=float64\PYGZgt{}
      jac: array([2.])
  message: \PYGZsq{}CONVERGENCE: NORM\PYGZus{}OF\PYGZus{}PROJECTED\PYGZus{}GRADIENT\PYGZus{}\PYGZlt{}=\PYGZus{}PGTOL\PYGZsq{}
     nfev: 4
      nit: 1
     njev: 2
   status: 0
  success: True
        x: array([6.])
\end{sphinxVerbatim}

\end{sphinxuseclass}\end{sphinxVerbatimOutput}

\end{sphinxuseclass}
\sphinxAtStartPar
The minimum output of \sphinxcode{\sphinxupquote{quadratic()}} occurs at \$x=6\$, again, if we constrain \sphinxcode{\sphinxupquote{x \sphinxhyphen{} 6}} to be non\sphinxhyphen{}negative.
We use bounds to limit the search space directly, and we use constraints to limit the search space indirectly based on a formula.

\begin{sphinxuseclass}{cell}\begin{sphinxVerbatimInput}

\begin{sphinxuseclass}{cell_input}
\begin{sphinxVerbatim}[commandchars=\\\{\}]
\PYG{n}{sco}\PYG{o}{.}\PYG{n}{minimize}\PYG{p}{(}
    \PYG{n}{fun}\PYG{o}{=}\PYG{n}{quadratic}\PYG{p}{,}
    \PYG{n}{x0}\PYG{o}{=}\PYG{n}{np}\PYG{o}{.}\PYG{n}{array}\PYG{p}{(}\PYG{p}{[}\PYG{l+m+mi}{2001}\PYG{p}{]}\PYG{p}{)}\PYG{p}{,}
    \PYG{n}{constraints}\PYG{o}{=}\PYG{p}{(}\PYG{p}{\PYGZob{}}\PYG{l+s+s1}{\PYGZsq{}}\PYG{l+s+s1}{type}\PYG{l+s+s1}{\PYGZsq{}}\PYG{p}{:} \PYG{l+s+s1}{\PYGZsq{}}\PYG{l+s+s1}{ineq}\PYG{l+s+s1}{\PYGZsq{}}\PYG{p}{,} \PYG{l+s+s1}{\PYGZsq{}}\PYG{l+s+s1}{fun}\PYG{l+s+s1}{\PYGZsq{}}\PYG{p}{:} \PYG{k}{lambda} \PYG{n}{x}\PYG{p}{:} \PYG{n}{x} \PYG{o}{\PYGZhy{}} \PYG{l+m+mi}{6}\PYG{p}{\PYGZcb{}}\PYG{p}{)}
\PYG{p}{)}
\end{sphinxVerbatim}

\end{sphinxuseclass}\end{sphinxVerbatimInput}
\begin{sphinxVerbatimOutput}

\begin{sphinxuseclass}{cell_output}
\begin{sphinxVerbatim}[commandchars=\\\{\}]
     fun: 1.0000000000000018
     jac: array([2.])
 message: \PYGZsq{}Optimization terminated successfully\PYGZsq{}
    nfev: 6
     nit: 3
    njev: 3
  status: 0
 success: True
       x: array([6.])
\end{sphinxVerbatim}

\end{sphinxuseclass}\end{sphinxVerbatimOutput}

\end{sphinxuseclass}
\sphinxAtStartPar
We can use the \sphinxcode{\sphinxupquote{args=}} argument to pass additional arguments to \sphinxcode{\sphinxupquote{fun}}.
For example, we change the \sphinxcode{\sphinxupquote{a=}} argument in \sphinxcode{\sphinxupquote{quadratic()}} from the default of \sphinxcode{\sphinxupquote{a=5}} to \sphinxcode{\sphinxupquote{a=20}} with \sphinxcode{\sphinxupquote{args=(20,)}}.
Note that \sphinxcode{\sphinxupquote{args=}} expects a tuple, so we need a trailing comma \sphinxcode{\sphinxupquote{,}} if we have one argument.

\begin{sphinxuseclass}{cell}\begin{sphinxVerbatimInput}

\begin{sphinxuseclass}{cell_input}
\begin{sphinxVerbatim}[commandchars=\\\{\}]
\PYG{n}{sco}\PYG{o}{.}\PYG{n}{minimize}\PYG{p}{(}
    \PYG{n}{fun}\PYG{o}{=}\PYG{n}{quadratic}\PYG{p}{,}
    \PYG{n}{args}\PYG{o}{=}\PYG{p}{(}\PYG{l+m+mi}{20}\PYG{p}{,}\PYG{p}{)}\PYG{p}{,}
    \PYG{n}{x0}\PYG{o}{=}\PYG{n}{np}\PYG{o}{.}\PYG{n}{array}\PYG{p}{(}\PYG{p}{[}\PYG{l+m+mi}{2001}\PYG{p}{]}\PYG{p}{)}\PYG{p}{,}
\PYG{p}{)}
\end{sphinxVerbatim}

\end{sphinxuseclass}\end{sphinxVerbatimInput}
\begin{sphinxVerbatimOutput}

\begin{sphinxuseclass}{cell_output}
\begin{sphinxVerbatim}[commandchars=\\\{\}]
      fun: 7.090392030754976e\PYGZhy{}17
 hess\PYGZus{}inv: array([[0.5]])
      jac: array([\PYGZhy{}1.9397e\PYGZhy{}09])
  message: \PYGZsq{}Optimization terminated successfully.\PYGZsq{}
     nfev: 18
      nit: 4
     njev: 9
   status: 0
  success: True
        x: array([20.])
\end{sphinxVerbatim}

\end{sphinxuseclass}\end{sphinxVerbatimOutput}

\end{sphinxuseclass}

\subsection{The Minimum Variance Portfolio}
\label{\detokenize{herron_04_lecture:the-minimum-variance-portfolio}}
\sphinxAtStartPar
We can find the minimum variance portfolio with \sphinxcode{\sphinxupquote{minimize()}} function from SciPy’s \sphinxcode{\sphinxupquote{optimize}} module.
The \sphinxcode{\sphinxupquote{minimize()}} function with vary an input array \sphinxcode{\sphinxupquote{x}} (starting from argument \sphinxcode{\sphinxupquote{x0=}}) to minimize the objective function \sphinxcode{\sphinxupquote{fun=}} subject to the bounds and constraints in \sphinxcode{\sphinxupquote{bounds=}} and \sphinxcode{\sphinxupquote{constraints=}}.
We will define a function \sphinxcode{\sphinxupquote{port\_vol()}} to calculate portfolio volatility.
The first argument to \sphinxcode{\sphinxupquote{port\_vol()}} must be the input array \sphinxcode{\sphinxupquote{x}} that the \sphinxcode{\sphinxupquote{minimize()}} function searches over.
For clarity, we will call this first argument \sphinxcode{\sphinxupquote{x}}, but the argument’s name is not important.

\begin{sphinxuseclass}{cell}\begin{sphinxVerbatimInput}

\begin{sphinxuseclass}{cell_input}
\begin{sphinxVerbatim}[commandchars=\\\{\}]
\PYG{k}{def} \PYG{n+nf}{port\PYGZus{}vol}\PYG{p}{(}\PYG{n}{x}\PYG{p}{,} \PYG{n}{r}\PYG{p}{,} \PYG{n}{ppy}\PYG{p}{)}\PYG{p}{:}
    \PYG{k}{return} \PYG{n}{np}\PYG{o}{.}\PYG{n}{sqrt}\PYG{p}{(}\PYG{n}{ppy}\PYG{p}{)} \PYG{o}{*} \PYG{n}{r}\PYG{o}{.}\PYG{n}{dot}\PYG{p}{(}\PYG{n}{x}\PYG{p}{)}\PYG{o}{.}\PYG{n}{std}\PYG{p}{(}\PYG{p}{)}
\end{sphinxVerbatim}

\end{sphinxuseclass}\end{sphinxVerbatimInput}

\end{sphinxuseclass}
\sphinxAtStartPar
We will eventually need a mean portfolio return function, too.

\begin{sphinxuseclass}{cell}\begin{sphinxVerbatimInput}

\begin{sphinxuseclass}{cell_input}
\begin{sphinxVerbatim}[commandchars=\\\{\}]
\PYG{k}{def} \PYG{n+nf}{port\PYGZus{}mean}\PYG{p}{(}\PYG{n}{x}\PYG{p}{,} \PYG{n}{r}\PYG{p}{,} \PYG{n}{ppy}\PYG{p}{)}\PYG{p}{:}
    \PYG{k}{return} \PYG{n}{ppy} \PYG{o}{*} \PYG{n}{r}\PYG{o}{.}\PYG{n}{dot}\PYG{p}{(}\PYG{n}{x}\PYG{p}{)}\PYG{o}{.}\PYG{n}{mean}\PYG{p}{(}\PYG{p}{)}
\end{sphinxVerbatim}

\end{sphinxuseclass}\end{sphinxVerbatimInput}

\end{sphinxuseclass}
\begin{sphinxuseclass}{cell}\begin{sphinxVerbatimInput}

\begin{sphinxuseclass}{cell_input}
\begin{sphinxVerbatim}[commandchars=\\\{\}]
\PYG{n}{res\PYGZus{}mv} \PYG{o}{=} \PYG{n}{sco}\PYG{o}{.}\PYG{n}{minimize}\PYG{p}{(}
    \PYG{n}{fun}\PYG{o}{=}\PYG{n}{port\PYGZus{}vol}\PYG{p}{,} \PYG{c+c1}{\PYGZsh{} objective function that we minimize}
    \PYG{n}{x0}\PYG{o}{=}\PYG{n}{np}\PYG{o}{.}\PYG{n}{ones}\PYG{p}{(}\PYG{n}{returns}\PYG{o}{.}\PYG{n}{shape}\PYG{p}{[}\PYG{l+m+mi}{1}\PYG{p}{]}\PYG{p}{)} \PYG{o}{/} \PYG{n}{returns}\PYG{o}{.}\PYG{n}{shape}\PYG{p}{[}\PYG{l+m+mi}{1}\PYG{p}{]}\PYG{p}{,} \PYG{c+c1}{\PYGZsh{} initial portfolio weights}
    \PYG{n}{args}\PYG{o}{=}\PYG{p}{(}\PYG{n}{returns}\PYG{p}{,} \PYG{l+m+mi}{252}\PYG{p}{)}\PYG{p}{,} \PYG{c+c1}{\PYGZsh{} additional arguments to our objective function}
    \PYG{n}{bounds}\PYG{o}{=}\PYG{p}{[}\PYG{p}{(}\PYG{l+m+mi}{0}\PYG{p}{,}\PYG{l+m+mi}{1}\PYG{p}{)} \PYG{k}{for} \PYG{n}{\PYGZus{}} \PYG{o+ow}{in} \PYG{n}{returns}\PYG{p}{]}\PYG{p}{,} \PYG{c+c1}{\PYGZsh{} bounds limit the search space for each portfolio weights}
    \PYG{n}{constraints}\PYG{o}{=}\PYG{p}{(}
        \PYG{p}{\PYGZob{}}\PYG{l+s+s1}{\PYGZsq{}}\PYG{l+s+s1}{type}\PYG{l+s+s1}{\PYGZsq{}}\PYG{p}{:} \PYG{l+s+s1}{\PYGZsq{}}\PYG{l+s+s1}{eq}\PYG{l+s+s1}{\PYGZsq{}}\PYG{p}{,} \PYG{l+s+s1}{\PYGZsq{}}\PYG{l+s+s1}{fun}\PYG{l+s+s1}{\PYGZsq{}}\PYG{p}{:} \PYG{k}{lambda} \PYG{n}{x}\PYG{p}{:} \PYG{n}{x}\PYG{o}{.}\PYG{n}{sum}\PYG{p}{(}\PYG{p}{)} \PYG{o}{\PYGZhy{}} \PYG{l+m+mi}{1}\PYG{p}{\PYGZcb{}} \PYG{c+c1}{\PYGZsh{} minimize drives \PYGZdq{}eq\PYGZdq{} constraints to zero}
    \PYG{p}{)}
\PYG{p}{)}

\PYG{n+nb}{print}\PYG{p}{(}\PYG{n}{res\PYGZus{}mv}\PYG{p}{)}
\end{sphinxVerbatim}

\end{sphinxuseclass}\end{sphinxVerbatimInput}
\begin{sphinxVerbatimOutput}

\begin{sphinxuseclass}{cell_output}
\begin{sphinxVerbatim}[commandchars=\\\{\}]
     fun: 0.28980781598721805
     jac: array([0.2903, 0.29  , 0.2894, 0.2897, 0.397 , 0.3458])
 message: \PYGZsq{}Optimization terminated successfully\PYGZsq{}
    nfev: 56
     nit: 8
    njev: 8
  status: 0
 success: True
       x: array([3.0827e\PYGZhy{}01, 0.0000e+00, 2.3112e\PYGZhy{}01, 4.6061e\PYGZhy{}01, 7.4593e\PYGZhy{}17,
       0.0000e+00])
\end{sphinxVerbatim}

\end{sphinxuseclass}\end{sphinxVerbatimOutput}

\end{sphinxuseclass}
\sphinxAtStartPar
What are the attributes of this minimum variance portfolio?

\begin{sphinxuseclass}{cell}\begin{sphinxVerbatimInput}

\begin{sphinxuseclass}{cell_input}
\begin{sphinxVerbatim}[commandchars=\\\{\}]
\PYG{k}{def} \PYG{n+nf}{print\PYGZus{}port\PYGZus{}res}\PYG{p}{(}\PYG{n}{w}\PYG{p}{,} \PYG{n}{r}\PYG{p}{,} \PYG{n}{title}\PYG{p}{,} \PYG{n}{ppy}\PYG{o}{=}\PYG{l+m+mi}{252}\PYG{p}{,} \PYG{n}{tgt}\PYG{o}{=}\PYG{k+kc}{None}\PYG{p}{)}\PYG{p}{:}
    \PYG{n}{width} \PYG{o}{=} \PYG{n+nb}{len}\PYG{p}{(}\PYG{n}{title}\PYG{p}{)}
    \PYG{n}{rp} \PYG{o}{=} \PYG{n}{r}\PYG{o}{.}\PYG{n}{dot}\PYG{p}{(}\PYG{n}{w}\PYG{p}{)}
    \PYG{n}{mu} \PYG{o}{=} \PYG{n}{ppy} \PYG{o}{*} \PYG{n}{rp}\PYG{o}{.}\PYG{n}{mean}\PYG{p}{(}\PYG{p}{)}
    \PYG{n}{sigma} \PYG{o}{=} \PYG{n}{np}\PYG{o}{.}\PYG{n}{sqrt}\PYG{p}{(}\PYG{n}{ppy}\PYG{p}{)} \PYG{o}{*} \PYG{n}{rp}\PYG{o}{.}\PYG{n}{std}\PYG{p}{(}\PYG{p}{)}
    \PYG{k}{if} \PYG{n}{tgt} \PYG{o+ow}{is} \PYG{o+ow}{not} \PYG{k+kc}{None}\PYG{p}{:}
        \PYG{n}{er} \PYG{o}{=} \PYG{n}{rp}\PYG{o}{.}\PYG{n}{sub}\PYG{p}{(}\PYG{n}{tgt}\PYG{p}{)}
        \PYG{n}{sr} \PYG{o}{=} \PYG{n}{np}\PYG{o}{.}\PYG{n}{sqrt}\PYG{p}{(}\PYG{n}{ppy}\PYG{p}{)} \PYG{o}{*} \PYG{n}{er}\PYG{o}{.}\PYG{n}{mean}\PYG{p}{(}\PYG{p}{)} \PYG{o}{/} \PYG{n}{er}\PYG{o}{.}\PYG{n}{std}\PYG{p}{(}\PYG{p}{)}
    \PYG{k}{else}\PYG{p}{:}
        \PYG{n}{sr} \PYG{o}{=} \PYG{k+kc}{None}
    
    \PYG{k}{return} \PYG{n+nb}{print}\PYG{p}{(}
        \PYG{n}{title}\PYG{p}{,}
        \PYG{l+s+s1}{\PYGZsq{}}\PYG{l+s+s1}{=}\PYG{l+s+s1}{\PYGZsq{}} \PYG{o}{*} \PYG{n}{width}\PYG{p}{,}
        \PYG{l+s+s1}{\PYGZsq{}}\PYG{l+s+s1}{\PYGZsq{}}\PYG{p}{,}
        \PYG{l+s+s1}{\PYGZsq{}}\PYG{l+s+s1}{Performance}\PYG{l+s+s1}{\PYGZsq{}}\PYG{p}{,}
        \PYG{l+s+s1}{\PYGZsq{}}\PYG{l+s+s1}{\PYGZhy{}}\PYG{l+s+s1}{\PYGZsq{}} \PYG{o}{*} \PYG{n}{width}\PYG{p}{,}
        \PYG{l+s+s1}{\PYGZsq{}}\PYG{l+s+s1}{Return:}\PYG{l+s+s1}{\PYGZsq{}}\PYG{o}{.}\PYG{n}{ljust}\PYG{p}{(}\PYG{n}{width} \PYG{o}{\PYGZhy{}} \PYG{l+m+mi}{6}\PYG{p}{)} \PYG{o}{+} \PYG{l+s+sa}{f}\PYG{l+s+s1}{\PYGZsq{}}\PYG{l+s+si}{\PYGZob{}}\PYG{n}{mu}\PYG{l+s+si}{:}\PYG{l+s+s1}{0.4f}\PYG{l+s+si}{\PYGZcb{}}\PYG{l+s+s1}{\PYGZsq{}}\PYG{p}{,}
        \PYG{l+s+s1}{\PYGZsq{}}\PYG{l+s+s1}{Volatility:}\PYG{l+s+s1}{\PYGZsq{}}\PYG{o}{.}\PYG{n}{ljust}\PYG{p}{(}\PYG{n}{width} \PYG{o}{\PYGZhy{}} \PYG{l+m+mi}{6}\PYG{p}{)} \PYG{o}{+} \PYG{l+s+sa}{f}\PYG{l+s+s1}{\PYGZsq{}}\PYG{l+s+si}{\PYGZob{}}\PYG{n}{sigma}\PYG{l+s+si}{:}\PYG{l+s+s1}{0.4f}\PYG{l+s+si}{\PYGZcb{}}\PYG{l+s+s1}{\PYGZsq{}}\PYG{p}{,}
        \PYG{l+s+s1}{\PYGZsq{}}\PYG{l+s+s1}{Sharpe Ratio:}\PYG{l+s+s1}{\PYGZsq{}}\PYG{o}{.}\PYG{n}{ljust}\PYG{p}{(}\PYG{n}{width} \PYG{o}{\PYGZhy{}} \PYG{l+m+mi}{6}\PYG{p}{)} \PYG{o}{+} \PYG{l+s+sa}{f}\PYG{l+s+s1}{\PYGZsq{}}\PYG{l+s+si}{\PYGZob{}}\PYG{n}{sr}\PYG{l+s+si}{:}\PYG{l+s+s1}{0.4f}\PYG{l+s+si}{\PYGZcb{}}\PYG{l+s+se}{\PYGZbs{}n}\PYG{l+s+s1}{\PYGZsq{}} \PYG{k}{if} \PYG{n}{sr} \PYG{o+ow}{is} \PYG{o+ow}{not} \PYG{k+kc}{None} \PYG{k}{else} \PYG{l+s+s1}{\PYGZsq{}}\PYG{l+s+s1}{\PYGZsq{}}\PYG{p}{,}
        \PYG{l+s+s1}{\PYGZsq{}}\PYG{l+s+s1}{Weights}\PYG{l+s+s1}{\PYGZsq{}}\PYG{p}{,} 
        \PYG{l+s+s1}{\PYGZsq{}}\PYG{l+s+s1}{\PYGZhy{}}\PYG{l+s+s1}{\PYGZsq{}} \PYG{o}{*} \PYG{n}{width}\PYG{p}{,} 
        \PYG{l+s+s1}{\PYGZsq{}}\PYG{l+s+se}{\PYGZbs{}n}\PYG{l+s+s1}{\PYGZsq{}}\PYG{o}{.}\PYG{n}{join}\PYG{p}{(}\PYG{p}{[}\PYG{l+s+sa}{f}\PYG{l+s+s1}{\PYGZsq{}}\PYG{l+s+si}{\PYGZob{}}\PYG{n}{\PYGZus{}r}\PYG{l+s+si}{\PYGZcb{}}\PYG{l+s+s1}{:}\PYG{l+s+s1}{\PYGZsq{}}\PYG{o}{.}\PYG{n}{ljust}\PYG{p}{(}\PYG{n}{width} \PYG{o}{\PYGZhy{}} \PYG{l+m+mi}{6}\PYG{p}{)} \PYG{o}{+} \PYG{l+s+sa}{f}\PYG{l+s+s1}{\PYGZsq{}}\PYG{l+s+si}{\PYGZob{}}\PYG{n}{\PYGZus{}w}\PYG{l+s+si}{:}\PYG{l+s+s1}{0.4f}\PYG{l+s+si}{\PYGZcb{}}\PYG{l+s+s1}{\PYGZsq{}} \PYG{k}{for} \PYG{n}{\PYGZus{}r}\PYG{p}{,} \PYG{n}{\PYGZus{}w} \PYG{o+ow}{in} \PYG{n+nb}{zip}\PYG{p}{(}\PYG{n}{r}\PYG{o}{.}\PYG{n}{columns}\PYG{p}{,} \PYG{n}{w}\PYG{p}{)}\PYG{p}{]}\PYG{p}{)}\PYG{p}{,}
        \PYG{n}{sep}\PYG{o}{=}\PYG{l+s+s1}{\PYGZsq{}}\PYG{l+s+se}{\PYGZbs{}n}\PYG{l+s+s1}{\PYGZsq{}}\PYG{p}{,}
    \PYG{p}{)}
\end{sphinxVerbatim}

\end{sphinxuseclass}\end{sphinxVerbatimInput}

\end{sphinxuseclass}
\begin{sphinxuseclass}{cell}\begin{sphinxVerbatimInput}

\begin{sphinxuseclass}{cell_input}
\begin{sphinxVerbatim}[commandchars=\\\{\}]
\PYG{n}{print\PYGZus{}port\PYGZus{}res}\PYG{p}{(}\PYG{n}{w}\PYG{o}{=}\PYG{n}{res\PYGZus{}mv}\PYG{p}{[}\PYG{l+s+s1}{\PYGZsq{}}\PYG{l+s+s1}{x}\PYG{l+s+s1}{\PYGZsq{}}\PYG{p}{]}\PYG{p}{,} \PYG{n}{r}\PYG{o}{=}\PYG{n}{returns}\PYG{p}{,} \PYG{n}{title}\PYG{o}{=}\PYG{l+s+s1}{\PYGZsq{}}\PYG{l+s+s1}{Minimum Variance Portfolio}\PYG{l+s+s1}{\PYGZsq{}}\PYG{p}{)}
\end{sphinxVerbatim}

\end{sphinxuseclass}\end{sphinxVerbatimInput}
\begin{sphinxVerbatimOutput}

\begin{sphinxuseclass}{cell_output}
\begin{sphinxVerbatim}[commandchars=\\\{\}]
Minimum Variance Portfolio
==========================

Performance
\PYGZhy{}\PYGZhy{}\PYGZhy{}\PYGZhy{}\PYGZhy{}\PYGZhy{}\PYGZhy{}\PYGZhy{}\PYGZhy{}\PYGZhy{}\PYGZhy{}\PYGZhy{}\PYGZhy{}\PYGZhy{}\PYGZhy{}\PYGZhy{}\PYGZhy{}\PYGZhy{}\PYGZhy{}\PYGZhy{}\PYGZhy{}\PYGZhy{}\PYGZhy{}\PYGZhy{}\PYGZhy{}\PYGZhy{}
Return:             0.2984
Volatility:         0.2898

Weights
\PYGZhy{}\PYGZhy{}\PYGZhy{}\PYGZhy{}\PYGZhy{}\PYGZhy{}\PYGZhy{}\PYGZhy{}\PYGZhy{}\PYGZhy{}\PYGZhy{}\PYGZhy{}\PYGZhy{}\PYGZhy{}\PYGZhy{}\PYGZhy{}\PYGZhy{}\PYGZhy{}\PYGZhy{}\PYGZhy{}\PYGZhy{}\PYGZhy{}\PYGZhy{}\PYGZhy{}\PYGZhy{}\PYGZhy{}
AAPL:               0.3083
AMZN:               0.0000
GOOG:               0.2311
MSFT:               0.4606
NVDA:               0.0000
TSLA:               0.0000
\end{sphinxVerbatim}

\end{sphinxuseclass}\end{sphinxVerbatimOutput}

\end{sphinxuseclass}

\subsection{The (Mean\sphinxhyphen{}Variance) Efficient Frontier}
\label{\detokenize{herron_04_lecture:the-mean-variance-efficient-frontier}}
\sphinxAtStartPar
We will use the \sphinxcode{\sphinxupquote{minimize()}} function to map the efficient frontier.
Here is a basic outline:
\begin{enumerate}
\sphinxsetlistlabels{\arabic}{enumi}{enumii}{}{.}%
\item {} 
\sphinxAtStartPar
Create a NumPy array \sphinxcode{\sphinxupquote{tret}} of target returns

\item {} 
\sphinxAtStartPar
Create an empty list \sphinxcode{\sphinxupquote{res\_ef}} of \sphinxcode{\sphinxupquote{minimize()}} results

\item {} 
\sphinxAtStartPar
Loop over \sphinxcode{\sphinxupquote{tret}}, passing each as a constraint to the \sphinxcode{\sphinxupquote{minimize()}} function

\item {} 
\sphinxAtStartPar
Append each \sphinxcode{\sphinxupquote{minimize()}} result to \sphinxcode{\sphinxupquote{res\_ef}}

\end{enumerate}

\begin{sphinxuseclass}{cell}\begin{sphinxVerbatimInput}

\begin{sphinxuseclass}{cell_input}
\begin{sphinxVerbatim}[commandchars=\\\{\}]
\PYG{n}{tret} \PYG{o}{=} \PYG{l+m+mi}{252} \PYG{o}{*} \PYG{n}{np}\PYG{o}{.}\PYG{n}{linspace}\PYG{p}{(}\PYG{n}{returns}\PYG{o}{.}\PYG{n}{mean}\PYG{p}{(}\PYG{p}{)}\PYG{o}{.}\PYG{n}{min}\PYG{p}{(}\PYG{p}{)}\PYG{p}{,} \PYG{n}{returns}\PYG{o}{.}\PYG{n}{mean}\PYG{p}{(}\PYG{p}{)}\PYG{o}{.}\PYG{n}{max}\PYG{p}{(}\PYG{p}{)}\PYG{p}{,} \PYG{l+m+mi}{25}\PYG{p}{)}

\PYG{n}{tret}
\end{sphinxVerbatim}

\end{sphinxuseclass}\end{sphinxVerbatimInput}
\begin{sphinxVerbatimOutput}

\begin{sphinxuseclass}{cell_output}
\begin{sphinxVerbatim}[commandchars=\\\{\}]
array([0.076 , 0.1061, 0.1362, 0.1663, 0.1964, 0.2266, 0.2567, 0.2868,
       0.3169, 0.347 , 0.3772, 0.4073, 0.4374, 0.4675, 0.4976, 0.5277,
       0.5579, 0.588 , 0.6181, 0.6482, 0.6783, 0.7085, 0.7386, 0.7687,
       0.7988])
\end{sphinxVerbatim}

\end{sphinxuseclass}\end{sphinxVerbatimOutput}

\end{sphinxuseclass}
\sphinxAtStartPar
We will loop over these target returns, finding the minimum variance portfolio for each target return.

\begin{sphinxuseclass}{cell}\begin{sphinxVerbatimInput}

\begin{sphinxuseclass}{cell_input}
\begin{sphinxVerbatim}[commandchars=\\\{\}]
\PYG{n}{res\PYGZus{}ef} \PYG{o}{=} \PYG{p}{[}\PYG{p}{]}

\PYG{k}{for} \PYG{n}{t} \PYG{o+ow}{in} \PYG{n}{tret}\PYG{p}{:}
    \PYG{n}{\PYGZus{}} \PYG{o}{=} \PYG{n}{sco}\PYG{o}{.}\PYG{n}{minimize}\PYG{p}{(}
        \PYG{n}{fun}\PYG{o}{=}\PYG{n}{port\PYGZus{}vol}\PYG{p}{,} \PYG{c+c1}{\PYGZsh{} minimize portfolio volatility}
        \PYG{n}{x0}\PYG{o}{=}\PYG{n}{np}\PYG{o}{.}\PYG{n}{ones}\PYG{p}{(}\PYG{n}{returns}\PYG{o}{.}\PYG{n}{shape}\PYG{p}{[}\PYG{l+m+mi}{1}\PYG{p}{]}\PYG{p}{)} \PYG{o}{/} \PYG{n}{returns}\PYG{o}{.}\PYG{n}{shape}\PYG{p}{[}\PYG{l+m+mi}{1}\PYG{p}{]}\PYG{p}{,} \PYG{c+c1}{\PYGZsh{} initial portfolio weights}
        \PYG{n}{args}\PYG{o}{=}\PYG{p}{(}\PYG{n}{returns}\PYG{p}{,} \PYG{l+m+mi}{252}\PYG{p}{)}\PYG{p}{,} \PYG{c+c1}{\PYGZsh{} additional arguments to fun, in order}
        \PYG{n}{bounds}\PYG{o}{=}\PYG{p}{[}\PYG{p}{(}\PYG{l+m+mi}{0}\PYG{p}{,} \PYG{l+m+mi}{1}\PYG{p}{)} \PYG{k}{for} \PYG{n}{c} \PYG{o+ow}{in} \PYG{n}{returns}\PYG{o}{.}\PYG{n}{columns}\PYG{p}{]}\PYG{p}{,} \PYG{c+c1}{\PYGZsh{} bounds limit the search space for each portfolio weight}
        \PYG{n}{constraints}\PYG{o}{=}\PYG{p}{(}
            \PYG{p}{\PYGZob{}}\PYG{l+s+s1}{\PYGZsq{}}\PYG{l+s+s1}{type}\PYG{l+s+s1}{\PYGZsq{}}\PYG{p}{:} \PYG{l+s+s1}{\PYGZsq{}}\PYG{l+s+s1}{eq}\PYG{l+s+s1}{\PYGZsq{}}\PYG{p}{,} \PYG{l+s+s1}{\PYGZsq{}}\PYG{l+s+s1}{fun}\PYG{l+s+s1}{\PYGZsq{}}\PYG{p}{:} \PYG{k}{lambda} \PYG{n}{x}\PYG{p}{:} \PYG{n}{x}\PYG{o}{.}\PYG{n}{sum}\PYG{p}{(}\PYG{p}{)} \PYG{o}{\PYGZhy{}} \PYG{l+m+mi}{1}\PYG{p}{\PYGZcb{}}\PYG{p}{,} \PYG{c+c1}{\PYGZsh{} constrain sum of weights to one}
            \PYG{p}{\PYGZob{}}\PYG{l+s+s1}{\PYGZsq{}}\PYG{l+s+s1}{type}\PYG{l+s+s1}{\PYGZsq{}}\PYG{p}{:} \PYG{l+s+s1}{\PYGZsq{}}\PYG{l+s+s1}{eq}\PYG{l+s+s1}{\PYGZsq{}}\PYG{p}{,} \PYG{l+s+s1}{\PYGZsq{}}\PYG{l+s+s1}{fun}\PYG{l+s+s1}{\PYGZsq{}}\PYG{p}{:} \PYG{k}{lambda} \PYG{n}{x}\PYG{p}{:} \PYG{n}{port\PYGZus{}mean}\PYG{p}{(}\PYG{n}{x}\PYG{o}{=}\PYG{n}{x}\PYG{p}{,} \PYG{n}{r}\PYG{o}{=}\PYG{n}{returns}\PYG{p}{,} \PYG{n}{ppy}\PYG{o}{=}\PYG{l+m+mi}{252}\PYG{p}{)} \PYG{o}{\PYGZhy{}} \PYG{n}{t}\PYG{p}{\PYGZcb{}} \PYG{c+c1}{\PYGZsh{} constrains portfolio mean return to the target return}

        \PYG{p}{)}
    \PYG{p}{)}
    \PYG{n}{res\PYGZus{}ef}\PYG{o}{.}\PYG{n}{append}\PYG{p}{(}\PYG{n}{\PYGZus{}}\PYG{p}{)}
\end{sphinxVerbatim}

\end{sphinxuseclass}\end{sphinxVerbatimInput}

\end{sphinxuseclass}
\sphinxAtStartPar
List \sphinxcode{\sphinxupquote{res\_ef}} contains the results of all 25 minimum\sphinxhyphen{}variance portfolios.
For example, \sphinxcode{\sphinxupquote{res\_ef{[}0{]}}} is the minimum variance portfolio for the lowest target return.

\begin{sphinxuseclass}{cell}\begin{sphinxVerbatimInput}

\begin{sphinxuseclass}{cell_input}
\begin{sphinxVerbatim}[commandchars=\\\{\}]
\PYG{n}{res\PYGZus{}ef}\PYG{p}{[}\PYG{l+m+mi}{0}\PYG{p}{]}
\end{sphinxVerbatim}

\end{sphinxuseclass}\end{sphinxVerbatimInput}
\begin{sphinxVerbatimOutput}

\begin{sphinxuseclass}{cell_output}
\begin{sphinxVerbatim}[commandchars=\\\{\}]
     fun: 0.38644121151979843
     jac: array([0.2148, 0.3864, 0.2244, 0.2158, 0.3411, 0.3183])
 message: \PYGZsq{}Optimization terminated successfully\PYGZsq{}
    nfev: 21
     nit: 3
    njev: 3
  status: 0
 success: True
       x: array([0.0000e+00, 1.0000e+00, 0.0000e+00, 2.9160e\PYGZhy{}11, 0.0000e+00,
       5.2856e\PYGZhy{}17])
\end{sphinxVerbatim}

\end{sphinxuseclass}\end{sphinxVerbatimOutput}

\end{sphinxuseclass}
\sphinxAtStartPar
I typically check that all portfolio volatility minimization succeeds.
If a portfolio volatility minimization fails, we should check our function, bounds, and constraints.

\begin{sphinxuseclass}{cell}\begin{sphinxVerbatimInput}

\begin{sphinxuseclass}{cell_input}
\begin{sphinxVerbatim}[commandchars=\\\{\}]
\PYG{k}{for} \PYG{n}{r} \PYG{o+ow}{in} \PYG{n}{res\PYGZus{}ef}\PYG{p}{:}
    \PYG{k}{assert} \PYG{n}{r}\PYG{p}{[}\PYG{l+s+s1}{\PYGZsq{}}\PYG{l+s+s1}{success}\PYG{l+s+s1}{\PYGZsq{}}\PYG{p}{]} 
\end{sphinxVerbatim}

\end{sphinxuseclass}\end{sphinxVerbatimInput}

\end{sphinxuseclass}
\sphinxAtStartPar
We can combine the target returns and volatilities into a data frame \sphinxcode{\sphinxupquote{ef}}.

\begin{sphinxuseclass}{cell}\begin{sphinxVerbatimInput}

\begin{sphinxuseclass}{cell_input}
\begin{sphinxVerbatim}[commandchars=\\\{\}]
\PYG{n}{ef} \PYG{o}{=} \PYG{n}{pd}\PYG{o}{.}\PYG{n}{DataFrame}\PYG{p}{(}
    \PYG{p}{\PYGZob{}}
        \PYG{l+s+s1}{\PYGZsq{}}\PYG{l+s+s1}{tret}\PYG{l+s+s1}{\PYGZsq{}}\PYG{p}{:} \PYG{n}{tret}\PYG{p}{,}
        \PYG{l+s+s1}{\PYGZsq{}}\PYG{l+s+s1}{tvol}\PYG{l+s+s1}{\PYGZsq{}}\PYG{p}{:} \PYG{n}{np}\PYG{o}{.}\PYG{n}{array}\PYG{p}{(}\PYG{p}{[}\PYG{n}{r}\PYG{p}{[}\PYG{l+s+s1}{\PYGZsq{}}\PYG{l+s+s1}{fun}\PYG{l+s+s1}{\PYGZsq{}}\PYG{p}{]} \PYG{k}{if} \PYG{n}{r}\PYG{p}{[}\PYG{l+s+s1}{\PYGZsq{}}\PYG{l+s+s1}{success}\PYG{l+s+s1}{\PYGZsq{}}\PYG{p}{]} \PYG{k}{else} \PYG{n}{np}\PYG{o}{.}\PYG{n}{nan} \PYG{k}{for} \PYG{n}{r} \PYG{o+ow}{in} \PYG{n}{res\PYGZus{}ef}\PYG{p}{]}\PYG{p}{)}
    \PYG{p}{\PYGZcb{}}
\PYG{p}{)}

\PYG{n}{ef}\PYG{o}{.}\PYG{n}{head}\PYG{p}{(}\PYG{p}{)}
\end{sphinxVerbatim}

\end{sphinxuseclass}\end{sphinxVerbatimInput}
\begin{sphinxVerbatimOutput}

\begin{sphinxuseclass}{cell_output}
\begin{sphinxVerbatim}[commandchars=\\\{\}]
    tret   tvol
0 0.0760 0.3864
1 0.1061 0.3609
2 0.1362 0.3388
3 0.1663 0.3207
4 0.1964 0.3073
\end{sphinxVerbatim}

\end{sphinxuseclass}\end{sphinxVerbatimOutput}

\end{sphinxuseclass}
\begin{sphinxuseclass}{cell}\begin{sphinxVerbatimInput}

\begin{sphinxuseclass}{cell_input}
\begin{sphinxVerbatim}[commandchars=\\\{\}]
\PYG{n}{ef}\PYG{o}{.}\PYG{n}{mul}\PYG{p}{(}\PYG{l+m+mi}{100}\PYG{p}{)}\PYG{o}{.}\PYG{n}{plot}\PYG{p}{(}\PYG{n}{x}\PYG{o}{=}\PYG{l+s+s1}{\PYGZsq{}}\PYG{l+s+s1}{tvol}\PYG{l+s+s1}{\PYGZsq{}}\PYG{p}{,} \PYG{n}{y}\PYG{o}{=}\PYG{l+s+s1}{\PYGZsq{}}\PYG{l+s+s1}{tret}\PYG{l+s+s1}{\PYGZsq{}}\PYG{p}{,} \PYG{n}{legend}\PYG{o}{=}\PYG{k+kc}{False}\PYG{p}{)}
\PYG{n}{plt}\PYG{o}{.}\PYG{n}{ylabel}\PYG{p}{(}\PYG{l+s+s1}{\PYGZsq{}}\PYG{l+s+s1}{Annualized Mean Return (}\PYG{l+s+s1}{\PYGZpc{}}\PYG{l+s+s1}{)}\PYG{l+s+s1}{\PYGZsq{}}\PYG{p}{)}
\PYG{n}{plt}\PYG{o}{.}\PYG{n}{xlabel}\PYG{p}{(}\PYG{l+s+s1}{\PYGZsq{}}\PYG{l+s+s1}{Annualized Volatility (}\PYG{l+s+s1}{\PYGZpc{}}\PYG{l+s+s1}{)}\PYG{l+s+s1}{\PYGZsq{}}\PYG{p}{)}
\PYG{n}{plt}\PYG{o}{.}\PYG{n}{title}\PYG{p}{(}
    \PYG{l+s+sa}{f}\PYG{l+s+s1}{\PYGZsq{}}\PYG{l+s+s1}{Efficient Frontier}\PYG{l+s+s1}{\PYGZsq{}} \PYG{o}{+}
    \PYG{l+s+sa}{f}\PYG{l+s+s1}{\PYGZsq{}}\PYG{l+s+se}{\PYGZbs{}n}\PYG{l+s+s1}{for }\PYG{l+s+si}{\PYGZob{}}\PYG{l+s+s2}{\PYGZdq{}}\PYG{l+s+s2}{, }\PYG{l+s+s2}{\PYGZdq{}}\PYG{o}{.}\PYG{n}{join}\PYG{p}{(}\PYG{n}{returns}\PYG{o}{.}\PYG{n}{columns}\PYG{p}{)}\PYG{l+s+si}{\PYGZcb{}}\PYG{l+s+s1}{\PYGZsq{}} \PYG{o}{+}
    \PYG{l+s+sa}{f}\PYG{l+s+s1}{\PYGZsq{}}\PYG{l+s+se}{\PYGZbs{}n}\PYG{l+s+s1}{from }\PYG{l+s+si}{\PYGZob{}}\PYG{n}{returns}\PYG{o}{.}\PYG{n}{index}\PYG{p}{[}\PYG{l+m+mi}{0}\PYG{p}{]}\PYG{l+s+si}{:}\PYG{l+s+s1}{\PYGZpc{}B \PYGZpc{}d, \PYGZpc{}Y}\PYG{l+s+si}{\PYGZcb{}}\PYG{l+s+s1}{ to }\PYG{l+s+si}{\PYGZob{}}\PYG{n}{returns}\PYG{o}{.}\PYG{n}{index}\PYG{p}{[}\PYG{o}{\PYGZhy{}}\PYG{l+m+mi}{1}\PYG{p}{]}\PYG{l+s+si}{:}\PYG{l+s+s1}{\PYGZpc{}B \PYGZpc{}d, \PYGZpc{}Y}\PYG{l+s+si}{\PYGZcb{}}\PYG{l+s+s1}{\PYGZsq{}}
\PYG{p}{)}

\PYG{k}{for} \PYG{n}{t}\PYG{p}{,} \PYG{n}{x}\PYG{p}{,} \PYG{n}{y} \PYG{o+ow}{in} \PYG{n+nb}{zip}\PYG{p}{(}
    \PYG{n}{returns}\PYG{o}{.}\PYG{n}{columns}\PYG{p}{,} 
    \PYG{n}{returns}\PYG{o}{.}\PYG{n}{std}\PYG{p}{(}\PYG{p}{)}\PYG{o}{.}\PYG{n}{mul}\PYG{p}{(}\PYG{l+m+mi}{100}\PYG{o}{*}\PYG{n}{np}\PYG{o}{.}\PYG{n}{sqrt}\PYG{p}{(}\PYG{l+m+mi}{252}\PYG{p}{)}\PYG{p}{)}\PYG{p}{,}
    \PYG{n}{returns}\PYG{o}{.}\PYG{n}{mean}\PYG{p}{(}\PYG{p}{)}\PYG{o}{.}\PYG{n}{mul}\PYG{p}{(}\PYG{l+m+mi}{100}\PYG{o}{*}\PYG{l+m+mi}{252}\PYG{p}{)}
\PYG{p}{)}\PYG{p}{:}
    \PYG{n}{plt}\PYG{o}{.}\PYG{n}{annotate}\PYG{p}{(}\PYG{n}{text}\PYG{o}{=}\PYG{n}{t}\PYG{p}{,} \PYG{n}{xy}\PYG{o}{=}\PYG{p}{(}\PYG{n}{x}\PYG{p}{,} \PYG{n}{y}\PYG{p}{)}\PYG{p}{)}
    
\PYG{n}{plt}\PYG{o}{.}\PYG{n}{show}\PYG{p}{(}\PYG{p}{)}
\end{sphinxVerbatim}

\end{sphinxuseclass}\end{sphinxVerbatimInput}
\begin{sphinxVerbatimOutput}

\begin{sphinxuseclass}{cell_output}
\noindent\sphinxincludegraphics{{7e347d0b73f704ed78b2f892bc9af615815f989047da46c70f384f666f59f176}.png}

\end{sphinxuseclass}\end{sphinxVerbatimOutput}

\end{sphinxuseclass}
\sphinxstepscope


\section{Herron Topic 4 \sphinxhyphen{} Practice (Blank)}
\label{\detokenize{herron_04_practice:herron-topic-4-practice-blank}}\label{\detokenize{herron_04_practice::doc}}

\subsection{Announcements}
\label{\detokenize{herron_04_practice:announcements}}

\subsection{Practice}
\label{\detokenize{herron_04_practice:practice}}

\subsubsection{Find the maximum Sharpe Ratio portfolio of MATANA stocks over the last three years}
\label{\detokenize{herron_04_practice:find-the-maximum-sharpe-ratio-portfolio-of-matana-stocks-over-the-last-three-years}}
\sphinxAtStartPar
Note that \sphinxcode{\sphinxupquote{sco.minimize()}} finds \sphinxstyleemphasis{minimums}, so you need to minimize the \sphinxstyleemphasis{negative} Sharpe Ratio.


\subsubsection{Find the maximum Sharpe Ratio portfolio of MATANA stocks over the last three years, but allow short weights up to 10\% on each stock}
\label{\detokenize{herron_04_practice:find-the-maximum-sharpe-ratio-portfolio-of-matana-stocks-over-the-last-three-years-but-allow-short-weights-up-to-10-on-each-stock}}

\subsubsection{Find the maximum Sharpe Ratio portfolio of MATANA stocks over the last three years, but allow total short weights of up to 30\%}
\label{\detokenize{herron_04_practice:find-the-maximum-sharpe-ratio-portfolio-of-matana-stocks-over-the-last-three-years-but-allow-total-short-weights-of-up-to-30}}

\subsubsection{Find the maximum Sharpe Ratio portfolio of MATANA stocks over the last three years, but do not allow any weight to exceed 30\% in magnitude}
\label{\detokenize{herron_04_practice:find-the-maximum-sharpe-ratio-portfolio-of-matana-stocks-over-the-last-three-years-but-do-not-allow-any-weight-to-exceed-30-in-magnitude}}

\subsubsection{Find the minimum 95\% Value at Risk (Var) portfolio of MATANA stocks over the last three years}
\label{\detokenize{herron_04_practice:find-the-minimum-95-value-at-risk-var-portfolio-of-matana-stocks-over-the-last-three-years}}
\sphinxAtStartPar
More on VaR \sphinxhref{https://en.wikipedia.org/wiki/Value\_at\_risk}{here}.


\subsubsection{Find the minimum maximum draw down portfolio of MATANA stocks over the last three years}
\label{\detokenize{herron_04_practice:find-the-minimum-maximum-draw-down-portfolio-of-matana-stocks-over-the-last-three-years}}

\subsubsection{Find the minimum maximum draw down portfolio with all available data for the current Dow\sphinxhyphen{}Jones Industrial Average (DJIA) stocks}
\label{\detokenize{herron_04_practice:find-the-minimum-maximum-draw-down-portfolio-with-all-available-data-for-the-current-dow-jones-industrial-average-djia-stocks}}
\sphinxAtStartPar
You can find the \sphinxhref{https://en.wikipedia.org/wiki/Dow\_Jones\_Industrial\_Average}{DJIA tickers on Wikipedia}.


\subsubsection{Plot the (mean\sphinxhyphen{}variance) efficient frontier with all available data for the current the DJIA stocks}
\label{\detokenize{herron_04_practice:plot-the-mean-variance-efficient-frontier-with-all-available-data-for-the-current-the-djia-stocks}}

\subsubsection{Find the maximum Sharpe Ratio portfolio with all available data for the current the DJIA stocks}
\label{\detokenize{herron_04_practice:find-the-maximum-sharpe-ratio-portfolio-with-all-available-data-for-the-current-the-djia-stocks}}

\subsubsection{Compare the \$\textbackslash{}frac\{1\}\{n\}\$ and maximum Sharpe Ratio portfolios with all available data for the current DJIA stocks}
\label{\detokenize{herron_04_practice:compare-the-frac-1-n-and-maximum-sharpe-ratio-portfolios-with-all-available-data-for-the-current-djia-stocks}}
\sphinxAtStartPar
Use all but the last 252 trading days to estimate the maximum Sharpe Ratio portfolio weights.
Then use the last 252 trading days of data to compare the \$\textbackslash{}frac\{1\}\{n\}\$  maximum Sharpe Ratio portfolios.

\sphinxstepscope


\section{Herron Topic 4 \sphinxhyphen{} Practice (Monday 2:45 PM, Section 3)}
\label{\detokenize{herron_04_practice_03:herron-topic-4-practice-monday-2-45-pm-section-3}}\label{\detokenize{herron_04_practice_03::doc}}

\subsection{Announcements}
\label{\detokenize{herron_04_practice_03:announcements}}\begin{itemize}
\item {} 
\sphinxAtStartPar
Quiz 6 this week
\begin{itemize}
\item {} 
\sphinxAtStartPar
I will post it at about 6 PM on Wednesday, 3/31

\item {} 
\sphinxAtStartPar
It will be due by 11:59 PM on Friday, 3/31

\end{itemize}

\item {} 
\sphinxAtStartPar
Please complete the week ten survey
\begin{itemize}
\item {} 
\sphinxAtStartPar
I am considering dropping a topic to allow more in\sphinxhyphen{}class group work and easier access to me

\item {} 
\sphinxAtStartPar
I am also curious why the quantitative courses are less popular this summer

\item {} 
\sphinxAtStartPar
Please complete by 11:59 PM on Friday, 3/31

\item {} 
\sphinxAtStartPar
\sphinxstyleemphasis{\sphinxstylestrong{The week ten survey is anonymous and voluntary}}

\end{itemize}

\item {} 
\sphinxAtStartPar
I will post project 2 as soon as I can

\item {} 
\sphinxAtStartPar
Assessment exam
\begin{itemize}
\item {} 
\sphinxAtStartPar
20 questions multiple on Canvas

\item {} 
\sphinxAtStartPar
\sphinxstyleemphasis{\sphinxstylestrong{You must be in the class room}}

\item {} 
\sphinxAtStartPar
No specific studying, but I suggest putting core course resources on your laptop (e.g., notes and PowerPoints)

\end{itemize}

\end{itemize}


\subsection{Practice}
\label{\detokenize{herron_04_practice_03:practice}}
\begin{sphinxuseclass}{cell}\begin{sphinxVerbatimInput}

\begin{sphinxuseclass}{cell_input}
\begin{sphinxVerbatim}[commandchars=\\\{\}]
\PYG{k+kn}{import} \PYG{n+nn}{matplotlib}\PYG{n+nn}{.}\PYG{n+nn}{pyplot} \PYG{k}{as} \PYG{n+nn}{plt}
\PYG{k+kn}{import} \PYG{n+nn}{numpy} \PYG{k}{as} \PYG{n+nn}{np}
\PYG{k+kn}{import} \PYG{n+nn}{pandas} \PYG{k}{as} \PYG{n+nn}{pd}
\end{sphinxVerbatim}

\end{sphinxuseclass}\end{sphinxVerbatimInput}

\end{sphinxuseclass}
\begin{sphinxuseclass}{cell}\begin{sphinxVerbatimInput}

\begin{sphinxuseclass}{cell_input}
\begin{sphinxVerbatim}[commandchars=\\\{\}]
\PYG{o}{\PYGZpc{}}\PYG{k}{config} InlineBackend.figure\PYGZus{}format = \PYGZsq{}retina\PYGZsq{}
\PYG{o}{\PYGZpc{}}\PYG{k}{precision} 4
\PYG{n}{pd}\PYG{o}{.}\PYG{n}{options}\PYG{o}{.}\PYG{n}{display}\PYG{o}{.}\PYG{n}{float\PYGZus{}format} \PYG{o}{=} \PYG{l+s+s1}{\PYGZsq{}}\PYG{l+s+si}{\PYGZob{}:.4f\PYGZcb{}}\PYG{l+s+s1}{\PYGZsq{}}\PYG{o}{.}\PYG{n}{format}
\end{sphinxVerbatim}

\end{sphinxuseclass}\end{sphinxVerbatimInput}

\end{sphinxuseclass}
\begin{sphinxuseclass}{cell}\begin{sphinxVerbatimInput}

\begin{sphinxuseclass}{cell_input}
\begin{sphinxVerbatim}[commandchars=\\\{\}]
\PYG{k+kn}{import} \PYG{n+nn}{yfinance} \PYG{k}{as} \PYG{n+nn}{yf}
\PYG{k+kn}{import} \PYG{n+nn}{pandas\PYGZus{}datareader} \PYG{k}{as} \PYG{n+nn}{pdr}
\PYG{k+kn}{import} \PYG{n+nn}{requests\PYGZus{}cache}
\PYG{n}{session} \PYG{o}{=} \PYG{n}{requests\PYGZus{}cache}\PYG{o}{.}\PYG{n}{CachedSession}\PYG{p}{(}\PYG{p}{)}
\end{sphinxVerbatim}

\end{sphinxuseclass}\end{sphinxVerbatimInput}

\end{sphinxuseclass}
\begin{sphinxuseclass}{cell}\begin{sphinxVerbatimInput}

\begin{sphinxuseclass}{cell_input}
\begin{sphinxVerbatim}[commandchars=\\\{\}]
\PYG{k+kn}{import} \PYG{n+nn}{scipy}\PYG{n+nn}{.}\PYG{n+nn}{optimize} \PYG{k}{as} \PYG{n+nn}{sco}
\end{sphinxVerbatim}

\end{sphinxuseclass}\end{sphinxVerbatimInput}

\end{sphinxuseclass}

\subsubsection{Find the maximum Sharpe Ratio portfolio of MATANA stocks over the last three years}
\label{\detokenize{herron_04_practice_03:find-the-maximum-sharpe-ratio-portfolio-of-matana-stocks-over-the-last-three-years}}
\sphinxAtStartPar
\sphinxstyleemphasis{\sphinxstylestrong{Note that \sphinxcode{\sphinxupquote{sco.minimize()}} finds minimums, so we need to minimize the negative Sharpe Ratio.}}

\sphinxAtStartPar
The following code downloads data for the MATANA stocks and assigns daily decimal returns from 2020 through 2022 to data frame \sphinxcode{\sphinxupquote{returns}}.
We will stop in 2022 to make it easier to compare our results, whether we use the risk\sphinxhyphen{}free rate or value\sphinxhyphen{}weighted market portfolio as our benchmark or not.
Recall, the Fama and French benchmark factors are only available with a lag, and are only available through December 2022 as I type.

\begin{sphinxuseclass}{cell}\begin{sphinxVerbatimInput}

\begin{sphinxuseclass}{cell_input}
\begin{sphinxVerbatim}[commandchars=\\\{\}]
\PYG{n}{tickers} \PYG{o}{=} \PYG{l+s+s1}{\PYGZsq{}}\PYG{l+s+s1}{MSFT AAPL TSLA AMZN NVDA GOOG}\PYG{l+s+s1}{\PYGZsq{}}

\PYG{n}{matana} \PYG{o}{=} \PYG{p}{(}
    \PYG{n}{yf}\PYG{o}{.}\PYG{n}{download}\PYG{p}{(}\PYG{n}{tickers}\PYG{o}{=}\PYG{n}{tickers}\PYG{p}{,} \PYG{n}{progress}\PYG{o}{=}\PYG{k+kc}{False}\PYG{p}{)}
    \PYG{o}{.}\PYG{n}{assign}\PYG{p}{(}\PYG{n}{Date}\PYG{o}{=}\PYG{k}{lambda} \PYG{n}{x}\PYG{p}{:} \PYG{n}{x}\PYG{o}{.}\PYG{n}{index}\PYG{o}{.}\PYG{n}{tz\PYGZus{}localize}\PYG{p}{(}\PYG{k+kc}{None}\PYG{p}{)}\PYG{p}{)}
    \PYG{o}{.}\PYG{n}{set\PYGZus{}index}\PYG{p}{(}\PYG{l+s+s1}{\PYGZsq{}}\PYG{l+s+s1}{Date}\PYG{l+s+s1}{\PYGZsq{}}\PYG{p}{)}
    \PYG{o}{.}\PYG{n}{rename\PYGZus{}axis}\PYG{p}{(}\PYG{n}{columns}\PYG{o}{=}\PYG{p}{[}\PYG{l+s+s1}{\PYGZsq{}}\PYG{l+s+s1}{Variable}\PYG{l+s+s1}{\PYGZsq{}}\PYG{p}{,} \PYG{l+s+s1}{\PYGZsq{}}\PYG{l+s+s1}{Ticker}\PYG{l+s+s1}{\PYGZsq{}}\PYG{p}{]}\PYG{p}{)}
\PYG{p}{)}

\PYG{n}{returns} \PYG{o}{=} \PYG{n}{matana}\PYG{p}{[}\PYG{l+s+s1}{\PYGZsq{}}\PYG{l+s+s1}{Adj Close}\PYG{l+s+s1}{\PYGZsq{}}\PYG{p}{]}\PYG{o}{.}\PYG{n}{pct\PYGZus{}change}\PYG{p}{(}\PYG{p}{)}\PYG{o}{.}\PYG{n}{loc}\PYG{p}{[}\PYG{l+s+s1}{\PYGZsq{}}\PYG{l+s+s1}{2020}\PYG{l+s+s1}{\PYGZsq{}}\PYG{p}{:}\PYG{l+s+s1}{\PYGZsq{}}\PYG{l+s+s1}{2022}\PYG{l+s+s1}{\PYGZsq{}}\PYG{p}{]}
\PYG{n}{returns}\PYG{o}{.}\PYG{n}{describe}\PYG{p}{(}\PYG{p}{)}
\end{sphinxVerbatim}

\end{sphinxuseclass}\end{sphinxVerbatimInput}
\begin{sphinxVerbatimOutput}

\begin{sphinxuseclass}{cell_output}
\begin{sphinxVerbatim}[commandchars=\\\{\}]
Ticker     AAPL     AMZN     GOOG     MSFT     NVDA     TSLA
count  756.0000 756.0000 756.0000 756.0000 756.0000 756.0000
mean     0.0011   0.0002   0.0006   0.0008   0.0018   0.0030
std      0.0233   0.0246   0.0217   0.0219   0.0352   0.0455
min     \PYGZhy{}0.1286  \PYGZhy{}0.1405  \PYGZhy{}0.1110  \PYGZhy{}0.1474  \PYGZhy{}0.1845  \PYGZhy{}0.2106
25\PYGZpc{}     \PYGZhy{}0.0110  \PYGZhy{}0.0129  \PYGZhy{}0.0097  \PYGZhy{}0.0097  \PYGZhy{}0.0177  \PYGZhy{}0.0215
50\PYGZpc{}      0.0006   0.0006   0.0012   0.0007   0.0029   0.0020
75\PYGZpc{}      0.0142   0.0123   0.0114   0.0123   0.0222   0.0251
max      0.1198   0.1354   0.0940   0.1422   0.1716   0.1989
\end{sphinxVerbatim}

\end{sphinxuseclass}\end{sphinxVerbatimOutput}

\end{sphinxuseclass}
\begin{sphinxuseclass}{cell}\begin{sphinxVerbatimInput}

\begin{sphinxuseclass}{cell_input}
\begin{sphinxVerbatim}[commandchars=\\\{\}]
\PYG{k}{def} \PYG{n+nf}{port\PYGZus{}sharpe}\PYG{p}{(}\PYG{n}{x}\PYG{p}{,} \PYG{n}{r}\PYG{p}{,} \PYG{n}{ppy}\PYG{p}{,} \PYG{n}{tgt}\PYG{p}{)}\PYG{p}{:}
    \PYG{l+s+sd}{\PYGZdq{}\PYGZdq{}\PYGZdq{}}
\PYG{l+s+sd}{    x: portfolio weights}
\PYG{l+s+sd}{    r: data frame of returns}
\PYG{l+s+sd}{    ppy: periods per year for annualization}
\PYG{l+s+sd}{    tgt: target or benchmark}
\PYG{l+s+sd}{    \PYGZdq{}\PYGZdq{}\PYGZdq{}}
    \PYG{n}{rp} \PYG{o}{=} \PYG{n}{r}\PYG{o}{.}\PYG{n}{dot}\PYG{p}{(}\PYG{n}{x}\PYG{p}{)} \PYG{c+c1}{\PYGZsh{} portfolio return}
    \PYG{n}{er} \PYG{o}{=} \PYG{n}{rp}\PYG{o}{.}\PYG{n}{sub}\PYG{p}{(}\PYG{n}{tgt}\PYG{p}{)}\PYG{o}{.}\PYG{n}{dropna}\PYG{p}{(}\PYG{p}{)} \PYG{c+c1}{\PYGZsh{} portfolio excess return}
    \PYG{k}{return} \PYG{n}{np}\PYG{o}{.}\PYG{n}{sqrt}\PYG{p}{(}\PYG{n}{ppy}\PYG{p}{)} \PYG{o}{*} \PYG{n}{er}\PYG{o}{.}\PYG{n}{mean}\PYG{p}{(}\PYG{p}{)} \PYG{o}{/} \PYG{n}{er}\PYG{o}{.}\PYG{n}{std}\PYG{p}{(}\PYG{p}{)} \PYG{c+c1}{\PYGZsh{} portfolio Sharpe Ratio}
\end{sphinxVerbatim}

\end{sphinxuseclass}\end{sphinxVerbatimInput}

\end{sphinxuseclass}
\begin{sphinxuseclass}{cell}\begin{sphinxVerbatimInput}

\begin{sphinxuseclass}{cell_input}
\begin{sphinxVerbatim}[commandchars=\\\{\}]
\PYG{k}{def} \PYG{n+nf}{port\PYGZus{}sharpe\PYGZus{}neg}\PYG{p}{(}\PYG{n}{x}\PYG{p}{,} \PYG{n}{r}\PYG{p}{,} \PYG{n}{ppy}\PYG{p}{,} \PYG{n}{tgt}\PYG{p}{)}\PYG{p}{:}
    \PYG{k}{return} \PYG{o}{\PYGZhy{}}\PYG{l+m+mi}{1} \PYG{o}{*} \PYG{n}{port\PYGZus{}sharpe}\PYG{p}{(}\PYG{n}{x}\PYG{p}{,} \PYG{n}{r}\PYG{p}{,} \PYG{n}{ppy}\PYG{p}{,} \PYG{n}{tgt}\PYG{p}{)}
\end{sphinxVerbatim}

\end{sphinxuseclass}\end{sphinxVerbatimInput}

\end{sphinxuseclass}
\begin{sphinxuseclass}{cell}\begin{sphinxVerbatimInput}

\begin{sphinxuseclass}{cell_input}
\begin{sphinxVerbatim}[commandchars=\\\{\}]
\PYG{n}{res\PYGZus{}sharpe\PYGZus{}1} \PYG{o}{=} \PYG{n}{sco}\PYG{o}{.}\PYG{n}{minimize}\PYG{p}{(}
    \PYG{n}{fun}\PYG{o}{=}\PYG{n}{port\PYGZus{}sharpe\PYGZus{}neg}\PYG{p}{,}
    \PYG{n}{x0}\PYG{o}{=}\PYG{n}{np}\PYG{o}{.}\PYG{n}{ones}\PYG{p}{(}\PYG{n}{returns}\PYG{o}{.}\PYG{n}{shape}\PYG{p}{[}\PYG{l+m+mi}{1}\PYG{p}{]}\PYG{p}{)} \PYG{o}{/} \PYG{n}{returns}\PYG{o}{.}\PYG{n}{shape}\PYG{p}{[}\PYG{l+m+mi}{1}\PYG{p}{]}\PYG{p}{,}
    \PYG{n}{args}\PYG{o}{=}\PYG{p}{(}\PYG{n}{returns}\PYG{p}{,} \PYG{l+m+mi}{252}\PYG{p}{,} \PYG{l+m+mi}{0}\PYG{p}{)}\PYG{p}{,}
    \PYG{n}{bounds}\PYG{o}{=}\PYG{p}{[}\PYG{p}{(}\PYG{l+m+mi}{0}\PYG{p}{,}\PYG{l+m+mi}{1}\PYG{p}{)} \PYG{k}{for} \PYG{n}{\PYGZus{}} \PYG{o+ow}{in} \PYG{n+nb}{range}\PYG{p}{(}\PYG{n}{returns}\PYG{o}{.}\PYG{n}{shape}\PYG{p}{[}\PYG{l+m+mi}{1}\PYG{p}{]}\PYG{p}{)}\PYG{p}{]}\PYG{p}{,}
    \PYG{n}{constraints}\PYG{o}{=}\PYG{p}{(}
        \PYG{p}{\PYGZob{}}\PYG{l+s+s1}{\PYGZsq{}}\PYG{l+s+s1}{type}\PYG{l+s+s1}{\PYGZsq{}}\PYG{p}{:} \PYG{l+s+s1}{\PYGZsq{}}\PYG{l+s+s1}{eq}\PYG{l+s+s1}{\PYGZsq{}}\PYG{p}{,} \PYG{l+s+s1}{\PYGZsq{}}\PYG{l+s+s1}{fun}\PYG{l+s+s1}{\PYGZsq{}}\PYG{p}{:} \PYG{k}{lambda} \PYG{n}{x}\PYG{p}{:} \PYG{n}{x}\PYG{o}{.}\PYG{n}{sum}\PYG{p}{(}\PYG{p}{)} \PYG{o}{\PYGZhy{}} \PYG{l+m+mi}{1}\PYG{p}{\PYGZcb{}} \PYG{c+c1}{\PYGZsh{} want eq constraint to = 0}
    \PYG{p}{)}
\PYG{p}{)}

\PYG{n}{res\PYGZus{}sharpe\PYGZus{}1}
\end{sphinxVerbatim}

\end{sphinxuseclass}\end{sphinxVerbatimInput}
\begin{sphinxVerbatimOutput}

\begin{sphinxuseclass}{cell_output}
\begin{sphinxVerbatim}[commandchars=\\\{\}]
     fun: \PYGZhy{}1.0891990698266503
     jac: array([ 1.1642e\PYGZhy{}04,  3.6243e\PYGZhy{}01,  1.1961e\PYGZhy{}01,  6.4995e\PYGZhy{}02,  4.9071e\PYGZhy{}04,
       \PYGZhy{}2.5976e\PYGZhy{}04])
 message: \PYGZsq{}Optimization terminated successfully\PYGZsq{}
    nfev: 63
     nit: 9
    njev: 9
  status: 0
 success: True
       x: array([6.6102e\PYGZhy{}02, 8.1640e\PYGZhy{}17, 7.1991e\PYGZhy{}17, 0.0000e+00, 3.1307e\PYGZhy{}01,
       6.2083e\PYGZhy{}01])
\end{sphinxVerbatim}

\end{sphinxuseclass}\end{sphinxVerbatimOutput}

\end{sphinxuseclass}
\begin{sphinxuseclass}{cell}\begin{sphinxVerbatimInput}

\begin{sphinxuseclass}{cell_input}
\begin{sphinxVerbatim}[commandchars=\\\{\}]
\PYG{n}{port\PYGZus{}sharpe}\PYG{p}{(}\PYG{n}{x}\PYG{o}{=}\PYG{n}{res\PYGZus{}sharpe\PYGZus{}1}\PYG{p}{[}\PYG{l+s+s1}{\PYGZsq{}}\PYG{l+s+s1}{x}\PYG{l+s+s1}{\PYGZsq{}}\PYG{p}{]}\PYG{p}{,} \PYG{n}{r}\PYG{o}{=}\PYG{n}{returns}\PYG{p}{,} \PYG{n}{ppy}\PYG{o}{=}\PYG{l+m+mi}{252}\PYG{p}{,} \PYG{n}{tgt}\PYG{o}{=}\PYG{l+m+mi}{0}\PYG{p}{)}
\end{sphinxVerbatim}

\end{sphinxuseclass}\end{sphinxVerbatimInput}
\begin{sphinxVerbatimOutput}

\begin{sphinxuseclass}{cell_output}
\begin{sphinxVerbatim}[commandchars=\\\{\}]
1.0891990698266503
\end{sphinxVerbatim}

\end{sphinxuseclass}\end{sphinxVerbatimOutput}

\end{sphinxuseclass}

\subsubsection{Find the maximum Sharpe Ratio portfolio of MATANA stocks over the last three years, but allow short weights up to 10\% on each stock}
\label{\detokenize{herron_04_practice_03:find-the-maximum-sharpe-ratio-portfolio-of-matana-stocks-over-the-last-three-years-but-allow-short-weights-up-to-10-on-each-stock}}
\begin{sphinxuseclass}{cell}\begin{sphinxVerbatimInput}

\begin{sphinxuseclass}{cell_input}
\begin{sphinxVerbatim}[commandchars=\\\{\}]
\PYG{n}{res\PYGZus{}sharpe\PYGZus{}2} \PYG{o}{=} \PYG{n}{sco}\PYG{o}{.}\PYG{n}{minimize}\PYG{p}{(}
    \PYG{n}{fun}\PYG{o}{=}\PYG{n}{port\PYGZus{}sharpe\PYGZus{}neg}\PYG{p}{,}
    \PYG{n}{x0}\PYG{o}{=}\PYG{n}{np}\PYG{o}{.}\PYG{n}{ones}\PYG{p}{(}\PYG{n}{returns}\PYG{o}{.}\PYG{n}{shape}\PYG{p}{[}\PYG{l+m+mi}{1}\PYG{p}{]}\PYG{p}{)} \PYG{o}{/} \PYG{n}{returns}\PYG{o}{.}\PYG{n}{shape}\PYG{p}{[}\PYG{l+m+mi}{1}\PYG{p}{]}\PYG{p}{,}
    \PYG{n}{args}\PYG{o}{=}\PYG{p}{(}\PYG{n}{returns}\PYG{p}{,} \PYG{l+m+mi}{252}\PYG{p}{,} \PYG{l+m+mi}{0}\PYG{p}{)}\PYG{p}{,}
    \PYG{n}{bounds}\PYG{o}{=}\PYG{p}{[}\PYG{p}{(}\PYG{o}{\PYGZhy{}}\PYG{l+m+mf}{.1}\PYG{p}{,}\PYG{l+m+mf}{1.5}\PYG{p}{)} \PYG{k}{for} \PYG{n}{\PYGZus{}} \PYG{o+ow}{in} \PYG{n+nb}{range}\PYG{p}{(}\PYG{n}{returns}\PYG{o}{.}\PYG{n}{shape}\PYG{p}{[}\PYG{l+m+mi}{1}\PYG{p}{]}\PYG{p}{)}\PYG{p}{]}\PYG{p}{,}
    \PYG{n}{constraints}\PYG{o}{=}\PYG{p}{(}
        \PYG{p}{\PYGZob{}}\PYG{l+s+s1}{\PYGZsq{}}\PYG{l+s+s1}{type}\PYG{l+s+s1}{\PYGZsq{}}\PYG{p}{:} \PYG{l+s+s1}{\PYGZsq{}}\PYG{l+s+s1}{eq}\PYG{l+s+s1}{\PYGZsq{}}\PYG{p}{,} \PYG{l+s+s1}{\PYGZsq{}}\PYG{l+s+s1}{fun}\PYG{l+s+s1}{\PYGZsq{}}\PYG{p}{:} \PYG{k}{lambda} \PYG{n}{x}\PYG{p}{:} \PYG{n}{x}\PYG{o}{.}\PYG{n}{sum}\PYG{p}{(}\PYG{p}{)} \PYG{o}{\PYGZhy{}} \PYG{l+m+mi}{1}\PYG{p}{\PYGZcb{}} \PYG{c+c1}{\PYGZsh{} want eq constraint to = 0}
    \PYG{p}{)}
\PYG{p}{)}

\PYG{n}{res\PYGZus{}sharpe\PYGZus{}2}
\end{sphinxVerbatim}

\end{sphinxuseclass}\end{sphinxVerbatimInput}
\begin{sphinxVerbatimOutput}

\begin{sphinxuseclass}{cell_output}
\begin{sphinxVerbatim}[commandchars=\\\{\}]
     fun: \PYGZhy{}1.1368990608896157
     jac: array([0.041 , 0.3444, 0.1157, 0.0727, 0.041 , 0.041 ])
 message: \PYGZsq{}Optimization terminated successfully\PYGZsq{}
    nfev: 63
     nit: 9
    njev: 9
  status: 0
 success: True
       x: array([ 0.2997, \PYGZhy{}0.1   , \PYGZhy{}0.1   , \PYGZhy{}0.1   ,  0.3745,  0.6258])
\end{sphinxVerbatim}

\end{sphinxuseclass}\end{sphinxVerbatimOutput}

\end{sphinxuseclass}
\begin{sphinxuseclass}{cell}\begin{sphinxVerbatimInput}

\begin{sphinxuseclass}{cell_input}
\begin{sphinxVerbatim}[commandchars=\\\{\}]
\PYG{p}{(}
    \PYG{n}{pd}\PYG{o}{.}\PYG{n}{DataFrame}\PYG{p}{(}
        \PYG{n}{data}\PYG{o}{=}\PYG{p}{\PYGZob{}}
            \PYG{l+s+s1}{\PYGZsq{}}\PYG{l+s+s1}{Long Only}\PYG{l+s+s1}{\PYGZsq{}}\PYG{p}{:}\PYG{n}{res\PYGZus{}sharpe\PYGZus{}1}\PYG{p}{[}\PYG{l+s+s1}{\PYGZsq{}}\PYG{l+s+s1}{x}\PYG{l+s+s1}{\PYGZsq{}}\PYG{p}{]}\PYG{p}{,} 
            \PYG{l+s+s1}{\PYGZsq{}}\PYG{l+s+s1}{Up to 10}\PYG{l+s+s1}{\PYGZpc{}}\PYG{l+s+s1}{ Short per Stock}\PYG{l+s+s1}{\PYGZsq{}}\PYG{p}{:}\PYG{n}{res\PYGZus{}sharpe\PYGZus{}2}\PYG{p}{[}\PYG{l+s+s1}{\PYGZsq{}}\PYG{l+s+s1}{x}\PYG{l+s+s1}{\PYGZsq{}}\PYG{p}{]}
        \PYG{p}{\PYGZcb{}}\PYG{p}{,}
        \PYG{n}{index}\PYG{o}{=}\PYG{n}{returns}\PYG{o}{.}\PYG{n}{columns}
    \PYG{p}{)}
    \PYG{o}{.}\PYG{n}{rename\PYGZus{}axis}\PYG{p}{(}\PYG{l+s+s1}{\PYGZsq{}}\PYG{l+s+s1}{Portfolio Weight}\PYG{l+s+s1}{\PYGZsq{}}\PYG{p}{)}
    \PYG{o}{.}\PYG{n}{plot}\PYG{p}{(}\PYG{n}{kind}\PYG{o}{=}\PYG{l+s+s1}{\PYGZsq{}}\PYG{l+s+s1}{barh}\PYG{l+s+s1}{\PYGZsq{}}\PYG{p}{)}
\PYG{p}{)}
\PYG{n}{plt}\PYG{o}{.}\PYG{n}{title}\PYG{p}{(}\PYG{l+s+s1}{\PYGZsq{}}\PYG{l+s+s1}{Comparison Max. Sharpe Ratio Portfolio Weights}\PYG{l+s+s1}{\PYGZsq{}}\PYG{p}{)}
\PYG{n}{plt}\PYG{o}{.}\PYG{n}{show}\PYG{p}{(}\PYG{p}{)}
\end{sphinxVerbatim}

\end{sphinxuseclass}\end{sphinxVerbatimInput}
\begin{sphinxVerbatimOutput}

\begin{sphinxuseclass}{cell_output}
\noindent\sphinxincludegraphics{{004163d333021c4843934510f0bca8e93ff174a2d8ca8aae95ac07868ddfaaea}.png}

\end{sphinxuseclass}\end{sphinxVerbatimOutput}

\end{sphinxuseclass}
\sphinxAtStartPar
By relaxing the long\sphinxhyphen{}only constrain (via changes to \sphinxcode{\sphinxupquote{bounds=}}), the weights on AMZN, GOOG, and MSFT go from zero to \sphinxhyphen{}10\%.
Also, the Sharpe Ratio increases because we relax a binding constraint.

\begin{sphinxuseclass}{cell}\begin{sphinxVerbatimInput}

\begin{sphinxuseclass}{cell_input}
\begin{sphinxVerbatim}[commandchars=\\\{\}]
\PYG{n}{port\PYGZus{}sharpe}\PYG{p}{(}\PYG{n}{res\PYGZus{}sharpe\PYGZus{}1}\PYG{p}{[}\PYG{l+s+s1}{\PYGZsq{}}\PYG{l+s+s1}{x}\PYG{l+s+s1}{\PYGZsq{}}\PYG{p}{]}\PYG{p}{,} \PYG{n}{r}\PYG{o}{=}\PYG{n}{returns}\PYG{p}{,} \PYG{n}{ppy}\PYG{o}{=}\PYG{l+m+mi}{252}\PYG{p}{,} \PYG{n}{tgt}\PYG{o}{=}\PYG{l+m+mi}{0}\PYG{p}{)}
\end{sphinxVerbatim}

\end{sphinxuseclass}\end{sphinxVerbatimInput}
\begin{sphinxVerbatimOutput}

\begin{sphinxuseclass}{cell_output}
\begin{sphinxVerbatim}[commandchars=\\\{\}]
1.0891990698266503
\end{sphinxVerbatim}

\end{sphinxuseclass}\end{sphinxVerbatimOutput}

\end{sphinxuseclass}
\begin{sphinxuseclass}{cell}\begin{sphinxVerbatimInput}

\begin{sphinxuseclass}{cell_input}
\begin{sphinxVerbatim}[commandchars=\\\{\}]
\PYG{n}{port\PYGZus{}sharpe}\PYG{p}{(}\PYG{n}{res\PYGZus{}sharpe\PYGZus{}2}\PYG{p}{[}\PYG{l+s+s1}{\PYGZsq{}}\PYG{l+s+s1}{x}\PYG{l+s+s1}{\PYGZsq{}}\PYG{p}{]}\PYG{p}{,} \PYG{n}{r}\PYG{o}{=}\PYG{n}{returns}\PYG{p}{,} \PYG{n}{ppy}\PYG{o}{=}\PYG{l+m+mi}{252}\PYG{p}{,} \PYG{n}{tgt}\PYG{o}{=}\PYG{l+m+mi}{0}\PYG{p}{)}
\end{sphinxVerbatim}

\end{sphinxuseclass}\end{sphinxVerbatimInput}
\begin{sphinxVerbatimOutput}

\begin{sphinxuseclass}{cell_output}
\begin{sphinxVerbatim}[commandchars=\\\{\}]
1.1368990608896157
\end{sphinxVerbatim}

\end{sphinxuseclass}\end{sphinxVerbatimOutput}

\end{sphinxuseclass}

\subsubsection{Find the maximum Sharpe Ratio portfolio of MATANA stocks over the last three years, but allow total short weights of up to 30\%}
\label{\detokenize{herron_04_practice_03:find-the-maximum-sharpe-ratio-portfolio-of-matana-stocks-over-the-last-three-years-but-allow-total-short-weights-of-up-to-30}}
\sphinxAtStartPar
We can find the negative values in a NumPy array as follows.

\begin{sphinxuseclass}{cell}\begin{sphinxVerbatimInput}

\begin{sphinxuseclass}{cell_input}
\begin{sphinxVerbatim}[commandchars=\\\{\}]
\PYG{n}{x} \PYG{o}{=} \PYG{n}{np}\PYG{o}{.}\PYG{n}{arange}\PYG{p}{(}\PYG{l+m+mi}{10}\PYG{p}{)} \PYG{o}{\PYGZhy{}} \PYG{l+m+mi}{5}
\PYG{n}{x\PYGZus{}neg} \PYG{o}{=} \PYG{n}{x}\PYG{p}{[}\PYG{n}{x} \PYG{o}{\PYGZlt{}} \PYG{l+m+mi}{0}\PYG{p}{]}
\PYG{n+nb}{print}\PYG{p}{(}\PYG{l+s+sa}{f}\PYG{l+s+s1}{\PYGZsq{}}\PYG{l+s+s1}{All Values: }\PYG{l+s+si}{\PYGZob{}}\PYG{n}{x}\PYG{l+s+si}{\PYGZcb{}}\PYG{l+s+se}{\PYGZbs{}n}\PYG{l+s+s1}{Negative Values: }\PYG{l+s+si}{\PYGZob{}}\PYG{n}{x\PYGZus{}neg}\PYG{l+s+si}{\PYGZcb{}}\PYG{l+s+s1}{\PYGZsq{}}\PYG{p}{)}
\end{sphinxVerbatim}

\end{sphinxuseclass}\end{sphinxVerbatimInput}
\begin{sphinxVerbatimOutput}

\begin{sphinxuseclass}{cell_output}
\begin{sphinxVerbatim}[commandchars=\\\{\}]
All Values: [\PYGZhy{}5 \PYGZhy{}4 \PYGZhy{}3 \PYGZhy{}2 \PYGZhy{}1  0  1  2  3  4]
Negative Values: [\PYGZhy{}5 \PYGZhy{}4 \PYGZhy{}3 \PYGZhy{}2 \PYGZhy{}1]
\end{sphinxVerbatim}

\end{sphinxuseclass}\end{sphinxVerbatimOutput}

\end{sphinxuseclass}
\begin{sphinxuseclass}{cell}\begin{sphinxVerbatimInput}

\begin{sphinxuseclass}{cell_input}
\begin{sphinxVerbatim}[commandchars=\\\{\}]
\PYG{n}{res\PYGZus{}sharpe\PYGZus{}3} \PYG{o}{=} \PYG{n}{sco}\PYG{o}{.}\PYG{n}{minimize}\PYG{p}{(}
    \PYG{n}{fun}\PYG{o}{=}\PYG{n}{port\PYGZus{}sharpe\PYGZus{}neg}\PYG{p}{,}
    \PYG{n}{x0}\PYG{o}{=}\PYG{n}{np}\PYG{o}{.}\PYG{n}{ones}\PYG{p}{(}\PYG{n}{returns}\PYG{o}{.}\PYG{n}{shape}\PYG{p}{[}\PYG{l+m+mi}{1}\PYG{p}{]}\PYG{p}{)} \PYG{o}{/} \PYG{n}{returns}\PYG{o}{.}\PYG{n}{shape}\PYG{p}{[}\PYG{l+m+mi}{1}\PYG{p}{]}\PYG{p}{,}
    \PYG{n}{args}\PYG{o}{=}\PYG{p}{(}\PYG{n}{returns}\PYG{p}{,} \PYG{l+m+mi}{252}\PYG{p}{,} \PYG{l+m+mi}{0}\PYG{p}{)}\PYG{p}{,}
    \PYG{n}{bounds}\PYG{o}{=}\PYG{p}{[}\PYG{p}{(}\PYG{o}{\PYGZhy{}}\PYG{l+m+mf}{0.3}\PYG{p}{,}\PYG{l+m+mf}{1.3}\PYG{p}{)} \PYG{k}{for} \PYG{n}{\PYGZus{}} \PYG{o+ow}{in} \PYG{n+nb}{range}\PYG{p}{(}\PYG{n}{returns}\PYG{o}{.}\PYG{n}{shape}\PYG{p}{[}\PYG{l+m+mi}{1}\PYG{p}{]}\PYG{p}{)}\PYG{p}{]}\PYG{p}{,}
    \PYG{n}{constraints}\PYG{o}{=}\PYG{p}{(}
        \PYG{p}{\PYGZob{}}\PYG{l+s+s1}{\PYGZsq{}}\PYG{l+s+s1}{type}\PYG{l+s+s1}{\PYGZsq{}}\PYG{p}{:} \PYG{l+s+s1}{\PYGZsq{}}\PYG{l+s+s1}{eq}\PYG{l+s+s1}{\PYGZsq{}}\PYG{p}{,} \PYG{l+s+s1}{\PYGZsq{}}\PYG{l+s+s1}{fun}\PYG{l+s+s1}{\PYGZsq{}}\PYG{p}{:} \PYG{k}{lambda} \PYG{n}{x}\PYG{p}{:} \PYG{n}{x}\PYG{o}{.}\PYG{n}{sum}\PYG{p}{(}\PYG{p}{)} \PYG{o}{\PYGZhy{}} \PYG{l+m+mi}{1}\PYG{p}{\PYGZcb{}}\PYG{p}{,} \PYG{c+c1}{\PYGZsh{} want eq constraint to = 0}
        \PYG{p}{\PYGZob{}}\PYG{l+s+s1}{\PYGZsq{}}\PYG{l+s+s1}{type}\PYG{l+s+s1}{\PYGZsq{}}\PYG{p}{:} \PYG{l+s+s1}{\PYGZsq{}}\PYG{l+s+s1}{ineq}\PYG{l+s+s1}{\PYGZsq{}}\PYG{p}{,} \PYG{l+s+s1}{\PYGZsq{}}\PYG{l+s+s1}{fun}\PYG{l+s+s1}{\PYGZsq{}}\PYG{p}{:} \PYG{k}{lambda} \PYG{n}{x}\PYG{p}{:} \PYG{n}{x}\PYG{p}{[}\PYG{n}{x} \PYG{o}{\PYGZlt{}} \PYG{l+m+mi}{0}\PYG{p}{]}\PYG{o}{.}\PYG{n}{sum}\PYG{p}{(}\PYG{p}{)} \PYG{o}{+} \PYG{l+m+mf}{0.3}\PYG{p}{\PYGZcb{}} \PYG{c+c1}{\PYGZsh{} want ineq constraint to \PYGZgt{}= 0}
    \PYG{p}{)}
\PYG{p}{)}

\PYG{n}{res\PYGZus{}sharpe\PYGZus{}3}
\end{sphinxVerbatim}

\end{sphinxuseclass}\end{sphinxVerbatimInput}
\begin{sphinxVerbatimOutput}

\begin{sphinxuseclass}{cell_output}
\begin{sphinxVerbatim}[commandchars=\\\{\}]
     fun: \PYGZhy{}1.1826812160909803
     jac: array([0.0782, 0.336 , 0.1519, 0.1096, 0.0773, 0.0773])
 message: \PYGZsq{}Optimization terminated successfully\PYGZsq{}
    nfev: 100
     nit: 10
    njev: 10
  status: 0
 success: True
       x: array([ 3.5659e\PYGZhy{}01, \PYGZhy{}3.0000e\PYGZhy{}01, \PYGZhy{}7.2559e\PYGZhy{}10,  1.7959e\PYGZhy{}05,  3.4963e\PYGZhy{}01,
        5.9376e\PYGZhy{}01])
\end{sphinxVerbatim}

\end{sphinxuseclass}\end{sphinxVerbatimOutput}

\end{sphinxuseclass}
\begin{sphinxuseclass}{cell}\begin{sphinxVerbatimInput}

\begin{sphinxuseclass}{cell_input}
\begin{sphinxVerbatim}[commandchars=\\\{\}]
\PYG{p}{(}
    \PYG{n}{pd}\PYG{o}{.}\PYG{n}{DataFrame}\PYG{p}{(}
        \PYG{n}{data}\PYG{o}{=}\PYG{p}{\PYGZob{}}
            \PYG{l+s+s1}{\PYGZsq{}}\PYG{l+s+s1}{Long Only}\PYG{l+s+s1}{\PYGZsq{}}\PYG{p}{:}\PYG{n}{res\PYGZus{}sharpe\PYGZus{}1}\PYG{p}{[}\PYG{l+s+s1}{\PYGZsq{}}\PYG{l+s+s1}{x}\PYG{l+s+s1}{\PYGZsq{}}\PYG{p}{]}\PYG{p}{,} 
            \PYG{l+s+s1}{\PYGZsq{}}\PYG{l+s+s1}{Up to 30}\PYG{l+s+s1}{\PYGZpc{}}\PYG{l+s+s1}{ Short Total}\PYG{l+s+s1}{\PYGZsq{}}\PYG{p}{:}\PYG{n}{res\PYGZus{}sharpe\PYGZus{}2}\PYG{p}{[}\PYG{l+s+s1}{\PYGZsq{}}\PYG{l+s+s1}{x}\PYG{l+s+s1}{\PYGZsq{}}\PYG{p}{]}
        \PYG{p}{\PYGZcb{}}\PYG{p}{,}
        \PYG{n}{index}\PYG{o}{=}\PYG{n}{returns}\PYG{o}{.}\PYG{n}{columns}
    \PYG{p}{)}
    \PYG{o}{.}\PYG{n}{rename\PYGZus{}axis}\PYG{p}{(}\PYG{l+s+s1}{\PYGZsq{}}\PYG{l+s+s1}{Portfolio Weight}\PYG{l+s+s1}{\PYGZsq{}}\PYG{p}{)}
    \PYG{o}{.}\PYG{n}{plot}\PYG{p}{(}\PYG{n}{kind}\PYG{o}{=}\PYG{l+s+s1}{\PYGZsq{}}\PYG{l+s+s1}{barh}\PYG{l+s+s1}{\PYGZsq{}}\PYG{p}{)}
\PYG{p}{)}
\PYG{n}{plt}\PYG{o}{.}\PYG{n}{title}\PYG{p}{(}\PYG{l+s+s1}{\PYGZsq{}}\PYG{l+s+s1}{Comparison Max. Sharpe Ratio Portfolios}\PYG{l+s+s1}{\PYGZsq{}}\PYG{p}{)}
\PYG{n}{plt}\PYG{o}{.}\PYG{n}{show}\PYG{p}{(}\PYG{p}{)}
\end{sphinxVerbatim}

\end{sphinxuseclass}\end{sphinxVerbatimInput}
\begin{sphinxVerbatimOutput}

\begin{sphinxuseclass}{cell_output}
\noindent\sphinxincludegraphics{{9191ee6b277076599561ec2ccfc4e2c05a524e1504c24b98f5dfcbe9c357f2a1}.png}

\end{sphinxuseclass}\end{sphinxVerbatimOutput}

\end{sphinxuseclass}
\sphinxAtStartPar
Again, by relaxing the long\sphinxhyphen{}only constrain, the weights on AMZN, GOOG, and MSFT go from zero to \sphinxhyphen{}10\%.
Also, the Sharpe Ratio increases because we relax a binding constraints.
The Sharpe Ratio is higher here than in the previous exercise, but this will not always be the case, since we relax different constraints here and in the previous exercise.

\begin{sphinxuseclass}{cell}\begin{sphinxVerbatimInput}

\begin{sphinxuseclass}{cell_input}
\begin{sphinxVerbatim}[commandchars=\\\{\}]
\PYG{n}{port\PYGZus{}sharpe}\PYG{p}{(}\PYG{n}{res\PYGZus{}sharpe\PYGZus{}1}\PYG{p}{[}\PYG{l+s+s1}{\PYGZsq{}}\PYG{l+s+s1}{x}\PYG{l+s+s1}{\PYGZsq{}}\PYG{p}{]}\PYG{p}{,} \PYG{n}{r}\PYG{o}{=}\PYG{n}{returns}\PYG{p}{,} \PYG{n}{ppy}\PYG{o}{=}\PYG{l+m+mi}{252}\PYG{p}{,} \PYG{n}{tgt}\PYG{o}{=}\PYG{l+m+mi}{0}\PYG{p}{)}
\end{sphinxVerbatim}

\end{sphinxuseclass}\end{sphinxVerbatimInput}
\begin{sphinxVerbatimOutput}

\begin{sphinxuseclass}{cell_output}
\begin{sphinxVerbatim}[commandchars=\\\{\}]
1.0891990698266503
\end{sphinxVerbatim}

\end{sphinxuseclass}\end{sphinxVerbatimOutput}

\end{sphinxuseclass}
\begin{sphinxuseclass}{cell}\begin{sphinxVerbatimInput}

\begin{sphinxuseclass}{cell_input}
\begin{sphinxVerbatim}[commandchars=\\\{\}]
\PYG{n}{port\PYGZus{}sharpe}\PYG{p}{(}\PYG{n}{res\PYGZus{}sharpe\PYGZus{}3}\PYG{p}{[}\PYG{l+s+s1}{\PYGZsq{}}\PYG{l+s+s1}{x}\PYG{l+s+s1}{\PYGZsq{}}\PYG{p}{]}\PYG{p}{,} \PYG{n}{r}\PYG{o}{=}\PYG{n}{returns}\PYG{p}{,} \PYG{n}{ppy}\PYG{o}{=}\PYG{l+m+mi}{252}\PYG{p}{,} \PYG{n}{tgt}\PYG{o}{=}\PYG{l+m+mi}{0}\PYG{p}{)}
\end{sphinxVerbatim}

\end{sphinxuseclass}\end{sphinxVerbatimInput}
\begin{sphinxVerbatimOutput}

\begin{sphinxuseclass}{cell_output}
\begin{sphinxVerbatim}[commandchars=\\\{\}]
1.1826812160909803
\end{sphinxVerbatim}

\end{sphinxuseclass}\end{sphinxVerbatimOutput}

\end{sphinxuseclass}

\subsubsection{Find the maximum Sharpe Ratio portfolio of MATANA stocks over the last three years, but do not allow any weight to exceed 30\% in magnitude}
\label{\detokenize{herron_04_practice_03:find-the-maximum-sharpe-ratio-portfolio-of-matana-stocks-over-the-last-three-years-but-do-not-allow-any-weight-to-exceed-30-in-magnitude}}
\sphinxAtStartPar
We can do this easily with \sphinxcode{\sphinxupquote{bounds=}}.

\begin{sphinxuseclass}{cell}\begin{sphinxVerbatimInput}

\begin{sphinxuseclass}{cell_input}
\begin{sphinxVerbatim}[commandchars=\\\{\}]
\PYG{n}{res\PYGZus{}sharpe\PYGZus{}4} \PYG{o}{=} \PYG{n}{sco}\PYG{o}{.}\PYG{n}{minimize}\PYG{p}{(}
    \PYG{n}{fun}\PYG{o}{=}\PYG{n}{port\PYGZus{}sharpe\PYGZus{}neg}\PYG{p}{,}
    \PYG{n}{x0}\PYG{o}{=}\PYG{n}{np}\PYG{o}{.}\PYG{n}{ones}\PYG{p}{(}\PYG{n}{returns}\PYG{o}{.}\PYG{n}{shape}\PYG{p}{[}\PYG{l+m+mi}{1}\PYG{p}{]}\PYG{p}{)} \PYG{o}{/} \PYG{n}{returns}\PYG{o}{.}\PYG{n}{shape}\PYG{p}{[}\PYG{l+m+mi}{1}\PYG{p}{]}\PYG{p}{,}
    \PYG{n}{args}\PYG{o}{=}\PYG{p}{(}\PYG{n}{returns}\PYG{p}{,} \PYG{l+m+mi}{252}\PYG{p}{,} \PYG{l+m+mi}{0}\PYG{p}{)}\PYG{p}{,}
    \PYG{n}{tol}\PYG{o}{=}\PYG{l+m+mf}{1e\PYGZhy{}6}\PYG{p}{,}
    \PYG{n}{bounds}\PYG{o}{=}\PYG{p}{[}\PYG{p}{(}\PYG{l+m+mi}{0}\PYG{p}{,}\PYG{l+m+mf}{0.3}\PYG{p}{)} \PYG{k}{for} \PYG{n}{\PYGZus{}} \PYG{o+ow}{in} \PYG{n+nb}{range}\PYG{p}{(}\PYG{n}{returns}\PYG{o}{.}\PYG{n}{shape}\PYG{p}{[}\PYG{l+m+mi}{1}\PYG{p}{]}\PYG{p}{)}\PYG{p}{]}\PYG{p}{,}
    \PYG{n}{constraints}\PYG{o}{=}\PYG{p}{(}
        \PYG{p}{\PYGZob{}}\PYG{l+s+s1}{\PYGZsq{}}\PYG{l+s+s1}{type}\PYG{l+s+s1}{\PYGZsq{}}\PYG{p}{:} \PYG{l+s+s1}{\PYGZsq{}}\PYG{l+s+s1}{eq}\PYG{l+s+s1}{\PYGZsq{}}\PYG{p}{,} \PYG{l+s+s1}{\PYGZsq{}}\PYG{l+s+s1}{fun}\PYG{l+s+s1}{\PYGZsq{}}\PYG{p}{:} \PYG{k}{lambda} \PYG{n}{x}\PYG{p}{:} \PYG{n}{x}\PYG{o}{.}\PYG{n}{sum}\PYG{p}{(}\PYG{p}{)} \PYG{o}{\PYGZhy{}} \PYG{l+m+mi}{1}\PYG{p}{\PYGZcb{}}\PYG{p}{,} \PYG{c+c1}{\PYGZsh{} want eq constraint to = 0}
    \PYG{p}{)}
\PYG{p}{)}

\PYG{n}{res\PYGZus{}sharpe\PYGZus{}4}
\end{sphinxVerbatim}

\end{sphinxuseclass}\end{sphinxVerbatimInput}
\begin{sphinxVerbatimOutput}

\begin{sphinxuseclass}{cell_output}
\begin{sphinxVerbatim}[commandchars=\\\{\}]
     fun: \PYGZhy{}1.0342974862116496
     jac: array([ 0.1105,  0.5249,  0.2314,  0.176 ,  0.0964, \PYGZhy{}0.2655])
 message: \PYGZsq{}Optimization terminated successfully\PYGZsq{}
    nfev: 21
     nit: 3
    njev: 3
  status: 0
 success: True
       x: array([3.0000e\PYGZhy{}01, 0.0000e+00, 1.0408e\PYGZhy{}17, 1.0000e\PYGZhy{}01, 3.0000e\PYGZhy{}01,
       3.0000e\PYGZhy{}01])
\end{sphinxVerbatim}

\end{sphinxuseclass}\end{sphinxVerbatimOutput}

\end{sphinxuseclass}
\sphinxAtStartPar
\sphinxstyleemphasis{\sphinxstylestrong{I removed the version that achieved the same result with constraints, because it did not work the same on different computers.}}
I cannot find documentation for why this solution fails, but I suspect using the \sphinxcode{\sphinxupquote{.max()}} method in a constraint makes convergence slower because changes to non\sphinxhyphen{}max values in \sphinxcode{\sphinxupquote{x}} do not change the constraint function output.


\subsubsection{Find the minimum 95\% Value at Risk (Var) portfolio of MATANA stocks over the last three years}
\label{\detokenize{herron_04_practice_03:find-the-minimum-95-value-at-risk-var-portfolio-of-matana-stocks-over-the-last-three-years}}
\sphinxAtStartPar
More on VaR \sphinxhref{https://en.wikipedia.org/wiki/Value\_at\_risk}{here}.

\begin{sphinxuseclass}{cell}\begin{sphinxVerbatimInput}

\begin{sphinxuseclass}{cell_input}
\begin{sphinxVerbatim}[commandchars=\\\{\}]
\PYG{k}{def} \PYG{n+nf}{port\PYGZus{}var}\PYG{p}{(}\PYG{n}{x}\PYG{p}{,} \PYG{n}{r}\PYG{p}{,} \PYG{n}{q}\PYG{p}{)}\PYG{p}{:}
    \PYG{k}{return} \PYG{n}{r}\PYG{o}{.}\PYG{n}{dot}\PYG{p}{(}\PYG{n}{x}\PYG{p}{)}\PYG{o}{.}\PYG{n}{quantile}\PYG{p}{(}\PYG{n}{q}\PYG{p}{)}
\end{sphinxVerbatim}

\end{sphinxuseclass}\end{sphinxVerbatimInput}

\end{sphinxuseclass}
\begin{sphinxuseclass}{cell}\begin{sphinxVerbatimInput}

\begin{sphinxuseclass}{cell_input}
\begin{sphinxVerbatim}[commandchars=\\\{\}]
\PYG{k}{def} \PYG{n+nf}{port\PYGZus{}var\PYGZus{}neg}\PYG{p}{(}\PYG{n}{x}\PYG{p}{,} \PYG{n}{r}\PYG{p}{,} \PYG{n}{q}\PYG{p}{)}\PYG{p}{:}
    \PYG{k}{return} \PYG{o}{\PYGZhy{}}\PYG{l+m+mi}{1} \PYG{o}{*} \PYG{n}{port\PYGZus{}var}\PYG{p}{(}\PYG{n}{x}\PYG{o}{=}\PYG{n}{x}\PYG{p}{,} \PYG{n}{r}\PYG{o}{=}\PYG{n}{r}\PYG{p}{,} \PYG{n}{q}\PYG{o}{=}\PYG{n}{q}\PYG{p}{)}
\end{sphinxVerbatim}

\end{sphinxuseclass}\end{sphinxVerbatimInput}

\end{sphinxuseclass}
\begin{sphinxuseclass}{cell}\begin{sphinxVerbatimInput}

\begin{sphinxuseclass}{cell_input}
\begin{sphinxVerbatim}[commandchars=\\\{\}]
\PYG{n}{res\PYGZus{}var\PYGZus{}1} \PYG{o}{=} \PYG{n}{sco}\PYG{o}{.}\PYG{n}{minimize}\PYG{p}{(}
    \PYG{n}{fun}\PYG{o}{=}\PYG{n}{port\PYGZus{}var\PYGZus{}neg}\PYG{p}{,}
    \PYG{n}{x0}\PYG{o}{=}\PYG{n}{np}\PYG{o}{.}\PYG{n}{ones}\PYG{p}{(}\PYG{n}{returns}\PYG{o}{.}\PYG{n}{shape}\PYG{p}{[}\PYG{l+m+mi}{1}\PYG{p}{]}\PYG{p}{)} \PYG{o}{/} \PYG{n}{returns}\PYG{o}{.}\PYG{n}{shape}\PYG{p}{[}\PYG{l+m+mi}{1}\PYG{p}{]}\PYG{p}{,}
    \PYG{n}{args}\PYG{o}{=}\PYG{p}{(}\PYG{n}{returns}\PYG{p}{,} \PYG{l+m+mf}{0.05}\PYG{p}{)}\PYG{p}{,}
    \PYG{n}{bounds}\PYG{o}{=}\PYG{p}{[}\PYG{p}{(}\PYG{l+m+mi}{0}\PYG{p}{,}\PYG{l+m+mi}{1}\PYG{p}{)} \PYG{k}{for} \PYG{n}{\PYGZus{}} \PYG{o+ow}{in} \PYG{n}{returns}\PYG{p}{]}\PYG{p}{,}
    \PYG{n}{constraints}\PYG{o}{=}\PYG{p}{(}
        \PYG{p}{\PYGZob{}}\PYG{l+s+s1}{\PYGZsq{}}\PYG{l+s+s1}{type}\PYG{l+s+s1}{\PYGZsq{}}\PYG{p}{:} \PYG{l+s+s1}{\PYGZsq{}}\PYG{l+s+s1}{eq}\PYG{l+s+s1}{\PYGZsq{}}\PYG{p}{,} \PYG{l+s+s1}{\PYGZsq{}}\PYG{l+s+s1}{fun}\PYG{l+s+s1}{\PYGZsq{}}\PYG{p}{:} \PYG{k}{lambda} \PYG{n}{x}\PYG{p}{:} \PYG{n}{x}\PYG{o}{.}\PYG{n}{sum}\PYG{p}{(}\PYG{p}{)} \PYG{o}{\PYGZhy{}} \PYG{l+m+mi}{1}\PYG{p}{\PYGZcb{}}\PYG{p}{,} \PYG{c+c1}{\PYGZsh{} minimize drives \PYGZdq{}eq\PYGZdq{} constraints to zero}
    \PYG{p}{)}
\PYG{p}{)}

\PYG{n}{res\PYGZus{}var\PYGZus{}1}
\end{sphinxVerbatim}

\end{sphinxuseclass}\end{sphinxVerbatimInput}
\begin{sphinxVerbatimOutput}

\begin{sphinxuseclass}{cell_output}
\begin{sphinxVerbatim}[commandchars=\\\{\}]
     fun: 0.03592250509338485
     jac: array([0.0359, 0.0419, 0.037 , 0.0385, 0.0275, 0.0416])
 message: \PYGZsq{}Optimization terminated successfully\PYGZsq{}
    nfev: 45
     nit: 5
    njev: 5
  status: 0
 success: True
       x: array([1.8180e\PYGZhy{}01, 1.4369e\PYGZhy{}01, 2.2517e\PYGZhy{}01, 2.4347e\PYGZhy{}01, 2.0587e\PYGZhy{}01,
       2.6479e\PYGZhy{}17])
\end{sphinxVerbatim}

\end{sphinxuseclass}\end{sphinxVerbatimOutput}

\end{sphinxuseclass}
\begin{sphinxuseclass}{cell}\begin{sphinxVerbatimInput}

\begin{sphinxuseclass}{cell_input}
\begin{sphinxVerbatim}[commandchars=\\\{\}]
\PYG{n}{port\PYGZus{}var}\PYG{p}{(}\PYG{n}{x}\PYG{o}{=}\PYG{n}{res\PYGZus{}var\PYGZus{}1}\PYG{p}{[}\PYG{l+s+s1}{\PYGZsq{}}\PYG{l+s+s1}{x}\PYG{l+s+s1}{\PYGZsq{}}\PYG{p}{]}\PYG{p}{,} \PYG{n}{r}\PYG{o}{=}\PYG{n}{returns}\PYG{p}{,} \PYG{n}{q}\PYG{o}{=}\PYG{l+m+mf}{0.05}\PYG{p}{)}
\end{sphinxVerbatim}

\end{sphinxuseclass}\end{sphinxVerbatimInput}
\begin{sphinxVerbatimOutput}

\begin{sphinxuseclass}{cell_output}
\begin{sphinxVerbatim}[commandchars=\\\{\}]
\PYGZhy{}0.03592250509338485
\end{sphinxVerbatim}

\end{sphinxuseclass}\end{sphinxVerbatimOutput}

\end{sphinxuseclass}
\sphinxAtStartPar
It might be helpful to slightly change then minimum VaR portfolio weights to show that we minimized VaR.

\begin{sphinxuseclass}{cell}\begin{sphinxVerbatimInput}

\begin{sphinxuseclass}{cell_input}
\begin{sphinxVerbatim}[commandchars=\\\{\}]
\PYG{k}{def} \PYG{n+nf}{tweak}\PYG{p}{(}\PYG{n}{x}\PYG{p}{,} \PYG{n}{d}\PYG{o}{=}\PYG{l+m+mf}{0.05}\PYG{p}{)}\PYG{p}{:}
    \PYG{n}{y} \PYG{o}{=} \PYG{n}{np}\PYG{o}{.}\PYG{n}{zeros}\PYG{p}{(}\PYG{n}{x}\PYG{o}{.}\PYG{n}{shape}\PYG{p}{[}\PYG{l+m+mi}{0}\PYG{p}{]}\PYG{p}{)}
    \PYG{n}{y}\PYG{p}{[}\PYG{l+m+mi}{0}\PYG{p}{]}\PYG{p}{,} \PYG{n}{y}\PYG{p}{[}\PYG{l+m+mi}{1}\PYG{p}{]} \PYG{o}{=} \PYG{n}{d}\PYG{p}{,} \PYG{o}{\PYGZhy{}}\PYG{l+m+mi}{1} \PYG{o}{*} \PYG{n}{d}
    \PYG{k}{return} \PYG{n}{x} \PYG{o}{+} \PYG{n}{y}
\end{sphinxVerbatim}

\end{sphinxuseclass}\end{sphinxVerbatimInput}

\end{sphinxuseclass}
\begin{sphinxuseclass}{cell}\begin{sphinxVerbatimInput}

\begin{sphinxuseclass}{cell_input}
\begin{sphinxVerbatim}[commandchars=\\\{\}]
\PYG{n}{port\PYGZus{}var}\PYG{p}{(}\PYG{n}{x}\PYG{o}{=}\PYG{n}{tweak}\PYG{p}{(}\PYG{n}{res\PYGZus{}var\PYGZus{}1}\PYG{p}{[}\PYG{l+s+s1}{\PYGZsq{}}\PYG{l+s+s1}{x}\PYG{l+s+s1}{\PYGZsq{}}\PYG{p}{]}\PYG{p}{)}\PYG{p}{,} \PYG{n}{r}\PYG{o}{=}\PYG{n}{returns}\PYG{p}{,} \PYG{n}{q}\PYG{o}{=}\PYG{l+m+mf}{0.05}\PYG{p}{)}
\end{sphinxVerbatim}

\end{sphinxuseclass}\end{sphinxVerbatimInput}
\begin{sphinxVerbatimOutput}

\begin{sphinxuseclass}{cell_output}
\begin{sphinxVerbatim}[commandchars=\\\{\}]
\PYGZhy{}0.03622949371096426
\end{sphinxVerbatim}

\end{sphinxuseclass}\end{sphinxVerbatimOutput}

\end{sphinxuseclass}

\subsubsection{Find the minimum maximum draw down portfolio of MATANA stocks over the last three years}
\label{\detokenize{herron_04_practice_03:find-the-minimum-maximum-draw-down-portfolio-of-matana-stocks-over-the-last-three-years}}
\begin{sphinxuseclass}{cell}\begin{sphinxVerbatimInput}

\begin{sphinxuseclass}{cell_input}
\begin{sphinxVerbatim}[commandchars=\\\{\}]
\PYG{k}{def} \PYG{n+nf}{port\PYGZus{}draw\PYGZus{}down\PYGZus{}max}\PYG{p}{(}\PYG{n}{x}\PYG{p}{,} \PYG{n}{r}\PYG{p}{)}\PYG{p}{:}
    \PYG{n}{rp} \PYG{o}{=} \PYG{n}{r}\PYG{o}{.}\PYG{n}{dot}\PYG{p}{(}\PYG{n}{x}\PYG{p}{)}
    \PYG{n}{price} \PYG{o}{=} \PYG{n}{rp}\PYG{o}{.}\PYG{n}{add}\PYG{p}{(}\PYG{l+m+mi}{1}\PYG{p}{)}\PYG{o}{.}\PYG{n}{cumprod}\PYG{p}{(}\PYG{p}{)}
    \PYG{n}{cum\PYGZus{}max} \PYG{o}{=} \PYG{n}{price}\PYG{o}{.}\PYG{n}{cummax}\PYG{p}{(}\PYG{p}{)}
    \PYG{n}{draw\PYGZus{}down} \PYG{o}{=} \PYG{p}{(}\PYG{n}{cum\PYGZus{}max} \PYG{o}{\PYGZhy{}} \PYG{n}{price}\PYG{p}{)} \PYG{o}{/} \PYG{n}{cum\PYGZus{}max}
    \PYG{k}{return} \PYG{n}{draw\PYGZus{}down}\PYG{o}{.}\PYG{n}{max}\PYG{p}{(}\PYG{p}{)}
\end{sphinxVerbatim}

\end{sphinxuseclass}\end{sphinxVerbatimInput}

\end{sphinxuseclass}
\begin{sphinxuseclass}{cell}\begin{sphinxVerbatimInput}

\begin{sphinxuseclass}{cell_input}
\begin{sphinxVerbatim}[commandchars=\\\{\}]
\PYG{n}{res\PYGZus{}dd\PYGZus{}1} \PYG{o}{=} \PYG{n}{sco}\PYG{o}{.}\PYG{n}{minimize}\PYG{p}{(}
    \PYG{n}{fun}\PYG{o}{=}\PYG{n}{port\PYGZus{}draw\PYGZus{}down\PYGZus{}max}\PYG{p}{,}
    \PYG{n}{x0}\PYG{o}{=}\PYG{n}{np}\PYG{o}{.}\PYG{n}{ones}\PYG{p}{(}\PYG{n}{returns}\PYG{o}{.}\PYG{n}{shape}\PYG{p}{[}\PYG{l+m+mi}{1}\PYG{p}{]}\PYG{p}{)} \PYG{o}{/} \PYG{n}{returns}\PYG{o}{.}\PYG{n}{shape}\PYG{p}{[}\PYG{l+m+mi}{1}\PYG{p}{]}\PYG{p}{,}
    \PYG{n}{args}\PYG{o}{=}\PYG{p}{(}\PYG{n}{returns}\PYG{p}{,}\PYG{p}{)}\PYG{p}{,}
    \PYG{n}{bounds}\PYG{o}{=}\PYG{p}{[}\PYG{p}{(}\PYG{l+m+mi}{0}\PYG{p}{,}\PYG{l+m+mi}{1}\PYG{p}{)} \PYG{k}{for} \PYG{n}{\PYGZus{}} \PYG{o+ow}{in} \PYG{n}{returns}\PYG{p}{]}\PYG{p}{,}
    \PYG{n}{constraints}\PYG{o}{=}\PYG{p}{(}
        \PYG{p}{\PYGZob{}}\PYG{l+s+s1}{\PYGZsq{}}\PYG{l+s+s1}{type}\PYG{l+s+s1}{\PYGZsq{}}\PYG{p}{:} \PYG{l+s+s1}{\PYGZsq{}}\PYG{l+s+s1}{eq}\PYG{l+s+s1}{\PYGZsq{}}\PYG{p}{,} \PYG{l+s+s1}{\PYGZsq{}}\PYG{l+s+s1}{fun}\PYG{l+s+s1}{\PYGZsq{}}\PYG{p}{:} \PYG{k}{lambda} \PYG{n}{x}\PYG{p}{:} \PYG{n}{x}\PYG{o}{.}\PYG{n}{sum}\PYG{p}{(}\PYG{p}{)} \PYG{o}{\PYGZhy{}} \PYG{l+m+mi}{1}\PYG{p}{\PYGZcb{}}\PYG{p}{,} \PYG{c+c1}{\PYGZsh{} minimize drives \PYGZdq{}eq\PYGZdq{} constraints to zero}
    \PYG{p}{)}
\PYG{p}{)}

\PYG{n}{res\PYGZus{}dd\PYGZus{}1}
\end{sphinxVerbatim}

\end{sphinxuseclass}\end{sphinxVerbatimInput}
\begin{sphinxVerbatimOutput}

\begin{sphinxuseclass}{cell_output}
\begin{sphinxVerbatim}[commandchars=\\\{\}]
     fun: 0.2973739393829374
     jac: array([0.2865, 0.5131, 0.4028, 0.2927, 0.5359, 0.7698])
 message: \PYGZsq{}Optimization terminated successfully\PYGZsq{}
    nfev: 164
     nit: 19
    njev: 19
  status: 0
 success: True
       x: array([6.8105e\PYGZhy{}01, 5.4865e\PYGZhy{}16, 2.2106e\PYGZhy{}15, 3.1895e\PYGZhy{}01, 4.8634e\PYGZhy{}16,
       5.7586e\PYGZhy{}16])
\end{sphinxVerbatim}

\end{sphinxuseclass}\end{sphinxVerbatimOutput}

\end{sphinxuseclass}
\begin{sphinxuseclass}{cell}\begin{sphinxVerbatimInput}

\begin{sphinxuseclass}{cell_input}
\begin{sphinxVerbatim}[commandchars=\\\{\}]
\PYG{n}{port\PYGZus{}draw\PYGZus{}down\PYGZus{}max}\PYG{p}{(}\PYG{n}{x}\PYG{o}{=}\PYG{n}{res\PYGZus{}dd\PYGZus{}1}\PYG{p}{[}\PYG{l+s+s1}{\PYGZsq{}}\PYG{l+s+s1}{x}\PYG{l+s+s1}{\PYGZsq{}}\PYG{p}{]}\PYG{p}{,} \PYG{n}{r}\PYG{o}{=}\PYG{n}{returns}\PYG{p}{)}
\end{sphinxVerbatim}

\end{sphinxuseclass}\end{sphinxVerbatimInput}
\begin{sphinxVerbatimOutput}

\begin{sphinxuseclass}{cell_output}
\begin{sphinxVerbatim}[commandchars=\\\{\}]
0.2974
\end{sphinxVerbatim}

\end{sphinxuseclass}\end{sphinxVerbatimOutput}

\end{sphinxuseclass}
\sphinxAtStartPar
Again. it might be helpful to slightly change then minimum VaR portfolio weights to show that we minimized VaR.

\begin{sphinxuseclass}{cell}\begin{sphinxVerbatimInput}

\begin{sphinxuseclass}{cell_input}
\begin{sphinxVerbatim}[commandchars=\\\{\}]
\PYG{n}{port\PYGZus{}draw\PYGZus{}down\PYGZus{}max}\PYG{p}{(}\PYG{n}{x}\PYG{o}{=}\PYG{n}{tweak}\PYG{p}{(}\PYG{n}{res\PYGZus{}dd\PYGZus{}1}\PYG{p}{[}\PYG{l+s+s1}{\PYGZsq{}}\PYG{l+s+s1}{x}\PYG{l+s+s1}{\PYGZsq{}}\PYG{p}{]}\PYG{p}{)}\PYG{p}{,} \PYG{n}{r}\PYG{o}{=}\PYG{n}{returns}\PYG{p}{)}
\end{sphinxVerbatim}

\end{sphinxuseclass}\end{sphinxVerbatimInput}
\begin{sphinxVerbatimOutput}

\begin{sphinxuseclass}{cell_output}
\begin{sphinxVerbatim}[commandchars=\\\{\}]
0.3065
\end{sphinxVerbatim}

\end{sphinxuseclass}\end{sphinxVerbatimOutput}

\end{sphinxuseclass}

\subsubsection{Find the minimum maximum draw down portfolio with all available data for the current Dow\sphinxhyphen{}Jones Industrial Average (DJIA) stocks}
\label{\detokenize{herron_04_practice_03:find-the-minimum-maximum-draw-down-portfolio-with-all-available-data-for-the-current-dow-jones-industrial-average-djia-stocks}}
\sphinxAtStartPar
You can find the \sphinxhref{https://en.wikipedia.org/wiki/Dow\_Jones\_Industrial\_Average}{DJIA tickers on Wikipedia}.

\begin{sphinxuseclass}{cell}\begin{sphinxVerbatimInput}

\begin{sphinxuseclass}{cell_input}
\begin{sphinxVerbatim}[commandchars=\\\{\}]
\PYG{n}{wiki} \PYG{o}{=} \PYG{n}{pd}\PYG{o}{.}\PYG{n}{read\PYGZus{}html}\PYG{p}{(}\PYG{l+s+s1}{\PYGZsq{}}\PYG{l+s+s1}{https://en.wikipedia.org/wiki/Dow\PYGZus{}Jones\PYGZus{}Industrial\PYGZus{}Average}\PYG{l+s+s1}{\PYGZsq{}}\PYG{p}{)}
\end{sphinxVerbatim}

\end{sphinxuseclass}\end{sphinxVerbatimInput}

\end{sphinxuseclass}
\begin{sphinxuseclass}{cell}\begin{sphinxVerbatimInput}

\begin{sphinxuseclass}{cell_input}
\begin{sphinxVerbatim}[commandchars=\\\{\}]
\PYG{n}{djia} \PYG{o}{=} \PYG{p}{(}
    \PYG{n}{yf}\PYG{o}{.}\PYG{n}{download}\PYG{p}{(}\PYG{n}{tickers}\PYG{o}{=}\PYG{n}{wiki}\PYG{p}{[}\PYG{l+m+mi}{1}\PYG{p}{]}\PYG{p}{[}\PYG{l+s+s1}{\PYGZsq{}}\PYG{l+s+s1}{Symbol}\PYG{l+s+s1}{\PYGZsq{}}\PYG{p}{]}\PYG{o}{.}\PYG{n}{to\PYGZus{}list}\PYG{p}{(}\PYG{p}{)}\PYG{p}{,} \PYG{n}{progress}\PYG{o}{=}\PYG{k+kc}{False}\PYG{p}{)}
    \PYG{o}{.}\PYG{n}{assign}\PYG{p}{(}\PYG{n}{Date}\PYG{o}{=}\PYG{k}{lambda} \PYG{n}{x}\PYG{p}{:} \PYG{n}{x}\PYG{o}{.}\PYG{n}{index}\PYG{o}{.}\PYG{n}{tz\PYGZus{}localize}\PYG{p}{(}\PYG{k+kc}{None}\PYG{p}{)}\PYG{p}{)}
    \PYG{o}{.}\PYG{n}{set\PYGZus{}index}\PYG{p}{(}\PYG{l+s+s1}{\PYGZsq{}}\PYG{l+s+s1}{Date}\PYG{l+s+s1}{\PYGZsq{}}\PYG{p}{)}
    \PYG{o}{.}\PYG{n}{rename\PYGZus{}axis}\PYG{p}{(}\PYG{n}{columns}\PYG{o}{=}\PYG{p}{[}\PYG{l+s+s1}{\PYGZsq{}}\PYG{l+s+s1}{Variable}\PYG{l+s+s1}{\PYGZsq{}}\PYG{p}{,} \PYG{l+s+s1}{\PYGZsq{}}\PYG{l+s+s1}{Ticker}\PYG{l+s+s1}{\PYGZsq{}}\PYG{p}{]}\PYG{p}{)}
\PYG{p}{)}

\PYG{n}{returns\PYGZus{}2} \PYG{o}{=} \PYG{n}{djia}\PYG{p}{[}\PYG{l+s+s1}{\PYGZsq{}}\PYG{l+s+s1}{Adj Close}\PYG{l+s+s1}{\PYGZsq{}}\PYG{p}{]}\PYG{o}{.}\PYG{n}{pct\PYGZus{}change}\PYG{p}{(}\PYG{p}{)}\PYG{o}{.}\PYG{n}{loc}\PYG{p}{[}\PYG{p}{:}\PYG{l+s+s1}{\PYGZsq{}}\PYG{l+s+s1}{2022}\PYG{l+s+s1}{\PYGZsq{}}\PYG{p}{]}\PYG{o}{.}\PYG{n}{dropna}\PYG{p}{(}\PYG{p}{)}
\PYG{n}{returns\PYGZus{}2}\PYG{o}{.}\PYG{n}{describe}\PYG{p}{(}\PYG{p}{)}
\end{sphinxVerbatim}

\end{sphinxuseclass}\end{sphinxVerbatimInput}
\begin{sphinxVerbatimOutput}

\begin{sphinxuseclass}{cell_output}
\begin{sphinxVerbatim}[commandchars=\\\{\}]
Ticker     AAPL     AMGN      AXP       BA      CAT      CRM     CSCO  \PYGZbs{}
count  954.0000 954.0000 954.0000 954.0000 954.0000 954.0000 954.0000   
mean     0.0013   0.0006   0.0007  \PYGZhy{}0.0001   0.0009   0.0001   0.0002   
std      0.0218   0.0167   0.0259   0.0350   0.0211   0.0258   0.0192   
min     \PYGZhy{}0.1286  \PYGZhy{}0.0826  \PYGZhy{}0.1482  \PYGZhy{}0.2385  \PYGZhy{}0.1428  \PYGZhy{}0.1589  \PYGZhy{}0.1373   
25\PYGZpc{}     \PYGZhy{}0.0093  \PYGZhy{}0.0073  \PYGZhy{}0.0108  \PYGZhy{}0.0149  \PYGZhy{}0.0100  \PYGZhy{}0.0121  \PYGZhy{}0.0083   
50\PYGZpc{}      0.0010   0.0002   0.0004  \PYGZhy{}0.0010   0.0009   0.0004   0.0003   
75\PYGZpc{}      0.0135   0.0080   0.0114   0.0143   0.0116   0.0134   0.0091   
max      0.1198   0.1090   0.2188   0.2432   0.1033   0.2604   0.1337   

Ticker      CVX      DIS      DOW  ...      MRK     MSFT      NKE       PG  \PYGZbs{}
count  954.0000 954.0000 954.0000  ... 954.0000 954.0000 954.0000 954.0000   
mean     0.0009   0.0000   0.0006  ...   0.0006   0.0010   0.0006   0.0006   
std      0.0244   0.0226   0.0260  ...   0.0153   0.0202   0.0221   0.0143   
min     \PYGZhy{}0.2212  \PYGZhy{}0.1316  \PYGZhy{}0.2166  ...  \PYGZhy{}0.0986  \PYGZhy{}0.1474  \PYGZhy{}0.1281  \PYGZhy{}0.0874   
25\PYGZpc{}     \PYGZhy{}0.0093  \PYGZhy{}0.0109  \PYGZhy{}0.0122  ...  \PYGZhy{}0.0066  \PYGZhy{}0.0088  \PYGZhy{}0.0098  \PYGZhy{}0.0055   
50\PYGZpc{}      0.0007  \PYGZhy{}0.0004   0.0002  ...   0.0003   0.0011   0.0007   0.0010   
75\PYGZpc{}      0.0112   0.0100   0.0138  ...   0.0084   0.0111   0.0115   0.0075   
max      0.2274   0.1441   0.2091  ...   0.0837   0.1422   0.1553   0.1201   

Ticker      TRV      UNH        V       VZ      WBA      WMT  
count  954.0000 954.0000 954.0000 954.0000 954.0000 954.0000  
mean     0.0007   0.0010   0.0005  \PYGZhy{}0.0001  \PYGZhy{}0.0001   0.0006  
std      0.0193   0.0198   0.0194   0.0126   0.0220   0.0150  
min     \PYGZhy{}0.2080  \PYGZhy{}0.1728  \PYGZhy{}0.1355  \PYGZhy{}0.0674  \PYGZhy{}0.1281  \PYGZhy{}0.1138  
25\PYGZpc{}     \PYGZhy{}0.0077  \PYGZhy{}0.0078  \PYGZhy{}0.0089  \PYGZhy{}0.0057  \PYGZhy{}0.0099  \PYGZhy{}0.0061  
50\PYGZpc{}      0.0018   0.0010   0.0012   0.0000  \PYGZhy{}0.0002   0.0005  
75\PYGZpc{}      0.0092   0.0094   0.0097   0.0058   0.0101   0.0071  
max      0.1329   0.1280   0.1384   0.0721   0.1260   0.1171  

[8 rows x 30 columns]
\end{sphinxVerbatim}

\end{sphinxuseclass}\end{sphinxVerbatimOutput}

\end{sphinxuseclass}
\begin{sphinxuseclass}{cell}\begin{sphinxVerbatimInput}

\begin{sphinxuseclass}{cell_input}
\begin{sphinxVerbatim}[commandchars=\\\{\}]
\PYG{n}{res\PYGZus{}dd\PYGZus{}2} \PYG{o}{=} \PYG{n}{sco}\PYG{o}{.}\PYG{n}{minimize}\PYG{p}{(}
    \PYG{n}{fun}\PYG{o}{=}\PYG{n}{port\PYGZus{}draw\PYGZus{}down\PYGZus{}max}\PYG{p}{,}
    \PYG{n}{x0}\PYG{o}{=}\PYG{n}{np}\PYG{o}{.}\PYG{n}{ones}\PYG{p}{(}\PYG{n}{returns\PYGZus{}2}\PYG{o}{.}\PYG{n}{shape}\PYG{p}{[}\PYG{l+m+mi}{1}\PYG{p}{]}\PYG{p}{)} \PYG{o}{/} \PYG{n}{returns\PYGZus{}2}\PYG{o}{.}\PYG{n}{shape}\PYG{p}{[}\PYG{l+m+mi}{1}\PYG{p}{]}\PYG{p}{,}
    \PYG{n}{args}\PYG{o}{=}\PYG{p}{(}\PYG{n}{returns\PYGZus{}2}\PYG{p}{,}\PYG{p}{)}\PYG{p}{,}
    \PYG{n}{bounds}\PYG{o}{=}\PYG{p}{[}\PYG{p}{(}\PYG{l+m+mi}{0}\PYG{p}{,}\PYG{l+m+mi}{1}\PYG{p}{)} \PYG{k}{for} \PYG{n}{\PYGZus{}} \PYG{o+ow}{in} \PYG{n}{returns\PYGZus{}2}\PYG{p}{]}\PYG{p}{,}
    \PYG{n}{constraints}\PYG{o}{=}\PYG{p}{(}
        \PYG{p}{\PYGZob{}}\PYG{l+s+s1}{\PYGZsq{}}\PYG{l+s+s1}{type}\PYG{l+s+s1}{\PYGZsq{}}\PYG{p}{:} \PYG{l+s+s1}{\PYGZsq{}}\PYG{l+s+s1}{eq}\PYG{l+s+s1}{\PYGZsq{}}\PYG{p}{,} \PYG{l+s+s1}{\PYGZsq{}}\PYG{l+s+s1}{fun}\PYG{l+s+s1}{\PYGZsq{}}\PYG{p}{:} \PYG{k}{lambda} \PYG{n}{x}\PYG{p}{:} \PYG{n}{x}\PYG{o}{.}\PYG{n}{sum}\PYG{p}{(}\PYG{p}{)} \PYG{o}{\PYGZhy{}} \PYG{l+m+mi}{1}\PYG{p}{\PYGZcb{}}\PYG{p}{,} \PYG{c+c1}{\PYGZsh{} minimize drives \PYGZdq{}eq\PYGZdq{} constraints to zero}
    \PYG{p}{)}
\PYG{p}{)}

\PYG{n}{res\PYGZus{}dd\PYGZus{}2}
\end{sphinxVerbatim}

\end{sphinxuseclass}\end{sphinxVerbatimInput}
\begin{sphinxVerbatimOutput}

\begin{sphinxuseclass}{cell_output}
\begin{sphinxVerbatim}[commandchars=\\\{\}]
     fun: 0.14672468509782907
     jac: array([0.1248, 0.2206, 0.3222, 0.757 , 0.3871, 0.2441, 0.2924, 0.4504,
       0.3766, 0.7404, 0.3316, 0.247 , 0.2301, 0.2645, 0.2266, 0.1252,
       0.3681, 0.1685, 0.2405, 0.2484, 0.2348, 0.1306, 0.3488, 0.1296,
       0.3978, 0.2202, 0.1876, 0.171 , 0.22  , 0.1148])
 message: \PYGZsq{}Optimization terminated successfully\PYGZsq{}
    nfev: 452
     nit: 14
    njev: 14
  status: 0
 success: True
       x: array([5.2593e\PYGZhy{}18, 1.4610e\PYGZhy{}02, 3.4999e\PYGZhy{}17, 1.3552e\PYGZhy{}17, 8.6975e\PYGZhy{}17,
       2.6923e\PYGZhy{}03, 2.9377e\PYGZhy{}18, 4.5178e\PYGZhy{}17, 2.6362e\PYGZhy{}18, 2.8238e\PYGZhy{}19,
       8.5599e\PYGZhy{}17, 3.8562e\PYGZhy{}19, 1.2979e\PYGZhy{}17, 9.1035e\PYGZhy{}18, 1.0157e\PYGZhy{}17,
       1.4280e\PYGZhy{}01, 6.8501e\PYGZhy{}17, 2.8469e\PYGZhy{}17, 2.7362e\PYGZhy{}17, 4.4210e\PYGZhy{}17,
       1.9194e\PYGZhy{}01, 1.3857e\PYGZhy{}01, 1.6587e\PYGZhy{}17, 3.3822e\PYGZhy{}17, 2.6444e\PYGZhy{}17,
       3.5995e\PYGZhy{}17, 0.0000e+00, 1.1371e\PYGZhy{}01, 1.0140e\PYGZhy{}16, 3.9568e\PYGZhy{}01])
\end{sphinxVerbatim}

\end{sphinxuseclass}\end{sphinxVerbatimOutput}

\end{sphinxuseclass}
\begin{sphinxuseclass}{cell}\begin{sphinxVerbatimInput}

\begin{sphinxuseclass}{cell_input}
\begin{sphinxVerbatim}[commandchars=\\\{\}]
\PYG{n}{port\PYGZus{}draw\PYGZus{}down\PYGZus{}max}\PYG{p}{(}\PYG{n}{x}\PYG{o}{=}\PYG{n}{res\PYGZus{}dd\PYGZus{}2}\PYG{p}{[}\PYG{l+s+s1}{\PYGZsq{}}\PYG{l+s+s1}{x}\PYG{l+s+s1}{\PYGZsq{}}\PYG{p}{]}\PYG{p}{,} \PYG{n}{r}\PYG{o}{=}\PYG{n}{returns\PYGZus{}2}\PYG{p}{)}
\end{sphinxVerbatim}

\end{sphinxuseclass}\end{sphinxVerbatimInput}
\begin{sphinxVerbatimOutput}

\begin{sphinxuseclass}{cell_output}
\begin{sphinxVerbatim}[commandchars=\\\{\}]
0.1467
\end{sphinxVerbatim}

\end{sphinxuseclass}\end{sphinxVerbatimOutput}

\end{sphinxuseclass}
\sphinxAtStartPar
Again. it might be helpful to slightly change then minimum VaR portfolio weights to show that we minimized VaR.

\begin{sphinxuseclass}{cell}\begin{sphinxVerbatimInput}

\begin{sphinxuseclass}{cell_input}
\begin{sphinxVerbatim}[commandchars=\\\{\}]
\PYG{n}{port\PYGZus{}draw\PYGZus{}down\PYGZus{}max}\PYG{p}{(}\PYG{n}{x}\PYG{o}{=}\PYG{n}{tweak}\PYG{p}{(}\PYG{n}{res\PYGZus{}dd\PYGZus{}2}\PYG{p}{[}\PYG{l+s+s1}{\PYGZsq{}}\PYG{l+s+s1}{x}\PYG{l+s+s1}{\PYGZsq{}}\PYG{p}{]}\PYG{p}{)}\PYG{p}{,} \PYG{n}{r}\PYG{o}{=}\PYG{n}{returns\PYGZus{}2}\PYG{p}{)}
\end{sphinxVerbatim}

\end{sphinxuseclass}\end{sphinxVerbatimInput}
\begin{sphinxVerbatimOutput}

\begin{sphinxuseclass}{cell_output}
\begin{sphinxVerbatim}[commandchars=\\\{\}]
0.1535
\end{sphinxVerbatim}

\end{sphinxuseclass}\end{sphinxVerbatimOutput}

\end{sphinxuseclass}

\subsubsection{Plot the (mean\sphinxhyphen{}variance) efficient frontier with all available data for the current the DJIA stocks}
\label{\detokenize{herron_04_practice_03:plot-the-mean-variance-efficient-frontier-with-all-available-data-for-the-current-the-djia-stocks}}
\sphinxAtStartPar
The range of target returns in \sphinxcode{\sphinxupquote{tret}} span from the minimum to the maximum mean single\sphinxhyphen{}stock returns.

\begin{sphinxuseclass}{cell}\begin{sphinxVerbatimInput}

\begin{sphinxuseclass}{cell_input}
\begin{sphinxVerbatim}[commandchars=\\\{\}]
\PYG{n}{\PYGZus{}} \PYG{o}{=} \PYG{n}{returns\PYGZus{}2}\PYG{o}{.}\PYG{n}{mean}\PYG{p}{(}\PYG{p}{)}\PYG{o}{.}\PYG{n}{mul}\PYG{p}{(}\PYG{l+m+mi}{252}\PYG{p}{)}
\PYG{n}{tret} \PYG{o}{=} \PYG{n}{np}\PYG{o}{.}\PYG{n}{linspace}\PYG{p}{(}\PYG{n}{\PYGZus{}}\PYG{o}{.}\PYG{n}{min}\PYG{p}{(}\PYG{p}{)}\PYG{p}{,} \PYG{n}{\PYGZus{}}\PYG{o}{.}\PYG{n}{max}\PYG{p}{(}\PYG{p}{)}\PYG{p}{,} \PYG{l+m+mi}{25}\PYG{p}{)}
\end{sphinxVerbatim}

\end{sphinxuseclass}\end{sphinxVerbatimInput}

\end{sphinxuseclass}
\sphinxAtStartPar
We will loop over these target returns, finding the minimum variance portfolio for each target return.

\begin{sphinxuseclass}{cell}\begin{sphinxVerbatimInput}

\begin{sphinxuseclass}{cell_input}
\begin{sphinxVerbatim}[commandchars=\\\{\}]
\PYG{k}{def} \PYG{n+nf}{port\PYGZus{}vol}\PYG{p}{(}\PYG{n}{x}\PYG{p}{,} \PYG{n}{r}\PYG{p}{,} \PYG{n}{ppy}\PYG{p}{)}\PYG{p}{:}
    \PYG{k}{return} \PYG{n}{np}\PYG{o}{.}\PYG{n}{sqrt}\PYG{p}{(}\PYG{n}{ppy}\PYG{p}{)} \PYG{o}{*} \PYG{n}{r}\PYG{o}{.}\PYG{n}{dot}\PYG{p}{(}\PYG{n}{x}\PYG{p}{)}\PYG{o}{.}\PYG{n}{std}\PYG{p}{(}\PYG{p}{)}
\end{sphinxVerbatim}

\end{sphinxuseclass}\end{sphinxVerbatimInput}

\end{sphinxuseclass}
\begin{sphinxuseclass}{cell}\begin{sphinxVerbatimInput}

\begin{sphinxuseclass}{cell_input}
\begin{sphinxVerbatim}[commandchars=\\\{\}]
\PYG{k}{def} \PYG{n+nf}{port\PYGZus{}mean}\PYG{p}{(}\PYG{n}{x}\PYG{p}{,} \PYG{n}{r}\PYG{p}{,} \PYG{n}{ppy}\PYG{p}{)}\PYG{p}{:}
    \PYG{k}{return} \PYG{n}{ppy} \PYG{o}{*} \PYG{n}{r}\PYG{o}{.}\PYG{n}{dot}\PYG{p}{(}\PYG{n}{x}\PYG{p}{)}\PYG{o}{.}\PYG{n}{mean}\PYG{p}{(}\PYG{p}{)}
\end{sphinxVerbatim}

\end{sphinxuseclass}\end{sphinxVerbatimInput}

\end{sphinxuseclass}
\begin{sphinxuseclass}{cell}\begin{sphinxVerbatimInput}

\begin{sphinxuseclass}{cell_input}
\begin{sphinxVerbatim}[commandchars=\\\{\}]
\PYG{n}{res\PYGZus{}ef} \PYG{o}{=} \PYG{p}{[}\PYG{p}{]}

\PYG{k}{for} \PYG{n}{t} \PYG{o+ow}{in} \PYG{n}{tret}\PYG{p}{:}
    \PYG{n}{\PYGZus{}} \PYG{o}{=} \PYG{n}{sco}\PYG{o}{.}\PYG{n}{minimize}\PYG{p}{(}
        \PYG{n}{fun}\PYG{o}{=}\PYG{n}{port\PYGZus{}vol}\PYG{p}{,} \PYG{c+c1}{\PYGZsh{} minimize portfolio volatility}
        \PYG{n}{x0}\PYG{o}{=}\PYG{n}{np}\PYG{o}{.}\PYG{n}{ones}\PYG{p}{(}\PYG{n}{returns\PYGZus{}2}\PYG{o}{.}\PYG{n}{shape}\PYG{p}{[}\PYG{l+m+mi}{1}\PYG{p}{]}\PYG{p}{)} \PYG{o}{/} \PYG{n}{returns\PYGZus{}2}\PYG{o}{.}\PYG{n}{shape}\PYG{p}{[}\PYG{l+m+mi}{1}\PYG{p}{]}\PYG{p}{,} \PYG{c+c1}{\PYGZsh{} initial portfolio weights}
        \PYG{n}{args}\PYG{o}{=}\PYG{p}{(}\PYG{n}{returns\PYGZus{}2}\PYG{p}{,} \PYG{l+m+mi}{252}\PYG{p}{)}\PYG{p}{,} \PYG{c+c1}{\PYGZsh{} additional arguments to fun, in order}
        \PYG{n}{bounds}\PYG{o}{=}\PYG{p}{[}\PYG{p}{(}\PYG{l+m+mi}{0}\PYG{p}{,} \PYG{l+m+mi}{1}\PYG{p}{)} \PYG{k}{for} \PYG{n}{c} \PYG{o+ow}{in} \PYG{n}{returns\PYGZus{}2}\PYG{o}{.}\PYG{n}{columns}\PYG{p}{]}\PYG{p}{,} \PYG{c+c1}{\PYGZsh{} bounds limit the search space for each portfolio weight}
        \PYG{n}{constraints}\PYG{o}{=}\PYG{p}{(}
            \PYG{p}{\PYGZob{}}\PYG{l+s+s1}{\PYGZsq{}}\PYG{l+s+s1}{type}\PYG{l+s+s1}{\PYGZsq{}}\PYG{p}{:} \PYG{l+s+s1}{\PYGZsq{}}\PYG{l+s+s1}{eq}\PYG{l+s+s1}{\PYGZsq{}}\PYG{p}{,} \PYG{l+s+s1}{\PYGZsq{}}\PYG{l+s+s1}{fun}\PYG{l+s+s1}{\PYGZsq{}}\PYG{p}{:} \PYG{k}{lambda} \PYG{n}{x}\PYG{p}{:} \PYG{n}{x}\PYG{o}{.}\PYG{n}{sum}\PYG{p}{(}\PYG{p}{)} \PYG{o}{\PYGZhy{}} \PYG{l+m+mi}{1}\PYG{p}{\PYGZcb{}}\PYG{p}{,} \PYG{c+c1}{\PYGZsh{} constrain sum of weights to one}
            \PYG{p}{\PYGZob{}}\PYG{l+s+s1}{\PYGZsq{}}\PYG{l+s+s1}{type}\PYG{l+s+s1}{\PYGZsq{}}\PYG{p}{:} \PYG{l+s+s1}{\PYGZsq{}}\PYG{l+s+s1}{eq}\PYG{l+s+s1}{\PYGZsq{}}\PYG{p}{,} \PYG{l+s+s1}{\PYGZsq{}}\PYG{l+s+s1}{fun}\PYG{l+s+s1}{\PYGZsq{}}\PYG{p}{:} \PYG{k}{lambda} \PYG{n}{x}\PYG{p}{:} \PYG{n}{port\PYGZus{}mean}\PYG{p}{(}\PYG{n}{x}\PYG{o}{=}\PYG{n}{x}\PYG{p}{,} \PYG{n}{r}\PYG{o}{=}\PYG{n}{returns\PYGZus{}2}\PYG{p}{,} \PYG{n}{ppy}\PYG{o}{=}\PYG{l+m+mi}{252}\PYG{p}{)} \PYG{o}{\PYGZhy{}} \PYG{n}{t}\PYG{p}{\PYGZcb{}} \PYG{c+c1}{\PYGZsh{} constrains portfolio mean return to the target return}

        \PYG{p}{)}
    \PYG{p}{)}
    \PYG{n}{res\PYGZus{}ef}\PYG{o}{.}\PYG{n}{append}\PYG{p}{(}\PYG{n}{\PYGZus{}}\PYG{p}{)}
\end{sphinxVerbatim}

\end{sphinxuseclass}\end{sphinxVerbatimInput}

\end{sphinxuseclass}
\sphinxAtStartPar
List \sphinxcode{\sphinxupquote{res\_ef}} contains the results of all 25 minimum\sphinxhyphen{}variance portfolios.
For example, \sphinxcode{\sphinxupquote{res\_ef{[}0{]}}} is the minimum variance portfolio for the lowest target return.

\begin{sphinxuseclass}{cell}\begin{sphinxVerbatimInput}

\begin{sphinxuseclass}{cell_input}
\begin{sphinxVerbatim}[commandchars=\\\{\}]
\PYG{n}{res\PYGZus{}ef}\PYG{p}{[}\PYG{l+m+mi}{0}\PYG{p}{]}
\end{sphinxVerbatim}

\end{sphinxuseclass}\end{sphinxVerbatimInput}
\begin{sphinxVerbatimOutput}

\begin{sphinxuseclass}{cell_output}
\begin{sphinxVerbatim}[commandchars=\\\{\}]
     fun: 0.39340634182496936
     jac: array([0.2109, 0.112 , 0.2034, 0.242 , 0.1532, 0.1866, 0.1751, 0.1646,
       0.1595, 0.1857, 0.186 , 0.1659, 0.1516, 0.1489, 0.3934, 0.0812,
       0.1762, 0.0979, 0.1063, 0.1259, 0.0721, 0.2056, 0.1586, 0.0962,
       0.1303, 0.1379, 0.163 , 0.0718, 0.1416, 0.0877])
 message: \PYGZsq{}Optimization terminated successfully\PYGZsq{}
    nfev: 63
     nit: 2
    njev: 2
  status: 0
 success: True
       x: array([0.0000e+00, 6.5646e\PYGZhy{}12, 1.4979e\PYGZhy{}11, 0.0000e+00, 0.0000e+00,
       0.0000e+00, 0.0000e+00, 0.0000e+00, 8.6573e\PYGZhy{}12, 1.3497e\PYGZhy{}12,
       5.6112e\PYGZhy{}12, 6.4123e\PYGZhy{}12, 0.0000e+00, 1.3199e\PYGZhy{}11, 1.0000e+00,
       0.0000e+00, 0.0000e+00, 9.4935e\PYGZhy{}12, 0.0000e+00, 0.0000e+00,
       0.0000e+00, 0.0000e+00, 8.8827e\PYGZhy{}12, 0.0000e+00, 0.0000e+00,
       0.0000e+00, 2.4949e\PYGZhy{}12, 8.0527e\PYGZhy{}12, 2.2320e\PYGZhy{}12, 5.6294e\PYGZhy{}12])
\end{sphinxVerbatim}

\end{sphinxuseclass}\end{sphinxVerbatimOutput}

\end{sphinxuseclass}
\sphinxAtStartPar
I typically check that all portfolio volatility minimization succeeds.
If a portfolio volatility minimization fails, we should check our function, bounds, and constraints.

\begin{sphinxuseclass}{cell}\begin{sphinxVerbatimInput}

\begin{sphinxuseclass}{cell_input}
\begin{sphinxVerbatim}[commandchars=\\\{\}]
\PYG{k}{for} \PYG{n}{r} \PYG{o+ow}{in} \PYG{n}{res\PYGZus{}ef}\PYG{p}{:}
    \PYG{k}{assert} \PYG{n}{r}\PYG{p}{[}\PYG{l+s+s1}{\PYGZsq{}}\PYG{l+s+s1}{success}\PYG{l+s+s1}{\PYGZsq{}}\PYG{p}{]} 
\end{sphinxVerbatim}

\end{sphinxuseclass}\end{sphinxVerbatimInput}

\end{sphinxuseclass}
\sphinxAtStartPar
We can combine the target returns and volatilities into a data frame \sphinxcode{\sphinxupquote{ef}}.

\begin{sphinxuseclass}{cell}\begin{sphinxVerbatimInput}

\begin{sphinxuseclass}{cell_input}
\begin{sphinxVerbatim}[commandchars=\\\{\}]
\PYG{n}{ef} \PYG{o}{=} \PYG{n}{pd}\PYG{o}{.}\PYG{n}{DataFrame}\PYG{p}{(}
    \PYG{p}{\PYGZob{}}
        \PYG{l+s+s1}{\PYGZsq{}}\PYG{l+s+s1}{tret}\PYG{l+s+s1}{\PYGZsq{}}\PYG{p}{:} \PYG{n}{tret}\PYG{p}{,}
        \PYG{l+s+s1}{\PYGZsq{}}\PYG{l+s+s1}{tvol}\PYG{l+s+s1}{\PYGZsq{}}\PYG{p}{:} \PYG{n}{np}\PYG{o}{.}\PYG{n}{array}\PYG{p}{(}\PYG{p}{[}\PYG{n}{r}\PYG{p}{[}\PYG{l+s+s1}{\PYGZsq{}}\PYG{l+s+s1}{fun}\PYG{l+s+s1}{\PYGZsq{}}\PYG{p}{]} \PYG{k}{if} \PYG{n}{r}\PYG{p}{[}\PYG{l+s+s1}{\PYGZsq{}}\PYG{l+s+s1}{success}\PYG{l+s+s1}{\PYGZsq{}}\PYG{p}{]} \PYG{k}{else} \PYG{n}{np}\PYG{o}{.}\PYG{n}{nan} \PYG{k}{for} \PYG{n}{r} \PYG{o+ow}{in} \PYG{n}{res\PYGZus{}ef}\PYG{p}{]}\PYG{p}{)}
    \PYG{p}{\PYGZcb{}}
\PYG{p}{)}

\PYG{n}{ef}\PYG{o}{.}\PYG{n}{head}\PYG{p}{(}\PYG{p}{)}
\end{sphinxVerbatim}

\end{sphinxuseclass}\end{sphinxVerbatimInput}
\begin{sphinxVerbatimOutput}

\begin{sphinxuseclass}{cell_output}
\begin{sphinxVerbatim}[commandchars=\\\{\}]
     tret   tvol
0 \PYGZhy{}0.0814 0.3934
1 \PYGZhy{}0.0640 0.2313
2 \PYGZhy{}0.0466 0.1906
3 \PYGZhy{}0.0292 0.1832
4 \PYGZhy{}0.0117 0.1773
\end{sphinxVerbatim}

\end{sphinxuseclass}\end{sphinxVerbatimOutput}

\end{sphinxuseclass}
\begin{sphinxuseclass}{cell}\begin{sphinxVerbatimInput}

\begin{sphinxuseclass}{cell_input}
\begin{sphinxVerbatim}[commandchars=\\\{\}]
\PYG{n}{ef}\PYG{o}{.}\PYG{n}{mul}\PYG{p}{(}\PYG{l+m+mi}{100}\PYG{p}{)}\PYG{o}{.}\PYG{n}{plot}\PYG{p}{(}\PYG{n}{x}\PYG{o}{=}\PYG{l+s+s1}{\PYGZsq{}}\PYG{l+s+s1}{tvol}\PYG{l+s+s1}{\PYGZsq{}}\PYG{p}{,} \PYG{n}{y}\PYG{o}{=}\PYG{l+s+s1}{\PYGZsq{}}\PYG{l+s+s1}{tret}\PYG{l+s+s1}{\PYGZsq{}}\PYG{p}{,} \PYG{n}{legend}\PYG{o}{=}\PYG{k+kc}{False}\PYG{p}{)}
\PYG{n}{plt}\PYG{o}{.}\PYG{n}{ylabel}\PYG{p}{(}\PYG{l+s+s1}{\PYGZsq{}}\PYG{l+s+s1}{Annualized Mean Return (}\PYG{l+s+s1}{\PYGZpc{}}\PYG{l+s+s1}{)}\PYG{l+s+s1}{\PYGZsq{}}\PYG{p}{)}
\PYG{n}{plt}\PYG{o}{.}\PYG{n}{xlabel}\PYG{p}{(}\PYG{l+s+s1}{\PYGZsq{}}\PYG{l+s+s1}{Annualized Volatility (}\PYG{l+s+s1}{\PYGZpc{}}\PYG{l+s+s1}{)}\PYG{l+s+s1}{\PYGZsq{}}\PYG{p}{)}
\PYG{n}{plt}\PYG{o}{.}\PYG{n}{title}\PYG{p}{(}
    \PYG{l+s+sa}{f}\PYG{l+s+s1}{\PYGZsq{}}\PYG{l+s+s1}{Efficient Frontier for Dow\PYGZhy{}Jones Industrial Average Stocks}\PYG{l+s+s1}{\PYGZsq{}} \PYG{o}{+}
    \PYG{l+s+sa}{f}\PYG{l+s+s1}{\PYGZsq{}}\PYG{l+s+se}{\PYGZbs{}n}\PYG{l+s+s1}{from }\PYG{l+s+si}{\PYGZob{}}\PYG{n}{returns\PYGZus{}2}\PYG{o}{.}\PYG{n}{index}\PYG{p}{[}\PYG{l+m+mi}{0}\PYG{p}{]}\PYG{l+s+si}{:}\PYG{l+s+s1}{\PYGZpc{}B \PYGZpc{}d, \PYGZpc{}Y}\PYG{l+s+si}{\PYGZcb{}}\PYG{l+s+s1}{ to }\PYG{l+s+si}{\PYGZob{}}\PYG{n}{returns\PYGZus{}2}\PYG{o}{.}\PYG{n}{index}\PYG{p}{[}\PYG{o}{\PYGZhy{}}\PYG{l+m+mi}{1}\PYG{p}{]}\PYG{l+s+si}{:}\PYG{l+s+s1}{\PYGZpc{}B \PYGZpc{}d, \PYGZpc{}Y}\PYG{l+s+si}{\PYGZcb{}}\PYG{l+s+s1}{\PYGZsq{}}
\PYG{p}{)}

\PYG{k}{for} \PYG{n}{t}\PYG{p}{,} \PYG{n}{x}\PYG{p}{,} \PYG{n}{y} \PYG{o+ow}{in} \PYG{n+nb}{zip}\PYG{p}{(}
    \PYG{n}{returns\PYGZus{}2}\PYG{o}{.}\PYG{n}{columns}\PYG{p}{,} 
    \PYG{n}{returns\PYGZus{}2}\PYG{o}{.}\PYG{n}{std}\PYG{p}{(}\PYG{p}{)}\PYG{o}{.}\PYG{n}{mul}\PYG{p}{(}\PYG{l+m+mi}{100}\PYG{o}{*}\PYG{n}{np}\PYG{o}{.}\PYG{n}{sqrt}\PYG{p}{(}\PYG{l+m+mi}{252}\PYG{p}{)}\PYG{p}{)}\PYG{p}{,}
    \PYG{n}{returns\PYGZus{}2}\PYG{o}{.}\PYG{n}{mean}\PYG{p}{(}\PYG{p}{)}\PYG{o}{.}\PYG{n}{mul}\PYG{p}{(}\PYG{l+m+mi}{100}\PYG{o}{*}\PYG{l+m+mi}{252}\PYG{p}{)}
\PYG{p}{)}\PYG{p}{:}
    \PYG{n}{plt}\PYG{o}{.}\PYG{n}{annotate}\PYG{p}{(}\PYG{n}{text}\PYG{o}{=}\PYG{n}{t}\PYG{p}{,} \PYG{n}{xy}\PYG{o}{=}\PYG{p}{(}\PYG{n}{x}\PYG{p}{,} \PYG{n}{y}\PYG{p}{)}\PYG{p}{)}
    
\PYG{n}{plt}\PYG{o}{.}\PYG{n}{show}\PYG{p}{(}\PYG{p}{)}
\end{sphinxVerbatim}

\end{sphinxuseclass}\end{sphinxVerbatimInput}
\begin{sphinxVerbatimOutput}

\begin{sphinxuseclass}{cell_output}
\noindent\sphinxincludegraphics{{e5867a76dfc7c654c2b4f0ef5f5c02b0270e4e589d65c147c4f33893740deef9}.png}

\end{sphinxuseclass}\end{sphinxVerbatimOutput}

\end{sphinxuseclass}

\subsubsection{Find the maximum Sharpe Ratio portfolio with all available data for the current the DJIA stocks}
\label{\detokenize{herron_04_practice_03:find-the-maximum-sharpe-ratio-portfolio-with-all-available-data-for-the-current-the-djia-stocks}}
\begin{sphinxuseclass}{cell}\begin{sphinxVerbatimInput}

\begin{sphinxuseclass}{cell_input}
\begin{sphinxVerbatim}[commandchars=\\\{\}]
\PYG{n}{res\PYGZus{}sharpe\PYGZus{}6} \PYG{o}{=} \PYG{n}{sco}\PYG{o}{.}\PYG{n}{minimize}\PYG{p}{(}
    \PYG{n}{fun}\PYG{o}{=}\PYG{n}{port\PYGZus{}sharpe\PYGZus{}neg}\PYG{p}{,}
    \PYG{n}{x0}\PYG{o}{=}\PYG{n}{np}\PYG{o}{.}\PYG{n}{ones}\PYG{p}{(}\PYG{n}{returns\PYGZus{}2}\PYG{o}{.}\PYG{n}{shape}\PYG{p}{[}\PYG{l+m+mi}{1}\PYG{p}{]}\PYG{p}{)} \PYG{o}{/} \PYG{n}{returns\PYGZus{}2}\PYG{o}{.}\PYG{n}{shape}\PYG{p}{[}\PYG{l+m+mi}{1}\PYG{p}{]}\PYG{p}{,}
    \PYG{n}{args}\PYG{o}{=}\PYG{p}{(}\PYG{n}{returns\PYGZus{}2}\PYG{p}{,} \PYG{l+m+mi}{252}\PYG{p}{,} \PYG{l+m+mi}{0}\PYG{p}{)}\PYG{p}{,}
    \PYG{n}{bounds}\PYG{o}{=}\PYG{p}{[}\PYG{p}{(}\PYG{l+m+mi}{0}\PYG{p}{,}\PYG{l+m+mi}{1}\PYG{p}{)} \PYG{k}{for} \PYG{n}{\PYGZus{}} \PYG{o+ow}{in} \PYG{n+nb}{range}\PYG{p}{(}\PYG{n}{returns\PYGZus{}2}\PYG{o}{.}\PYG{n}{shape}\PYG{p}{[}\PYG{l+m+mi}{1}\PYG{p}{]}\PYG{p}{)}\PYG{p}{]}\PYG{p}{,}
    \PYG{n}{constraints}\PYG{o}{=}\PYG{p}{(}
        \PYG{p}{\PYGZob{}}\PYG{l+s+s1}{\PYGZsq{}}\PYG{l+s+s1}{type}\PYG{l+s+s1}{\PYGZsq{}}\PYG{p}{:} \PYG{l+s+s1}{\PYGZsq{}}\PYG{l+s+s1}{eq}\PYG{l+s+s1}{\PYGZsq{}}\PYG{p}{,} \PYG{l+s+s1}{\PYGZsq{}}\PYG{l+s+s1}{fun}\PYG{l+s+s1}{\PYGZsq{}}\PYG{p}{:} \PYG{k}{lambda} \PYG{n}{x}\PYG{p}{:} \PYG{n}{x}\PYG{o}{.}\PYG{n}{sum}\PYG{p}{(}\PYG{p}{)} \PYG{o}{\PYGZhy{}} \PYG{l+m+mi}{1}\PYG{p}{\PYGZcb{}} \PYG{c+c1}{\PYGZsh{} want eq constraint to = 0}
    \PYG{p}{)}
\PYG{p}{)}
\end{sphinxVerbatim}

\end{sphinxuseclass}\end{sphinxVerbatimInput}

\end{sphinxuseclass}
\begin{sphinxuseclass}{cell}\begin{sphinxVerbatimInput}

\begin{sphinxuseclass}{cell_input}
\begin{sphinxVerbatim}[commandchars=\\\{\}]
\PYG{n}{port\PYGZus{}sharpe}\PYG{p}{(}\PYG{n}{x}\PYG{o}{=}\PYG{n}{res\PYGZus{}sharpe\PYGZus{}6}\PYG{p}{[}\PYG{l+s+s1}{\PYGZsq{}}\PYG{l+s+s1}{x}\PYG{l+s+s1}{\PYGZsq{}}\PYG{p}{]}\PYG{p}{,} \PYG{n}{r}\PYG{o}{=}\PYG{n}{returns\PYGZus{}2}\PYG{p}{,} \PYG{n}{ppy}\PYG{o}{=}\PYG{l+m+mi}{252}\PYG{p}{,} \PYG{n}{tgt}\PYG{o}{=}\PYG{l+m+mi}{0}\PYG{p}{)}
\end{sphinxVerbatim}

\end{sphinxuseclass}\end{sphinxVerbatimInput}
\begin{sphinxVerbatimOutput}

\begin{sphinxuseclass}{cell_output}
\begin{sphinxVerbatim}[commandchars=\\\{\}]
1.092759027029442
\end{sphinxVerbatim}

\end{sphinxuseclass}\end{sphinxVerbatimOutput}

\end{sphinxuseclass}

\subsubsection{Compare the \$\textbackslash{}frac\{1\}\{n\}\$ and maximum Sharpe Ratio portfolios with all available data for the current DJIA stocks}
\label{\detokenize{herron_04_practice_03:compare-the-frac-1-n-and-maximum-sharpe-ratio-portfolios-with-all-available-data-for-the-current-djia-stocks}}
\sphinxAtStartPar
Use all but the last 252 trading days to estimate the maximum Sharpe Ratio portfolio weights.
Then use the last 252 trading days of data to compare the \$\textbackslash{}frac\{1\}\{n\}\$  maximum Sharpe Ratio portfolios.

\begin{sphinxuseclass}{cell}\begin{sphinxVerbatimInput}

\begin{sphinxuseclass}{cell_input}
\begin{sphinxVerbatim}[commandchars=\\\{\}]
\PYG{n}{res\PYGZus{}sharpe\PYGZus{}x} \PYG{o}{=} \PYG{n}{sco}\PYG{o}{.}\PYG{n}{minimize}\PYG{p}{(}
    \PYG{n}{fun}\PYG{o}{=}\PYG{n}{port\PYGZus{}sharpe\PYGZus{}neg}\PYG{p}{,}
    \PYG{n}{x0}\PYG{o}{=}\PYG{n}{np}\PYG{o}{.}\PYG{n}{ones}\PYG{p}{(}\PYG{n}{returns\PYGZus{}2}\PYG{o}{.}\PYG{n}{shape}\PYG{p}{[}\PYG{l+m+mi}{1}\PYG{p}{]}\PYG{p}{)} \PYG{o}{/} \PYG{n}{returns\PYGZus{}2}\PYG{o}{.}\PYG{n}{shape}\PYG{p}{[}\PYG{l+m+mi}{1}\PYG{p}{]}\PYG{p}{,}
    \PYG{n}{args}\PYG{o}{=}\PYG{p}{(}\PYG{n}{returns\PYGZus{}2}\PYG{o}{.}\PYG{n}{iloc}\PYG{p}{[}\PYG{p}{:}\PYG{o}{\PYGZhy{}}\PYG{l+m+mi}{252}\PYG{p}{]}\PYG{p}{,} \PYG{l+m+mi}{252}\PYG{p}{,} \PYG{l+m+mi}{0}\PYG{p}{)}\PYG{p}{,}
    \PYG{n}{bounds}\PYG{o}{=}\PYG{p}{[}\PYG{p}{(}\PYG{l+m+mi}{0}\PYG{p}{,}\PYG{l+m+mi}{1}\PYG{p}{)} \PYG{k}{for} \PYG{n}{\PYGZus{}} \PYG{o+ow}{in} \PYG{n+nb}{range}\PYG{p}{(}\PYG{n}{returns\PYGZus{}2}\PYG{o}{.}\PYG{n}{shape}\PYG{p}{[}\PYG{l+m+mi}{1}\PYG{p}{]}\PYG{p}{)}\PYG{p}{]}\PYG{p}{,}
    \PYG{n}{constraints}\PYG{o}{=}\PYG{p}{(}
        \PYG{p}{\PYGZob{}}\PYG{l+s+s1}{\PYGZsq{}}\PYG{l+s+s1}{type}\PYG{l+s+s1}{\PYGZsq{}}\PYG{p}{:} \PYG{l+s+s1}{\PYGZsq{}}\PYG{l+s+s1}{eq}\PYG{l+s+s1}{\PYGZsq{}}\PYG{p}{,} \PYG{l+s+s1}{\PYGZsq{}}\PYG{l+s+s1}{fun}\PYG{l+s+s1}{\PYGZsq{}}\PYG{p}{:} \PYG{k}{lambda} \PYG{n}{x}\PYG{p}{:} \PYG{n}{x}\PYG{o}{.}\PYG{n}{sum}\PYG{p}{(}\PYG{p}{)} \PYG{o}{\PYGZhy{}} \PYG{l+m+mi}{1}\PYG{p}{\PYGZcb{}}\PYG{p}{,} \PYG{c+c1}{\PYGZsh{} want eq constraint to = 0}
    \PYG{p}{)}
\PYG{p}{)}

\PYG{k}{assert} \PYG{n}{res\PYGZus{}sharpe\PYGZus{}x}\PYG{p}{[}\PYG{l+s+s1}{\PYGZsq{}}\PYG{l+s+s1}{success}\PYG{l+s+s1}{\PYGZsq{}}\PYG{p}{]}
\PYG{c+c1}{\PYGZsh{} res\PYGZus{}sharpe\PYGZus{}x}
\end{sphinxVerbatim}

\end{sphinxuseclass}\end{sphinxVerbatimInput}

\end{sphinxuseclass}
\begin{sphinxuseclass}{cell}\begin{sphinxVerbatimInput}

\begin{sphinxuseclass}{cell_input}
\begin{sphinxVerbatim}[commandchars=\\\{\}]
\PYG{n}{plt}\PYG{o}{.}\PYG{n}{barh}\PYG{p}{(}
    \PYG{n}{y}\PYG{o}{=}\PYG{n}{returns\PYGZus{}2}\PYG{o}{.}\PYG{n}{columns}\PYG{p}{,}
    \PYG{n}{width}\PYG{o}{=}\PYG{n}{res\PYGZus{}sharpe\PYGZus{}x}\PYG{p}{[}\PYG{l+s+s1}{\PYGZsq{}}\PYG{l+s+s1}{x}\PYG{l+s+s1}{\PYGZsq{}}\PYG{p}{]}\PYG{p}{,}
    \PYG{n}{label}\PYG{o}{=}\PYG{l+s+s1}{\PYGZsq{}}\PYG{l+s+s1}{Maximum Sharpe Ratio}\PYG{l+s+s1}{\PYGZsq{}}
\PYG{p}{)}
\PYG{n}{plt}\PYG{o}{.}\PYG{n}{axvline}\PYG{p}{(}\PYG{l+m+mi}{1}\PYG{o}{/}\PYG{l+m+mi}{30}\PYG{p}{,} \PYG{n}{color}\PYG{o}{=}\PYG{l+s+s1}{\PYGZsq{}}\PYG{l+s+s1}{red}\PYG{l+s+s1}{\PYGZsq{}}\PYG{p}{,} \PYG{n}{label}\PYG{o}{=}\PYG{l+s+s1}{\PYGZsq{}}\PYG{l+s+s1}{Equal Weight}\PYG{l+s+s1}{\PYGZsq{}}\PYG{p}{)}
\PYG{n}{plt}\PYG{o}{.}\PYG{n}{legend}\PYG{p}{(}\PYG{p}{)}
\PYG{n}{plt}\PYG{o}{.}\PYG{n}{xlabel}\PYG{p}{(}\PYG{l+s+s1}{\PYGZsq{}}\PYG{l+s+s1}{Portfolio Weight}\PYG{l+s+s1}{\PYGZsq{}}\PYG{p}{)}
\PYG{n}{plt}\PYG{o}{.}\PYG{n}{title}\PYG{p}{(}
    \PYG{l+s+s1}{\PYGZsq{}}\PYG{l+s+s1}{Portfolio Weights for Dow\PYGZhy{}Jones Industrial Average Stocks}\PYG{l+s+s1}{\PYGZsq{}} \PYG{o}{+}
    \PYG{l+s+sa}{f}\PYG{l+s+s1}{\PYGZsq{}}\PYG{l+s+se}{\PYGZbs{}n}\PYG{l+s+s1}{from }\PYG{l+s+si}{\PYGZob{}}\PYG{n}{returns\PYGZus{}2}\PYG{o}{.}\PYG{n}{index}\PYG{p}{[}\PYG{l+m+mi}{0}\PYG{p}{]}\PYG{l+s+si}{:}\PYG{l+s+s1}{\PYGZpc{}b \PYGZpc{}d, \PYGZpc{}Y}\PYG{l+s+si}{\PYGZcb{}}\PYG{l+s+s1}{ to }\PYG{l+s+si}{\PYGZob{}}\PYG{n}{returns\PYGZus{}2}\PYG{o}{.}\PYG{n}{index}\PYG{p}{[}\PYG{o}{\PYGZhy{}}\PYG{l+m+mi}{1}\PYG{p}{]}\PYG{l+s+si}{:}\PYG{l+s+s1}{\PYGZpc{}b \PYGZpc{}d, \PYGZpc{}Y}\PYG{l+s+si}{\PYGZcb{}}\PYG{l+s+s1}{\PYGZsq{}}
\PYG{p}{)}
\PYG{n}{plt}\PYG{o}{.}\PYG{n}{show}\PYG{p}{(}\PYG{p}{)}
\end{sphinxVerbatim}

\end{sphinxuseclass}\end{sphinxVerbatimInput}
\begin{sphinxVerbatimOutput}

\begin{sphinxuseclass}{cell_output}
\noindent\sphinxincludegraphics{{b191669025532c59a146a0130d6d2041907f6e9199cd8d603be86812a5a44281}.png}

\end{sphinxuseclass}\end{sphinxVerbatimOutput}

\end{sphinxuseclass}
\begin{sphinxuseclass}{cell}\begin{sphinxVerbatimInput}

\begin{sphinxuseclass}{cell_input}
\begin{sphinxVerbatim}[commandchars=\\\{\}]
\PYG{n}{port\PYGZus{}sharpe}\PYG{p}{(}\PYG{n}{x}\PYG{o}{=}\PYG{n}{res\PYGZus{}sharpe\PYGZus{}x}\PYG{p}{[}\PYG{l+s+s1}{\PYGZsq{}}\PYG{l+s+s1}{x}\PYG{l+s+s1}{\PYGZsq{}}\PYG{p}{]}\PYG{p}{,} \PYG{n}{r}\PYG{o}{=}\PYG{n}{returns\PYGZus{}2}\PYG{o}{.}\PYG{n}{iloc}\PYG{p}{[}\PYG{p}{:}\PYG{o}{\PYGZhy{}}\PYG{l+m+mi}{252}\PYG{p}{]}\PYG{p}{,} \PYG{n}{ppy}\PYG{o}{=}\PYG{l+m+mi}{252}\PYG{p}{,} \PYG{n}{tgt}\PYG{o}{=}\PYG{l+m+mi}{0}\PYG{p}{)}
\end{sphinxVerbatim}

\end{sphinxuseclass}\end{sphinxVerbatimInput}
\begin{sphinxVerbatimOutput}

\begin{sphinxuseclass}{cell_output}
\begin{sphinxVerbatim}[commandchars=\\\{\}]
1.63277447434889
\end{sphinxVerbatim}

\end{sphinxuseclass}\end{sphinxVerbatimOutput}

\end{sphinxuseclass}
\begin{sphinxuseclass}{cell}\begin{sphinxVerbatimInput}

\begin{sphinxuseclass}{cell_input}
\begin{sphinxVerbatim}[commandchars=\\\{\}]
\PYG{n}{port\PYGZus{}sharpe}\PYG{p}{(}\PYG{n}{x}\PYG{o}{=}\PYG{n}{np}\PYG{o}{.}\PYG{n}{ones}\PYG{p}{(}\PYG{n}{returns\PYGZus{}2}\PYG{o}{.}\PYG{n}{shape}\PYG{p}{[}\PYG{l+m+mi}{1}\PYG{p}{]}\PYG{p}{)}\PYG{o}{/}\PYG{n}{returns\PYGZus{}2}\PYG{o}{.}\PYG{n}{shape}\PYG{p}{[}\PYG{l+m+mi}{1}\PYG{p}{]}\PYG{p}{,} \PYG{n}{r}\PYG{o}{=}\PYG{n}{returns\PYGZus{}2}\PYG{o}{.}\PYG{n}{iloc}\PYG{p}{[}\PYG{p}{:}\PYG{o}{\PYGZhy{}}\PYG{l+m+mi}{252}\PYG{p}{]}\PYG{p}{,} \PYG{n}{ppy}\PYG{o}{=}\PYG{l+m+mi}{252}\PYG{p}{,} \PYG{n}{tgt}\PYG{o}{=}\PYG{l+m+mi}{0}\PYG{p}{)}
\end{sphinxVerbatim}

\end{sphinxuseclass}\end{sphinxVerbatimInput}
\begin{sphinxVerbatimOutput}

\begin{sphinxuseclass}{cell_output}
\begin{sphinxVerbatim}[commandchars=\\\{\}]
0.8019384916171576
\end{sphinxVerbatim}

\end{sphinxuseclass}\end{sphinxVerbatimOutput}

\end{sphinxuseclass}
\sphinxAtStartPar
Out of sample:

\begin{sphinxuseclass}{cell}\begin{sphinxVerbatimInput}

\begin{sphinxuseclass}{cell_input}
\begin{sphinxVerbatim}[commandchars=\\\{\}]
\PYG{n}{port\PYGZus{}sharpe}\PYG{p}{(}\PYG{n}{x}\PYG{o}{=}\PYG{n}{res\PYGZus{}sharpe\PYGZus{}x}\PYG{p}{[}\PYG{l+s+s1}{\PYGZsq{}}\PYG{l+s+s1}{x}\PYG{l+s+s1}{\PYGZsq{}}\PYG{p}{]}\PYG{p}{,} \PYG{n}{r}\PYG{o}{=}\PYG{n}{returns\PYGZus{}2}\PYG{o}{.}\PYG{n}{iloc}\PYG{p}{[}\PYG{o}{\PYGZhy{}}\PYG{l+m+mi}{252}\PYG{p}{:}\PYG{p}{]}\PYG{p}{,} \PYG{n}{ppy}\PYG{o}{=}\PYG{l+m+mi}{252}\PYG{p}{,} \PYG{n}{tgt}\PYG{o}{=}\PYG{l+m+mi}{0}\PYG{p}{)}
\end{sphinxVerbatim}

\end{sphinxuseclass}\end{sphinxVerbatimInput}
\begin{sphinxVerbatimOutput}

\begin{sphinxuseclass}{cell_output}
\begin{sphinxVerbatim}[commandchars=\\\{\}]
\PYGZhy{}0.7218303076039226
\end{sphinxVerbatim}

\end{sphinxuseclass}\end{sphinxVerbatimOutput}

\end{sphinxuseclass}
\begin{sphinxuseclass}{cell}\begin{sphinxVerbatimInput}

\begin{sphinxuseclass}{cell_input}
\begin{sphinxVerbatim}[commandchars=\\\{\}]
\PYG{n}{port\PYGZus{}sharpe}\PYG{p}{(}\PYG{n}{x}\PYG{o}{=}\PYG{n}{np}\PYG{o}{.}\PYG{n}{ones}\PYG{p}{(}\PYG{n}{returns\PYGZus{}2}\PYG{o}{.}\PYG{n}{shape}\PYG{p}{[}\PYG{l+m+mi}{1}\PYG{p}{]}\PYG{p}{)}\PYG{o}{/}\PYG{n}{returns\PYGZus{}2}\PYG{o}{.}\PYG{n}{shape}\PYG{p}{[}\PYG{l+m+mi}{1}\PYG{p}{]}\PYG{p}{,} \PYG{n}{r}\PYG{o}{=}\PYG{n}{returns\PYGZus{}2}\PYG{o}{.}\PYG{n}{iloc}\PYG{p}{[}\PYG{o}{\PYGZhy{}}\PYG{l+m+mi}{252}\PYG{p}{:}\PYG{p}{]}\PYG{p}{,} \PYG{n}{ppy}\PYG{o}{=}\PYG{l+m+mi}{252}\PYG{p}{,} \PYG{n}{tgt}\PYG{o}{=}\PYG{l+m+mi}{0}\PYG{p}{)}
\end{sphinxVerbatim}

\end{sphinxuseclass}\end{sphinxVerbatimInput}
\begin{sphinxVerbatimOutput}

\begin{sphinxuseclass}{cell_output}
\begin{sphinxVerbatim}[commandchars=\\\{\}]
\PYGZhy{}0.2615614328449418
\end{sphinxVerbatim}

\end{sphinxuseclass}\end{sphinxVerbatimOutput}

\end{sphinxuseclass}
\sphinxAtStartPar
It is hard to beat the \$\textbackslash{}frac\{1\}\{n\}\$ portfolio because mean returns (and covariances) are hard to predict!

\sphinxstepscope


\section{Herron Topic 4 \sphinxhyphen{} Practice (Wednesday 11:45 AM, Section 4)}
\label{\detokenize{herron_04_practice_04:herron-topic-4-practice-wednesday-11-45-am-section-4}}\label{\detokenize{herron_04_practice_04::doc}}

\subsection{Announcements}
\label{\detokenize{herron_04_practice_04:announcements}}\begin{itemize}
\item {} 
\sphinxAtStartPar
Quiz 6 this week
\begin{itemize}
\item {} 
\sphinxAtStartPar
I will post it at about 6 PM on Wednesday, 3/31

\item {} 
\sphinxAtStartPar
It will be due by 11:59 PM on Friday, 3/31

\end{itemize}

\item {} 
\sphinxAtStartPar
Please complete the week ten survey
\begin{itemize}
\item {} 
\sphinxAtStartPar
I am considering dropping a topic to allow more in\sphinxhyphen{}class group work and easier access to me

\item {} 
\sphinxAtStartPar
I am also curious why the quantitative courses are less popular this summer

\item {} 
\sphinxAtStartPar
Please complete by 11:59 PM on Friday, 3/31

\item {} 
\sphinxAtStartPar
\sphinxstyleemphasis{\sphinxstylestrong{The week ten survey is anonymous and voluntary}}

\end{itemize}

\item {} 
\sphinxAtStartPar
I will post project 2 as soon as I can

\item {} 
\sphinxAtStartPar
Assessment exam
\begin{itemize}
\item {} 
\sphinxAtStartPar
20 questions multiple on Canvas

\item {} 
\sphinxAtStartPar
You must be in the room

\item {} 
\sphinxAtStartPar
No specific studying, but I suggest putting core course resources on your laptop (e.g., notes and PowerPoints)

\end{itemize}

\end{itemize}


\subsection{Practice}
\label{\detokenize{herron_04_practice_04:practice}}
\begin{sphinxuseclass}{cell}\begin{sphinxVerbatimInput}

\begin{sphinxuseclass}{cell_input}
\begin{sphinxVerbatim}[commandchars=\\\{\}]
\PYG{k+kn}{import} \PYG{n+nn}{matplotlib}\PYG{n+nn}{.}\PYG{n+nn}{pyplot} \PYG{k}{as} \PYG{n+nn}{plt}
\PYG{k+kn}{import} \PYG{n+nn}{numpy} \PYG{k}{as} \PYG{n+nn}{np}
\PYG{k+kn}{import} \PYG{n+nn}{pandas} \PYG{k}{as} \PYG{n+nn}{pd}
\end{sphinxVerbatim}

\end{sphinxuseclass}\end{sphinxVerbatimInput}

\end{sphinxuseclass}
\begin{sphinxuseclass}{cell}\begin{sphinxVerbatimInput}

\begin{sphinxuseclass}{cell_input}
\begin{sphinxVerbatim}[commandchars=\\\{\}]
\PYG{o}{\PYGZpc{}}\PYG{k}{config} InlineBackend.figure\PYGZus{}format = \PYGZsq{}retina\PYGZsq{}
\PYG{o}{\PYGZpc{}}\PYG{k}{precision} 4
\PYG{n}{pd}\PYG{o}{.}\PYG{n}{options}\PYG{o}{.}\PYG{n}{display}\PYG{o}{.}\PYG{n}{float\PYGZus{}format} \PYG{o}{=} \PYG{l+s+s1}{\PYGZsq{}}\PYG{l+s+si}{\PYGZob{}:.4f\PYGZcb{}}\PYG{l+s+s1}{\PYGZsq{}}\PYG{o}{.}\PYG{n}{format}
\end{sphinxVerbatim}

\end{sphinxuseclass}\end{sphinxVerbatimInput}

\end{sphinxuseclass}
\begin{sphinxuseclass}{cell}\begin{sphinxVerbatimInput}

\begin{sphinxuseclass}{cell_input}
\begin{sphinxVerbatim}[commandchars=\\\{\}]
\PYG{k+kn}{import} \PYG{n+nn}{yfinance} \PYG{k}{as} \PYG{n+nn}{yf}
\PYG{k+kn}{import} \PYG{n+nn}{pandas\PYGZus{}datareader} \PYG{k}{as} \PYG{n+nn}{pdr}
\PYG{k+kn}{import} \PYG{n+nn}{requests\PYGZus{}cache}
\PYG{n}{session} \PYG{o}{=} \PYG{n}{requests\PYGZus{}cache}\PYG{o}{.}\PYG{n}{CachedSession}\PYG{p}{(}\PYG{p}{)}
\end{sphinxVerbatim}

\end{sphinxuseclass}\end{sphinxVerbatimInput}

\end{sphinxuseclass}
\begin{sphinxuseclass}{cell}\begin{sphinxVerbatimInput}

\begin{sphinxuseclass}{cell_input}
\begin{sphinxVerbatim}[commandchars=\\\{\}]
\PYG{k+kn}{import} \PYG{n+nn}{scipy}\PYG{n+nn}{.}\PYG{n+nn}{optimize} \PYG{k}{as} \PYG{n+nn}{sco}
\end{sphinxVerbatim}

\end{sphinxuseclass}\end{sphinxVerbatimInput}

\end{sphinxuseclass}

\subsubsection{Find the maximum Sharpe Ratio portfolio of MATANA stocks over the last three years}
\label{\detokenize{herron_04_practice_04:find-the-maximum-sharpe-ratio-portfolio-of-matana-stocks-over-the-last-three-years}}
\sphinxAtStartPar
\sphinxstyleemphasis{\sphinxstylestrong{Note that \sphinxcode{\sphinxupquote{sco.minimize()}} finds minimums, so we need to minimize the negative Sharpe Ratio.}}

\sphinxAtStartPar
The following code downloads data for the MATANA stocks and assigns daily decimal returns from 2020 through 2022 to data frame \sphinxcode{\sphinxupquote{returns}}.
We will stop in 2022 to make it easier to compare our results, whether we use the risk\sphinxhyphen{}free rate or value\sphinxhyphen{}weighted market portfolio as our benchmark or not.
Recall, the Fama and French benchmark factors are only available with a lag, and are only available through December 2022 as I type.

\begin{sphinxuseclass}{cell}\begin{sphinxVerbatimInput}

\begin{sphinxuseclass}{cell_input}
\begin{sphinxVerbatim}[commandchars=\\\{\}]
\PYG{n}{tickers} \PYG{o}{=} \PYG{l+s+s1}{\PYGZsq{}}\PYG{l+s+s1}{MSFT AAPL TSLA AMZN NVDA GOOG}\PYG{l+s+s1}{\PYGZsq{}}

\PYG{n}{matana} \PYG{o}{=} \PYG{p}{(}
    \PYG{n}{yf}\PYG{o}{.}\PYG{n}{download}\PYG{p}{(}\PYG{n}{tickers}\PYG{o}{=}\PYG{n}{tickers}\PYG{p}{,} \PYG{n}{progress}\PYG{o}{=}\PYG{k+kc}{False}\PYG{p}{)}
    \PYG{o}{.}\PYG{n}{assign}\PYG{p}{(}\PYG{n}{Date}\PYG{o}{=}\PYG{k}{lambda} \PYG{n}{x}\PYG{p}{:} \PYG{n}{x}\PYG{o}{.}\PYG{n}{index}\PYG{o}{.}\PYG{n}{tz\PYGZus{}localize}\PYG{p}{(}\PYG{k+kc}{None}\PYG{p}{)}\PYG{p}{)}
    \PYG{o}{.}\PYG{n}{set\PYGZus{}index}\PYG{p}{(}\PYG{l+s+s1}{\PYGZsq{}}\PYG{l+s+s1}{Date}\PYG{l+s+s1}{\PYGZsq{}}\PYG{p}{)}
    \PYG{o}{.}\PYG{n}{rename\PYGZus{}axis}\PYG{p}{(}\PYG{n}{columns}\PYG{o}{=}\PYG{p}{[}\PYG{l+s+s1}{\PYGZsq{}}\PYG{l+s+s1}{Variable}\PYG{l+s+s1}{\PYGZsq{}}\PYG{p}{,} \PYG{l+s+s1}{\PYGZsq{}}\PYG{l+s+s1}{Ticker}\PYG{l+s+s1}{\PYGZsq{}}\PYG{p}{]}\PYG{p}{)}
\PYG{p}{)}

\PYG{n}{returns} \PYG{o}{=} \PYG{n}{matana}\PYG{p}{[}\PYG{l+s+s1}{\PYGZsq{}}\PYG{l+s+s1}{Adj Close}\PYG{l+s+s1}{\PYGZsq{}}\PYG{p}{]}\PYG{o}{.}\PYG{n}{pct\PYGZus{}change}\PYG{p}{(}\PYG{p}{)}\PYG{o}{.}\PYG{n}{loc}\PYG{p}{[}\PYG{l+s+s1}{\PYGZsq{}}\PYG{l+s+s1}{2020}\PYG{l+s+s1}{\PYGZsq{}}\PYG{p}{:}\PYG{l+s+s1}{\PYGZsq{}}\PYG{l+s+s1}{2022}\PYG{l+s+s1}{\PYGZsq{}}\PYG{p}{]}
\PYG{n}{returns}\PYG{o}{.}\PYG{n}{describe}\PYG{p}{(}\PYG{p}{)}
\end{sphinxVerbatim}

\end{sphinxuseclass}\end{sphinxVerbatimInput}
\begin{sphinxVerbatimOutput}

\begin{sphinxuseclass}{cell_output}
\begin{sphinxVerbatim}[commandchars=\\\{\}]
Ticker     AAPL     AMZN     GOOG     MSFT     NVDA     TSLA
count  756.0000 756.0000 756.0000 756.0000 756.0000 756.0000
mean     0.0011   0.0002   0.0006   0.0008   0.0018   0.0030
std      0.0233   0.0246   0.0217   0.0219   0.0352   0.0455
min     \PYGZhy{}0.1286  \PYGZhy{}0.1405  \PYGZhy{}0.1110  \PYGZhy{}0.1474  \PYGZhy{}0.1845  \PYGZhy{}0.2106
25\PYGZpc{}     \PYGZhy{}0.0110  \PYGZhy{}0.0129  \PYGZhy{}0.0097  \PYGZhy{}0.0097  \PYGZhy{}0.0177  \PYGZhy{}0.0215
50\PYGZpc{}      0.0006   0.0006   0.0012   0.0007   0.0029   0.0020
75\PYGZpc{}      0.0142   0.0123   0.0114   0.0123   0.0222   0.0251
max      0.1198   0.1354   0.0940   0.1422   0.1716   0.1989
\end{sphinxVerbatim}

\end{sphinxuseclass}\end{sphinxVerbatimOutput}

\end{sphinxuseclass}
\begin{sphinxuseclass}{cell}\begin{sphinxVerbatimInput}

\begin{sphinxuseclass}{cell_input}
\begin{sphinxVerbatim}[commandchars=\\\{\}]
\PYG{k}{def} \PYG{n+nf}{port\PYGZus{}sharpe}\PYG{p}{(}\PYG{n}{x}\PYG{p}{,} \PYG{n}{r}\PYG{p}{,} \PYG{n}{tgt}\PYG{p}{,} \PYG{n}{ppy}\PYG{p}{)}\PYG{p}{:}
    \PYG{l+s+sd}{\PYGZdq{}\PYGZdq{}\PYGZdq{}}
\PYG{l+s+sd}{    x: portfolio weights (must be first arg to use with sco.minimize())}
\PYG{l+s+sd}{    r: data frame of portfolio returns}
\PYG{l+s+sd}{    tgt: target or benchmark for Sharpe Ratio}
\PYG{l+s+sd}{    ppy: return periods per year}
\PYG{l+s+sd}{    \PYGZdq{}\PYGZdq{}\PYGZdq{}}
    \PYG{n}{rp} \PYG{o}{=} \PYG{n}{r}\PYG{o}{.}\PYG{n}{dot}\PYG{p}{(}\PYG{n}{x}\PYG{p}{)}
    \PYG{n}{er} \PYG{o}{=} \PYG{n}{rp}\PYG{o}{.}\PYG{n}{sub}\PYG{p}{(}\PYG{n}{tgt}\PYG{p}{)}
    \PYG{k}{return} \PYG{n}{np}\PYG{o}{.}\PYG{n}{sqrt}\PYG{p}{(}\PYG{n}{ppy}\PYG{p}{)} \PYG{o}{*} \PYG{n}{er}\PYG{o}{.}\PYG{n}{mean}\PYG{p}{(}\PYG{p}{)} \PYG{o}{/} \PYG{n}{er}\PYG{o}{.}\PYG{n}{std}\PYG{p}{(}\PYG{p}{)}
\end{sphinxVerbatim}

\end{sphinxuseclass}\end{sphinxVerbatimInput}

\end{sphinxuseclass}
\begin{sphinxuseclass}{cell}\begin{sphinxVerbatimInput}

\begin{sphinxuseclass}{cell_input}
\begin{sphinxVerbatim}[commandchars=\\\{\}]
\PYG{k}{def} \PYG{n+nf}{port\PYGZus{}sharpe\PYGZus{}neg}\PYG{p}{(}\PYG{n}{x}\PYG{p}{,} \PYG{n}{r}\PYG{p}{,} \PYG{n}{tgt}\PYG{p}{,} \PYG{n}{ppy}\PYG{p}{)}\PYG{p}{:}
    \PYG{k}{return} \PYG{o}{\PYGZhy{}}\PYG{l+m+mi}{1} \PYG{o}{*} \PYG{n}{port\PYGZus{}sharpe}\PYG{p}{(}\PYG{n}{x}\PYG{p}{,} \PYG{n}{r}\PYG{p}{,} \PYG{n}{tgt}\PYG{p}{,} \PYG{n}{ppy}\PYG{p}{)}
\end{sphinxVerbatim}

\end{sphinxuseclass}\end{sphinxVerbatimInput}

\end{sphinxuseclass}
\begin{sphinxuseclass}{cell}\begin{sphinxVerbatimInput}

\begin{sphinxuseclass}{cell_input}
\begin{sphinxVerbatim}[commandchars=\\\{\}]
\PYG{n}{res\PYGZus{}sharpe\PYGZus{}1} \PYG{o}{=} \PYG{n}{sco}\PYG{o}{.}\PYG{n}{minimize}\PYG{p}{(}
    \PYG{n}{fun}\PYG{o}{=}\PYG{n}{port\PYGZus{}sharpe\PYGZus{}neg}\PYG{p}{,}
    \PYG{n}{x0}\PYG{o}{=}\PYG{n}{np}\PYG{o}{.}\PYG{n}{ones}\PYG{p}{(}\PYG{n}{returns}\PYG{o}{.}\PYG{n}{shape}\PYG{p}{[}\PYG{l+m+mi}{1}\PYG{p}{]}\PYG{p}{)} \PYG{o}{/} \PYG{n}{returns}\PYG{o}{.}\PYG{n}{shape}\PYG{p}{[}\PYG{l+m+mi}{1}\PYG{p}{]}\PYG{p}{,}
    \PYG{n}{args}\PYG{o}{=}\PYG{p}{(}\PYG{n}{returns}\PYG{p}{,} \PYG{l+m+mi}{0}\PYG{p}{,} \PYG{l+m+mi}{252}\PYG{p}{)}\PYG{p}{,}
    \PYG{n}{bounds}\PYG{o}{=}\PYG{p}{[}\PYG{p}{(}\PYG{l+m+mi}{0}\PYG{p}{,}\PYG{l+m+mi}{1}\PYG{p}{)} \PYG{k}{for} \PYG{n}{\PYGZus{}} \PYG{o+ow}{in} \PYG{n}{returns}\PYG{p}{]}\PYG{p}{,}
    \PYG{n}{constraints}\PYG{o}{=}\PYG{p}{(}
        \PYG{p}{\PYGZob{}}\PYG{l+s+s1}{\PYGZsq{}}\PYG{l+s+s1}{type}\PYG{l+s+s1}{\PYGZsq{}}\PYG{p}{:} \PYG{l+s+s1}{\PYGZsq{}}\PYG{l+s+s1}{eq}\PYG{l+s+s1}{\PYGZsq{}}\PYG{p}{,} \PYG{l+s+s1}{\PYGZsq{}}\PYG{l+s+s1}{fun}\PYG{l+s+s1}{\PYGZsq{}}\PYG{p}{:} \PYG{k}{lambda} \PYG{n}{x}\PYG{p}{:} \PYG{n}{x}\PYG{o}{.}\PYG{n}{sum}\PYG{p}{(}\PYG{p}{)} \PYG{o}{\PYGZhy{}} \PYG{l+m+mi}{1}\PYG{p}{\PYGZcb{}} \PYG{c+c1}{\PYGZsh{} eq constraints are driven to 0}
    \PYG{p}{)}
\PYG{p}{)}

\PYG{n}{res\PYGZus{}sharpe\PYGZus{}1}
\end{sphinxVerbatim}

\end{sphinxuseclass}\end{sphinxVerbatimInput}
\begin{sphinxVerbatimOutput}

\begin{sphinxuseclass}{cell_output}
\begin{sphinxVerbatim}[commandchars=\\\{\}]
     fun: \PYGZhy{}1.0891990229341466
     jac: array([ 1.1644e\PYGZhy{}04,  3.6243e\PYGZhy{}01,  1.1961e\PYGZhy{}01,  6.4995e\PYGZhy{}02,  4.9078e\PYGZhy{}04,
       \PYGZhy{}2.5971e\PYGZhy{}04])
 message: \PYGZsq{}Optimization terminated successfully\PYGZsq{}
    nfev: 63
     nit: 9
    njev: 9
  status: 0
 success: True
       x: array([6.6101e\PYGZhy{}02, 3.5779e\PYGZhy{}17, 4.3043e\PYGZhy{}17, 7.2642e\PYGZhy{}18, 3.1307e\PYGZhy{}01,
       6.2083e\PYGZhy{}01])
\end{sphinxVerbatim}

\end{sphinxuseclass}\end{sphinxVerbatimOutput}

\end{sphinxuseclass}
\begin{sphinxuseclass}{cell}\begin{sphinxVerbatimInput}

\begin{sphinxuseclass}{cell_input}
\begin{sphinxVerbatim}[commandchars=\\\{\}]
\PYG{n}{port\PYGZus{}sharpe}\PYG{p}{(}\PYG{n}{x}\PYG{o}{=}\PYG{n}{res\PYGZus{}sharpe\PYGZus{}1}\PYG{p}{[}\PYG{l+s+s1}{\PYGZsq{}}\PYG{l+s+s1}{x}\PYG{l+s+s1}{\PYGZsq{}}\PYG{p}{]}\PYG{p}{,} \PYG{n}{r}\PYG{o}{=}\PYG{n}{returns}\PYG{p}{,} \PYG{n}{tgt}\PYG{o}{=}\PYG{l+m+mi}{0}\PYG{p}{,} \PYG{n}{ppy}\PYG{o}{=}\PYG{l+m+mi}{252}\PYG{p}{)}
\end{sphinxVerbatim}

\end{sphinxuseclass}\end{sphinxVerbatimInput}
\begin{sphinxVerbatimOutput}

\begin{sphinxuseclass}{cell_output}
\begin{sphinxVerbatim}[commandchars=\\\{\}]
1.0891990229341466
\end{sphinxVerbatim}

\end{sphinxuseclass}\end{sphinxVerbatimOutput}

\end{sphinxuseclass}

\subsubsection{Find the maximum Sharpe Ratio portfolio of MATANA stocks over the last three years, but allow short weights up to 10\% on each stock}
\label{\detokenize{herron_04_practice_04:find-the-maximum-sharpe-ratio-portfolio-of-matana-stocks-over-the-last-three-years-but-allow-short-weights-up-to-10-on-each-stock}}
\begin{sphinxuseclass}{cell}\begin{sphinxVerbatimInput}

\begin{sphinxuseclass}{cell_input}
\begin{sphinxVerbatim}[commandchars=\\\{\}]
\PYG{n}{res\PYGZus{}sharpe\PYGZus{}2} \PYG{o}{=} \PYG{n}{sco}\PYG{o}{.}\PYG{n}{minimize}\PYG{p}{(}
    \PYG{n}{fun}\PYG{o}{=}\PYG{n}{port\PYGZus{}sharpe\PYGZus{}neg}\PYG{p}{,}
    \PYG{n}{x0}\PYG{o}{=}\PYG{n}{np}\PYG{o}{.}\PYG{n}{ones}\PYG{p}{(}\PYG{n}{returns}\PYG{o}{.}\PYG{n}{shape}\PYG{p}{[}\PYG{l+m+mi}{1}\PYG{p}{]}\PYG{p}{)} \PYG{o}{/} \PYG{n}{returns}\PYG{o}{.}\PYG{n}{shape}\PYG{p}{[}\PYG{l+m+mi}{1}\PYG{p}{]}\PYG{p}{,}
    \PYG{n}{args}\PYG{o}{=}\PYG{p}{(}\PYG{n}{returns}\PYG{p}{,} \PYG{l+m+mi}{0}\PYG{p}{,} \PYG{l+m+mi}{252}\PYG{p}{)}\PYG{p}{,}
    \PYG{n}{bounds}\PYG{o}{=}\PYG{p}{[}\PYG{p}{(}\PYG{o}{\PYGZhy{}}\PYG{l+m+mf}{0.1}\PYG{p}{,}\PYG{l+m+mf}{1.5}\PYG{p}{)} \PYG{k}{for} \PYG{n}{\PYGZus{}} \PYG{o+ow}{in} \PYG{n}{returns}\PYG{p}{]}\PYG{p}{,} \PYG{c+c1}{\PYGZsh{} short up to 5 stocks 10\PYGZpc{}, then long the other stock 150\PYGZpc{}}
    \PYG{n}{constraints}\PYG{o}{=}\PYG{p}{(}
        \PYG{p}{\PYGZob{}}\PYG{l+s+s1}{\PYGZsq{}}\PYG{l+s+s1}{type}\PYG{l+s+s1}{\PYGZsq{}}\PYG{p}{:} \PYG{l+s+s1}{\PYGZsq{}}\PYG{l+s+s1}{eq}\PYG{l+s+s1}{\PYGZsq{}}\PYG{p}{,} \PYG{l+s+s1}{\PYGZsq{}}\PYG{l+s+s1}{fun}\PYG{l+s+s1}{\PYGZsq{}}\PYG{p}{:} \PYG{k}{lambda} \PYG{n}{x}\PYG{p}{:} \PYG{n}{x}\PYG{o}{.}\PYG{n}{sum}\PYG{p}{(}\PYG{p}{)} \PYG{o}{\PYGZhy{}} \PYG{l+m+mi}{1}\PYG{p}{\PYGZcb{}} \PYG{c+c1}{\PYGZsh{} eq constraints are driven to 0}
    \PYG{p}{)}
\PYG{p}{)}

\PYG{n}{res\PYGZus{}sharpe\PYGZus{}2}
\end{sphinxVerbatim}

\end{sphinxuseclass}\end{sphinxVerbatimInput}
\begin{sphinxVerbatimOutput}

\begin{sphinxuseclass}{cell_output}
\begin{sphinxVerbatim}[commandchars=\\\{\}]
     fun: \PYGZhy{}1.1368990243538257
     jac: array([0.041 , 0.3444, 0.1157, 0.0727, 0.041 , 0.041 ])
 message: \PYGZsq{}Optimization terminated successfully\PYGZsq{}
    nfev: 63
     nit: 9
    njev: 9
  status: 0
 success: True
       x: array([ 0.2997, \PYGZhy{}0.1   , \PYGZhy{}0.1   , \PYGZhy{}0.1   ,  0.3745,  0.6258])
\end{sphinxVerbatim}

\end{sphinxuseclass}\end{sphinxVerbatimOutput}

\end{sphinxuseclass}
\begin{sphinxuseclass}{cell}\begin{sphinxVerbatimInput}

\begin{sphinxuseclass}{cell_input}
\begin{sphinxVerbatim}[commandchars=\\\{\}]
\PYG{p}{(}
    \PYG{n}{pd}\PYG{o}{.}\PYG{n}{DataFrame}\PYG{p}{(}
        \PYG{n}{data}\PYG{o}{=}\PYG{p}{\PYGZob{}}
            \PYG{l+s+s1}{\PYGZsq{}}\PYG{l+s+s1}{Long Only}\PYG{l+s+s1}{\PYGZsq{}}\PYG{p}{:}\PYG{n}{res\PYGZus{}sharpe\PYGZus{}1}\PYG{p}{[}\PYG{l+s+s1}{\PYGZsq{}}\PYG{l+s+s1}{x}\PYG{l+s+s1}{\PYGZsq{}}\PYG{p}{]}\PYG{p}{,} 
            \PYG{l+s+s1}{\PYGZsq{}}\PYG{l+s+s1}{Up to 10}\PYG{l+s+s1}{\PYGZpc{}}\PYG{l+s+s1}{ Short per Stock}\PYG{l+s+s1}{\PYGZsq{}}\PYG{p}{:}\PYG{n}{res\PYGZus{}sharpe\PYGZus{}2}\PYG{p}{[}\PYG{l+s+s1}{\PYGZsq{}}\PYG{l+s+s1}{x}\PYG{l+s+s1}{\PYGZsq{}}\PYG{p}{]}
        \PYG{p}{\PYGZcb{}}\PYG{p}{,}
        \PYG{n}{index}\PYG{o}{=}\PYG{n}{returns}\PYG{o}{.}\PYG{n}{columns}
    \PYG{p}{)}
    \PYG{o}{.}\PYG{n}{rename\PYGZus{}axis}\PYG{p}{(}\PYG{l+s+s1}{\PYGZsq{}}\PYG{l+s+s1}{Portfolio Weight}\PYG{l+s+s1}{\PYGZsq{}}\PYG{p}{)}
    \PYG{o}{.}\PYG{n}{plot}\PYG{p}{(}\PYG{n}{kind}\PYG{o}{=}\PYG{l+s+s1}{\PYGZsq{}}\PYG{l+s+s1}{barh}\PYG{l+s+s1}{\PYGZsq{}}\PYG{p}{)}
\PYG{p}{)}
\PYG{n}{plt}\PYG{o}{.}\PYG{n}{title}\PYG{p}{(}\PYG{l+s+s1}{\PYGZsq{}}\PYG{l+s+s1}{Comparison Max. Sharpe Ratio Portfolio Weights}\PYG{l+s+s1}{\PYGZsq{}}\PYG{p}{)}
\PYG{n}{plt}\PYG{o}{.}\PYG{n}{show}\PYG{p}{(}\PYG{p}{)}
\end{sphinxVerbatim}

\end{sphinxuseclass}\end{sphinxVerbatimInput}
\begin{sphinxVerbatimOutput}

\begin{sphinxuseclass}{cell_output}
\noindent\sphinxincludegraphics{{004163d333021c4843934510f0bca8e93ff174a2d8ca8aae95ac07868ddfaaea}.png}

\end{sphinxuseclass}\end{sphinxVerbatimOutput}

\end{sphinxuseclass}
\sphinxAtStartPar
By relaxing the long\sphinxhyphen{}only constrain (via changes to \sphinxcode{\sphinxupquote{bounds=}}), the weights on AMZN, GOOG, and MSFT go from zero to \sphinxhyphen{}10\%.
Also, the Sharpe Ratio increases because we relax a binding constraint.

\begin{sphinxuseclass}{cell}\begin{sphinxVerbatimInput}

\begin{sphinxuseclass}{cell_input}
\begin{sphinxVerbatim}[commandchars=\\\{\}]
\PYG{n}{port\PYGZus{}sharpe}\PYG{p}{(}\PYG{n}{res\PYGZus{}sharpe\PYGZus{}1}\PYG{p}{[}\PYG{l+s+s1}{\PYGZsq{}}\PYG{l+s+s1}{x}\PYG{l+s+s1}{\PYGZsq{}}\PYG{p}{]}\PYG{p}{,} \PYG{n}{r}\PYG{o}{=}\PYG{n}{returns}\PYG{p}{,} \PYG{n}{tgt}\PYG{o}{=}\PYG{l+m+mi}{0}\PYG{p}{,} \PYG{n}{ppy}\PYG{o}{=}\PYG{l+m+mi}{252}\PYG{p}{)}
\end{sphinxVerbatim}

\end{sphinxuseclass}\end{sphinxVerbatimInput}
\begin{sphinxVerbatimOutput}

\begin{sphinxuseclass}{cell_output}
\begin{sphinxVerbatim}[commandchars=\\\{\}]
1.0891990229341466
\end{sphinxVerbatim}

\end{sphinxuseclass}\end{sphinxVerbatimOutput}

\end{sphinxuseclass}
\begin{sphinxuseclass}{cell}\begin{sphinxVerbatimInput}

\begin{sphinxuseclass}{cell_input}
\begin{sphinxVerbatim}[commandchars=\\\{\}]
\PYG{n}{port\PYGZus{}sharpe}\PYG{p}{(}\PYG{n}{res\PYGZus{}sharpe\PYGZus{}2}\PYG{p}{[}\PYG{l+s+s1}{\PYGZsq{}}\PYG{l+s+s1}{x}\PYG{l+s+s1}{\PYGZsq{}}\PYG{p}{]}\PYG{p}{,} \PYG{n}{r}\PYG{o}{=}\PYG{n}{returns}\PYG{p}{,} \PYG{n}{tgt}\PYG{o}{=}\PYG{l+m+mi}{0}\PYG{p}{,} \PYG{n}{ppy}\PYG{o}{=}\PYG{l+m+mi}{252}\PYG{p}{)}
\end{sphinxVerbatim}

\end{sphinxuseclass}\end{sphinxVerbatimInput}
\begin{sphinxVerbatimOutput}

\begin{sphinxuseclass}{cell_output}
\begin{sphinxVerbatim}[commandchars=\\\{\}]
1.1368990243538257
\end{sphinxVerbatim}

\end{sphinxuseclass}\end{sphinxVerbatimOutput}

\end{sphinxuseclass}

\subsubsection{Find the maximum Sharpe Ratio portfolio of MATANA stocks over the last three years, but allow total short weights of up to 30\%}
\label{\detokenize{herron_04_practice_04:find-the-maximum-sharpe-ratio-portfolio-of-matana-stocks-over-the-last-three-years-but-allow-total-short-weights-of-up-to-30}}
\sphinxAtStartPar
We can find the negative values in a NumPy array as follows.

\begin{sphinxuseclass}{cell}\begin{sphinxVerbatimInput}

\begin{sphinxuseclass}{cell_input}
\begin{sphinxVerbatim}[commandchars=\\\{\}]
\PYG{n}{x} \PYG{o}{=} \PYG{n}{np}\PYG{o}{.}\PYG{n}{arange}\PYG{p}{(}\PYG{l+m+mi}{6}\PYG{p}{)} \PYG{o}{\PYGZhy{}} \PYG{l+m+mi}{3}
\PYG{n}{x}\PYG{p}{[}\PYG{n}{x} \PYG{o}{\PYGZlt{}} \PYG{l+m+mi}{0}\PYG{p}{]}
\end{sphinxVerbatim}

\end{sphinxuseclass}\end{sphinxVerbatimInput}
\begin{sphinxVerbatimOutput}

\begin{sphinxuseclass}{cell_output}
\begin{sphinxVerbatim}[commandchars=\\\{\}]
array([\PYGZhy{}3, \PYGZhy{}2, \PYGZhy{}1])
\end{sphinxVerbatim}

\end{sphinxuseclass}\end{sphinxVerbatimOutput}

\end{sphinxuseclass}
\begin{sphinxuseclass}{cell}\begin{sphinxVerbatimInput}

\begin{sphinxuseclass}{cell_input}
\begin{sphinxVerbatim}[commandchars=\\\{\}]
\PYG{n}{res\PYGZus{}sharpe\PYGZus{}3} \PYG{o}{=} \PYG{n}{sco}\PYG{o}{.}\PYG{n}{minimize}\PYG{p}{(}
    \PYG{n}{fun}\PYG{o}{=}\PYG{n}{port\PYGZus{}sharpe\PYGZus{}neg}\PYG{p}{,}
    \PYG{n}{x0}\PYG{o}{=}\PYG{n}{np}\PYG{o}{.}\PYG{n}{ones}\PYG{p}{(}\PYG{n}{returns}\PYG{o}{.}\PYG{n}{shape}\PYG{p}{[}\PYG{l+m+mi}{1}\PYG{p}{]}\PYG{p}{)} \PYG{o}{/} \PYG{n}{returns}\PYG{o}{.}\PYG{n}{shape}\PYG{p}{[}\PYG{l+m+mi}{1}\PYG{p}{]}\PYG{p}{,}
    \PYG{n}{args}\PYG{o}{=}\PYG{p}{(}\PYG{n}{returns}\PYG{p}{,} \PYG{l+m+mi}{0}\PYG{p}{,} \PYG{l+m+mi}{252}\PYG{p}{)}\PYG{p}{,}
    \PYG{n}{bounds}\PYG{o}{=}\PYG{p}{[}\PYG{p}{(}\PYG{o}{\PYGZhy{}}\PYG{l+m+mf}{0.3}\PYG{p}{,}\PYG{l+m+mf}{1.3}\PYG{p}{)} \PYG{k}{for} \PYG{n}{\PYGZus{}} \PYG{o+ow}{in} \PYG{n}{returns}\PYG{p}{]}\PYG{p}{,} \PYG{c+c1}{\PYGZsh{} short up to 5 stocks 10\PYGZpc{}, then long the other stock 150\PYGZpc{}}
    \PYG{n}{constraints}\PYG{o}{=}\PYG{p}{(}
        \PYG{p}{\PYGZob{}}\PYG{l+s+s1}{\PYGZsq{}}\PYG{l+s+s1}{type}\PYG{l+s+s1}{\PYGZsq{}}\PYG{p}{:} \PYG{l+s+s1}{\PYGZsq{}}\PYG{l+s+s1}{eq}\PYG{l+s+s1}{\PYGZsq{}}\PYG{p}{,} \PYG{l+s+s1}{\PYGZsq{}}\PYG{l+s+s1}{fun}\PYG{l+s+s1}{\PYGZsq{}}\PYG{p}{:} \PYG{k}{lambda} \PYG{n}{x}\PYG{p}{:} \PYG{n}{x}\PYG{o}{.}\PYG{n}{sum}\PYG{p}{(}\PYG{p}{)} \PYG{o}{\PYGZhy{}} \PYG{l+m+mi}{1}\PYG{p}{\PYGZcb{}}\PYG{p}{,} \PYG{c+c1}{\PYGZsh{} eq constraints are driven to 0}
        \PYG{p}{\PYGZob{}}\PYG{l+s+s1}{\PYGZsq{}}\PYG{l+s+s1}{type}\PYG{l+s+s1}{\PYGZsq{}}\PYG{p}{:} \PYG{l+s+s1}{\PYGZsq{}}\PYG{l+s+s1}{ineq}\PYG{l+s+s1}{\PYGZsq{}}\PYG{p}{,} \PYG{l+s+s1}{\PYGZsq{}}\PYG{l+s+s1}{fun}\PYG{l+s+s1}{\PYGZsq{}}\PYG{p}{:} \PYG{k}{lambda} \PYG{n}{x}\PYG{p}{:} \PYG{n}{x}\PYG{p}{[}\PYG{n}{x}\PYG{o}{\PYGZlt{}}\PYG{l+m+mi}{0}\PYG{p}{]}\PYG{o}{.}\PYG{n}{sum}\PYG{p}{(}\PYG{p}{)} \PYG{o}{+} \PYG{l+m+mf}{0.3}\PYG{p}{\PYGZcb{}} \PYG{c+c1}{\PYGZsh{} ineq constraints must be non\PYGZhy{}negative}
    \PYG{p}{)}
\PYG{p}{)}

\PYG{n}{res\PYGZus{}sharpe\PYGZus{}3}
\end{sphinxVerbatim}

\end{sphinxuseclass}\end{sphinxVerbatimInput}
\begin{sphinxVerbatimOutput}

\begin{sphinxuseclass}{cell_output}
\begin{sphinxVerbatim}[commandchars=\\\{\}]
     fun: \PYGZhy{}1.1826806520985051
     jac: array([0.0784, 0.3361, 0.152 , 0.1097, 0.0773, 0.0772])
 message: \PYGZsq{}Optimization terminated successfully\PYGZsq{}
    nfev: 100
     nit: 10
    njev: 10
  status: 0
 success: True
       x: array([ 3.5706e\PYGZhy{}01, \PYGZhy{}3.0000e\PYGZhy{}01, \PYGZhy{}4.1390e\PYGZhy{}09,  1.8703e\PYGZhy{}05,  3.4943e\PYGZhy{}01,
        5.9349e\PYGZhy{}01])
\end{sphinxVerbatim}

\end{sphinxuseclass}\end{sphinxVerbatimOutput}

\end{sphinxuseclass}
\begin{sphinxuseclass}{cell}\begin{sphinxVerbatimInput}

\begin{sphinxuseclass}{cell_input}
\begin{sphinxVerbatim}[commandchars=\\\{\}]
\PYG{p}{(}
    \PYG{n}{pd}\PYG{o}{.}\PYG{n}{DataFrame}\PYG{p}{(}
        \PYG{n}{data}\PYG{o}{=}\PYG{p}{\PYGZob{}}
            \PYG{l+s+s1}{\PYGZsq{}}\PYG{l+s+s1}{Long Only}\PYG{l+s+s1}{\PYGZsq{}}\PYG{p}{:}\PYG{n}{res\PYGZus{}sharpe\PYGZus{}1}\PYG{p}{[}\PYG{l+s+s1}{\PYGZsq{}}\PYG{l+s+s1}{x}\PYG{l+s+s1}{\PYGZsq{}}\PYG{p}{]}\PYG{p}{,} 
            \PYG{l+s+s1}{\PYGZsq{}}\PYG{l+s+s1}{Up to 30}\PYG{l+s+s1}{\PYGZpc{}}\PYG{l+s+s1}{ Short Total}\PYG{l+s+s1}{\PYGZsq{}}\PYG{p}{:}\PYG{n}{res\PYGZus{}sharpe\PYGZus{}2}\PYG{p}{[}\PYG{l+s+s1}{\PYGZsq{}}\PYG{l+s+s1}{x}\PYG{l+s+s1}{\PYGZsq{}}\PYG{p}{]}
        \PYG{p}{\PYGZcb{}}\PYG{p}{,}
        \PYG{n}{index}\PYG{o}{=}\PYG{n}{returns}\PYG{o}{.}\PYG{n}{columns}
    \PYG{p}{)}
    \PYG{o}{.}\PYG{n}{rename\PYGZus{}axis}\PYG{p}{(}\PYG{l+s+s1}{\PYGZsq{}}\PYG{l+s+s1}{Portfolio Weight}\PYG{l+s+s1}{\PYGZsq{}}\PYG{p}{)}
    \PYG{o}{.}\PYG{n}{plot}\PYG{p}{(}\PYG{n}{kind}\PYG{o}{=}\PYG{l+s+s1}{\PYGZsq{}}\PYG{l+s+s1}{barh}\PYG{l+s+s1}{\PYGZsq{}}\PYG{p}{)}
\PYG{p}{)}
\PYG{n}{plt}\PYG{o}{.}\PYG{n}{title}\PYG{p}{(}\PYG{l+s+s1}{\PYGZsq{}}\PYG{l+s+s1}{Comparison Max. Sharpe Ratio Portfolios}\PYG{l+s+s1}{\PYGZsq{}}\PYG{p}{)}
\PYG{n}{plt}\PYG{o}{.}\PYG{n}{show}\PYG{p}{(}\PYG{p}{)}
\end{sphinxVerbatim}

\end{sphinxuseclass}\end{sphinxVerbatimInput}
\begin{sphinxVerbatimOutput}

\begin{sphinxuseclass}{cell_output}
\noindent\sphinxincludegraphics{{9191ee6b277076599561ec2ccfc4e2c05a524e1504c24b98f5dfcbe9c357f2a1}.png}

\end{sphinxuseclass}\end{sphinxVerbatimOutput}

\end{sphinxuseclass}
\sphinxAtStartPar
Again, by relaxing the long\sphinxhyphen{}only constrain, the weights on AMZN, GOOG, and MSFT go from zero to \sphinxhyphen{}10\%.
Also, the Sharpe Ratio increases because we relax a binding constraints.
The Sharpe Ratio is higher here than in the previous exercise, but this will not always be the case, since we relax different constraints here and in the previous exercise.

\begin{sphinxuseclass}{cell}\begin{sphinxVerbatimInput}

\begin{sphinxuseclass}{cell_input}
\begin{sphinxVerbatim}[commandchars=\\\{\}]
\PYG{n}{port\PYGZus{}sharpe}\PYG{p}{(}\PYG{n}{res\PYGZus{}sharpe\PYGZus{}1}\PYG{p}{[}\PYG{l+s+s1}{\PYGZsq{}}\PYG{l+s+s1}{x}\PYG{l+s+s1}{\PYGZsq{}}\PYG{p}{]}\PYG{p}{,} \PYG{n}{r}\PYG{o}{=}\PYG{n}{returns}\PYG{p}{,} \PYG{n}{ppy}\PYG{o}{=}\PYG{l+m+mi}{252}\PYG{p}{,} \PYG{n}{tgt}\PYG{o}{=}\PYG{l+m+mi}{0}\PYG{p}{)}
\end{sphinxVerbatim}

\end{sphinxuseclass}\end{sphinxVerbatimInput}
\begin{sphinxVerbatimOutput}

\begin{sphinxuseclass}{cell_output}
\begin{sphinxVerbatim}[commandchars=\\\{\}]
1.0891990229341466
\end{sphinxVerbatim}

\end{sphinxuseclass}\end{sphinxVerbatimOutput}

\end{sphinxuseclass}
\begin{sphinxuseclass}{cell}\begin{sphinxVerbatimInput}

\begin{sphinxuseclass}{cell_input}
\begin{sphinxVerbatim}[commandchars=\\\{\}]
\PYG{n}{port\PYGZus{}sharpe}\PYG{p}{(}\PYG{n}{res\PYGZus{}sharpe\PYGZus{}3}\PYG{p}{[}\PYG{l+s+s1}{\PYGZsq{}}\PYG{l+s+s1}{x}\PYG{l+s+s1}{\PYGZsq{}}\PYG{p}{]}\PYG{p}{,} \PYG{n}{r}\PYG{o}{=}\PYG{n}{returns}\PYG{p}{,} \PYG{n}{ppy}\PYG{o}{=}\PYG{l+m+mi}{252}\PYG{p}{,} \PYG{n}{tgt}\PYG{o}{=}\PYG{l+m+mi}{0}\PYG{p}{)}
\end{sphinxVerbatim}

\end{sphinxuseclass}\end{sphinxVerbatimInput}
\begin{sphinxVerbatimOutput}

\begin{sphinxuseclass}{cell_output}
\begin{sphinxVerbatim}[commandchars=\\\{\}]
1.1826806520985051
\end{sphinxVerbatim}

\end{sphinxuseclass}\end{sphinxVerbatimOutput}

\end{sphinxuseclass}

\subsubsection{Find the maximum Sharpe Ratio portfolio of MATANA stocks over the last three years, but do not allow any weight to exceed 30\% in magnitude}
\label{\detokenize{herron_04_practice_04:find-the-maximum-sharpe-ratio-portfolio-of-matana-stocks-over-the-last-three-years-but-do-not-allow-any-weight-to-exceed-30-in-magnitude}}
\begin{sphinxuseclass}{cell}\begin{sphinxVerbatimInput}

\begin{sphinxuseclass}{cell_input}
\begin{sphinxVerbatim}[commandchars=\\\{\}]
\PYG{n}{res\PYGZus{}sharpe\PYGZus{}4} \PYG{o}{=} \PYG{n}{sco}\PYG{o}{.}\PYG{n}{minimize}\PYG{p}{(}
    \PYG{n}{fun}\PYG{o}{=}\PYG{n}{port\PYGZus{}sharpe\PYGZus{}neg}\PYG{p}{,}
    \PYG{n}{x0}\PYG{o}{=}\PYG{n}{np}\PYG{o}{.}\PYG{n}{ones}\PYG{p}{(}\PYG{n}{returns}\PYG{o}{.}\PYG{n}{shape}\PYG{p}{[}\PYG{l+m+mi}{1}\PYG{p}{]}\PYG{p}{)} \PYG{o}{/} \PYG{n}{returns}\PYG{o}{.}\PYG{n}{shape}\PYG{p}{[}\PYG{l+m+mi}{1}\PYG{p}{]}\PYG{p}{,}
    \PYG{n}{args}\PYG{o}{=}\PYG{p}{(}\PYG{n}{returns}\PYG{p}{,} \PYG{l+m+mi}{252}\PYG{p}{,} \PYG{l+m+mi}{0}\PYG{p}{)}\PYG{p}{,}
    \PYG{n}{tol}\PYG{o}{=}\PYG{l+m+mf}{1e\PYGZhy{}6}\PYG{p}{,}
    \PYG{n}{bounds}\PYG{o}{=}\PYG{p}{[}\PYG{p}{(}\PYG{l+m+mi}{0}\PYG{p}{,}\PYG{l+m+mf}{0.3}\PYG{p}{)} \PYG{k}{for} \PYG{n}{\PYGZus{}} \PYG{o+ow}{in} \PYG{n+nb}{range}\PYG{p}{(}\PYG{n}{returns}\PYG{o}{.}\PYG{n}{shape}\PYG{p}{[}\PYG{l+m+mi}{1}\PYG{p}{]}\PYG{p}{)}\PYG{p}{]}\PYG{p}{,}
    \PYG{n}{constraints}\PYG{o}{=}\PYG{p}{(}
        \PYG{p}{\PYGZob{}}\PYG{l+s+s1}{\PYGZsq{}}\PYG{l+s+s1}{type}\PYG{l+s+s1}{\PYGZsq{}}\PYG{p}{:} \PYG{l+s+s1}{\PYGZsq{}}\PYG{l+s+s1}{eq}\PYG{l+s+s1}{\PYGZsq{}}\PYG{p}{,} \PYG{l+s+s1}{\PYGZsq{}}\PYG{l+s+s1}{fun}\PYG{l+s+s1}{\PYGZsq{}}\PYG{p}{:} \PYG{k}{lambda} \PYG{n}{x}\PYG{p}{:} \PYG{n}{x}\PYG{o}{.}\PYG{n}{sum}\PYG{p}{(}\PYG{p}{)} \PYG{o}{\PYGZhy{}} \PYG{l+m+mi}{1}\PYG{p}{\PYGZcb{}}\PYG{p}{,} \PYG{c+c1}{\PYGZsh{} want eq constraint to = 0}
    \PYG{p}{)}
\PYG{p}{)}

\PYG{n}{res\PYGZus{}sharpe\PYGZus{}4}
\end{sphinxVerbatim}

\end{sphinxuseclass}\end{sphinxVerbatimInput}
\begin{sphinxVerbatimOutput}

\begin{sphinxuseclass}{cell_output}
\begin{sphinxVerbatim}[commandchars=\\\{\}]
     fun: 0.0
     jac: array([0., 0., 0., 0., 0., 0.])
 message: \PYGZsq{}Optimization terminated successfully\PYGZsq{}
    nfev: 7
     nit: 1
    njev: 1
  status: 0
 success: True
       x: array([0.1667, 0.1667, 0.1667, 0.1667, 0.1667, 0.1667])
\end{sphinxVerbatim}

\end{sphinxuseclass}\end{sphinxVerbatimOutput}

\end{sphinxuseclass}

\subsubsection{Find the minimum 95\% Value at Risk (Var) portfolio of MATANA stocks over the last three years}
\label{\detokenize{herron_04_practice_04:find-the-minimum-95-value-at-risk-var-portfolio-of-matana-stocks-over-the-last-three-years}}
\sphinxAtStartPar
More on VaR \sphinxhref{https://en.wikipedia.org/wiki/Value\_at\_risk}{here}.

\begin{sphinxuseclass}{cell}\begin{sphinxVerbatimInput}

\begin{sphinxuseclass}{cell_input}
\begin{sphinxVerbatim}[commandchars=\\\{\}]
\PYG{k}{def} \PYG{n+nf}{port\PYGZus{}var}\PYG{p}{(}\PYG{n}{x}\PYG{p}{,} \PYG{n}{r}\PYG{p}{,} \PYG{n}{q}\PYG{p}{)}\PYG{p}{:}
    \PYG{k}{return} \PYG{n}{r}\PYG{o}{.}\PYG{n}{dot}\PYG{p}{(}\PYG{n}{x}\PYG{p}{)}\PYG{o}{.}\PYG{n}{quantile}\PYG{p}{(}\PYG{n}{q}\PYG{p}{)}
\end{sphinxVerbatim}

\end{sphinxuseclass}\end{sphinxVerbatimInput}

\end{sphinxuseclass}
\begin{sphinxuseclass}{cell}\begin{sphinxVerbatimInput}

\begin{sphinxuseclass}{cell_input}
\begin{sphinxVerbatim}[commandchars=\\\{\}]
\PYG{k}{def} \PYG{n+nf}{port\PYGZus{}var\PYGZus{}neg}\PYG{p}{(}\PYG{n}{x}\PYG{p}{,} \PYG{n}{r}\PYG{p}{,} \PYG{n}{q}\PYG{p}{)}\PYG{p}{:}
    \PYG{k}{return} \PYG{o}{\PYGZhy{}}\PYG{l+m+mi}{1} \PYG{o}{*} \PYG{n}{port\PYGZus{}var}\PYG{p}{(}\PYG{n}{x}\PYG{o}{=}\PYG{n}{x}\PYG{p}{,} \PYG{n}{r}\PYG{o}{=}\PYG{n}{r}\PYG{p}{,} \PYG{n}{q}\PYG{o}{=}\PYG{n}{q}\PYG{p}{)}
\end{sphinxVerbatim}

\end{sphinxuseclass}\end{sphinxVerbatimInput}

\end{sphinxuseclass}
\begin{sphinxuseclass}{cell}\begin{sphinxVerbatimInput}

\begin{sphinxuseclass}{cell_input}
\begin{sphinxVerbatim}[commandchars=\\\{\}]
\PYG{n}{res\PYGZus{}var\PYGZus{}1} \PYG{o}{=} \PYG{n}{sco}\PYG{o}{.}\PYG{n}{minimize}\PYG{p}{(}
    \PYG{n}{fun}\PYG{o}{=}\PYG{n}{port\PYGZus{}var\PYGZus{}neg}\PYG{p}{,}
    \PYG{n}{x0}\PYG{o}{=}\PYG{n}{np}\PYG{o}{.}\PYG{n}{ones}\PYG{p}{(}\PYG{n}{returns}\PYG{o}{.}\PYG{n}{shape}\PYG{p}{[}\PYG{l+m+mi}{1}\PYG{p}{]}\PYG{p}{)} \PYG{o}{/} \PYG{n}{returns}\PYG{o}{.}\PYG{n}{shape}\PYG{p}{[}\PYG{l+m+mi}{1}\PYG{p}{]}\PYG{p}{,}
    \PYG{n}{args}\PYG{o}{=}\PYG{p}{(}\PYG{n}{returns}\PYG{p}{,} \PYG{l+m+mf}{0.05}\PYG{p}{)}\PYG{p}{,}
    \PYG{n}{bounds}\PYG{o}{=}\PYG{p}{[}\PYG{p}{(}\PYG{l+m+mi}{0}\PYG{p}{,}\PYG{l+m+mi}{1}\PYG{p}{)} \PYG{k}{for} \PYG{n}{\PYGZus{}} \PYG{o+ow}{in} \PYG{n}{returns}\PYG{p}{]}\PYG{p}{,}
    \PYG{n}{constraints}\PYG{o}{=}\PYG{p}{(}
        \PYG{p}{\PYGZob{}}\PYG{l+s+s1}{\PYGZsq{}}\PYG{l+s+s1}{type}\PYG{l+s+s1}{\PYGZsq{}}\PYG{p}{:} \PYG{l+s+s1}{\PYGZsq{}}\PYG{l+s+s1}{eq}\PYG{l+s+s1}{\PYGZsq{}}\PYG{p}{,} \PYG{l+s+s1}{\PYGZsq{}}\PYG{l+s+s1}{fun}\PYG{l+s+s1}{\PYGZsq{}}\PYG{p}{:} \PYG{k}{lambda} \PYG{n}{x}\PYG{p}{:} \PYG{n}{x}\PYG{o}{.}\PYG{n}{sum}\PYG{p}{(}\PYG{p}{)} \PYG{o}{\PYGZhy{}} \PYG{l+m+mi}{1}\PYG{p}{\PYGZcb{}}\PYG{p}{,} \PYG{c+c1}{\PYGZsh{} minimize drives \PYGZdq{}eq\PYGZdq{} constraints to zero}
    \PYG{p}{)}
\PYG{p}{)}

\PYG{n}{res\PYGZus{}var\PYGZus{}1}
\end{sphinxVerbatim}

\end{sphinxuseclass}\end{sphinxVerbatimInput}
\begin{sphinxVerbatimOutput}

\begin{sphinxuseclass}{cell_output}
\begin{sphinxVerbatim}[commandchars=\\\{\}]
     fun: 0.03592251751051852
     jac: array([0.0359, 0.0419, 0.037 , 0.0385, 0.0275, 0.0416])
 message: \PYGZsq{}Optimization terminated successfully\PYGZsq{}
    nfev: 45
     nit: 5
    njev: 5
  status: 0
 success: True
       x: array([1.8180e\PYGZhy{}01, 1.4369e\PYGZhy{}01, 2.2517e\PYGZhy{}01, 2.4347e\PYGZhy{}01, 2.0587e\PYGZhy{}01,
       4.4129e\PYGZhy{}19])
\end{sphinxVerbatim}

\end{sphinxuseclass}\end{sphinxVerbatimOutput}

\end{sphinxuseclass}
\begin{sphinxuseclass}{cell}\begin{sphinxVerbatimInput}

\begin{sphinxuseclass}{cell_input}
\begin{sphinxVerbatim}[commandchars=\\\{\}]
\PYG{n}{port\PYGZus{}var}\PYG{p}{(}\PYG{n}{x}\PYG{o}{=}\PYG{n}{res\PYGZus{}var\PYGZus{}1}\PYG{p}{[}\PYG{l+s+s1}{\PYGZsq{}}\PYG{l+s+s1}{x}\PYG{l+s+s1}{\PYGZsq{}}\PYG{p}{]}\PYG{p}{,} \PYG{n}{r}\PYG{o}{=}\PYG{n}{returns}\PYG{p}{,} \PYG{n}{q}\PYG{o}{=}\PYG{l+m+mf}{0.05}\PYG{p}{)}
\end{sphinxVerbatim}

\end{sphinxuseclass}\end{sphinxVerbatimInput}
\begin{sphinxVerbatimOutput}

\begin{sphinxuseclass}{cell_output}
\begin{sphinxVerbatim}[commandchars=\\\{\}]
\PYGZhy{}0.03592251751051852
\end{sphinxVerbatim}

\end{sphinxuseclass}\end{sphinxVerbatimOutput}

\end{sphinxuseclass}
\sphinxAtStartPar
It might be helpful to slightly change then minimum VaR portfolio weights to show that we minimized VaR.

\begin{sphinxuseclass}{cell}\begin{sphinxVerbatimInput}

\begin{sphinxuseclass}{cell_input}
\begin{sphinxVerbatim}[commandchars=\\\{\}]
\PYG{k}{def} \PYG{n+nf}{tweak}\PYG{p}{(}\PYG{n}{x}\PYG{p}{,} \PYG{n}{d}\PYG{o}{=}\PYG{l+m+mf}{0.05}\PYG{p}{)}\PYG{p}{:}
    \PYG{n}{y} \PYG{o}{=} \PYG{n}{np}\PYG{o}{.}\PYG{n}{zeros}\PYG{p}{(}\PYG{n}{x}\PYG{o}{.}\PYG{n}{shape}\PYG{p}{[}\PYG{l+m+mi}{0}\PYG{p}{]}\PYG{p}{)}
    \PYG{n}{y}\PYG{p}{[}\PYG{l+m+mi}{0}\PYG{p}{]}\PYG{p}{,} \PYG{n}{y}\PYG{p}{[}\PYG{l+m+mi}{1}\PYG{p}{]} \PYG{o}{=} \PYG{n}{d}\PYG{p}{,} \PYG{o}{\PYGZhy{}}\PYG{l+m+mi}{1} \PYG{o}{*} \PYG{n}{d}
    \PYG{k}{return} \PYG{n}{x} \PYG{o}{+} \PYG{n}{y}
\end{sphinxVerbatim}

\end{sphinxuseclass}\end{sphinxVerbatimInput}

\end{sphinxuseclass}
\begin{sphinxuseclass}{cell}\begin{sphinxVerbatimInput}

\begin{sphinxuseclass}{cell_input}
\begin{sphinxVerbatim}[commandchars=\\\{\}]
\PYG{n}{port\PYGZus{}var}\PYG{p}{(}\PYG{n}{x}\PYG{o}{=}\PYG{n}{tweak}\PYG{p}{(}\PYG{n}{res\PYGZus{}var\PYGZus{}1}\PYG{p}{[}\PYG{l+s+s1}{\PYGZsq{}}\PYG{l+s+s1}{x}\PYG{l+s+s1}{\PYGZsq{}}\PYG{p}{]}\PYG{p}{)}\PYG{p}{,} \PYG{n}{r}\PYG{o}{=}\PYG{n}{returns}\PYG{p}{,} \PYG{n}{q}\PYG{o}{=}\PYG{l+m+mf}{0.05}\PYG{p}{)}
\end{sphinxVerbatim}

\end{sphinxuseclass}\end{sphinxVerbatimInput}
\begin{sphinxVerbatimOutput}

\begin{sphinxuseclass}{cell_output}
\begin{sphinxVerbatim}[commandchars=\\\{\}]
\PYGZhy{}0.036229497993286966
\end{sphinxVerbatim}

\end{sphinxuseclass}\end{sphinxVerbatimOutput}

\end{sphinxuseclass}

\subsubsection{Find the minimum maximum draw down portfolio of MATANA stocks over the last three years}
\label{\detokenize{herron_04_practice_04:find-the-minimum-maximum-draw-down-portfolio-of-matana-stocks-over-the-last-three-years}}
\begin{sphinxuseclass}{cell}\begin{sphinxVerbatimInput}

\begin{sphinxuseclass}{cell_input}
\begin{sphinxVerbatim}[commandchars=\\\{\}]
\PYG{k}{def} \PYG{n+nf}{port\PYGZus{}draw\PYGZus{}down\PYGZus{}max}\PYG{p}{(}\PYG{n}{x}\PYG{p}{,} \PYG{n}{r}\PYG{p}{)}\PYG{p}{:}
    \PYG{n}{rp} \PYG{o}{=} \PYG{n}{r}\PYG{o}{.}\PYG{n}{dot}\PYG{p}{(}\PYG{n}{x}\PYG{p}{)}
    \PYG{n}{price} \PYG{o}{=} \PYG{n}{rp}\PYG{o}{.}\PYG{n}{add}\PYG{p}{(}\PYG{l+m+mi}{1}\PYG{p}{)}\PYG{o}{.}\PYG{n}{cumprod}\PYG{p}{(}\PYG{p}{)}
    \PYG{n}{cum\PYGZus{}max} \PYG{o}{=} \PYG{n}{price}\PYG{o}{.}\PYG{n}{cummax}\PYG{p}{(}\PYG{p}{)}
    \PYG{n}{draw\PYGZus{}down} \PYG{o}{=} \PYG{p}{(}\PYG{n}{cum\PYGZus{}max} \PYG{o}{\PYGZhy{}} \PYG{n}{price}\PYG{p}{)} \PYG{o}{/} \PYG{n}{cum\PYGZus{}max}
    \PYG{k}{return} \PYG{n}{draw\PYGZus{}down}\PYG{o}{.}\PYG{n}{max}\PYG{p}{(}\PYG{p}{)}
\end{sphinxVerbatim}

\end{sphinxuseclass}\end{sphinxVerbatimInput}

\end{sphinxuseclass}
\begin{sphinxuseclass}{cell}\begin{sphinxVerbatimInput}

\begin{sphinxuseclass}{cell_input}
\begin{sphinxVerbatim}[commandchars=\\\{\}]
\PYG{n}{res\PYGZus{}dd\PYGZus{}1} \PYG{o}{=} \PYG{n}{sco}\PYG{o}{.}\PYG{n}{minimize}\PYG{p}{(}
    \PYG{n}{fun}\PYG{o}{=}\PYG{n}{port\PYGZus{}draw\PYGZus{}down\PYGZus{}max}\PYG{p}{,}
    \PYG{n}{x0}\PYG{o}{=}\PYG{n}{np}\PYG{o}{.}\PYG{n}{ones}\PYG{p}{(}\PYG{n}{returns}\PYG{o}{.}\PYG{n}{shape}\PYG{p}{[}\PYG{l+m+mi}{1}\PYG{p}{]}\PYG{p}{)} \PYG{o}{/} \PYG{n}{returns}\PYG{o}{.}\PYG{n}{shape}\PYG{p}{[}\PYG{l+m+mi}{1}\PYG{p}{]}\PYG{p}{,}
    \PYG{n}{args}\PYG{o}{=}\PYG{p}{(}\PYG{n}{returns}\PYG{p}{,}\PYG{p}{)}\PYG{p}{,}
    \PYG{n}{bounds}\PYG{o}{=}\PYG{p}{[}\PYG{p}{(}\PYG{l+m+mi}{0}\PYG{p}{,}\PYG{l+m+mi}{1}\PYG{p}{)} \PYG{k}{for} \PYG{n}{\PYGZus{}} \PYG{o+ow}{in} \PYG{n}{returns}\PYG{p}{]}\PYG{p}{,}
    \PYG{n}{constraints}\PYG{o}{=}\PYG{p}{(}
        \PYG{p}{\PYGZob{}}\PYG{l+s+s1}{\PYGZsq{}}\PYG{l+s+s1}{type}\PYG{l+s+s1}{\PYGZsq{}}\PYG{p}{:} \PYG{l+s+s1}{\PYGZsq{}}\PYG{l+s+s1}{eq}\PYG{l+s+s1}{\PYGZsq{}}\PYG{p}{,} \PYG{l+s+s1}{\PYGZsq{}}\PYG{l+s+s1}{fun}\PYG{l+s+s1}{\PYGZsq{}}\PYG{p}{:} \PYG{k}{lambda} \PYG{n}{x}\PYG{p}{:} \PYG{n}{x}\PYG{o}{.}\PYG{n}{sum}\PYG{p}{(}\PYG{p}{)} \PYG{o}{\PYGZhy{}} \PYG{l+m+mi}{1}\PYG{p}{\PYGZcb{}}\PYG{p}{,} \PYG{c+c1}{\PYGZsh{} minimize drives \PYGZdq{}eq\PYGZdq{} constraints to zero}
    \PYG{p}{)}
\PYG{p}{)}

\PYG{n}{res\PYGZus{}dd\PYGZus{}1}
\end{sphinxVerbatim}

\end{sphinxuseclass}\end{sphinxVerbatimInput}
\begin{sphinxVerbatimOutput}

\begin{sphinxuseclass}{cell_output}
\begin{sphinxVerbatim}[commandchars=\\\{\}]
     fun: 0.2973740650996654
     jac: array([0.2865, 0.5131, 0.4028, 0.2927, 0.5359, 0.7698])
 message: \PYGZsq{}Optimization terminated successfully\PYGZsq{}
    nfev: 164
     nit: 19
    njev: 19
  status: 0
 success: True
       x: array([6.8105e\PYGZhy{}01, 3.7577e\PYGZhy{}16, 1.3585e\PYGZhy{}16, 3.1895e\PYGZhy{}01, 7.7695e\PYGZhy{}16,
       7.2563e\PYGZhy{}16])
\end{sphinxVerbatim}

\end{sphinxuseclass}\end{sphinxVerbatimOutput}

\end{sphinxuseclass}
\begin{sphinxuseclass}{cell}\begin{sphinxVerbatimInput}

\begin{sphinxuseclass}{cell_input}
\begin{sphinxVerbatim}[commandchars=\\\{\}]
\PYG{n}{port\PYGZus{}draw\PYGZus{}down\PYGZus{}max}\PYG{p}{(}\PYG{n}{x}\PYG{o}{=}\PYG{n}{res\PYGZus{}dd\PYGZus{}1}\PYG{p}{[}\PYG{l+s+s1}{\PYGZsq{}}\PYG{l+s+s1}{x}\PYG{l+s+s1}{\PYGZsq{}}\PYG{p}{]}\PYG{p}{,} \PYG{n}{r}\PYG{o}{=}\PYG{n}{returns}\PYG{p}{)}
\end{sphinxVerbatim}

\end{sphinxuseclass}\end{sphinxVerbatimInput}
\begin{sphinxVerbatimOutput}

\begin{sphinxuseclass}{cell_output}
\begin{sphinxVerbatim}[commandchars=\\\{\}]
0.2974
\end{sphinxVerbatim}

\end{sphinxuseclass}\end{sphinxVerbatimOutput}

\end{sphinxuseclass}
\sphinxAtStartPar
Again. it might be helpful to slightly change then minimum VaR portfolio weights to show that we minimized VaR.

\begin{sphinxuseclass}{cell}\begin{sphinxVerbatimInput}

\begin{sphinxuseclass}{cell_input}
\begin{sphinxVerbatim}[commandchars=\\\{\}]
\PYG{n}{port\PYGZus{}draw\PYGZus{}down\PYGZus{}max}\PYG{p}{(}\PYG{n}{x}\PYG{o}{=}\PYG{n}{tweak}\PYG{p}{(}\PYG{n}{res\PYGZus{}dd\PYGZus{}1}\PYG{p}{[}\PYG{l+s+s1}{\PYGZsq{}}\PYG{l+s+s1}{x}\PYG{l+s+s1}{\PYGZsq{}}\PYG{p}{]}\PYG{p}{)}\PYG{p}{,} \PYG{n}{r}\PYG{o}{=}\PYG{n}{returns}\PYG{p}{)}
\end{sphinxVerbatim}

\end{sphinxuseclass}\end{sphinxVerbatimInput}
\begin{sphinxVerbatimOutput}

\begin{sphinxuseclass}{cell_output}
\begin{sphinxVerbatim}[commandchars=\\\{\}]
0.3065
\end{sphinxVerbatim}

\end{sphinxuseclass}\end{sphinxVerbatimOutput}

\end{sphinxuseclass}

\subsubsection{Find the minimum maximum draw down portfolio with all available data for the current Dow\sphinxhyphen{}Jones Industrial Average (DJIA) stocks}
\label{\detokenize{herron_04_practice_04:find-the-minimum-maximum-draw-down-portfolio-with-all-available-data-for-the-current-dow-jones-industrial-average-djia-stocks}}
\sphinxAtStartPar
You can find the \sphinxhref{https://en.wikipedia.org/wiki/Dow\_Jones\_Industrial\_Average}{DJIA tickers on Wikipedia}.

\begin{sphinxuseclass}{cell}\begin{sphinxVerbatimInput}

\begin{sphinxuseclass}{cell_input}
\begin{sphinxVerbatim}[commandchars=\\\{\}]
\PYG{n}{wiki} \PYG{o}{=} \PYG{n}{pd}\PYG{o}{.}\PYG{n}{read\PYGZus{}html}\PYG{p}{(}\PYG{l+s+s1}{\PYGZsq{}}\PYG{l+s+s1}{https://en.wikipedia.org/wiki/Dow\PYGZus{}Jones\PYGZus{}Industrial\PYGZus{}Average}\PYG{l+s+s1}{\PYGZsq{}}\PYG{p}{)}
\PYG{n}{tickers} \PYG{o}{=} \PYG{n}{wiki}\PYG{p}{[}\PYG{l+m+mi}{1}\PYG{p}{]}\PYG{p}{[}\PYG{l+s+s1}{\PYGZsq{}}\PYG{l+s+s1}{Symbol}\PYG{l+s+s1}{\PYGZsq{}}\PYG{p}{]}\PYG{o}{.}\PYG{n}{to\PYGZus{}list}\PYG{p}{(}\PYG{p}{)}

\PYG{n}{djia} \PYG{o}{=} \PYG{p}{(}
    \PYG{n}{yf}\PYG{o}{.}\PYG{n}{download}\PYG{p}{(}\PYG{n}{tickers}\PYG{o}{=}\PYG{n}{tickers}\PYG{p}{,} \PYG{n}{progress}\PYG{o}{=}\PYG{k+kc}{False}\PYG{p}{)}
    \PYG{o}{.}\PYG{n}{assign}\PYG{p}{(}\PYG{n}{Date}\PYG{o}{=}\PYG{k}{lambda} \PYG{n}{x}\PYG{p}{:} \PYG{n}{x}\PYG{o}{.}\PYG{n}{index}\PYG{o}{.}\PYG{n}{tz\PYGZus{}localize}\PYG{p}{(}\PYG{k+kc}{None}\PYG{p}{)}\PYG{p}{)}
    \PYG{o}{.}\PYG{n}{set\PYGZus{}index}\PYG{p}{(}\PYG{l+s+s1}{\PYGZsq{}}\PYG{l+s+s1}{Date}\PYG{l+s+s1}{\PYGZsq{}}\PYG{p}{)}
    \PYG{o}{.}\PYG{n}{rename\PYGZus{}axis}\PYG{p}{(}\PYG{n}{columns}\PYG{o}{=}\PYG{p}{[}\PYG{l+s+s1}{\PYGZsq{}}\PYG{l+s+s1}{Variable}\PYG{l+s+s1}{\PYGZsq{}}\PYG{p}{,} \PYG{l+s+s1}{\PYGZsq{}}\PYG{l+s+s1}{Ticker}\PYG{l+s+s1}{\PYGZsq{}}\PYG{p}{]}\PYG{p}{)}
\PYG{p}{)}

\PYG{n}{returns\PYGZus{}2} \PYG{o}{=} \PYG{n}{djia}\PYG{p}{[}\PYG{l+s+s1}{\PYGZsq{}}\PYG{l+s+s1}{Adj Close}\PYG{l+s+s1}{\PYGZsq{}}\PYG{p}{]}\PYG{o}{.}\PYG{n}{pct\PYGZus{}change}\PYG{p}{(}\PYG{p}{)}\PYG{o}{.}\PYG{n}{loc}\PYG{p}{[}\PYG{l+s+s1}{\PYGZsq{}}\PYG{l+s+s1}{2020}\PYG{l+s+s1}{\PYGZsq{}}\PYG{p}{:}\PYG{l+s+s1}{\PYGZsq{}}\PYG{l+s+s1}{2022}\PYG{l+s+s1}{\PYGZsq{}}\PYG{p}{]}
\PYG{n}{returns\PYGZus{}2}\PYG{o}{.}\PYG{n}{describe}\PYG{p}{(}\PYG{p}{)}
\end{sphinxVerbatim}

\end{sphinxuseclass}\end{sphinxVerbatimInput}
\begin{sphinxVerbatimOutput}

\begin{sphinxuseclass}{cell_output}
\begin{sphinxVerbatim}[commandchars=\\\{\}]
Ticker     AAPL     AMGN      AXP       BA      CAT      CRM     CSCO  \PYGZbs{}
count  756.0000 756.0000 756.0000 756.0000 756.0000 756.0000 756.0000   
mean     0.0011   0.0004   0.0007   0.0000   0.0010   0.0001   0.0003   
std      0.0233   0.0176   0.0286   0.0383   0.0224   0.0278   0.0199   
min     \PYGZhy{}0.1286  \PYGZhy{}0.0826  \PYGZhy{}0.1482  \PYGZhy{}0.2385  \PYGZhy{}0.1428  \PYGZhy{}0.1589  \PYGZhy{}0.1373   
25\PYGZpc{}     \PYGZhy{}0.0110  \PYGZhy{}0.0080  \PYGZhy{}0.0130  \PYGZhy{}0.0173  \PYGZhy{}0.0106  \PYGZhy{}0.0137  \PYGZhy{}0.0085   
50\PYGZpc{}      0.0006  \PYGZhy{}0.0003   0.0004  \PYGZhy{}0.0014   0.0010   0.0005   0.0000   
75\PYGZpc{}      0.0142   0.0081   0.0123   0.0152   0.0123   0.0149   0.0093   
max      0.1198   0.1090   0.2188   0.2432   0.1033   0.2604   0.1337   

Ticker      CVX      DIS      DOW  ...      MRK     MSFT      NKE       PG  \PYGZbs{}
count  756.0000 756.0000 756.0000  ... 756.0000 756.0000 756.0000 756.0000   
mean     0.0011  \PYGZhy{}0.0004   0.0005  ...   0.0006   0.0008   0.0005   0.0005   
std      0.0268   0.0241   0.0272  ...   0.0161   0.0219   0.0238   0.0152   
min     \PYGZhy{}0.2212  \PYGZhy{}0.1316  \PYGZhy{}0.2166  ...  \PYGZhy{}0.0986  \PYGZhy{}0.1474  \PYGZhy{}0.1281  \PYGZhy{}0.0874   
25\PYGZpc{}     \PYGZhy{}0.0108  \PYGZhy{}0.0123  \PYGZhy{}0.0124  ...  \PYGZhy{}0.0073  \PYGZhy{}0.0097  \PYGZhy{}0.0104  \PYGZhy{}0.0060   
50\PYGZpc{}      0.0008  \PYGZhy{}0.0012  \PYGZhy{}0.0002  ...   0.0000   0.0007   0.0000   0.0008   
75\PYGZpc{}      0.0133   0.0106   0.0142  ...   0.0084   0.0123   0.0123   0.0075   
max      0.2274   0.1441   0.2091  ...   0.0837   0.1422   0.1553   0.1201   

Ticker      TRV      UNH        V       VZ      WBA      WMT  
count  756.0000 756.0000 756.0000 756.0000 756.0000 756.0000  
mean     0.0007   0.0011   0.0004  \PYGZhy{}0.0003  \PYGZhy{}0.0002   0.0004  
std      0.0210   0.0208   0.0211   0.0132   0.0231   0.0162  
min     \PYGZhy{}0.2080  \PYGZhy{}0.1728  \PYGZhy{}0.1355  \PYGZhy{}0.0674  \PYGZhy{}0.1096  \PYGZhy{}0.1138  
25\PYGZpc{}     \PYGZhy{}0.0087  \PYGZhy{}0.0080  \PYGZhy{}0.0099  \PYGZhy{}0.0061  \PYGZhy{}0.0104  \PYGZhy{}0.0067  
50\PYGZpc{}      0.0017   0.0011   0.0007  \PYGZhy{}0.0002  \PYGZhy{}0.0007   0.0001  
75\PYGZpc{}      0.0101   0.0095   0.0106   0.0054   0.0107   0.0077  
max      0.1329   0.1280   0.1384   0.0721   0.1260   0.1171  

[8 rows x 30 columns]
\end{sphinxVerbatim}

\end{sphinxuseclass}\end{sphinxVerbatimOutput}

\end{sphinxuseclass}
\begin{sphinxuseclass}{cell}\begin{sphinxVerbatimInput}

\begin{sphinxuseclass}{cell_input}
\begin{sphinxVerbatim}[commandchars=\\\{\}]
\PYG{n}{res\PYGZus{}dd\PYGZus{}2} \PYG{o}{=} \PYG{n}{sco}\PYG{o}{.}\PYG{n}{minimize}\PYG{p}{(}
    \PYG{n}{fun}\PYG{o}{=}\PYG{n}{port\PYGZus{}draw\PYGZus{}down\PYGZus{}max}\PYG{p}{,}
    \PYG{n}{x0}\PYG{o}{=}\PYG{n}{np}\PYG{o}{.}\PYG{n}{ones}\PYG{p}{(}\PYG{n}{returns\PYGZus{}2}\PYG{o}{.}\PYG{n}{shape}\PYG{p}{[}\PYG{l+m+mi}{1}\PYG{p}{]}\PYG{p}{)} \PYG{o}{/} \PYG{n}{returns\PYGZus{}2}\PYG{o}{.}\PYG{n}{shape}\PYG{p}{[}\PYG{l+m+mi}{1}\PYG{p}{]}\PYG{p}{,}
    \PYG{n}{args}\PYG{o}{=}\PYG{p}{(}\PYG{n}{returns\PYGZus{}2}\PYG{p}{,}\PYG{p}{)}\PYG{p}{,}
    \PYG{n}{bounds}\PYG{o}{=}\PYG{p}{[}\PYG{p}{(}\PYG{l+m+mi}{0}\PYG{p}{,}\PYG{l+m+mi}{1}\PYG{p}{)} \PYG{k}{for} \PYG{n}{\PYGZus{}} \PYG{o+ow}{in} \PYG{n}{returns\PYGZus{}2}\PYG{p}{]}\PYG{p}{,}
    \PYG{n}{constraints}\PYG{o}{=}\PYG{p}{(}
        \PYG{p}{\PYGZob{}}\PYG{l+s+s1}{\PYGZsq{}}\PYG{l+s+s1}{type}\PYG{l+s+s1}{\PYGZsq{}}\PYG{p}{:} \PYG{l+s+s1}{\PYGZsq{}}\PYG{l+s+s1}{eq}\PYG{l+s+s1}{\PYGZsq{}}\PYG{p}{,} \PYG{l+s+s1}{\PYGZsq{}}\PYG{l+s+s1}{fun}\PYG{l+s+s1}{\PYGZsq{}}\PYG{p}{:} \PYG{k}{lambda} \PYG{n}{x}\PYG{p}{:} \PYG{n}{x}\PYG{o}{.}\PYG{n}{sum}\PYG{p}{(}\PYG{p}{)} \PYG{o}{\PYGZhy{}} \PYG{l+m+mi}{1}\PYG{p}{\PYGZcb{}}\PYG{p}{,} \PYG{c+c1}{\PYGZsh{} minimize drives \PYGZdq{}eq\PYGZdq{} constraints to zero}
    \PYG{p}{)}
\PYG{p}{)}

\PYG{n}{res\PYGZus{}dd\PYGZus{}2}
\end{sphinxVerbatim}

\end{sphinxuseclass}\end{sphinxVerbatimInput}
\begin{sphinxVerbatimOutput}

\begin{sphinxuseclass}{cell_output}
\begin{sphinxVerbatim}[commandchars=\\\{\}]
     fun: 0.1452611244311275
     jac: array([0.1974, 0.064 , 0.2211, 0.2242, 0.1603, 0.0966, 0.162 , 0.1158,
       0.2333, 0.1664, 0.1609, 0.1217, 0.0695, 0.0094, 0.2134, 0.0657,
       0.1275, 0.0835, 0.0717, 0.1171, 0.0113, 0.1175, 0.1914, 0.1776,
       0.0756, 0.1588, 0.1089, 0.0882, 0.1333, 0.2522])
 message: \PYGZsq{}Optimization terminated successfully\PYGZsq{}
    nfev: 879
     nit: 27
    njev: 27
  status: 0
 success: True
       x: array([7.4317e\PYGZhy{}17, 2.6652e\PYGZhy{}02, 4.9022e\PYGZhy{}17, 0.0000e+00, 8.4813e\PYGZhy{}17,
       4.0861e\PYGZhy{}17, 0.0000e+00, 4.8595e\PYGZhy{}19, 8.3224e\PYGZhy{}17, 4.6493e\PYGZhy{}20,
       2.0382e\PYGZhy{}17, 1.0975e\PYGZhy{}16, 7.4798e\PYGZhy{}20, 1.2327e\PYGZhy{}17, 3.6385e\PYGZhy{}17,
       1.3723e\PYGZhy{}01, 7.9003e\PYGZhy{}17, 2.1395e\PYGZhy{}19, 5.0830e\PYGZhy{}17, 0.0000e+00,
       1.9045e\PYGZhy{}01, 1.3055e\PYGZhy{}01, 2.6484e\PYGZhy{}17, 3.0680e\PYGZhy{}17, 3.8500e\PYGZhy{}17,
       3.7442e\PYGZhy{}19, 2.4126e\PYGZhy{}17, 1.2647e\PYGZhy{}01, 0.0000e+00, 3.8864e\PYGZhy{}01])
\end{sphinxVerbatim}

\end{sphinxuseclass}\end{sphinxVerbatimOutput}

\end{sphinxuseclass}
\begin{sphinxuseclass}{cell}\begin{sphinxVerbatimInput}

\begin{sphinxuseclass}{cell_input}
\begin{sphinxVerbatim}[commandchars=\\\{\}]
\PYG{n}{port\PYGZus{}draw\PYGZus{}down\PYGZus{}max}\PYG{p}{(}\PYG{n}{x}\PYG{o}{=}\PYG{n}{res\PYGZus{}dd\PYGZus{}2}\PYG{p}{[}\PYG{l+s+s1}{\PYGZsq{}}\PYG{l+s+s1}{x}\PYG{l+s+s1}{\PYGZsq{}}\PYG{p}{]}\PYG{p}{,} \PYG{n}{r}\PYG{o}{=}\PYG{n}{returns\PYGZus{}2}\PYG{p}{)}
\end{sphinxVerbatim}

\end{sphinxuseclass}\end{sphinxVerbatimInput}
\begin{sphinxVerbatimOutput}

\begin{sphinxuseclass}{cell_output}
\begin{sphinxVerbatim}[commandchars=\\\{\}]
0.1453
\end{sphinxVerbatim}

\end{sphinxuseclass}\end{sphinxVerbatimOutput}

\end{sphinxuseclass}
\sphinxAtStartPar
Again. it might be helpful to slightly change then minimum VaR portfolio weights to show that we minimized VaR.

\begin{sphinxuseclass}{cell}\begin{sphinxVerbatimInput}

\begin{sphinxuseclass}{cell_input}
\begin{sphinxVerbatim}[commandchars=\\\{\}]
\PYG{n}{port\PYGZus{}draw\PYGZus{}down\PYGZus{}max}\PYG{p}{(}\PYG{n}{x}\PYG{o}{=}\PYG{n}{tweak}\PYG{p}{(}\PYG{n}{res\PYGZus{}dd\PYGZus{}2}\PYG{p}{[}\PYG{l+s+s1}{\PYGZsq{}}\PYG{l+s+s1}{x}\PYG{l+s+s1}{\PYGZsq{}}\PYG{p}{]}\PYG{p}{)}\PYG{p}{,} \PYG{n}{r}\PYG{o}{=}\PYG{n}{returns\PYGZus{}2}\PYG{p}{)}
\end{sphinxVerbatim}

\end{sphinxuseclass}\end{sphinxVerbatimInput}
\begin{sphinxVerbatimOutput}

\begin{sphinxuseclass}{cell_output}
\begin{sphinxVerbatim}[commandchars=\\\{\}]
0.1519
\end{sphinxVerbatim}

\end{sphinxuseclass}\end{sphinxVerbatimOutput}

\end{sphinxuseclass}

\subsubsection{Plot the (mean\sphinxhyphen{}variance) efficient frontier with all available data for the current the DJIA stocks}
\label{\detokenize{herron_04_practice_04:plot-the-mean-variance-efficient-frontier-with-all-available-data-for-the-current-the-djia-stocks}}
\sphinxAtStartPar
The range of target returns in \sphinxcode{\sphinxupquote{tret}} span from the minimum to the maximum mean single\sphinxhyphen{}stock returns.

\begin{sphinxuseclass}{cell}\begin{sphinxVerbatimInput}

\begin{sphinxuseclass}{cell_input}
\begin{sphinxVerbatim}[commandchars=\\\{\}]
\PYG{n}{\PYGZus{}} \PYG{o}{=} \PYG{n}{returns\PYGZus{}2}\PYG{o}{.}\PYG{n}{mean}\PYG{p}{(}\PYG{p}{)}\PYG{o}{.}\PYG{n}{mul}\PYG{p}{(}\PYG{l+m+mi}{252}\PYG{p}{)}
\PYG{n}{tret} \PYG{o}{=} \PYG{n}{np}\PYG{o}{.}\PYG{n}{linspace}\PYG{p}{(}\PYG{n}{\PYGZus{}}\PYG{o}{.}\PYG{n}{min}\PYG{p}{(}\PYG{p}{)}\PYG{p}{,} \PYG{n}{\PYGZus{}}\PYG{o}{.}\PYG{n}{max}\PYG{p}{(}\PYG{p}{)}\PYG{p}{,} \PYG{l+m+mi}{25}\PYG{p}{)}
\end{sphinxVerbatim}

\end{sphinxuseclass}\end{sphinxVerbatimInput}

\end{sphinxuseclass}
\sphinxAtStartPar
We will loop over these target returns, finding the minimum variance portfolio for each target return.

\begin{sphinxuseclass}{cell}\begin{sphinxVerbatimInput}

\begin{sphinxuseclass}{cell_input}
\begin{sphinxVerbatim}[commandchars=\\\{\}]
\PYG{k}{def} \PYG{n+nf}{port\PYGZus{}vol}\PYG{p}{(}\PYG{n}{x}\PYG{p}{,} \PYG{n}{r}\PYG{p}{,} \PYG{n}{ppy}\PYG{p}{)}\PYG{p}{:}
    \PYG{k}{return} \PYG{n}{np}\PYG{o}{.}\PYG{n}{sqrt}\PYG{p}{(}\PYG{n}{ppy}\PYG{p}{)} \PYG{o}{*} \PYG{n}{r}\PYG{o}{.}\PYG{n}{dot}\PYG{p}{(}\PYG{n}{x}\PYG{p}{)}\PYG{o}{.}\PYG{n}{std}\PYG{p}{(}\PYG{p}{)}
\end{sphinxVerbatim}

\end{sphinxuseclass}\end{sphinxVerbatimInput}

\end{sphinxuseclass}
\begin{sphinxuseclass}{cell}\begin{sphinxVerbatimInput}

\begin{sphinxuseclass}{cell_input}
\begin{sphinxVerbatim}[commandchars=\\\{\}]
\PYG{k}{def} \PYG{n+nf}{port\PYGZus{}mean}\PYG{p}{(}\PYG{n}{x}\PYG{p}{,} \PYG{n}{r}\PYG{p}{,} \PYG{n}{ppy}\PYG{p}{)}\PYG{p}{:}
    \PYG{k}{return} \PYG{n}{ppy} \PYG{o}{*} \PYG{n}{r}\PYG{o}{.}\PYG{n}{dot}\PYG{p}{(}\PYG{n}{x}\PYG{p}{)}\PYG{o}{.}\PYG{n}{mean}\PYG{p}{(}\PYG{p}{)}
\end{sphinxVerbatim}

\end{sphinxuseclass}\end{sphinxVerbatimInput}

\end{sphinxuseclass}
\begin{sphinxuseclass}{cell}\begin{sphinxVerbatimInput}

\begin{sphinxuseclass}{cell_input}
\begin{sphinxVerbatim}[commandchars=\\\{\}]
\PYG{n}{res\PYGZus{}ef} \PYG{o}{=} \PYG{p}{[}\PYG{p}{]}

\PYG{k}{for} \PYG{n}{t} \PYG{o+ow}{in} \PYG{n}{tret}\PYG{p}{:}
    \PYG{n}{\PYGZus{}} \PYG{o}{=} \PYG{n}{sco}\PYG{o}{.}\PYG{n}{minimize}\PYG{p}{(}
        \PYG{n}{fun}\PYG{o}{=}\PYG{n}{port\PYGZus{}vol}\PYG{p}{,} \PYG{c+c1}{\PYGZsh{} minimize portfolio volatility}
        \PYG{n}{x0}\PYG{o}{=}\PYG{n}{np}\PYG{o}{.}\PYG{n}{ones}\PYG{p}{(}\PYG{n}{returns\PYGZus{}2}\PYG{o}{.}\PYG{n}{shape}\PYG{p}{[}\PYG{l+m+mi}{1}\PYG{p}{]}\PYG{p}{)} \PYG{o}{/} \PYG{n}{returns\PYGZus{}2}\PYG{o}{.}\PYG{n}{shape}\PYG{p}{[}\PYG{l+m+mi}{1}\PYG{p}{]}\PYG{p}{,} \PYG{c+c1}{\PYGZsh{} initial portfolio weights}
        \PYG{n}{args}\PYG{o}{=}\PYG{p}{(}\PYG{n}{returns\PYGZus{}2}\PYG{p}{,} \PYG{l+m+mi}{252}\PYG{p}{)}\PYG{p}{,} \PYG{c+c1}{\PYGZsh{} additional arguments to fun, in order}
        \PYG{n}{bounds}\PYG{o}{=}\PYG{p}{[}\PYG{p}{(}\PYG{l+m+mi}{0}\PYG{p}{,} \PYG{l+m+mi}{1}\PYG{p}{)} \PYG{k}{for} \PYG{n}{c} \PYG{o+ow}{in} \PYG{n}{returns\PYGZus{}2}\PYG{o}{.}\PYG{n}{columns}\PYG{p}{]}\PYG{p}{,} \PYG{c+c1}{\PYGZsh{} bounds limit the search space for each portfolio weight}
        \PYG{n}{constraints}\PYG{o}{=}\PYG{p}{(}
            \PYG{p}{\PYGZob{}}\PYG{l+s+s1}{\PYGZsq{}}\PYG{l+s+s1}{type}\PYG{l+s+s1}{\PYGZsq{}}\PYG{p}{:} \PYG{l+s+s1}{\PYGZsq{}}\PYG{l+s+s1}{eq}\PYG{l+s+s1}{\PYGZsq{}}\PYG{p}{,} \PYG{l+s+s1}{\PYGZsq{}}\PYG{l+s+s1}{fun}\PYG{l+s+s1}{\PYGZsq{}}\PYG{p}{:} \PYG{k}{lambda} \PYG{n}{x}\PYG{p}{:} \PYG{n}{x}\PYG{o}{.}\PYG{n}{sum}\PYG{p}{(}\PYG{p}{)} \PYG{o}{\PYGZhy{}} \PYG{l+m+mi}{1}\PYG{p}{\PYGZcb{}}\PYG{p}{,} \PYG{c+c1}{\PYGZsh{} constrain sum of weights to one}
            \PYG{p}{\PYGZob{}}\PYG{l+s+s1}{\PYGZsq{}}\PYG{l+s+s1}{type}\PYG{l+s+s1}{\PYGZsq{}}\PYG{p}{:} \PYG{l+s+s1}{\PYGZsq{}}\PYG{l+s+s1}{eq}\PYG{l+s+s1}{\PYGZsq{}}\PYG{p}{,} \PYG{l+s+s1}{\PYGZsq{}}\PYG{l+s+s1}{fun}\PYG{l+s+s1}{\PYGZsq{}}\PYG{p}{:} \PYG{k}{lambda} \PYG{n}{x}\PYG{p}{:} \PYG{n}{port\PYGZus{}mean}\PYG{p}{(}\PYG{n}{x}\PYG{o}{=}\PYG{n}{x}\PYG{p}{,} \PYG{n}{r}\PYG{o}{=}\PYG{n}{returns\PYGZus{}2}\PYG{p}{,} \PYG{n}{ppy}\PYG{o}{=}\PYG{l+m+mi}{252}\PYG{p}{)} \PYG{o}{\PYGZhy{}} \PYG{n}{t}\PYG{p}{\PYGZcb{}} \PYG{c+c1}{\PYGZsh{} constrains portfolio mean return to the target return}

        \PYG{p}{)}
    \PYG{p}{)}
    \PYG{n}{res\PYGZus{}ef}\PYG{o}{.}\PYG{n}{append}\PYG{p}{(}\PYG{n}{\PYGZus{}}\PYG{p}{)}
\end{sphinxVerbatim}

\end{sphinxuseclass}\end{sphinxVerbatimInput}

\end{sphinxuseclass}
\sphinxAtStartPar
List \sphinxcode{\sphinxupquote{res\_ef}} contains the results of all 25 minimum\sphinxhyphen{}variance portfolios.
For example, \sphinxcode{\sphinxupquote{res\_ef{[}0{]}}} is the minimum variance portfolio for the lowest target return.

\begin{sphinxuseclass}{cell}\begin{sphinxVerbatimInput}

\begin{sphinxuseclass}{cell_input}
\begin{sphinxVerbatim}[commandchars=\\\{\}]
\PYG{n}{res\PYGZus{}ef}\PYG{p}{[}\PYG{l+m+mi}{0}\PYG{p}{]}
\end{sphinxVerbatim}

\end{sphinxuseclass}\end{sphinxVerbatimInput}
\begin{sphinxVerbatimOutput}

\begin{sphinxuseclass}{cell_output}
\begin{sphinxVerbatim}[commandchars=\\\{\}]
     fun: 0.4206024526789833
     jac: array([0.226 , 0.1289, 0.2273, 0.2753, 0.1603, 0.2102, 0.1879, 0.1846,
       0.1792, 0.1971, 0.2016, 0.1831, 0.1662, 0.1612, 0.4206, 0.0915,
       0.1934, 0.1133, 0.122 , 0.1283, 0.0833, 0.2267, 0.1728, 0.1111,
       0.1504, 0.1597, 0.1808, 0.0838, 0.154 , 0.0955])
 message: \PYGZsq{}Optimization terminated successfully\PYGZsq{}
    nfev: 93
     nit: 3
    njev: 3
  status: 0
 success: True
       x: array([1.7984e\PYGZhy{}15, 1.9459e\PYGZhy{}15, 0.0000e+00, 0.0000e+00, 3.1500e\PYGZhy{}15,
       0.0000e+00, 0.0000e+00, 9.2610e\PYGZhy{}16, 0.0000e+00, 0.0000e+00,
       6.8625e\PYGZhy{}16, 0.0000e+00, 0.0000e+00, 0.0000e+00, 1.0000e+00,
       0.0000e+00, 0.0000e+00, 0.0000e+00, 1.4480e\PYGZhy{}15, 0.0000e+00,
       0.0000e+00, 0.0000e+00, 1.0790e\PYGZhy{}15, 0.0000e+00, 1.1208e\PYGZhy{}15,
       0.0000e+00, 1.2956e\PYGZhy{}16, 5.8070e\PYGZhy{}16, 0.0000e+00, 0.0000e+00])
\end{sphinxVerbatim}

\end{sphinxuseclass}\end{sphinxVerbatimOutput}

\end{sphinxuseclass}
\sphinxAtStartPar
I typically check that all portfolio volatility minimization succeeds.
If a portfolio volatility minimization fails, we should check our function, bounds, and constraints.

\begin{sphinxuseclass}{cell}\begin{sphinxVerbatimInput}

\begin{sphinxuseclass}{cell_input}
\begin{sphinxVerbatim}[commandchars=\\\{\}]
\PYG{k}{for} \PYG{n}{r} \PYG{o+ow}{in} \PYG{n}{res\PYGZus{}ef}\PYG{p}{:}
    \PYG{k}{assert} \PYG{n}{r}\PYG{p}{[}\PYG{l+s+s1}{\PYGZsq{}}\PYG{l+s+s1}{success}\PYG{l+s+s1}{\PYGZsq{}}\PYG{p}{]} 
\end{sphinxVerbatim}

\end{sphinxuseclass}\end{sphinxVerbatimInput}

\end{sphinxuseclass}
\sphinxAtStartPar
We can combine the target returns and volatilities into a data frame \sphinxcode{\sphinxupquote{ef}}.

\begin{sphinxuseclass}{cell}\begin{sphinxVerbatimInput}

\begin{sphinxuseclass}{cell_input}
\begin{sphinxVerbatim}[commandchars=\\\{\}]
\PYG{n}{ef} \PYG{o}{=} \PYG{n}{pd}\PYG{o}{.}\PYG{n}{DataFrame}\PYG{p}{(}
    \PYG{p}{\PYGZob{}}
        \PYG{l+s+s1}{\PYGZsq{}}\PYG{l+s+s1}{tret}\PYG{l+s+s1}{\PYGZsq{}}\PYG{p}{:} \PYG{n}{tret}\PYG{p}{,}
        \PYG{l+s+s1}{\PYGZsq{}}\PYG{l+s+s1}{tvol}\PYG{l+s+s1}{\PYGZsq{}}\PYG{p}{:} \PYG{n}{np}\PYG{o}{.}\PYG{n}{array}\PYG{p}{(}\PYG{p}{[}\PYG{n}{r}\PYG{p}{[}\PYG{l+s+s1}{\PYGZsq{}}\PYG{l+s+s1}{fun}\PYG{l+s+s1}{\PYGZsq{}}\PYG{p}{]} \PYG{k}{if} \PYG{n}{r}\PYG{p}{[}\PYG{l+s+s1}{\PYGZsq{}}\PYG{l+s+s1}{success}\PYG{l+s+s1}{\PYGZsq{}}\PYG{p}{]} \PYG{k}{else} \PYG{n}{np}\PYG{o}{.}\PYG{n}{nan} \PYG{k}{for} \PYG{n}{r} \PYG{o+ow}{in} \PYG{n}{res\PYGZus{}ef}\PYG{p}{]}\PYG{p}{)}
    \PYG{p}{\PYGZcb{}}
\PYG{p}{)}

\PYG{n}{ef}\PYG{o}{.}\PYG{n}{head}\PYG{p}{(}\PYG{p}{)}
\end{sphinxVerbatim}

\end{sphinxuseclass}\end{sphinxVerbatimInput}
\begin{sphinxVerbatimOutput}

\begin{sphinxuseclass}{cell_output}
\begin{sphinxVerbatim}[commandchars=\\\{\}]
     tret   tvol
0 \PYGZhy{}0.1535 0.4206
1 \PYGZhy{}0.1358 0.3435
2 \PYGZhy{}0.1181 0.2757
3 \PYGZhy{}0.1004 0.2253
4 \PYGZhy{}0.0827 0.2027
\end{sphinxVerbatim}

\end{sphinxuseclass}\end{sphinxVerbatimOutput}

\end{sphinxuseclass}
\begin{sphinxuseclass}{cell}\begin{sphinxVerbatimInput}

\begin{sphinxuseclass}{cell_input}
\begin{sphinxVerbatim}[commandchars=\\\{\}]
\PYG{n}{ef}\PYG{o}{.}\PYG{n}{mul}\PYG{p}{(}\PYG{l+m+mi}{100}\PYG{p}{)}\PYG{o}{.}\PYG{n}{plot}\PYG{p}{(}\PYG{n}{x}\PYG{o}{=}\PYG{l+s+s1}{\PYGZsq{}}\PYG{l+s+s1}{tvol}\PYG{l+s+s1}{\PYGZsq{}}\PYG{p}{,} \PYG{n}{y}\PYG{o}{=}\PYG{l+s+s1}{\PYGZsq{}}\PYG{l+s+s1}{tret}\PYG{l+s+s1}{\PYGZsq{}}\PYG{p}{,} \PYG{n}{legend}\PYG{o}{=}\PYG{k+kc}{False}\PYG{p}{)}
\PYG{n}{plt}\PYG{o}{.}\PYG{n}{ylabel}\PYG{p}{(}\PYG{l+s+s1}{\PYGZsq{}}\PYG{l+s+s1}{Annualized Mean Return (}\PYG{l+s+s1}{\PYGZpc{}}\PYG{l+s+s1}{)}\PYG{l+s+s1}{\PYGZsq{}}\PYG{p}{)}
\PYG{n}{plt}\PYG{o}{.}\PYG{n}{xlabel}\PYG{p}{(}\PYG{l+s+s1}{\PYGZsq{}}\PYG{l+s+s1}{Annualized Volatility (}\PYG{l+s+s1}{\PYGZpc{}}\PYG{l+s+s1}{)}\PYG{l+s+s1}{\PYGZsq{}}\PYG{p}{)}
\PYG{n}{plt}\PYG{o}{.}\PYG{n}{title}\PYG{p}{(}
    \PYG{l+s+sa}{f}\PYG{l+s+s1}{\PYGZsq{}}\PYG{l+s+s1}{Efficient Frontier for Dow\PYGZhy{}Jones Industrial Average Stocks}\PYG{l+s+s1}{\PYGZsq{}} \PYG{o}{+}
    \PYG{l+s+sa}{f}\PYG{l+s+s1}{\PYGZsq{}}\PYG{l+s+se}{\PYGZbs{}n}\PYG{l+s+s1}{from }\PYG{l+s+si}{\PYGZob{}}\PYG{n}{returns\PYGZus{}2}\PYG{o}{.}\PYG{n}{index}\PYG{p}{[}\PYG{l+m+mi}{0}\PYG{p}{]}\PYG{l+s+si}{:}\PYG{l+s+s1}{\PYGZpc{}B \PYGZpc{}d, \PYGZpc{}Y}\PYG{l+s+si}{\PYGZcb{}}\PYG{l+s+s1}{ to }\PYG{l+s+si}{\PYGZob{}}\PYG{n}{returns\PYGZus{}2}\PYG{o}{.}\PYG{n}{index}\PYG{p}{[}\PYG{o}{\PYGZhy{}}\PYG{l+m+mi}{1}\PYG{p}{]}\PYG{l+s+si}{:}\PYG{l+s+s1}{\PYGZpc{}B \PYGZpc{}d, \PYGZpc{}Y}\PYG{l+s+si}{\PYGZcb{}}\PYG{l+s+s1}{\PYGZsq{}}
\PYG{p}{)}

\PYG{k}{for} \PYG{n}{t}\PYG{p}{,} \PYG{n}{x}\PYG{p}{,} \PYG{n}{y} \PYG{o+ow}{in} \PYG{n+nb}{zip}\PYG{p}{(}
    \PYG{n}{returns\PYGZus{}2}\PYG{o}{.}\PYG{n}{columns}\PYG{p}{,} 
    \PYG{n}{returns\PYGZus{}2}\PYG{o}{.}\PYG{n}{std}\PYG{p}{(}\PYG{p}{)}\PYG{o}{.}\PYG{n}{mul}\PYG{p}{(}\PYG{l+m+mi}{100}\PYG{o}{*}\PYG{n}{np}\PYG{o}{.}\PYG{n}{sqrt}\PYG{p}{(}\PYG{l+m+mi}{252}\PYG{p}{)}\PYG{p}{)}\PYG{p}{,}
    \PYG{n}{returns\PYGZus{}2}\PYG{o}{.}\PYG{n}{mean}\PYG{p}{(}\PYG{p}{)}\PYG{o}{.}\PYG{n}{mul}\PYG{p}{(}\PYG{l+m+mi}{100}\PYG{o}{*}\PYG{l+m+mi}{252}\PYG{p}{)}
\PYG{p}{)}\PYG{p}{:}
    \PYG{n}{plt}\PYG{o}{.}\PYG{n}{annotate}\PYG{p}{(}\PYG{n}{text}\PYG{o}{=}\PYG{n}{t}\PYG{p}{,} \PYG{n}{xy}\PYG{o}{=}\PYG{p}{(}\PYG{n}{x}\PYG{p}{,} \PYG{n}{y}\PYG{p}{)}\PYG{p}{)}
    
\PYG{n}{plt}\PYG{o}{.}\PYG{n}{show}\PYG{p}{(}\PYG{p}{)}
\end{sphinxVerbatim}

\end{sphinxuseclass}\end{sphinxVerbatimInput}
\begin{sphinxVerbatimOutput}

\begin{sphinxuseclass}{cell_output}
\noindent\sphinxincludegraphics{{43956a28cfa168eb1d249195f6a0fa16932abafc198ea4b80a17b3280a195942}.png}

\end{sphinxuseclass}\end{sphinxVerbatimOutput}

\end{sphinxuseclass}

\subsubsection{Find the maximum Sharpe Ratio portfolio with all available data for the current the DJIA stocks}
\label{\detokenize{herron_04_practice_04:find-the-maximum-sharpe-ratio-portfolio-with-all-available-data-for-the-current-the-djia-stocks}}
\begin{sphinxuseclass}{cell}\begin{sphinxVerbatimInput}

\begin{sphinxuseclass}{cell_input}
\begin{sphinxVerbatim}[commandchars=\\\{\}]
\PYG{n}{res\PYGZus{}sharpe\PYGZus{}6} \PYG{o}{=} \PYG{n}{sco}\PYG{o}{.}\PYG{n}{minimize}\PYG{p}{(}
    \PYG{n}{fun}\PYG{o}{=}\PYG{n}{port\PYGZus{}sharpe\PYGZus{}neg}\PYG{p}{,}
    \PYG{n}{x0}\PYG{o}{=}\PYG{n}{np}\PYG{o}{.}\PYG{n}{ones}\PYG{p}{(}\PYG{n}{returns\PYGZus{}2}\PYG{o}{.}\PYG{n}{shape}\PYG{p}{[}\PYG{l+m+mi}{1}\PYG{p}{]}\PYG{p}{)} \PYG{o}{/} \PYG{n}{returns\PYGZus{}2}\PYG{o}{.}\PYG{n}{shape}\PYG{p}{[}\PYG{l+m+mi}{1}\PYG{p}{]}\PYG{p}{,}
    \PYG{n}{args}\PYG{o}{=}\PYG{p}{(}\PYG{n}{returns\PYGZus{}2}\PYG{p}{,} \PYG{l+m+mi}{252}\PYG{p}{,} \PYG{l+m+mi}{0}\PYG{p}{)}\PYG{p}{,}
    \PYG{n}{bounds}\PYG{o}{=}\PYG{p}{[}\PYG{p}{(}\PYG{l+m+mi}{0}\PYG{p}{,}\PYG{l+m+mi}{1}\PYG{p}{)} \PYG{k}{for} \PYG{n}{\PYGZus{}} \PYG{o+ow}{in} \PYG{n+nb}{range}\PYG{p}{(}\PYG{n}{returns\PYGZus{}2}\PYG{o}{.}\PYG{n}{shape}\PYG{p}{[}\PYG{l+m+mi}{1}\PYG{p}{]}\PYG{p}{)}\PYG{p}{]}\PYG{p}{,}
    \PYG{n}{constraints}\PYG{o}{=}\PYG{p}{(}
        \PYG{p}{\PYGZob{}}\PYG{l+s+s1}{\PYGZsq{}}\PYG{l+s+s1}{type}\PYG{l+s+s1}{\PYGZsq{}}\PYG{p}{:} \PYG{l+s+s1}{\PYGZsq{}}\PYG{l+s+s1}{eq}\PYG{l+s+s1}{\PYGZsq{}}\PYG{p}{,} \PYG{l+s+s1}{\PYGZsq{}}\PYG{l+s+s1}{fun}\PYG{l+s+s1}{\PYGZsq{}}\PYG{p}{:} \PYG{k}{lambda} \PYG{n}{x}\PYG{p}{:} \PYG{n}{x}\PYG{o}{.}\PYG{n}{sum}\PYG{p}{(}\PYG{p}{)} \PYG{o}{\PYGZhy{}} \PYG{l+m+mi}{1}\PYG{p}{\PYGZcb{}} \PYG{c+c1}{\PYGZsh{} want eq constraint to = 0}
    \PYG{p}{)}
\PYG{p}{)}
\end{sphinxVerbatim}

\end{sphinxuseclass}\end{sphinxVerbatimInput}

\end{sphinxuseclass}
\begin{sphinxuseclass}{cell}\begin{sphinxVerbatimInput}

\begin{sphinxuseclass}{cell_input}
\begin{sphinxVerbatim}[commandchars=\\\{\}]
\PYG{n}{port\PYGZus{}sharpe}\PYG{p}{(}\PYG{n}{x}\PYG{o}{=}\PYG{n}{res\PYGZus{}sharpe\PYGZus{}6}\PYG{p}{[}\PYG{l+s+s1}{\PYGZsq{}}\PYG{l+s+s1}{x}\PYG{l+s+s1}{\PYGZsq{}}\PYG{p}{]}\PYG{p}{,} \PYG{n}{r}\PYG{o}{=}\PYG{n}{returns\PYGZus{}2}\PYG{p}{,} \PYG{n}{ppy}\PYG{o}{=}\PYG{l+m+mi}{252}\PYG{p}{,} \PYG{n}{tgt}\PYG{o}{=}\PYG{l+m+mi}{0}\PYG{p}{)}
\end{sphinxVerbatim}

\end{sphinxuseclass}\end{sphinxVerbatimInput}
\begin{sphinxVerbatimOutput}

\begin{sphinxuseclass}{cell_output}
\begin{sphinxVerbatim}[commandchars=\\\{\}]
0.44191670887963796
\end{sphinxVerbatim}

\end{sphinxuseclass}\end{sphinxVerbatimOutput}

\end{sphinxuseclass}

\subsubsection{Compare the \$\textbackslash{}frac\{1\}\{n\}\$ and maximum Sharpe Ratio portfolios with all available data for the current DJIA stocks}
\label{\detokenize{herron_04_practice_04:compare-the-frac-1-n-and-maximum-sharpe-ratio-portfolios-with-all-available-data-for-the-current-djia-stocks}}
\sphinxAtStartPar
Use all but the last 252 trading days to estimate the maximum Sharpe Ratio portfolio weights.
Then use the last 252 trading days of data to compare the \$\textbackslash{}frac\{1\}\{n\}\$  maximum Sharpe Ratio portfolios.

\begin{sphinxuseclass}{cell}\begin{sphinxVerbatimInput}

\begin{sphinxuseclass}{cell_input}
\begin{sphinxVerbatim}[commandchars=\\\{\}]
\PYG{n}{res\PYGZus{}sharpe\PYGZus{}x} \PYG{o}{=} \PYG{n}{sco}\PYG{o}{.}\PYG{n}{minimize}\PYG{p}{(}
    \PYG{n}{fun}\PYG{o}{=}\PYG{n}{port\PYGZus{}sharpe\PYGZus{}neg}\PYG{p}{,}
    \PYG{n}{x0}\PYG{o}{=}\PYG{n}{np}\PYG{o}{.}\PYG{n}{ones}\PYG{p}{(}\PYG{n}{returns\PYGZus{}2}\PYG{o}{.}\PYG{n}{shape}\PYG{p}{[}\PYG{l+m+mi}{1}\PYG{p}{]}\PYG{p}{)} \PYG{o}{/} \PYG{n}{returns\PYGZus{}2}\PYG{o}{.}\PYG{n}{shape}\PYG{p}{[}\PYG{l+m+mi}{1}\PYG{p}{]}\PYG{p}{,}
    \PYG{n}{args}\PYG{o}{=}\PYG{p}{(}\PYG{n}{returns\PYGZus{}2}\PYG{o}{.}\PYG{n}{loc}\PYG{p}{[}\PYG{p}{:}\PYG{l+s+s1}{\PYGZsq{}}\PYG{l+s+s1}{2021}\PYG{l+s+s1}{\PYGZsq{}}\PYG{p}{]}\PYG{p}{,} \PYG{l+m+mi}{0}\PYG{p}{,} \PYG{l+m+mi}{252}\PYG{p}{)}\PYG{p}{,}
    \PYG{n}{bounds}\PYG{o}{=}\PYG{p}{[}\PYG{p}{(}\PYG{l+m+mi}{0}\PYG{p}{,}\PYG{l+m+mi}{1}\PYG{p}{)} \PYG{k}{for} \PYG{n}{\PYGZus{}} \PYG{o+ow}{in} \PYG{n}{returns\PYGZus{}2}\PYG{p}{]}\PYG{p}{,} \PYG{c+c1}{\PYGZsh{} short up to 5 stocks 10\PYGZpc{}, then long the other stock 150\PYGZpc{}}
    \PYG{n}{constraints}\PYG{o}{=}\PYG{p}{(}
        \PYG{p}{\PYGZob{}}\PYG{l+s+s1}{\PYGZsq{}}\PYG{l+s+s1}{type}\PYG{l+s+s1}{\PYGZsq{}}\PYG{p}{:} \PYG{l+s+s1}{\PYGZsq{}}\PYG{l+s+s1}{eq}\PYG{l+s+s1}{\PYGZsq{}}\PYG{p}{,} \PYG{l+s+s1}{\PYGZsq{}}\PYG{l+s+s1}{fun}\PYG{l+s+s1}{\PYGZsq{}}\PYG{p}{:} \PYG{k}{lambda} \PYG{n}{x}\PYG{p}{:} \PYG{n}{x}\PYG{o}{.}\PYG{n}{sum}\PYG{p}{(}\PYG{p}{)} \PYG{o}{\PYGZhy{}} \PYG{l+m+mi}{1}\PYG{p}{\PYGZcb{}}\PYG{p}{,} \PYG{c+c1}{\PYGZsh{} eq constraints are driven to 0}
    \PYG{p}{)}
\PYG{p}{)}

\PYG{n}{res\PYGZus{}sharpe\PYGZus{}x}
\end{sphinxVerbatim}

\end{sphinxuseclass}\end{sphinxVerbatimInput}
\begin{sphinxVerbatimOutput}

\begin{sphinxuseclass}{cell_output}
\begin{sphinxVerbatim}[commandchars=\\\{\}]
     fun: \PYGZhy{}1.45101908912907
     jac: array([\PYGZhy{}3.3379e\PYGZhy{}05,  7.1544e\PYGZhy{}01,  3.6336e\PYGZhy{}01,  1.5581e+00, \PYGZhy{}5.4866e\PYGZhy{}05,
        2.6187e\PYGZhy{}01,  3.1877e\PYGZhy{}01,  6.4522e\PYGZhy{}01,  6.0754e\PYGZhy{}01,  5.1480e\PYGZhy{}01,
        2.2540e\PYGZhy{}02, \PYGZhy{}2.8282e\PYGZhy{}05,  4.4621e\PYGZhy{}01,  4.4353e\PYGZhy{}01,  1.1298e+00,
        2.2024e\PYGZhy{}01,  4.9128e\PYGZhy{}01,  3.9944e\PYGZhy{}01,  2.4492e\PYGZhy{}01,  4.7485e\PYGZhy{}01,
        6.0061e\PYGZhy{}01,  3.6554e\PYGZhy{}04, \PYGZhy{}1.0265e\PYGZhy{}03,  1.1015e\PYGZhy{}01,  4.1630e\PYGZhy{}01,
        1.2803e\PYGZhy{}02,  6.8370e\PYGZhy{}01,  5.0606e\PYGZhy{}01,  5.0418e\PYGZhy{}01,  1.4875e\PYGZhy{}01])
 message: \PYGZsq{}Optimization terminated successfully\PYGZsq{}
    nfev: 218
     nit: 7
    njev: 7
  status: 0
 success: True
       x: array([5.1283e\PYGZhy{}01, 0.0000e+00, 4.7142e\PYGZhy{}17, 7.0902e\PYGZhy{}16, 1.0012e\PYGZhy{}02,
       0.0000e+00, 1.5765e\PYGZhy{}16, 1.1636e\PYGZhy{}16, 0.0000e+00, 0.0000e+00,
       0.0000e+00, 2.5708e\PYGZhy{}01, 0.0000e+00, 0.0000e+00, 1.2567e\PYGZhy{}15,
       0.0000e+00, 0.0000e+00, 0.0000e+00, 0.0000e+00, 1.7025e\PYGZhy{}16,
       4.7163e\PYGZhy{}18, 1.7922e\PYGZhy{}01, 4.0854e\PYGZhy{}02, 0.0000e+00, 7.3800e\PYGZhy{}17,
       0.0000e+00, 3.5034e\PYGZhy{}16, 1.1696e\PYGZhy{}17, 2.2730e\PYGZhy{}16, 0.0000e+00])
\end{sphinxVerbatim}

\end{sphinxuseclass}\end{sphinxVerbatimOutput}

\end{sphinxuseclass}
\begin{sphinxuseclass}{cell}\begin{sphinxVerbatimInput}

\begin{sphinxuseclass}{cell_input}
\begin{sphinxVerbatim}[commandchars=\\\{\}]
\PYG{n}{plt}\PYG{o}{.}\PYG{n}{barh}\PYG{p}{(}
    \PYG{n}{y}\PYG{o}{=}\PYG{n}{returns\PYGZus{}2}\PYG{o}{.}\PYG{n}{columns}\PYG{p}{,}
    \PYG{n}{width}\PYG{o}{=}\PYG{n}{res\PYGZus{}sharpe\PYGZus{}x}\PYG{p}{[}\PYG{l+s+s1}{\PYGZsq{}}\PYG{l+s+s1}{x}\PYG{l+s+s1}{\PYGZsq{}}\PYG{p}{]}\PYG{p}{,}
    \PYG{n}{label}\PYG{o}{=}\PYG{l+s+s1}{\PYGZsq{}}\PYG{l+s+s1}{Maximum Sharpe Ratio}\PYG{l+s+s1}{\PYGZsq{}}
\PYG{p}{)}
\PYG{n}{plt}\PYG{o}{.}\PYG{n}{axvline}\PYG{p}{(}\PYG{l+m+mi}{1}\PYG{o}{/}\PYG{l+m+mi}{30}\PYG{p}{,} \PYG{n}{color}\PYG{o}{=}\PYG{l+s+s1}{\PYGZsq{}}\PYG{l+s+s1}{red}\PYG{l+s+s1}{\PYGZsq{}}\PYG{p}{,} \PYG{n}{label}\PYG{o}{=}\PYG{l+s+s1}{\PYGZsq{}}\PYG{l+s+s1}{Equal Weight}\PYG{l+s+s1}{\PYGZsq{}}\PYG{p}{)}
\PYG{n}{plt}\PYG{o}{.}\PYG{n}{legend}\PYG{p}{(}\PYG{p}{)}
\PYG{n}{plt}\PYG{o}{.}\PYG{n}{xlabel}\PYG{p}{(}\PYG{l+s+s1}{\PYGZsq{}}\PYG{l+s+s1}{Portfolio Weight}\PYG{l+s+s1}{\PYGZsq{}}\PYG{p}{)}
\PYG{n}{plt}\PYG{o}{.}\PYG{n}{title}\PYG{p}{(}
    \PYG{l+s+s1}{\PYGZsq{}}\PYG{l+s+s1}{Portfolio Weights for Dow\PYGZhy{}Jones Industrial Average Stocks}\PYG{l+s+s1}{\PYGZsq{}} \PYG{o}{+}
    \PYG{l+s+sa}{f}\PYG{l+s+s1}{\PYGZsq{}}\PYG{l+s+se}{\PYGZbs{}n}\PYG{l+s+s1}{from }\PYG{l+s+si}{\PYGZob{}}\PYG{n}{returns\PYGZus{}2}\PYG{o}{.}\PYG{n}{index}\PYG{p}{[}\PYG{l+m+mi}{0}\PYG{p}{]}\PYG{l+s+si}{:}\PYG{l+s+s1}{\PYGZpc{}b \PYGZpc{}d, \PYGZpc{}Y}\PYG{l+s+si}{\PYGZcb{}}\PYG{l+s+s1}{ to }\PYG{l+s+si}{\PYGZob{}}\PYG{n}{returns\PYGZus{}2}\PYG{o}{.}\PYG{n}{index}\PYG{p}{[}\PYG{o}{\PYGZhy{}}\PYG{l+m+mi}{1}\PYG{p}{]}\PYG{l+s+si}{:}\PYG{l+s+s1}{\PYGZpc{}b \PYGZpc{}d, \PYGZpc{}Y}\PYG{l+s+si}{\PYGZcb{}}\PYG{l+s+s1}{\PYGZsq{}}
\PYG{p}{)}
\PYG{n}{plt}\PYG{o}{.}\PYG{n}{show}\PYG{p}{(}\PYG{p}{)}
\end{sphinxVerbatim}

\end{sphinxuseclass}\end{sphinxVerbatimInput}
\begin{sphinxVerbatimOutput}

\begin{sphinxuseclass}{cell_output}
\noindent\sphinxincludegraphics{{ed3faa8fc73c2e844f027f25089fe052145d884b806b964b4e67bef53f91593e}.png}

\end{sphinxuseclass}\end{sphinxVerbatimOutput}

\end{sphinxuseclass}
\begin{sphinxuseclass}{cell}\begin{sphinxVerbatimInput}

\begin{sphinxuseclass}{cell_input}
\begin{sphinxVerbatim}[commandchars=\\\{\}]
\PYG{n}{port\PYGZus{}sharpe}\PYG{p}{(}\PYG{n}{x}\PYG{o}{=}\PYG{n}{res\PYGZus{}sharpe\PYGZus{}x}\PYG{p}{[}\PYG{l+s+s1}{\PYGZsq{}}\PYG{l+s+s1}{x}\PYG{l+s+s1}{\PYGZsq{}}\PYG{p}{]}\PYG{p}{,} \PYG{n}{r}\PYG{o}{=}\PYG{n}{returns\PYGZus{}2}\PYG{o}{.}\PYG{n}{loc}\PYG{p}{[}\PYG{l+s+s1}{\PYGZsq{}}\PYG{l+s+s1}{2022}\PYG{l+s+s1}{\PYGZsq{}}\PYG{p}{]}\PYG{p}{,} \PYG{n}{tgt}\PYG{o}{=}\PYG{l+m+mi}{0}\PYG{p}{,} \PYG{n}{ppy}\PYG{o}{=}\PYG{l+m+mi}{252}\PYG{p}{)}
\end{sphinxVerbatim}

\end{sphinxuseclass}\end{sphinxVerbatimInput}
\begin{sphinxVerbatimOutput}

\begin{sphinxuseclass}{cell_output}
\begin{sphinxVerbatim}[commandchars=\\\{\}]
\PYGZhy{}0.753001293587445
\end{sphinxVerbatim}

\end{sphinxuseclass}\end{sphinxVerbatimOutput}

\end{sphinxuseclass}
\begin{sphinxuseclass}{cell}\begin{sphinxVerbatimInput}

\begin{sphinxuseclass}{cell_input}
\begin{sphinxVerbatim}[commandchars=\\\{\}]
\PYG{n}{port\PYGZus{}sharpe}\PYG{p}{(}\PYG{n}{x}\PYG{o}{=}\PYG{n}{np}\PYG{o}{.}\PYG{n}{ones}\PYG{p}{(}\PYG{n}{returns\PYGZus{}2}\PYG{o}{.}\PYG{n}{shape}\PYG{p}{[}\PYG{l+m+mi}{1}\PYG{p}{]}\PYG{p}{)}\PYG{o}{/}\PYG{n}{returns\PYGZus{}2}\PYG{o}{.}\PYG{n}{shape}\PYG{p}{[}\PYG{l+m+mi}{1}\PYG{p}{]}\PYG{p}{,} \PYG{n}{r}\PYG{o}{=}\PYG{n}{returns\PYGZus{}2}\PYG{o}{.}\PYG{n}{loc}\PYG{p}{[}\PYG{l+s+s1}{\PYGZsq{}}\PYG{l+s+s1}{2022}\PYG{l+s+s1}{\PYGZsq{}}\PYG{p}{]}\PYG{p}{,} \PYG{n}{tgt}\PYG{o}{=}\PYG{l+m+mi}{0}\PYG{p}{,} \PYG{n}{ppy}\PYG{o}{=}\PYG{l+m+mi}{252}\PYG{p}{)}
\end{sphinxVerbatim}

\end{sphinxuseclass}\end{sphinxVerbatimInput}
\begin{sphinxVerbatimOutput}

\begin{sphinxuseclass}{cell_output}
\begin{sphinxVerbatim}[commandchars=\\\{\}]
\PYGZhy{}0.25436552600712903
\end{sphinxVerbatim}

\end{sphinxuseclass}\end{sphinxVerbatimOutput}

\end{sphinxuseclass}
\sphinxAtStartPar
It is hard to beat the \$\textbackslash{}frac\{1\}\{n\}\$ portfolio because mean returns (and covariances) are hard to predict!


\bigskip\hrule\bigskip


\sphinxAtStartPar
What if we picked a different period?

\begin{sphinxuseclass}{cell}\begin{sphinxVerbatimInput}

\begin{sphinxuseclass}{cell_input}
\begin{sphinxVerbatim}[commandchars=\\\{\}]
\PYG{n}{returns\PYGZus{}3} \PYG{o}{=} \PYG{n}{djia}\PYG{p}{[}\PYG{l+s+s1}{\PYGZsq{}}\PYG{l+s+s1}{Adj Close}\PYG{l+s+s1}{\PYGZsq{}}\PYG{p}{]}\PYG{o}{.}\PYG{n}{pct\PYGZus{}change}\PYG{p}{(}\PYG{p}{)}\PYG{o}{.}\PYG{n}{loc}\PYG{p}{[}\PYG{l+s+s1}{\PYGZsq{}}\PYG{l+s+s1}{2010}\PYG{l+s+s1}{\PYGZsq{}}\PYG{p}{:}\PYG{l+s+s1}{\PYGZsq{}}\PYG{l+s+s1}{2019}\PYG{l+s+s1}{\PYGZsq{}}\PYG{p}{]}\PYG{o}{.}\PYG{n}{dropna}\PYG{p}{(}\PYG{n}{axis}\PYG{o}{=}\PYG{l+m+mi}{1}\PYG{p}{)}
\end{sphinxVerbatim}

\end{sphinxuseclass}\end{sphinxVerbatimInput}

\end{sphinxuseclass}
\begin{sphinxuseclass}{cell}\begin{sphinxVerbatimInput}

\begin{sphinxuseclass}{cell_input}
\begin{sphinxVerbatim}[commandchars=\\\{\}]
\PYG{n}{res\PYGZus{}sharpe\PYGZus{}xp1} \PYG{o}{=} \PYG{n}{sco}\PYG{o}{.}\PYG{n}{minimize}\PYG{p}{(}
    \PYG{n}{fun}\PYG{o}{=}\PYG{n}{port\PYGZus{}sharpe\PYGZus{}neg}\PYG{p}{,}
    \PYG{n}{x0}\PYG{o}{=}\PYG{n}{np}\PYG{o}{.}\PYG{n}{ones}\PYG{p}{(}\PYG{n}{returns\PYGZus{}3}\PYG{o}{.}\PYG{n}{shape}\PYG{p}{[}\PYG{l+m+mi}{1}\PYG{p}{]}\PYG{p}{)} \PYG{o}{/} \PYG{n}{returns\PYGZus{}3}\PYG{o}{.}\PYG{n}{shape}\PYG{p}{[}\PYG{l+m+mi}{1}\PYG{p}{]}\PYG{p}{,}
    \PYG{n}{args}\PYG{o}{=}\PYG{p}{(}\PYG{n}{returns\PYGZus{}3}\PYG{o}{.}\PYG{n}{loc}\PYG{p}{[}\PYG{p}{:}\PYG{l+s+s1}{\PYGZsq{}}\PYG{l+s+s1}{2017}\PYG{l+s+s1}{\PYGZsq{}}\PYG{p}{]}\PYG{p}{,} \PYG{l+m+mi}{0}\PYG{p}{,} \PYG{l+m+mi}{252}\PYG{p}{)}\PYG{p}{,}
    \PYG{n}{bounds}\PYG{o}{=}\PYG{p}{[}\PYG{p}{(}\PYG{l+m+mi}{0}\PYG{p}{,}\PYG{l+m+mi}{1}\PYG{p}{)} \PYG{k}{for} \PYG{n}{\PYGZus{}} \PYG{o+ow}{in} \PYG{n}{returns\PYGZus{}3}\PYG{p}{]}\PYG{p}{,} \PYG{c+c1}{\PYGZsh{} short up to 5 stocks 10\PYGZpc{}, then long the other stock 150\PYGZpc{}}
    \PYG{n}{constraints}\PYG{o}{=}\PYG{p}{(}
        \PYG{p}{\PYGZob{}}\PYG{l+s+s1}{\PYGZsq{}}\PYG{l+s+s1}{type}\PYG{l+s+s1}{\PYGZsq{}}\PYG{p}{:} \PYG{l+s+s1}{\PYGZsq{}}\PYG{l+s+s1}{eq}\PYG{l+s+s1}{\PYGZsq{}}\PYG{p}{,} \PYG{l+s+s1}{\PYGZsq{}}\PYG{l+s+s1}{fun}\PYG{l+s+s1}{\PYGZsq{}}\PYG{p}{:} \PYG{k}{lambda} \PYG{n}{x}\PYG{p}{:} \PYG{n}{x}\PYG{o}{.}\PYG{n}{sum}\PYG{p}{(}\PYG{p}{)} \PYG{o}{\PYGZhy{}} \PYG{l+m+mi}{1}\PYG{p}{\PYGZcb{}}\PYG{p}{,} \PYG{c+c1}{\PYGZsh{} eq constraints are driven to 0}
    \PYG{p}{)}
\PYG{p}{)}
\end{sphinxVerbatim}

\end{sphinxuseclass}\end{sphinxVerbatimInput}

\end{sphinxuseclass}
\begin{sphinxuseclass}{cell}\begin{sphinxVerbatimInput}

\begin{sphinxuseclass}{cell_input}
\begin{sphinxVerbatim}[commandchars=\\\{\}]
\PYG{n}{res\PYGZus{}sharpe\PYGZus{}xp1}
\end{sphinxVerbatim}

\end{sphinxuseclass}\end{sphinxVerbatimInput}
\begin{sphinxVerbatimOutput}

\begin{sphinxuseclass}{cell_output}
\begin{sphinxVerbatim}[commandchars=\\\{\}]
     fun: \PYGZhy{}1.731278080433011
     jac: array([ 8.2898e\PYGZhy{}04,  1.7132e\PYGZhy{}01,  6.2183e\PYGZhy{}01,  6.6705e\PYGZhy{}04,  5.1273e\PYGZhy{}01,
        5.8156e\PYGZhy{}02,  7.1817e\PYGZhy{}01,  5.8716e\PYGZhy{}01,  2.4546e\PYGZhy{}01,  1.0825e+00,
       \PYGZhy{}6.6045e\PYGZhy{}04,  2.9742e\PYGZhy{}01,  7.8327e\PYGZhy{}01,  3.4385e\PYGZhy{}01,  1.2937e\PYGZhy{}02,
        6.9605e\PYGZhy{}01,  2.2715e\PYGZhy{}01, \PYGZhy{}5.5020e\PYGZhy{}04,  2.2086e\PYGZhy{}01,  4.1068e\PYGZhy{}01,
        2.3549e\PYGZhy{}01,  9.3803e\PYGZhy{}02,  1.7122e\PYGZhy{}01,  1.2081e\PYGZhy{}01,  9.5281e\PYGZhy{}04,
       \PYGZhy{}6.2662e\PYGZhy{}04, \PYGZhy{}7.7012e\PYGZhy{}04,  3.4866e\PYGZhy{}01,  5.7884e\PYGZhy{}02])
 message: \PYGZsq{}Optimization terminated successfully\PYGZsq{}
    nfev: 273
     nit: 9
    njev: 9
  status: 0
 success: True
       x: array([1.1385e\PYGZhy{}01, 3.4971e\PYGZhy{}16, 5.0862e\PYGZhy{}16, 7.1293e\PYGZhy{}02, 0.0000e+00,
       1.4777e\PYGZhy{}18, 0.0000e+00, 3.7423e\PYGZhy{}17, 1.6828e\PYGZhy{}16, 3.6013e\PYGZhy{}16,
       3.1237e\PYGZhy{}01, 1.3929e\PYGZhy{}16, 0.0000e+00, 2.2603e\PYGZhy{}17, 1.0282e\PYGZhy{}16,
       2.0842e\PYGZhy{}16, 1.6927e\PYGZhy{}16, 2.0434e\PYGZhy{}01, 6.9230e\PYGZhy{}17, 3.2044e\PYGZhy{}17,
       8.4599e\PYGZhy{}17, 1.3271e\PYGZhy{}16, 5.6371e\PYGZhy{}17, 2.0042e\PYGZhy{}16, 2.3077e\PYGZhy{}01,
       5.2373e\PYGZhy{}02, 1.5005e\PYGZhy{}02, 5.5219e\PYGZhy{}17, 7.0648e\PYGZhy{}17])
\end{sphinxVerbatim}

\end{sphinxuseclass}\end{sphinxVerbatimOutput}

\end{sphinxuseclass}
\begin{sphinxuseclass}{cell}\begin{sphinxVerbatimInput}

\begin{sphinxuseclass}{cell_input}
\begin{sphinxVerbatim}[commandchars=\\\{\}]
\PYG{n}{plt}\PYG{o}{.}\PYG{n}{barh}\PYG{p}{(}
    \PYG{n}{y}\PYG{o}{=}\PYG{n}{returns\PYGZus{}3}\PYG{o}{.}\PYG{n}{columns}\PYG{p}{,}
    \PYG{n}{width}\PYG{o}{=}\PYG{n}{res\PYGZus{}sharpe\PYGZus{}xp1}\PYG{p}{[}\PYG{l+s+s1}{\PYGZsq{}}\PYG{l+s+s1}{x}\PYG{l+s+s1}{\PYGZsq{}}\PYG{p}{]}\PYG{p}{,}
    \PYG{n}{label}\PYG{o}{=}\PYG{l+s+s1}{\PYGZsq{}}\PYG{l+s+s1}{Maximum Sharpe Ratio}\PYG{l+s+s1}{\PYGZsq{}}
\PYG{p}{)}
\PYG{n}{plt}\PYG{o}{.}\PYG{n}{axvline}\PYG{p}{(}\PYG{l+m+mi}{1}\PYG{o}{/}\PYG{l+m+mi}{30}\PYG{p}{,} \PYG{n}{color}\PYG{o}{=}\PYG{l+s+s1}{\PYGZsq{}}\PYG{l+s+s1}{red}\PYG{l+s+s1}{\PYGZsq{}}\PYG{p}{,} \PYG{n}{label}\PYG{o}{=}\PYG{l+s+s1}{\PYGZsq{}}\PYG{l+s+s1}{Equal Weight}\PYG{l+s+s1}{\PYGZsq{}}\PYG{p}{)}
\PYG{n}{plt}\PYG{o}{.}\PYG{n}{legend}\PYG{p}{(}\PYG{p}{)}
\PYG{n}{plt}\PYG{o}{.}\PYG{n}{xlabel}\PYG{p}{(}\PYG{l+s+s1}{\PYGZsq{}}\PYG{l+s+s1}{Portfolio Weight}\PYG{l+s+s1}{\PYGZsq{}}\PYG{p}{)}
\PYG{n}{plt}\PYG{o}{.}\PYG{n}{title}\PYG{p}{(}
    \PYG{l+s+s1}{\PYGZsq{}}\PYG{l+s+s1}{Portfolio Weights for Dow\PYGZhy{}Jones Industrial Average Stocks}\PYG{l+s+s1}{\PYGZsq{}} \PYG{o}{+}
    \PYG{l+s+sa}{f}\PYG{l+s+s1}{\PYGZsq{}}\PYG{l+s+se}{\PYGZbs{}n}\PYG{l+s+s1}{from }\PYG{l+s+si}{\PYGZob{}}\PYG{n}{returns\PYGZus{}3}\PYG{o}{.}\PYG{n}{index}\PYG{p}{[}\PYG{l+m+mi}{0}\PYG{p}{]}\PYG{l+s+si}{:}\PYG{l+s+s1}{\PYGZpc{}b \PYGZpc{}d, \PYGZpc{}Y}\PYG{l+s+si}{\PYGZcb{}}\PYG{l+s+s1}{ to }\PYG{l+s+si}{\PYGZob{}}\PYG{n}{returns\PYGZus{}3}\PYG{o}{.}\PYG{n}{index}\PYG{p}{[}\PYG{o}{\PYGZhy{}}\PYG{l+m+mi}{1}\PYG{p}{]}\PYG{l+s+si}{:}\PYG{l+s+s1}{\PYGZpc{}b \PYGZpc{}d, \PYGZpc{}Y}\PYG{l+s+si}{\PYGZcb{}}\PYG{l+s+s1}{\PYGZsq{}}
\PYG{p}{)}
\PYG{n}{plt}\PYG{o}{.}\PYG{n}{show}\PYG{p}{(}\PYG{p}{)}
\end{sphinxVerbatim}

\end{sphinxuseclass}\end{sphinxVerbatimInput}
\begin{sphinxVerbatimOutput}

\begin{sphinxuseclass}{cell_output}
\noindent\sphinxincludegraphics{{d1b7fb21fa5827c49ea187152662da7786148c1cccd535eb399fd6eda3421d9e}.png}

\end{sphinxuseclass}\end{sphinxVerbatimOutput}

\end{sphinxuseclass}
\begin{sphinxuseclass}{cell}\begin{sphinxVerbatimInput}

\begin{sphinxuseclass}{cell_input}
\begin{sphinxVerbatim}[commandchars=\\\{\}]
\PYG{n}{port\PYGZus{}sharpe}\PYG{p}{(}\PYG{n}{x}\PYG{o}{=}\PYG{n}{res\PYGZus{}sharpe\PYGZus{}xp1}\PYG{p}{[}\PYG{l+s+s1}{\PYGZsq{}}\PYG{l+s+s1}{x}\PYG{l+s+s1}{\PYGZsq{}}\PYG{p}{]}\PYG{p}{,} \PYG{n}{r}\PYG{o}{=}\PYG{n}{returns\PYGZus{}3}\PYG{o}{.}\PYG{n}{loc}\PYG{p}{[}\PYG{l+s+s1}{\PYGZsq{}}\PYG{l+s+s1}{2018}\PYG{l+s+s1}{\PYGZsq{}}\PYG{p}{:}\PYG{l+s+s1}{\PYGZsq{}}\PYG{l+s+s1}{2019}\PYG{l+s+s1}{\PYGZsq{}}\PYG{p}{]}\PYG{p}{,} \PYG{n}{tgt}\PYG{o}{=}\PYG{l+m+mi}{0}\PYG{p}{,} \PYG{n}{ppy}\PYG{o}{=}\PYG{l+m+mi}{252}\PYG{p}{)}
\end{sphinxVerbatim}

\end{sphinxuseclass}\end{sphinxVerbatimInput}
\begin{sphinxVerbatimOutput}

\begin{sphinxuseclass}{cell_output}
\begin{sphinxVerbatim}[commandchars=\\\{\}]
1.0491764710165667
\end{sphinxVerbatim}

\end{sphinxuseclass}\end{sphinxVerbatimOutput}

\end{sphinxuseclass}
\begin{sphinxuseclass}{cell}\begin{sphinxVerbatimInput}

\begin{sphinxuseclass}{cell_input}
\begin{sphinxVerbatim}[commandchars=\\\{\}]
\PYG{n}{port\PYGZus{}sharpe}\PYG{p}{(}\PYG{n}{x}\PYG{o}{=}\PYG{n}{np}\PYG{o}{.}\PYG{n}{ones}\PYG{p}{(}\PYG{n}{returns\PYGZus{}3}\PYG{o}{.}\PYG{n}{shape}\PYG{p}{[}\PYG{l+m+mi}{1}\PYG{p}{]}\PYG{p}{)}\PYG{o}{/}\PYG{n}{returns\PYGZus{}3}\PYG{o}{.}\PYG{n}{shape}\PYG{p}{[}\PYG{l+m+mi}{1}\PYG{p}{]}\PYG{p}{,} \PYG{n}{r}\PYG{o}{=}\PYG{n}{returns\PYGZus{}3}\PYG{o}{.}\PYG{n}{loc}\PYG{p}{[}\PYG{l+s+s1}{\PYGZsq{}}\PYG{l+s+s1}{2018}\PYG{l+s+s1}{\PYGZsq{}}\PYG{p}{:}\PYG{l+s+s1}{\PYGZsq{}}\PYG{l+s+s1}{2019}\PYG{l+s+s1}{\PYGZsq{}}\PYG{p}{]}\PYG{p}{,} \PYG{n}{tgt}\PYG{o}{=}\PYG{l+m+mi}{0}\PYG{p}{,} \PYG{n}{ppy}\PYG{o}{=}\PYG{l+m+mi}{252}\PYG{p}{)}
\end{sphinxVerbatim}

\end{sphinxuseclass}\end{sphinxVerbatimInput}
\begin{sphinxVerbatimOutput}

\begin{sphinxuseclass}{cell_output}
\begin{sphinxVerbatim}[commandchars=\\\{\}]
0.9908235050926001
\end{sphinxVerbatim}

\end{sphinxuseclass}\end{sphinxVerbatimOutput}

\end{sphinxuseclass}
\sphinxAtStartPar
It is hard to beat the \$\textbackslash{}frac\{1\}\{n\}\$ portfolio!
The \$\textbackslash{}frac\{1\}\{n\}\$ portfolio is hard to beat because return means and covariances are hard to predict!

\sphinxstepscope


\section{Herron Topic 4 \sphinxhyphen{} Practice (Wednesday 2:45 PM, Section 2)}
\label{\detokenize{herron_04_practice_02:herron-topic-4-practice-wednesday-2-45-pm-section-2}}\label{\detokenize{herron_04_practice_02::doc}}

\subsection{Announcements}
\label{\detokenize{herron_04_practice_02:announcements}}\begin{itemize}
\item {} 
\sphinxAtStartPar
Quiz 6 this week
\begin{itemize}
\item {} 
\sphinxAtStartPar
I will post it at about 6 PM on Wednesday, 3/31

\item {} 
\sphinxAtStartPar
It will be due by 11:59 PM on Friday, 3/31

\end{itemize}

\item {} 
\sphinxAtStartPar
Please complete the week ten survey
\begin{itemize}
\item {} 
\sphinxAtStartPar
I am considering dropping a topic to allow more in\sphinxhyphen{}class group work and easier access to me

\item {} 
\sphinxAtStartPar
I am also curious why the quantitative courses are less popular this summer

\item {} 
\sphinxAtStartPar
Please complete by 11:59 PM on Friday, 3/31

\item {} 
\sphinxAtStartPar
\sphinxstyleemphasis{\sphinxstylestrong{The week ten survey is anonymous and voluntary}}

\end{itemize}

\item {} 
\sphinxAtStartPar
I will post project 2 as soon as I can

\item {} 
\sphinxAtStartPar
Assessment exam
\begin{itemize}
\item {} 
\sphinxAtStartPar
20 questions multiple on Canvas

\item {} 
\sphinxAtStartPar
You must be in the room

\item {} 
\sphinxAtStartPar
No specific studying, but I suggest putting core course resources on your laptop (e.g., notes and PowerPoints)

\end{itemize}

\end{itemize}


\subsection{Practice}
\label{\detokenize{herron_04_practice_02:practice}}
\begin{sphinxuseclass}{cell}\begin{sphinxVerbatimInput}

\begin{sphinxuseclass}{cell_input}
\begin{sphinxVerbatim}[commandchars=\\\{\}]
\PYG{k+kn}{import} \PYG{n+nn}{matplotlib}\PYG{n+nn}{.}\PYG{n+nn}{pyplot} \PYG{k}{as} \PYG{n+nn}{plt}
\PYG{k+kn}{import} \PYG{n+nn}{numpy} \PYG{k}{as} \PYG{n+nn}{np}
\PYG{k+kn}{import} \PYG{n+nn}{pandas} \PYG{k}{as} \PYG{n+nn}{pd}
\end{sphinxVerbatim}

\end{sphinxuseclass}\end{sphinxVerbatimInput}

\end{sphinxuseclass}
\begin{sphinxuseclass}{cell}\begin{sphinxVerbatimInput}

\begin{sphinxuseclass}{cell_input}
\begin{sphinxVerbatim}[commandchars=\\\{\}]
\PYG{o}{\PYGZpc{}}\PYG{k}{config} InlineBackend.figure\PYGZus{}format = \PYGZsq{}retina\PYGZsq{}
\PYG{o}{\PYGZpc{}}\PYG{k}{precision} 4
\PYG{n}{pd}\PYG{o}{.}\PYG{n}{options}\PYG{o}{.}\PYG{n}{display}\PYG{o}{.}\PYG{n}{float\PYGZus{}format} \PYG{o}{=} \PYG{l+s+s1}{\PYGZsq{}}\PYG{l+s+si}{\PYGZob{}:.4f\PYGZcb{}}\PYG{l+s+s1}{\PYGZsq{}}\PYG{o}{.}\PYG{n}{format}
\end{sphinxVerbatim}

\end{sphinxuseclass}\end{sphinxVerbatimInput}

\end{sphinxuseclass}
\begin{sphinxuseclass}{cell}\begin{sphinxVerbatimInput}

\begin{sphinxuseclass}{cell_input}
\begin{sphinxVerbatim}[commandchars=\\\{\}]
\PYG{k+kn}{import} \PYG{n+nn}{yfinance} \PYG{k}{as} \PYG{n+nn}{yf}
\PYG{k+kn}{import} \PYG{n+nn}{pandas\PYGZus{}datareader} \PYG{k}{as} \PYG{n+nn}{pdr}
\PYG{k+kn}{import} \PYG{n+nn}{requests\PYGZus{}cache}
\PYG{n}{session} \PYG{o}{=} \PYG{n}{requests\PYGZus{}cache}\PYG{o}{.}\PYG{n}{CachedSession}\PYG{p}{(}\PYG{p}{)}
\end{sphinxVerbatim}

\end{sphinxuseclass}\end{sphinxVerbatimInput}

\end{sphinxuseclass}
\begin{sphinxuseclass}{cell}\begin{sphinxVerbatimInput}

\begin{sphinxuseclass}{cell_input}
\begin{sphinxVerbatim}[commandchars=\\\{\}]
\PYG{k+kn}{import} \PYG{n+nn}{scipy}\PYG{n+nn}{.}\PYG{n+nn}{optimize} \PYG{k}{as} \PYG{n+nn}{sco}
\end{sphinxVerbatim}

\end{sphinxuseclass}\end{sphinxVerbatimInput}

\end{sphinxuseclass}

\subsubsection{Find the maximum Sharpe Ratio portfolio of MATANA stocks over the last three calendar years}
\label{\detokenize{herron_04_practice_02:find-the-maximum-sharpe-ratio-portfolio-of-matana-stocks-over-the-last-three-calendar-years}}
\sphinxAtStartPar
\sphinxstyleemphasis{\sphinxstylestrong{Note that \sphinxcode{\sphinxupquote{sco.minimize()}} finds minimums, so we need to minimize the negative Sharpe Ratio.}}

\sphinxAtStartPar
The following code downloads data for the MATANA stocks and assigns daily decimal returns from 2020 through 2022 to data frame \sphinxcode{\sphinxupquote{returns}}.
We will stop in 2022 to make it easier to compare our results, whether we use the risk\sphinxhyphen{}free rate or value\sphinxhyphen{}weighted market portfolio as our benchmark or not.
Recall, the Fama and French benchmark factors are only available with a lag, and are only available through December 2022 as I type.

\begin{sphinxuseclass}{cell}\begin{sphinxVerbatimInput}

\begin{sphinxuseclass}{cell_input}
\begin{sphinxVerbatim}[commandchars=\\\{\}]
\PYG{n}{tickers} \PYG{o}{=} \PYG{l+s+s1}{\PYGZsq{}}\PYG{l+s+s1}{MSFT AAPL TSLA AMZN NVDA GOOG}\PYG{l+s+s1}{\PYGZsq{}}

\PYG{n}{matana} \PYG{o}{=} \PYG{p}{(}
    \PYG{n}{yf}\PYG{o}{.}\PYG{n}{download}\PYG{p}{(}\PYG{n}{tickers}\PYG{o}{=}\PYG{n}{tickers}\PYG{p}{,} \PYG{n}{progress}\PYG{o}{=}\PYG{k+kc}{False}\PYG{p}{)}
    \PYG{o}{.}\PYG{n}{assign}\PYG{p}{(}\PYG{n}{Date}\PYG{o}{=}\PYG{k}{lambda} \PYG{n}{x}\PYG{p}{:} \PYG{n}{x}\PYG{o}{.}\PYG{n}{index}\PYG{o}{.}\PYG{n}{tz\PYGZus{}localize}\PYG{p}{(}\PYG{k+kc}{None}\PYG{p}{)}\PYG{p}{)}
    \PYG{o}{.}\PYG{n}{set\PYGZus{}index}\PYG{p}{(}\PYG{l+s+s1}{\PYGZsq{}}\PYG{l+s+s1}{Date}\PYG{l+s+s1}{\PYGZsq{}}\PYG{p}{)}
    \PYG{o}{.}\PYG{n}{rename\PYGZus{}axis}\PYG{p}{(}\PYG{n}{columns}\PYG{o}{=}\PYG{p}{[}\PYG{l+s+s1}{\PYGZsq{}}\PYG{l+s+s1}{Variable}\PYG{l+s+s1}{\PYGZsq{}}\PYG{p}{,} \PYG{l+s+s1}{\PYGZsq{}}\PYG{l+s+s1}{Ticker}\PYG{l+s+s1}{\PYGZsq{}}\PYG{p}{]}\PYG{p}{)}
\PYG{p}{)}

\PYG{n}{returns} \PYG{o}{=} \PYG{n}{matana}\PYG{p}{[}\PYG{l+s+s1}{\PYGZsq{}}\PYG{l+s+s1}{Adj Close}\PYG{l+s+s1}{\PYGZsq{}}\PYG{p}{]}\PYG{o}{.}\PYG{n}{pct\PYGZus{}change}\PYG{p}{(}\PYG{p}{)}\PYG{o}{.}\PYG{n}{loc}\PYG{p}{[}\PYG{l+s+s1}{\PYGZsq{}}\PYG{l+s+s1}{2020}\PYG{l+s+s1}{\PYGZsq{}}\PYG{p}{:}\PYG{l+s+s1}{\PYGZsq{}}\PYG{l+s+s1}{2022}\PYG{l+s+s1}{\PYGZsq{}}\PYG{p}{]}
\PYG{n}{returns}\PYG{o}{.}\PYG{n}{describe}\PYG{p}{(}\PYG{p}{)}
\end{sphinxVerbatim}

\end{sphinxuseclass}\end{sphinxVerbatimInput}
\begin{sphinxVerbatimOutput}

\begin{sphinxuseclass}{cell_output}
\begin{sphinxVerbatim}[commandchars=\\\{\}]
Ticker     AAPL     AMZN     GOOG     MSFT     NVDA     TSLA
count  756.0000 756.0000 756.0000 756.0000 756.0000 756.0000
mean     0.0011   0.0002   0.0006   0.0008   0.0018   0.0030
std      0.0233   0.0246   0.0217   0.0219   0.0352   0.0455
min     \PYGZhy{}0.1286  \PYGZhy{}0.1405  \PYGZhy{}0.1110  \PYGZhy{}0.1474  \PYGZhy{}0.1845  \PYGZhy{}0.2106
25\PYGZpc{}     \PYGZhy{}0.0110  \PYGZhy{}0.0129  \PYGZhy{}0.0097  \PYGZhy{}0.0097  \PYGZhy{}0.0177  \PYGZhy{}0.0215
50\PYGZpc{}      0.0006   0.0006   0.0012   0.0007   0.0029   0.0020
75\PYGZpc{}      0.0142   0.0123   0.0114   0.0123   0.0222   0.0251
max      0.1198   0.1354   0.0940   0.1422   0.1716   0.1989
\end{sphinxVerbatim}

\end{sphinxuseclass}\end{sphinxVerbatimOutput}

\end{sphinxuseclass}
\begin{sphinxuseclass}{cell}\begin{sphinxVerbatimInput}

\begin{sphinxuseclass}{cell_input}
\begin{sphinxVerbatim}[commandchars=\\\{\}]
\PYG{k}{def} \PYG{n+nf}{port\PYGZus{}sharpe}\PYG{p}{(}\PYG{n}{x}\PYG{p}{,} \PYG{n}{r}\PYG{p}{,} \PYG{n}{tgt}\PYG{p}{,} \PYG{n}{ppy}\PYG{p}{)}\PYG{p}{:}
    \PYG{n}{rp} \PYG{o}{=} \PYG{n}{r}\PYG{o}{.}\PYG{n}{dot}\PYG{p}{(}\PYG{n}{x}\PYG{p}{)}
    \PYG{n}{er} \PYG{o}{=} \PYG{n}{rp}\PYG{o}{.}\PYG{n}{sub}\PYG{p}{(}\PYG{n}{tgt}\PYG{p}{)}
    \PYG{k}{return} \PYG{n}{np}\PYG{o}{.}\PYG{n}{sqrt}\PYG{p}{(}\PYG{n}{ppy}\PYG{p}{)} \PYG{o}{*} \PYG{n}{er}\PYG{o}{.}\PYG{n}{mean}\PYG{p}{(}\PYG{p}{)} \PYG{o}{/} \PYG{n}{er}\PYG{o}{.}\PYG{n}{std}\PYG{p}{(}\PYG{p}{)}
\end{sphinxVerbatim}

\end{sphinxuseclass}\end{sphinxVerbatimInput}

\end{sphinxuseclass}
\begin{sphinxuseclass}{cell}\begin{sphinxVerbatimInput}

\begin{sphinxuseclass}{cell_input}
\begin{sphinxVerbatim}[commandchars=\\\{\}]
\PYG{k}{def} \PYG{n+nf}{port\PYGZus{}sharpe\PYGZus{}neg}\PYG{p}{(}\PYG{n}{x}\PYG{p}{,} \PYG{n}{r}\PYG{p}{,} \PYG{n}{tgt}\PYG{p}{,} \PYG{n}{ppy}\PYG{p}{)}\PYG{p}{:}
    \PYG{k}{return} \PYG{o}{\PYGZhy{}}\PYG{l+m+mi}{1} \PYG{o}{*} \PYG{n}{port\PYGZus{}sharpe}\PYG{p}{(}\PYG{n}{x}\PYG{p}{,} \PYG{n}{r}\PYG{p}{,} \PYG{n}{tgt}\PYG{p}{,} \PYG{n}{ppy}\PYG{p}{)}
\end{sphinxVerbatim}

\end{sphinxuseclass}\end{sphinxVerbatimInput}

\end{sphinxuseclass}
\begin{sphinxuseclass}{cell}\begin{sphinxVerbatimInput}

\begin{sphinxuseclass}{cell_input}
\begin{sphinxVerbatim}[commandchars=\\\{\}]
\PYG{k}{def} \PYG{n+nf}{get\PYGZus{}ew}\PYG{p}{(}\PYG{n}{r}\PYG{p}{)}\PYG{p}{:}
    \PYG{k}{return} \PYG{n}{np}\PYG{o}{.}\PYG{n}{ones}\PYG{p}{(}\PYG{n}{r}\PYG{o}{.}\PYG{n}{shape}\PYG{p}{[}\PYG{l+m+mi}{1}\PYG{p}{]}\PYG{p}{)} \PYG{o}{/} \PYG{n}{r}\PYG{o}{.}\PYG{n}{shape}\PYG{p}{[}\PYG{l+m+mi}{1}\PYG{p}{]}
\end{sphinxVerbatim}

\end{sphinxuseclass}\end{sphinxVerbatimInput}

\end{sphinxuseclass}
\begin{sphinxuseclass}{cell}\begin{sphinxVerbatimInput}

\begin{sphinxuseclass}{cell_input}
\begin{sphinxVerbatim}[commandchars=\\\{\}]
\PYG{n}{get\PYGZus{}ew}\PYG{p}{(}\PYG{n}{returns}\PYG{p}{)}
\end{sphinxVerbatim}

\end{sphinxuseclass}\end{sphinxVerbatimInput}
\begin{sphinxVerbatimOutput}

\begin{sphinxuseclass}{cell_output}
\begin{sphinxVerbatim}[commandchars=\\\{\}]
array([0.1667, 0.1667, 0.1667, 0.1667, 0.1667, 0.1667])
\end{sphinxVerbatim}

\end{sphinxuseclass}\end{sphinxVerbatimOutput}

\end{sphinxuseclass}
\begin{sphinxuseclass}{cell}\begin{sphinxVerbatimInput}

\begin{sphinxuseclass}{cell_input}
\begin{sphinxVerbatim}[commandchars=\\\{\}]
\PYG{p}{[}\PYG{p}{(}\PYG{l+m+mi}{0}\PYG{p}{,} \PYG{l+m+mi}{1}\PYG{p}{)} \PYG{k}{for} \PYG{n}{i} \PYG{o+ow}{in} \PYG{n}{returns}\PYG{p}{]}
\end{sphinxVerbatim}

\end{sphinxuseclass}\end{sphinxVerbatimInput}
\begin{sphinxVerbatimOutput}

\begin{sphinxuseclass}{cell_output}
\begin{sphinxVerbatim}[commandchars=\\\{\}]
[(0, 1), (0, 1), (0, 1), (0, 1), (0, 1), (0, 1)]
\end{sphinxVerbatim}

\end{sphinxuseclass}\end{sphinxVerbatimOutput}

\end{sphinxuseclass}
\begin{sphinxuseclass}{cell}\begin{sphinxVerbatimInput}

\begin{sphinxuseclass}{cell_input}
\begin{sphinxVerbatim}[commandchars=\\\{\}]
\PYG{n}{res\PYGZus{}sharpe\PYGZus{}1} \PYG{o}{=} \PYG{n}{sco}\PYG{o}{.}\PYG{n}{minimize}\PYG{p}{(}
    \PYG{n}{fun}\PYG{o}{=}\PYG{n}{port\PYGZus{}sharpe\PYGZus{}neg}\PYG{p}{,}
    \PYG{n}{x0}\PYG{o}{=}\PYG{n}{get\PYGZus{}ew}\PYG{p}{(}\PYG{n}{returns}\PYG{p}{)}\PYG{p}{,}
    \PYG{n}{args}\PYG{o}{=}\PYG{p}{(}\PYG{n}{returns}\PYG{p}{,} \PYG{l+m+mi}{0}\PYG{p}{,} \PYG{l+m+mi}{252}\PYG{p}{)}\PYG{p}{,}
    \PYG{n}{bounds}\PYG{o}{=}\PYG{p}{[}\PYG{p}{(}\PYG{l+m+mi}{0}\PYG{p}{,} \PYG{l+m+mi}{1}\PYG{p}{)} \PYG{k}{for} \PYG{n}{i} \PYG{o+ow}{in} \PYG{n}{returns}\PYG{p}{]}\PYG{p}{,}
    \PYG{n}{constraints}\PYG{o}{=}\PYG{p}{(}
        \PYG{p}{\PYGZob{}}\PYG{l+s+s1}{\PYGZsq{}}\PYG{l+s+s1}{type}\PYG{l+s+s1}{\PYGZsq{}}\PYG{p}{:} \PYG{l+s+s1}{\PYGZsq{}}\PYG{l+s+s1}{eq}\PYG{l+s+s1}{\PYGZsq{}}\PYG{p}{,} \PYG{l+s+s1}{\PYGZsq{}}\PYG{l+s+s1}{fun}\PYG{l+s+s1}{\PYGZsq{}}\PYG{p}{:} \PYG{k}{lambda} \PYG{n}{x}\PYG{p}{:} \PYG{n}{x}\PYG{o}{.}\PYG{n}{sum}\PYG{p}{(}\PYG{p}{)} \PYG{o}{\PYGZhy{}} \PYG{l+m+mi}{1}\PYG{p}{\PYGZcb{}} \PYG{c+c1}{\PYGZsh{} eq constraint met when equal to zero}
    \PYG{p}{)}
\PYG{p}{)}

\PYG{n}{res\PYGZus{}sharpe\PYGZus{}1}
\end{sphinxVerbatim}

\end{sphinxuseclass}\end{sphinxVerbatimInput}
\begin{sphinxVerbatimOutput}

\begin{sphinxuseclass}{cell_output}
\begin{sphinxVerbatim}[commandchars=\\\{\}]
     fun: \PYGZhy{}1.0891990431411531
     jac: array([ 1.1602e\PYGZhy{}04,  3.6243e\PYGZhy{}01,  1.1961e\PYGZhy{}01,  6.4995e\PYGZhy{}02,  4.9046e\PYGZhy{}04,
       \PYGZhy{}2.5962e\PYGZhy{}04])
 message: \PYGZsq{}Optimization terminated successfully\PYGZsq{}
    nfev: 63
     nit: 9
    njev: 9
  status: 0
 success: True
       x: array([6.6101e\PYGZhy{}02, 2.5912e\PYGZhy{}17, 0.0000e+00, 0.0000e+00, 3.1307e\PYGZhy{}01,
       6.2083e\PYGZhy{}01])
\end{sphinxVerbatim}

\end{sphinxuseclass}\end{sphinxVerbatimOutput}

\end{sphinxuseclass}
\begin{sphinxuseclass}{cell}\begin{sphinxVerbatimInput}

\begin{sphinxuseclass}{cell_input}
\begin{sphinxVerbatim}[commandchars=\\\{\}]
\PYG{n}{port\PYGZus{}sharpe}\PYG{p}{(}\PYG{n}{x}\PYG{o}{=}\PYG{n}{res\PYGZus{}sharpe\PYGZus{}1}\PYG{p}{[}\PYG{l+s+s1}{\PYGZsq{}}\PYG{l+s+s1}{x}\PYG{l+s+s1}{\PYGZsq{}}\PYG{p}{]}\PYG{p}{,} \PYG{n}{r}\PYG{o}{=}\PYG{n}{returns}\PYG{p}{,} \PYG{n}{tgt}\PYG{o}{=}\PYG{l+m+mi}{0}\PYG{p}{,} \PYG{n}{ppy}\PYG{o}{=}\PYG{l+m+mi}{252}\PYG{p}{)}
\end{sphinxVerbatim}

\end{sphinxuseclass}\end{sphinxVerbatimInput}
\begin{sphinxVerbatimOutput}

\begin{sphinxuseclass}{cell_output}
\begin{sphinxVerbatim}[commandchars=\\\{\}]
1.0891990431411531
\end{sphinxVerbatim}

\end{sphinxuseclass}\end{sphinxVerbatimOutput}

\end{sphinxuseclass}

\subsubsection{Find the maximum Sharpe Ratio portfolio of MATANA stocks over the last three years, but allow short weights up to 10\% on each stock}
\label{\detokenize{herron_04_practice_02:find-the-maximum-sharpe-ratio-portfolio-of-matana-stocks-over-the-last-three-years-but-allow-short-weights-up-to-10-on-each-stock}}
\begin{sphinxuseclass}{cell}\begin{sphinxVerbatimInput}

\begin{sphinxuseclass}{cell_input}
\begin{sphinxVerbatim}[commandchars=\\\{\}]
\PYG{n}{res\PYGZus{}sharpe\PYGZus{}2} \PYG{o}{=} \PYG{n}{sco}\PYG{o}{.}\PYG{n}{minimize}\PYG{p}{(}
    \PYG{n}{fun}\PYG{o}{=}\PYG{n}{port\PYGZus{}sharpe\PYGZus{}neg}\PYG{p}{,}
    \PYG{n}{x0}\PYG{o}{=}\PYG{n}{get\PYGZus{}ew}\PYG{p}{(}\PYG{n}{returns}\PYG{p}{)}\PYG{p}{,}
    \PYG{n}{args}\PYG{o}{=}\PYG{p}{(}\PYG{n}{returns}\PYG{p}{,} \PYG{l+m+mi}{0}\PYG{p}{,} \PYG{l+m+mi}{252}\PYG{p}{)}\PYG{p}{,}
    \PYG{n}{bounds}\PYG{o}{=}\PYG{p}{[}\PYG{p}{(}\PYG{o}{\PYGZhy{}}\PYG{l+m+mf}{0.1}\PYG{p}{,} \PYG{l+m+mf}{1.5}\PYG{p}{)} \PYG{k}{for} \PYG{n}{i} \PYG{o+ow}{in} \PYG{n}{returns}\PYG{p}{]}\PYG{p}{,}
    \PYG{n}{constraints}\PYG{o}{=}\PYG{p}{(}
        \PYG{p}{\PYGZob{}}\PYG{l+s+s1}{\PYGZsq{}}\PYG{l+s+s1}{type}\PYG{l+s+s1}{\PYGZsq{}}\PYG{p}{:} \PYG{l+s+s1}{\PYGZsq{}}\PYG{l+s+s1}{eq}\PYG{l+s+s1}{\PYGZsq{}}\PYG{p}{,} \PYG{l+s+s1}{\PYGZsq{}}\PYG{l+s+s1}{fun}\PYG{l+s+s1}{\PYGZsq{}}\PYG{p}{:} \PYG{k}{lambda} \PYG{n}{x}\PYG{p}{:} \PYG{n}{x}\PYG{o}{.}\PYG{n}{sum}\PYG{p}{(}\PYG{p}{)} \PYG{o}{\PYGZhy{}} \PYG{l+m+mi}{1}\PYG{p}{\PYGZcb{}} \PYG{c+c1}{\PYGZsh{} eq constraint met when equal to zero}
    \PYG{p}{)}
\PYG{p}{)}

\PYG{n}{res\PYGZus{}sharpe\PYGZus{}2}
\end{sphinxVerbatim}

\end{sphinxuseclass}\end{sphinxVerbatimInput}
\begin{sphinxVerbatimOutput}

\begin{sphinxuseclass}{cell_output}
\begin{sphinxVerbatim}[commandchars=\\\{\}]
     fun: \PYGZhy{}1.1368989996984662
     jac: array([0.041 , 0.3444, 0.1157, 0.0727, 0.041 , 0.041 ])
 message: \PYGZsq{}Optimization terminated successfully\PYGZsq{}
    nfev: 63
     nit: 9
    njev: 9
  status: 0
 success: True
       x: array([ 0.2997, \PYGZhy{}0.1   , \PYGZhy{}0.1   , \PYGZhy{}0.1   ,  0.3745,  0.6258])
\end{sphinxVerbatim}

\end{sphinxuseclass}\end{sphinxVerbatimOutput}

\end{sphinxuseclass}
\begin{sphinxuseclass}{cell}\begin{sphinxVerbatimInput}

\begin{sphinxuseclass}{cell_input}
\begin{sphinxVerbatim}[commandchars=\\\{\}]
\PYG{p}{(}
    \PYG{n}{pd}\PYG{o}{.}\PYG{n}{DataFrame}\PYG{p}{(}
        \PYG{n}{data}\PYG{o}{=}\PYG{p}{\PYGZob{}}
            \PYG{l+s+s1}{\PYGZsq{}}\PYG{l+s+s1}{Long Only}\PYG{l+s+s1}{\PYGZsq{}}\PYG{p}{:}\PYG{n}{res\PYGZus{}sharpe\PYGZus{}1}\PYG{p}{[}\PYG{l+s+s1}{\PYGZsq{}}\PYG{l+s+s1}{x}\PYG{l+s+s1}{\PYGZsq{}}\PYG{p}{]}\PYG{p}{,} 
            \PYG{l+s+s1}{\PYGZsq{}}\PYG{l+s+s1}{Up to 10}\PYG{l+s+s1}{\PYGZpc{}}\PYG{l+s+s1}{ Short per Stock}\PYG{l+s+s1}{\PYGZsq{}}\PYG{p}{:}\PYG{n}{res\PYGZus{}sharpe\PYGZus{}2}\PYG{p}{[}\PYG{l+s+s1}{\PYGZsq{}}\PYG{l+s+s1}{x}\PYG{l+s+s1}{\PYGZsq{}}\PYG{p}{]}
        \PYG{p}{\PYGZcb{}}\PYG{p}{,}
        \PYG{n}{index}\PYG{o}{=}\PYG{n}{returns}\PYG{o}{.}\PYG{n}{columns}
    \PYG{p}{)}
    \PYG{o}{.}\PYG{n}{rename\PYGZus{}axis}\PYG{p}{(}\PYG{l+s+s1}{\PYGZsq{}}\PYG{l+s+s1}{Portfolio Weight}\PYG{l+s+s1}{\PYGZsq{}}\PYG{p}{)}
    \PYG{o}{.}\PYG{n}{plot}\PYG{p}{(}\PYG{n}{kind}\PYG{o}{=}\PYG{l+s+s1}{\PYGZsq{}}\PYG{l+s+s1}{barh}\PYG{l+s+s1}{\PYGZsq{}}\PYG{p}{)}
\PYG{p}{)}
\PYG{n}{plt}\PYG{o}{.}\PYG{n}{title}\PYG{p}{(}\PYG{l+s+s1}{\PYGZsq{}}\PYG{l+s+s1}{Comparison Max. Sharpe Ratio Portfolio Weights}\PYG{l+s+s1}{\PYGZsq{}}\PYG{p}{)}
\PYG{n}{plt}\PYG{o}{.}\PYG{n}{show}\PYG{p}{(}\PYG{p}{)}
\end{sphinxVerbatim}

\end{sphinxuseclass}\end{sphinxVerbatimInput}
\begin{sphinxVerbatimOutput}

\begin{sphinxuseclass}{cell_output}
\noindent\sphinxincludegraphics{{004163d333021c4843934510f0bca8e93ff174a2d8ca8aae95ac07868ddfaaea}.png}

\end{sphinxuseclass}\end{sphinxVerbatimOutput}

\end{sphinxuseclass}
\sphinxAtStartPar
By relaxing the long\sphinxhyphen{}only constrain (via changes to \sphinxcode{\sphinxupquote{bounds=}}), the weights on AMZN, GOOG, and MSFT go from zero to \sphinxhyphen{}10\%.
Also, the Sharpe Ratio increases because we relax a binding constraint.

\begin{sphinxuseclass}{cell}\begin{sphinxVerbatimInput}

\begin{sphinxuseclass}{cell_input}
\begin{sphinxVerbatim}[commandchars=\\\{\}]
\PYG{n}{port\PYGZus{}sharpe}\PYG{p}{(}\PYG{n}{res\PYGZus{}sharpe\PYGZus{}1}\PYG{p}{[}\PYG{l+s+s1}{\PYGZsq{}}\PYG{l+s+s1}{x}\PYG{l+s+s1}{\PYGZsq{}}\PYG{p}{]}\PYG{p}{,} \PYG{n}{r}\PYG{o}{=}\PYG{n}{returns}\PYG{p}{,} \PYG{n}{tgt}\PYG{o}{=}\PYG{l+m+mi}{0}\PYG{p}{,} \PYG{n}{ppy}\PYG{o}{=}\PYG{l+m+mi}{252}\PYG{p}{)}
\end{sphinxVerbatim}

\end{sphinxuseclass}\end{sphinxVerbatimInput}
\begin{sphinxVerbatimOutput}

\begin{sphinxuseclass}{cell_output}
\begin{sphinxVerbatim}[commandchars=\\\{\}]
1.0891990431411531
\end{sphinxVerbatim}

\end{sphinxuseclass}\end{sphinxVerbatimOutput}

\end{sphinxuseclass}
\begin{sphinxuseclass}{cell}\begin{sphinxVerbatimInput}

\begin{sphinxuseclass}{cell_input}
\begin{sphinxVerbatim}[commandchars=\\\{\}]
\PYG{n}{port\PYGZus{}sharpe}\PYG{p}{(}\PYG{n}{res\PYGZus{}sharpe\PYGZus{}2}\PYG{p}{[}\PYG{l+s+s1}{\PYGZsq{}}\PYG{l+s+s1}{x}\PYG{l+s+s1}{\PYGZsq{}}\PYG{p}{]}\PYG{p}{,} \PYG{n}{r}\PYG{o}{=}\PYG{n}{returns}\PYG{p}{,} \PYG{n}{tgt}\PYG{o}{=}\PYG{l+m+mi}{0}\PYG{p}{,} \PYG{n}{ppy}\PYG{o}{=}\PYG{l+m+mi}{252}\PYG{p}{)}
\end{sphinxVerbatim}

\end{sphinxuseclass}\end{sphinxVerbatimInput}
\begin{sphinxVerbatimOutput}

\begin{sphinxuseclass}{cell_output}
\begin{sphinxVerbatim}[commandchars=\\\{\}]
1.1368989996984662
\end{sphinxVerbatim}

\end{sphinxuseclass}\end{sphinxVerbatimOutput}

\end{sphinxuseclass}

\subsubsection{Find the maximum Sharpe Ratio portfolio of MATANA stocks over the last three years, but allow total short weights of up to 30\%}
\label{\detokenize{herron_04_practice_02:find-the-maximum-sharpe-ratio-portfolio-of-matana-stocks-over-the-last-three-years-but-allow-total-short-weights-of-up-to-30}}
\sphinxAtStartPar
We can find the negative values in a NumPy array as follows.

\begin{sphinxuseclass}{cell}\begin{sphinxVerbatimInput}

\begin{sphinxuseclass}{cell_input}
\begin{sphinxVerbatim}[commandchars=\\\{\}]
\PYG{n}{x} \PYG{o}{=} \PYG{n}{np}\PYG{o}{.}\PYG{n}{arange}\PYG{p}{(}\PYG{l+m+mi}{6}\PYG{p}{)} \PYG{o}{\PYGZhy{}} \PYG{l+m+mi}{3}
\PYG{n}{x}\PYG{p}{[}\PYG{n}{x} \PYG{o}{\PYGZlt{}} \PYG{l+m+mi}{0}\PYG{p}{]}
\end{sphinxVerbatim}

\end{sphinxuseclass}\end{sphinxVerbatimInput}
\begin{sphinxVerbatimOutput}

\begin{sphinxuseclass}{cell_output}
\begin{sphinxVerbatim}[commandchars=\\\{\}]
array([\PYGZhy{}3, \PYGZhy{}2, \PYGZhy{}1])
\end{sphinxVerbatim}

\end{sphinxuseclass}\end{sphinxVerbatimOutput}

\end{sphinxuseclass}
\begin{sphinxuseclass}{cell}\begin{sphinxVerbatimInput}

\begin{sphinxuseclass}{cell_input}
\begin{sphinxVerbatim}[commandchars=\\\{\}]
\PYG{n}{res\PYGZus{}sharpe\PYGZus{}3} \PYG{o}{=} \PYG{n}{sco}\PYG{o}{.}\PYG{n}{minimize}\PYG{p}{(}
    \PYG{n}{fun}\PYG{o}{=}\PYG{n}{port\PYGZus{}sharpe\PYGZus{}neg}\PYG{p}{,}
    \PYG{n}{x0}\PYG{o}{=}\PYG{n}{get\PYGZus{}ew}\PYG{p}{(}\PYG{n}{returns}\PYG{p}{)}\PYG{p}{,}
    \PYG{n}{args}\PYG{o}{=}\PYG{p}{(}\PYG{n}{returns}\PYG{p}{,} \PYG{l+m+mi}{0}\PYG{p}{,} \PYG{l+m+mi}{252}\PYG{p}{)}\PYG{p}{,}
    \PYG{n}{bounds}\PYG{o}{=}\PYG{p}{[}\PYG{p}{(}\PYG{o}{\PYGZhy{}}\PYG{l+m+mf}{0.3}\PYG{p}{,} \PYG{l+m+mf}{1.3}\PYG{p}{)} \PYG{k}{for} \PYG{n}{i} \PYG{o+ow}{in} \PYG{n}{returns}\PYG{p}{]}\PYG{p}{,}
    \PYG{n}{constraints}\PYG{o}{=}\PYG{p}{(}
        \PYG{p}{\PYGZob{}}\PYG{l+s+s1}{\PYGZsq{}}\PYG{l+s+s1}{type}\PYG{l+s+s1}{\PYGZsq{}}\PYG{p}{:} \PYG{l+s+s1}{\PYGZsq{}}\PYG{l+s+s1}{eq}\PYG{l+s+s1}{\PYGZsq{}}\PYG{p}{,} \PYG{l+s+s1}{\PYGZsq{}}\PYG{l+s+s1}{fun}\PYG{l+s+s1}{\PYGZsq{}}\PYG{p}{:} \PYG{k}{lambda} \PYG{n}{x}\PYG{p}{:} \PYG{n}{x}\PYG{o}{.}\PYG{n}{sum}\PYG{p}{(}\PYG{p}{)} \PYG{o}{\PYGZhy{}} \PYG{l+m+mi}{1}\PYG{p}{\PYGZcb{}}\PYG{p}{,} \PYG{c+c1}{\PYGZsh{} eq constraint met when = 0}
        \PYG{p}{\PYGZob{}}\PYG{l+s+s1}{\PYGZsq{}}\PYG{l+s+s1}{type}\PYG{l+s+s1}{\PYGZsq{}}\PYG{p}{:} \PYG{l+s+s1}{\PYGZsq{}}\PYG{l+s+s1}{ineq}\PYG{l+s+s1}{\PYGZsq{}}\PYG{p}{,} \PYG{l+s+s1}{\PYGZsq{}}\PYG{l+s+s1}{fun}\PYG{l+s+s1}{\PYGZsq{}}\PYG{p}{:} \PYG{k}{lambda} \PYG{n}{x}\PYG{p}{:} \PYG{n}{x}\PYG{p}{[}\PYG{n}{x}\PYG{o}{\PYGZlt{}}\PYG{l+m+mi}{0}\PYG{p}{]}\PYG{o}{.}\PYG{n}{sum}\PYG{p}{(}\PYG{p}{)} \PYG{o}{+} \PYG{l+m+mf}{0.3}\PYG{p}{\PYGZcb{}} \PYG{c+c1}{\PYGZsh{} ineq constraint met when \PYGZgt{}= 0}
    \PYG{p}{)}
\PYG{p}{)}

\PYG{n}{res\PYGZus{}sharpe\PYGZus{}3}
\end{sphinxVerbatim}

\end{sphinxuseclass}\end{sphinxVerbatimInput}
\begin{sphinxVerbatimOutput}

\begin{sphinxuseclass}{cell_output}
\begin{sphinxVerbatim}[commandchars=\\\{\}]
     fun: \PYGZhy{}1.1826810753664976
     jac: array([0.0782, 0.336 , 0.152 , 0.1096, 0.0773, 0.0773])
 message: \PYGZsq{}Optimization terminated successfully\PYGZsq{}
    nfev: 100
     nit: 10
    njev: 10
  status: 0
 success: True
       x: array([ 3.5667e\PYGZhy{}01, \PYGZhy{}3.0000e\PYGZhy{}01, \PYGZhy{}1.7501e\PYGZhy{}09,  1.7759e\PYGZhy{}05,  3.4960e\PYGZhy{}01,
        5.9371e\PYGZhy{}01])
\end{sphinxVerbatim}

\end{sphinxuseclass}\end{sphinxVerbatimOutput}

\end{sphinxuseclass}
\begin{sphinxuseclass}{cell}\begin{sphinxVerbatimInput}

\begin{sphinxuseclass}{cell_input}
\begin{sphinxVerbatim}[commandchars=\\\{\}]
\PYG{p}{(}
    \PYG{n}{pd}\PYG{o}{.}\PYG{n}{DataFrame}\PYG{p}{(}
        \PYG{n}{data}\PYG{o}{=}\PYG{p}{\PYGZob{}}
            \PYG{l+s+s1}{\PYGZsq{}}\PYG{l+s+s1}{Long Only}\PYG{l+s+s1}{\PYGZsq{}}\PYG{p}{:}\PYG{n}{res\PYGZus{}sharpe\PYGZus{}1}\PYG{p}{[}\PYG{l+s+s1}{\PYGZsq{}}\PYG{l+s+s1}{x}\PYG{l+s+s1}{\PYGZsq{}}\PYG{p}{]}\PYG{p}{,} 
            \PYG{l+s+s1}{\PYGZsq{}}\PYG{l+s+s1}{Up to 30}\PYG{l+s+s1}{\PYGZpc{}}\PYG{l+s+s1}{ Short Total}\PYG{l+s+s1}{\PYGZsq{}}\PYG{p}{:}\PYG{n}{res\PYGZus{}sharpe\PYGZus{}2}\PYG{p}{[}\PYG{l+s+s1}{\PYGZsq{}}\PYG{l+s+s1}{x}\PYG{l+s+s1}{\PYGZsq{}}\PYG{p}{]}
        \PYG{p}{\PYGZcb{}}\PYG{p}{,}
        \PYG{n}{index}\PYG{o}{=}\PYG{n}{returns}\PYG{o}{.}\PYG{n}{columns}
    \PYG{p}{)}
    \PYG{o}{.}\PYG{n}{rename\PYGZus{}axis}\PYG{p}{(}\PYG{l+s+s1}{\PYGZsq{}}\PYG{l+s+s1}{Portfolio Weight}\PYG{l+s+s1}{\PYGZsq{}}\PYG{p}{)}
    \PYG{o}{.}\PYG{n}{plot}\PYG{p}{(}\PYG{n}{kind}\PYG{o}{=}\PYG{l+s+s1}{\PYGZsq{}}\PYG{l+s+s1}{barh}\PYG{l+s+s1}{\PYGZsq{}}\PYG{p}{)}
\PYG{p}{)}
\PYG{n}{plt}\PYG{o}{.}\PYG{n}{title}\PYG{p}{(}\PYG{l+s+s1}{\PYGZsq{}}\PYG{l+s+s1}{Comparison Max. Sharpe Ratio Portfolios}\PYG{l+s+s1}{\PYGZsq{}}\PYG{p}{)}
\PYG{n}{plt}\PYG{o}{.}\PYG{n}{show}\PYG{p}{(}\PYG{p}{)}
\end{sphinxVerbatim}

\end{sphinxuseclass}\end{sphinxVerbatimInput}
\begin{sphinxVerbatimOutput}

\begin{sphinxuseclass}{cell_output}
\noindent\sphinxincludegraphics{{9191ee6b277076599561ec2ccfc4e2c05a524e1504c24b98f5dfcbe9c357f2a1}.png}

\end{sphinxuseclass}\end{sphinxVerbatimOutput}

\end{sphinxuseclass}
\sphinxAtStartPar
Again, by relaxing the long\sphinxhyphen{}only constrain, the weights on AMZN, GOOG, and MSFT go from zero to \sphinxhyphen{}10\%.
Also, the Sharpe Ratio increases because we relax a binding constraints.
The Sharpe Ratio is higher here than in the previous exercise, but this will not always be the case, since we relax different constraints here and in the previous exercise.

\begin{sphinxuseclass}{cell}\begin{sphinxVerbatimInput}

\begin{sphinxuseclass}{cell_input}
\begin{sphinxVerbatim}[commandchars=\\\{\}]
\PYG{n}{port\PYGZus{}sharpe}\PYG{p}{(}\PYG{n}{res\PYGZus{}sharpe\PYGZus{}1}\PYG{p}{[}\PYG{l+s+s1}{\PYGZsq{}}\PYG{l+s+s1}{x}\PYG{l+s+s1}{\PYGZsq{}}\PYG{p}{]}\PYG{p}{,} \PYG{n}{r}\PYG{o}{=}\PYG{n}{returns}\PYG{p}{,} \PYG{n}{ppy}\PYG{o}{=}\PYG{l+m+mi}{252}\PYG{p}{,} \PYG{n}{tgt}\PYG{o}{=}\PYG{l+m+mi}{0}\PYG{p}{)}
\end{sphinxVerbatim}

\end{sphinxuseclass}\end{sphinxVerbatimInput}
\begin{sphinxVerbatimOutput}

\begin{sphinxuseclass}{cell_output}
\begin{sphinxVerbatim}[commandchars=\\\{\}]
1.0891990431411531
\end{sphinxVerbatim}

\end{sphinxuseclass}\end{sphinxVerbatimOutput}

\end{sphinxuseclass}
\begin{sphinxuseclass}{cell}\begin{sphinxVerbatimInput}

\begin{sphinxuseclass}{cell_input}
\begin{sphinxVerbatim}[commandchars=\\\{\}]
\PYG{n}{port\PYGZus{}sharpe}\PYG{p}{(}\PYG{n}{res\PYGZus{}sharpe\PYGZus{}3}\PYG{p}{[}\PYG{l+s+s1}{\PYGZsq{}}\PYG{l+s+s1}{x}\PYG{l+s+s1}{\PYGZsq{}}\PYG{p}{]}\PYG{p}{,} \PYG{n}{r}\PYG{o}{=}\PYG{n}{returns}\PYG{p}{,} \PYG{n}{ppy}\PYG{o}{=}\PYG{l+m+mi}{252}\PYG{p}{,} \PYG{n}{tgt}\PYG{o}{=}\PYG{l+m+mi}{0}\PYG{p}{)}
\end{sphinxVerbatim}

\end{sphinxuseclass}\end{sphinxVerbatimInput}
\begin{sphinxVerbatimOutput}

\begin{sphinxuseclass}{cell_output}
\begin{sphinxVerbatim}[commandchars=\\\{\}]
1.1826810753664976
\end{sphinxVerbatim}

\end{sphinxuseclass}\end{sphinxVerbatimOutput}

\end{sphinxuseclass}

\subsubsection{Find the maximum Sharpe Ratio portfolio of MATANA stocks over the last three years, but do not allow any weight to exceed 30\% in magnitude}
\label{\detokenize{herron_04_practice_02:find-the-maximum-sharpe-ratio-portfolio-of-matana-stocks-over-the-last-three-years-but-do-not-allow-any-weight-to-exceed-30-in-magnitude}}
\begin{sphinxuseclass}{cell}\begin{sphinxVerbatimInput}

\begin{sphinxuseclass}{cell_input}
\begin{sphinxVerbatim}[commandchars=\\\{\}]
\PYG{n}{res\PYGZus{}sharpe\PYGZus{}4} \PYG{o}{=} \PYG{n}{sco}\PYG{o}{.}\PYG{n}{minimize}\PYG{p}{(}
    \PYG{n}{fun}\PYG{o}{=}\PYG{n}{port\PYGZus{}sharpe\PYGZus{}neg}\PYG{p}{,}
    \PYG{n}{x0}\PYG{o}{=}\PYG{n}{np}\PYG{o}{.}\PYG{n}{ones}\PYG{p}{(}\PYG{n}{returns}\PYG{o}{.}\PYG{n}{shape}\PYG{p}{[}\PYG{l+m+mi}{1}\PYG{p}{]}\PYG{p}{)} \PYG{o}{/} \PYG{n}{returns}\PYG{o}{.}\PYG{n}{shape}\PYG{p}{[}\PYG{l+m+mi}{1}\PYG{p}{]}\PYG{p}{,}
    \PYG{n}{args}\PYG{o}{=}\PYG{p}{(}\PYG{n}{returns}\PYG{p}{,} \PYG{l+m+mi}{252}\PYG{p}{,} \PYG{l+m+mi}{0}\PYG{p}{)}\PYG{p}{,}
    \PYG{n}{tol}\PYG{o}{=}\PYG{l+m+mf}{1e\PYGZhy{}6}\PYG{p}{,}
    \PYG{n}{bounds}\PYG{o}{=}\PYG{p}{[}\PYG{p}{(}\PYG{l+m+mi}{0}\PYG{p}{,}\PYG{l+m+mf}{0.3}\PYG{p}{)} \PYG{k}{for} \PYG{n}{\PYGZus{}} \PYG{o+ow}{in} \PYG{n+nb}{range}\PYG{p}{(}\PYG{n}{returns}\PYG{o}{.}\PYG{n}{shape}\PYG{p}{[}\PYG{l+m+mi}{1}\PYG{p}{]}\PYG{p}{)}\PYG{p}{]}\PYG{p}{,}
    \PYG{n}{constraints}\PYG{o}{=}\PYG{p}{(}
        \PYG{p}{\PYGZob{}}\PYG{l+s+s1}{\PYGZsq{}}\PYG{l+s+s1}{type}\PYG{l+s+s1}{\PYGZsq{}}\PYG{p}{:} \PYG{l+s+s1}{\PYGZsq{}}\PYG{l+s+s1}{eq}\PYG{l+s+s1}{\PYGZsq{}}\PYG{p}{,} \PYG{l+s+s1}{\PYGZsq{}}\PYG{l+s+s1}{fun}\PYG{l+s+s1}{\PYGZsq{}}\PYG{p}{:} \PYG{k}{lambda} \PYG{n}{x}\PYG{p}{:} \PYG{n}{x}\PYG{o}{.}\PYG{n}{sum}\PYG{p}{(}\PYG{p}{)} \PYG{o}{\PYGZhy{}} \PYG{l+m+mi}{1}\PYG{p}{\PYGZcb{}}\PYG{p}{,} \PYG{c+c1}{\PYGZsh{} want eq constraint to = 0}
    \PYG{p}{)}
\PYG{p}{)}

\PYG{n}{res\PYGZus{}sharpe\PYGZus{}4}
\end{sphinxVerbatim}

\end{sphinxuseclass}\end{sphinxVerbatimInput}
\begin{sphinxVerbatimOutput}

\begin{sphinxuseclass}{cell_output}
\begin{sphinxVerbatim}[commandchars=\\\{\}]
     fun: 0.0
     jac: array([0., 0., 0., 0., 0., 0.])
 message: \PYGZsq{}Optimization terminated successfully\PYGZsq{}
    nfev: 7
     nit: 1
    njev: 1
  status: 0
 success: True
       x: array([0.1667, 0.1667, 0.1667, 0.1667, 0.1667, 0.1667])
\end{sphinxVerbatim}

\end{sphinxuseclass}\end{sphinxVerbatimOutput}

\end{sphinxuseclass}

\subsubsection{Find the minimum 95\% Value at Risk (Var) portfolio of MATANA stocks over the last three years}
\label{\detokenize{herron_04_practice_02:find-the-minimum-95-value-at-risk-var-portfolio-of-matana-stocks-over-the-last-three-years}}
\sphinxAtStartPar
More on VaR \sphinxhref{https://en.wikipedia.org/wiki/Value\_at\_risk}{here}.

\begin{sphinxuseclass}{cell}\begin{sphinxVerbatimInput}

\begin{sphinxuseclass}{cell_input}
\begin{sphinxVerbatim}[commandchars=\\\{\}]
\PYG{k}{def} \PYG{n+nf}{port\PYGZus{}var}\PYG{p}{(}\PYG{n}{x}\PYG{p}{,} \PYG{n}{r}\PYG{p}{,} \PYG{n}{q}\PYG{p}{)}\PYG{p}{:}
    \PYG{k}{return} \PYG{n}{r}\PYG{o}{.}\PYG{n}{dot}\PYG{p}{(}\PYG{n}{x}\PYG{p}{)}\PYG{o}{.}\PYG{n}{quantile}\PYG{p}{(}\PYG{n}{q}\PYG{p}{)}
\end{sphinxVerbatim}

\end{sphinxuseclass}\end{sphinxVerbatimInput}

\end{sphinxuseclass}
\begin{sphinxuseclass}{cell}\begin{sphinxVerbatimInput}

\begin{sphinxuseclass}{cell_input}
\begin{sphinxVerbatim}[commandchars=\\\{\}]
\PYG{k}{def} \PYG{n+nf}{port\PYGZus{}var\PYGZus{}neg}\PYG{p}{(}\PYG{n}{x}\PYG{p}{,} \PYG{n}{r}\PYG{p}{,} \PYG{n}{q}\PYG{p}{)}\PYG{p}{:}
    \PYG{k}{return} \PYG{o}{\PYGZhy{}}\PYG{l+m+mi}{1} \PYG{o}{*} \PYG{n}{port\PYGZus{}var}\PYG{p}{(}\PYG{n}{x}\PYG{o}{=}\PYG{n}{x}\PYG{p}{,} \PYG{n}{r}\PYG{o}{=}\PYG{n}{r}\PYG{p}{,} \PYG{n}{q}\PYG{o}{=}\PYG{n}{q}\PYG{p}{)}
\end{sphinxVerbatim}

\end{sphinxuseclass}\end{sphinxVerbatimInput}

\end{sphinxuseclass}
\begin{sphinxuseclass}{cell}\begin{sphinxVerbatimInput}

\begin{sphinxuseclass}{cell_input}
\begin{sphinxVerbatim}[commandchars=\\\{\}]
\PYG{n}{res\PYGZus{}var\PYGZus{}1} \PYG{o}{=} \PYG{n}{sco}\PYG{o}{.}\PYG{n}{minimize}\PYG{p}{(}
    \PYG{n}{fun}\PYG{o}{=}\PYG{n}{port\PYGZus{}var\PYGZus{}neg}\PYG{p}{,}
    \PYG{n}{x0}\PYG{o}{=}\PYG{n}{np}\PYG{o}{.}\PYG{n}{ones}\PYG{p}{(}\PYG{n}{returns}\PYG{o}{.}\PYG{n}{shape}\PYG{p}{[}\PYG{l+m+mi}{1}\PYG{p}{]}\PYG{p}{)} \PYG{o}{/} \PYG{n}{returns}\PYG{o}{.}\PYG{n}{shape}\PYG{p}{[}\PYG{l+m+mi}{1}\PYG{p}{]}\PYG{p}{,}
    \PYG{n}{args}\PYG{o}{=}\PYG{p}{(}\PYG{n}{returns}\PYG{p}{,} \PYG{l+m+mf}{0.05}\PYG{p}{)}\PYG{p}{,}
    \PYG{n}{bounds}\PYG{o}{=}\PYG{p}{[}\PYG{p}{(}\PYG{l+m+mi}{0}\PYG{p}{,}\PYG{l+m+mi}{1}\PYG{p}{)} \PYG{k}{for} \PYG{n}{\PYGZus{}} \PYG{o+ow}{in} \PYG{n}{returns}\PYG{p}{]}\PYG{p}{,}
    \PYG{n}{constraints}\PYG{o}{=}\PYG{p}{(}
        \PYG{p}{\PYGZob{}}\PYG{l+s+s1}{\PYGZsq{}}\PYG{l+s+s1}{type}\PYG{l+s+s1}{\PYGZsq{}}\PYG{p}{:} \PYG{l+s+s1}{\PYGZsq{}}\PYG{l+s+s1}{eq}\PYG{l+s+s1}{\PYGZsq{}}\PYG{p}{,} \PYG{l+s+s1}{\PYGZsq{}}\PYG{l+s+s1}{fun}\PYG{l+s+s1}{\PYGZsq{}}\PYG{p}{:} \PYG{k}{lambda} \PYG{n}{x}\PYG{p}{:} \PYG{n}{x}\PYG{o}{.}\PYG{n}{sum}\PYG{p}{(}\PYG{p}{)} \PYG{o}{\PYGZhy{}} \PYG{l+m+mi}{1}\PYG{p}{\PYGZcb{}}\PYG{p}{,} \PYG{c+c1}{\PYGZsh{} minimize drives \PYGZdq{}eq\PYGZdq{} constraints to zero}
    \PYG{p}{)}
\PYG{p}{)}

\PYG{n}{res\PYGZus{}var\PYGZus{}1}
\end{sphinxVerbatim}

\end{sphinxuseclass}\end{sphinxVerbatimInput}
\begin{sphinxVerbatimOutput}

\begin{sphinxuseclass}{cell_output}
\begin{sphinxVerbatim}[commandchars=\\\{\}]
     fun: 0.035922474334542105
     jac: array([0.0359, 0.0419, 0.037 , 0.0385, 0.0275, 0.0416])
 message: \PYGZsq{}Optimization terminated successfully\PYGZsq{}
    nfev: 45
     nit: 5
    njev: 5
  status: 0
 success: True
       x: array([1.8180e\PYGZhy{}01, 1.4369e\PYGZhy{}01, 2.2517e\PYGZhy{}01, 2.4347e\PYGZhy{}01, 2.0587e\PYGZhy{}01,
       5.2421e\PYGZhy{}20])
\end{sphinxVerbatim}

\end{sphinxuseclass}\end{sphinxVerbatimOutput}

\end{sphinxuseclass}
\begin{sphinxuseclass}{cell}\begin{sphinxVerbatimInput}

\begin{sphinxuseclass}{cell_input}
\begin{sphinxVerbatim}[commandchars=\\\{\}]
\PYG{n}{port\PYGZus{}var}\PYG{p}{(}\PYG{n}{x}\PYG{o}{=}\PYG{n}{res\PYGZus{}var\PYGZus{}1}\PYG{p}{[}\PYG{l+s+s1}{\PYGZsq{}}\PYG{l+s+s1}{x}\PYG{l+s+s1}{\PYGZsq{}}\PYG{p}{]}\PYG{p}{,} \PYG{n}{r}\PYG{o}{=}\PYG{n}{returns}\PYG{p}{,} \PYG{n}{q}\PYG{o}{=}\PYG{l+m+mf}{0.05}\PYG{p}{)}
\end{sphinxVerbatim}

\end{sphinxuseclass}\end{sphinxVerbatimInput}
\begin{sphinxVerbatimOutput}

\begin{sphinxuseclass}{cell_output}
\begin{sphinxVerbatim}[commandchars=\\\{\}]
\PYGZhy{}0.035922474334542105
\end{sphinxVerbatim}

\end{sphinxuseclass}\end{sphinxVerbatimOutput}

\end{sphinxuseclass}
\sphinxAtStartPar
It might be helpful to slightly change then minimum VaR portfolio weights to show that we minimized VaR.

\begin{sphinxuseclass}{cell}\begin{sphinxVerbatimInput}

\begin{sphinxuseclass}{cell_input}
\begin{sphinxVerbatim}[commandchars=\\\{\}]
\PYG{k}{def} \PYG{n+nf}{tweak}\PYG{p}{(}\PYG{n}{x}\PYG{p}{,} \PYG{n}{d}\PYG{o}{=}\PYG{l+m+mf}{0.05}\PYG{p}{)}\PYG{p}{:}
    \PYG{n}{y} \PYG{o}{=} \PYG{n}{np}\PYG{o}{.}\PYG{n}{zeros}\PYG{p}{(}\PYG{n}{x}\PYG{o}{.}\PYG{n}{shape}\PYG{p}{[}\PYG{l+m+mi}{0}\PYG{p}{]}\PYG{p}{)}
    \PYG{n}{y}\PYG{p}{[}\PYG{l+m+mi}{0}\PYG{p}{]}\PYG{p}{,} \PYG{n}{y}\PYG{p}{[}\PYG{l+m+mi}{1}\PYG{p}{]} \PYG{o}{=} \PYG{n}{d}\PYG{p}{,} \PYG{o}{\PYGZhy{}}\PYG{l+m+mi}{1} \PYG{o}{*} \PYG{n}{d}
    \PYG{k}{return} \PYG{n}{x} \PYG{o}{+} \PYG{n}{y}
\end{sphinxVerbatim}

\end{sphinxuseclass}\end{sphinxVerbatimInput}

\end{sphinxuseclass}
\begin{sphinxuseclass}{cell}\begin{sphinxVerbatimInput}

\begin{sphinxuseclass}{cell_input}
\begin{sphinxVerbatim}[commandchars=\\\{\}]
\PYG{n}{port\PYGZus{}var}\PYG{p}{(}\PYG{n}{x}\PYG{o}{=}\PYG{n}{tweak}\PYG{p}{(}\PYG{n}{res\PYGZus{}var\PYGZus{}1}\PYG{p}{[}\PYG{l+s+s1}{\PYGZsq{}}\PYG{l+s+s1}{x}\PYG{l+s+s1}{\PYGZsq{}}\PYG{p}{]}\PYG{p}{)}\PYG{p}{,} \PYG{n}{r}\PYG{o}{=}\PYG{n}{returns}\PYG{p}{,} \PYG{n}{q}\PYG{o}{=}\PYG{l+m+mf}{0.05}\PYG{p}{)}
\end{sphinxVerbatim}

\end{sphinxuseclass}\end{sphinxVerbatimInput}
\begin{sphinxVerbatimOutput}

\begin{sphinxuseclass}{cell_output}
\begin{sphinxVerbatim}[commandchars=\\\{\}]
\PYGZhy{}0.036229546375465875
\end{sphinxVerbatim}

\end{sphinxuseclass}\end{sphinxVerbatimOutput}

\end{sphinxuseclass}

\subsubsection{Find the minimum maximum draw down portfolio of MATANA stocks over the last three years}
\label{\detokenize{herron_04_practice_02:find-the-minimum-maximum-draw-down-portfolio-of-matana-stocks-over-the-last-three-years}}
\begin{sphinxuseclass}{cell}\begin{sphinxVerbatimInput}

\begin{sphinxuseclass}{cell_input}
\begin{sphinxVerbatim}[commandchars=\\\{\}]
\PYG{k}{def} \PYG{n+nf}{port\PYGZus{}draw\PYGZus{}down\PYGZus{}max}\PYG{p}{(}\PYG{n}{x}\PYG{p}{,} \PYG{n}{r}\PYG{p}{)}\PYG{p}{:}
    \PYG{n}{rp} \PYG{o}{=} \PYG{n}{r}\PYG{o}{.}\PYG{n}{dot}\PYG{p}{(}\PYG{n}{x}\PYG{p}{)}
    \PYG{n}{price} \PYG{o}{=} \PYG{n}{rp}\PYG{o}{.}\PYG{n}{add}\PYG{p}{(}\PYG{l+m+mi}{1}\PYG{p}{)}\PYG{o}{.}\PYG{n}{cumprod}\PYG{p}{(}\PYG{p}{)}
    \PYG{n}{cum\PYGZus{}max} \PYG{o}{=} \PYG{n}{price}\PYG{o}{.}\PYG{n}{cummax}\PYG{p}{(}\PYG{p}{)}
    \PYG{n}{draw\PYGZus{}down} \PYG{o}{=} \PYG{p}{(}\PYG{n}{cum\PYGZus{}max} \PYG{o}{\PYGZhy{}} \PYG{n}{price}\PYG{p}{)} \PYG{o}{/} \PYG{n}{cum\PYGZus{}max}
    \PYG{k}{return} \PYG{n}{draw\PYGZus{}down}\PYG{o}{.}\PYG{n}{max}\PYG{p}{(}\PYG{p}{)}
\end{sphinxVerbatim}

\end{sphinxuseclass}\end{sphinxVerbatimInput}

\end{sphinxuseclass}
\begin{sphinxuseclass}{cell}\begin{sphinxVerbatimInput}

\begin{sphinxuseclass}{cell_input}
\begin{sphinxVerbatim}[commandchars=\\\{\}]
\PYG{n}{res\PYGZus{}dd\PYGZus{}1} \PYG{o}{=} \PYG{n}{sco}\PYG{o}{.}\PYG{n}{minimize}\PYG{p}{(}
    \PYG{n}{fun}\PYG{o}{=}\PYG{n}{port\PYGZus{}draw\PYGZus{}down\PYGZus{}max}\PYG{p}{,}
    \PYG{n}{x0}\PYG{o}{=}\PYG{n}{np}\PYG{o}{.}\PYG{n}{ones}\PYG{p}{(}\PYG{n}{returns}\PYG{o}{.}\PYG{n}{shape}\PYG{p}{[}\PYG{l+m+mi}{1}\PYG{p}{]}\PYG{p}{)} \PYG{o}{/} \PYG{n}{returns}\PYG{o}{.}\PYG{n}{shape}\PYG{p}{[}\PYG{l+m+mi}{1}\PYG{p}{]}\PYG{p}{,}
    \PYG{n}{args}\PYG{o}{=}\PYG{p}{(}\PYG{n}{returns}\PYG{p}{,}\PYG{p}{)}\PYG{p}{,}
    \PYG{n}{bounds}\PYG{o}{=}\PYG{p}{[}\PYG{p}{(}\PYG{l+m+mi}{0}\PYG{p}{,}\PYG{l+m+mi}{1}\PYG{p}{)} \PYG{k}{for} \PYG{n}{\PYGZus{}} \PYG{o+ow}{in} \PYG{n}{returns}\PYG{p}{]}\PYG{p}{,}
    \PYG{n}{constraints}\PYG{o}{=}\PYG{p}{(}
        \PYG{p}{\PYGZob{}}\PYG{l+s+s1}{\PYGZsq{}}\PYG{l+s+s1}{type}\PYG{l+s+s1}{\PYGZsq{}}\PYG{p}{:} \PYG{l+s+s1}{\PYGZsq{}}\PYG{l+s+s1}{eq}\PYG{l+s+s1}{\PYGZsq{}}\PYG{p}{,} \PYG{l+s+s1}{\PYGZsq{}}\PYG{l+s+s1}{fun}\PYG{l+s+s1}{\PYGZsq{}}\PYG{p}{:} \PYG{k}{lambda} \PYG{n}{x}\PYG{p}{:} \PYG{n}{x}\PYG{o}{.}\PYG{n}{sum}\PYG{p}{(}\PYG{p}{)} \PYG{o}{\PYGZhy{}} \PYG{l+m+mi}{1}\PYG{p}{\PYGZcb{}}\PYG{p}{,} \PYG{c+c1}{\PYGZsh{} minimize drives \PYGZdq{}eq\PYGZdq{} constraints to zero}
    \PYG{p}{)}
\PYG{p}{)}

\PYG{n}{res\PYGZus{}dd\PYGZus{}1}
\end{sphinxVerbatim}

\end{sphinxuseclass}\end{sphinxVerbatimInput}
\begin{sphinxVerbatimOutput}

\begin{sphinxuseclass}{cell_output}
\begin{sphinxVerbatim}[commandchars=\\\{\}]
     fun: 0.2973734368777809
     jac: array([0.2865, 0.5131, 0.4028, 0.2927, 0.5359, 0.7698])
 message: \PYGZsq{}Optimization terminated successfully\PYGZsq{}
    nfev: 178
     nit: 20
    njev: 20
  status: 0
 success: True
       x: array([6.8104e\PYGZhy{}01, 9.6518e\PYGZhy{}18, 2.2588e\PYGZhy{}17, 3.1896e\PYGZhy{}01, 1.2995e\PYGZhy{}17,
       3.4603e\PYGZhy{}18])
\end{sphinxVerbatim}

\end{sphinxuseclass}\end{sphinxVerbatimOutput}

\end{sphinxuseclass}
\begin{sphinxuseclass}{cell}\begin{sphinxVerbatimInput}

\begin{sphinxuseclass}{cell_input}
\begin{sphinxVerbatim}[commandchars=\\\{\}]
\PYG{n}{port\PYGZus{}draw\PYGZus{}down\PYGZus{}max}\PYG{p}{(}\PYG{n}{x}\PYG{o}{=}\PYG{n}{res\PYGZus{}dd\PYGZus{}1}\PYG{p}{[}\PYG{l+s+s1}{\PYGZsq{}}\PYG{l+s+s1}{x}\PYG{l+s+s1}{\PYGZsq{}}\PYG{p}{]}\PYG{p}{,} \PYG{n}{r}\PYG{o}{=}\PYG{n}{returns}\PYG{p}{)}
\end{sphinxVerbatim}

\end{sphinxuseclass}\end{sphinxVerbatimInput}
\begin{sphinxVerbatimOutput}

\begin{sphinxuseclass}{cell_output}
\begin{sphinxVerbatim}[commandchars=\\\{\}]
0.2974
\end{sphinxVerbatim}

\end{sphinxuseclass}\end{sphinxVerbatimOutput}

\end{sphinxuseclass}
\sphinxAtStartPar
Again. it might be helpful to slightly change then minimum VaR portfolio weights to show that we minimized VaR.

\begin{sphinxuseclass}{cell}\begin{sphinxVerbatimInput}

\begin{sphinxuseclass}{cell_input}
\begin{sphinxVerbatim}[commandchars=\\\{\}]
\PYG{n}{port\PYGZus{}draw\PYGZus{}down\PYGZus{}max}\PYG{p}{(}\PYG{n}{x}\PYG{o}{=}\PYG{n}{tweak}\PYG{p}{(}\PYG{n}{res\PYGZus{}dd\PYGZus{}1}\PYG{p}{[}\PYG{l+s+s1}{\PYGZsq{}}\PYG{l+s+s1}{x}\PYG{l+s+s1}{\PYGZsq{}}\PYG{p}{]}\PYG{p}{)}\PYG{p}{,} \PYG{n}{r}\PYG{o}{=}\PYG{n}{returns}\PYG{p}{)}
\end{sphinxVerbatim}

\end{sphinxuseclass}\end{sphinxVerbatimInput}
\begin{sphinxVerbatimOutput}

\begin{sphinxuseclass}{cell_output}
\begin{sphinxVerbatim}[commandchars=\\\{\}]
0.3065
\end{sphinxVerbatim}

\end{sphinxuseclass}\end{sphinxVerbatimOutput}

\end{sphinxuseclass}

\subsubsection{Find the minimum maximum draw down portfolio with all available data for the current Dow\sphinxhyphen{}Jones Industrial Average (DJIA) stocks}
\label{\detokenize{herron_04_practice_02:find-the-minimum-maximum-draw-down-portfolio-with-all-available-data-for-the-current-dow-jones-industrial-average-djia-stocks}}
\sphinxAtStartPar
You can find the \sphinxhref{https://en.wikipedia.org/wiki/Dow\_Jones\_Industrial\_Average}{DJIA tickers on Wikipedia}.

\begin{sphinxuseclass}{cell}\begin{sphinxVerbatimInput}

\begin{sphinxuseclass}{cell_input}
\begin{sphinxVerbatim}[commandchars=\\\{\}]
\PYG{n}{wiki} \PYG{o}{=} \PYG{n}{pd}\PYG{o}{.}\PYG{n}{read\PYGZus{}html}\PYG{p}{(}\PYG{l+s+s1}{\PYGZsq{}}\PYG{l+s+s1}{https://en.wikipedia.org/wiki/Dow\PYGZus{}Jones\PYGZus{}Industrial\PYGZus{}Average}\PYG{l+s+s1}{\PYGZsq{}}\PYG{p}{)}
\PYG{n}{tickers} \PYG{o}{=} \PYG{n}{wiki}\PYG{p}{[}\PYG{l+m+mi}{1}\PYG{p}{]}\PYG{p}{[}\PYG{l+s+s1}{\PYGZsq{}}\PYG{l+s+s1}{Symbol}\PYG{l+s+s1}{\PYGZsq{}}\PYG{p}{]}\PYG{o}{.}\PYG{n}{to\PYGZus{}list}\PYG{p}{(}\PYG{p}{)}

\PYG{n}{djia} \PYG{o}{=} \PYG{p}{(}
    \PYG{n}{yf}\PYG{o}{.}\PYG{n}{download}\PYG{p}{(}\PYG{n}{tickers}\PYG{o}{=}\PYG{n}{tickers}\PYG{p}{,} \PYG{n}{progress}\PYG{o}{=}\PYG{k+kc}{False}\PYG{p}{)}
    \PYG{o}{.}\PYG{n}{assign}\PYG{p}{(}\PYG{n}{Date}\PYG{o}{=}\PYG{k}{lambda} \PYG{n}{x}\PYG{p}{:} \PYG{n}{x}\PYG{o}{.}\PYG{n}{index}\PYG{o}{.}\PYG{n}{tz\PYGZus{}localize}\PYG{p}{(}\PYG{k+kc}{None}\PYG{p}{)}\PYG{p}{)}
    \PYG{o}{.}\PYG{n}{set\PYGZus{}index}\PYG{p}{(}\PYG{l+s+s1}{\PYGZsq{}}\PYG{l+s+s1}{Date}\PYG{l+s+s1}{\PYGZsq{}}\PYG{p}{)}
    \PYG{o}{.}\PYG{n}{rename\PYGZus{}axis}\PYG{p}{(}\PYG{n}{columns}\PYG{o}{=}\PYG{p}{[}\PYG{l+s+s1}{\PYGZsq{}}\PYG{l+s+s1}{Variable}\PYG{l+s+s1}{\PYGZsq{}}\PYG{p}{,} \PYG{l+s+s1}{\PYGZsq{}}\PYG{l+s+s1}{Ticker}\PYG{l+s+s1}{\PYGZsq{}}\PYG{p}{]}\PYG{p}{)}
\PYG{p}{)}

\PYG{n}{returns\PYGZus{}2} \PYG{o}{=} \PYG{n}{djia}\PYG{p}{[}\PYG{l+s+s1}{\PYGZsq{}}\PYG{l+s+s1}{Adj Close}\PYG{l+s+s1}{\PYGZsq{}}\PYG{p}{]}\PYG{o}{.}\PYG{n}{pct\PYGZus{}change}\PYG{p}{(}\PYG{p}{)}\PYG{o}{.}\PYG{n}{loc}\PYG{p}{[}\PYG{l+s+s1}{\PYGZsq{}}\PYG{l+s+s1}{2020}\PYG{l+s+s1}{\PYGZsq{}}\PYG{p}{:}\PYG{l+s+s1}{\PYGZsq{}}\PYG{l+s+s1}{2022}\PYG{l+s+s1}{\PYGZsq{}}\PYG{p}{]}
\PYG{n}{returns\PYGZus{}2}\PYG{o}{.}\PYG{n}{describe}\PYG{p}{(}\PYG{p}{)}
\end{sphinxVerbatim}

\end{sphinxuseclass}\end{sphinxVerbatimInput}
\begin{sphinxVerbatimOutput}

\begin{sphinxuseclass}{cell_output}
\begin{sphinxVerbatim}[commandchars=\\\{\}]
Ticker     AAPL     AMGN      AXP       BA      CAT      CRM     CSCO  \PYGZbs{}
count  756.0000 756.0000 756.0000 756.0000 756.0000 756.0000 756.0000   
mean     0.0011   0.0004   0.0007   0.0000   0.0010   0.0001   0.0003   
std      0.0233   0.0176   0.0286   0.0383   0.0224   0.0278   0.0199   
min     \PYGZhy{}0.1286  \PYGZhy{}0.0826  \PYGZhy{}0.1482  \PYGZhy{}0.2385  \PYGZhy{}0.1428  \PYGZhy{}0.1589  \PYGZhy{}0.1373   
25\PYGZpc{}     \PYGZhy{}0.0110  \PYGZhy{}0.0080  \PYGZhy{}0.0130  \PYGZhy{}0.0173  \PYGZhy{}0.0106  \PYGZhy{}0.0137  \PYGZhy{}0.0085   
50\PYGZpc{}      0.0006  \PYGZhy{}0.0003   0.0004  \PYGZhy{}0.0014   0.0010   0.0005   0.0000   
75\PYGZpc{}      0.0142   0.0081   0.0123   0.0152   0.0123   0.0149   0.0093   
max      0.1198   0.1090   0.2188   0.2432   0.1033   0.2604   0.1337   

Ticker      CVX      DIS      DOW  ...      MRK     MSFT      NKE       PG  \PYGZbs{}
count  756.0000 756.0000 756.0000  ... 756.0000 756.0000 756.0000 756.0000   
mean     0.0011  \PYGZhy{}0.0004   0.0005  ...   0.0006   0.0008   0.0005   0.0005   
std      0.0268   0.0241   0.0272  ...   0.0161   0.0219   0.0238   0.0152   
min     \PYGZhy{}0.2212  \PYGZhy{}0.1316  \PYGZhy{}0.2166  ...  \PYGZhy{}0.0986  \PYGZhy{}0.1474  \PYGZhy{}0.1281  \PYGZhy{}0.0874   
25\PYGZpc{}     \PYGZhy{}0.0108  \PYGZhy{}0.0123  \PYGZhy{}0.0124  ...  \PYGZhy{}0.0073  \PYGZhy{}0.0097  \PYGZhy{}0.0104  \PYGZhy{}0.0060   
50\PYGZpc{}      0.0008  \PYGZhy{}0.0012  \PYGZhy{}0.0002  ...   0.0000   0.0007   0.0000   0.0008   
75\PYGZpc{}      0.0133   0.0106   0.0142  ...   0.0084   0.0123   0.0123   0.0075   
max      0.2274   0.1441   0.2091  ...   0.0837   0.1422   0.1553   0.1201   

Ticker      TRV      UNH        V       VZ      WBA      WMT  
count  756.0000 756.0000 756.0000 756.0000 756.0000 756.0000  
mean     0.0007   0.0011   0.0004  \PYGZhy{}0.0003  \PYGZhy{}0.0002   0.0004  
std      0.0210   0.0208   0.0211   0.0132   0.0231   0.0162  
min     \PYGZhy{}0.2080  \PYGZhy{}0.1728  \PYGZhy{}0.1355  \PYGZhy{}0.0674  \PYGZhy{}0.1096  \PYGZhy{}0.1138  
25\PYGZpc{}     \PYGZhy{}0.0087  \PYGZhy{}0.0080  \PYGZhy{}0.0099  \PYGZhy{}0.0061  \PYGZhy{}0.0104  \PYGZhy{}0.0067  
50\PYGZpc{}      0.0017   0.0011   0.0007  \PYGZhy{}0.0002  \PYGZhy{}0.0007   0.0001  
75\PYGZpc{}      0.0101   0.0095   0.0106   0.0054   0.0107   0.0077  
max      0.1329   0.1280   0.1384   0.0721   0.1260   0.1171  

[8 rows x 30 columns]
\end{sphinxVerbatim}

\end{sphinxuseclass}\end{sphinxVerbatimOutput}

\end{sphinxuseclass}
\begin{sphinxuseclass}{cell}\begin{sphinxVerbatimInput}

\begin{sphinxuseclass}{cell_input}
\begin{sphinxVerbatim}[commandchars=\\\{\}]
\PYG{n}{res\PYGZus{}dd\PYGZus{}2} \PYG{o}{=} \PYG{n}{sco}\PYG{o}{.}\PYG{n}{minimize}\PYG{p}{(}
    \PYG{n}{fun}\PYG{o}{=}\PYG{n}{port\PYGZus{}draw\PYGZus{}down\PYGZus{}max}\PYG{p}{,}
    \PYG{n}{x0}\PYG{o}{=}\PYG{n}{np}\PYG{o}{.}\PYG{n}{ones}\PYG{p}{(}\PYG{n}{returns\PYGZus{}2}\PYG{o}{.}\PYG{n}{shape}\PYG{p}{[}\PYG{l+m+mi}{1}\PYG{p}{]}\PYG{p}{)} \PYG{o}{/} \PYG{n}{returns\PYGZus{}2}\PYG{o}{.}\PYG{n}{shape}\PYG{p}{[}\PYG{l+m+mi}{1}\PYG{p}{]}\PYG{p}{,}
    \PYG{n}{args}\PYG{o}{=}\PYG{p}{(}\PYG{n}{returns\PYGZus{}2}\PYG{p}{,}\PYG{p}{)}\PYG{p}{,}
    \PYG{n}{bounds}\PYG{o}{=}\PYG{p}{[}\PYG{p}{(}\PYG{l+m+mi}{0}\PYG{p}{,}\PYG{l+m+mi}{1}\PYG{p}{)} \PYG{k}{for} \PYG{n}{\PYGZus{}} \PYG{o+ow}{in} \PYG{n}{returns\PYGZus{}2}\PYG{p}{]}\PYG{p}{,}
    \PYG{n}{constraints}\PYG{o}{=}\PYG{p}{(}
        \PYG{p}{\PYGZob{}}\PYG{l+s+s1}{\PYGZsq{}}\PYG{l+s+s1}{type}\PYG{l+s+s1}{\PYGZsq{}}\PYG{p}{:} \PYG{l+s+s1}{\PYGZsq{}}\PYG{l+s+s1}{eq}\PYG{l+s+s1}{\PYGZsq{}}\PYG{p}{,} \PYG{l+s+s1}{\PYGZsq{}}\PYG{l+s+s1}{fun}\PYG{l+s+s1}{\PYGZsq{}}\PYG{p}{:} \PYG{k}{lambda} \PYG{n}{x}\PYG{p}{:} \PYG{n}{x}\PYG{o}{.}\PYG{n}{sum}\PYG{p}{(}\PYG{p}{)} \PYG{o}{\PYGZhy{}} \PYG{l+m+mi}{1}\PYG{p}{\PYGZcb{}}\PYG{p}{,} \PYG{c+c1}{\PYGZsh{} minimize drives \PYGZdq{}eq\PYGZdq{} constraints to zero}
    \PYG{p}{)}
\PYG{p}{)}

\PYG{n}{res\PYGZus{}dd\PYGZus{}2}
\end{sphinxVerbatim}

\end{sphinxuseclass}\end{sphinxVerbatimInput}
\begin{sphinxVerbatimOutput}

\begin{sphinxuseclass}{cell_output}
\begin{sphinxVerbatim}[commandchars=\\\{\}]
     fun: 0.14526116917692314
     jac: array([0.1974, 0.064 , 0.2211, 0.2242, 0.1603, 0.0966, 0.162 , 0.1158,
       0.2333, 0.1664, 0.1609, 0.1217, 0.0695, 0.0094, 0.2134, 0.0657,
       0.1275, 0.0835, 0.0717, 0.1171, 0.0113, 0.1175, 0.1914, 0.1776,
       0.0756, 0.1588, 0.1089, 0.0882, 0.1333, 0.2522])
 message: \PYGZsq{}Optimization terminated successfully\PYGZsq{}
    nfev: 879
     nit: 27
    njev: 27
  status: 0
 success: True
       x: array([8.5088e\PYGZhy{}17, 2.6650e\PYGZhy{}02, 0.0000e+00, 1.9963e\PYGZhy{}17, 4.8467e\PYGZhy{}17,
       4.6527e\PYGZhy{}18, 1.4882e\PYGZhy{}17, 5.6870e\PYGZhy{}17, 1.4465e\PYGZhy{}16, 1.0899e\PYGZhy{}16,
       1.6653e\PYGZhy{}18, 2.1579e\PYGZhy{}18, 4.2017e\PYGZhy{}17, 3.0508e\PYGZhy{}17, 7.0291e\PYGZhy{}17,
       1.3723e\PYGZhy{}01, 3.8140e\PYGZhy{}17, 3.8649e\PYGZhy{}17, 2.6616e\PYGZhy{}17, 4.6360e\PYGZhy{}17,
       1.9045e\PYGZhy{}01, 1.3055e\PYGZhy{}01, 7.8858e\PYGZhy{}17, 1.1369e\PYGZhy{}18, 2.2045e\PYGZhy{}18,
       2.0718e\PYGZhy{}17, 1.0429e\PYGZhy{}17, 1.2647e\PYGZhy{}01, 9.0356e\PYGZhy{}17, 3.8864e\PYGZhy{}01])
\end{sphinxVerbatim}

\end{sphinxuseclass}\end{sphinxVerbatimOutput}

\end{sphinxuseclass}
\begin{sphinxuseclass}{cell}\begin{sphinxVerbatimInput}

\begin{sphinxuseclass}{cell_input}
\begin{sphinxVerbatim}[commandchars=\\\{\}]
\PYG{n}{port\PYGZus{}draw\PYGZus{}down\PYGZus{}max}\PYG{p}{(}\PYG{n}{x}\PYG{o}{=}\PYG{n}{res\PYGZus{}dd\PYGZus{}2}\PYG{p}{[}\PYG{l+s+s1}{\PYGZsq{}}\PYG{l+s+s1}{x}\PYG{l+s+s1}{\PYGZsq{}}\PYG{p}{]}\PYG{p}{,} \PYG{n}{r}\PYG{o}{=}\PYG{n}{returns\PYGZus{}2}\PYG{p}{)}
\end{sphinxVerbatim}

\end{sphinxuseclass}\end{sphinxVerbatimInput}
\begin{sphinxVerbatimOutput}

\begin{sphinxuseclass}{cell_output}
\begin{sphinxVerbatim}[commandchars=\\\{\}]
0.1453
\end{sphinxVerbatim}

\end{sphinxuseclass}\end{sphinxVerbatimOutput}

\end{sphinxuseclass}
\sphinxAtStartPar
Again. it might be helpful to slightly change then minimum VaR portfolio weights to show that we minimized VaR.

\begin{sphinxuseclass}{cell}\begin{sphinxVerbatimInput}

\begin{sphinxuseclass}{cell_input}
\begin{sphinxVerbatim}[commandchars=\\\{\}]
\PYG{n}{port\PYGZus{}draw\PYGZus{}down\PYGZus{}max}\PYG{p}{(}\PYG{n}{x}\PYG{o}{=}\PYG{n}{tweak}\PYG{p}{(}\PYG{n}{res\PYGZus{}dd\PYGZus{}2}\PYG{p}{[}\PYG{l+s+s1}{\PYGZsq{}}\PYG{l+s+s1}{x}\PYG{l+s+s1}{\PYGZsq{}}\PYG{p}{]}\PYG{p}{)}\PYG{p}{,} \PYG{n}{r}\PYG{o}{=}\PYG{n}{returns\PYGZus{}2}\PYG{p}{)}
\end{sphinxVerbatim}

\end{sphinxuseclass}\end{sphinxVerbatimInput}
\begin{sphinxVerbatimOutput}

\begin{sphinxuseclass}{cell_output}
\begin{sphinxVerbatim}[commandchars=\\\{\}]
0.1519
\end{sphinxVerbatim}

\end{sphinxuseclass}\end{sphinxVerbatimOutput}

\end{sphinxuseclass}

\subsubsection{Plot the (mean\sphinxhyphen{}variance) efficient frontier with all available data for the current the DJIA stocks}
\label{\detokenize{herron_04_practice_02:plot-the-mean-variance-efficient-frontier-with-all-available-data-for-the-current-the-djia-stocks}}
\sphinxAtStartPar
The range of target returns in \sphinxcode{\sphinxupquote{tret}} span from the minimum to the maximum mean single\sphinxhyphen{}stock returns.

\begin{sphinxuseclass}{cell}\begin{sphinxVerbatimInput}

\begin{sphinxuseclass}{cell_input}
\begin{sphinxVerbatim}[commandchars=\\\{\}]
\PYG{n}{\PYGZus{}} \PYG{o}{=} \PYG{n}{returns\PYGZus{}2}\PYG{o}{.}\PYG{n}{mean}\PYG{p}{(}\PYG{p}{)}\PYG{o}{.}\PYG{n}{mul}\PYG{p}{(}\PYG{l+m+mi}{252}\PYG{p}{)}
\PYG{n}{tret} \PYG{o}{=} \PYG{n}{np}\PYG{o}{.}\PYG{n}{linspace}\PYG{p}{(}\PYG{n}{\PYGZus{}}\PYG{o}{.}\PYG{n}{min}\PYG{p}{(}\PYG{p}{)}\PYG{p}{,} \PYG{n}{\PYGZus{}}\PYG{o}{.}\PYG{n}{max}\PYG{p}{(}\PYG{p}{)}\PYG{p}{,} \PYG{l+m+mi}{25}\PYG{p}{)}
\end{sphinxVerbatim}

\end{sphinxuseclass}\end{sphinxVerbatimInput}

\end{sphinxuseclass}
\sphinxAtStartPar
We will loop over these target returns, finding the minimum variance portfolio for each target return.

\begin{sphinxuseclass}{cell}\begin{sphinxVerbatimInput}

\begin{sphinxuseclass}{cell_input}
\begin{sphinxVerbatim}[commandchars=\\\{\}]
\PYG{k}{def} \PYG{n+nf}{port\PYGZus{}vol}\PYG{p}{(}\PYG{n}{x}\PYG{p}{,} \PYG{n}{r}\PYG{p}{,} \PYG{n}{ppy}\PYG{p}{)}\PYG{p}{:}
    \PYG{k}{return} \PYG{n}{np}\PYG{o}{.}\PYG{n}{sqrt}\PYG{p}{(}\PYG{n}{ppy}\PYG{p}{)} \PYG{o}{*} \PYG{n}{r}\PYG{o}{.}\PYG{n}{dot}\PYG{p}{(}\PYG{n}{x}\PYG{p}{)}\PYG{o}{.}\PYG{n}{std}\PYG{p}{(}\PYG{p}{)}
\end{sphinxVerbatim}

\end{sphinxuseclass}\end{sphinxVerbatimInput}

\end{sphinxuseclass}
\begin{sphinxuseclass}{cell}\begin{sphinxVerbatimInput}

\begin{sphinxuseclass}{cell_input}
\begin{sphinxVerbatim}[commandchars=\\\{\}]
\PYG{k}{def} \PYG{n+nf}{port\PYGZus{}mean}\PYG{p}{(}\PYG{n}{x}\PYG{p}{,} \PYG{n}{r}\PYG{p}{,} \PYG{n}{ppy}\PYG{p}{)}\PYG{p}{:}
    \PYG{k}{return} \PYG{n}{ppy} \PYG{o}{*} \PYG{n}{r}\PYG{o}{.}\PYG{n}{dot}\PYG{p}{(}\PYG{n}{x}\PYG{p}{)}\PYG{o}{.}\PYG{n}{mean}\PYG{p}{(}\PYG{p}{)}
\end{sphinxVerbatim}

\end{sphinxuseclass}\end{sphinxVerbatimInput}

\end{sphinxuseclass}
\begin{sphinxuseclass}{cell}\begin{sphinxVerbatimInput}

\begin{sphinxuseclass}{cell_input}
\begin{sphinxVerbatim}[commandchars=\\\{\}]
\PYG{n}{res\PYGZus{}ef} \PYG{o}{=} \PYG{p}{[}\PYG{p}{]}

\PYG{k}{for} \PYG{n}{t} \PYG{o+ow}{in} \PYG{n}{tret}\PYG{p}{:}
    \PYG{n}{\PYGZus{}} \PYG{o}{=} \PYG{n}{sco}\PYG{o}{.}\PYG{n}{minimize}\PYG{p}{(}
        \PYG{n}{fun}\PYG{o}{=}\PYG{n}{port\PYGZus{}vol}\PYG{p}{,} \PYG{c+c1}{\PYGZsh{} minimize portfolio volatility}
        \PYG{n}{x0}\PYG{o}{=}\PYG{n}{np}\PYG{o}{.}\PYG{n}{ones}\PYG{p}{(}\PYG{n}{returns\PYGZus{}2}\PYG{o}{.}\PYG{n}{shape}\PYG{p}{[}\PYG{l+m+mi}{1}\PYG{p}{]}\PYG{p}{)} \PYG{o}{/} \PYG{n}{returns\PYGZus{}2}\PYG{o}{.}\PYG{n}{shape}\PYG{p}{[}\PYG{l+m+mi}{1}\PYG{p}{]}\PYG{p}{,} \PYG{c+c1}{\PYGZsh{} initial portfolio weights}
        \PYG{n}{args}\PYG{o}{=}\PYG{p}{(}\PYG{n}{returns\PYGZus{}2}\PYG{p}{,} \PYG{l+m+mi}{252}\PYG{p}{)}\PYG{p}{,} \PYG{c+c1}{\PYGZsh{} additional arguments to fun, in order}
        \PYG{n}{bounds}\PYG{o}{=}\PYG{p}{[}\PYG{p}{(}\PYG{l+m+mi}{0}\PYG{p}{,} \PYG{l+m+mi}{1}\PYG{p}{)} \PYG{k}{for} \PYG{n}{c} \PYG{o+ow}{in} \PYG{n}{returns\PYGZus{}2}\PYG{o}{.}\PYG{n}{columns}\PYG{p}{]}\PYG{p}{,} \PYG{c+c1}{\PYGZsh{} bounds limit the search space for each portfolio weight}
        \PYG{n}{constraints}\PYG{o}{=}\PYG{p}{(}
            \PYG{p}{\PYGZob{}}\PYG{l+s+s1}{\PYGZsq{}}\PYG{l+s+s1}{type}\PYG{l+s+s1}{\PYGZsq{}}\PYG{p}{:} \PYG{l+s+s1}{\PYGZsq{}}\PYG{l+s+s1}{eq}\PYG{l+s+s1}{\PYGZsq{}}\PYG{p}{,} \PYG{l+s+s1}{\PYGZsq{}}\PYG{l+s+s1}{fun}\PYG{l+s+s1}{\PYGZsq{}}\PYG{p}{:} \PYG{k}{lambda} \PYG{n}{x}\PYG{p}{:} \PYG{n}{x}\PYG{o}{.}\PYG{n}{sum}\PYG{p}{(}\PYG{p}{)} \PYG{o}{\PYGZhy{}} \PYG{l+m+mi}{1}\PYG{p}{\PYGZcb{}}\PYG{p}{,} \PYG{c+c1}{\PYGZsh{} constrain sum of weights to one}
            \PYG{p}{\PYGZob{}}\PYG{l+s+s1}{\PYGZsq{}}\PYG{l+s+s1}{type}\PYG{l+s+s1}{\PYGZsq{}}\PYG{p}{:} \PYG{l+s+s1}{\PYGZsq{}}\PYG{l+s+s1}{eq}\PYG{l+s+s1}{\PYGZsq{}}\PYG{p}{,} \PYG{l+s+s1}{\PYGZsq{}}\PYG{l+s+s1}{fun}\PYG{l+s+s1}{\PYGZsq{}}\PYG{p}{:} \PYG{k}{lambda} \PYG{n}{x}\PYG{p}{:} \PYG{n}{port\PYGZus{}mean}\PYG{p}{(}\PYG{n}{x}\PYG{o}{=}\PYG{n}{x}\PYG{p}{,} \PYG{n}{r}\PYG{o}{=}\PYG{n}{returns\PYGZus{}2}\PYG{p}{,} \PYG{n}{ppy}\PYG{o}{=}\PYG{l+m+mi}{252}\PYG{p}{)} \PYG{o}{\PYGZhy{}} \PYG{n}{t}\PYG{p}{\PYGZcb{}} \PYG{c+c1}{\PYGZsh{} constrains portfolio mean return to the target return}

        \PYG{p}{)}
    \PYG{p}{)}
    \PYG{n}{res\PYGZus{}ef}\PYG{o}{.}\PYG{n}{append}\PYG{p}{(}\PYG{n}{\PYGZus{}}\PYG{p}{)}
\end{sphinxVerbatim}

\end{sphinxuseclass}\end{sphinxVerbatimInput}

\end{sphinxuseclass}
\sphinxAtStartPar
List \sphinxcode{\sphinxupquote{res\_ef}} contains the results of all 25 minimum\sphinxhyphen{}variance portfolios.
For example, \sphinxcode{\sphinxupquote{res\_ef{[}0{]}}} is the minimum variance portfolio for the lowest target return.

\begin{sphinxuseclass}{cell}\begin{sphinxVerbatimInput}

\begin{sphinxuseclass}{cell_input}
\begin{sphinxVerbatim}[commandchars=\\\{\}]
\PYG{n}{res\PYGZus{}ef}\PYG{p}{[}\PYG{l+m+mi}{0}\PYG{p}{]}
\end{sphinxVerbatim}

\end{sphinxuseclass}\end{sphinxVerbatimInput}
\begin{sphinxVerbatimOutput}

\begin{sphinxuseclass}{cell_output}
\begin{sphinxVerbatim}[commandchars=\\\{\}]
     fun: 0.42060231866698816
     jac: array([0.226 , 0.1289, 0.2273, 0.2753, 0.1603, 0.2102, 0.1879, 0.1846,
       0.1792, 0.1971, 0.2016, 0.1831, 0.1662, 0.1612, 0.4206, 0.0915,
       0.1934, 0.1133, 0.122 , 0.1283, 0.0833, 0.2267, 0.1728, 0.1111,
       0.1504, 0.1597, 0.1808, 0.0838, 0.154 , 0.0955])
 message: \PYGZsq{}Optimization terminated successfully\PYGZsq{}
    nfev: 62
     nit: 2
    njev: 2
  status: 0
 success: True
       x: array([0.0000e+00, 2.3870e\PYGZhy{}15, 4.9336e\PYGZhy{}15, 1.2490e\PYGZhy{}16, 5.4123e\PYGZhy{}16,
       9.8532e\PYGZhy{}16, 0.0000e+00, 0.0000e+00, 1.2915e\PYGZhy{}07, 0.0000e+00,
       0.0000e+00, 2.1858e\PYGZhy{}15, 0.0000e+00, 0.0000e+00, 1.0000e+00,
       0.0000e+00, 0.0000e+00, 1.2282e\PYGZhy{}15, 7.4246e\PYGZhy{}16, 2.4980e\PYGZhy{}15,
       1.0408e\PYGZhy{}15, 0.0000e+00, 2.2204e\PYGZhy{}16, 3.0531e\PYGZhy{}16, 1.2906e\PYGZhy{}15,
       0.0000e+00, 0.0000e+00, 2.4980e\PYGZhy{}16, 0.0000e+00, 0.0000e+00])
\end{sphinxVerbatim}

\end{sphinxuseclass}\end{sphinxVerbatimOutput}

\end{sphinxuseclass}
\sphinxAtStartPar
I typically check that all portfolio volatility minimization succeeds.
If a portfolio volatility minimization fails, we should check our function, bounds, and constraints.

\begin{sphinxuseclass}{cell}\begin{sphinxVerbatimInput}

\begin{sphinxuseclass}{cell_input}
\begin{sphinxVerbatim}[commandchars=\\\{\}]
\PYG{k}{for} \PYG{n}{r} \PYG{o+ow}{in} \PYG{n}{res\PYGZus{}ef}\PYG{p}{:}
    \PYG{k}{assert} \PYG{n}{r}\PYG{p}{[}\PYG{l+s+s1}{\PYGZsq{}}\PYG{l+s+s1}{success}\PYG{l+s+s1}{\PYGZsq{}}\PYG{p}{]} 
\end{sphinxVerbatim}

\end{sphinxuseclass}\end{sphinxVerbatimInput}

\end{sphinxuseclass}
\sphinxAtStartPar
We can combine the target returns and volatilities into a data frame \sphinxcode{\sphinxupquote{ef}}.

\begin{sphinxuseclass}{cell}\begin{sphinxVerbatimInput}

\begin{sphinxuseclass}{cell_input}
\begin{sphinxVerbatim}[commandchars=\\\{\}]
\PYG{n}{ef} \PYG{o}{=} \PYG{n}{pd}\PYG{o}{.}\PYG{n}{DataFrame}\PYG{p}{(}
    \PYG{p}{\PYGZob{}}
        \PYG{l+s+s1}{\PYGZsq{}}\PYG{l+s+s1}{tret}\PYG{l+s+s1}{\PYGZsq{}}\PYG{p}{:} \PYG{n}{tret}\PYG{p}{,}
        \PYG{l+s+s1}{\PYGZsq{}}\PYG{l+s+s1}{tvol}\PYG{l+s+s1}{\PYGZsq{}}\PYG{p}{:} \PYG{n}{np}\PYG{o}{.}\PYG{n}{array}\PYG{p}{(}\PYG{p}{[}\PYG{n}{r}\PYG{p}{[}\PYG{l+s+s1}{\PYGZsq{}}\PYG{l+s+s1}{fun}\PYG{l+s+s1}{\PYGZsq{}}\PYG{p}{]} \PYG{k}{if} \PYG{n}{r}\PYG{p}{[}\PYG{l+s+s1}{\PYGZsq{}}\PYG{l+s+s1}{success}\PYG{l+s+s1}{\PYGZsq{}}\PYG{p}{]} \PYG{k}{else} \PYG{n}{np}\PYG{o}{.}\PYG{n}{nan} \PYG{k}{for} \PYG{n}{r} \PYG{o+ow}{in} \PYG{n}{res\PYGZus{}ef}\PYG{p}{]}\PYG{p}{)}
    \PYG{p}{\PYGZcb{}}
\PYG{p}{)}

\PYG{n}{ef}\PYG{o}{.}\PYG{n}{head}\PYG{p}{(}\PYG{p}{)}
\end{sphinxVerbatim}

\end{sphinxuseclass}\end{sphinxVerbatimInput}
\begin{sphinxVerbatimOutput}

\begin{sphinxuseclass}{cell_output}
\begin{sphinxVerbatim}[commandchars=\\\{\}]
     tret   tvol
0 \PYGZhy{}0.1535 0.4206
1 \PYGZhy{}0.1358 0.3435
2 \PYGZhy{}0.1181 0.2757
3 \PYGZhy{}0.1004 0.2253
4 \PYGZhy{}0.0827 0.2027
\end{sphinxVerbatim}

\end{sphinxuseclass}\end{sphinxVerbatimOutput}

\end{sphinxuseclass}
\begin{sphinxuseclass}{cell}\begin{sphinxVerbatimInput}

\begin{sphinxuseclass}{cell_input}
\begin{sphinxVerbatim}[commandchars=\\\{\}]
\PYG{n}{ef}\PYG{o}{.}\PYG{n}{mul}\PYG{p}{(}\PYG{l+m+mi}{100}\PYG{p}{)}\PYG{o}{.}\PYG{n}{plot}\PYG{p}{(}\PYG{n}{x}\PYG{o}{=}\PYG{l+s+s1}{\PYGZsq{}}\PYG{l+s+s1}{tvol}\PYG{l+s+s1}{\PYGZsq{}}\PYG{p}{,} \PYG{n}{y}\PYG{o}{=}\PYG{l+s+s1}{\PYGZsq{}}\PYG{l+s+s1}{tret}\PYG{l+s+s1}{\PYGZsq{}}\PYG{p}{,} \PYG{n}{legend}\PYG{o}{=}\PYG{k+kc}{False}\PYG{p}{)}
\PYG{n}{plt}\PYG{o}{.}\PYG{n}{ylabel}\PYG{p}{(}\PYG{l+s+s1}{\PYGZsq{}}\PYG{l+s+s1}{Annualized Mean Return (}\PYG{l+s+s1}{\PYGZpc{}}\PYG{l+s+s1}{)}\PYG{l+s+s1}{\PYGZsq{}}\PYG{p}{)}
\PYG{n}{plt}\PYG{o}{.}\PYG{n}{xlabel}\PYG{p}{(}\PYG{l+s+s1}{\PYGZsq{}}\PYG{l+s+s1}{Annualized Volatility (}\PYG{l+s+s1}{\PYGZpc{}}\PYG{l+s+s1}{)}\PYG{l+s+s1}{\PYGZsq{}}\PYG{p}{)}
\PYG{n}{plt}\PYG{o}{.}\PYG{n}{title}\PYG{p}{(}
    \PYG{l+s+sa}{f}\PYG{l+s+s1}{\PYGZsq{}}\PYG{l+s+s1}{Efficient Frontier for Dow\PYGZhy{}Jones Industrial Average Stocks}\PYG{l+s+s1}{\PYGZsq{}} \PYG{o}{+}
    \PYG{l+s+sa}{f}\PYG{l+s+s1}{\PYGZsq{}}\PYG{l+s+se}{\PYGZbs{}n}\PYG{l+s+s1}{from }\PYG{l+s+si}{\PYGZob{}}\PYG{n}{returns\PYGZus{}2}\PYG{o}{.}\PYG{n}{index}\PYG{p}{[}\PYG{l+m+mi}{0}\PYG{p}{]}\PYG{l+s+si}{:}\PYG{l+s+s1}{\PYGZpc{}B \PYGZpc{}d, \PYGZpc{}Y}\PYG{l+s+si}{\PYGZcb{}}\PYG{l+s+s1}{ to }\PYG{l+s+si}{\PYGZob{}}\PYG{n}{returns\PYGZus{}2}\PYG{o}{.}\PYG{n}{index}\PYG{p}{[}\PYG{o}{\PYGZhy{}}\PYG{l+m+mi}{1}\PYG{p}{]}\PYG{l+s+si}{:}\PYG{l+s+s1}{\PYGZpc{}B \PYGZpc{}d, \PYGZpc{}Y}\PYG{l+s+si}{\PYGZcb{}}\PYG{l+s+s1}{\PYGZsq{}}
\PYG{p}{)}

\PYG{k}{for} \PYG{n}{t}\PYG{p}{,} \PYG{n}{x}\PYG{p}{,} \PYG{n}{y} \PYG{o+ow}{in} \PYG{n+nb}{zip}\PYG{p}{(}
    \PYG{n}{returns\PYGZus{}2}\PYG{o}{.}\PYG{n}{columns}\PYG{p}{,} 
    \PYG{n}{returns\PYGZus{}2}\PYG{o}{.}\PYG{n}{std}\PYG{p}{(}\PYG{p}{)}\PYG{o}{.}\PYG{n}{mul}\PYG{p}{(}\PYG{l+m+mi}{100}\PYG{o}{*}\PYG{n}{np}\PYG{o}{.}\PYG{n}{sqrt}\PYG{p}{(}\PYG{l+m+mi}{252}\PYG{p}{)}\PYG{p}{)}\PYG{p}{,}
    \PYG{n}{returns\PYGZus{}2}\PYG{o}{.}\PYG{n}{mean}\PYG{p}{(}\PYG{p}{)}\PYG{o}{.}\PYG{n}{mul}\PYG{p}{(}\PYG{l+m+mi}{100}\PYG{o}{*}\PYG{l+m+mi}{252}\PYG{p}{)}
\PYG{p}{)}\PYG{p}{:}
    \PYG{n}{plt}\PYG{o}{.}\PYG{n}{annotate}\PYG{p}{(}\PYG{n}{text}\PYG{o}{=}\PYG{n}{t}\PYG{p}{,} \PYG{n}{xy}\PYG{o}{=}\PYG{p}{(}\PYG{n}{x}\PYG{p}{,} \PYG{n}{y}\PYG{p}{)}\PYG{p}{)}
    
\PYG{n}{plt}\PYG{o}{.}\PYG{n}{show}\PYG{p}{(}\PYG{p}{)}
\end{sphinxVerbatim}

\end{sphinxuseclass}\end{sphinxVerbatimInput}
\begin{sphinxVerbatimOutput}

\begin{sphinxuseclass}{cell_output}
\noindent\sphinxincludegraphics{{2fa0e7ee8aaa8a5c63c9b4d6ad2f9c62b0e8921eb649922944b307b8f6be9fc0}.png}

\end{sphinxuseclass}\end{sphinxVerbatimOutput}

\end{sphinxuseclass}

\subsubsection{Find the maximum Sharpe Ratio portfolio with all available data for the current the DJIA stocks}
\label{\detokenize{herron_04_practice_02:find-the-maximum-sharpe-ratio-portfolio-with-all-available-data-for-the-current-the-djia-stocks}}
\begin{sphinxuseclass}{cell}\begin{sphinxVerbatimInput}

\begin{sphinxuseclass}{cell_input}
\begin{sphinxVerbatim}[commandchars=\\\{\}]
\PYG{n}{res\PYGZus{}sharpe\PYGZus{}6} \PYG{o}{=} \PYG{n}{sco}\PYG{o}{.}\PYG{n}{minimize}\PYG{p}{(}
    \PYG{n}{fun}\PYG{o}{=}\PYG{n}{port\PYGZus{}sharpe\PYGZus{}neg}\PYG{p}{,}
    \PYG{n}{x0}\PYG{o}{=}\PYG{n}{np}\PYG{o}{.}\PYG{n}{ones}\PYG{p}{(}\PYG{n}{returns\PYGZus{}2}\PYG{o}{.}\PYG{n}{shape}\PYG{p}{[}\PYG{l+m+mi}{1}\PYG{p}{]}\PYG{p}{)} \PYG{o}{/} \PYG{n}{returns\PYGZus{}2}\PYG{o}{.}\PYG{n}{shape}\PYG{p}{[}\PYG{l+m+mi}{1}\PYG{p}{]}\PYG{p}{,}
    \PYG{n}{args}\PYG{o}{=}\PYG{p}{(}\PYG{n}{returns\PYGZus{}2}\PYG{p}{,} \PYG{l+m+mi}{252}\PYG{p}{,} \PYG{l+m+mi}{0}\PYG{p}{)}\PYG{p}{,}
    \PYG{n}{bounds}\PYG{o}{=}\PYG{p}{[}\PYG{p}{(}\PYG{l+m+mi}{0}\PYG{p}{,}\PYG{l+m+mi}{1}\PYG{p}{)} \PYG{k}{for} \PYG{n}{\PYGZus{}} \PYG{o+ow}{in} \PYG{n+nb}{range}\PYG{p}{(}\PYG{n}{returns\PYGZus{}2}\PYG{o}{.}\PYG{n}{shape}\PYG{p}{[}\PYG{l+m+mi}{1}\PYG{p}{]}\PYG{p}{)}\PYG{p}{]}\PYG{p}{,}
    \PYG{n}{constraints}\PYG{o}{=}\PYG{p}{(}
        \PYG{p}{\PYGZob{}}\PYG{l+s+s1}{\PYGZsq{}}\PYG{l+s+s1}{type}\PYG{l+s+s1}{\PYGZsq{}}\PYG{p}{:} \PYG{l+s+s1}{\PYGZsq{}}\PYG{l+s+s1}{eq}\PYG{l+s+s1}{\PYGZsq{}}\PYG{p}{,} \PYG{l+s+s1}{\PYGZsq{}}\PYG{l+s+s1}{fun}\PYG{l+s+s1}{\PYGZsq{}}\PYG{p}{:} \PYG{k}{lambda} \PYG{n}{x}\PYG{p}{:} \PYG{n}{x}\PYG{o}{.}\PYG{n}{sum}\PYG{p}{(}\PYG{p}{)} \PYG{o}{\PYGZhy{}} \PYG{l+m+mi}{1}\PYG{p}{\PYGZcb{}} \PYG{c+c1}{\PYGZsh{} want eq constraint to = 0}
    \PYG{p}{)}
\PYG{p}{)}
\end{sphinxVerbatim}

\end{sphinxuseclass}\end{sphinxVerbatimInput}

\end{sphinxuseclass}
\begin{sphinxuseclass}{cell}\begin{sphinxVerbatimInput}

\begin{sphinxuseclass}{cell_input}
\begin{sphinxVerbatim}[commandchars=\\\{\}]
\PYG{n}{port\PYGZus{}sharpe}\PYG{p}{(}\PYG{n}{x}\PYG{o}{=}\PYG{n}{res\PYGZus{}sharpe\PYGZus{}6}\PYG{p}{[}\PYG{l+s+s1}{\PYGZsq{}}\PYG{l+s+s1}{x}\PYG{l+s+s1}{\PYGZsq{}}\PYG{p}{]}\PYG{p}{,} \PYG{n}{r}\PYG{o}{=}\PYG{n}{returns\PYGZus{}2}\PYG{p}{,} \PYG{n}{ppy}\PYG{o}{=}\PYG{l+m+mi}{252}\PYG{p}{,} \PYG{n}{tgt}\PYG{o}{=}\PYG{l+m+mi}{0}\PYG{p}{)}
\end{sphinxVerbatim}

\end{sphinxuseclass}\end{sphinxVerbatimInput}
\begin{sphinxVerbatimOutput}

\begin{sphinxuseclass}{cell_output}
\begin{sphinxVerbatim}[commandchars=\\\{\}]
0.4419167465321825
\end{sphinxVerbatim}

\end{sphinxuseclass}\end{sphinxVerbatimOutput}

\end{sphinxuseclass}

\subsubsection{Compare the \$\textbackslash{}frac\{1\}\{n\}\$ and maximum Sharpe Ratio portfolios with all available data for the current DJIA stocks}
\label{\detokenize{herron_04_practice_02:compare-the-frac-1-n-and-maximum-sharpe-ratio-portfolios-with-all-available-data-for-the-current-djia-stocks}}
\sphinxAtStartPar
Use all but the last 252 trading days to estimate the maximum Sharpe Ratio portfolio weights.
Then use the last 252 trading days of data to compare the \$\textbackslash{}frac\{1\}\{n\}\$  maximum Sharpe Ratio portfolios.

\begin{sphinxuseclass}{cell}\begin{sphinxVerbatimInput}

\begin{sphinxuseclass}{cell_input}
\begin{sphinxVerbatim}[commandchars=\\\{\}]
\PYG{n}{res\PYGZus{}sharpe\PYGZus{}x} \PYG{o}{=} \PYG{n}{sco}\PYG{o}{.}\PYG{n}{minimize}\PYG{p}{(}
    \PYG{n}{fun}\PYG{o}{=}\PYG{n}{port\PYGZus{}sharpe\PYGZus{}neg}\PYG{p}{,}
    \PYG{n}{x0}\PYG{o}{=}\PYG{n}{get\PYGZus{}ew}\PYG{p}{(}\PYG{n}{returns\PYGZus{}2}\PYG{p}{)}\PYG{p}{,}
    \PYG{n}{args}\PYG{o}{=}\PYG{p}{(}\PYG{n}{returns\PYGZus{}2}\PYG{o}{.}\PYG{n}{loc}\PYG{p}{[}\PYG{l+s+s1}{\PYGZsq{}}\PYG{l+s+s1}{2020}\PYG{l+s+s1}{\PYGZsq{}}\PYG{p}{:}\PYG{l+s+s1}{\PYGZsq{}}\PYG{l+s+s1}{2021}\PYG{l+s+s1}{\PYGZsq{}}\PYG{p}{]}\PYG{p}{,} \PYG{l+m+mi}{0}\PYG{p}{,} \PYG{l+m+mi}{252}\PYG{p}{)}\PYG{p}{,}
    \PYG{n}{bounds}\PYG{o}{=}\PYG{p}{[}\PYG{p}{(}\PYG{l+m+mi}{0}\PYG{p}{,} \PYG{l+m+mi}{1}\PYG{p}{)} \PYG{k}{for} \PYG{n}{i} \PYG{o+ow}{in} \PYG{n}{returns\PYGZus{}2}\PYG{p}{]}\PYG{p}{,}
    \PYG{n}{constraints}\PYG{o}{=}\PYG{p}{(}
        \PYG{p}{\PYGZob{}}\PYG{l+s+s1}{\PYGZsq{}}\PYG{l+s+s1}{type}\PYG{l+s+s1}{\PYGZsq{}}\PYG{p}{:} \PYG{l+s+s1}{\PYGZsq{}}\PYG{l+s+s1}{eq}\PYG{l+s+s1}{\PYGZsq{}}\PYG{p}{,} \PYG{l+s+s1}{\PYGZsq{}}\PYG{l+s+s1}{fun}\PYG{l+s+s1}{\PYGZsq{}}\PYG{p}{:} \PYG{k}{lambda} \PYG{n}{x}\PYG{p}{:} \PYG{n}{x}\PYG{o}{.}\PYG{n}{sum}\PYG{p}{(}\PYG{p}{)} \PYG{o}{\PYGZhy{}} \PYG{l+m+mi}{1}\PYG{p}{\PYGZcb{}} \PYG{c+c1}{\PYGZsh{} eq constraint met when equal to zero}
    \PYG{p}{)}
\PYG{p}{)}

\PYG{n}{res\PYGZus{}sharpe\PYGZus{}x}
\end{sphinxVerbatim}

\end{sphinxuseclass}\end{sphinxVerbatimInput}
\begin{sphinxVerbatimOutput}

\begin{sphinxuseclass}{cell_output}
\begin{sphinxVerbatim}[commandchars=\\\{\}]
     fun: \PYGZhy{}1.451019869956015
     jac: array([\PYGZhy{}3.3244e\PYGZhy{}05,  7.1544e\PYGZhy{}01,  3.6336e\PYGZhy{}01,  1.5581e+00, \PYGZhy{}5.4896e\PYGZhy{}05,
        2.6187e\PYGZhy{}01,  3.1877e\PYGZhy{}01,  6.4522e\PYGZhy{}01,  6.0754e\PYGZhy{}01,  5.1480e\PYGZhy{}01,
        2.2540e\PYGZhy{}02, \PYGZhy{}2.8297e\PYGZhy{}05,  4.4621e\PYGZhy{}01,  4.4353e\PYGZhy{}01,  1.1298e+00,
        2.2024e\PYGZhy{}01,  4.9128e\PYGZhy{}01,  3.9944e\PYGZhy{}01,  2.4492e\PYGZhy{}01,  4.7485e\PYGZhy{}01,
        6.0061e\PYGZhy{}01,  3.6570e\PYGZhy{}04, \PYGZhy{}1.0264e\PYGZhy{}03,  1.1015e\PYGZhy{}01,  4.1630e\PYGZhy{}01,
        1.2804e\PYGZhy{}02,  6.8370e\PYGZhy{}01,  5.0606e\PYGZhy{}01,  5.0418e\PYGZhy{}01,  1.4876e\PYGZhy{}01])
 message: \PYGZsq{}Optimization terminated successfully\PYGZsq{}
    nfev: 218
     nit: 7
    njev: 7
  status: 0
 success: True
       x: array([5.1283e\PYGZhy{}01, 7.3929e\PYGZhy{}17, 0.0000e+00, 0.0000e+00, 1.0011e\PYGZhy{}02,
       6.1795e\PYGZhy{}16, 1.2347e\PYGZhy{}15, 0.0000e+00, 0.0000e+00, 4.7353e\PYGZhy{}16,
       3.4684e\PYGZhy{}16, 2.5709e\PYGZhy{}01, 0.0000e+00, 4.1536e\PYGZhy{}16, 0.0000e+00,
       5.3411e\PYGZhy{}17, 2.8629e\PYGZhy{}16, 6.5371e\PYGZhy{}17, 8.9074e\PYGZhy{}17, 2.4575e\PYGZhy{}16,
       8.2074e\PYGZhy{}17, 1.7922e\PYGZhy{}01, 4.0854e\PYGZhy{}02, 3.3561e\PYGZhy{}16, 2.8254e\PYGZhy{}16,
       3.0830e\PYGZhy{}16, 0.0000e+00, 0.0000e+00, 0.0000e+00, 0.0000e+00])
\end{sphinxVerbatim}

\end{sphinxuseclass}\end{sphinxVerbatimOutput}

\end{sphinxuseclass}
\begin{sphinxuseclass}{cell}\begin{sphinxVerbatimInput}

\begin{sphinxuseclass}{cell_input}
\begin{sphinxVerbatim}[commandchars=\\\{\}]
\PYG{n}{plt}\PYG{o}{.}\PYG{n}{barh}\PYG{p}{(}
    \PYG{n}{y}\PYG{o}{=}\PYG{n}{returns\PYGZus{}2}\PYG{o}{.}\PYG{n}{columns}\PYG{p}{,}
    \PYG{n}{width}\PYG{o}{=}\PYG{n}{res\PYGZus{}sharpe\PYGZus{}x}\PYG{p}{[}\PYG{l+s+s1}{\PYGZsq{}}\PYG{l+s+s1}{x}\PYG{l+s+s1}{\PYGZsq{}}\PYG{p}{]}\PYG{p}{,}
    \PYG{n}{label}\PYG{o}{=}\PYG{l+s+s1}{\PYGZsq{}}\PYG{l+s+s1}{Maximum Sharpe Ratio}\PYG{l+s+s1}{\PYGZsq{}}
\PYG{p}{)}
\PYG{n}{plt}\PYG{o}{.}\PYG{n}{axvline}\PYG{p}{(}\PYG{l+m+mi}{1}\PYG{o}{/}\PYG{l+m+mi}{30}\PYG{p}{,} \PYG{n}{color}\PYG{o}{=}\PYG{l+s+s1}{\PYGZsq{}}\PYG{l+s+s1}{red}\PYG{l+s+s1}{\PYGZsq{}}\PYG{p}{,} \PYG{n}{label}\PYG{o}{=}\PYG{l+s+s1}{\PYGZsq{}}\PYG{l+s+s1}{Equal Weight}\PYG{l+s+s1}{\PYGZsq{}}\PYG{p}{)}
\PYG{n}{plt}\PYG{o}{.}\PYG{n}{legend}\PYG{p}{(}\PYG{p}{)}
\PYG{n}{plt}\PYG{o}{.}\PYG{n}{xlabel}\PYG{p}{(}\PYG{l+s+s1}{\PYGZsq{}}\PYG{l+s+s1}{Portfolio Weight}\PYG{l+s+s1}{\PYGZsq{}}\PYG{p}{)}
\PYG{n}{plt}\PYG{o}{.}\PYG{n}{title}\PYG{p}{(}
    \PYG{l+s+s1}{\PYGZsq{}}\PYG{l+s+s1}{Portfolio Weights for Dow\PYGZhy{}Jones Industrial Average Stocks}\PYG{l+s+s1}{\PYGZsq{}} \PYG{o}{+}
    \PYG{l+s+sa}{f}\PYG{l+s+s1}{\PYGZsq{}}\PYG{l+s+se}{\PYGZbs{}n}\PYG{l+s+s1}{from }\PYG{l+s+si}{\PYGZob{}}\PYG{n}{returns\PYGZus{}2}\PYG{o}{.}\PYG{n}{index}\PYG{p}{[}\PYG{l+m+mi}{0}\PYG{p}{]}\PYG{l+s+si}{:}\PYG{l+s+s1}{\PYGZpc{}b \PYGZpc{}d, \PYGZpc{}Y}\PYG{l+s+si}{\PYGZcb{}}\PYG{l+s+s1}{ to }\PYG{l+s+si}{\PYGZob{}}\PYG{n}{returns\PYGZus{}2}\PYG{o}{.}\PYG{n}{index}\PYG{p}{[}\PYG{o}{\PYGZhy{}}\PYG{l+m+mi}{1}\PYG{p}{]}\PYG{l+s+si}{:}\PYG{l+s+s1}{\PYGZpc{}b \PYGZpc{}d, \PYGZpc{}Y}\PYG{l+s+si}{\PYGZcb{}}\PYG{l+s+s1}{\PYGZsq{}}
\PYG{p}{)}
\PYG{n}{plt}\PYG{o}{.}\PYG{n}{show}\PYG{p}{(}\PYG{p}{)}
\end{sphinxVerbatim}

\end{sphinxuseclass}\end{sphinxVerbatimInput}
\begin{sphinxVerbatimOutput}

\begin{sphinxuseclass}{cell_output}
\noindent\sphinxincludegraphics{{ed3faa8fc73c2e844f027f25089fe052145d884b806b964b4e67bef53f91593e}.png}

\end{sphinxuseclass}\end{sphinxVerbatimOutput}

\end{sphinxuseclass}
\begin{sphinxuseclass}{cell}\begin{sphinxVerbatimInput}

\begin{sphinxuseclass}{cell_input}
\begin{sphinxVerbatim}[commandchars=\\\{\}]
\PYG{n}{port\PYGZus{}sharpe}\PYG{p}{(}\PYG{n}{res\PYGZus{}sharpe\PYGZus{}x}\PYG{p}{[}\PYG{l+s+s1}{\PYGZsq{}}\PYG{l+s+s1}{x}\PYG{l+s+s1}{\PYGZsq{}}\PYG{p}{]}\PYG{p}{,} \PYG{n}{returns\PYGZus{}2}\PYG{o}{.}\PYG{n}{loc}\PYG{p}{[}\PYG{l+s+s1}{\PYGZsq{}}\PYG{l+s+s1}{2022}\PYG{l+s+s1}{\PYGZsq{}}\PYG{p}{]}\PYG{p}{,} \PYG{l+m+mi}{0}\PYG{p}{,} \PYG{l+m+mi}{252}\PYG{p}{)}
\end{sphinxVerbatim}

\end{sphinxuseclass}\end{sphinxVerbatimInput}
\begin{sphinxVerbatimOutput}

\begin{sphinxuseclass}{cell_output}
\begin{sphinxVerbatim}[commandchars=\\\{\}]
\PYGZhy{}0.7530020412398378
\end{sphinxVerbatim}

\end{sphinxuseclass}\end{sphinxVerbatimOutput}

\end{sphinxuseclass}
\begin{sphinxuseclass}{cell}\begin{sphinxVerbatimInput}

\begin{sphinxuseclass}{cell_input}
\begin{sphinxVerbatim}[commandchars=\\\{\}]
\PYG{n}{port\PYGZus{}sharpe}\PYG{p}{(}\PYG{n}{get\PYGZus{}ew}\PYG{p}{(}\PYG{n}{returns\PYGZus{}2}\PYG{p}{)}\PYG{p}{,} \PYG{n}{returns\PYGZus{}2}\PYG{o}{.}\PYG{n}{loc}\PYG{p}{[}\PYG{l+s+s1}{\PYGZsq{}}\PYG{l+s+s1}{2022}\PYG{l+s+s1}{\PYGZsq{}}\PYG{p}{]}\PYG{p}{,} \PYG{l+m+mi}{0}\PYG{p}{,} \PYG{l+m+mi}{252}\PYG{p}{)}
\end{sphinxVerbatim}

\end{sphinxuseclass}\end{sphinxVerbatimInput}
\begin{sphinxVerbatimOutput}

\begin{sphinxuseclass}{cell_output}
\begin{sphinxVerbatim}[commandchars=\\\{\}]
\PYGZhy{}0.25436549146826076
\end{sphinxVerbatim}

\end{sphinxuseclass}\end{sphinxVerbatimOutput}

\end{sphinxuseclass}
\sphinxAtStartPar
It is hard to beat the \$\textbackslash{}frac\{1\}\{n\}\$ portfolio because mean returns (and covariances) are hard to predict!


\bigskip\hrule\bigskip


\sphinxAtStartPar
Side discussion on the \sphinxcode{\sphinxupquote{.dot()}} method.

\begin{sphinxuseclass}{cell}\begin{sphinxVerbatimInput}

\begin{sphinxuseclass}{cell_input}
\begin{sphinxVerbatim}[commandchars=\\\{\}]
\PYG{n}{weights} \PYG{o}{=} \PYG{n}{get\PYGZus{}ew}\PYG{p}{(}\PYG{n}{returns}\PYG{p}{)}
\end{sphinxVerbatim}

\end{sphinxuseclass}\end{sphinxVerbatimInput}

\end{sphinxuseclass}
\begin{sphinxuseclass}{cell}\begin{sphinxVerbatimInput}

\begin{sphinxuseclass}{cell_input}
\begin{sphinxVerbatim}[commandchars=\\\{\}]
\PYG{n}{np}\PYG{o}{.}\PYG{n}{allclose}\PYG{p}{(}
    \PYG{n}{weights}\PYG{o}{.}\PYG{n}{dot}\PYG{p}{(}\PYG{n}{returns}\PYG{o}{.}\PYG{n}{transpose}\PYG{p}{(}\PYG{p}{)}\PYG{p}{)}\PYG{p}{,}
    \PYG{n}{returns}\PYG{o}{.}\PYG{n}{dot}\PYG{p}{(}\PYG{n}{weights}\PYG{p}{)}
\PYG{p}{)}
\end{sphinxVerbatim}

\end{sphinxuseclass}\end{sphinxVerbatimInput}
\begin{sphinxVerbatimOutput}

\begin{sphinxuseclass}{cell_output}
\begin{sphinxVerbatim}[commandchars=\\\{\}]
True
\end{sphinxVerbatim}

\end{sphinxuseclass}\end{sphinxVerbatimOutput}

\end{sphinxuseclass}
\begin{sphinxuseclass}{cell}\begin{sphinxVerbatimInput}

\begin{sphinxuseclass}{cell_input}
\begin{sphinxVerbatim}[commandchars=\\\{\}]
\PYG{n}{np}\PYG{o}{.}\PYG{n}{allclose}\PYG{p}{(}
    \PYG{n}{returns} \PYG{o}{@} \PYG{n}{weights}\PYG{p}{,}
    \PYG{n}{returns}\PYG{o}{.}\PYG{n}{dot}\PYG{p}{(}\PYG{n}{weights}\PYG{p}{)}
\PYG{p}{)}
\end{sphinxVerbatim}

\end{sphinxuseclass}\end{sphinxVerbatimInput}
\begin{sphinxVerbatimOutput}

\begin{sphinxuseclass}{cell_output}
\begin{sphinxVerbatim}[commandchars=\\\{\}]
True
\end{sphinxVerbatim}

\end{sphinxuseclass}\end{sphinxVerbatimOutput}

\end{sphinxuseclass}
\begin{sphinxuseclass}{cell}\begin{sphinxVerbatimInput}

\begin{sphinxuseclass}{cell_input}
\begin{sphinxVerbatim}[commandchars=\\\{\}]
\PYG{n}{np}\PYG{o}{.}\PYG{n}{allclose}\PYG{p}{(}
    \PYG{n}{weights} \PYG{o}{@} \PYG{n}{returns}\PYG{o}{.}\PYG{n}{transpose}\PYG{p}{(}\PYG{p}{)}\PYG{p}{,}
    \PYG{n}{returns}\PYG{o}{.}\PYG{n}{dot}\PYG{p}{(}\PYG{n}{weights}\PYG{p}{)}
\PYG{p}{)}
\end{sphinxVerbatim}

\end{sphinxuseclass}\end{sphinxVerbatimInput}
\begin{sphinxVerbatimOutput}

\begin{sphinxuseclass}{cell_output}
\begin{sphinxVerbatim}[commandchars=\\\{\}]
True
\end{sphinxVerbatim}

\end{sphinxuseclass}\end{sphinxVerbatimOutput}

\end{sphinxuseclass}
\sphinxstepscope


\chapter{Herron Topic 5 \sphinxhyphen{} Simulations}
\label{\detokenize{herron_05_lecture:herron-topic-5-simulations}}\label{\detokenize{herron_05_lecture::doc}}
\sphinxAtStartPar
In finance, we typically perform simulations with \sphinxhref{https://en.wikipedia.org/wiki/Monte\_Carlo\_method}{Monte Carlo methods}.
Monte Carlo methods are used throughout science, engineering, and mathematics, and they are especially useful in finance due to the randomness of asset prices.
This lecture notebook provides an introduction to \sphinxhref{https://en.wikipedia.org/wiki/Monte\_Carlo\_methods\_in\_finance}{Monte Carlo methods in finance}.

\sphinxAtStartPar
The basic idea behind Monte Carlo methods in finance is to create many possible paths of financial variables (e.g., stock prices, interest rates, exchange rates) based on their historical behavior, statistical properties, and any other relevant factors.
We use these paths to simulate the performance of financial instruments or portfolios, and then we aggregate these results to estimate metrics of interest.
This approach helps us model and analyze complex financial problems that may be difficult or impossible to solve analytically.

\sphinxAtStartPar
We could spend a semester (or more) on simulations and Monte Carlo methods.
In this lecture notebook, we will limit our focus to:
\begin{enumerate}
\sphinxsetlistlabels{\arabic}{enumi}{enumii}{}{.}%
\item {} 
\sphinxAtStartPar
Option pricing: Monte Carlo methods can be used to estimate the fair value of financial derivatives, such as options and warrants. By simulating the potential future paths of the underlying asset and calculating the payoffs of the derivative at each path, the expected payoff can be computed and discounted to present value to determine the option’s price.

\item {} 
\sphinxAtStartPar
Value at Risk (VaR): Monte Carlo simulations can be used to compute VaR, a widely used risk management metric that estimates the potential loss in the value of a portfolio over a specific time horizon, given a certain level of confidence (e.g., 95\% or 99\%). By simulating numerous scenarios and observing the distribution of potential losses, the VaR can be calculated as the threshold below which a certain percentage of losses fall.

\end{enumerate}

\begin{sphinxuseclass}{cell}\begin{sphinxVerbatimInput}

\begin{sphinxuseclass}{cell_input}
\begin{sphinxVerbatim}[commandchars=\\\{\}]
\PYG{k+kn}{import} \PYG{n+nn}{matplotlib}\PYG{n+nn}{.}\PYG{n+nn}{pyplot} \PYG{k}{as} \PYG{n+nn}{plt}
\PYG{k+kn}{import} \PYG{n+nn}{numpy} \PYG{k}{as} \PYG{n+nn}{np}
\PYG{k+kn}{import} \PYG{n+nn}{pandas} \PYG{k}{as} \PYG{n+nn}{pd}
\end{sphinxVerbatim}

\end{sphinxuseclass}\end{sphinxVerbatimInput}

\end{sphinxuseclass}
\begin{sphinxuseclass}{cell}\begin{sphinxVerbatimInput}

\begin{sphinxuseclass}{cell_input}
\begin{sphinxVerbatim}[commandchars=\\\{\}]
\PYG{o}{\PYGZpc{}}\PYG{k}{config} InlineBackend.figure\PYGZus{}format = \PYGZsq{}retina\PYGZsq{}
\PYG{o}{\PYGZpc{}}\PYG{k}{precision} 2
\PYG{n}{pd}\PYG{o}{.}\PYG{n}{options}\PYG{o}{.}\PYG{n}{display}\PYG{o}{.}\PYG{n}{float\PYGZus{}format} \PYG{o}{=} \PYG{l+s+s1}{\PYGZsq{}}\PYG{l+s+si}{\PYGZob{}:.2f\PYGZcb{}}\PYG{l+s+s1}{\PYGZsq{}}\PYG{o}{.}\PYG{n}{format}
\end{sphinxVerbatim}

\end{sphinxuseclass}\end{sphinxVerbatimInput}

\end{sphinxuseclass}
\begin{sphinxuseclass}{cell}\begin{sphinxVerbatimInput}

\begin{sphinxuseclass}{cell_input}
\begin{sphinxVerbatim}[commandchars=\\\{\}]
\PYG{k+kn}{import} \PYG{n+nn}{yfinance} \PYG{k}{as} \PYG{n+nn}{yf}
\PYG{k+kn}{import} \PYG{n+nn}{pandas\PYGZus{}datareader} \PYG{k}{as} \PYG{n+nn}{pdr}
\PYG{k+kn}{import} \PYG{n+nn}{requests\PYGZus{}cache}
\PYG{n}{session} \PYG{o}{=} \PYG{n}{requests\PYGZus{}cache}\PYG{o}{.}\PYG{n}{CachedSession}\PYG{p}{(}\PYG{p}{)}
\end{sphinxVerbatim}

\end{sphinxuseclass}\end{sphinxVerbatimInput}

\end{sphinxuseclass}

\section{Option Pricing}
\label{\detokenize{herron_05_lecture:option-pricing}}
\sphinxAtStartPar
We can use Monte Carlo methods to value stock options.
First, we simulate several hundred or thousand possible (but random) price paths for the underlying stock.
Then, we calculate the payoff for each path.
On some paths, the option will expire “in the money” with \$S\_T > K\$ and pay \$S\_T \sphinxhyphen{} K\$.
On other paths, the option will expire “out of the money” with \$S\_T < K\$ and pay 0.
We average these payoffs and discount them to today.
The present value of these payoffs is the option price.
This is an illustrative example, and there is a lot of depth to \sphinxhref{https://en.wikipedia.org/wiki/Monte\_Carlo\_methods\_for\_option\_pricing}{Monte Carlo methods for option pricing}.


\subsection{Simulating Stock Prices}
\label{\detokenize{herron_05_lecture:simulating-stock-prices}}
\sphinxAtStartPar
We can simulate stock prices with the following stochastic differential equation (SDE) for Geometric Brownian Motion (GBM): \$dS = \textbackslash{}mu S dt + \textbackslash{}sigma S dW\_t\$.
GBM does not account for mean\sphinxhyphen{}reversion and time\sphinxhyphen{}dependent volatility.
So GBM is often used for stocks and not for bond prices, which tend to display long\sphinxhyphen{}term reversion to the face value.
In the GBM SDE:
\begin{enumerate}
\sphinxsetlistlabels{\arabic}{enumi}{enumii}{}{.}%
\item {} 
\sphinxAtStartPar
\$S\$ is the stock price

\item {} 
\sphinxAtStartPar
\$\textbackslash{}mu\$ is the drift coefficient (i.e., instantaneous expected return)

\item {} 
\sphinxAtStartPar
\$\textbackslash{}sigma\$ is the diffusion coefficient (i.e., volatility of the drift)

\item {} 
\sphinxAtStartPar
\$W\_t\$ is the Wiener Process or Brownian Motion

\end{enumerate}

\sphinxAtStartPar
The GBM SDE has a closed\sphinxhyphen{}form solution: \$S(t) = S\_0 \textbackslash{}exp\textbackslash{}left(\textbackslash{}left(\textbackslash{}mu \sphinxhyphen{} \textbackslash{}frac\{1\}\{2\}\textbackslash{}sigma\textasciicircum{}2\textbackslash{}right)t + \textbackslash{}sigma W\_t\textbackslash{}right)\$.
We can apply this closed form solution recursively: \$S(t\_\{i+1\}) = S(t\_i) \textbackslash{}exp\textbackslash{}left(\textbackslash{}left(\textbackslash{}mu \sphinxhyphen{} \textbackslash{}frac\{1\}\{2\}\textbackslash{}sigma\textasciicircum{}2\textbackslash{}right)(t\_\{i+1\} \sphinxhyphen{} t\_i) + \textbackslash{}sigma \textbackslash{}sqrt\{t\_\{i+1\} \sphinxhyphen{} t\_i\} Z\_\{i+1\}\textbackslash{}right)\$.
Here, \$Z\_i\$ is a Standard Normal random variable (\$dW\_t\$ are independent and normally distributed) and \$i = 0, \textbackslash{}ldots, T\sphinxhyphen{}1\$ is the time index.

\sphinxAtStartPar
We can use this closed form solution to simulate stock prices for AAPL.

\begin{sphinxuseclass}{cell}\begin{sphinxVerbatimInput}

\begin{sphinxuseclass}{cell_input}
\begin{sphinxVerbatim}[commandchars=\\\{\}]
\PYG{n}{aapl} \PYG{o}{=} \PYG{p}{(}
    \PYG{n}{yf}\PYG{o}{.}\PYG{n}{download}\PYG{p}{(}\PYG{n}{tickers}\PYG{o}{=}\PYG{l+s+s1}{\PYGZsq{}}\PYG{l+s+s1}{AAPL}\PYG{l+s+s1}{\PYGZsq{}}\PYG{p}{,} \PYG{n}{progress}\PYG{o}{=}\PYG{k+kc}{False}\PYG{p}{)}
    \PYG{o}{.}\PYG{n}{assign}\PYG{p}{(}
        \PYG{n}{Date}\PYG{o}{=}\PYG{k}{lambda} \PYG{n}{x}\PYG{p}{:} \PYG{n}{x}\PYG{o}{.}\PYG{n}{index}\PYG{o}{.}\PYG{n}{tz\PYGZus{}localize}\PYG{p}{(}\PYG{k+kc}{None}\PYG{p}{)}\PYG{p}{,}
        \PYG{n}{Return}\PYG{o}{=}\PYG{k}{lambda} \PYG{n}{x}\PYG{p}{:} \PYG{n}{x}\PYG{p}{[}\PYG{l+s+s1}{\PYGZsq{}}\PYG{l+s+s1}{Adj Close}\PYG{l+s+s1}{\PYGZsq{}}\PYG{p}{]}\PYG{o}{.}\PYG{n}{pct\PYGZus{}change}\PYG{p}{(}\PYG{p}{)}
    \PYG{p}{)}
    \PYG{o}{.}\PYG{n}{set\PYGZus{}index}\PYG{p}{(}\PYG{l+s+s1}{\PYGZsq{}}\PYG{l+s+s1}{Date}\PYG{l+s+s1}{\PYGZsq{}}\PYG{p}{)}
    \PYG{o}{.}\PYG{n}{rename\PYGZus{}axis}\PYG{p}{(}\PYG{n}{columns}\PYG{o}{=}\PYG{p}{[}\PYG{l+s+s1}{\PYGZsq{}}\PYG{l+s+s1}{Variable}\PYG{l+s+s1}{\PYGZsq{}}\PYG{p}{]}\PYG{p}{)}
\PYG{p}{)}
\end{sphinxVerbatim}

\end{sphinxuseclass}\end{sphinxVerbatimInput}

\end{sphinxuseclass}
\begin{sphinxuseclass}{cell}\begin{sphinxVerbatimInput}

\begin{sphinxuseclass}{cell_input}
\begin{sphinxVerbatim}[commandchars=\\\{\}]
\PYG{n}{aapl}\PYG{o}{.}\PYG{n}{describe}\PYG{p}{(}\PYG{p}{)}
\end{sphinxVerbatim}

\end{sphinxuseclass}\end{sphinxVerbatimInput}
\begin{sphinxVerbatimOutput}

\begin{sphinxuseclass}{cell_output}
\begin{sphinxVerbatim}[commandchars=\\\{\}]
Variable     Open     High      Low    Close  Adj Close        Volume   Return
count    10674.00 10674.00 10674.00 10674.00   10674.00      10674.00 10673.00
mean        17.35    17.55    17.16    17.36      16.67  325931256.70     0.00
std         36.64    37.09    36.22    36.68      36.33  337395659.86     0.03
min          0.05     0.05     0.05     0.05       0.04          0.00    \PYGZhy{}0.52
25\PYGZpc{}          0.29     0.30     0.28     0.29       0.24  119995300.00    \PYGZhy{}0.01
50\PYGZpc{}          0.49     0.50     0.48     0.49       0.41  213094000.00     0.00
75\PYGZpc{}         16.90    17.10    16.72    16.94      14.73  405541500.00     0.01
max        182.63   182.94   179.12   182.01     180.68 7421640800.00     0.33
\end{sphinxVerbatim}

\end{sphinxuseclass}\end{sphinxVerbatimOutput}

\end{sphinxuseclass}
\sphinxAtStartPar
We will use returns from 2021 to predict prices in 2022.

\begin{sphinxuseclass}{cell}\begin{sphinxVerbatimInput}

\begin{sphinxuseclass}{cell_input}
\begin{sphinxVerbatim}[commandchars=\\\{\}]
\PYG{n}{train} \PYG{o}{=} \PYG{n}{aapl}\PYG{o}{.}\PYG{n}{loc}\PYG{p}{[}\PYG{l+s+s1}{\PYGZsq{}}\PYG{l+s+s1}{2021}\PYG{l+s+s1}{\PYGZsq{}}\PYG{p}{]}
\PYG{n}{test} \PYG{o}{=} \PYG{n}{aapl}\PYG{o}{.}\PYG{n}{loc}\PYG{p}{[}\PYG{l+s+s1}{\PYGZsq{}}\PYG{l+s+s1}{2022}\PYG{l+s+s1}{\PYGZsq{}}\PYG{p}{]}
\end{sphinxVerbatim}

\end{sphinxuseclass}\end{sphinxVerbatimInput}

\end{sphinxuseclass}
\sphinxAtStartPar
We will use the following function to simulate price paths.
Throughout this lecture notebook, we will use one\sphinxhyphen{}trading\sphinxhyphen{}day steps (i.e., \sphinxcode{\sphinxupquote{dt=1}}).

\begin{sphinxuseclass}{cell}\begin{sphinxVerbatimInput}

\begin{sphinxuseclass}{cell_input}
\begin{sphinxVerbatim}[commandchars=\\\{\}]
\PYG{k}{def} \PYG{n+nf}{simulate\PYGZus{}gbm}\PYG{p}{(}\PYG{n}{S\PYGZus{}0}\PYG{p}{,} \PYG{n}{mu}\PYG{p}{,} \PYG{n}{sigma}\PYG{p}{,} \PYG{n}{n\PYGZus{}steps}\PYG{p}{,} \PYG{n}{dt}\PYG{o}{=}\PYG{l+m+mi}{1}\PYG{p}{,} \PYG{n}{seed}\PYG{o}{=}\PYG{l+m+mi}{42}\PYG{p}{)}\PYG{p}{:}
    \PYG{l+s+sd}{\PYGZsq{}\PYGZsq{}\PYGZsq{}}
\PYG{l+s+sd}{    Function to simulate stock prices following Geometric Brownian Motion (GBM).}
\PYG{l+s+sd}{    }
\PYG{l+s+sd}{    Parameters}
\PYG{l+s+sd}{    \PYGZhy{}\PYGZhy{}\PYGZhy{}\PYGZhy{}\PYGZhy{}\PYGZhy{}\PYGZhy{}\PYGZhy{}\PYGZhy{}\PYGZhy{}\PYGZhy{}\PYGZhy{}}
\PYG{l+s+sd}{    S\PYGZus{}0 : float}
\PYG{l+s+sd}{        Initial stock price}
\PYG{l+s+sd}{    mu : float}
\PYG{l+s+sd}{        Drift coefficient}
\PYG{l+s+sd}{    sigma : float}
\PYG{l+s+sd}{        Diffusion coefficient}
\PYG{l+s+sd}{    n\PYGZus{}steps : int}
\PYG{l+s+sd}{        Length of the forecast horizon in time increments, so T = n\PYGZus{}steps * dt}
\PYG{l+s+sd}{    dt : int}
\PYG{l+s+sd}{        Time increment, typically one day}
\PYG{l+s+sd}{    seed : int}
\PYG{l+s+sd}{        Random seed for reproducibility}

\PYG{l+s+sd}{    Returns}
\PYG{l+s+sd}{    \PYGZhy{}\PYGZhy{}\PYGZhy{}\PYGZhy{}\PYGZhy{}\PYGZhy{}\PYGZhy{}\PYGZhy{}\PYGZhy{}\PYGZhy{}\PYGZhy{}}
\PYG{l+s+sd}{    S\PYGZus{}t : np.ndarray}
\PYG{l+s+sd}{        Array (length: n\PYGZus{}steps + 1) of simulated prices}
\PYG{l+s+sd}{    \PYGZsq{}\PYGZsq{}\PYGZsq{}}

    \PYG{n}{np}\PYG{o}{.}\PYG{n}{random}\PYG{o}{.}\PYG{n}{seed}\PYG{p}{(}\PYG{n}{seed}\PYG{p}{)}
    \PYG{n}{dW} \PYG{o}{=} \PYG{n}{np}\PYG{o}{.}\PYG{n}{random}\PYG{o}{.}\PYG{n}{normal}\PYG{p}{(}\PYG{n}{scale}\PYG{o}{=}\PYG{n}{np}\PYG{o}{.}\PYG{n}{sqrt}\PYG{p}{(}\PYG{n}{dt}\PYG{p}{)}\PYG{p}{,} \PYG{n}{size}\PYG{o}{=}\PYG{n}{n\PYGZus{}steps}\PYG{p}{)}
    \PYG{n}{W} \PYG{o}{=} \PYG{n}{dW}\PYG{o}{.}\PYG{n}{cumsum}\PYG{p}{(}\PYG{p}{)}
    
    \PYG{n}{t} \PYG{o}{=} \PYG{n}{np}\PYG{o}{.}\PYG{n}{linspace}\PYG{p}{(}\PYG{n}{dt}\PYG{p}{,} \PYG{n}{n\PYGZus{}steps} \PYG{o}{*} \PYG{n}{dt}\PYG{p}{,} \PYG{n}{n\PYGZus{}steps}\PYG{p}{)}
    
    \PYG{n}{S\PYGZus{}t} \PYG{o}{=} \PYG{n}{S\PYGZus{}0} \PYG{o}{*} \PYG{n}{np}\PYG{o}{.}\PYG{n}{exp}\PYG{p}{(}\PYG{p}{(}\PYG{n}{mu} \PYG{o}{\PYGZhy{}} \PYG{l+m+mf}{0.5} \PYG{o}{*} \PYG{n}{sigma}\PYG{o}{*}\PYG{o}{*}\PYG{l+m+mi}{2}\PYG{p}{)} \PYG{o}{*} \PYG{n}{t} \PYG{o}{+} \PYG{n}{sigma} \PYG{o}{*} \PYG{n}{W}\PYG{p}{)}
    \PYG{n}{S\PYGZus{}t} \PYG{o}{=} \PYG{n}{np}\PYG{o}{.}\PYG{n}{insert}\PYG{p}{(}\PYG{n}{S\PYGZus{}t}\PYG{p}{,} \PYG{l+m+mi}{0}\PYG{p}{,} \PYG{n}{S\PYGZus{}0}\PYG{p}{)}
    
    \PYG{k}{return} \PYG{n}{S\PYGZus{}t}
\end{sphinxVerbatim}

\end{sphinxuseclass}\end{sphinxVerbatimInput}

\end{sphinxuseclass}
\sphinxAtStartPar
Now we will simulate price paths.
Here is one simulated price path:

\begin{sphinxuseclass}{cell}\begin{sphinxVerbatimInput}

\begin{sphinxuseclass}{cell_input}
\begin{sphinxVerbatim}[commandchars=\\\{\}]
\PYG{n}{simulate\PYGZus{}gbm}\PYG{p}{(}
    \PYG{n}{S\PYGZus{}0}\PYG{o}{=}\PYG{n}{train}\PYG{p}{[}\PYG{l+s+s1}{\PYGZsq{}}\PYG{l+s+s1}{Adj Close}\PYG{l+s+s1}{\PYGZsq{}}\PYG{p}{]}\PYG{o}{.}\PYG{n}{iloc}\PYG{p}{[}\PYG{o}{\PYGZhy{}}\PYG{l+m+mi}{1}\PYG{p}{]}\PYG{p}{,}
    \PYG{n}{mu}\PYG{o}{=}\PYG{n}{train}\PYG{p}{[}\PYG{l+s+s1}{\PYGZsq{}}\PYG{l+s+s1}{Return}\PYG{l+s+s1}{\PYGZsq{}}\PYG{p}{]}\PYG{o}{.}\PYG{n}{pipe}\PYG{p}{(}\PYG{n}{np}\PYG{o}{.}\PYG{n}{log1p}\PYG{p}{)}\PYG{o}{.}\PYG{n}{mean}\PYG{p}{(}\PYG{p}{)}\PYG{p}{,}
    \PYG{n}{sigma}\PYG{o}{=}\PYG{n}{train}\PYG{p}{[}\PYG{l+s+s1}{\PYGZsq{}}\PYG{l+s+s1}{Return}\PYG{l+s+s1}{\PYGZsq{}}\PYG{p}{]}\PYG{o}{.}\PYG{n}{pipe}\PYG{p}{(}\PYG{n}{np}\PYG{o}{.}\PYG{n}{log1p}\PYG{p}{)}\PYG{o}{.}\PYG{n}{std}\PYG{p}{(}\PYG{p}{)}\PYG{p}{,}
    \PYG{n}{n\PYGZus{}steps}\PYG{o}{=}\PYG{n}{test}\PYG{o}{.}\PYG{n}{shape}\PYG{p}{[}\PYG{l+m+mi}{0}\PYG{p}{]}
\PYG{p}{)}
\end{sphinxVerbatim}

\end{sphinxuseclass}\end{sphinxVerbatimInput}
\begin{sphinxVerbatimOutput}

\begin{sphinxuseclass}{cell_output}
\begin{sphinxVerbatim}[commandchars=\\\{\}]
array([176.28, 177.85, 177.65, 179.67, 184.24, 183.76, 183.27, 188.1 ,
       190.6 , 189.39, 191.22, 190.03, 188.83, 189.76, 184.3 , 179.53,
       178.13, 175.49, 176.55, 174.21, 170.55, 174.73, 174.29, 174.66,
       170.95, 169.67, 170.14, 167.25, 168.43, 167.01, 166.42, 165.02,
       170.1 , 170.24, 167.6 , 169.97, 166.9 , 167.63, 162.68, 159.47,
       160.14, 162.19, 162.8 , 162.68, 162.07, 158.5 , 156.87, 155.9 ,
       158.69, 159.72, 155.5 , 156.46, 155.68, 154.18, 155.85, 158.57,
       161.09, 159.14, 158.53, 159.53, 162.18, 161.13, 160.83, 158.2 ,
       155.41, 157.58, 161.16, 161.15, 163.9 , 165.01, 163.51, 164.62,
       168.85, 168.94, 173.35, 166.49, 168.85, 169.26, 168.64, 169.06,
       164.01, 163.61, 164.71, 168.78, 167.58, 165.63, 164.49, 167.07,
       168.11, 166.89, 168.43, 168.86, 171.65, 169.93, 169.23, 168.37,
       164.69, 165.64, 166.5 , 166.69, 166.25, 162.74, 161.83, 161.13,
       159.27, 159.03, 160.22, 165.24, 165.87, 166.72, 166.7 , 161.89,
       162.  , 162.32, 168.94, 168.61, 169.59, 169.68, 166.75, 169.97,
       172.18, 174.53, 172.22, 176.27, 172.59, 174.38, 180.72, 178.1 ,
       176.7 , 177.16, 175.94, 171.87, 172.23, 169.54, 171.  , 168.71,
       173.08, 171.13, 170.44, 172.83, 169.68, 170.47, 174.21, 170.02,
       170.7 , 171.58, 173.9 , 170.71, 167.36, 168.92, 169.9 , 170.75,
       171.87, 170.21, 171.02, 172.  , 170.24, 175.52, 177.03, 173.91,
       175.91, 173.41, 175.76, 179.2 , 177.08, 179.98, 181.35, 183.92,
       189.72, 189.18, 187.14, 184.72, 182.55, 182.52, 183.7 , 184.7 ,
       187.33, 187.56, 192.12, 191.52, 200.15, 202.35, 199.84, 196.69,
       198.41, 197.92, 200.38, 202.09, 202.07, 199.6 , 195.08, 193.91,
       196.76, 197.64, 193.99, 194.73, 196.12, 193.61, 194.28, 194.67,
       191.38, 192.67, 194.59, 198.16, 201.7 , 197.57, 194.87, 196.67,
       198.48, 200.31, 213.12, 215.28, 219.41, 222.97, 225.52, 224.64,
       227.59, 225.06, 224.46, 222.98, 223.5 , 232.08, 225.56, 228.26,
       222.75, 221.33, 225.41, 225.88, 222.3 , 220.03, 222.64, 220.32,
       221.31, 221.7 , 219.66, 227.47, 230.01, 223.  , 223.89, 221.8 ,
       225.04, 222.48, 222.31, 224.32, 227.66, 223.61, 222.67, 221.24,
       219.2 , 225.64, 227.33, 223.07])
\end{sphinxVerbatim}

\end{sphinxuseclass}\end{sphinxVerbatimOutput}

\end{sphinxuseclass}
\sphinxAtStartPar
We will combine \sphinxcode{\sphinxupquote{simulate\_gbm()}} with a list comprehension and \sphinxcode{\sphinxupquote{pd.concat()}} to simulate many price paths.
To simplify this combination, we will write a helper function \sphinxcode{\sphinxupquote{simulate\_gbm\_series()}} that:
\begin{enumerate}
\sphinxsetlistlabels{\arabic}{enumi}{enumii}{}{.}%
\item {} 
\sphinxAtStartPar
Returns a series

\item {} 
\sphinxAtStartPar
Helps us vary the \sphinxcode{\sphinxupquote{seed}} argument

\end{enumerate}

\begin{sphinxuseclass}{cell}\begin{sphinxVerbatimInput}

\begin{sphinxuseclass}{cell_input}
\begin{sphinxVerbatim}[commandchars=\\\{\}]
\PYG{k}{def} \PYG{n+nf}{simulate\PYGZus{}gbm\PYGZus{}series}\PYG{p}{(}\PYG{n}{seed}\PYG{p}{,} \PYG{n}{train}\PYG{o}{=}\PYG{n}{train}\PYG{p}{,} \PYG{n}{test}\PYG{o}{=}\PYG{n}{test}\PYG{p}{)}\PYG{p}{:}
    \PYG{n}{S\PYGZus{}t} \PYG{o}{=} \PYG{n}{simulate\PYGZus{}gbm}\PYG{p}{(}
        \PYG{n}{S\PYGZus{}0}\PYG{o}{=}\PYG{n}{train}\PYG{p}{[}\PYG{l+s+s1}{\PYGZsq{}}\PYG{l+s+s1}{Adj Close}\PYG{l+s+s1}{\PYGZsq{}}\PYG{p}{]}\PYG{o}{.}\PYG{n}{iloc}\PYG{p}{[}\PYG{o}{\PYGZhy{}}\PYG{l+m+mi}{1}\PYG{p}{]}\PYG{p}{,}
        \PYG{n}{mu}\PYG{o}{=}\PYG{n}{train}\PYG{p}{[}\PYG{l+s+s1}{\PYGZsq{}}\PYG{l+s+s1}{Return}\PYG{l+s+s1}{\PYGZsq{}}\PYG{p}{]}\PYG{o}{.}\PYG{n}{pipe}\PYG{p}{(}\PYG{n}{np}\PYG{o}{.}\PYG{n}{log1p}\PYG{p}{)}\PYG{o}{.}\PYG{n}{mean}\PYG{p}{(}\PYG{p}{)}\PYG{p}{,}
        \PYG{n}{sigma}\PYG{o}{=}\PYG{n}{train}\PYG{p}{[}\PYG{l+s+s1}{\PYGZsq{}}\PYG{l+s+s1}{Return}\PYG{l+s+s1}{\PYGZsq{}}\PYG{p}{]}\PYG{o}{.}\PYG{n}{pipe}\PYG{p}{(}\PYG{n}{np}\PYG{o}{.}\PYG{n}{log1p}\PYG{p}{)}\PYG{o}{.}\PYG{n}{std}\PYG{p}{(}\PYG{p}{)}\PYG{p}{,}
        \PYG{n}{n\PYGZus{}steps}\PYG{o}{=}\PYG{n}{test}\PYG{o}{.}\PYG{n}{shape}\PYG{p}{[}\PYG{l+m+mi}{0}\PYG{p}{]}\PYG{p}{,}
        \PYG{n}{seed}\PYG{o}{=}\PYG{n}{seed}
    \PYG{p}{)}
    \PYG{k}{return} \PYG{n}{pd}\PYG{o}{.}\PYG{n}{Series}\PYG{p}{(}\PYG{n}{data}\PYG{o}{=}\PYG{n}{S\PYGZus{}t}\PYG{p}{,} \PYG{n}{index}\PYG{o}{=}\PYG{n}{test}\PYG{o}{.}\PYG{n}{index}\PYG{o}{.}\PYG{n}{insert}\PYG{p}{(}\PYG{l+m+mi}{0}\PYG{p}{,} \PYG{n}{train}\PYG{o}{.}\PYG{n}{index}\PYG{p}{[}\PYG{o}{\PYGZhy{}}\PYG{l+m+mi}{1}\PYG{p}{]}\PYG{p}{)}\PYG{p}{)}
\end{sphinxVerbatim}

\end{sphinxuseclass}\end{sphinxVerbatimInput}

\end{sphinxuseclass}
\begin{sphinxuseclass}{cell}\begin{sphinxVerbatimInput}

\begin{sphinxuseclass}{cell_input}
\begin{sphinxVerbatim}[commandchars=\\\{\}]
\PYG{n}{n} \PYG{o}{=} \PYG{l+m+mi}{100}

\PYG{n}{S\PYGZus{}t} \PYG{o}{=} \PYG{n}{pd}\PYG{o}{.}\PYG{n}{concat}\PYG{p}{(}
    \PYG{n}{objs}\PYG{o}{=}\PYG{p}{[}\PYG{n}{simulate\PYGZus{}gbm\PYGZus{}series}\PYG{p}{(}\PYG{n}{seed}\PYG{o}{=}\PYG{n}{seed}\PYG{p}{)} \PYG{k}{for} \PYG{n}{seed} \PYG{o+ow}{in} \PYG{n+nb}{range}\PYG{p}{(}\PYG{n}{n}\PYG{p}{)}\PYG{p}{]}\PYG{p}{,}
    \PYG{n}{axis}\PYG{o}{=}\PYG{l+m+mi}{1}\PYG{p}{,}
    \PYG{n}{keys}\PYG{o}{=}\PYG{n+nb}{range}\PYG{p}{(}\PYG{n}{n}\PYG{p}{)}\PYG{p}{,}
    \PYG{n}{names}\PYG{o}{=}\PYG{p}{[}\PYG{l+s+s1}{\PYGZsq{}}\PYG{l+s+s1}{Simulation}\PYG{l+s+s1}{\PYGZsq{}}\PYG{p}{]}
\PYG{p}{)}
\end{sphinxVerbatim}

\end{sphinxuseclass}\end{sphinxVerbatimInput}

\end{sphinxuseclass}
\begin{sphinxuseclass}{cell}\begin{sphinxVerbatimInput}

\begin{sphinxuseclass}{cell_input}
\begin{sphinxVerbatim}[commandchars=\\\{\}]
\PYG{n}{S\PYGZus{}t}
\end{sphinxVerbatim}

\end{sphinxuseclass}\end{sphinxVerbatimInput}
\begin{sphinxVerbatimOutput}

\begin{sphinxuseclass}{cell_output}
\begin{sphinxVerbatim}[commandchars=\\\{\}]
Simulation     0      1      2      3      4      5      6      7      8   \PYGZbs{}
Date                                                                        
2021\PYGZhy{}12\PYGZhy{}31 176.28 176.28 176.28 176.28 176.28 176.28 176.28 176.28 176.28   
2022\PYGZhy{}01\PYGZhy{}03 181.45 181.05 175.30 181.52 176.60 177.70 175.59 181.24 176.72   
2022\PYGZhy{}01\PYGZhy{}04 182.80 179.50 175.33 182.97 178.19 176.96 177.82 180.10 179.98   
2022\PYGZhy{}01\PYGZhy{}05 185.84 178.19 169.69 183.45 175.59 184.08 178.62 180.39 174.71   
2022\PYGZhy{}01\PYGZhy{}06 192.75 175.38 174.33 178.31 177.72 183.54 176.28 181.74 171.11   
...           ...    ...    ...    ...    ...    ...    ...    ...    ...   
2022\PYGZhy{}12\PYGZhy{}23 264.37 309.02 209.29 243.87 280.98 228.05 238.47 208.71 217.12   
2022\PYGZhy{}12\PYGZhy{}27 269.20 307.19 210.77 252.25 279.95 227.38 235.26 207.00 213.48   
2022\PYGZhy{}12\PYGZhy{}28 266.04 316.44 212.71 256.82 276.97 225.13 239.62 206.86 212.20   
2022\PYGZhy{}12\PYGZhy{}29 260.22 310.31 211.81 257.77 275.77 226.87 239.04 202.55 212.04   
2022\PYGZhy{}12\PYGZhy{}30 262.65 308.94 211.09 265.15 276.65 226.59 243.48 202.23 214.07   

Simulation     9   ...     90     91     92     93     94     95     96  \PYGZbs{}
Date               ...                                                    
2021\PYGZhy{}12\PYGZhy{}31 176.28  ... 176.28 176.28 176.28 176.28 176.28 176.28 176.28   
2022\PYGZhy{}01\PYGZhy{}03 176.47  ... 175.84 174.78 177.11 181.07 178.55 174.62 176.66   
2022\PYGZhy{}01\PYGZhy{}04 175.85  ... 175.39 172.06 178.16 184.44 183.09 173.20 176.71   
2022\PYGZhy{}01\PYGZhy{}05 172.95  ... 173.94 170.75 176.28 190.75 181.24 174.32 175.06   
2022\PYGZhy{}01\PYGZhy{}06 173.10  ... 171.94 170.18 169.30 192.36 186.53 176.16 176.24   
...           ...  ...    ...    ...    ...    ...    ...    ...    ...   
2022\PYGZhy{}12\PYGZhy{}23 253.82  ... 239.90 350.58 237.84 239.82 211.79 240.60 222.35   
2022\PYGZhy{}12\PYGZhy{}27 260.27  ... 233.63 350.19 239.30 241.52 213.10 241.15 226.32   
2022\PYGZhy{}12\PYGZhy{}28 256.62  ... 237.98 348.81 243.23 238.31 212.25 233.59 236.23   
2022\PYGZhy{}12\PYGZhy{}29 259.85  ... 242.62 363.57 240.01 239.94 211.93 229.14 238.43   
2022\PYGZhy{}12\PYGZhy{}30 262.42  ... 237.68 372.17 242.60 239.73 205.42 229.94 237.87   

Simulation     97     98     99  
Date                             
2021\PYGZhy{}12\PYGZhy{}31 176.28 176.28 176.28  
2022\PYGZhy{}01\PYGZhy{}03 171.64 177.79 176.07  
2022\PYGZhy{}01\PYGZhy{}04 170.66 182.64 182.08  
2022\PYGZhy{}01\PYGZhy{}05 168.84 185.75 183.09  
2022\PYGZhy{}01\PYGZhy{}06 171.73 183.87 187.17  
...           ...    ...    ...  
2022\PYGZhy{}12\PYGZhy{}23 183.09 213.59 252.19  
2022\PYGZhy{}12\PYGZhy{}27 184.40 213.79 251.71  
2022\PYGZhy{}12\PYGZhy{}28 186.87 211.93 249.90  
2022\PYGZhy{}12\PYGZhy{}29 180.97 208.95 251.67  
2022\PYGZhy{}12\PYGZhy{}30 183.23 213.77 249.60  

[252 rows x 100 columns]
\end{sphinxVerbatim}

\end{sphinxuseclass}\end{sphinxVerbatimOutput}

\end{sphinxuseclass}
\sphinxAtStartPar
Below, we prefix the simulated price path column names with \sphinxcode{\sphinxupquote{\_}} to hide them from the legend.
However, this feature triggers a warning \sphinxstyleemphasis{100 times}!
We will suppress these 100 warnings with the warnings package.
Generally, we should avoid suppressing warnings, however it is the easiest option here.

\begin{sphinxuseclass}{cell}\begin{sphinxVerbatimInput}

\begin{sphinxuseclass}{cell_input}
\begin{sphinxVerbatim}[commandchars=\\\{\}]
\PYG{k+kn}{import} \PYG{n+nn}{warnings}
\end{sphinxVerbatim}

\end{sphinxuseclass}\end{sphinxVerbatimInput}

\end{sphinxuseclass}
\begin{sphinxuseclass}{cell}\begin{sphinxVerbatimInput}

\begin{sphinxuseclass}{cell_input}
\begin{sphinxVerbatim}[commandchars=\\\{\}]
\PYG{n}{fig}\PYG{p}{,} \PYG{n}{ax} \PYG{o}{=} \PYG{n}{plt}\PYG{o}{.}\PYG{n}{subplots}\PYG{p}{(}\PYG{l+m+mi}{1}\PYG{p}{,}\PYG{l+m+mi}{1}\PYG{p}{)}
\PYG{k}{with} \PYG{n}{warnings}\PYG{o}{.}\PYG{n}{catch\PYGZus{}warnings}\PYG{p}{(}\PYG{p}{)}\PYG{p}{:}
    \PYG{n}{warnings}\PYG{o}{.}\PYG{n}{simplefilter}\PYG{p}{(}\PYG{l+s+s2}{\PYGZdq{}}\PYG{l+s+s2}{ignore}\PYG{l+s+s2}{\PYGZdq{}}\PYG{p}{)}
    \PYG{n}{S\PYGZus{}t}\PYG{o}{.}\PYG{n}{add\PYGZus{}prefix}\PYG{p}{(}\PYG{l+s+s1}{\PYGZsq{}}\PYG{l+s+s1}{\PYGZus{}}\PYG{l+s+s1}{\PYGZsq{}}\PYG{p}{)}\PYG{o}{.}\PYG{n}{plot}\PYG{p}{(}\PYG{n}{alpha}\PYG{o}{=}\PYG{l+m+mf}{0.1}\PYG{p}{,} \PYG{n}{ax}\PYG{o}{=}\PYG{n}{ax}\PYG{p}{)}
\PYG{n}{S\PYGZus{}t}\PYG{o}{.}\PYG{n}{mean}\PYG{p}{(}\PYG{n}{axis}\PYG{o}{=}\PYG{l+m+mi}{1}\PYG{p}{)}\PYG{o}{.}\PYG{n}{plot}\PYG{p}{(}\PYG{n}{label}\PYG{o}{=}\PYG{l+s+s1}{\PYGZsq{}}\PYG{l+s+s1}{Mean}\PYG{l+s+s1}{\PYGZsq{}}\PYG{p}{,} \PYG{n}{ax}\PYG{o}{=}\PYG{n}{ax}\PYG{p}{)}
\PYG{n}{aapl}\PYG{o}{.}\PYG{n}{loc}\PYG{p}{[}\PYG{n}{S\PYGZus{}t}\PYG{o}{.}\PYG{n}{index}\PYG{p}{,} \PYG{p}{[}\PYG{l+s+s1}{\PYGZsq{}}\PYG{l+s+s1}{Adj Close}\PYG{l+s+s1}{\PYGZsq{}}\PYG{p}{]}\PYG{p}{]}\PYG{o}{.}\PYG{n}{plot}\PYG{p}{(}\PYG{n}{label}\PYG{o}{=}\PYG{l+s+s1}{\PYGZsq{}}\PYG{l+s+s1}{Observed}\PYG{l+s+s1}{\PYGZsq{}}\PYG{p}{,} \PYG{n}{ax}\PYG{o}{=}\PYG{n}{ax}\PYG{p}{)}
\PYG{n}{plt}\PYG{o}{.}\PYG{n}{legend}\PYG{p}{(}\PYG{p}{)}
\PYG{n}{plt}\PYG{o}{.}\PYG{n}{ylabel}\PYG{p}{(}\PYG{l+s+s1}{\PYGZsq{}}\PYG{l+s+s1}{Price (\PYGZdl{})}\PYG{l+s+s1}{\PYGZsq{}}\PYG{p}{)}
\PYG{n}{plt}\PYG{o}{.}\PYG{n}{title}\PYG{p}{(}
    \PYG{l+s+s1}{\PYGZsq{}}\PYG{l+s+s1}{Apple Simulated and Observed Prices}\PYG{l+s+s1}{\PYGZsq{}} \PYG{o}{+} 
    \PYG{l+s+sa}{f}\PYG{l+s+s1}{\PYGZsq{}}\PYG{l+s+se}{\PYGZbs{}n}\PYG{l+s+s1}{Trained from }\PYG{l+s+si}{\PYGZob{}}\PYG{n}{train}\PYG{o}{.}\PYG{n}{index}\PYG{p}{[}\PYG{l+m+mi}{0}\PYG{p}{]}\PYG{l+s+si}{:}\PYG{l+s+s1}{\PYGZpc{}Y\PYGZhy{}\PYGZpc{}m\PYGZhy{}\PYGZpc{}d}\PYG{l+s+si}{\PYGZcb{}}\PYG{l+s+s1}{ to }\PYG{l+s+si}{\PYGZob{}}\PYG{n}{train}\PYG{o}{.}\PYG{n}{index}\PYG{p}{[}\PYG{o}{\PYGZhy{}}\PYG{l+m+mi}{1}\PYG{p}{]}\PYG{l+s+si}{:}\PYG{l+s+s1}{\PYGZpc{}Y\PYGZhy{}\PYGZpc{}m\PYGZhy{}\PYGZpc{}d}\PYG{l+s+si}{\PYGZcb{}}\PYG{l+s+s1}{\PYGZsq{}}
\PYG{p}{)}
\PYG{n}{plt}\PYG{o}{.}\PYG{n}{show}\PYG{p}{(}\PYG{p}{)}
\end{sphinxVerbatim}

\end{sphinxuseclass}\end{sphinxVerbatimInput}
\begin{sphinxVerbatimOutput}

\begin{sphinxuseclass}{cell_output}
\noindent\sphinxincludegraphics{{72b966d79a0e359e38ea7033254ccf1979dd347e703223a9ed6c6ddf877bdd54}.png}

\end{sphinxuseclass}\end{sphinxVerbatimOutput}

\end{sphinxuseclass}

\subsection{Pricing Options}
\label{\detokenize{herron_05_lecture:pricing-options}}
\sphinxAtStartPar
We can use simulated price paths to price options!
We will use the Black and Scholes (1973) formula as a benchmark.
Black and Scholes (1973) provide a closed form (analytic) solution to price European options.

\begin{sphinxuseclass}{cell}\begin{sphinxVerbatimInput}

\begin{sphinxuseclass}{cell_input}
\begin{sphinxVerbatim}[commandchars=\\\{\}]
\PYG{k+kn}{from} \PYG{n+nn}{scipy}\PYG{n+nn}{.}\PYG{n+nn}{stats} \PYG{k+kn}{import} \PYG{n}{norm}
\end{sphinxVerbatim}

\end{sphinxuseclass}\end{sphinxVerbatimInput}

\end{sphinxuseclass}
\begin{sphinxuseclass}{cell}\begin{sphinxVerbatimInput}

\begin{sphinxuseclass}{cell_input}
\begin{sphinxVerbatim}[commandchars=\\\{\}]
\PYG{k}{def} \PYG{n+nf}{price\PYGZus{}bs}\PYG{p}{(}\PYG{n}{S\PYGZus{}0}\PYG{p}{,} \PYG{n}{K}\PYG{p}{,} \PYG{n}{T}\PYG{p}{,} \PYG{n}{r}\PYG{p}{,} \PYG{n}{sigma}\PYG{p}{,} \PYG{n+nb}{type}\PYG{o}{=}\PYG{l+s+s1}{\PYGZsq{}}\PYG{l+s+s1}{call}\PYG{l+s+s1}{\PYGZsq{}}\PYG{p}{)}\PYG{p}{:}
    \PYG{l+s+sd}{\PYGZsq{}\PYGZsq{}\PYGZsq{}}
\PYG{l+s+sd}{    Function used for calculating the price of European options using the analytical form of the Black\PYGZhy{}Scholes model.}
\PYG{l+s+sd}{    }
\PYG{l+s+sd}{    Parameters}
\PYG{l+s+sd}{    \PYGZhy{}\PYGZhy{}\PYGZhy{}\PYGZhy{}\PYGZhy{}\PYGZhy{}\PYGZhy{}\PYGZhy{}\PYGZhy{}\PYGZhy{}\PYGZhy{}\PYGZhy{}}
\PYG{l+s+sd}{    S\PYGZus{}0 : float}
\PYG{l+s+sd}{        Initial stock price}
\PYG{l+s+sd}{    K : float}
\PYG{l+s+sd}{        Strike price}
\PYG{l+s+sd}{    T : float}
\PYG{l+s+sd}{        Time to expiration in days}
\PYG{l+s+sd}{    r : float}
\PYG{l+s+sd}{        Daily risk\PYGZhy{}free rate}
\PYG{l+s+sd}{    sigma : float}
\PYG{l+s+sd}{        Standard deviation of daily stock returns}
\PYG{l+s+sd}{    type : str}
\PYG{l+s+sd}{        Type of the option. Allowable: [\PYGZsq{}call\PYGZsq{}, \PYGZsq{}put\PYGZsq{}]}
\PYG{l+s+sd}{    }
\PYG{l+s+sd}{    Returns}
\PYG{l+s+sd}{    \PYGZhy{}\PYGZhy{}\PYGZhy{}\PYGZhy{}\PYGZhy{}\PYGZhy{}\PYGZhy{}\PYGZhy{}\PYGZhy{}\PYGZhy{}\PYGZhy{}}
\PYG{l+s+sd}{    option\PYGZus{}premium : float}
\PYG{l+s+sd}{        The premium on the option calculated using the Black\PYGZhy{}Scholes model}
\PYG{l+s+sd}{    \PYGZsq{}\PYGZsq{}\PYGZsq{}}
    
    \PYG{n}{d1} \PYG{o}{=} \PYG{p}{(}\PYG{n}{np}\PYG{o}{.}\PYG{n}{log}\PYG{p}{(}\PYG{n}{S\PYGZus{}0} \PYG{o}{/} \PYG{n}{K}\PYG{p}{)} \PYG{o}{+} \PYG{p}{(}\PYG{n}{r} \PYG{o}{+} \PYG{l+m+mf}{0.5} \PYG{o}{*} \PYG{n}{sigma} \PYG{o}{*}\PYG{o}{*} \PYG{l+m+mi}{2}\PYG{p}{)} \PYG{o}{*} \PYG{n}{T}\PYG{p}{)} \PYG{o}{/} \PYG{p}{(}\PYG{n}{sigma} \PYG{o}{*} \PYG{n}{np}\PYG{o}{.}\PYG{n}{sqrt}\PYG{p}{(}\PYG{n}{T}\PYG{p}{)}\PYG{p}{)}
    \PYG{n}{d2} \PYG{o}{=} \PYG{n}{d1} \PYG{o}{\PYGZhy{}} \PYG{n}{sigma} \PYG{o}{*} \PYG{n}{np}\PYG{o}{.}\PYG{n}{sqrt}\PYG{p}{(}\PYG{n}{T}\PYG{p}{)}
    
    \PYG{k}{if} \PYG{n+nb}{type} \PYG{o}{==} \PYG{l+s+s1}{\PYGZsq{}}\PYG{l+s+s1}{call}\PYG{l+s+s1}{\PYGZsq{}}\PYG{p}{:}
        \PYG{n}{val} \PYG{o}{=} \PYG{p}{(}\PYG{n}{norm}\PYG{o}{.}\PYG{n}{cdf}\PYG{p}{(}\PYG{n}{d1}\PYG{p}{,} \PYG{l+m+mi}{0}\PYG{p}{,} \PYG{l+m+mi}{1}\PYG{p}{)} \PYG{o}{*} \PYG{n}{S\PYGZus{}0}\PYG{p}{)} \PYG{o}{\PYGZhy{}} \PYG{p}{(}\PYG{n}{norm}\PYG{o}{.}\PYG{n}{cdf}\PYG{p}{(}\PYG{n}{d2}\PYG{p}{,} \PYG{l+m+mi}{0}\PYG{p}{,} \PYG{l+m+mi}{1}\PYG{p}{)} \PYG{o}{*} \PYG{n}{K} \PYG{o}{*} \PYG{n}{np}\PYG{o}{.}\PYG{n}{exp}\PYG{p}{(}\PYG{o}{\PYGZhy{}}\PYG{n}{r} \PYG{o}{*} \PYG{n}{T}\PYG{p}{)}\PYG{p}{)}
    \PYG{k}{elif} \PYG{n+nb}{type} \PYG{o}{==} \PYG{l+s+s1}{\PYGZsq{}}\PYG{l+s+s1}{put}\PYG{l+s+s1}{\PYGZsq{}}\PYG{p}{:}
        \PYG{n}{val} \PYG{o}{=} \PYG{p}{(}\PYG{n}{norm}\PYG{o}{.}\PYG{n}{cdf}\PYG{p}{(}\PYG{o}{\PYGZhy{}}\PYG{n}{d2}\PYG{p}{,} \PYG{l+m+mi}{0}\PYG{p}{,} \PYG{l+m+mi}{1}\PYG{p}{)} \PYG{o}{*} \PYG{n}{K} \PYG{o}{*} \PYG{n}{np}\PYG{o}{.}\PYG{n}{exp}\PYG{p}{(}\PYG{o}{\PYGZhy{}}\PYG{n}{r} \PYG{o}{*} \PYG{n}{T}\PYG{p}{)}\PYG{p}{)} \PYG{o}{\PYGZhy{}} \PYG{p}{(}\PYG{n}{norm}\PYG{o}{.}\PYG{n}{cdf}\PYG{p}{(}\PYG{o}{\PYGZhy{}}\PYG{n}{d1}\PYG{p}{,} \PYG{l+m+mi}{0}\PYG{p}{,} \PYG{l+m+mi}{1}\PYG{p}{)} \PYG{o}{*} \PYG{n}{S\PYGZus{}0}\PYG{p}{)}
    \PYG{k}{else}\PYG{p}{:}
        \PYG{k}{raise} \PYG{n+ne}{ValueError}\PYG{p}{(}\PYG{l+s+s1}{\PYGZsq{}}\PYG{l+s+s1}{Wrong input for type!}\PYG{l+s+s1}{\PYGZsq{}}\PYG{p}{)}
        
    \PYG{k}{return} \PYG{n}{val}
\end{sphinxVerbatim}

\end{sphinxuseclass}\end{sphinxVerbatimInput}

\end{sphinxuseclass}
\sphinxAtStartPar
We can use the AAPL parameters above to price a European call option on AAPL stock.
We will calculate its price at the end of 2021 with an expiration at the end of 2022, assuming a 5\% risk\sphinxhyphen{}free rate.

\begin{sphinxuseclass}{cell}\begin{sphinxVerbatimInput}

\begin{sphinxuseclass}{cell_input}
\begin{sphinxVerbatim}[commandchars=\\\{\}]
\PYG{n}{S\PYGZus{}0} \PYG{o}{=} \PYG{n}{train}\PYG{p}{[}\PYG{l+s+s1}{\PYGZsq{}}\PYG{l+s+s1}{Adj Close}\PYG{l+s+s1}{\PYGZsq{}}\PYG{p}{]}\PYG{o}{.}\PYG{n}{iloc}\PYG{p}{[}\PYG{o}{\PYGZhy{}}\PYG{l+m+mi}{1}\PYG{p}{]}
\PYG{n}{K} \PYG{o}{=} \PYG{l+m+mi}{100}
\PYG{n}{T} \PYG{o}{=} \PYG{n}{test}\PYG{o}{.}\PYG{n}{shape}\PYG{p}{[}\PYG{l+m+mi}{0}\PYG{p}{]}
\PYG{n}{r} \PYG{o}{=} \PYG{l+m+mf}{0.05}\PYG{o}{/}\PYG{l+m+mi}{252}
\PYG{n}{sigma} \PYG{o}{=} \PYG{n}{train}\PYG{p}{[}\PYG{l+s+s1}{\PYGZsq{}}\PYG{l+s+s1}{Return}\PYG{l+s+s1}{\PYGZsq{}}\PYG{p}{]}\PYG{o}{.}\PYG{n}{pipe}\PYG{p}{(}\PYG{n}{np}\PYG{o}{.}\PYG{n}{log1p}\PYG{p}{)}\PYG{o}{.}\PYG{n}{std}\PYG{p}{(}\PYG{p}{)}
\end{sphinxVerbatim}

\end{sphinxuseclass}\end{sphinxVerbatimInput}

\end{sphinxuseclass}
\begin{sphinxuseclass}{cell}\begin{sphinxVerbatimInput}

\begin{sphinxuseclass}{cell_input}
\begin{sphinxVerbatim}[commandchars=\\\{\}]
\PYG{n}{S\PYGZus{}0}
\end{sphinxVerbatim}

\end{sphinxuseclass}\end{sphinxVerbatimInput}
\begin{sphinxVerbatimOutput}

\begin{sphinxuseclass}{cell_output}
\begin{sphinxVerbatim}[commandchars=\\\{\}]
176.27622985839844
\end{sphinxVerbatim}

\end{sphinxuseclass}\end{sphinxVerbatimOutput}

\end{sphinxuseclass}
\begin{sphinxuseclass}{cell}\begin{sphinxVerbatimInput}

\begin{sphinxuseclass}{cell_input}
\begin{sphinxVerbatim}[commandchars=\\\{\}]
\PYG{n}{\PYGZus{}} \PYG{o}{=} \PYG{n}{price\PYGZus{}bs}\PYG{p}{(}
    \PYG{n}{S\PYGZus{}0}\PYG{o}{=}\PYG{n}{S\PYGZus{}0}\PYG{p}{,}
    \PYG{n}{K}\PYG{o}{=}\PYG{n}{K}\PYG{p}{,}
    \PYG{n}{T}\PYG{o}{=}\PYG{n}{T}\PYG{p}{,}
    \PYG{n}{r}\PYG{o}{=}\PYG{n}{r}\PYG{p}{,}
    \PYG{n}{sigma}\PYG{o}{=}\PYG{n}{sigma}
\PYG{p}{)}

\PYG{n+nb}{print}\PYG{p}{(}\PYG{l+s+sa}{f}\PYG{l+s+s1}{\PYGZsq{}}\PYG{l+s+s1}{Black and Scholes (1973) option price: }\PYG{l+s+si}{\PYGZob{}}\PYG{n}{\PYGZus{}}\PYG{l+s+si}{:}\PYG{l+s+s1}{0.2f}\PYG{l+s+si}{\PYGZcb{}}\PYG{l+s+s1}{\PYGZsq{}}\PYG{p}{)}
\end{sphinxVerbatim}

\end{sphinxuseclass}\end{sphinxVerbatimInput}
\begin{sphinxVerbatimOutput}

\begin{sphinxuseclass}{cell_output}
\begin{sphinxVerbatim}[commandchars=\\\{\}]
Black and Scholes (1973) option price: 81.21
\end{sphinxVerbatim}

\end{sphinxuseclass}\end{sphinxVerbatimOutput}

\end{sphinxuseclass}
\sphinxAtStartPar
To simulate the Black and Scholes (1973) option price, we must simulate AAPL prices with the same 5\% risk\sphinxhyphen{}free rate as the drift.
We will write a new helper function to use the same inputs as above.

\begin{sphinxuseclass}{cell}\begin{sphinxVerbatimInput}

\begin{sphinxuseclass}{cell_input}
\begin{sphinxVerbatim}[commandchars=\\\{\}]
\PYG{k}{def} \PYG{n+nf}{simulate\PYGZus{}gbm\PYGZus{}series}\PYG{p}{(}\PYG{n}{seed}\PYG{p}{,} \PYG{n}{S\PYGZus{}0}\PYG{o}{=}\PYG{n}{S\PYGZus{}0}\PYG{p}{,} \PYG{n}{T}\PYG{o}{=}\PYG{n}{T}\PYG{p}{,} \PYG{n}{r}\PYG{o}{=}\PYG{n}{r}\PYG{p}{,} \PYG{n}{sigma}\PYG{o}{=}\PYG{n}{sigma}\PYG{p}{,} \PYG{n}{train}\PYG{o}{=}\PYG{n}{train}\PYG{p}{,} \PYG{n}{test}\PYG{o}{=}\PYG{n}{test}\PYG{p}{)}\PYG{p}{:}
    \PYG{n}{S\PYGZus{}t} \PYG{o}{=} \PYG{n}{simulate\PYGZus{}gbm}\PYG{p}{(}
        \PYG{n}{S\PYGZus{}0}\PYG{o}{=}\PYG{n}{S\PYGZus{}0}\PYG{p}{,}
        \PYG{n}{mu}\PYG{o}{=}\PYG{n}{r}\PYG{p}{,}
        \PYG{n}{sigma}\PYG{o}{=}\PYG{n}{sigma}\PYG{p}{,}
        \PYG{n}{n\PYGZus{}steps}\PYG{o}{=}\PYG{n}{T}\PYG{p}{,}
        \PYG{n}{seed}\PYG{o}{=}\PYG{n}{seed}
    \PYG{p}{)}
    \PYG{k}{return} \PYG{n}{pd}\PYG{o}{.}\PYG{n}{Series}\PYG{p}{(}\PYG{n}{data}\PYG{o}{=}\PYG{n}{S\PYGZus{}t}\PYG{p}{,} \PYG{n}{index}\PYG{o}{=}\PYG{n}{test}\PYG{o}{.}\PYG{n}{index}\PYG{o}{.}\PYG{n}{insert}\PYG{p}{(}\PYG{l+m+mi}{0}\PYG{p}{,} \PYG{n}{train}\PYG{o}{.}\PYG{n}{index}\PYG{p}{[}\PYG{o}{\PYGZhy{}}\PYG{l+m+mi}{1}\PYG{p}{]}\PYG{p}{)}\PYG{p}{)}
\end{sphinxVerbatim}

\end{sphinxuseclass}\end{sphinxVerbatimInput}

\end{sphinxuseclass}
\sphinxAtStartPar
We will simulate more price paths to increase the precision of out option price.

\begin{sphinxuseclass}{cell}\begin{sphinxVerbatimInput}

\begin{sphinxuseclass}{cell_input}
\begin{sphinxVerbatim}[commandchars=\\\{\}]
\PYG{n}{n} \PYG{o}{=} \PYG{l+m+mi}{10\PYGZus{}000}

\PYG{n}{S\PYGZus{}t} \PYG{o}{=} \PYG{n}{pd}\PYG{o}{.}\PYG{n}{concat}\PYG{p}{(}
    \PYG{n}{objs}\PYG{o}{=}\PYG{p}{[}\PYG{n}{simulate\PYGZus{}gbm\PYGZus{}series}\PYG{p}{(}\PYG{n}{seed}\PYG{o}{=}\PYG{n}{seed}\PYG{p}{)} \PYG{k}{for} \PYG{n}{seed} \PYG{o+ow}{in} \PYG{n+nb}{range}\PYG{p}{(}\PYG{n}{n}\PYG{p}{)}\PYG{p}{]}\PYG{p}{,}
    \PYG{n}{axis}\PYG{o}{=}\PYG{l+m+mi}{1}\PYG{p}{,}
    \PYG{n}{keys}\PYG{o}{=}\PYG{n+nb}{range}\PYG{p}{(}\PYG{n}{n}\PYG{p}{)}\PYG{p}{,}
    \PYG{n}{names}\PYG{o}{=}\PYG{p}{[}\PYG{l+s+s1}{\PYGZsq{}}\PYG{l+s+s1}{Simulation}\PYG{l+s+s1}{\PYGZsq{}}\PYG{p}{]}
\PYG{p}{)}
\end{sphinxVerbatim}

\end{sphinxuseclass}\end{sphinxVerbatimInput}

\end{sphinxuseclass}
\begin{sphinxuseclass}{cell}\begin{sphinxVerbatimInput}

\begin{sphinxuseclass}{cell_input}
\begin{sphinxVerbatim}[commandchars=\\\{\}]
\PYG{n}{S\PYGZus{}t}
\end{sphinxVerbatim}

\end{sphinxuseclass}\end{sphinxVerbatimInput}
\begin{sphinxVerbatimOutput}

\begin{sphinxuseclass}{cell_output}
\begin{sphinxVerbatim}[commandchars=\\\{\}]
Simulation   0      1      2      3      4      5      6      7      8     \PYGZbs{}
Date                                                                        
2021\PYGZhy{}12\PYGZhy{}31 176.28 176.28 176.28 176.28 176.28 176.28 176.28 176.28 176.28   
2022\PYGZhy{}01\PYGZhy{}03 181.27 180.87 175.13 181.34 176.43 177.52 175.42 181.06 176.54   
2022\PYGZhy{}01\PYGZhy{}04 182.44 179.15 174.99 182.61 177.84 176.61 177.47 179.75 179.63   
2022\PYGZhy{}01\PYGZhy{}05 185.30 177.67 169.19 182.91 175.08 183.54 178.09 179.85 174.20   
2022\PYGZhy{}01\PYGZhy{}06 191.99 174.70 173.65 177.61 177.02 182.82 175.59 181.03 170.43   
...           ...    ...    ...    ...    ...    ...    ...    ...    ...   
2022\PYGZhy{}12\PYGZhy{}23 207.42 242.46 164.21 191.34 220.46 178.93 187.11 163.75 170.35   
2022\PYGZhy{}12\PYGZhy{}27 211.01 240.78 165.21 197.72 219.43 178.22 184.40 162.25 167.33   
2022\PYGZhy{}12\PYGZhy{}28 208.33 247.79 166.57 201.11 216.88 176.29 187.63 161.99 166.16   
2022\PYGZhy{}12\PYGZhy{}29 203.57 242.75 165.70 201.65 215.73 177.48 187.00 158.45 165.87   
2022\PYGZhy{}12\PYGZhy{}30 205.27 241.45 164.97 207.22 216.20 177.08 190.29 158.04 167.30   

Simulation   9     ...   9990   9991   9992   9993   9994   9995   9996  \PYGZbs{}
Date               ...                                                    
2021\PYGZhy{}12\PYGZhy{}31 176.28  ... 176.28 176.28 176.28 176.28 176.28 176.28 176.28   
2022\PYGZhy{}01\PYGZhy{}03 176.29  ... 178.14 175.95 174.19 173.95 172.99 172.26 173.57   
2022\PYGZhy{}01\PYGZhy{}04 175.50  ... 177.45 176.81 175.57 176.18 171.43 171.36 178.19   
2022\PYGZhy{}01\PYGZhy{}05 172.44  ... 180.20 175.37 176.23 174.00 167.70 169.66 180.88   
2022\PYGZhy{}01\PYGZhy{}06 172.42  ... 183.12 176.18 175.12 176.93 167.96 172.82 182.73   
...           ...  ...    ...    ...    ...    ...    ...    ...    ...   
2022\PYGZhy{}12\PYGZhy{}23 199.15  ... 156.18 165.90 137.42 252.22 144.24 129.41 242.87   
2022\PYGZhy{}12\PYGZhy{}27 204.00  ... 160.32 164.82 138.60 255.43 145.55 133.71 241.07   
2022\PYGZhy{}12\PYGZhy{}28 200.95  ... 158.05 164.04 139.88 259.64 146.61 131.03 240.96   
2022\PYGZhy{}12\PYGZhy{}29 203.28  ... 156.62 164.92 139.07 262.25 144.18 128.91 242.47   
2022\PYGZhy{}12\PYGZhy{}30 205.09  ... 156.43 164.43 134.79 268.96 140.79 130.94 244.51   

Simulation   9997   9998   9999  
Date                             
2021\PYGZhy{}12\PYGZhy{}31 176.28 176.28 176.28  
2022\PYGZhy{}01\PYGZhy{}03 176.58 177.60 174.86  
2022\PYGZhy{}01\PYGZhy{}04 175.19 176.40 176.52  
2022\PYGZhy{}01\PYGZhy{}05 175.69 176.73 176.66  
2022\PYGZhy{}01\PYGZhy{}06 176.78 172.54 175.32  
...           ...    ...    ...  
2022\PYGZhy{}12\PYGZhy{}23 204.14 194.58 186.51  
2022\PYGZhy{}12\PYGZhy{}27 205.05 187.88 192.04  
2022\PYGZhy{}12\PYGZhy{}28 213.31 191.79 192.79  
2022\PYGZhy{}12\PYGZhy{}29 217.28 187.56 189.87  
2022\PYGZhy{}12\PYGZhy{}30 223.76 187.76 192.34  

[252 rows x 10000 columns]
\end{sphinxVerbatim}

\end{sphinxuseclass}\end{sphinxVerbatimOutput}

\end{sphinxuseclass}
\sphinxAtStartPar
We can compare this price to a simulated price.
The payoff on the call option is \$S\_T \sphinxhyphen{} K\$ or \$0\$, whichever is higher.
The price of the option is the present value of the mean payoff, discounted at the risk\sphinxhyphen{}free rate.

\begin{sphinxuseclass}{cell}\begin{sphinxVerbatimInput}

\begin{sphinxuseclass}{cell_input}
\begin{sphinxVerbatim}[commandchars=\\\{\}]
\PYG{n}{payoff} \PYG{o}{=} \PYG{n}{np}\PYG{o}{.}\PYG{n}{maximum}\PYG{p}{(}\PYG{n}{S\PYGZus{}t}\PYG{o}{.}\PYG{n}{iloc}\PYG{p}{[}\PYG{o}{\PYGZhy{}}\PYG{l+m+mi}{1}\PYG{p}{]} \PYG{o}{\PYGZhy{}} \PYG{n}{K}\PYG{p}{,} \PYG{l+m+mi}{0}\PYG{p}{)}
\PYG{n}{\PYGZus{}} \PYG{o}{=} \PYG{n}{payoff}\PYG{o}{.}\PYG{n}{mean}\PYG{p}{(}\PYG{p}{)} \PYG{o}{*} \PYG{n}{np}\PYG{o}{.}\PYG{n}{exp}\PYG{p}{(}\PYG{o}{\PYGZhy{}}\PYG{n}{r} \PYG{o}{*} \PYG{n}{T}\PYG{p}{)}

\PYG{n+nb}{print}\PYG{p}{(}\PYG{l+s+sa}{f}\PYG{l+s+s1}{\PYGZsq{}}\PYG{l+s+s1}{Simulated option price: }\PYG{l+s+si}{\PYGZob{}}\PYG{n}{\PYGZus{}}\PYG{l+s+si}{:}\PYG{l+s+s1}{0.2f}\PYG{l+s+si}{\PYGZcb{}}\PYG{l+s+s1}{\PYGZsq{}}\PYG{p}{)}
\end{sphinxVerbatim}

\end{sphinxuseclass}\end{sphinxVerbatimInput}
\begin{sphinxVerbatimOutput}

\begin{sphinxuseclass}{cell_output}
\begin{sphinxVerbatim}[commandchars=\\\{\}]
Simulated option price: 81.92
\end{sphinxVerbatim}

\end{sphinxuseclass}\end{sphinxVerbatimOutput}

\end{sphinxuseclass}
\sphinxAtStartPar
The option prices do not match exactly.
However, we can simulate more price paths to bring our simulated option price closer to the analytic solution.


\section{Estimating Value\sphinxhyphen{}at\sphinxhyphen{}Risk using Monte Carlo}
\label{\detokenize{herron_05_lecture:estimating-value-at-risk-using-monte-carlo}}
\sphinxAtStartPar
Value\sphinxhyphen{}at\sphinxhyphen{}Risk (VaR) measures the risk associated with a portfolio
VaR reports the worst expected loss, at a given level of confidence, over a certain horizon under normal market conditions.
For example, say the 1\sphinxhyphen{}day 95\% VaR of our portfolio is \$100.
This implies that that 95\% of the time (under normal market conditions), we should not lose more than \textbackslash{}\$100 over one day.
We typically present VaR as a positive value, so a VaR of \$100 implies a loss of less than \$100.

\sphinxAtStartPar
We can calculate VaR several ways, including:
\begin{itemize}
\item {} 
\sphinxAtStartPar
Parametric Approach (Variance\sphinxhyphen{}Covariance)

\item {} 
\sphinxAtStartPar
Historical Simulation Approach

\item {} 
\sphinxAtStartPar
Monte Carlo simulations

\end{itemize}

\sphinxAtStartPar
We only consider the last method to calculate the 1\sphinxhyphen{}day VaR of an portfolio of 20 shares each of META and GOOG.

\begin{sphinxuseclass}{cell}\begin{sphinxVerbatimInput}

\begin{sphinxuseclass}{cell_input}
\begin{sphinxVerbatim}[commandchars=\\\{\}]
\PYG{n}{tickers} \PYG{o}{=} \PYG{p}{[}\PYG{l+s+s1}{\PYGZsq{}}\PYG{l+s+s1}{GOOG}\PYG{l+s+s1}{\PYGZsq{}}\PYG{p}{,} \PYG{l+s+s1}{\PYGZsq{}}\PYG{l+s+s1}{META}\PYG{l+s+s1}{\PYGZsq{}}\PYG{p}{]}
\PYG{n}{shares} \PYG{o}{=} \PYG{n}{np}\PYG{o}{.}\PYG{n}{array}\PYG{p}{(}\PYG{p}{[}\PYG{l+m+mi}{20}\PYG{p}{,} \PYG{l+m+mi}{20}\PYG{p}{]}\PYG{p}{)}
\PYG{n}{T} \PYG{o}{=} \PYG{l+m+mi}{1}
\PYG{n}{n\PYGZus{}sims} \PYG{o}{=} \PYG{l+m+mi}{10\PYGZus{}000}
\end{sphinxVerbatim}

\end{sphinxuseclass}\end{sphinxVerbatimInput}

\end{sphinxuseclass}
\sphinxAtStartPar
However, we will download all data from Yahoo! Finance and subset our data later.

\begin{sphinxuseclass}{cell}\begin{sphinxVerbatimInput}

\begin{sphinxuseclass}{cell_input}
\begin{sphinxVerbatim}[commandchars=\\\{\}]
\PYG{n}{df} \PYG{o}{=} \PYG{p}{(}
    \PYG{n}{yf}\PYG{o}{.}\PYG{n}{download}\PYG{p}{(}\PYG{n}{tickers}\PYG{o}{=}\PYG{n}{tickers}\PYG{p}{,} \PYG{n}{progress}\PYG{o}{=}\PYG{k+kc}{False}\PYG{p}{)}
    \PYG{o}{.}\PYG{n}{assign}\PYG{p}{(}
        \PYG{n}{Date}\PYG{o}{=}\PYG{k}{lambda} \PYG{n}{x}\PYG{p}{:} \PYG{n}{x}\PYG{o}{.}\PYG{n}{index}\PYG{o}{.}\PYG{n}{tz\PYGZus{}localize}\PYG{p}{(}\PYG{k+kc}{None}\PYG{p}{)}\PYG{p}{,}
    \PYG{p}{)}
    \PYG{o}{.}\PYG{n}{set\PYGZus{}index}\PYG{p}{(}\PYG{l+s+s1}{\PYGZsq{}}\PYG{l+s+s1}{Date}\PYG{l+s+s1}{\PYGZsq{}}\PYG{p}{)}
    \PYG{o}{.}\PYG{n}{rename\PYGZus{}axis}\PYG{p}{(}\PYG{n}{columns}\PYG{o}{=}\PYG{p}{[}\PYG{l+s+s1}{\PYGZsq{}}\PYG{l+s+s1}{Variable}\PYG{l+s+s1}{\PYGZsq{}}\PYG{p}{,} \PYG{l+s+s1}{\PYGZsq{}}\PYG{l+s+s1}{Ticker}\PYG{l+s+s1}{\PYGZsq{}}\PYG{p}{]}\PYG{p}{)}
\PYG{p}{)}
\end{sphinxVerbatim}

\end{sphinxuseclass}\end{sphinxVerbatimInput}

\end{sphinxuseclass}
\sphinxAtStartPar
Next, we calculate daily returns during 2022.
Choosing the window to define “normal market conditions” is part art, part science, and beyod the scope of this lecture notebook.

\begin{sphinxuseclass}{cell}\begin{sphinxVerbatimInput}

\begin{sphinxuseclass}{cell_input}
\begin{sphinxVerbatim}[commandchars=\\\{\}]
\PYG{n}{returns} \PYG{o}{=} \PYG{n}{df}\PYG{p}{[}\PYG{l+s+s1}{\PYGZsq{}}\PYG{l+s+s1}{Adj Close}\PYG{l+s+s1}{\PYGZsq{}}\PYG{p}{]}\PYG{o}{.}\PYG{n}{pct\PYGZus{}change}\PYG{p}{(}\PYG{p}{)}\PYG{o}{.}\PYG{n}{loc}\PYG{p}{[}\PYG{l+s+s1}{\PYGZsq{}}\PYG{l+s+s1}{2022}\PYG{l+s+s1}{\PYGZsq{}}\PYG{p}{]}
\end{sphinxVerbatim}

\end{sphinxuseclass}\end{sphinxVerbatimInput}

\end{sphinxuseclass}
\begin{sphinxuseclass}{cell}\begin{sphinxVerbatimInput}

\begin{sphinxuseclass}{cell_input}
\begin{sphinxVerbatim}[commandchars=\\\{\}]
\PYG{n}{returns}
\end{sphinxVerbatim}

\end{sphinxuseclass}\end{sphinxVerbatimInput}
\begin{sphinxVerbatimOutput}

\begin{sphinxuseclass}{cell_output}
\begin{sphinxVerbatim}[commandchars=\\\{\}]
Ticker      GOOG  META
Date                  
2022\PYGZhy{}01\PYGZhy{}03  0.00  0.01
2022\PYGZhy{}01\PYGZhy{}04 \PYGZhy{}0.00 \PYGZhy{}0.01
2022\PYGZhy{}01\PYGZhy{}05 \PYGZhy{}0.05 \PYGZhy{}0.04
2022\PYGZhy{}01\PYGZhy{}06 \PYGZhy{}0.00  0.03
2022\PYGZhy{}01\PYGZhy{}07 \PYGZhy{}0.00 \PYGZhy{}0.00
...          ...   ...
2022\PYGZhy{}12\PYGZhy{}23  0.02  0.01
2022\PYGZhy{}12\PYGZhy{}27 \PYGZhy{}0.02 \PYGZhy{}0.01
2022\PYGZhy{}12\PYGZhy{}28 \PYGZhy{}0.02 \PYGZhy{}0.01
2022\PYGZhy{}12\PYGZhy{}29  0.03  0.04
2022\PYGZhy{}12\PYGZhy{}30 \PYGZhy{}0.00  0.00

[251 rows x 2 columns]
\end{sphinxVerbatim}

\end{sphinxuseclass}\end{sphinxVerbatimOutput}

\end{sphinxuseclass}
\sphinxAtStartPar
We will need the variance\sphinxhyphen{}covariance matrix.

\begin{sphinxuseclass}{cell}\begin{sphinxVerbatimInput}

\begin{sphinxuseclass}{cell_input}
\begin{sphinxVerbatim}[commandchars=\\\{\}]
\PYG{n}{cov\PYGZus{}mat} \PYG{o}{=} \PYG{n}{returns}\PYG{o}{.}\PYG{n}{cov}\PYG{p}{(}\PYG{p}{)}

\PYG{n}{cov\PYGZus{}mat} \PYG{o}{*} \PYG{l+m+mi}{1\PYGZus{}000\PYGZus{}000}
\end{sphinxVerbatim}

\end{sphinxuseclass}\end{sphinxVerbatimInput}
\begin{sphinxVerbatimOutput}

\begin{sphinxuseclass}{cell_output}
\begin{sphinxVerbatim}[commandchars=\\\{\}]
Ticker   GOOG    META
Ticker               
GOOG   596.48  674.26
META   674.26 1638.50
\end{sphinxVerbatim}

\end{sphinxuseclass}\end{sphinxVerbatimOutput}

\end{sphinxuseclass}
\sphinxAtStartPar
We will use the variance\sphinxhyphen{}covariance matrix to calculate the Cholesky decomposition.

\begin{sphinxuseclass}{cell}\begin{sphinxVerbatimInput}

\begin{sphinxuseclass}{cell_input}
\begin{sphinxVerbatim}[commandchars=\\\{\}]
\PYG{n}{chol\PYGZus{}mat} \PYG{o}{=} \PYG{n}{np}\PYG{o}{.}\PYG{n}{linalg}\PYG{o}{.}\PYG{n}{cholesky}\PYG{p}{(}\PYG{n}{cov\PYGZus{}mat}\PYG{p}{)}

\PYG{n}{chol\PYGZus{}mat}
\end{sphinxVerbatim}

\end{sphinxuseclass}\end{sphinxVerbatimInput}
\begin{sphinxVerbatimOutput}

\begin{sphinxuseclass}{cell_output}
\begin{sphinxVerbatim}[commandchars=\\\{\}]
array([[0.02, 0.  ],
       [0.03, 0.03]])
\end{sphinxVerbatim}

\end{sphinxuseclass}\end{sphinxVerbatimOutput}

\end{sphinxuseclass}
\sphinxAtStartPar
The Cholesky decomposition helps us generate random variables with the same variance and covariance as the observed data.

\begin{sphinxuseclass}{cell}\begin{sphinxVerbatimInput}

\begin{sphinxuseclass}{cell_input}
\begin{sphinxVerbatim}[commandchars=\\\{\}]
\PYG{n}{rv} \PYG{o}{=} \PYG{n}{np}\PYG{o}{.}\PYG{n}{random}\PYG{o}{.}\PYG{n}{normal}\PYG{p}{(}\PYG{n}{size}\PYG{o}{=}\PYG{p}{(}\PYG{n}{n\PYGZus{}sims}\PYG{p}{,} \PYG{n+nb}{len}\PYG{p}{(}\PYG{n}{tickers}\PYG{p}{)}\PYG{p}{)}\PYG{p}{)}
\end{sphinxVerbatim}

\end{sphinxuseclass}\end{sphinxVerbatimInput}

\end{sphinxuseclass}
\begin{sphinxuseclass}{cell}\begin{sphinxVerbatimInput}

\begin{sphinxuseclass}{cell_input}
\begin{sphinxVerbatim}[commandchars=\\\{\}]
\PYG{n}{correlated\PYGZus{}rv} \PYG{o}{=} \PYG{p}{(}\PYG{n}{chol\PYGZus{}mat} \PYG{o}{@} \PYG{n}{rv}\PYG{o}{.}\PYG{n}{T}\PYG{p}{)}\PYG{o}{.}\PYG{n}{T}
\end{sphinxVerbatim}

\end{sphinxuseclass}\end{sphinxVerbatimInput}

\end{sphinxuseclass}
\begin{sphinxuseclass}{cell}\begin{sphinxVerbatimInput}

\begin{sphinxuseclass}{cell_input}
\begin{sphinxVerbatim}[commandchars=\\\{\}]
\PYG{n}{correlated\PYGZus{}rv}
\end{sphinxVerbatim}

\end{sphinxuseclass}\end{sphinxVerbatimInput}
\begin{sphinxVerbatimOutput}

\begin{sphinxuseclass}{cell_output}
\begin{sphinxVerbatim}[commandchars=\\\{\}]
array([[ 0.04,  0.05],
       [ 0.02,  0.02],
       [\PYGZhy{}0.02, \PYGZhy{}0.04],
       ...,
       [\PYGZhy{}0.04,  0.01],
       [\PYGZhy{}0.  , \PYGZhy{}0.09],
       [\PYGZhy{}0.02,  0.  ]])
\end{sphinxVerbatim}

\end{sphinxuseclass}\end{sphinxVerbatimOutput}

\end{sphinxuseclass}
\sphinxAtStartPar
These random variables have a variance\sphinxhyphen{}covariance matrix similar to the real data.

\begin{sphinxuseclass}{cell}\begin{sphinxVerbatimInput}

\begin{sphinxuseclass}{cell_input}
\begin{sphinxVerbatim}[commandchars=\\\{\}]
\PYG{n}{np}\PYG{o}{.}\PYG{n}{cov}\PYG{p}{(}\PYG{n}{correlated\PYGZus{}rv}\PYG{o}{.}\PYG{n}{T}\PYG{p}{)}  \PYG{o}{*} \PYG{l+m+mi}{1\PYGZus{}000\PYGZus{}000}
\end{sphinxVerbatim}

\end{sphinxuseclass}\end{sphinxVerbatimInput}
\begin{sphinxVerbatimOutput}

\begin{sphinxuseclass}{cell_output}
\begin{sphinxVerbatim}[commandchars=\\\{\}]
array([[ 597.18,  685.37],
       [ 685.37, 1647.31]])
\end{sphinxVerbatim}

\end{sphinxuseclass}\end{sphinxVerbatimOutput}

\end{sphinxuseclass}
\begin{sphinxuseclass}{cell}\begin{sphinxVerbatimInput}

\begin{sphinxuseclass}{cell_input}
\begin{sphinxVerbatim}[commandchars=\\\{\}]
\PYG{n}{np}\PYG{o}{.}\PYG{n}{allclose}\PYG{p}{(}\PYG{n}{cov\PYGZus{}mat}\PYG{p}{,} \PYG{n}{np}\PYG{o}{.}\PYG{n}{cov}\PYG{p}{(}\PYG{n}{correlated\PYGZus{}rv}\PYG{o}{.}\PYG{n}{T}\PYG{p}{)}\PYG{p}{,} \PYG{n}{rtol}\PYG{o}{=}\PYG{l+m+mf}{0.05}\PYG{p}{)}
\end{sphinxVerbatim}

\end{sphinxuseclass}\end{sphinxVerbatimInput}
\begin{sphinxVerbatimOutput}

\begin{sphinxuseclass}{cell_output}
\begin{sphinxVerbatim}[commandchars=\\\{\}]
True
\end{sphinxVerbatim}

\end{sphinxuseclass}\end{sphinxVerbatimOutput}

\end{sphinxuseclass}
\begin{sphinxuseclass}{cell}\begin{sphinxVerbatimInput}

\begin{sphinxuseclass}{cell_input}
\begin{sphinxVerbatim}[commandchars=\\\{\}]
\PYG{n}{np}\PYG{o}{.}\PYG{n}{mean}\PYG{p}{(}\PYG{n}{correlated\PYGZus{}rv}\PYG{p}{,} \PYG{n}{axis}\PYG{o}{=}\PYG{l+m+mi}{0}\PYG{p}{)} \PYG{o}{*} \PYG{l+m+mi}{100}
\end{sphinxVerbatim}

\end{sphinxuseclass}\end{sphinxVerbatimInput}
\begin{sphinxVerbatimOutput}

\begin{sphinxuseclass}{cell_output}
\begin{sphinxVerbatim}[commandchars=\\\{\}]
array([0.02, 0.02])
\end{sphinxVerbatim}

\end{sphinxuseclass}\end{sphinxVerbatimOutput}

\end{sphinxuseclass}
\begin{sphinxuseclass}{cell}\begin{sphinxVerbatimInput}

\begin{sphinxuseclass}{cell_input}
\begin{sphinxVerbatim}[commandchars=\\\{\}]
\PYG{n}{returns}\PYG{o}{.}\PYG{n}{mean}\PYG{p}{(}\PYG{p}{)}\PYG{o}{.}\PYG{n}{values} \PYG{o}{*} \PYG{l+m+mi}{100}
\end{sphinxVerbatim}

\end{sphinxuseclass}\end{sphinxVerbatimInput}
\begin{sphinxVerbatimOutput}

\begin{sphinxuseclass}{cell_output}
\begin{sphinxVerbatim}[commandchars=\\\{\}]
array([\PYGZhy{}0.16, \PYGZhy{}0.32])
\end{sphinxVerbatim}

\end{sphinxuseclass}\end{sphinxVerbatimOutput}

\end{sphinxuseclass}
\sphinxAtStartPar
Here are the parameters for the simulated price paths:

\begin{sphinxuseclass}{cell}\begin{sphinxVerbatimInput}

\begin{sphinxuseclass}{cell_input}
\begin{sphinxVerbatim}[commandchars=\\\{\}]
\PYG{n}{mu} \PYG{o}{=} \PYG{n}{returns}\PYG{o}{.}\PYG{n}{mean}\PYG{p}{(}\PYG{p}{)}\PYG{o}{.}\PYG{n}{values}
\PYG{n}{sigma} \PYG{o}{=} \PYG{n}{returns}\PYG{o}{.}\PYG{n}{std}\PYG{p}{(}\PYG{p}{)}\PYG{o}{.}\PYG{n}{values}
\PYG{n}{S\PYGZus{}0} \PYG{o}{=} \PYG{n}{df}\PYG{o}{.}\PYG{n}{loc}\PYG{p}{[}\PYG{l+s+s1}{\PYGZsq{}}\PYG{l+s+s1}{2021}\PYG{l+s+s1}{\PYGZsq{}}\PYG{p}{,} \PYG{l+s+s1}{\PYGZsq{}}\PYG{l+s+s1}{Adj Close}\PYG{l+s+s1}{\PYGZsq{}}\PYG{p}{]}\PYG{o}{.}\PYG{n}{iloc}\PYG{p}{[}\PYG{o}{\PYGZhy{}}\PYG{l+m+mi}{1}\PYG{p}{]}\PYG{o}{.}\PYG{n}{values}
\PYG{n}{P\PYGZus{}0} \PYG{o}{=} \PYG{n}{S\PYGZus{}0}\PYG{o}{.}\PYG{n}{dot}\PYG{p}{(}\PYG{n}{shares}\PYG{p}{)}
\end{sphinxVerbatim}

\end{sphinxuseclass}\end{sphinxVerbatimInput}

\end{sphinxuseclass}
\sphinxAtStartPar
Calculate terminal prices using the GBM formula above:

\begin{sphinxuseclass}{cell}\begin{sphinxVerbatimInput}

\begin{sphinxuseclass}{cell_input}
\begin{sphinxVerbatim}[commandchars=\\\{\}]
\PYG{n}{S\PYGZus{}T} \PYG{o}{=} \PYG{n}{S\PYGZus{}0} \PYG{o}{*} \PYG{n}{np}\PYG{o}{.}\PYG{n}{exp}\PYG{p}{(}\PYG{p}{(}\PYG{n}{r} \PYG{o}{\PYGZhy{}} \PYG{l+m+mf}{0.5} \PYG{o}{*} \PYG{n}{sigma} \PYG{o}{*}\PYG{o}{*} \PYG{l+m+mi}{2}\PYG{p}{)} \PYG{o}{*} \PYG{n}{T} \PYG{o}{+} \PYG{n}{sigma} \PYG{o}{*} \PYG{n}{np}\PYG{o}{.}\PYG{n}{sqrt}\PYG{p}{(}\PYG{n}{T}\PYG{p}{)} \PYG{o}{*} \PYG{n}{correlated\PYGZus{}rv}\PYG{p}{)}

\PYG{n}{S\PYGZus{}T}
\end{sphinxVerbatim}

\end{sphinxuseclass}\end{sphinxVerbatimInput}
\begin{sphinxVerbatimOutput}

\begin{sphinxuseclass}{cell_output}
\begin{sphinxVerbatim}[commandchars=\\\{\}]
array([[144.79, 336.82],
       [144.74, 336.38],
       [144.61, 335.63],
       ...,
       [144.53, 336.3 ],
       [144.65, 334.95],
       [144.6 , 336.2 ]])
\end{sphinxVerbatim}

\end{sphinxuseclass}\end{sphinxVerbatimOutput}

\end{sphinxuseclass}
\sphinxAtStartPar
Calculate terminal portfolio values and returns.
Note that these are dollar values, since VaR is typically expressed in dollar values.

\begin{sphinxuseclass}{cell}\begin{sphinxVerbatimInput}

\begin{sphinxuseclass}{cell_input}
\begin{sphinxVerbatim}[commandchars=\\\{\}]
\PYG{n}{P\PYGZus{}T} \PYG{o}{=} \PYG{n}{S\PYGZus{}T}\PYG{o}{.}\PYG{n}{dot}\PYG{p}{(}\PYG{n}{shares}\PYG{p}{)}

\PYG{n}{P\PYGZus{}T}
\end{sphinxVerbatim}

\end{sphinxuseclass}\end{sphinxVerbatimInput}
\begin{sphinxVerbatimOutput}

\begin{sphinxuseclass}{cell_output}
\begin{sphinxVerbatim}[commandchars=\\\{\}]
array([9632.23, 9622.51, 9604.65, ..., 9616.48, 9592.05, 9615.98])
\end{sphinxVerbatim}

\end{sphinxuseclass}\end{sphinxVerbatimOutput}

\end{sphinxuseclass}
\begin{sphinxuseclass}{cell}\begin{sphinxVerbatimInput}

\begin{sphinxuseclass}{cell_input}
\begin{sphinxVerbatim}[commandchars=\\\{\}]
\PYG{n}{P\PYGZus{}diff} \PYG{o}{=} \PYG{n}{P\PYGZus{}T} \PYG{o}{\PYGZhy{}} \PYG{n}{P\PYGZus{}0}

\PYG{n}{P\PYGZus{}diff}
\end{sphinxVerbatim}

\end{sphinxuseclass}\end{sphinxVerbatimInput}
\begin{sphinxVerbatimOutput}

\begin{sphinxuseclass}{cell_output}
\begin{sphinxVerbatim}[commandchars=\\\{\}]
array([ 11.64,   1.92, \PYGZhy{}15.94, ...,  \PYGZhy{}4.11, \PYGZhy{}28.54,  \PYGZhy{}4.61])
\end{sphinxVerbatim}

\end{sphinxuseclass}\end{sphinxVerbatimOutput}

\end{sphinxuseclass}
\begin{sphinxuseclass}{cell}\begin{sphinxVerbatimInput}

\begin{sphinxuseclass}{cell_input}
\begin{sphinxVerbatim}[commandchars=\\\{\}]
\PYG{n}{P\PYGZus{}diff}\PYG{o}{.}\PYG{n}{mean}\PYG{p}{(}\PYG{p}{)}
\end{sphinxVerbatim}

\end{sphinxuseclass}\end{sphinxVerbatimInput}
\begin{sphinxVerbatimOutput}

\begin{sphinxuseclass}{cell_output}
\begin{sphinxVerbatim}[commandchars=\\\{\}]
\PYGZhy{}4.376750155672959
\end{sphinxVerbatim}

\end{sphinxuseclass}\end{sphinxVerbatimOutput}

\end{sphinxuseclass}
\sphinxAtStartPar
Next, we calculate VaR.

\begin{sphinxuseclass}{cell}\begin{sphinxVerbatimInput}

\begin{sphinxuseclass}{cell_input}
\begin{sphinxVerbatim}[commandchars=\\\{\}]
\PYG{n}{percentiles} \PYG{o}{=} \PYG{p}{[}\PYG{l+m+mf}{0.01}\PYG{p}{,} \PYG{l+m+mf}{0.05}\PYG{p}{,} \PYG{l+m+mf}{0.1}\PYG{p}{]}
\PYG{n}{var} \PYG{o}{=} \PYG{n}{np}\PYG{o}{.}\PYG{n}{percentile}\PYG{p}{(}\PYG{n}{P\PYGZus{}diff}\PYG{p}{,} \PYG{n}{percentiles}\PYG{p}{)}

\PYG{k}{for} \PYG{n}{x}\PYG{p}{,} \PYG{n}{y} \PYG{o+ow}{in} \PYG{n+nb}{zip}\PYG{p}{(}\PYG{n}{percentiles}\PYG{p}{,} \PYG{n}{var}\PYG{p}{)}\PYG{p}{:}
    \PYG{n+nb}{print}\PYG{p}{(}\PYG{l+s+sa}{f}\PYG{l+s+s1}{\PYGZsq{}}\PYG{l+s+s1}{1\PYGZhy{}day VaR with }\PYG{l+s+si}{\PYGZob{}}\PYG{l+m+mi}{100}\PYG{o}{\PYGZhy{}}\PYG{n}{x}\PYG{l+s+si}{\PYGZcb{}}\PYG{l+s+s1}{\PYGZpc{} confidence: \PYGZdl{}}\PYG{l+s+si}{\PYGZob{}}\PYG{o}{\PYGZhy{}}\PYG{n}{y}\PYG{l+s+si}{:}\PYG{l+s+s1}{.2f}\PYG{l+s+si}{\PYGZcb{}}\PYG{l+s+s1}{\PYGZsq{}}\PYG{p}{)}
\end{sphinxVerbatim}

\end{sphinxuseclass}\end{sphinxVerbatimInput}
\begin{sphinxVerbatimOutput}

\begin{sphinxuseclass}{cell_output}
\begin{sphinxVerbatim}[commandchars=\\\{\}]
1\PYGZhy{}day VaR with 99.99\PYGZpc{} confidence: \PYGZdl{}48.02
1\PYGZhy{}day VaR with 99.95\PYGZpc{} confidence: \PYGZdl{}44.65
1\PYGZhy{}day VaR with 99.9\PYGZpc{} confidence: \PYGZdl{}42.63
\end{sphinxVerbatim}

\end{sphinxuseclass}\end{sphinxVerbatimOutput}

\end{sphinxuseclass}
\sphinxAtStartPar
Finally, we will plot VaR:

\begin{sphinxuseclass}{cell}\begin{sphinxVerbatimInput}

\begin{sphinxuseclass}{cell_input}
\begin{sphinxVerbatim}[commandchars=\\\{\}]
\PYG{n}{fig}\PYG{p}{,} \PYG{n}{ax} \PYG{o}{=} \PYG{n}{plt}\PYG{o}{.}\PYG{n}{subplots}\PYG{p}{(}\PYG{p}{)}
\PYG{n}{ax}\PYG{o}{.}\PYG{n}{hist}\PYG{p}{(}\PYG{n}{P\PYGZus{}diff}\PYG{p}{,} \PYG{n}{bins}\PYG{o}{=}\PYG{l+m+mi}{100}\PYG{p}{,} \PYG{n}{density}\PYG{o}{=}\PYG{k+kc}{True}\PYG{p}{)}
\PYG{n}{ax}\PYG{o}{.}\PYG{n}{set\PYGZus{}title}\PYG{p}{(}\PYG{l+s+sa}{f}\PYG{l+s+s1}{\PYGZsq{}}\PYG{l+s+s1}{Distribution of 1\PYGZhy{}Day Changes in Portfolio Value}\PYG{l+s+se}{\PYGZbs{}n}\PYG{l+s+s1}{ from }\PYG{l+s+si}{\PYGZob{}}\PYG{n}{n\PYGZus{}sims}\PYG{l+s+si}{\PYGZcb{}}\PYG{l+s+s1}{ Simulations}\PYG{l+s+s1}{\PYGZsq{}}\PYG{p}{)}
\PYG{n}{ax}\PYG{o}{.}\PYG{n}{axvline}\PYG{p}{(}\PYG{n}{x}\PYG{o}{=}\PYG{n}{var}\PYG{p}{[}\PYG{l+m+mi}{2}\PYG{p}{]}\PYG{p}{,} \PYG{n}{color}\PYG{o}{=}\PYG{l+s+s1}{\PYGZsq{}}\PYG{l+s+s1}{red}\PYG{l+s+s1}{\PYGZsq{}}\PYG{p}{,} \PYG{n}{ls}\PYG{o}{=}\PYG{l+s+s1}{\PYGZsq{}}\PYG{l+s+s1}{\PYGZhy{}\PYGZhy{}}\PYG{l+s+s1}{\PYGZsq{}}\PYG{p}{)}
\PYG{n}{ax}\PYG{o}{.}\PYG{n}{text}\PYG{p}{(}\PYG{n}{x}\PYG{o}{=}\PYG{n}{var}\PYG{p}{[}\PYG{l+m+mi}{2}\PYG{p}{]}\PYG{p}{,} \PYG{n}{y}\PYG{o}{=}\PYG{l+m+mi}{1}\PYG{p}{,} \PYG{n}{s}\PYG{o}{=}\PYG{l+s+s1}{\PYGZsq{}}\PYG{l+s+s1}{99}\PYG{l+s+s1}{\PYGZpc{}}\PYG{l+s+s1}{ 1\PYGZhy{}Day VaR}\PYG{l+s+s1}{\PYGZsq{}}\PYG{p}{,} \PYG{n}{color}\PYG{o}{=}\PYG{l+s+s1}{\PYGZsq{}}\PYG{l+s+s1}{red}\PYG{l+s+s1}{\PYGZsq{}}\PYG{p}{,} \PYG{n}{ha}\PYG{o}{=}\PYG{l+s+s1}{\PYGZsq{}}\PYG{l+s+s1}{right}\PYG{l+s+s1}{\PYGZsq{}}\PYG{p}{,} \PYG{n}{va}\PYG{o}{=}\PYG{l+s+s1}{\PYGZsq{}}\PYG{l+s+s1}{top}\PYG{l+s+s1}{\PYGZsq{}}\PYG{p}{,} \PYG{n}{rotation}\PYG{o}{=}\PYG{l+m+mi}{90}\PYG{p}{,} \PYG{n}{transform}\PYG{o}{=}\PYG{n}{ax}\PYG{o}{.}\PYG{n}{get\PYGZus{}xaxis\PYGZus{}transform}\PYG{p}{(}\PYG{p}{)}\PYG{p}{)}
\PYG{n}{ax}\PYG{o}{.}\PYG{n}{set\PYGZus{}ylabel}\PYG{p}{(}\PYG{l+s+s1}{\PYGZsq{}}\PYG{l+s+s1}{Density}\PYG{l+s+s1}{\PYGZsq{}}\PYG{p}{)}
\PYG{n}{ax}\PYG{o}{.}\PYG{n}{set\PYGZus{}xlabel}\PYG{p}{(}\PYG{l+s+s1}{\PYGZsq{}}\PYG{l+s+s1}{1\PYGZhy{}Day Change in Portfolio Value (\PYGZdl{})}\PYG{l+s+s1}{\PYGZsq{}}\PYG{p}{)}
\PYG{n}{plt}\PYG{o}{.}\PYG{n}{show}\PYG{p}{(}\PYG{p}{)}
\end{sphinxVerbatim}

\end{sphinxuseclass}\end{sphinxVerbatimInput}
\begin{sphinxVerbatimOutput}

\begin{sphinxuseclass}{cell_output}
\noindent\sphinxincludegraphics{{bf986aec5eb57850436872d90b758d098940c93b27c249ef2621ab4d88d9af03}.png}

\end{sphinxuseclass}\end{sphinxVerbatimOutput}

\end{sphinxuseclass}
\sphinxstepscope


\section{Herron Topic 5 \sphinxhyphen{} Practice (Blank)}
\label{\detokenize{herron_05_practice:herron-topic-5-practice-blank}}\label{\detokenize{herron_05_practice::doc}}

\subsection{Announcements}
\label{\detokenize{herron_05_practice:announcements}}

\subsection{Practice}
\label{\detokenize{herron_05_practice:practice}}
\begin{sphinxuseclass}{cell}\begin{sphinxVerbatimInput}

\begin{sphinxuseclass}{cell_input}
\begin{sphinxVerbatim}[commandchars=\\\{\}]
\PYG{k+kn}{import} \PYG{n+nn}{matplotlib}\PYG{n+nn}{.}\PYG{n+nn}{pyplot} \PYG{k}{as} \PYG{n+nn}{plt}
\PYG{k+kn}{import} \PYG{n+nn}{numpy} \PYG{k}{as} \PYG{n+nn}{np}
\PYG{k+kn}{import} \PYG{n+nn}{pandas} \PYG{k}{as} \PYG{n+nn}{pd}
\end{sphinxVerbatim}

\end{sphinxuseclass}\end{sphinxVerbatimInput}

\end{sphinxuseclass}
\begin{sphinxuseclass}{cell}\begin{sphinxVerbatimInput}

\begin{sphinxuseclass}{cell_input}
\begin{sphinxVerbatim}[commandchars=\\\{\}]
\PYG{o}{\PYGZpc{}}\PYG{k}{config} InlineBackend.figure\PYGZus{}format = \PYGZsq{}retina\PYGZsq{}
\PYG{o}{\PYGZpc{}}\PYG{k}{precision} 2
\PYG{n}{pd}\PYG{o}{.}\PYG{n}{options}\PYG{o}{.}\PYG{n}{display}\PYG{o}{.}\PYG{n}{float\PYGZus{}format} \PYG{o}{=} \PYG{l+s+s1}{\PYGZsq{}}\PYG{l+s+si}{\PYGZob{}:.2f\PYGZcb{}}\PYG{l+s+s1}{\PYGZsq{}}\PYG{o}{.}\PYG{n}{format}
\end{sphinxVerbatim}

\end{sphinxuseclass}\end{sphinxVerbatimInput}

\end{sphinxuseclass}
\begin{sphinxuseclass}{cell}\begin{sphinxVerbatimInput}

\begin{sphinxuseclass}{cell_input}
\begin{sphinxVerbatim}[commandchars=\\\{\}]
\PYG{k+kn}{import} \PYG{n+nn}{yfinance} \PYG{k}{as} \PYG{n+nn}{yf}
\PYG{k+kn}{import} \PYG{n+nn}{pandas\PYGZus{}datareader} \PYG{k}{as} \PYG{n+nn}{pdr}
\PYG{k+kn}{import} \PYG{n+nn}{requests\PYGZus{}cache}
\PYG{n}{session} \PYG{o}{=} \PYG{n}{requests\PYGZus{}cache}\PYG{o}{.}\PYG{n}{CachedSession}\PYG{p}{(}\PYG{p}{)}
\end{sphinxVerbatim}

\end{sphinxuseclass}\end{sphinxVerbatimInput}

\end{sphinxuseclass}

\subsubsection{Estimate \$\textbackslash{}pi\$ by simulating darts thrown at a dart board}
\label{\detokenize{herron_05_practice:estimate-pi-by-simulating-darts-thrown-at-a-dart-board}}
\sphinxAtStartPar
\sphinxstyleemphasis{Hints:}
Select random \$x\$s and \$y\$s such that \$\sphinxhyphen{}r \textbackslash{}leq x \textbackslash{}leq +r\$ and \$\sphinxhyphen{}r \textbackslash{}leq x \textbackslash{}leq +r\$.
Darts are on the board if \$x\textasciicircum{}2 + y\textasciicircum{}2 \textbackslash{}leq r\textasciicircum{}2\$.
The area of the circlular board is \$\textbackslash{}pi r\textasciicircum{}2\$, and the area of square around the board is \$(2r)\textasciicircum{}2 = 4r\textasciicircum{}2\$.
The fraction \$f\$ of darts on the board is the same as the ratio of circle area to square area, so \$f = \textbackslash{}frac\{\textbackslash{}pi r\textasciicircum{}2\}\{4 r\textasciicircum{}2\}\$.


\subsubsection{Simulate your wealth \$W\_T\$ by randomly sampling market returns}
\label{\detokenize{herron_05_practice:simulate-your-wealth-w-t-by-randomly-sampling-market-returns}}
\sphinxAtStartPar
Use monthly market returns from the French Data Library.
Only invest one cash flow \$W\_0\$, and plot the distribution of \$W\_T\$.


\subsubsection{Repeat the exercise above but add end\sphinxhyphen{}of\sphinxhyphen{}month investments \$C\_t\$}
\label{\detokenize{herron_05_practice:repeat-the-exercise-above-but-add-end-of-month-investments-c-t}}
\sphinxstepscope


\section{Herron Topic 5 \sphinxhyphen{} Practice (Monday 2:45 PM, Section 3)}
\label{\detokenize{herron_05_practice_03:herron-topic-5-practice-monday-2-45-pm-section-3}}\label{\detokenize{herron_05_practice_03::doc}}

\subsection{Announcements}
\label{\detokenize{herron_05_practice_03:announcements}}\begin{itemize}
\item {} 
\sphinxAtStartPar
Due dates
\begin{itemize}
\item {} 
\sphinxAtStartPar
Team project 2 is due by 11:59 PM on Wednesday, 4/26

\item {} 
\sphinxAtStartPar
Teammates review 2 is due by 11:59 PM on Wednesday, 4/26

\item {} 
\sphinxAtStartPar
30,000 DataCamp XP are due by 11:59 PM on \sphinxstyleemphasis{Friday, 4/28}

\item {} 
\sphinxAtStartPar
TRACE course evaluations are due by 11:59 PM on \sphinxstyleemphasis{Friday, 4/28}

\end{itemize}

\item {} 
\sphinxAtStartPar
\sphinxstyleemphasis{\sphinxstylestrong{TRACE course evaluations are optional (and anonymous), but I value your feedback!}}
\begin{itemize}
\item {} 
\sphinxAtStartPar
I always find student feedback useful, and I read and re\sphinxhyphen{}read it!

\item {} 
\sphinxAtStartPar
I tweak my courses every semester based on student feedback!

\item {} 
\sphinxAtStartPar
Your feedback does not have to be perfect to be useful!

\item {} 
\sphinxAtStartPar
If you had a magic wand, what would you change about my course and approach?

\end{itemize}

\item {} 
\sphinxAtStartPar
This week and next, we can discuss whatever topics you want, including simulations and team project 2

\end{itemize}


\subsection{Practice}
\label{\detokenize{herron_05_practice_03:practice}}
\begin{sphinxuseclass}{cell}\begin{sphinxVerbatimInput}

\begin{sphinxuseclass}{cell_input}
\begin{sphinxVerbatim}[commandchars=\\\{\}]
\PYG{k+kn}{import} \PYG{n+nn}{matplotlib}\PYG{n+nn}{.}\PYG{n+nn}{pyplot} \PYG{k}{as} \PYG{n+nn}{plt}
\PYG{k+kn}{import} \PYG{n+nn}{numpy} \PYG{k}{as} \PYG{n+nn}{np}
\PYG{k+kn}{import} \PYG{n+nn}{pandas} \PYG{k}{as} \PYG{n+nn}{pd}
\end{sphinxVerbatim}

\end{sphinxuseclass}\end{sphinxVerbatimInput}

\end{sphinxuseclass}
\begin{sphinxuseclass}{cell}\begin{sphinxVerbatimInput}

\begin{sphinxuseclass}{cell_input}
\begin{sphinxVerbatim}[commandchars=\\\{\}]
\PYG{o}{\PYGZpc{}}\PYG{k}{config} InlineBackend.figure\PYGZus{}format = \PYGZsq{}retina\PYGZsq{}
\PYG{o}{\PYGZpc{}}\PYG{k}{precision} 2
\PYG{n}{pd}\PYG{o}{.}\PYG{n}{options}\PYG{o}{.}\PYG{n}{display}\PYG{o}{.}\PYG{n}{float\PYGZus{}format} \PYG{o}{=} \PYG{l+s+s1}{\PYGZsq{}}\PYG{l+s+si}{\PYGZob{}:.2f\PYGZcb{}}\PYG{l+s+s1}{\PYGZsq{}}\PYG{o}{.}\PYG{n}{format}
\end{sphinxVerbatim}

\end{sphinxuseclass}\end{sphinxVerbatimInput}

\end{sphinxuseclass}
\begin{sphinxuseclass}{cell}\begin{sphinxVerbatimInput}

\begin{sphinxuseclass}{cell_input}
\begin{sphinxVerbatim}[commandchars=\\\{\}]
\PYG{k+kn}{import} \PYG{n+nn}{yfinance} \PYG{k}{as} \PYG{n+nn}{yf}
\PYG{k+kn}{import} \PYG{n+nn}{pandas\PYGZus{}datareader} \PYG{k}{as} \PYG{n+nn}{pdr}
\PYG{k+kn}{import} \PYG{n+nn}{requests\PYGZus{}cache}
\PYG{n}{session} \PYG{o}{=} \PYG{n}{requests\PYGZus{}cache}\PYG{o}{.}\PYG{n}{CachedSession}\PYG{p}{(}\PYG{p}{)}
\end{sphinxVerbatim}

\end{sphinxuseclass}\end{sphinxVerbatimInput}

\end{sphinxuseclass}

\subsubsection{Estimate \$\textbackslash{}pi\$ by simulating darts thrown at a dart board}
\label{\detokenize{herron_05_practice_03:estimate-pi-by-simulating-darts-thrown-at-a-dart-board}}
\sphinxAtStartPar
\sphinxstyleemphasis{Hints:}
Select random \$x\$s and \$y\$s such that \$\sphinxhyphen{}r \textbackslash{}leq x \textbackslash{}leq +r\$ and \$\sphinxhyphen{}r \textbackslash{}leq x \textbackslash{}leq +r\$.
Darts are on the board if \$x\textasciicircum{}2 + y\textasciicircum{}2 \textbackslash{}leq r\textasciicircum{}2\$.
The area of the circlular board is \$\textbackslash{}pi r\textasciicircum{}2\$, and the area of square around the board is \$(2r)\textasciicircum{}2 = 4r\textasciicircum{}2\$.
The fraction \$f\$ of darts on the board is the same as the ratio of circle area to square area, so \$f = \textbackslash{}frac\{\textbackslash{}pi r\textasciicircum{}2\}\{4 r\textasciicircum{}2\}\$.

\sphinxAtStartPar
First we throw darts at the board.
Darts with \$x\textasciicircum{}2 + y\textasciicircum{}2 \textbackslash{}leq r\textasciicircum{}2\$ are on the board.

\begin{sphinxuseclass}{cell}\begin{sphinxVerbatimInput}

\begin{sphinxuseclass}{cell_input}
\begin{sphinxVerbatim}[commandchars=\\\{\}]
\PYG{k}{def} \PYG{n+nf}{throw\PYGZus{}darts}\PYG{p}{(}\PYG{n}{n}\PYG{o}{=}\PYG{l+m+mi}{10\PYGZus{}000}\PYG{p}{,} \PYG{n}{r}\PYG{o}{=}\PYG{l+m+mi}{1}\PYG{p}{,} \PYG{n}{seed}\PYG{o}{=}\PYG{l+m+mi}{42}\PYG{p}{)}\PYG{p}{:}
    \PYG{n}{np}\PYG{o}{.}\PYG{n}{random}\PYG{o}{.}\PYG{n}{seed}\PYG{p}{(}\PYG{n}{seed}\PYG{p}{)}
    \PYG{k}{return} \PYG{p}{(}
        \PYG{n}{pd}\PYG{o}{.}\PYG{n}{DataFrame}\PYG{p}{(}
            \PYG{n}{data}\PYG{o}{=}\PYG{n}{np}\PYG{o}{.}\PYG{n}{random}\PYG{o}{.}\PYG{n}{uniform}\PYG{p}{(}\PYG{n}{low}\PYG{o}{=}\PYG{o}{\PYGZhy{}}\PYG{n}{r}\PYG{p}{,} \PYG{n}{high}\PYG{o}{=}\PYG{n}{r}\PYG{p}{,} \PYG{n}{size}\PYG{o}{=}\PYG{l+m+mi}{2}\PYG{o}{*}\PYG{n}{n}\PYG{p}{)}\PYG{o}{.}\PYG{n}{reshape}\PYG{p}{(}\PYG{n}{n}\PYG{p}{,} \PYG{l+m+mi}{2}\PYG{p}{)}\PYG{p}{,} 
            \PYG{n}{columns}\PYG{o}{=}\PYG{p}{[}\PYG{l+s+s1}{\PYGZsq{}}\PYG{l+s+s1}{x}\PYG{l+s+s1}{\PYGZsq{}}\PYG{p}{,} \PYG{l+s+s1}{\PYGZsq{}}\PYG{l+s+s1}{y}\PYG{l+s+s1}{\PYGZsq{}}\PYG{p}{]}
        \PYG{p}{)}
        \PYG{o}{.}\PYG{n}{assign}\PYG{p}{(}\PYG{n}{board}\PYG{o}{=}\PYG{k}{lambda} \PYG{n}{x}\PYG{p}{:} \PYG{n}{x}\PYG{p}{[}\PYG{l+s+s1}{\PYGZsq{}}\PYG{l+s+s1}{x}\PYG{l+s+s1}{\PYGZsq{}}\PYG{p}{]}\PYG{o}{*}\PYG{o}{*}\PYG{l+m+mi}{2} \PYG{o}{+} \PYG{n}{x}\PYG{p}{[}\PYG{l+s+s1}{\PYGZsq{}}\PYG{l+s+s1}{y}\PYG{l+s+s1}{\PYGZsq{}}\PYG{p}{]}\PYG{o}{*}\PYG{o}{*}\PYG{l+m+mi}{2} \PYG{o}{\PYGZlt{}}\PYG{o}{=} \PYG{n}{r}\PYG{o}{*}\PYG{o}{*}\PYG{l+m+mi}{2}\PYG{p}{)}
        \PYG{o}{.}\PYG{n}{rename\PYGZus{}axis}\PYG{p}{(}\PYG{n}{index}\PYG{o}{=}\PYG{l+s+s1}{\PYGZsq{}}\PYG{l+s+s1}{n}\PYG{l+s+s1}{\PYGZsq{}}\PYG{p}{,} \PYG{n}{columns}\PYG{o}{=}\PYG{l+s+s1}{\PYGZsq{}}\PYG{l+s+s1}{Variable}\PYG{l+s+s1}{\PYGZsq{}}\PYG{p}{)}
    \PYG{p}{)}
\end{sphinxVerbatim}

\end{sphinxuseclass}\end{sphinxVerbatimInput}

\end{sphinxuseclass}
\begin{sphinxuseclass}{cell}\begin{sphinxVerbatimInput}

\begin{sphinxuseclass}{cell_input}
\begin{sphinxVerbatim}[commandchars=\\\{\}]
\PYG{n}{throw\PYGZus{}darts}\PYG{p}{(}\PYG{p}{)}
\end{sphinxVerbatim}

\end{sphinxuseclass}\end{sphinxVerbatimInput}
\begin{sphinxVerbatimOutput}

\begin{sphinxuseclass}{cell_output}
\begin{sphinxVerbatim}[commandchars=\\\{\}]
Variable     x     y  board
n                          
0        \PYGZhy{}0.25  0.90   True
1         0.46  0.20   True
2        \PYGZhy{}0.69 \PYGZhy{}0.69   True
3        \PYGZhy{}0.88  0.73  False
4         0.20  0.42   True
...        ...   ...    ...
9995      0.15  0.52   True
9996     \PYGZhy{}0.82 \PYGZhy{}0.01   True
9997      0.80  0.75  False
9998     \PYGZhy{}0.91 \PYGZhy{}0.39   True
9999     \PYGZhy{}0.11 \PYGZhy{}0.66   True

[10000 rows x 3 columns]
\end{sphinxVerbatim}

\end{sphinxuseclass}\end{sphinxVerbatimOutput}

\end{sphinxuseclass}
\sphinxAtStartPar
Next, we visualize these darts with a scatter plot.
Seaborn’s \sphinxcode{\sphinxupquote{scatterplot()}} helps color darts by location (i.e., on or off board).
The \sphinxcode{\sphinxupquote{.pipe()}} method lets us send the output of the \sphinxcode{\sphinxupquote{.assign()}} method to \sphinxcode{\sphinxupquote{sns.scatterplot()}} without assigning a temporary data frame.

\begin{sphinxuseclass}{cell}\begin{sphinxVerbatimInput}

\begin{sphinxuseclass}{cell_input}
\begin{sphinxVerbatim}[commandchars=\\\{\}]
\PYG{k+kn}{import} \PYG{n+nn}{seaborn} \PYG{k}{as} \PYG{n+nn}{sns}
\end{sphinxVerbatim}

\end{sphinxuseclass}\end{sphinxVerbatimInput}

\end{sphinxuseclass}
\begin{sphinxuseclass}{cell}\begin{sphinxVerbatimInput}

\begin{sphinxuseclass}{cell_input}
\begin{sphinxVerbatim}[commandchars=\\\{\}]
\PYG{p}{(}
    \PYG{n}{throw\PYGZus{}darts}\PYG{p}{(}\PYG{p}{)}
    \PYG{o}{.}\PYG{n}{assign}\PYG{p}{(}\PYG{n}{Location}\PYG{o}{=}\PYG{k}{lambda} \PYG{n}{x}\PYG{p}{:} \PYG{n}{np}\PYG{o}{.}\PYG{n}{where}\PYG{p}{(}\PYG{n}{x}\PYG{p}{[}\PYG{l+s+s1}{\PYGZsq{}}\PYG{l+s+s1}{board}\PYG{l+s+s1}{\PYGZsq{}}\PYG{p}{]}\PYG{p}{,} \PYG{l+s+s1}{\PYGZsq{}}\PYG{l+s+s1}{On Board}\PYG{l+s+s1}{\PYGZsq{}}\PYG{p}{,} \PYG{l+s+s1}{\PYGZsq{}}\PYG{l+s+s1}{Off Board}\PYG{l+s+s1}{\PYGZsq{}}\PYG{p}{)}\PYG{p}{)}
    \PYG{o}{.}\PYG{n}{pipe}\PYG{p}{(}\PYG{k}{lambda} \PYG{n}{x}\PYG{p}{:} \PYG{n}{sns}\PYG{o}{.}\PYG{n}{scatterplot}\PYG{p}{(}\PYG{n}{x}\PYG{o}{=}\PYG{l+s+s1}{\PYGZsq{}}\PYG{l+s+s1}{x}\PYG{l+s+s1}{\PYGZsq{}}\PYG{p}{,} \PYG{n}{y}\PYG{o}{=}\PYG{l+s+s1}{\PYGZsq{}}\PYG{l+s+s1}{y}\PYG{l+s+s1}{\PYGZsq{}}\PYG{p}{,} \PYG{n}{data}\PYG{o}{=}\PYG{n}{x}\PYG{p}{,} \PYG{n}{hue}\PYG{o}{=}\PYG{l+s+s1}{\PYGZsq{}}\PYG{l+s+s1}{Location}\PYG{l+s+s1}{\PYGZsq{}}\PYG{p}{)}\PYG{p}{)}
\PYG{p}{)}
\PYG{n}{plt}\PYG{o}{.}\PYG{n}{title}\PYG{p}{(}\PYG{l+s+s1}{\PYGZsq{}}\PYG{l+s+s1}{Simulated Dart Throws}\PYG{l+s+s1}{\PYGZsq{}}\PYG{p}{)}
\PYG{n}{plt}\PYG{o}{.}\PYG{n}{show}\PYG{p}{(}\PYG{p}{)}
\end{sphinxVerbatim}

\end{sphinxuseclass}\end{sphinxVerbatimInput}
\begin{sphinxVerbatimOutput}

\begin{sphinxuseclass}{cell_output}
\noindent\sphinxincludegraphics{{9c8f487cc54cb77b787ccddc2740c30d5e0ca308ad208d8bdfe30ab6f0ed140e}.png}

\end{sphinxuseclass}\end{sphinxVerbatimOutput}

\end{sphinxuseclass}
\sphinxAtStartPar
Finally, we use the hint above to estimate \$\textbackslash{}pi\$.
The hint above says \$f = \textbackslash{}frac\{\textbackslash{}pi r\textasciicircum{}2\}\{4 r\textasciicircum{}2\}\$, where \$f\$ is the fraction of darts on the board.
Therefore, \$\textbackslash{}pi = \textbackslash{}frac\{4fr\textasciicircum{}2\}\{r\textasciicircum{}2\} = 4f\$.

\begin{sphinxuseclass}{cell}\begin{sphinxVerbatimInput}

\begin{sphinxuseclass}{cell_input}
\begin{sphinxVerbatim}[commandchars=\\\{\}]
\PYG{n}{n} \PYG{o}{=} \PYG{l+m+mi}{10\PYGZus{}000}
\PYG{n}{pi} \PYG{o}{=} \PYG{l+m+mi}{4} \PYG{o}{*} \PYG{n}{throw\PYGZus{}darts}\PYG{p}{(}\PYG{n}{n}\PYG{o}{=}\PYG{n}{n}\PYG{p}{)}\PYG{p}{[}\PYG{l+s+s2}{\PYGZdq{}}\PYG{l+s+s2}{board}\PYG{l+s+s2}{\PYGZdq{}}\PYG{p}{]}\PYG{o}{.}\PYG{n}{mean}\PYG{p}{(}\PYG{p}{)}
\PYG{n+nb}{print}\PYG{p}{(}\PYG{l+s+sa}{f}\PYG{l+s+s1}{\PYGZsq{}}\PYG{l+s+s1}{Estimate of pi based on }\PYG{l+s+si}{\PYGZob{}}\PYG{n}{n}\PYG{l+s+si}{:}\PYG{l+s+s1}{,.0f}\PYG{l+s+si}{\PYGZcb{}}\PYG{l+s+s1}{ darts: }\PYG{l+s+si}{\PYGZob{}}\PYG{n}{pi}\PYG{l+s+si}{:}\PYG{l+s+s1}{0.4f}\PYG{l+s+si}{\PYGZcb{}}\PYG{l+s+s1}{\PYGZsq{}}\PYG{p}{)}
\end{sphinxVerbatim}

\end{sphinxuseclass}\end{sphinxVerbatimInput}
\begin{sphinxVerbatimOutput}

\begin{sphinxuseclass}{cell_output}
\begin{sphinxVerbatim}[commandchars=\\\{\}]
Estimate of pi based on 10,000 darts: 3.1544
\end{sphinxVerbatim}

\end{sphinxuseclass}\end{sphinxVerbatimOutput}

\end{sphinxuseclass}
\sphinxAtStartPar
We increase the precision of our \$\textbackslash{}pi\$ estimate by increasing the number of simulated darts \$n\$.

\begin{sphinxuseclass}{cell}\begin{sphinxVerbatimInput}

\begin{sphinxuseclass}{cell_input}
\begin{sphinxVerbatim}[commandchars=\\\{\}]
\PYG{k}{for} \PYG{n}{n} \PYG{o+ow}{in} \PYG{l+m+mi}{10}\PYG{o}{*}\PYG{o}{*}\PYG{n}{np}\PYG{o}{.}\PYG{n}{arange}\PYG{p}{(}\PYG{l+m+mi}{7}\PYG{p}{)}\PYG{p}{:}
    \PYG{n}{pi} \PYG{o}{=} \PYG{l+m+mi}{4} \PYG{o}{*} \PYG{n}{throw\PYGZus{}darts}\PYG{p}{(}\PYG{n}{n}\PYG{o}{=}\PYG{n}{n}\PYG{p}{)}\PYG{p}{[}\PYG{l+s+s2}{\PYGZdq{}}\PYG{l+s+s2}{board}\PYG{l+s+s2}{\PYGZdq{}}\PYG{p}{]}\PYG{o}{.}\PYG{n}{mean}\PYG{p}{(}\PYG{p}{)}
    \PYG{n+nb}{print}\PYG{p}{(}\PYG{l+s+sa}{f}\PYG{l+s+s1}{\PYGZsq{}}\PYG{l+s+s1}{Estimate of pi based on }\PYG{l+s+si}{\PYGZob{}}\PYG{n}{n}\PYG{l+s+si}{:}\PYG{l+s+s1}{\PYGZlt{}9,.0f}\PYG{l+s+si}{\PYGZcb{}}\PYG{l+s+s1}{ darts: }\PYG{l+s+si}{\PYGZob{}}\PYG{n}{pi}\PYG{l+s+si}{:}\PYG{l+s+s1}{0.4f}\PYG{l+s+si}{\PYGZcb{}}\PYG{l+s+s1}{\PYGZsq{}}\PYG{p}{)}
\end{sphinxVerbatim}

\end{sphinxuseclass}\end{sphinxVerbatimInput}
\begin{sphinxVerbatimOutput}

\begin{sphinxuseclass}{cell_output}
\begin{sphinxVerbatim}[commandchars=\\\{\}]
Estimate of pi based on 1         darts: 4.0000
Estimate of pi based on 10        darts: 3.2000
Estimate of pi based on 100       darts: 3.0400
Estimate of pi based on 1,000     darts: 3.1040
Estimate of pi based on 10,000    darts: 3.1544
Estimate of pi based on 100,000   darts: 3.1468
Estimate of pi based on 1,000,000 darts: 3.1420
\end{sphinxVerbatim}

\end{sphinxuseclass}\end{sphinxVerbatimOutput}

\end{sphinxuseclass}

\subsubsection{Simulate your wealth \$W\_T\$ by randomly sampling market returns}
\label{\detokenize{herron_05_practice_03:simulate-your-wealth-w-t-by-randomly-sampling-market-returns}}
\sphinxAtStartPar
Use monthly market returns from the French Data Library.
Only invest one cash flow \$W\_0\$, and plot the distribution of \$W\_T\$.

\sphinxAtStartPar
First, we download data from the French Data Library.
We convert these returns from percent to decimal to simplify compounding.

\begin{sphinxuseclass}{cell}\begin{sphinxVerbatimInput}

\begin{sphinxuseclass}{cell_input}
\begin{sphinxVerbatim}[commandchars=\\\{\}]
\PYG{n}{mkt} \PYG{o}{=} \PYG{p}{(}
    \PYG{n}{pdr}\PYG{o}{.}\PYG{n}{DataReader}\PYG{p}{(}
        \PYG{n}{name}\PYG{o}{=}\PYG{l+s+s1}{\PYGZsq{}}\PYG{l+s+s1}{F\PYGZhy{}F\PYGZus{}Research\PYGZus{}Data\PYGZus{}Factors\PYGZus{}daily}\PYG{l+s+s1}{\PYGZsq{}}\PYG{p}{,}
        \PYG{n}{data\PYGZus{}source}\PYG{o}{=}\PYG{l+s+s1}{\PYGZsq{}}\PYG{l+s+s1}{famafrench}\PYG{l+s+s1}{\PYGZsq{}}\PYG{p}{,}
        \PYG{n}{start}\PYG{o}{=}\PYG{l+s+s1}{\PYGZsq{}}\PYG{l+s+s1}{1900}\PYG{l+s+s1}{\PYGZsq{}}\PYG{p}{,}
        \PYG{n}{session}\PYG{o}{=}\PYG{n}{session}
    \PYG{p}{)}\PYG{p}{[}\PYG{l+m+mi}{0}\PYG{p}{]}
    \PYG{o}{.}\PYG{n}{assign}\PYG{p}{(}\PYG{n}{mkt}\PYG{o}{=}\PYG{k}{lambda} \PYG{n}{x}\PYG{p}{:} \PYG{p}{(}\PYG{n}{x}\PYG{p}{[}\PYG{l+s+s1}{\PYGZsq{}}\PYG{l+s+s1}{Mkt\PYGZhy{}RF}\PYG{l+s+s1}{\PYGZsq{}}\PYG{p}{]} \PYG{o}{+} \PYG{n}{x}\PYG{p}{[}\PYG{l+s+s1}{\PYGZsq{}}\PYG{l+s+s1}{RF}\PYG{l+s+s1}{\PYGZsq{}}\PYG{p}{]}\PYG{p}{)} \PYG{o}{/} \PYG{l+m+mi}{100}\PYG{p}{)}
    \PYG{p}{[}\PYG{l+s+s1}{\PYGZsq{}}\PYG{l+s+s1}{mkt}\PYG{l+s+s1}{\PYGZsq{}}\PYG{p}{]}
\PYG{p}{)}
\end{sphinxVerbatim}

\end{sphinxuseclass}\end{sphinxVerbatimInput}

\end{sphinxuseclass}
\sphinxAtStartPar
Next, we write a couple of helper functions.
The \sphinxcode{\sphinxupquote{get\_sample()}} function draws one random sample of \sphinxcode{\sphinxupquote{n}} observations from returns series \sphinxcode{\sphinxupquote{x}}.
This sample provides one simulated return history.

\begin{sphinxuseclass}{cell}\begin{sphinxVerbatimInput}

\begin{sphinxuseclass}{cell_input}
\begin{sphinxVerbatim}[commandchars=\\\{\}]
\PYG{k}{def} \PYG{n+nf}{get\PYGZus{}sample}\PYG{p}{(}\PYG{n}{x}\PYG{p}{,} \PYG{n}{n}\PYG{o}{=}\PYG{l+m+mi}{10\PYGZus{}000}\PYG{p}{,} \PYG{n}{seed}\PYG{o}{=}\PYG{l+m+mi}{42}\PYG{p}{,} \PYG{n}{start}\PYG{o}{=}\PYG{k+kc}{None}\PYG{p}{)}\PYG{p}{:}
    \PYG{k}{if} \PYG{n}{start} \PYG{o+ow}{is} \PYG{k+kc}{None}\PYG{p}{:}
        \PYG{n}{start} \PYG{o}{=} \PYG{n}{x}\PYG{o}{.}\PYG{n}{index}\PYG{p}{[}\PYG{o}{\PYGZhy{}}\PYG{l+m+mi}{1}\PYG{p}{]} \PYG{o}{+} \PYG{n}{pd}\PYG{o}{.}\PYG{n}{offsets}\PYG{o}{.}\PYG{n}{BDay}\PYG{p}{(}\PYG{p}{)}
    \PYG{k}{return} \PYG{p}{(}
        \PYG{n}{x}
        \PYG{o}{.}\PYG{n}{sample}\PYG{p}{(}\PYG{n}{n}\PYG{o}{=}\PYG{n}{n}\PYG{p}{,} \PYG{n}{replace}\PYG{o}{=}\PYG{k+kc}{True}\PYG{p}{,} \PYG{n}{random\PYGZus{}state}\PYG{o}{=}\PYG{n}{seed}\PYG{p}{,} \PYG{n}{ignore\PYGZus{}index}\PYG{o}{=}\PYG{k+kc}{True}\PYG{p}{)}
        \PYG{o}{.}\PYG{n}{set\PYGZus{}axis}\PYG{p}{(}\PYG{n}{pd}\PYG{o}{.}\PYG{n}{date\PYGZus{}range}\PYG{p}{(}\PYG{n}{start}\PYG{o}{=}\PYG{n}{start}\PYG{p}{,} \PYG{n}{periods}\PYG{o}{=}\PYG{n}{n}\PYG{p}{)}\PYG{p}{)}
        \PYG{o}{.}\PYG{n}{rename\PYGZus{}axis}\PYG{p}{(}\PYG{n}{index}\PYG{o}{=}\PYG{l+s+s1}{\PYGZsq{}}\PYG{l+s+s1}{Date}\PYG{l+s+s1}{\PYGZsq{}}\PYG{p}{)}
    \PYG{p}{)}    
\end{sphinxVerbatim}

\end{sphinxuseclass}\end{sphinxVerbatimInput}

\end{sphinxuseclass}
\sphinxAtStartPar
The \sphinxcode{\sphinxupquote{get\_samples()}} function calls the \sphinxcode{\sphinxupquote{get\_sample()}} function \sphinxcode{\sphinxupquote{m}} times to simulate \sphinxcode{\sphinxupquote{m}} return histories.
The \sphinxcode{\sphinxupquote{get\_samples()}} function combines these \sphinxcode{\sphinxupquote{m}} return histories into one data frame.

\begin{sphinxuseclass}{cell}\begin{sphinxVerbatimInput}

\begin{sphinxuseclass}{cell_input}
\begin{sphinxVerbatim}[commandchars=\\\{\}]
\PYG{k}{def} \PYG{n+nf}{get\PYGZus{}samples}\PYG{p}{(}\PYG{n}{x}\PYG{p}{,} \PYG{n}{m}\PYG{o}{=}\PYG{l+m+mi}{100}\PYG{p}{,} \PYG{n}{n}\PYG{o}{=}\PYG{l+m+mi}{10\PYGZus{}000}\PYG{p}{,} \PYG{n}{seed}\PYG{o}{=}\PYG{l+m+mi}{42}\PYG{p}{,} \PYG{n}{start}\PYG{o}{=}\PYG{k+kc}{None}\PYG{p}{)}\PYG{p}{:}
    \PYG{k}{return} \PYG{p}{(}
        \PYG{n}{pd}\PYG{o}{.}\PYG{n}{concat}\PYG{p}{(}
            \PYG{n}{objs}\PYG{o}{=}\PYG{p}{[}\PYG{n}{get\PYGZus{}sample}\PYG{p}{(}\PYG{n}{x}\PYG{o}{=}\PYG{n}{x}\PYG{p}{,} \PYG{n}{n}\PYG{o}{=}\PYG{n}{n}\PYG{p}{,} \PYG{n}{seed}\PYG{o}{=}\PYG{n}{seed}\PYG{o}{+}\PYG{n}{i}\PYG{p}{,} \PYG{n}{start}\PYG{o}{=}\PYG{n}{start}\PYG{p}{)} \PYG{k}{for} \PYG{n}{i} \PYG{o+ow}{in} \PYG{n+nb}{range}\PYG{p}{(}\PYG{n}{m}\PYG{p}{)}\PYG{p}{]}\PYG{p}{,}
            \PYG{n}{axis}\PYG{o}{=}\PYG{l+m+mi}{1}\PYG{p}{,}
            \PYG{n}{keys}\PYG{o}{=}\PYG{n+nb}{range}\PYG{p}{(}\PYG{n}{m}\PYG{p}{)}\PYG{p}{,}
            \PYG{n}{names}\PYG{o}{=}\PYG{l+s+s1}{\PYGZsq{}}\PYG{l+s+s1}{Sample}\PYG{l+s+s1}{\PYGZsq{}}
        \PYG{p}{)}
    \PYG{p}{)}
\end{sphinxVerbatim}

\end{sphinxuseclass}\end{sphinxVerbatimInput}

\end{sphinxuseclass}
\sphinxAtStartPar
Next, we use these helper functions to simulate 10,000 return histories of 10,000 trading days each.

\begin{sphinxuseclass}{cell}\begin{sphinxVerbatimInput}

\begin{sphinxuseclass}{cell_input}
\begin{sphinxVerbatim}[commandchars=\\\{\}]
\PYG{n}{mkts} \PYG{o}{=} \PYG{n}{get\PYGZus{}samples}\PYG{p}{(}\PYG{n}{x}\PYG{o}{=}\PYG{n}{mkt}\PYG{p}{,} \PYG{n}{m}\PYG{o}{=}\PYG{l+m+mi}{10\PYGZus{}000}\PYG{p}{,} \PYG{n}{n}\PYG{o}{=}\PYG{l+m+mi}{10\PYGZus{}000}\PYG{p}{)}

\PYG{n}{mkts}
\end{sphinxVerbatim}

\end{sphinxuseclass}\end{sphinxVerbatimInput}
\begin{sphinxVerbatimOutput}

\begin{sphinxuseclass}{cell_output}
\begin{sphinxVerbatim}[commandchars=\\\{\}]
Sample      0     1     2     3     4     5     6     7     8     9     ...  \PYGZbs{}
Date                                                                    ...   
2023\PYGZhy{}01\PYGZhy{}02  0.00  0.01  0.00  0.00  0.00  0.00 \PYGZhy{}0.00  0.00 \PYGZhy{}0.01  0.00  ...   
2023\PYGZhy{}01\PYGZhy{}03  0.01 \PYGZhy{}0.00  0.00  0.01  0.00 \PYGZhy{}0.01 \PYGZhy{}0.01  0.01  0.00 \PYGZhy{}0.01  ...   
2023\PYGZhy{}01\PYGZhy{}04 \PYGZhy{}0.03 \PYGZhy{}0.00 \PYGZhy{}0.01 \PYGZhy{}0.03 \PYGZhy{}0.00  0.02 \PYGZhy{}0.00  0.00 \PYGZhy{}0.01 \PYGZhy{}0.01  ...   
2023\PYGZhy{}01\PYGZhy{}05 \PYGZhy{}0.00 \PYGZhy{}0.00 \PYGZhy{}0.01 \PYGZhy{}0.00  0.01  0.01  0.00 \PYGZhy{}0.02  0.00 \PYGZhy{}0.01  ...   
2023\PYGZhy{}01\PYGZhy{}06 \PYGZhy{}0.00 \PYGZhy{}0.00 \PYGZhy{}0.00 \PYGZhy{}0.00 \PYGZhy{}0.01 \PYGZhy{}0.00  0.01  0.01 \PYGZhy{}0.00  0.01  ...   
...          ...   ...   ...   ...   ...   ...   ...   ...   ...   ...  ...   
2050\PYGZhy{}05\PYGZhy{}15  0.02 \PYGZhy{}0.01  0.01  0.00 \PYGZhy{}0.00 \PYGZhy{}0.00 \PYGZhy{}0.00  0.02  0.00 \PYGZhy{}0.00  ...   
2050\PYGZhy{}05\PYGZhy{}16  0.01 \PYGZhy{}0.00 \PYGZhy{}0.00  0.00  0.00 \PYGZhy{}0.01 \PYGZhy{}0.01  0.02  0.01  0.00  ...   
2050\PYGZhy{}05\PYGZhy{}17 \PYGZhy{}0.02  0.01  0.00  0.01  0.00  0.01  0.00  0.00  0.00  0.00  ...   
2050\PYGZhy{}05\PYGZhy{}18 \PYGZhy{}0.00  0.00  0.03  0.01  0.00  0.00 \PYGZhy{}0.00  0.00  0.00  0.00  ...   
2050\PYGZhy{}05\PYGZhy{}19  0.01 \PYGZhy{}0.02  0.00  0.01  0.00 \PYGZhy{}0.00  0.01 \PYGZhy{}0.01  0.00 \PYGZhy{}0.00  ...   

Sample      9990  9991  9992  9993  9994  9995  9996  9997  9998  9999  
Date                                                                    
2023\PYGZhy{}01\PYGZhy{}02  0.00 \PYGZhy{}0.00 \PYGZhy{}0.00  0.00  0.00  0.01 \PYGZhy{}0.00 \PYGZhy{}0.01 \PYGZhy{}0.00  0.01  
2023\PYGZhy{}01\PYGZhy{}03  0.00  0.01  0.01  0.01 \PYGZhy{}0.00  0.00  0.00 \PYGZhy{}0.01 \PYGZhy{}0.00  0.02  
2023\PYGZhy{}01\PYGZhy{}04 \PYGZhy{}0.01  0.01 \PYGZhy{}0.01 \PYGZhy{}0.01  0.01  0.00  0.01  0.01  0.00  0.02  
2023\PYGZhy{}01\PYGZhy{}05  0.01  0.02  0.03 \PYGZhy{}0.00 \PYGZhy{}0.00  0.01  0.01 \PYGZhy{}0.00  0.02  0.00  
2023\PYGZhy{}01\PYGZhy{}06  0.01  0.02  0.01 \PYGZhy{}0.01  0.02  0.01 \PYGZhy{}0.00  0.00  0.00  0.02  
...          ...   ...   ...   ...   ...   ...   ...   ...   ...   ...  
2050\PYGZhy{}05\PYGZhy{}15 \PYGZhy{}0.00 \PYGZhy{}0.01  0.02  0.00  0.00 \PYGZhy{}0.01 \PYGZhy{}0.02  0.01 \PYGZhy{}0.01 \PYGZhy{}0.04  
2050\PYGZhy{}05\PYGZhy{}16  0.00 \PYGZhy{}0.00  0.00 \PYGZhy{}0.00 \PYGZhy{}0.01 \PYGZhy{}0.00  0.01  0.00 \PYGZhy{}0.02  0.00  
2050\PYGZhy{}05\PYGZhy{}17  0.00 \PYGZhy{}0.00  0.00  0.00 \PYGZhy{}0.00  0.01 \PYGZhy{}0.00  0.01 \PYGZhy{}0.02 \PYGZhy{}0.00  
2050\PYGZhy{}05\PYGZhy{}18 \PYGZhy{}0.00  0.02  0.01 \PYGZhy{}0.00  0.00  0.00  0.00  0.01  0.00 \PYGZhy{}0.00  
2050\PYGZhy{}05\PYGZhy{}19  0.00 \PYGZhy{}0.00 \PYGZhy{}0.01 \PYGZhy{}0.00 \PYGZhy{}0.00 \PYGZhy{}0.00  0.00  0.01  0.01  0.01  

[10000 rows x 10000 columns]
\end{sphinxVerbatim}

\end{sphinxuseclass}\end{sphinxVerbatimOutput}

\end{sphinxuseclass}
\sphinxAtStartPar
We compound daily returns \$R\_t\$ to find future wealth \$W\_T\$, so \$W\_T = W\_0 \textbackslash{}times (1 + R\_1) \textbackslash{}times (1 + R\_2) \textbackslash{}times \textbackslash{}cdots \textbackslash{}times (1 + R\_T)\$.
We use the \sphinxcode{\sphinxupquote{.cumprod()}} method to compound the daily returns in data frame \sphinxcode{\sphinxupquote{mkts}}.

\begin{sphinxuseclass}{cell}\begin{sphinxVerbatimInput}

\begin{sphinxuseclass}{cell_input}
\begin{sphinxVerbatim}[commandchars=\\\{\}]
\PYG{n}{W\PYGZus{}0} \PYG{o}{=} \PYG{l+m+mi}{1\PYGZus{}000\PYGZus{}000}
\PYG{n}{W\PYGZus{}t} \PYG{o}{=} \PYG{n}{W\PYGZus{}0} \PYG{o}{*} \PYG{n}{mkts}\PYG{o}{.}\PYG{n}{add}\PYG{p}{(}\PYG{l+m+mi}{1}\PYG{p}{)}\PYG{o}{.}\PYG{n}{cumprod}\PYG{p}{(}\PYG{p}{)}
\end{sphinxVerbatim}

\end{sphinxuseclass}\end{sphinxVerbatimInput}

\end{sphinxuseclass}
\sphinxAtStartPar
We visualize terminal wealth \$W\_T\$ with a cumulative distribution plot.

\begin{sphinxuseclass}{cell}\begin{sphinxVerbatimInput}

\begin{sphinxuseclass}{cell_input}
\begin{sphinxVerbatim}[commandchars=\\\{\}]
\PYG{n}{W\PYGZus{}t}\PYG{o}{.}\PYG{n}{iloc}\PYG{p}{[}\PYG{o}{\PYGZhy{}}\PYG{l+m+mi}{1}\PYG{p}{]}\PYG{o}{.}\PYG{n}{plot}\PYG{p}{(}\PYG{n}{kind}\PYG{o}{=}\PYG{l+s+s1}{\PYGZsq{}}\PYG{l+s+s1}{hist}\PYG{l+s+s1}{\PYGZsq{}}\PYG{p}{,} \PYG{n}{density}\PYG{o}{=}\PYG{k+kc}{True}\PYG{p}{,} \PYG{n}{cumulative}\PYG{o}{=}\PYG{k+kc}{True}\PYG{p}{,} \PYG{n}{bins}\PYG{o}{=}\PYG{l+m+mi}{1\PYGZus{}000}\PYG{p}{)}
\PYG{n}{plt}\PYG{o}{.}\PYG{n}{semilogx}\PYG{p}{(}\PYG{p}{)}
\PYG{n}{plt}\PYG{o}{.}\PYG{n}{xlabel}\PYG{p}{(}\PYG{l+s+sa}{r}\PYG{l+s+s1}{\PYGZsq{}}\PYG{l+s+s1}{\PYGZdl{}W\PYGZus{}T\PYGZdl{} where \PYGZdl{}W\PYGZus{}0 = \PYGZdl{}}\PYG{l+s+s1}{\PYGZsq{}} \PYG{o}{+} \PYG{l+s+sa}{f}\PYG{l+s+s1}{\PYGZsq{}}\PYG{l+s+s1}{\PYGZbs{}}\PYG{l+s+s1}{\PYGZdl{}}\PYG{l+s+si}{\PYGZob{}}\PYG{n}{W\PYGZus{}0}\PYG{l+s+si}{:}\PYG{l+s+s1}{,.0f}\PYG{l+s+si}{\PYGZcb{}}\PYG{l+s+s1}{\PYGZsq{}}\PYG{p}{)}
\PYG{n}{plt}\PYG{o}{.}\PYG{n}{title}\PYG{p}{(}
    \PYG{l+s+sa}{r}\PYG{l+s+s1}{\PYGZsq{}}\PYG{l+s+s1}{Cumulative Distribution of \PYGZdl{}W\PYGZus{}T\PYGZdl{}}\PYG{l+s+s1}{\PYGZsq{}} \PYG{o}{+} 
    \PYG{l+s+sa}{f}\PYG{l+s+s1}{\PYGZsq{}}\PYG{l+s+se}{\PYGZbs{}n}\PYG{l+s+s1}{after }\PYG{l+s+si}{\PYGZob{}}\PYG{n}{W\PYGZus{}t}\PYG{o}{.}\PYG{n}{shape}\PYG{p}{[}\PYG{l+m+mi}{0}\PYG{p}{]}\PYG{l+s+si}{:}\PYG{l+s+s1}{,.0f}\PYG{l+s+si}{\PYGZcb{}}\PYG{l+s+s1}{ Trading Days}\PYG{l+s+s1}{\PYGZsq{}}
\PYG{p}{)}
\PYG{n}{plt}\PYG{o}{.}\PYG{n}{show}\PYG{p}{(}\PYG{p}{)}
\end{sphinxVerbatim}

\end{sphinxuseclass}\end{sphinxVerbatimInput}
\begin{sphinxVerbatimOutput}

\begin{sphinxuseclass}{cell_output}
\noindent\sphinxincludegraphics{{0211c97349d8ed9941ef650fa751fb416ffaa07a7972ee92543d9abd864642d2}.png}

\end{sphinxuseclass}\end{sphinxVerbatimOutput}

\end{sphinxuseclass}
\sphinxAtStartPar
The plot above shows the cumulative distribution of \$W\_T\$, suggesting we expect \$W\_T\$ greater than about \$17 million for about 75\% of samples.

\sphinxAtStartPar
The plot above may be difficult to read and interpret because \$W\_T\$ has large outliers.
The \sphinxcode{\sphinxupquote{.describe()}} method provides few salient values of \$W\_T\$.
We convert \$W\_T\$ from dollars to millions of dollars with \sphinxcode{\sphinxupquote{.div(1\_000\_000)}}.

\begin{sphinxuseclass}{cell}\begin{sphinxVerbatimInput}

\begin{sphinxuseclass}{cell_input}
\begin{sphinxVerbatim}[commandchars=\\\{\}]
\PYG{n}{W\PYGZus{}t}\PYG{o}{.}\PYG{n}{iloc}\PYG{p}{[}\PYG{o}{\PYGZhy{}}\PYG{l+m+mi}{1}\PYG{p}{]}\PYG{o}{.}\PYG{n}{div}\PYG{p}{(}\PYG{l+m+mi}{1\PYGZus{}000\PYGZus{}000}\PYG{p}{)}\PYG{o}{.}\PYG{n}{describe}\PYG{p}{(}\PYG{p}{)}
\end{sphinxVerbatim}

\end{sphinxuseclass}\end{sphinxVerbatimInput}
\begin{sphinxVerbatimOutput}

\begin{sphinxuseclass}{cell_output}
\begin{sphinxVerbatim}[commandchars=\\\{\}]
count   10000.00
mean       64.75
std       101.61
min         0.56
25\PYGZpc{}        17.56
50\PYGZpc{}        36.11
75\PYGZpc{}        74.11
max      3330.53
Name: 2050\PYGZhy{}05\PYGZhy{}19 00:00:00, dtype: float64
\end{sphinxVerbatim}

\end{sphinxuseclass}\end{sphinxVerbatimOutput}

\end{sphinxuseclass}

\subsubsection{Repeat the exercise above but add end\sphinxhyphen{}of\sphinxhyphen{}month investments \$C\_t\$}
\label{\detokenize{herron_05_practice_03:repeat-the-exercise-above-but-add-end-of-month-investments-c-t}}
\sphinxAtStartPar
We can use the same data frame \sphinxcode{\sphinxupquote{mkts}} of simulated market returns.
However, need to consider end\sphinxhyphen{}of\sphinxhyphen{}month investments of \$C\_t\$.
The easiest approach is to aggregate the daily market returns in \sphinxcode{\sphinxupquote{mkts}} to monthly market returns in data frame \sphinxcode{\sphinxupquote{mkts\_m}}.
The last month in \sphinxcode{\sphinxupquote{mkts\_m}} is not complete (i.e., fewer than about 21 trading days), so we drop is with \sphinxcode{\sphinxupquote{.iloc{[}:\sphinxhyphen{}1{]}}}.

\begin{sphinxuseclass}{cell}\begin{sphinxVerbatimInput}

\begin{sphinxuseclass}{cell_input}
\begin{sphinxVerbatim}[commandchars=\\\{\}]
\PYG{n}{mkts\PYGZus{}m} \PYG{o}{=} \PYG{p}{(}
    \PYG{n}{mkts}
    \PYG{o}{.}\PYG{n}{add}\PYG{p}{(}\PYG{l+m+mi}{1}\PYG{p}{)}
    \PYG{o}{.}\PYG{n}{resample}\PYG{p}{(}\PYG{n}{rule}\PYG{o}{=}\PYG{l+s+s1}{\PYGZsq{}}\PYG{l+s+s1}{M}\PYG{l+s+s1}{\PYGZsq{}}\PYG{p}{,} \PYG{n}{kind}\PYG{o}{=}\PYG{l+s+s1}{\PYGZsq{}}\PYG{l+s+s1}{period}\PYG{l+s+s1}{\PYGZsq{}}\PYG{p}{)}
    \PYG{o}{.}\PYG{n}{prod}\PYG{p}{(}\PYG{p}{)}
    \PYG{o}{.}\PYG{n}{sub}\PYG{p}{(}\PYG{l+m+mi}{1}\PYG{p}{)}
    \PYG{o}{.}\PYG{n}{iloc}\PYG{p}{[}\PYG{p}{:}\PYG{o}{\PYGZhy{}}\PYG{l+m+mi}{1}\PYG{p}{]}
\PYG{p}{)}
\end{sphinxVerbatim}

\end{sphinxuseclass}\end{sphinxVerbatimInput}

\end{sphinxuseclass}
\sphinxAtStartPar
The wealth at time \$t\$ is \$W\_t\$ and depends on:
\begin{enumerate}
\sphinxsetlistlabels{\arabic}{enumi}{enumii}{}{.}%
\item {} 
\sphinxAtStartPar
The wealth at the end of the previous month \$W\_\{t\sphinxhyphen{}1\}\$

\item {} 
\sphinxAtStartPar
The return over the previous month \$R\_t\$ (recall we label returns by their right edge)

\item {} 
\sphinxAtStartPar
The end\sphinxhyphen{}of\sphinxhyphen{}month investment \$C\_t\$

\end{enumerate}

\sphinxAtStartPar
Putting it all togther: \$W\_t = W\_\{t\sphinxhyphen{}1\} \textbackslash{}times (1 + R\_t) + C\_t\$

\sphinxAtStartPar
We have to loop over the monthly returns in \sphinxcode{\sphinxupquote{mkts\_m}} because we have to combine compounded returns and cash flows.
The \sphinxcode{\sphinxupquote{iterrows()}} methods provides an easy way to iterate (loop) over the rows in \sphinxcode{\sphinxupquote{mkts\_m}}.

\begin{sphinxuseclass}{cell}\begin{sphinxVerbatimInput}

\begin{sphinxuseclass}{cell_input}
\begin{sphinxVerbatim}[commandchars=\\\{\}]
\PYG{n}{C\PYGZus{}t} \PYG{o}{=} \PYG{l+m+mi}{1\PYGZus{}000}
\PYG{n}{W\PYGZus{}0} \PYG{o}{=} \PYG{l+m+mi}{1\PYGZus{}000\PYGZus{}000}
\PYG{n}{W\PYGZus{}last} \PYG{o}{=} \PYG{n}{W\PYGZus{}0}
\PYG{n}{W\PYGZus{}t} \PYG{o}{=} \PYG{p}{[}\PYG{p}{]}

\PYG{k}{for} \PYG{n}{d}\PYG{p}{,} \PYG{n}{m} \PYG{o+ow}{in} \PYG{n}{mkts\PYGZus{}m}\PYG{o}{.}\PYG{n}{iterrows}\PYG{p}{(}\PYG{p}{)}\PYG{p}{:}
    \PYG{n}{W\PYGZus{}last} \PYG{o}{=} \PYG{p}{(}\PYG{l+m+mi}{1} \PYG{o}{+} \PYG{n}{m}\PYG{p}{)} \PYG{o}{*} \PYG{n}{W\PYGZus{}last} \PYG{o}{+} \PYG{n}{C\PYGZus{}t}
    \PYG{n}{W\PYGZus{}t}\PYG{o}{.}\PYG{n}{append}\PYG{p}{(}\PYG{n}{W\PYGZus{}last}\PYG{p}{)}

\PYG{n}{W\PYGZus{}t} \PYG{o}{=} \PYG{n}{pd}\PYG{o}{.}\PYG{n}{concat}\PYG{p}{(}\PYG{n}{objs}\PYG{o}{=}\PYG{n}{W\PYGZus{}t}\PYG{p}{,} \PYG{n}{axis}\PYG{o}{=}\PYG{l+m+mi}{1}\PYG{p}{,} \PYG{n}{keys}\PYG{o}{=}\PYG{n}{mkts\PYGZus{}m}\PYG{o}{.}\PYG{n}{index}\PYG{p}{)}\PYG{o}{.}\PYG{n}{transpose}\PYG{p}{(}\PYG{p}{)}
\end{sphinxVerbatim}

\end{sphinxuseclass}\end{sphinxVerbatimInput}

\end{sphinxuseclass}
\sphinxAtStartPar
We repeat the cumulative distribution of wealth plot and descriptive statistics from above.

\begin{sphinxuseclass}{cell}\begin{sphinxVerbatimInput}

\begin{sphinxuseclass}{cell_input}
\begin{sphinxVerbatim}[commandchars=\\\{\}]
\PYG{n}{W\PYGZus{}t}\PYG{o}{.}\PYG{n}{iloc}\PYG{p}{[}\PYG{o}{\PYGZhy{}}\PYG{l+m+mi}{1}\PYG{p}{]}\PYG{o}{.}\PYG{n}{plot}\PYG{p}{(}\PYG{n}{kind}\PYG{o}{=}\PYG{l+s+s1}{\PYGZsq{}}\PYG{l+s+s1}{hist}\PYG{l+s+s1}{\PYGZsq{}}\PYG{p}{,} \PYG{n}{density}\PYG{o}{=}\PYG{k+kc}{True}\PYG{p}{,} \PYG{n}{cumulative}\PYG{o}{=}\PYG{k+kc}{True}\PYG{p}{,} \PYG{n}{bins}\PYG{o}{=}\PYG{l+m+mi}{1\PYGZus{}000}\PYG{p}{)}
\PYG{n}{plt}\PYG{o}{.}\PYG{n}{semilogx}\PYG{p}{(}\PYG{p}{)}
\PYG{n}{plt}\PYG{o}{.}\PYG{n}{xlabel}\PYG{p}{(}
    \PYG{l+s+sa}{r}\PYG{l+s+s1}{\PYGZsq{}}\PYG{l+s+s1}{\PYGZdl{}W\PYGZus{}T\PYGZdl{} where \PYGZdl{}W\PYGZus{}0 = \PYGZdl{}}\PYG{l+s+s1}{\PYGZsq{}} \PYG{o}{+} \PYG{l+s+sa}{f}\PYG{l+s+s1}{\PYGZsq{}}\PYG{l+s+s1}{\PYGZbs{}}\PYG{l+s+s1}{\PYGZdl{}}\PYG{l+s+si}{\PYGZob{}}\PYG{n}{W\PYGZus{}0}\PYG{l+s+si}{:}\PYG{l+s+s1}{,.0f}\PYG{l+s+si}{\PYGZcb{}}\PYG{l+s+s1}{\PYGZsq{}} \PYG{o}{+} 
    \PYG{l+s+sa}{r}\PYG{l+s+s1}{\PYGZsq{}}\PYG{l+s+s1}{ and  \PYGZdl{}C\PYGZus{}t = \PYGZdl{}}\PYG{l+s+s1}{\PYGZsq{}} \PYG{o}{+} \PYG{l+s+sa}{f}\PYG{l+s+s1}{\PYGZsq{}}\PYG{l+s+s1}{\PYGZbs{}}\PYG{l+s+s1}{\PYGZdl{}}\PYG{l+s+si}{\PYGZob{}}\PYG{n}{C\PYGZus{}t}\PYG{l+s+si}{:}\PYG{l+s+s1}{,.0f}\PYG{l+s+si}{\PYGZcb{}}\PYG{l+s+s1}{\PYGZsq{}}
\PYG{p}{)}
\PYG{n}{plt}\PYG{o}{.}\PYG{n}{title}\PYG{p}{(}
    \PYG{l+s+sa}{r}\PYG{l+s+s1}{\PYGZsq{}}\PYG{l+s+s1}{Cumulative Distribution of \PYGZdl{}W\PYGZus{}T\PYGZdl{}}\PYG{l+s+s1}{\PYGZsq{}} \PYG{o}{+} 
    \PYG{l+s+sa}{f}\PYG{l+s+s1}{\PYGZsq{}}\PYG{l+s+se}{\PYGZbs{}n}\PYG{l+s+s1}{after }\PYG{l+s+si}{\PYGZob{}}\PYG{n}{W\PYGZus{}t}\PYG{o}{.}\PYG{n}{shape}\PYG{p}{[}\PYG{l+m+mi}{0}\PYG{p}{]}\PYG{l+s+si}{:}\PYG{l+s+s1}{,.0f}\PYG{l+s+si}{\PYGZcb{}}\PYG{l+s+s1}{ Trading Months}\PYG{l+s+s1}{\PYGZsq{}}
\PYG{p}{)}
\PYG{n}{plt}\PYG{o}{.}\PYG{n}{show}\PYG{p}{(}\PYG{p}{)}
\end{sphinxVerbatim}

\end{sphinxuseclass}\end{sphinxVerbatimInput}
\begin{sphinxVerbatimOutput}

\begin{sphinxuseclass}{cell_output}
\noindent\sphinxincludegraphics{{6d78aab7a506358bd5a775b38f3ef6930893fe48e082ff3b0dfee6a188e7bc80}.png}

\end{sphinxuseclass}\end{sphinxVerbatimOutput}

\end{sphinxuseclass}
\begin{sphinxuseclass}{cell}\begin{sphinxVerbatimInput}

\begin{sphinxuseclass}{cell_input}
\begin{sphinxVerbatim}[commandchars=\\\{\}]
\PYG{n}{W\PYGZus{}t}\PYG{o}{.}\PYG{n}{iloc}\PYG{p}{[}\PYG{o}{\PYGZhy{}}\PYG{l+m+mi}{1}\PYG{p}{]}\PYG{o}{.}\PYG{n}{div}\PYG{p}{(}\PYG{l+m+mi}{1\PYGZus{}000\PYGZus{}000}\PYG{p}{)}\PYG{o}{.}\PYG{n}{describe}\PYG{p}{(}\PYG{p}{)}
\end{sphinxVerbatim}

\end{sphinxuseclass}\end{sphinxVerbatimInput}
\begin{sphinxVerbatimOutput}

\begin{sphinxuseclass}{cell_output}
\begin{sphinxVerbatim}[commandchars=\\\{\}]
count   10000.00
mean       69.20
std       106.22
min         0.94
25\PYGZpc{}        19.59
50\PYGZpc{}        39.27
75\PYGZpc{}        79.69
max      3467.92
Name: 2050\PYGZhy{}04, dtype: float64
\end{sphinxVerbatim}

\end{sphinxuseclass}\end{sphinxVerbatimOutput}

\end{sphinxuseclass}
\sphinxAtStartPar
The plot above shows the cumulative distribution of \$W\_T\$, suggesting we expect \$W\_T\$ greater than about \textbackslash{}\$20 million for about 75\% of samples.
Note, even though we deposit \$1,000 per month for almost 40 years, we do not gain much wealth over the previous example with only a lump sum!
Start saving early!

\sphinxstepscope


\section{Herron Topic 5 \sphinxhyphen{} Practice (Wednesday 11:45 AM, Section 4)}
\label{\detokenize{herron_05_practice_04:herron-topic-5-practice-wednesday-11-45-am-section-4}}\label{\detokenize{herron_05_practice_04::doc}}

\subsection{Announcements}
\label{\detokenize{herron_05_practice_04:announcements}}\begin{itemize}
\item {} 
\sphinxAtStartPar
Due dates
\begin{itemize}
\item {} 
\sphinxAtStartPar
Team project 2 is due by 11:59 PM on Wednesday, 4/26

\item {} 
\sphinxAtStartPar
Teammates review 2 is due by 11:59 PM on Wednesday, 4/26

\item {} 
\sphinxAtStartPar
30,000 DataCamp XP are due by 11:59 PM on \sphinxstyleemphasis{Friday, 4/28}

\item {} 
\sphinxAtStartPar
TRACE course evaluations are due by 11:59 PM on \sphinxstyleemphasis{Friday, 4/28}

\end{itemize}

\item {} 
\sphinxAtStartPar
\sphinxstyleemphasis{\sphinxstylestrong{TRACE course evaluations are optional (and anonymous), but I value your feedback!}}
\begin{itemize}
\item {} 
\sphinxAtStartPar
I always find student feedback useful, and I read and re\sphinxhyphen{}read it!

\item {} 
\sphinxAtStartPar
I tweak my courses every semester based on student feedback!

\item {} 
\sphinxAtStartPar
Your feedback does not have to be perfect to be useful!

\item {} 
\sphinxAtStartPar
If you had a magic wand, what would you change about my course and approach?

\end{itemize}

\item {} 
\sphinxAtStartPar
This week and next, we can discuss whatever topics you want, including simulations and team project 2

\end{itemize}


\subsection{Practice}
\label{\detokenize{herron_05_practice_04:practice}}
\begin{sphinxuseclass}{cell}\begin{sphinxVerbatimInput}

\begin{sphinxuseclass}{cell_input}
\begin{sphinxVerbatim}[commandchars=\\\{\}]
\PYG{k+kn}{import} \PYG{n+nn}{matplotlib}\PYG{n+nn}{.}\PYG{n+nn}{pyplot} \PYG{k}{as} \PYG{n+nn}{plt}
\PYG{k+kn}{import} \PYG{n+nn}{numpy} \PYG{k}{as} \PYG{n+nn}{np}
\PYG{k+kn}{import} \PYG{n+nn}{pandas} \PYG{k}{as} \PYG{n+nn}{pd}
\end{sphinxVerbatim}

\end{sphinxuseclass}\end{sphinxVerbatimInput}

\end{sphinxuseclass}
\begin{sphinxuseclass}{cell}\begin{sphinxVerbatimInput}

\begin{sphinxuseclass}{cell_input}
\begin{sphinxVerbatim}[commandchars=\\\{\}]
\PYG{o}{\PYGZpc{}}\PYG{k}{config} InlineBackend.figure\PYGZus{}format = \PYGZsq{}retina\PYGZsq{}
\PYG{o}{\PYGZpc{}}\PYG{k}{precision} 2
\PYG{n}{pd}\PYG{o}{.}\PYG{n}{options}\PYG{o}{.}\PYG{n}{display}\PYG{o}{.}\PYG{n}{float\PYGZus{}format} \PYG{o}{=} \PYG{l+s+s1}{\PYGZsq{}}\PYG{l+s+si}{\PYGZob{}:.2f\PYGZcb{}}\PYG{l+s+s1}{\PYGZsq{}}\PYG{o}{.}\PYG{n}{format}
\end{sphinxVerbatim}

\end{sphinxuseclass}\end{sphinxVerbatimInput}

\end{sphinxuseclass}
\begin{sphinxuseclass}{cell}\begin{sphinxVerbatimInput}

\begin{sphinxuseclass}{cell_input}
\begin{sphinxVerbatim}[commandchars=\\\{\}]
\PYG{k+kn}{import} \PYG{n+nn}{yfinance} \PYG{k}{as} \PYG{n+nn}{yf}
\PYG{k+kn}{import} \PYG{n+nn}{pandas\PYGZus{}datareader} \PYG{k}{as} \PYG{n+nn}{pdr}
\PYG{k+kn}{import} \PYG{n+nn}{requests\PYGZus{}cache}
\PYG{n}{session} \PYG{o}{=} \PYG{n}{requests\PYGZus{}cache}\PYG{o}{.}\PYG{n}{CachedSession}\PYG{p}{(}\PYG{p}{)}
\end{sphinxVerbatim}

\end{sphinxuseclass}\end{sphinxVerbatimInput}

\end{sphinxuseclass}

\subsubsection{Estimate \$\textbackslash{}pi\$ by simulating darts thrown at a dart board}
\label{\detokenize{herron_05_practice_04:estimate-pi-by-simulating-darts-thrown-at-a-dart-board}}
\sphinxAtStartPar
\sphinxstyleemphasis{Hints:}
Select random \$x\$s and \$y\$s such that \$\sphinxhyphen{}r \textbackslash{}leq x \textbackslash{}leq +r\$ and \$\sphinxhyphen{}r \textbackslash{}leq x \textbackslash{}leq +r\$.
Darts are on the board if \$x\textasciicircum{}2 + y\textasciicircum{}2 \textbackslash{}leq r\textasciicircum{}2\$.
The area of the circlular board is \$\textbackslash{}pi r\textasciicircum{}2\$, and the area of square around the board is \$(2r)\textasciicircum{}2 = 4r\textasciicircum{}2\$.
The fraction \$f\$ of darts on the board is the same as the ratio of circle area to square area, so \$f = \textbackslash{}frac\{\textbackslash{}pi r\textasciicircum{}2\}\{4 r\textasciicircum{}2\}\$.

\sphinxAtStartPar
First we throw darts at the board.
Darts with \$x\textasciicircum{}2 + y\textasciicircum{}2 \textbackslash{}leq r\textasciicircum{}2\$ are on the board.

\begin{sphinxuseclass}{cell}\begin{sphinxVerbatimInput}

\begin{sphinxuseclass}{cell_input}
\begin{sphinxVerbatim}[commandchars=\\\{\}]
\PYG{k}{def} \PYG{n+nf}{throw\PYGZus{}darts}\PYG{p}{(}\PYG{n}{n}\PYG{o}{=}\PYG{l+m+mi}{10\PYGZus{}000}\PYG{p}{,} \PYG{n}{r}\PYG{o}{=}\PYG{l+m+mi}{1}\PYG{p}{,} \PYG{n}{seed}\PYG{o}{=}\PYG{l+m+mi}{42}\PYG{p}{)}\PYG{p}{:}
    \PYG{n}{np}\PYG{o}{.}\PYG{n}{random}\PYG{o}{.}\PYG{n}{seed}\PYG{p}{(}\PYG{n}{seed}\PYG{p}{)}
    \PYG{k}{return} \PYG{p}{(}
        \PYG{n}{pd}\PYG{o}{.}\PYG{n}{DataFrame}\PYG{p}{(}
            \PYG{n}{data}\PYG{o}{=}\PYG{n}{np}\PYG{o}{.}\PYG{n}{random}\PYG{o}{.}\PYG{n}{uniform}\PYG{p}{(}\PYG{n}{low}\PYG{o}{=}\PYG{o}{\PYGZhy{}}\PYG{n}{r}\PYG{p}{,} \PYG{n}{high}\PYG{o}{=}\PYG{n}{r}\PYG{p}{,} \PYG{n}{size}\PYG{o}{=}\PYG{l+m+mi}{2}\PYG{o}{*}\PYG{n}{n}\PYG{p}{)}\PYG{o}{.}\PYG{n}{reshape}\PYG{p}{(}\PYG{n}{n}\PYG{p}{,} \PYG{l+m+mi}{2}\PYG{p}{)}\PYG{p}{,} 
            \PYG{n}{columns}\PYG{o}{=}\PYG{p}{[}\PYG{l+s+s1}{\PYGZsq{}}\PYG{l+s+s1}{x}\PYG{l+s+s1}{\PYGZsq{}}\PYG{p}{,} \PYG{l+s+s1}{\PYGZsq{}}\PYG{l+s+s1}{y}\PYG{l+s+s1}{\PYGZsq{}}\PYG{p}{]}
        \PYG{p}{)}
        \PYG{o}{.}\PYG{n}{assign}\PYG{p}{(}\PYG{n}{board}\PYG{o}{=}\PYG{k}{lambda} \PYG{n}{x}\PYG{p}{:} \PYG{n}{x}\PYG{p}{[}\PYG{l+s+s1}{\PYGZsq{}}\PYG{l+s+s1}{x}\PYG{l+s+s1}{\PYGZsq{}}\PYG{p}{]}\PYG{o}{*}\PYG{o}{*}\PYG{l+m+mi}{2} \PYG{o}{+} \PYG{n}{x}\PYG{p}{[}\PYG{l+s+s1}{\PYGZsq{}}\PYG{l+s+s1}{y}\PYG{l+s+s1}{\PYGZsq{}}\PYG{p}{]}\PYG{o}{*}\PYG{o}{*}\PYG{l+m+mi}{2} \PYG{o}{\PYGZlt{}}\PYG{o}{=} \PYG{n}{r}\PYG{o}{*}\PYG{o}{*}\PYG{l+m+mi}{2}\PYG{p}{)}
        \PYG{o}{.}\PYG{n}{rename\PYGZus{}axis}\PYG{p}{(}\PYG{n}{index}\PYG{o}{=}\PYG{l+s+s1}{\PYGZsq{}}\PYG{l+s+s1}{n}\PYG{l+s+s1}{\PYGZsq{}}\PYG{p}{,} \PYG{n}{columns}\PYG{o}{=}\PYG{l+s+s1}{\PYGZsq{}}\PYG{l+s+s1}{Variable}\PYG{l+s+s1}{\PYGZsq{}}\PYG{p}{)}
    \PYG{p}{)}
\end{sphinxVerbatim}

\end{sphinxuseclass}\end{sphinxVerbatimInput}

\end{sphinxuseclass}
\begin{sphinxuseclass}{cell}\begin{sphinxVerbatimInput}

\begin{sphinxuseclass}{cell_input}
\begin{sphinxVerbatim}[commandchars=\\\{\}]
\PYG{n}{throw\PYGZus{}darts}\PYG{p}{(}\PYG{p}{)}
\end{sphinxVerbatim}

\end{sphinxuseclass}\end{sphinxVerbatimInput}
\begin{sphinxVerbatimOutput}

\begin{sphinxuseclass}{cell_output}
\begin{sphinxVerbatim}[commandchars=\\\{\}]
Variable     x     y  board
n                          
0        \PYGZhy{}0.25  0.90   True
1         0.46  0.20   True
2        \PYGZhy{}0.69 \PYGZhy{}0.69   True
3        \PYGZhy{}0.88  0.73  False
4         0.20  0.42   True
...        ...   ...    ...
9995      0.15  0.52   True
9996     \PYGZhy{}0.82 \PYGZhy{}0.01   True
9997      0.80  0.75  False
9998     \PYGZhy{}0.91 \PYGZhy{}0.39   True
9999     \PYGZhy{}0.11 \PYGZhy{}0.66   True

[10000 rows x 3 columns]
\end{sphinxVerbatim}

\end{sphinxuseclass}\end{sphinxVerbatimOutput}

\end{sphinxuseclass}
\sphinxAtStartPar
Next, we visualize these darts with a scatter plot.
Seaborn’s \sphinxcode{\sphinxupquote{scatterplot()}} helps color darts by location (i.e., on or off board).
The \sphinxcode{\sphinxupquote{.pipe()}} method lets us send the output of the \sphinxcode{\sphinxupquote{.assign()}} method to \sphinxcode{\sphinxupquote{sns.scatterplot()}} without assigning a temporary data frame.

\begin{sphinxuseclass}{cell}\begin{sphinxVerbatimInput}

\begin{sphinxuseclass}{cell_input}
\begin{sphinxVerbatim}[commandchars=\\\{\}]
\PYG{k+kn}{import} \PYG{n+nn}{seaborn} \PYG{k}{as} \PYG{n+nn}{sns}
\end{sphinxVerbatim}

\end{sphinxuseclass}\end{sphinxVerbatimInput}

\end{sphinxuseclass}
\begin{sphinxuseclass}{cell}\begin{sphinxVerbatimInput}

\begin{sphinxuseclass}{cell_input}
\begin{sphinxVerbatim}[commandchars=\\\{\}]
\PYG{p}{(}
    \PYG{n}{throw\PYGZus{}darts}\PYG{p}{(}\PYG{p}{)}
    \PYG{o}{.}\PYG{n}{assign}\PYG{p}{(}\PYG{n}{Location}\PYG{o}{=}\PYG{k}{lambda} \PYG{n}{x}\PYG{p}{:} \PYG{n}{np}\PYG{o}{.}\PYG{n}{where}\PYG{p}{(}\PYG{n}{x}\PYG{p}{[}\PYG{l+s+s1}{\PYGZsq{}}\PYG{l+s+s1}{board}\PYG{l+s+s1}{\PYGZsq{}}\PYG{p}{]}\PYG{p}{,} \PYG{l+s+s1}{\PYGZsq{}}\PYG{l+s+s1}{On Board}\PYG{l+s+s1}{\PYGZsq{}}\PYG{p}{,} \PYG{l+s+s1}{\PYGZsq{}}\PYG{l+s+s1}{Off Board}\PYG{l+s+s1}{\PYGZsq{}}\PYG{p}{)}\PYG{p}{)}
    \PYG{o}{.}\PYG{n}{pipe}\PYG{p}{(}\PYG{k}{lambda} \PYG{n}{x}\PYG{p}{:} \PYG{n}{sns}\PYG{o}{.}\PYG{n}{scatterplot}\PYG{p}{(}\PYG{n}{x}\PYG{o}{=}\PYG{l+s+s1}{\PYGZsq{}}\PYG{l+s+s1}{x}\PYG{l+s+s1}{\PYGZsq{}}\PYG{p}{,} \PYG{n}{y}\PYG{o}{=}\PYG{l+s+s1}{\PYGZsq{}}\PYG{l+s+s1}{y}\PYG{l+s+s1}{\PYGZsq{}}\PYG{p}{,} \PYG{n}{data}\PYG{o}{=}\PYG{n}{x}\PYG{p}{,} \PYG{n}{hue}\PYG{o}{=}\PYG{l+s+s1}{\PYGZsq{}}\PYG{l+s+s1}{Location}\PYG{l+s+s1}{\PYGZsq{}}\PYG{p}{)}\PYG{p}{)}
\PYG{p}{)}
\PYG{n}{plt}\PYG{o}{.}\PYG{n}{title}\PYG{p}{(}\PYG{l+s+s1}{\PYGZsq{}}\PYG{l+s+s1}{Simulated Dart Throws}\PYG{l+s+s1}{\PYGZsq{}}\PYG{p}{)}
\PYG{n}{plt}\PYG{o}{.}\PYG{n}{show}\PYG{p}{(}\PYG{p}{)}
\end{sphinxVerbatim}

\end{sphinxuseclass}\end{sphinxVerbatimInput}
\begin{sphinxVerbatimOutput}

\begin{sphinxuseclass}{cell_output}
\noindent\sphinxincludegraphics{{9c8f487cc54cb77b787ccddc2740c30d5e0ca308ad208d8bdfe30ab6f0ed140e}.png}

\end{sphinxuseclass}\end{sphinxVerbatimOutput}

\end{sphinxuseclass}
\sphinxAtStartPar
Finally, we use the hint above to estimate \$\textbackslash{}pi\$.
The hint above says \$f = \textbackslash{}frac\{\textbackslash{}pi r\textasciicircum{}2\}\{4 r\textasciicircum{}2\}\$, where \$f\$ is the fraction of darts on the board.
Therefore, \$\textbackslash{}pi = \textbackslash{}frac\{4fr\textasciicircum{}2\}\{r\textasciicircum{}2\} = 4f\$.

\begin{sphinxuseclass}{cell}\begin{sphinxVerbatimInput}

\begin{sphinxuseclass}{cell_input}
\begin{sphinxVerbatim}[commandchars=\\\{\}]
\PYG{n}{n} \PYG{o}{=} \PYG{l+m+mi}{10\PYGZus{}000}
\PYG{n}{pi} \PYG{o}{=} \PYG{l+m+mi}{4} \PYG{o}{*} \PYG{n}{throw\PYGZus{}darts}\PYG{p}{(}\PYG{n}{n}\PYG{o}{=}\PYG{n}{n}\PYG{p}{)}\PYG{p}{[}\PYG{l+s+s2}{\PYGZdq{}}\PYG{l+s+s2}{board}\PYG{l+s+s2}{\PYGZdq{}}\PYG{p}{]}\PYG{o}{.}\PYG{n}{mean}\PYG{p}{(}\PYG{p}{)}
\PYG{n+nb}{print}\PYG{p}{(}\PYG{l+s+sa}{f}\PYG{l+s+s1}{\PYGZsq{}}\PYG{l+s+s1}{Estimate of pi based on }\PYG{l+s+si}{\PYGZob{}}\PYG{n}{n}\PYG{l+s+si}{:}\PYG{l+s+s1}{,.0f}\PYG{l+s+si}{\PYGZcb{}}\PYG{l+s+s1}{ darts: }\PYG{l+s+si}{\PYGZob{}}\PYG{n}{pi}\PYG{l+s+si}{:}\PYG{l+s+s1}{0.4f}\PYG{l+s+si}{\PYGZcb{}}\PYG{l+s+s1}{\PYGZsq{}}\PYG{p}{)}
\end{sphinxVerbatim}

\end{sphinxuseclass}\end{sphinxVerbatimInput}
\begin{sphinxVerbatimOutput}

\begin{sphinxuseclass}{cell_output}
\begin{sphinxVerbatim}[commandchars=\\\{\}]
Estimate of pi based on 10,000 darts: 3.1544
\end{sphinxVerbatim}

\end{sphinxuseclass}\end{sphinxVerbatimOutput}

\end{sphinxuseclass}
\sphinxAtStartPar
We increase the precision of our \$\textbackslash{}pi\$ estimate by increasing the number of simulated darts \$n\$.

\begin{sphinxuseclass}{cell}\begin{sphinxVerbatimInput}

\begin{sphinxuseclass}{cell_input}
\begin{sphinxVerbatim}[commandchars=\\\{\}]
\PYG{k}{for} \PYG{n}{n} \PYG{o+ow}{in} \PYG{l+m+mi}{10}\PYG{o}{*}\PYG{o}{*}\PYG{n}{np}\PYG{o}{.}\PYG{n}{arange}\PYG{p}{(}\PYG{l+m+mi}{7}\PYG{p}{)}\PYG{p}{:}
    \PYG{n}{pi} \PYG{o}{=} \PYG{l+m+mi}{4} \PYG{o}{*} \PYG{n}{throw\PYGZus{}darts}\PYG{p}{(}\PYG{n}{n}\PYG{o}{=}\PYG{n}{n}\PYG{p}{)}\PYG{p}{[}\PYG{l+s+s2}{\PYGZdq{}}\PYG{l+s+s2}{board}\PYG{l+s+s2}{\PYGZdq{}}\PYG{p}{]}\PYG{o}{.}\PYG{n}{mean}\PYG{p}{(}\PYG{p}{)}
    \PYG{n+nb}{print}\PYG{p}{(}\PYG{l+s+sa}{f}\PYG{l+s+s1}{\PYGZsq{}}\PYG{l+s+s1}{Estimate of pi based on }\PYG{l+s+si}{\PYGZob{}}\PYG{n}{n}\PYG{l+s+si}{:}\PYG{l+s+s1}{\PYGZlt{}9,.0f}\PYG{l+s+si}{\PYGZcb{}}\PYG{l+s+s1}{ darts: }\PYG{l+s+si}{\PYGZob{}}\PYG{n}{pi}\PYG{l+s+si}{:}\PYG{l+s+s1}{0.4f}\PYG{l+s+si}{\PYGZcb{}}\PYG{l+s+s1}{\PYGZsq{}}\PYG{p}{)}
\end{sphinxVerbatim}

\end{sphinxuseclass}\end{sphinxVerbatimInput}
\begin{sphinxVerbatimOutput}

\begin{sphinxuseclass}{cell_output}
\begin{sphinxVerbatim}[commandchars=\\\{\}]
Estimate of pi based on 1         darts: 4.0000
Estimate of pi based on 10        darts: 3.2000
Estimate of pi based on 100       darts: 3.0400
Estimate of pi based on 1,000     darts: 3.1040
Estimate of pi based on 10,000    darts: 3.1544
Estimate of pi based on 100,000   darts: 3.1468
Estimate of pi based on 1,000,000 darts: 3.1420
\end{sphinxVerbatim}

\end{sphinxuseclass}\end{sphinxVerbatimOutput}

\end{sphinxuseclass}

\subsubsection{Simulate your wealth \$W\_T\$ by randomly sampling market returns}
\label{\detokenize{herron_05_practice_04:simulate-your-wealth-w-t-by-randomly-sampling-market-returns}}
\sphinxAtStartPar
Use monthly market returns from the French Data Library.
Only invest one cash flow \$W\_0\$, and plot the distribution of \$W\_T\$.

\sphinxAtStartPar
First, we download data from the French Data Library.
We convert these returns from percent to decimal to simplify compounding.

\begin{sphinxuseclass}{cell}\begin{sphinxVerbatimInput}

\begin{sphinxuseclass}{cell_input}
\begin{sphinxVerbatim}[commandchars=\\\{\}]
\PYG{n}{mkt} \PYG{o}{=} \PYG{p}{(}
    \PYG{n}{pdr}\PYG{o}{.}\PYG{n}{DataReader}\PYG{p}{(}
        \PYG{n}{name}\PYG{o}{=}\PYG{l+s+s1}{\PYGZsq{}}\PYG{l+s+s1}{F\PYGZhy{}F\PYGZus{}Research\PYGZus{}Data\PYGZus{}Factors\PYGZus{}daily}\PYG{l+s+s1}{\PYGZsq{}}\PYG{p}{,}
        \PYG{n}{data\PYGZus{}source}\PYG{o}{=}\PYG{l+s+s1}{\PYGZsq{}}\PYG{l+s+s1}{famafrench}\PYG{l+s+s1}{\PYGZsq{}}\PYG{p}{,}
        \PYG{n}{start}\PYG{o}{=}\PYG{l+s+s1}{\PYGZsq{}}\PYG{l+s+s1}{1900}\PYG{l+s+s1}{\PYGZsq{}}\PYG{p}{,}
        \PYG{n}{session}\PYG{o}{=}\PYG{n}{session}
    \PYG{p}{)}\PYG{p}{[}\PYG{l+m+mi}{0}\PYG{p}{]}
    \PYG{o}{.}\PYG{n}{assign}\PYG{p}{(}\PYG{n}{mkt}\PYG{o}{=}\PYG{k}{lambda} \PYG{n}{x}\PYG{p}{:} \PYG{p}{(}\PYG{n}{x}\PYG{p}{[}\PYG{l+s+s1}{\PYGZsq{}}\PYG{l+s+s1}{Mkt\PYGZhy{}RF}\PYG{l+s+s1}{\PYGZsq{}}\PYG{p}{]} \PYG{o}{+} \PYG{n}{x}\PYG{p}{[}\PYG{l+s+s1}{\PYGZsq{}}\PYG{l+s+s1}{RF}\PYG{l+s+s1}{\PYGZsq{}}\PYG{p}{]}\PYG{p}{)} \PYG{o}{/} \PYG{l+m+mi}{100}\PYG{p}{)}
    \PYG{p}{[}\PYG{l+s+s1}{\PYGZsq{}}\PYG{l+s+s1}{mkt}\PYG{l+s+s1}{\PYGZsq{}}\PYG{p}{]}
\PYG{p}{)}
\end{sphinxVerbatim}

\end{sphinxuseclass}\end{sphinxVerbatimInput}

\end{sphinxuseclass}
\sphinxAtStartPar
Next, we write a couple of helper functions.
The \sphinxcode{\sphinxupquote{get\_sample()}} function draws one random sample of \sphinxcode{\sphinxupquote{n}} observations from returns series \sphinxcode{\sphinxupquote{x}}.
This sample provides one simulated return history.

\begin{sphinxuseclass}{cell}\begin{sphinxVerbatimInput}

\begin{sphinxuseclass}{cell_input}
\begin{sphinxVerbatim}[commandchars=\\\{\}]
\PYG{k}{def} \PYG{n+nf}{get\PYGZus{}sample}\PYG{p}{(}\PYG{n}{x}\PYG{p}{,} \PYG{n}{n}\PYG{o}{=}\PYG{l+m+mi}{10\PYGZus{}000}\PYG{p}{,} \PYG{n}{seed}\PYG{o}{=}\PYG{l+m+mi}{42}\PYG{p}{,} \PYG{n}{start}\PYG{o}{=}\PYG{k+kc}{None}\PYG{p}{)}\PYG{p}{:}
    \PYG{k}{if} \PYG{n}{start} \PYG{o+ow}{is} \PYG{k+kc}{None}\PYG{p}{:}
        \PYG{n}{start} \PYG{o}{=} \PYG{n}{x}\PYG{o}{.}\PYG{n}{index}\PYG{p}{[}\PYG{o}{\PYGZhy{}}\PYG{l+m+mi}{1}\PYG{p}{]} \PYG{o}{+} \PYG{n}{pd}\PYG{o}{.}\PYG{n}{offsets}\PYG{o}{.}\PYG{n}{BDay}\PYG{p}{(}\PYG{p}{)}
    \PYG{k}{return} \PYG{p}{(}
        \PYG{n}{x}
        \PYG{o}{.}\PYG{n}{sample}\PYG{p}{(}\PYG{n}{n}\PYG{o}{=}\PYG{n}{n}\PYG{p}{,} \PYG{n}{replace}\PYG{o}{=}\PYG{k+kc}{True}\PYG{p}{,} \PYG{n}{random\PYGZus{}state}\PYG{o}{=}\PYG{n}{seed}\PYG{p}{,} \PYG{n}{ignore\PYGZus{}index}\PYG{o}{=}\PYG{k+kc}{True}\PYG{p}{)}
        \PYG{o}{.}\PYG{n}{set\PYGZus{}axis}\PYG{p}{(}\PYG{n}{pd}\PYG{o}{.}\PYG{n}{date\PYGZus{}range}\PYG{p}{(}\PYG{n}{start}\PYG{o}{=}\PYG{n}{start}\PYG{p}{,} \PYG{n}{periods}\PYG{o}{=}\PYG{n}{n}\PYG{p}{)}\PYG{p}{)}
        \PYG{o}{.}\PYG{n}{rename\PYGZus{}axis}\PYG{p}{(}\PYG{n}{index}\PYG{o}{=}\PYG{l+s+s1}{\PYGZsq{}}\PYG{l+s+s1}{Date}\PYG{l+s+s1}{\PYGZsq{}}\PYG{p}{)}
    \PYG{p}{)}    
\end{sphinxVerbatim}

\end{sphinxuseclass}\end{sphinxVerbatimInput}

\end{sphinxuseclass}
\sphinxAtStartPar
The \sphinxcode{\sphinxupquote{get\_samples()}} function calls the \sphinxcode{\sphinxupquote{get\_sample()}} function \sphinxcode{\sphinxupquote{m}} times to simulate \sphinxcode{\sphinxupquote{m}} return histories.
The \sphinxcode{\sphinxupquote{get\_samples()}} function combines these \sphinxcode{\sphinxupquote{m}} return histories into one data frame.

\begin{sphinxuseclass}{cell}\begin{sphinxVerbatimInput}

\begin{sphinxuseclass}{cell_input}
\begin{sphinxVerbatim}[commandchars=\\\{\}]
\PYG{k}{def} \PYG{n+nf}{get\PYGZus{}samples}\PYG{p}{(}\PYG{n}{x}\PYG{p}{,} \PYG{n}{m}\PYG{o}{=}\PYG{l+m+mi}{100}\PYG{p}{,} \PYG{n}{n}\PYG{o}{=}\PYG{l+m+mi}{10\PYGZus{}000}\PYG{p}{,} \PYG{n}{seed}\PYG{o}{=}\PYG{l+m+mi}{42}\PYG{p}{,} \PYG{n}{start}\PYG{o}{=}\PYG{k+kc}{None}\PYG{p}{)}\PYG{p}{:}
    \PYG{k}{return} \PYG{p}{(}
        \PYG{n}{pd}\PYG{o}{.}\PYG{n}{concat}\PYG{p}{(}
            \PYG{n}{objs}\PYG{o}{=}\PYG{p}{[}\PYG{n}{get\PYGZus{}sample}\PYG{p}{(}\PYG{n}{x}\PYG{o}{=}\PYG{n}{x}\PYG{p}{,} \PYG{n}{n}\PYG{o}{=}\PYG{n}{n}\PYG{p}{,} \PYG{n}{seed}\PYG{o}{=}\PYG{n}{seed}\PYG{o}{+}\PYG{n}{i}\PYG{p}{,} \PYG{n}{start}\PYG{o}{=}\PYG{n}{start}\PYG{p}{)} \PYG{k}{for} \PYG{n}{i} \PYG{o+ow}{in} \PYG{n+nb}{range}\PYG{p}{(}\PYG{n}{m}\PYG{p}{)}\PYG{p}{]}\PYG{p}{,}
            \PYG{n}{axis}\PYG{o}{=}\PYG{l+m+mi}{1}\PYG{p}{,}
            \PYG{n}{keys}\PYG{o}{=}\PYG{n+nb}{range}\PYG{p}{(}\PYG{n}{m}\PYG{p}{)}\PYG{p}{,}
            \PYG{n}{names}\PYG{o}{=}\PYG{l+s+s1}{\PYGZsq{}}\PYG{l+s+s1}{Sample}\PYG{l+s+s1}{\PYGZsq{}}
        \PYG{p}{)}
    \PYG{p}{)}
\end{sphinxVerbatim}

\end{sphinxuseclass}\end{sphinxVerbatimInput}

\end{sphinxuseclass}
\sphinxAtStartPar
Next, we use these helper functions to simulate 10,000 return histories of 10,000 trading days each.

\begin{sphinxuseclass}{cell}\begin{sphinxVerbatimInput}

\begin{sphinxuseclass}{cell_input}
\begin{sphinxVerbatim}[commandchars=\\\{\}]
\PYG{n}{mkts} \PYG{o}{=} \PYG{n}{get\PYGZus{}samples}\PYG{p}{(}\PYG{n}{x}\PYG{o}{=}\PYG{n}{mkt}\PYG{p}{,} \PYG{n}{m}\PYG{o}{=}\PYG{l+m+mi}{10\PYGZus{}000}\PYG{p}{,} \PYG{n}{n}\PYG{o}{=}\PYG{l+m+mi}{10\PYGZus{}000}\PYG{p}{)}

\PYG{n}{mkts}
\end{sphinxVerbatim}

\end{sphinxuseclass}\end{sphinxVerbatimInput}
\begin{sphinxVerbatimOutput}

\begin{sphinxuseclass}{cell_output}
\begin{sphinxVerbatim}[commandchars=\\\{\}]
Sample      0     1     2     3     4     5     6     7     8     9     ...  \PYGZbs{}
Date                                                                    ...   
2023\PYGZhy{}01\PYGZhy{}02  0.00  0.01  0.00  0.00  0.00  0.00 \PYGZhy{}0.00  0.00 \PYGZhy{}0.01  0.00  ...   
2023\PYGZhy{}01\PYGZhy{}03  0.01 \PYGZhy{}0.00  0.00  0.01  0.00 \PYGZhy{}0.01 \PYGZhy{}0.01  0.01  0.00 \PYGZhy{}0.01  ...   
2023\PYGZhy{}01\PYGZhy{}04 \PYGZhy{}0.03 \PYGZhy{}0.00 \PYGZhy{}0.01 \PYGZhy{}0.03 \PYGZhy{}0.00  0.02 \PYGZhy{}0.00  0.00 \PYGZhy{}0.01 \PYGZhy{}0.01  ...   
2023\PYGZhy{}01\PYGZhy{}05 \PYGZhy{}0.00 \PYGZhy{}0.00 \PYGZhy{}0.01 \PYGZhy{}0.00  0.01  0.01  0.00 \PYGZhy{}0.02  0.00 \PYGZhy{}0.01  ...   
2023\PYGZhy{}01\PYGZhy{}06 \PYGZhy{}0.00 \PYGZhy{}0.00 \PYGZhy{}0.00 \PYGZhy{}0.00 \PYGZhy{}0.01 \PYGZhy{}0.00  0.01  0.01 \PYGZhy{}0.00  0.01  ...   
...          ...   ...   ...   ...   ...   ...   ...   ...   ...   ...  ...   
2050\PYGZhy{}05\PYGZhy{}15  0.02 \PYGZhy{}0.01  0.01  0.00 \PYGZhy{}0.00 \PYGZhy{}0.00 \PYGZhy{}0.00  0.02  0.00 \PYGZhy{}0.00  ...   
2050\PYGZhy{}05\PYGZhy{}16  0.01 \PYGZhy{}0.00 \PYGZhy{}0.00  0.00  0.00 \PYGZhy{}0.01 \PYGZhy{}0.01  0.02  0.01  0.00  ...   
2050\PYGZhy{}05\PYGZhy{}17 \PYGZhy{}0.02  0.01  0.00  0.01  0.00  0.01  0.00  0.00  0.00  0.00  ...   
2050\PYGZhy{}05\PYGZhy{}18 \PYGZhy{}0.00  0.00  0.03  0.01  0.00  0.00 \PYGZhy{}0.00  0.00  0.00  0.00  ...   
2050\PYGZhy{}05\PYGZhy{}19  0.01 \PYGZhy{}0.02  0.00  0.01  0.00 \PYGZhy{}0.00  0.01 \PYGZhy{}0.01  0.00 \PYGZhy{}0.00  ...   

Sample      9990  9991  9992  9993  9994  9995  9996  9997  9998  9999  
Date                                                                    
2023\PYGZhy{}01\PYGZhy{}02  0.00 \PYGZhy{}0.00 \PYGZhy{}0.00  0.00  0.00  0.01 \PYGZhy{}0.00 \PYGZhy{}0.01 \PYGZhy{}0.00  0.01  
2023\PYGZhy{}01\PYGZhy{}03  0.00  0.01  0.01  0.01 \PYGZhy{}0.00  0.00  0.00 \PYGZhy{}0.01 \PYGZhy{}0.00  0.02  
2023\PYGZhy{}01\PYGZhy{}04 \PYGZhy{}0.01  0.01 \PYGZhy{}0.01 \PYGZhy{}0.01  0.01  0.00  0.01  0.01  0.00  0.02  
2023\PYGZhy{}01\PYGZhy{}05  0.01  0.02  0.03 \PYGZhy{}0.00 \PYGZhy{}0.00  0.01  0.01 \PYGZhy{}0.00  0.02  0.00  
2023\PYGZhy{}01\PYGZhy{}06  0.01  0.02  0.01 \PYGZhy{}0.01  0.02  0.01 \PYGZhy{}0.00  0.00  0.00  0.02  
...          ...   ...   ...   ...   ...   ...   ...   ...   ...   ...  
2050\PYGZhy{}05\PYGZhy{}15 \PYGZhy{}0.00 \PYGZhy{}0.01  0.02  0.00  0.00 \PYGZhy{}0.01 \PYGZhy{}0.02  0.01 \PYGZhy{}0.01 \PYGZhy{}0.04  
2050\PYGZhy{}05\PYGZhy{}16  0.00 \PYGZhy{}0.00  0.00 \PYGZhy{}0.00 \PYGZhy{}0.01 \PYGZhy{}0.00  0.01  0.00 \PYGZhy{}0.02  0.00  
2050\PYGZhy{}05\PYGZhy{}17  0.00 \PYGZhy{}0.00  0.00  0.00 \PYGZhy{}0.00  0.01 \PYGZhy{}0.00  0.01 \PYGZhy{}0.02 \PYGZhy{}0.00  
2050\PYGZhy{}05\PYGZhy{}18 \PYGZhy{}0.00  0.02  0.01 \PYGZhy{}0.00  0.00  0.00  0.00  0.01  0.00 \PYGZhy{}0.00  
2050\PYGZhy{}05\PYGZhy{}19  0.00 \PYGZhy{}0.00 \PYGZhy{}0.01 \PYGZhy{}0.00 \PYGZhy{}0.00 \PYGZhy{}0.00  0.00  0.01  0.01  0.01  

[10000 rows x 10000 columns]
\end{sphinxVerbatim}

\end{sphinxuseclass}\end{sphinxVerbatimOutput}

\end{sphinxuseclass}
\sphinxAtStartPar
We compound daily returns \$R\_t\$ to find future wealth \$W\_T\$, so \$W\_T = W\_0 \textbackslash{}times (1 + R\_1) \textbackslash{}times (1 + R\_2) \textbackslash{}times \textbackslash{}cdots \textbackslash{}times (1 + R\_T)\$.
We use the \sphinxcode{\sphinxupquote{.cumprod()}} method to compound the daily returns in data frame \sphinxcode{\sphinxupquote{mkts}}.

\begin{sphinxuseclass}{cell}\begin{sphinxVerbatimInput}

\begin{sphinxuseclass}{cell_input}
\begin{sphinxVerbatim}[commandchars=\\\{\}]
\PYG{n}{W\PYGZus{}0} \PYG{o}{=} \PYG{l+m+mi}{1\PYGZus{}000\PYGZus{}000}
\PYG{n}{W\PYGZus{}t} \PYG{o}{=} \PYG{n}{W\PYGZus{}0} \PYG{o}{*} \PYG{n}{mkts}\PYG{o}{.}\PYG{n}{add}\PYG{p}{(}\PYG{l+m+mi}{1}\PYG{p}{)}\PYG{o}{.}\PYG{n}{cumprod}\PYG{p}{(}\PYG{p}{)}
\end{sphinxVerbatim}

\end{sphinxuseclass}\end{sphinxVerbatimInput}

\end{sphinxuseclass}
\sphinxAtStartPar
We visualize terminal wealth \$W\_T\$ with a cumulative distribution plot.

\begin{sphinxuseclass}{cell}\begin{sphinxVerbatimInput}

\begin{sphinxuseclass}{cell_input}
\begin{sphinxVerbatim}[commandchars=\\\{\}]
\PYG{n}{W\PYGZus{}t}\PYG{o}{.}\PYG{n}{iloc}\PYG{p}{[}\PYG{o}{\PYGZhy{}}\PYG{l+m+mi}{1}\PYG{p}{]}\PYG{o}{.}\PYG{n}{plot}\PYG{p}{(}\PYG{n}{kind}\PYG{o}{=}\PYG{l+s+s1}{\PYGZsq{}}\PYG{l+s+s1}{hist}\PYG{l+s+s1}{\PYGZsq{}}\PYG{p}{,} \PYG{n}{density}\PYG{o}{=}\PYG{k+kc}{True}\PYG{p}{,} \PYG{n}{cumulative}\PYG{o}{=}\PYG{k+kc}{True}\PYG{p}{,} \PYG{n}{bins}\PYG{o}{=}\PYG{l+m+mi}{1\PYGZus{}000}\PYG{p}{)}
\PYG{n}{plt}\PYG{o}{.}\PYG{n}{semilogx}\PYG{p}{(}\PYG{p}{)}
\PYG{n}{plt}\PYG{o}{.}\PYG{n}{xlabel}\PYG{p}{(}\PYG{l+s+sa}{r}\PYG{l+s+s1}{\PYGZsq{}}\PYG{l+s+s1}{\PYGZdl{}W\PYGZus{}T\PYGZdl{} where \PYGZdl{}W\PYGZus{}0 = \PYGZdl{}}\PYG{l+s+s1}{\PYGZsq{}} \PYG{o}{+} \PYG{l+s+sa}{f}\PYG{l+s+s1}{\PYGZsq{}}\PYG{l+s+s1}{\PYGZbs{}}\PYG{l+s+s1}{\PYGZdl{}}\PYG{l+s+si}{\PYGZob{}}\PYG{n}{W\PYGZus{}0}\PYG{l+s+si}{:}\PYG{l+s+s1}{,.0f}\PYG{l+s+si}{\PYGZcb{}}\PYG{l+s+s1}{\PYGZsq{}}\PYG{p}{)}
\PYG{n}{plt}\PYG{o}{.}\PYG{n}{title}\PYG{p}{(}
    \PYG{l+s+sa}{r}\PYG{l+s+s1}{\PYGZsq{}}\PYG{l+s+s1}{Cumulative Distribution of \PYGZdl{}W\PYGZus{}T\PYGZdl{}}\PYG{l+s+s1}{\PYGZsq{}} \PYG{o}{+} 
    \PYG{l+s+sa}{f}\PYG{l+s+s1}{\PYGZsq{}}\PYG{l+s+se}{\PYGZbs{}n}\PYG{l+s+s1}{after }\PYG{l+s+si}{\PYGZob{}}\PYG{n}{W\PYGZus{}t}\PYG{o}{.}\PYG{n}{shape}\PYG{p}{[}\PYG{l+m+mi}{0}\PYG{p}{]}\PYG{l+s+si}{:}\PYG{l+s+s1}{,.0f}\PYG{l+s+si}{\PYGZcb{}}\PYG{l+s+s1}{ Trading Days}\PYG{l+s+s1}{\PYGZsq{}}
\PYG{p}{)}
\PYG{n}{plt}\PYG{o}{.}\PYG{n}{show}\PYG{p}{(}\PYG{p}{)}
\end{sphinxVerbatim}

\end{sphinxuseclass}\end{sphinxVerbatimInput}
\begin{sphinxVerbatimOutput}

\begin{sphinxuseclass}{cell_output}
\noindent\sphinxincludegraphics{{0211c97349d8ed9941ef650fa751fb416ffaa07a7972ee92543d9abd864642d2}.png}

\end{sphinxuseclass}\end{sphinxVerbatimOutput}

\end{sphinxuseclass}
\sphinxAtStartPar
The plot above shows the cumulative distribution of \$W\_T\$, suggesting we expect \$W\_T\$ greater than about \$17 million for about 75\% of samples.

\sphinxAtStartPar
The plot above may be difficult to read and interpret because \$W\_T\$ has large outliers.
The \sphinxcode{\sphinxupquote{.describe()}} method provides few salient values of \$W\_T\$.
We convert \$W\_T\$ from dollars to millions of dollars with \sphinxcode{\sphinxupquote{.div(1\_000\_000)}}.

\begin{sphinxuseclass}{cell}\begin{sphinxVerbatimInput}

\begin{sphinxuseclass}{cell_input}
\begin{sphinxVerbatim}[commandchars=\\\{\}]
\PYG{n}{W\PYGZus{}t}\PYG{o}{.}\PYG{n}{iloc}\PYG{p}{[}\PYG{o}{\PYGZhy{}}\PYG{l+m+mi}{1}\PYG{p}{]}\PYG{o}{.}\PYG{n}{div}\PYG{p}{(}\PYG{l+m+mi}{1\PYGZus{}000\PYGZus{}000}\PYG{p}{)}\PYG{o}{.}\PYG{n}{describe}\PYG{p}{(}\PYG{p}{)}
\end{sphinxVerbatim}

\end{sphinxuseclass}\end{sphinxVerbatimInput}
\begin{sphinxVerbatimOutput}

\begin{sphinxuseclass}{cell_output}
\begin{sphinxVerbatim}[commandchars=\\\{\}]
count   10000.00
mean       64.75
std       101.61
min         0.56
25\PYGZpc{}        17.56
50\PYGZpc{}        36.11
75\PYGZpc{}        74.11
max      3330.53
Name: 2050\PYGZhy{}05\PYGZhy{}19 00:00:00, dtype: float64
\end{sphinxVerbatim}

\end{sphinxuseclass}\end{sphinxVerbatimOutput}

\end{sphinxuseclass}

\subsubsection{Repeat the exercise above but add end\sphinxhyphen{}of\sphinxhyphen{}month investments \$C\_t\$}
\label{\detokenize{herron_05_practice_04:repeat-the-exercise-above-but-add-end-of-month-investments-c-t}}
\sphinxAtStartPar
We can use the same data frame \sphinxcode{\sphinxupquote{mkts}} of simulated market returns.
However, need to consider end\sphinxhyphen{}of\sphinxhyphen{}month investments of \$C\_t\$.
The easiest approach is to aggregate the daily market returns in \sphinxcode{\sphinxupquote{mkts}} to monthly market returns in data frame \sphinxcode{\sphinxupquote{mkts\_m}}.
The last month in \sphinxcode{\sphinxupquote{mkts\_m}} is not complete (i.e., fewer than about 21 trading days), so we drop is with \sphinxcode{\sphinxupquote{.iloc{[}:\sphinxhyphen{}1{]}}}.

\begin{sphinxuseclass}{cell}\begin{sphinxVerbatimInput}

\begin{sphinxuseclass}{cell_input}
\begin{sphinxVerbatim}[commandchars=\\\{\}]
\PYG{n}{mkts\PYGZus{}m} \PYG{o}{=} \PYG{p}{(}
    \PYG{n}{mkts}
    \PYG{o}{.}\PYG{n}{add}\PYG{p}{(}\PYG{l+m+mi}{1}\PYG{p}{)}
    \PYG{o}{.}\PYG{n}{resample}\PYG{p}{(}\PYG{n}{rule}\PYG{o}{=}\PYG{l+s+s1}{\PYGZsq{}}\PYG{l+s+s1}{M}\PYG{l+s+s1}{\PYGZsq{}}\PYG{p}{,} \PYG{n}{kind}\PYG{o}{=}\PYG{l+s+s1}{\PYGZsq{}}\PYG{l+s+s1}{period}\PYG{l+s+s1}{\PYGZsq{}}\PYG{p}{)}
    \PYG{o}{.}\PYG{n}{prod}\PYG{p}{(}\PYG{p}{)}
    \PYG{o}{.}\PYG{n}{sub}\PYG{p}{(}\PYG{l+m+mi}{1}\PYG{p}{)}
    \PYG{o}{.}\PYG{n}{iloc}\PYG{p}{[}\PYG{p}{:}\PYG{o}{\PYGZhy{}}\PYG{l+m+mi}{1}\PYG{p}{]}
\PYG{p}{)}
\end{sphinxVerbatim}

\end{sphinxuseclass}\end{sphinxVerbatimInput}

\end{sphinxuseclass}
\sphinxAtStartPar
The wealth at time \$t\$ is \$W\_t\$ and depends on:
\begin{enumerate}
\sphinxsetlistlabels{\arabic}{enumi}{enumii}{}{.}%
\item {} 
\sphinxAtStartPar
The wealth at the end of the previous month \$W\_\{t\sphinxhyphen{}1\}\$

\item {} 
\sphinxAtStartPar
The return over the previous month \$R\_t\$ (recall we label returns by their right edge)

\item {} 
\sphinxAtStartPar
The end\sphinxhyphen{}of\sphinxhyphen{}month investment \$C\_t\$

\end{enumerate}

\sphinxAtStartPar
Putting it all togther: \$W\_t = W\_\{t\sphinxhyphen{}1\} \textbackslash{}times (1 + R\_t) + C\_t\$

\sphinxAtStartPar
We have to loop over the monthly returns in \sphinxcode{\sphinxupquote{mkts\_m}} because we have to combine compounded returns and cash flows.
The \sphinxcode{\sphinxupquote{iterrows()}} methods provides an easy way to iterate (loop) over the rows in \sphinxcode{\sphinxupquote{mkts\_m}}.

\begin{sphinxuseclass}{cell}\begin{sphinxVerbatimInput}

\begin{sphinxuseclass}{cell_input}
\begin{sphinxVerbatim}[commandchars=\\\{\}]
\PYG{n}{C\PYGZus{}t} \PYG{o}{=} \PYG{l+m+mi}{1\PYGZus{}000}
\PYG{n}{W\PYGZus{}0} \PYG{o}{=} \PYG{l+m+mi}{1\PYGZus{}000\PYGZus{}000}
\PYG{n}{W\PYGZus{}last} \PYG{o}{=} \PYG{n}{W\PYGZus{}0}
\PYG{n}{W\PYGZus{}t} \PYG{o}{=} \PYG{p}{[}\PYG{p}{]}

\PYG{k}{for} \PYG{n}{d}\PYG{p}{,} \PYG{n}{m} \PYG{o+ow}{in} \PYG{n}{mkts\PYGZus{}m}\PYG{o}{.}\PYG{n}{iterrows}\PYG{p}{(}\PYG{p}{)}\PYG{p}{:}
    \PYG{n}{W\PYGZus{}last} \PYG{o}{=} \PYG{p}{(}\PYG{l+m+mi}{1} \PYG{o}{+} \PYG{n}{m}\PYG{p}{)} \PYG{o}{*} \PYG{n}{W\PYGZus{}last} \PYG{o}{+} \PYG{n}{C\PYGZus{}t}
    \PYG{n}{W\PYGZus{}t}\PYG{o}{.}\PYG{n}{append}\PYG{p}{(}\PYG{n}{W\PYGZus{}last}\PYG{p}{)}

\PYG{n}{W\PYGZus{}t} \PYG{o}{=} \PYG{n}{pd}\PYG{o}{.}\PYG{n}{concat}\PYG{p}{(}\PYG{n}{objs}\PYG{o}{=}\PYG{n}{W\PYGZus{}t}\PYG{p}{,} \PYG{n}{axis}\PYG{o}{=}\PYG{l+m+mi}{1}\PYG{p}{,} \PYG{n}{keys}\PYG{o}{=}\PYG{n}{mkts\PYGZus{}m}\PYG{o}{.}\PYG{n}{index}\PYG{p}{)}\PYG{o}{.}\PYG{n}{transpose}\PYG{p}{(}\PYG{p}{)}
\end{sphinxVerbatim}

\end{sphinxuseclass}\end{sphinxVerbatimInput}

\end{sphinxuseclass}
\sphinxAtStartPar
We repeat the cumulative distribution of wealth plot and descriptive statistics from above.

\begin{sphinxuseclass}{cell}\begin{sphinxVerbatimInput}

\begin{sphinxuseclass}{cell_input}
\begin{sphinxVerbatim}[commandchars=\\\{\}]
\PYG{n}{W\PYGZus{}t}\PYG{o}{.}\PYG{n}{iloc}\PYG{p}{[}\PYG{o}{\PYGZhy{}}\PYG{l+m+mi}{1}\PYG{p}{]}\PYG{o}{.}\PYG{n}{plot}\PYG{p}{(}\PYG{n}{kind}\PYG{o}{=}\PYG{l+s+s1}{\PYGZsq{}}\PYG{l+s+s1}{hist}\PYG{l+s+s1}{\PYGZsq{}}\PYG{p}{,} \PYG{n}{density}\PYG{o}{=}\PYG{k+kc}{True}\PYG{p}{,} \PYG{n}{cumulative}\PYG{o}{=}\PYG{k+kc}{True}\PYG{p}{,} \PYG{n}{bins}\PYG{o}{=}\PYG{l+m+mi}{1\PYGZus{}000}\PYG{p}{)}
\PYG{n}{plt}\PYG{o}{.}\PYG{n}{semilogx}\PYG{p}{(}\PYG{p}{)}
\PYG{n}{plt}\PYG{o}{.}\PYG{n}{xlabel}\PYG{p}{(}
    \PYG{l+s+sa}{r}\PYG{l+s+s1}{\PYGZsq{}}\PYG{l+s+s1}{\PYGZdl{}W\PYGZus{}T\PYGZdl{} where \PYGZdl{}W\PYGZus{}0 = \PYGZdl{}}\PYG{l+s+s1}{\PYGZsq{}} \PYG{o}{+} \PYG{l+s+sa}{f}\PYG{l+s+s1}{\PYGZsq{}}\PYG{l+s+s1}{\PYGZbs{}}\PYG{l+s+s1}{\PYGZdl{}}\PYG{l+s+si}{\PYGZob{}}\PYG{n}{W\PYGZus{}0}\PYG{l+s+si}{:}\PYG{l+s+s1}{,.0f}\PYG{l+s+si}{\PYGZcb{}}\PYG{l+s+s1}{\PYGZsq{}} \PYG{o}{+} 
    \PYG{l+s+sa}{r}\PYG{l+s+s1}{\PYGZsq{}}\PYG{l+s+s1}{ and  \PYGZdl{}C\PYGZus{}t = \PYGZdl{}}\PYG{l+s+s1}{\PYGZsq{}} \PYG{o}{+} \PYG{l+s+sa}{f}\PYG{l+s+s1}{\PYGZsq{}}\PYG{l+s+s1}{\PYGZbs{}}\PYG{l+s+s1}{\PYGZdl{}}\PYG{l+s+si}{\PYGZob{}}\PYG{n}{C\PYGZus{}t}\PYG{l+s+si}{:}\PYG{l+s+s1}{,.0f}\PYG{l+s+si}{\PYGZcb{}}\PYG{l+s+s1}{\PYGZsq{}}
\PYG{p}{)}
\PYG{n}{plt}\PYG{o}{.}\PYG{n}{title}\PYG{p}{(}
    \PYG{l+s+sa}{r}\PYG{l+s+s1}{\PYGZsq{}}\PYG{l+s+s1}{Cumulative Distribution of \PYGZdl{}W\PYGZus{}T\PYGZdl{}}\PYG{l+s+s1}{\PYGZsq{}} \PYG{o}{+} 
    \PYG{l+s+sa}{f}\PYG{l+s+s1}{\PYGZsq{}}\PYG{l+s+se}{\PYGZbs{}n}\PYG{l+s+s1}{after }\PYG{l+s+si}{\PYGZob{}}\PYG{n}{W\PYGZus{}t}\PYG{o}{.}\PYG{n}{shape}\PYG{p}{[}\PYG{l+m+mi}{0}\PYG{p}{]}\PYG{l+s+si}{:}\PYG{l+s+s1}{,.0f}\PYG{l+s+si}{\PYGZcb{}}\PYG{l+s+s1}{ Trading Months}\PYG{l+s+s1}{\PYGZsq{}}
\PYG{p}{)}
\PYG{n}{plt}\PYG{o}{.}\PYG{n}{show}\PYG{p}{(}\PYG{p}{)}
\end{sphinxVerbatim}

\end{sphinxuseclass}\end{sphinxVerbatimInput}
\begin{sphinxVerbatimOutput}

\begin{sphinxuseclass}{cell_output}
\noindent\sphinxincludegraphics{{6d78aab7a506358bd5a775b38f3ef6930893fe48e082ff3b0dfee6a188e7bc80}.png}

\end{sphinxuseclass}\end{sphinxVerbatimOutput}

\end{sphinxuseclass}
\begin{sphinxuseclass}{cell}\begin{sphinxVerbatimInput}

\begin{sphinxuseclass}{cell_input}
\begin{sphinxVerbatim}[commandchars=\\\{\}]
\PYG{n}{W\PYGZus{}t}\PYG{o}{.}\PYG{n}{iloc}\PYG{p}{[}\PYG{o}{\PYGZhy{}}\PYG{l+m+mi}{1}\PYG{p}{]}\PYG{o}{.}\PYG{n}{div}\PYG{p}{(}\PYG{l+m+mi}{1\PYGZus{}000\PYGZus{}000}\PYG{p}{)}\PYG{o}{.}\PYG{n}{describe}\PYG{p}{(}\PYG{p}{)}
\end{sphinxVerbatim}

\end{sphinxuseclass}\end{sphinxVerbatimInput}
\begin{sphinxVerbatimOutput}

\begin{sphinxuseclass}{cell_output}
\begin{sphinxVerbatim}[commandchars=\\\{\}]
count   10000.00
mean       69.20
std       106.22
min         0.94
25\PYGZpc{}        19.59
50\PYGZpc{}        39.27
75\PYGZpc{}        79.69
max      3467.92
Name: 2050\PYGZhy{}04, dtype: float64
\end{sphinxVerbatim}

\end{sphinxuseclass}\end{sphinxVerbatimOutput}

\end{sphinxuseclass}
\sphinxAtStartPar
The plot above shows the cumulative distribution of \$W\_T\$, suggesting we expect \$W\_T\$ greater than about \textbackslash{}\$20 million for about 75\% of samples.
Note, even though we deposit \$1,000 per month for almost 40 years, we do not gain much wealth over the previous example with only a lump sum!
Start saving early!

\sphinxstepscope


\section{Herron Topic 5 \sphinxhyphen{} Practice (Wednesday 2:45 PM, Section 2)}
\label{\detokenize{herron_05_practice_02:herron-topic-5-practice-wednesday-2-45-pm-section-2}}\label{\detokenize{herron_05_practice_02::doc}}

\subsection{Announcements}
\label{\detokenize{herron_05_practice_02:announcements}}\begin{itemize}
\item {} 
\sphinxAtStartPar
Due dates
\begin{itemize}
\item {} 
\sphinxAtStartPar
Team project 2 is due by 11:59 PM on Wednesday, 4/26

\item {} 
\sphinxAtStartPar
Teammates review 2 is due by 11:59 PM on Wednesday, 4/26

\item {} 
\sphinxAtStartPar
30,000 DataCamp XP are due by 11:59 PM on \sphinxstyleemphasis{Friday, 4/28}

\item {} 
\sphinxAtStartPar
TRACE course evaluations are due by 11:59 PM on \sphinxstyleemphasis{Friday, 4/28}

\end{itemize}

\item {} 
\sphinxAtStartPar
\sphinxstyleemphasis{\sphinxstylestrong{TRACE course evaluations are optional (and anonymous), but I value your feedback!}}
\begin{itemize}
\item {} 
\sphinxAtStartPar
I always find student feedback useful, and I read and re\sphinxhyphen{}read it!

\item {} 
\sphinxAtStartPar
I tweak my courses every semester based on student feedback!

\item {} 
\sphinxAtStartPar
Your feedback does not have to be perfect to be useful!

\item {} 
\sphinxAtStartPar
If you had a magic wand, what would you change about my course and approach?

\end{itemize}

\item {} 
\sphinxAtStartPar
This week and next, we can discuss whatever topics you want, including simulations and team project 2

\end{itemize}


\subsection{Practice}
\label{\detokenize{herron_05_practice_02:practice}}
\begin{sphinxuseclass}{cell}\begin{sphinxVerbatimInput}

\begin{sphinxuseclass}{cell_input}
\begin{sphinxVerbatim}[commandchars=\\\{\}]
\PYG{k+kn}{import} \PYG{n+nn}{matplotlib}\PYG{n+nn}{.}\PYG{n+nn}{pyplot} \PYG{k}{as} \PYG{n+nn}{plt}
\PYG{k+kn}{import} \PYG{n+nn}{numpy} \PYG{k}{as} \PYG{n+nn}{np}
\PYG{k+kn}{import} \PYG{n+nn}{pandas} \PYG{k}{as} \PYG{n+nn}{pd}
\end{sphinxVerbatim}

\end{sphinxuseclass}\end{sphinxVerbatimInput}

\end{sphinxuseclass}
\begin{sphinxuseclass}{cell}\begin{sphinxVerbatimInput}

\begin{sphinxuseclass}{cell_input}
\begin{sphinxVerbatim}[commandchars=\\\{\}]
\PYG{o}{\PYGZpc{}}\PYG{k}{config} InlineBackend.figure\PYGZus{}format = \PYGZsq{}retina\PYGZsq{}
\PYG{o}{\PYGZpc{}}\PYG{k}{precision} 2
\PYG{n}{pd}\PYG{o}{.}\PYG{n}{options}\PYG{o}{.}\PYG{n}{display}\PYG{o}{.}\PYG{n}{float\PYGZus{}format} \PYG{o}{=} \PYG{l+s+s1}{\PYGZsq{}}\PYG{l+s+si}{\PYGZob{}:.2f\PYGZcb{}}\PYG{l+s+s1}{\PYGZsq{}}\PYG{o}{.}\PYG{n}{format}
\end{sphinxVerbatim}

\end{sphinxuseclass}\end{sphinxVerbatimInput}

\end{sphinxuseclass}
\begin{sphinxuseclass}{cell}\begin{sphinxVerbatimInput}

\begin{sphinxuseclass}{cell_input}
\begin{sphinxVerbatim}[commandchars=\\\{\}]
\PYG{k+kn}{import} \PYG{n+nn}{yfinance} \PYG{k}{as} \PYG{n+nn}{yf}
\PYG{k+kn}{import} \PYG{n+nn}{pandas\PYGZus{}datareader} \PYG{k}{as} \PYG{n+nn}{pdr}
\PYG{k+kn}{import} \PYG{n+nn}{requests\PYGZus{}cache}
\PYG{n}{session} \PYG{o}{=} \PYG{n}{requests\PYGZus{}cache}\PYG{o}{.}\PYG{n}{CachedSession}\PYG{p}{(}\PYG{p}{)}
\end{sphinxVerbatim}

\end{sphinxuseclass}\end{sphinxVerbatimInput}

\end{sphinxuseclass}

\subsubsection{Estimate \$\textbackslash{}pi\$ by simulating darts thrown at a dart board}
\label{\detokenize{herron_05_practice_02:estimate-pi-by-simulating-darts-thrown-at-a-dart-board}}
\sphinxAtStartPar
\sphinxstyleemphasis{Hints:}
Select random \$x\$s and \$y\$s such that \$\sphinxhyphen{}r \textbackslash{}leq x \textbackslash{}leq +r\$ and \$\sphinxhyphen{}r \textbackslash{}leq x \textbackslash{}leq +r\$.
Darts are on the board if \$x\textasciicircum{}2 + y\textasciicircum{}2 \textbackslash{}leq r\textasciicircum{}2\$.
The area of the circlular board is \$\textbackslash{}pi r\textasciicircum{}2\$, and the area of square around the board is \$(2r)\textasciicircum{}2 = 4r\textasciicircum{}2\$.
The fraction \$f\$ of darts on the board is the same as the ratio of circle area to square area, so \$f = \textbackslash{}frac\{\textbackslash{}pi r\textasciicircum{}2\}\{4 r\textasciicircum{}2\}\$.

\sphinxAtStartPar
First we throw darts at the board.
Darts with \$x\textasciicircum{}2 + y\textasciicircum{}2 \textbackslash{}leq r\textasciicircum{}2\$ are on the board.

\begin{sphinxuseclass}{cell}\begin{sphinxVerbatimInput}

\begin{sphinxuseclass}{cell_input}
\begin{sphinxVerbatim}[commandchars=\\\{\}]
\PYG{k}{def} \PYG{n+nf}{throw\PYGZus{}darts}\PYG{p}{(}\PYG{n}{n}\PYG{o}{=}\PYG{l+m+mi}{10\PYGZus{}000}\PYG{p}{,} \PYG{n}{r}\PYG{o}{=}\PYG{l+m+mi}{1}\PYG{p}{,} \PYG{n}{seed}\PYG{o}{=}\PYG{l+m+mi}{42}\PYG{p}{)}\PYG{p}{:}
    \PYG{n}{np}\PYG{o}{.}\PYG{n}{random}\PYG{o}{.}\PYG{n}{seed}\PYG{p}{(}\PYG{n}{seed}\PYG{p}{)}
    \PYG{k}{return} \PYG{p}{(}
        \PYG{n}{pd}\PYG{o}{.}\PYG{n}{DataFrame}\PYG{p}{(}
            \PYG{n}{data}\PYG{o}{=}\PYG{n}{np}\PYG{o}{.}\PYG{n}{random}\PYG{o}{.}\PYG{n}{uniform}\PYG{p}{(}\PYG{n}{low}\PYG{o}{=}\PYG{o}{\PYGZhy{}}\PYG{n}{r}\PYG{p}{,} \PYG{n}{high}\PYG{o}{=}\PYG{n}{r}\PYG{p}{,} \PYG{n}{size}\PYG{o}{=}\PYG{l+m+mi}{2}\PYG{o}{*}\PYG{n}{n}\PYG{p}{)}\PYG{o}{.}\PYG{n}{reshape}\PYG{p}{(}\PYG{n}{n}\PYG{p}{,} \PYG{l+m+mi}{2}\PYG{p}{)}\PYG{p}{,} 
            \PYG{n}{columns}\PYG{o}{=}\PYG{p}{[}\PYG{l+s+s1}{\PYGZsq{}}\PYG{l+s+s1}{x}\PYG{l+s+s1}{\PYGZsq{}}\PYG{p}{,} \PYG{l+s+s1}{\PYGZsq{}}\PYG{l+s+s1}{y}\PYG{l+s+s1}{\PYGZsq{}}\PYG{p}{]}
        \PYG{p}{)}
        \PYG{o}{.}\PYG{n}{assign}\PYG{p}{(}\PYG{n}{board}\PYG{o}{=}\PYG{k}{lambda} \PYG{n}{x}\PYG{p}{:} \PYG{n}{x}\PYG{p}{[}\PYG{l+s+s1}{\PYGZsq{}}\PYG{l+s+s1}{x}\PYG{l+s+s1}{\PYGZsq{}}\PYG{p}{]}\PYG{o}{*}\PYG{o}{*}\PYG{l+m+mi}{2} \PYG{o}{+} \PYG{n}{x}\PYG{p}{[}\PYG{l+s+s1}{\PYGZsq{}}\PYG{l+s+s1}{y}\PYG{l+s+s1}{\PYGZsq{}}\PYG{p}{]}\PYG{o}{*}\PYG{o}{*}\PYG{l+m+mi}{2} \PYG{o}{\PYGZlt{}}\PYG{o}{=} \PYG{n}{r}\PYG{o}{*}\PYG{o}{*}\PYG{l+m+mi}{2}\PYG{p}{)}
        \PYG{o}{.}\PYG{n}{rename\PYGZus{}axis}\PYG{p}{(}\PYG{n}{index}\PYG{o}{=}\PYG{l+s+s1}{\PYGZsq{}}\PYG{l+s+s1}{n}\PYG{l+s+s1}{\PYGZsq{}}\PYG{p}{,} \PYG{n}{columns}\PYG{o}{=}\PYG{l+s+s1}{\PYGZsq{}}\PYG{l+s+s1}{Variable}\PYG{l+s+s1}{\PYGZsq{}}\PYG{p}{)}
    \PYG{p}{)}
\end{sphinxVerbatim}

\end{sphinxuseclass}\end{sphinxVerbatimInput}

\end{sphinxuseclass}
\begin{sphinxuseclass}{cell}\begin{sphinxVerbatimInput}

\begin{sphinxuseclass}{cell_input}
\begin{sphinxVerbatim}[commandchars=\\\{\}]
\PYG{n}{throw\PYGZus{}darts}\PYG{p}{(}\PYG{p}{)}
\end{sphinxVerbatim}

\end{sphinxuseclass}\end{sphinxVerbatimInput}
\begin{sphinxVerbatimOutput}

\begin{sphinxuseclass}{cell_output}
\begin{sphinxVerbatim}[commandchars=\\\{\}]
Variable     x     y  board
n                          
0        \PYGZhy{}0.25  0.90   True
1         0.46  0.20   True
2        \PYGZhy{}0.69 \PYGZhy{}0.69   True
3        \PYGZhy{}0.88  0.73  False
4         0.20  0.42   True
...        ...   ...    ...
9995      0.15  0.52   True
9996     \PYGZhy{}0.82 \PYGZhy{}0.01   True
9997      0.80  0.75  False
9998     \PYGZhy{}0.91 \PYGZhy{}0.39   True
9999     \PYGZhy{}0.11 \PYGZhy{}0.66   True

[10000 rows x 3 columns]
\end{sphinxVerbatim}

\end{sphinxuseclass}\end{sphinxVerbatimOutput}

\end{sphinxuseclass}
\sphinxAtStartPar
Next, we visualize these darts with a scatter plot.
Seaborn’s \sphinxcode{\sphinxupquote{scatterplot()}} helps color darts by location (i.e., on or off board).
The \sphinxcode{\sphinxupquote{.pipe()}} method lets us send the output of the \sphinxcode{\sphinxupquote{.assign()}} method to \sphinxcode{\sphinxupquote{sns.scatterplot()}} without assigning a temporary data frame.

\begin{sphinxuseclass}{cell}\begin{sphinxVerbatimInput}

\begin{sphinxuseclass}{cell_input}
\begin{sphinxVerbatim}[commandchars=\\\{\}]
\PYG{k+kn}{import} \PYG{n+nn}{seaborn} \PYG{k}{as} \PYG{n+nn}{sns}
\end{sphinxVerbatim}

\end{sphinxuseclass}\end{sphinxVerbatimInput}

\end{sphinxuseclass}
\begin{sphinxuseclass}{cell}\begin{sphinxVerbatimInput}

\begin{sphinxuseclass}{cell_input}
\begin{sphinxVerbatim}[commandchars=\\\{\}]
\PYG{p}{(}
    \PYG{n}{throw\PYGZus{}darts}\PYG{p}{(}\PYG{p}{)}
    \PYG{o}{.}\PYG{n}{assign}\PYG{p}{(}\PYG{n}{Location}\PYG{o}{=}\PYG{k}{lambda} \PYG{n}{x}\PYG{p}{:} \PYG{n}{np}\PYG{o}{.}\PYG{n}{where}\PYG{p}{(}\PYG{n}{x}\PYG{p}{[}\PYG{l+s+s1}{\PYGZsq{}}\PYG{l+s+s1}{board}\PYG{l+s+s1}{\PYGZsq{}}\PYG{p}{]}\PYG{p}{,} \PYG{l+s+s1}{\PYGZsq{}}\PYG{l+s+s1}{On Board}\PYG{l+s+s1}{\PYGZsq{}}\PYG{p}{,} \PYG{l+s+s1}{\PYGZsq{}}\PYG{l+s+s1}{Off Board}\PYG{l+s+s1}{\PYGZsq{}}\PYG{p}{)}\PYG{p}{)}
    \PYG{o}{.}\PYG{n}{pipe}\PYG{p}{(}\PYG{k}{lambda} \PYG{n}{x}\PYG{p}{:} \PYG{n}{sns}\PYG{o}{.}\PYG{n}{scatterplot}\PYG{p}{(}\PYG{n}{x}\PYG{o}{=}\PYG{l+s+s1}{\PYGZsq{}}\PYG{l+s+s1}{x}\PYG{l+s+s1}{\PYGZsq{}}\PYG{p}{,} \PYG{n}{y}\PYG{o}{=}\PYG{l+s+s1}{\PYGZsq{}}\PYG{l+s+s1}{y}\PYG{l+s+s1}{\PYGZsq{}}\PYG{p}{,} \PYG{n}{data}\PYG{o}{=}\PYG{n}{x}\PYG{p}{,} \PYG{n}{hue}\PYG{o}{=}\PYG{l+s+s1}{\PYGZsq{}}\PYG{l+s+s1}{Location}\PYG{l+s+s1}{\PYGZsq{}}\PYG{p}{)}\PYG{p}{)}
\PYG{p}{)}
\PYG{n}{plt}\PYG{o}{.}\PYG{n}{title}\PYG{p}{(}\PYG{l+s+s1}{\PYGZsq{}}\PYG{l+s+s1}{Simulated Dart Throws}\PYG{l+s+s1}{\PYGZsq{}}\PYG{p}{)}
\PYG{n}{plt}\PYG{o}{.}\PYG{n}{show}\PYG{p}{(}\PYG{p}{)}
\end{sphinxVerbatim}

\end{sphinxuseclass}\end{sphinxVerbatimInput}
\begin{sphinxVerbatimOutput}

\begin{sphinxuseclass}{cell_output}
\noindent\sphinxincludegraphics{{9c8f487cc54cb77b787ccddc2740c30d5e0ca308ad208d8bdfe30ab6f0ed140e}.png}

\end{sphinxuseclass}\end{sphinxVerbatimOutput}

\end{sphinxuseclass}
\sphinxAtStartPar
Finally, we use the hint above to estimate \$\textbackslash{}pi\$.
The hint above says \$f = \textbackslash{}frac\{\textbackslash{}pi r\textasciicircum{}2\}\{4 r\textasciicircum{}2\}\$, where \$f\$ is the fraction of darts on the board.
Therefore, \$\textbackslash{}pi = \textbackslash{}frac\{4fr\textasciicircum{}2\}\{r\textasciicircum{}2\} = 4f\$.

\begin{sphinxuseclass}{cell}\begin{sphinxVerbatimInput}

\begin{sphinxuseclass}{cell_input}
\begin{sphinxVerbatim}[commandchars=\\\{\}]
\PYG{n}{n} \PYG{o}{=} \PYG{l+m+mi}{10\PYGZus{}000}
\PYG{n}{pi} \PYG{o}{=} \PYG{l+m+mi}{4} \PYG{o}{*} \PYG{n}{throw\PYGZus{}darts}\PYG{p}{(}\PYG{n}{n}\PYG{o}{=}\PYG{n}{n}\PYG{p}{)}\PYG{p}{[}\PYG{l+s+s2}{\PYGZdq{}}\PYG{l+s+s2}{board}\PYG{l+s+s2}{\PYGZdq{}}\PYG{p}{]}\PYG{o}{.}\PYG{n}{mean}\PYG{p}{(}\PYG{p}{)}
\PYG{n+nb}{print}\PYG{p}{(}\PYG{l+s+sa}{f}\PYG{l+s+s1}{\PYGZsq{}}\PYG{l+s+s1}{Estimate of pi based on }\PYG{l+s+si}{\PYGZob{}}\PYG{n}{n}\PYG{l+s+si}{:}\PYG{l+s+s1}{,.0f}\PYG{l+s+si}{\PYGZcb{}}\PYG{l+s+s1}{ darts: }\PYG{l+s+si}{\PYGZob{}}\PYG{n}{pi}\PYG{l+s+si}{:}\PYG{l+s+s1}{0.4f}\PYG{l+s+si}{\PYGZcb{}}\PYG{l+s+s1}{\PYGZsq{}}\PYG{p}{)}
\end{sphinxVerbatim}

\end{sphinxuseclass}\end{sphinxVerbatimInput}
\begin{sphinxVerbatimOutput}

\begin{sphinxuseclass}{cell_output}
\begin{sphinxVerbatim}[commandchars=\\\{\}]
Estimate of pi based on 10,000 darts: 3.1544
\end{sphinxVerbatim}

\end{sphinxuseclass}\end{sphinxVerbatimOutput}

\end{sphinxuseclass}
\sphinxAtStartPar
We increase the precision of our \$\textbackslash{}pi\$ estimate by increasing the number of simulated darts \$n\$.

\begin{sphinxuseclass}{cell}\begin{sphinxVerbatimInput}

\begin{sphinxuseclass}{cell_input}
\begin{sphinxVerbatim}[commandchars=\\\{\}]
\PYG{k}{for} \PYG{n}{n} \PYG{o+ow}{in} \PYG{l+m+mi}{10}\PYG{o}{*}\PYG{o}{*}\PYG{n}{np}\PYG{o}{.}\PYG{n}{arange}\PYG{p}{(}\PYG{l+m+mi}{7}\PYG{p}{)}\PYG{p}{:}
    \PYG{n}{pi} \PYG{o}{=} \PYG{l+m+mi}{4} \PYG{o}{*} \PYG{n}{throw\PYGZus{}darts}\PYG{p}{(}\PYG{n}{n}\PYG{o}{=}\PYG{n}{n}\PYG{p}{)}\PYG{p}{[}\PYG{l+s+s2}{\PYGZdq{}}\PYG{l+s+s2}{board}\PYG{l+s+s2}{\PYGZdq{}}\PYG{p}{]}\PYG{o}{.}\PYG{n}{mean}\PYG{p}{(}\PYG{p}{)}
    \PYG{n+nb}{print}\PYG{p}{(}\PYG{l+s+sa}{f}\PYG{l+s+s1}{\PYGZsq{}}\PYG{l+s+s1}{Estimate of pi based on }\PYG{l+s+si}{\PYGZob{}}\PYG{n}{n}\PYG{l+s+si}{:}\PYG{l+s+s1}{\PYGZlt{}9,.0f}\PYG{l+s+si}{\PYGZcb{}}\PYG{l+s+s1}{ darts: }\PYG{l+s+si}{\PYGZob{}}\PYG{n}{pi}\PYG{l+s+si}{:}\PYG{l+s+s1}{0.4f}\PYG{l+s+si}{\PYGZcb{}}\PYG{l+s+s1}{\PYGZsq{}}\PYG{p}{)}
\end{sphinxVerbatim}

\end{sphinxuseclass}\end{sphinxVerbatimInput}
\begin{sphinxVerbatimOutput}

\begin{sphinxuseclass}{cell_output}
\begin{sphinxVerbatim}[commandchars=\\\{\}]
Estimate of pi based on 1         darts: 4.0000
Estimate of pi based on 10        darts: 3.2000
Estimate of pi based on 100       darts: 3.0400
Estimate of pi based on 1,000     darts: 3.1040
Estimate of pi based on 10,000    darts: 3.1544
Estimate of pi based on 100,000   darts: 3.1468
Estimate of pi based on 1,000,000 darts: 3.1420
\end{sphinxVerbatim}

\end{sphinxuseclass}\end{sphinxVerbatimOutput}

\end{sphinxuseclass}

\subsubsection{Simulate your wealth \$W\_T\$ by randomly sampling market returns}
\label{\detokenize{herron_05_practice_02:simulate-your-wealth-w-t-by-randomly-sampling-market-returns}}
\sphinxAtStartPar
Use monthly market returns from the French Data Library.
Only invest one cash flow \$W\_0\$, and plot the distribution of \$W\_T\$.

\sphinxAtStartPar
First, we download data from the French Data Library.
We convert these returns from percent to decimal to simplify compounding.

\begin{sphinxuseclass}{cell}\begin{sphinxVerbatimInput}

\begin{sphinxuseclass}{cell_input}
\begin{sphinxVerbatim}[commandchars=\\\{\}]
\PYG{n}{mkt} \PYG{o}{=} \PYG{p}{(}
    \PYG{n}{pdr}\PYG{o}{.}\PYG{n}{DataReader}\PYG{p}{(}
        \PYG{n}{name}\PYG{o}{=}\PYG{l+s+s1}{\PYGZsq{}}\PYG{l+s+s1}{F\PYGZhy{}F\PYGZus{}Research\PYGZus{}Data\PYGZus{}Factors\PYGZus{}daily}\PYG{l+s+s1}{\PYGZsq{}}\PYG{p}{,}
        \PYG{n}{data\PYGZus{}source}\PYG{o}{=}\PYG{l+s+s1}{\PYGZsq{}}\PYG{l+s+s1}{famafrench}\PYG{l+s+s1}{\PYGZsq{}}\PYG{p}{,}
        \PYG{n}{start}\PYG{o}{=}\PYG{l+s+s1}{\PYGZsq{}}\PYG{l+s+s1}{1900}\PYG{l+s+s1}{\PYGZsq{}}\PYG{p}{,}
        \PYG{n}{session}\PYG{o}{=}\PYG{n}{session}
    \PYG{p}{)}\PYG{p}{[}\PYG{l+m+mi}{0}\PYG{p}{]}
    \PYG{o}{.}\PYG{n}{assign}\PYG{p}{(}\PYG{n}{mkt}\PYG{o}{=}\PYG{k}{lambda} \PYG{n}{x}\PYG{p}{:} \PYG{p}{(}\PYG{n}{x}\PYG{p}{[}\PYG{l+s+s1}{\PYGZsq{}}\PYG{l+s+s1}{Mkt\PYGZhy{}RF}\PYG{l+s+s1}{\PYGZsq{}}\PYG{p}{]} \PYG{o}{+} \PYG{n}{x}\PYG{p}{[}\PYG{l+s+s1}{\PYGZsq{}}\PYG{l+s+s1}{RF}\PYG{l+s+s1}{\PYGZsq{}}\PYG{p}{]}\PYG{p}{)} \PYG{o}{/} \PYG{l+m+mi}{100}\PYG{p}{)}
    \PYG{p}{[}\PYG{l+s+s1}{\PYGZsq{}}\PYG{l+s+s1}{mkt}\PYG{l+s+s1}{\PYGZsq{}}\PYG{p}{]}
\PYG{p}{)}
\end{sphinxVerbatim}

\end{sphinxuseclass}\end{sphinxVerbatimInput}

\end{sphinxuseclass}
\sphinxAtStartPar
Next, we write a couple of helper functions.
The \sphinxcode{\sphinxupquote{get\_sample()}} function draws one random sample of \sphinxcode{\sphinxupquote{n}} observations from returns series \sphinxcode{\sphinxupquote{x}}.
This sample provides one simulated return history.

\begin{sphinxuseclass}{cell}\begin{sphinxVerbatimInput}

\begin{sphinxuseclass}{cell_input}
\begin{sphinxVerbatim}[commandchars=\\\{\}]
\PYG{k}{def} \PYG{n+nf}{get\PYGZus{}sample}\PYG{p}{(}\PYG{n}{x}\PYG{p}{,} \PYG{n}{n}\PYG{o}{=}\PYG{l+m+mi}{10\PYGZus{}000}\PYG{p}{,} \PYG{n}{seed}\PYG{o}{=}\PYG{l+m+mi}{42}\PYG{p}{,} \PYG{n}{start}\PYG{o}{=}\PYG{k+kc}{None}\PYG{p}{)}\PYG{p}{:}
    \PYG{k}{if} \PYG{n}{start} \PYG{o+ow}{is} \PYG{k+kc}{None}\PYG{p}{:}
        \PYG{n}{start} \PYG{o}{=} \PYG{n}{x}\PYG{o}{.}\PYG{n}{index}\PYG{p}{[}\PYG{o}{\PYGZhy{}}\PYG{l+m+mi}{1}\PYG{p}{]} \PYG{o}{+} \PYG{n}{pd}\PYG{o}{.}\PYG{n}{offsets}\PYG{o}{.}\PYG{n}{BDay}\PYG{p}{(}\PYG{p}{)}
    \PYG{k}{return} \PYG{p}{(}
        \PYG{n}{x}
        \PYG{o}{.}\PYG{n}{sample}\PYG{p}{(}\PYG{n}{n}\PYG{o}{=}\PYG{n}{n}\PYG{p}{,} \PYG{n}{replace}\PYG{o}{=}\PYG{k+kc}{True}\PYG{p}{,} \PYG{n}{random\PYGZus{}state}\PYG{o}{=}\PYG{n}{seed}\PYG{p}{,} \PYG{n}{ignore\PYGZus{}index}\PYG{o}{=}\PYG{k+kc}{True}\PYG{p}{)}
        \PYG{o}{.}\PYG{n}{set\PYGZus{}axis}\PYG{p}{(}\PYG{n}{pd}\PYG{o}{.}\PYG{n}{date\PYGZus{}range}\PYG{p}{(}\PYG{n}{start}\PYG{o}{=}\PYG{n}{start}\PYG{p}{,} \PYG{n}{periods}\PYG{o}{=}\PYG{n}{n}\PYG{p}{)}\PYG{p}{)}
        \PYG{o}{.}\PYG{n}{rename\PYGZus{}axis}\PYG{p}{(}\PYG{n}{index}\PYG{o}{=}\PYG{l+s+s1}{\PYGZsq{}}\PYG{l+s+s1}{Date}\PYG{l+s+s1}{\PYGZsq{}}\PYG{p}{)}
    \PYG{p}{)}    
\end{sphinxVerbatim}

\end{sphinxuseclass}\end{sphinxVerbatimInput}

\end{sphinxuseclass}
\sphinxAtStartPar
The \sphinxcode{\sphinxupquote{get\_samples()}} function calls the \sphinxcode{\sphinxupquote{get\_sample()}} function \sphinxcode{\sphinxupquote{m}} times to simulate \sphinxcode{\sphinxupquote{m}} return histories.
The \sphinxcode{\sphinxupquote{get\_samples()}} function combines these \sphinxcode{\sphinxupquote{m}} return histories into one data frame.

\begin{sphinxuseclass}{cell}\begin{sphinxVerbatimInput}

\begin{sphinxuseclass}{cell_input}
\begin{sphinxVerbatim}[commandchars=\\\{\}]
\PYG{k}{def} \PYG{n+nf}{get\PYGZus{}samples}\PYG{p}{(}\PYG{n}{x}\PYG{p}{,} \PYG{n}{m}\PYG{o}{=}\PYG{l+m+mi}{100}\PYG{p}{,} \PYG{n}{n}\PYG{o}{=}\PYG{l+m+mi}{10\PYGZus{}000}\PYG{p}{,} \PYG{n}{seed}\PYG{o}{=}\PYG{l+m+mi}{42}\PYG{p}{,} \PYG{n}{start}\PYG{o}{=}\PYG{k+kc}{None}\PYG{p}{)}\PYG{p}{:}
    \PYG{k}{return} \PYG{p}{(}
        \PYG{n}{pd}\PYG{o}{.}\PYG{n}{concat}\PYG{p}{(}
            \PYG{n}{objs}\PYG{o}{=}\PYG{p}{[}\PYG{n}{get\PYGZus{}sample}\PYG{p}{(}\PYG{n}{x}\PYG{o}{=}\PYG{n}{x}\PYG{p}{,} \PYG{n}{n}\PYG{o}{=}\PYG{n}{n}\PYG{p}{,} \PYG{n}{seed}\PYG{o}{=}\PYG{n}{seed}\PYG{o}{+}\PYG{n}{i}\PYG{p}{,} \PYG{n}{start}\PYG{o}{=}\PYG{n}{start}\PYG{p}{)} \PYG{k}{for} \PYG{n}{i} \PYG{o+ow}{in} \PYG{n+nb}{range}\PYG{p}{(}\PYG{n}{m}\PYG{p}{)}\PYG{p}{]}\PYG{p}{,}
            \PYG{n}{axis}\PYG{o}{=}\PYG{l+m+mi}{1}\PYG{p}{,}
            \PYG{n}{keys}\PYG{o}{=}\PYG{n+nb}{range}\PYG{p}{(}\PYG{n}{m}\PYG{p}{)}\PYG{p}{,}
            \PYG{n}{names}\PYG{o}{=}\PYG{l+s+s1}{\PYGZsq{}}\PYG{l+s+s1}{Sample}\PYG{l+s+s1}{\PYGZsq{}}
        \PYG{p}{)}
    \PYG{p}{)}
\end{sphinxVerbatim}

\end{sphinxuseclass}\end{sphinxVerbatimInput}

\end{sphinxuseclass}
\sphinxAtStartPar
Next, we use these helper functions to simulate 10,000 return histories of 10,000 trading days each.

\begin{sphinxuseclass}{cell}\begin{sphinxVerbatimInput}

\begin{sphinxuseclass}{cell_input}
\begin{sphinxVerbatim}[commandchars=\\\{\}]
\PYG{n}{mkts} \PYG{o}{=} \PYG{n}{get\PYGZus{}samples}\PYG{p}{(}\PYG{n}{x}\PYG{o}{=}\PYG{n}{mkt}\PYG{p}{,} \PYG{n}{m}\PYG{o}{=}\PYG{l+m+mi}{10\PYGZus{}000}\PYG{p}{,} \PYG{n}{n}\PYG{o}{=}\PYG{l+m+mi}{10\PYGZus{}000}\PYG{p}{)}

\PYG{n}{mkts}
\end{sphinxVerbatim}

\end{sphinxuseclass}\end{sphinxVerbatimInput}
\begin{sphinxVerbatimOutput}

\begin{sphinxuseclass}{cell_output}
\begin{sphinxVerbatim}[commandchars=\\\{\}]
Sample      0     1     2     3     4     5     6     7     8     9     ...  \PYGZbs{}
Date                                                                    ...   
2023\PYGZhy{}01\PYGZhy{}02  0.00  0.01  0.00  0.00  0.00  0.00 \PYGZhy{}0.00  0.00 \PYGZhy{}0.01  0.00  ...   
2023\PYGZhy{}01\PYGZhy{}03  0.01 \PYGZhy{}0.00  0.00  0.01  0.00 \PYGZhy{}0.01 \PYGZhy{}0.01  0.01  0.00 \PYGZhy{}0.01  ...   
2023\PYGZhy{}01\PYGZhy{}04 \PYGZhy{}0.03 \PYGZhy{}0.00 \PYGZhy{}0.01 \PYGZhy{}0.03 \PYGZhy{}0.00  0.02 \PYGZhy{}0.00  0.00 \PYGZhy{}0.01 \PYGZhy{}0.01  ...   
2023\PYGZhy{}01\PYGZhy{}05 \PYGZhy{}0.00 \PYGZhy{}0.00 \PYGZhy{}0.01 \PYGZhy{}0.00  0.01  0.01  0.00 \PYGZhy{}0.02  0.00 \PYGZhy{}0.01  ...   
2023\PYGZhy{}01\PYGZhy{}06 \PYGZhy{}0.00 \PYGZhy{}0.00 \PYGZhy{}0.00 \PYGZhy{}0.00 \PYGZhy{}0.01 \PYGZhy{}0.00  0.01  0.01 \PYGZhy{}0.00  0.01  ...   
...          ...   ...   ...   ...   ...   ...   ...   ...   ...   ...  ...   
2050\PYGZhy{}05\PYGZhy{}15  0.02 \PYGZhy{}0.01  0.01  0.00 \PYGZhy{}0.00 \PYGZhy{}0.00 \PYGZhy{}0.00  0.02  0.00 \PYGZhy{}0.00  ...   
2050\PYGZhy{}05\PYGZhy{}16  0.01 \PYGZhy{}0.00 \PYGZhy{}0.00  0.00  0.00 \PYGZhy{}0.01 \PYGZhy{}0.01  0.02  0.01  0.00  ...   
2050\PYGZhy{}05\PYGZhy{}17 \PYGZhy{}0.02  0.01  0.00  0.01  0.00  0.01  0.00  0.00  0.00  0.00  ...   
2050\PYGZhy{}05\PYGZhy{}18 \PYGZhy{}0.00  0.00  0.03  0.01  0.00  0.00 \PYGZhy{}0.00  0.00  0.00  0.00  ...   
2050\PYGZhy{}05\PYGZhy{}19  0.01 \PYGZhy{}0.02  0.00  0.01  0.00 \PYGZhy{}0.00  0.01 \PYGZhy{}0.01  0.00 \PYGZhy{}0.00  ...   

Sample      9990  9991  9992  9993  9994  9995  9996  9997  9998  9999  
Date                                                                    
2023\PYGZhy{}01\PYGZhy{}02  0.00 \PYGZhy{}0.00 \PYGZhy{}0.00  0.00  0.00  0.01 \PYGZhy{}0.00 \PYGZhy{}0.01 \PYGZhy{}0.00  0.01  
2023\PYGZhy{}01\PYGZhy{}03  0.00  0.01  0.01  0.01 \PYGZhy{}0.00  0.00  0.00 \PYGZhy{}0.01 \PYGZhy{}0.00  0.02  
2023\PYGZhy{}01\PYGZhy{}04 \PYGZhy{}0.01  0.01 \PYGZhy{}0.01 \PYGZhy{}0.01  0.01  0.00  0.01  0.01  0.00  0.02  
2023\PYGZhy{}01\PYGZhy{}05  0.01  0.02  0.03 \PYGZhy{}0.00 \PYGZhy{}0.00  0.01  0.01 \PYGZhy{}0.00  0.02  0.00  
2023\PYGZhy{}01\PYGZhy{}06  0.01  0.02  0.01 \PYGZhy{}0.01  0.02  0.01 \PYGZhy{}0.00  0.00  0.00  0.02  
...          ...   ...   ...   ...   ...   ...   ...   ...   ...   ...  
2050\PYGZhy{}05\PYGZhy{}15 \PYGZhy{}0.00 \PYGZhy{}0.01  0.02  0.00  0.00 \PYGZhy{}0.01 \PYGZhy{}0.02  0.01 \PYGZhy{}0.01 \PYGZhy{}0.04  
2050\PYGZhy{}05\PYGZhy{}16  0.00 \PYGZhy{}0.00  0.00 \PYGZhy{}0.00 \PYGZhy{}0.01 \PYGZhy{}0.00  0.01  0.00 \PYGZhy{}0.02  0.00  
2050\PYGZhy{}05\PYGZhy{}17  0.00 \PYGZhy{}0.00  0.00  0.00 \PYGZhy{}0.00  0.01 \PYGZhy{}0.00  0.01 \PYGZhy{}0.02 \PYGZhy{}0.00  
2050\PYGZhy{}05\PYGZhy{}18 \PYGZhy{}0.00  0.02  0.01 \PYGZhy{}0.00  0.00  0.00  0.00  0.01  0.00 \PYGZhy{}0.00  
2050\PYGZhy{}05\PYGZhy{}19  0.00 \PYGZhy{}0.00 \PYGZhy{}0.01 \PYGZhy{}0.00 \PYGZhy{}0.00 \PYGZhy{}0.00  0.00  0.01  0.01  0.01  

[10000 rows x 10000 columns]
\end{sphinxVerbatim}

\end{sphinxuseclass}\end{sphinxVerbatimOutput}

\end{sphinxuseclass}
\sphinxAtStartPar
We compound daily returns \$R\_t\$ to find future wealth \$W\_T\$, so \$W\_T = W\_0 \textbackslash{}times (1 + R\_1) \textbackslash{}times (1 + R\_2) \textbackslash{}times \textbackslash{}cdots \textbackslash{}times (1 + R\_T)\$.
We use the \sphinxcode{\sphinxupquote{.cumprod()}} method to compound the daily returns in data frame \sphinxcode{\sphinxupquote{mkts}}.

\begin{sphinxuseclass}{cell}\begin{sphinxVerbatimInput}

\begin{sphinxuseclass}{cell_input}
\begin{sphinxVerbatim}[commandchars=\\\{\}]
\PYG{n}{W\PYGZus{}0} \PYG{o}{=} \PYG{l+m+mi}{1\PYGZus{}000\PYGZus{}000}
\PYG{n}{W\PYGZus{}t} \PYG{o}{=} \PYG{n}{W\PYGZus{}0} \PYG{o}{*} \PYG{n}{mkts}\PYG{o}{.}\PYG{n}{add}\PYG{p}{(}\PYG{l+m+mi}{1}\PYG{p}{)}\PYG{o}{.}\PYG{n}{cumprod}\PYG{p}{(}\PYG{p}{)}
\end{sphinxVerbatim}

\end{sphinxuseclass}\end{sphinxVerbatimInput}

\end{sphinxuseclass}
\sphinxAtStartPar
We visualize terminal wealth \$W\_T\$ with a cumulative distribution plot.

\begin{sphinxuseclass}{cell}\begin{sphinxVerbatimInput}

\begin{sphinxuseclass}{cell_input}
\begin{sphinxVerbatim}[commandchars=\\\{\}]
\PYG{n}{W\PYGZus{}t}\PYG{o}{.}\PYG{n}{iloc}\PYG{p}{[}\PYG{o}{\PYGZhy{}}\PYG{l+m+mi}{1}\PYG{p}{]}\PYG{o}{.}\PYG{n}{plot}\PYG{p}{(}\PYG{n}{kind}\PYG{o}{=}\PYG{l+s+s1}{\PYGZsq{}}\PYG{l+s+s1}{hist}\PYG{l+s+s1}{\PYGZsq{}}\PYG{p}{,} \PYG{n}{density}\PYG{o}{=}\PYG{k+kc}{True}\PYG{p}{,} \PYG{n}{cumulative}\PYG{o}{=}\PYG{k+kc}{True}\PYG{p}{,} \PYG{n}{bins}\PYG{o}{=}\PYG{l+m+mi}{1\PYGZus{}000}\PYG{p}{)}
\PYG{n}{plt}\PYG{o}{.}\PYG{n}{semilogx}\PYG{p}{(}\PYG{p}{)}
\PYG{n}{plt}\PYG{o}{.}\PYG{n}{xlabel}\PYG{p}{(}\PYG{l+s+sa}{r}\PYG{l+s+s1}{\PYGZsq{}}\PYG{l+s+s1}{\PYGZdl{}W\PYGZus{}T\PYGZdl{} where \PYGZdl{}W\PYGZus{}0 = \PYGZdl{}}\PYG{l+s+s1}{\PYGZsq{}} \PYG{o}{+} \PYG{l+s+sa}{f}\PYG{l+s+s1}{\PYGZsq{}}\PYG{l+s+s1}{\PYGZbs{}}\PYG{l+s+s1}{\PYGZdl{}}\PYG{l+s+si}{\PYGZob{}}\PYG{n}{W\PYGZus{}0}\PYG{l+s+si}{:}\PYG{l+s+s1}{,.0f}\PYG{l+s+si}{\PYGZcb{}}\PYG{l+s+s1}{\PYGZsq{}}\PYG{p}{)}
\PYG{n}{plt}\PYG{o}{.}\PYG{n}{title}\PYG{p}{(}
    \PYG{l+s+sa}{r}\PYG{l+s+s1}{\PYGZsq{}}\PYG{l+s+s1}{Cumulative Distribution of \PYGZdl{}W\PYGZus{}T\PYGZdl{}}\PYG{l+s+s1}{\PYGZsq{}} \PYG{o}{+} 
    \PYG{l+s+sa}{f}\PYG{l+s+s1}{\PYGZsq{}}\PYG{l+s+se}{\PYGZbs{}n}\PYG{l+s+s1}{after }\PYG{l+s+si}{\PYGZob{}}\PYG{n}{W\PYGZus{}t}\PYG{o}{.}\PYG{n}{shape}\PYG{p}{[}\PYG{l+m+mi}{0}\PYG{p}{]}\PYG{l+s+si}{:}\PYG{l+s+s1}{,.0f}\PYG{l+s+si}{\PYGZcb{}}\PYG{l+s+s1}{ Trading Days}\PYG{l+s+s1}{\PYGZsq{}}
\PYG{p}{)}
\PYG{n}{plt}\PYG{o}{.}\PYG{n}{show}\PYG{p}{(}\PYG{p}{)}
\end{sphinxVerbatim}

\end{sphinxuseclass}\end{sphinxVerbatimInput}
\begin{sphinxVerbatimOutput}

\begin{sphinxuseclass}{cell_output}
\noindent\sphinxincludegraphics{{0211c97349d8ed9941ef650fa751fb416ffaa07a7972ee92543d9abd864642d2}.png}

\end{sphinxuseclass}\end{sphinxVerbatimOutput}

\end{sphinxuseclass}
\sphinxAtStartPar
The plot above shows the cumulative distribution of \$W\_T\$, suggesting we expect \$W\_T\$ greater than about \$17 million for about 75\% of samples.

\sphinxAtStartPar
The plot above may be difficult to read and interpret because \$W\_T\$ has large outliers.
The \sphinxcode{\sphinxupquote{.describe()}} method provides few salient values of \$W\_T\$.
We convert \$W\_T\$ from dollars to millions of dollars with \sphinxcode{\sphinxupquote{.div(1\_000\_000)}}.

\begin{sphinxuseclass}{cell}\begin{sphinxVerbatimInput}

\begin{sphinxuseclass}{cell_input}
\begin{sphinxVerbatim}[commandchars=\\\{\}]
\PYG{n}{W\PYGZus{}t}\PYG{o}{.}\PYG{n}{iloc}\PYG{p}{[}\PYG{o}{\PYGZhy{}}\PYG{l+m+mi}{1}\PYG{p}{]}\PYG{o}{.}\PYG{n}{div}\PYG{p}{(}\PYG{l+m+mi}{1\PYGZus{}000\PYGZus{}000}\PYG{p}{)}\PYG{o}{.}\PYG{n}{describe}\PYG{p}{(}\PYG{p}{)}
\end{sphinxVerbatim}

\end{sphinxuseclass}\end{sphinxVerbatimInput}
\begin{sphinxVerbatimOutput}

\begin{sphinxuseclass}{cell_output}
\begin{sphinxVerbatim}[commandchars=\\\{\}]
count   10000.00
mean       64.75
std       101.61
min         0.56
25\PYGZpc{}        17.56
50\PYGZpc{}        36.11
75\PYGZpc{}        74.11
max      3330.53
Name: 2050\PYGZhy{}05\PYGZhy{}19 00:00:00, dtype: float64
\end{sphinxVerbatim}

\end{sphinxuseclass}\end{sphinxVerbatimOutput}

\end{sphinxuseclass}

\subsubsection{Repeat the exercise above but add end\sphinxhyphen{}of\sphinxhyphen{}month investments \$C\_t\$}
\label{\detokenize{herron_05_practice_02:repeat-the-exercise-above-but-add-end-of-month-investments-c-t}}
\sphinxAtStartPar
We can use the same data frame \sphinxcode{\sphinxupquote{mkts}} of simulated market returns.
However, need to consider end\sphinxhyphen{}of\sphinxhyphen{}month investments of \$C\_t\$.
The easiest approach is to aggregate the daily market returns in \sphinxcode{\sphinxupquote{mkts}} to monthly market returns in data frame \sphinxcode{\sphinxupquote{mkts\_m}}.
The last month in \sphinxcode{\sphinxupquote{mkts\_m}} is not complete (i.e., fewer than about 21 trading days), so we drop is with \sphinxcode{\sphinxupquote{.iloc{[}:\sphinxhyphen{}1{]}}}.

\begin{sphinxuseclass}{cell}\begin{sphinxVerbatimInput}

\begin{sphinxuseclass}{cell_input}
\begin{sphinxVerbatim}[commandchars=\\\{\}]
\PYG{n}{mkts\PYGZus{}m} \PYG{o}{=} \PYG{p}{(}
    \PYG{n}{mkts}
    \PYG{o}{.}\PYG{n}{add}\PYG{p}{(}\PYG{l+m+mi}{1}\PYG{p}{)}
    \PYG{o}{.}\PYG{n}{resample}\PYG{p}{(}\PYG{n}{rule}\PYG{o}{=}\PYG{l+s+s1}{\PYGZsq{}}\PYG{l+s+s1}{M}\PYG{l+s+s1}{\PYGZsq{}}\PYG{p}{,} \PYG{n}{kind}\PYG{o}{=}\PYG{l+s+s1}{\PYGZsq{}}\PYG{l+s+s1}{period}\PYG{l+s+s1}{\PYGZsq{}}\PYG{p}{)}
    \PYG{o}{.}\PYG{n}{prod}\PYG{p}{(}\PYG{p}{)}
    \PYG{o}{.}\PYG{n}{sub}\PYG{p}{(}\PYG{l+m+mi}{1}\PYG{p}{)}
    \PYG{o}{.}\PYG{n}{iloc}\PYG{p}{[}\PYG{p}{:}\PYG{o}{\PYGZhy{}}\PYG{l+m+mi}{1}\PYG{p}{]}
\PYG{p}{)}
\end{sphinxVerbatim}

\end{sphinxuseclass}\end{sphinxVerbatimInput}

\end{sphinxuseclass}
\sphinxAtStartPar
The wealth at time \$t\$ is \$W\_t\$ and depends on:
\begin{enumerate}
\sphinxsetlistlabels{\arabic}{enumi}{enumii}{}{.}%
\item {} 
\sphinxAtStartPar
The wealth at the end of the previous month \$W\_\{t\sphinxhyphen{}1\}\$

\item {} 
\sphinxAtStartPar
The return over the previous month \$R\_t\$ (recall we label returns by their right edge)

\item {} 
\sphinxAtStartPar
The end\sphinxhyphen{}of\sphinxhyphen{}month investment \$C\_t\$

\end{enumerate}

\sphinxAtStartPar
Putting it all togther: \$W\_t = W\_\{t\sphinxhyphen{}1\} \textbackslash{}times (1 + R\_t) + C\_t\$

\sphinxAtStartPar
We have to loop over the monthly returns in \sphinxcode{\sphinxupquote{mkts\_m}} because we have to combine compounded returns and cash flows.
The \sphinxcode{\sphinxupquote{iterrows()}} methods provides an easy way to iterate (loop) over the rows in \sphinxcode{\sphinxupquote{mkts\_m}}.

\begin{sphinxuseclass}{cell}\begin{sphinxVerbatimInput}

\begin{sphinxuseclass}{cell_input}
\begin{sphinxVerbatim}[commandchars=\\\{\}]
\PYG{n}{C\PYGZus{}t} \PYG{o}{=} \PYG{l+m+mi}{1\PYGZus{}000}
\PYG{n}{W\PYGZus{}0} \PYG{o}{=} \PYG{l+m+mi}{1\PYGZus{}000\PYGZus{}000}
\PYG{n}{W\PYGZus{}last} \PYG{o}{=} \PYG{n}{W\PYGZus{}0}
\PYG{n}{W\PYGZus{}t} \PYG{o}{=} \PYG{p}{[}\PYG{p}{]}

\PYG{k}{for} \PYG{n}{d}\PYG{p}{,} \PYG{n}{m} \PYG{o+ow}{in} \PYG{n}{mkts\PYGZus{}m}\PYG{o}{.}\PYG{n}{iterrows}\PYG{p}{(}\PYG{p}{)}\PYG{p}{:}
    \PYG{n}{W\PYGZus{}last} \PYG{o}{=} \PYG{p}{(}\PYG{l+m+mi}{1} \PYG{o}{+} \PYG{n}{m}\PYG{p}{)} \PYG{o}{*} \PYG{n}{W\PYGZus{}last} \PYG{o}{+} \PYG{n}{C\PYGZus{}t}
    \PYG{n}{W\PYGZus{}t}\PYG{o}{.}\PYG{n}{append}\PYG{p}{(}\PYG{n}{W\PYGZus{}last}\PYG{p}{)}

\PYG{n}{W\PYGZus{}t} \PYG{o}{=} \PYG{n}{pd}\PYG{o}{.}\PYG{n}{concat}\PYG{p}{(}\PYG{n}{objs}\PYG{o}{=}\PYG{n}{W\PYGZus{}t}\PYG{p}{,} \PYG{n}{axis}\PYG{o}{=}\PYG{l+m+mi}{1}\PYG{p}{,} \PYG{n}{keys}\PYG{o}{=}\PYG{n}{mkts\PYGZus{}m}\PYG{o}{.}\PYG{n}{index}\PYG{p}{)}\PYG{o}{.}\PYG{n}{transpose}\PYG{p}{(}\PYG{p}{)}
\end{sphinxVerbatim}

\end{sphinxuseclass}\end{sphinxVerbatimInput}

\end{sphinxuseclass}
\sphinxAtStartPar
We repeat the cumulative distribution of wealth plot and descriptive statistics from above.

\begin{sphinxuseclass}{cell}\begin{sphinxVerbatimInput}

\begin{sphinxuseclass}{cell_input}
\begin{sphinxVerbatim}[commandchars=\\\{\}]
\PYG{n}{W\PYGZus{}t}\PYG{o}{.}\PYG{n}{iloc}\PYG{p}{[}\PYG{o}{\PYGZhy{}}\PYG{l+m+mi}{1}\PYG{p}{]}\PYG{o}{.}\PYG{n}{plot}\PYG{p}{(}\PYG{n}{kind}\PYG{o}{=}\PYG{l+s+s1}{\PYGZsq{}}\PYG{l+s+s1}{hist}\PYG{l+s+s1}{\PYGZsq{}}\PYG{p}{,} \PYG{n}{density}\PYG{o}{=}\PYG{k+kc}{True}\PYG{p}{,} \PYG{n}{cumulative}\PYG{o}{=}\PYG{k+kc}{True}\PYG{p}{,} \PYG{n}{bins}\PYG{o}{=}\PYG{l+m+mi}{1\PYGZus{}000}\PYG{p}{)}
\PYG{n}{plt}\PYG{o}{.}\PYG{n}{semilogx}\PYG{p}{(}\PYG{p}{)}
\PYG{n}{plt}\PYG{o}{.}\PYG{n}{xlabel}\PYG{p}{(}
    \PYG{l+s+sa}{r}\PYG{l+s+s1}{\PYGZsq{}}\PYG{l+s+s1}{\PYGZdl{}W\PYGZus{}T\PYGZdl{} where \PYGZdl{}W\PYGZus{}0 = \PYGZdl{}}\PYG{l+s+s1}{\PYGZsq{}} \PYG{o}{+} \PYG{l+s+sa}{f}\PYG{l+s+s1}{\PYGZsq{}}\PYG{l+s+s1}{\PYGZbs{}}\PYG{l+s+s1}{\PYGZdl{}}\PYG{l+s+si}{\PYGZob{}}\PYG{n}{W\PYGZus{}0}\PYG{l+s+si}{:}\PYG{l+s+s1}{,.0f}\PYG{l+s+si}{\PYGZcb{}}\PYG{l+s+s1}{\PYGZsq{}} \PYG{o}{+} 
    \PYG{l+s+sa}{r}\PYG{l+s+s1}{\PYGZsq{}}\PYG{l+s+s1}{ and  \PYGZdl{}C\PYGZus{}t = \PYGZdl{}}\PYG{l+s+s1}{\PYGZsq{}} \PYG{o}{+} \PYG{l+s+sa}{f}\PYG{l+s+s1}{\PYGZsq{}}\PYG{l+s+s1}{\PYGZbs{}}\PYG{l+s+s1}{\PYGZdl{}}\PYG{l+s+si}{\PYGZob{}}\PYG{n}{C\PYGZus{}t}\PYG{l+s+si}{:}\PYG{l+s+s1}{,.0f}\PYG{l+s+si}{\PYGZcb{}}\PYG{l+s+s1}{\PYGZsq{}}
\PYG{p}{)}
\PYG{n}{plt}\PYG{o}{.}\PYG{n}{title}\PYG{p}{(}
    \PYG{l+s+sa}{r}\PYG{l+s+s1}{\PYGZsq{}}\PYG{l+s+s1}{Cumulative Distribution of \PYGZdl{}W\PYGZus{}T\PYGZdl{}}\PYG{l+s+s1}{\PYGZsq{}} \PYG{o}{+} 
    \PYG{l+s+sa}{f}\PYG{l+s+s1}{\PYGZsq{}}\PYG{l+s+se}{\PYGZbs{}n}\PYG{l+s+s1}{after }\PYG{l+s+si}{\PYGZob{}}\PYG{n}{W\PYGZus{}t}\PYG{o}{.}\PYG{n}{shape}\PYG{p}{[}\PYG{l+m+mi}{0}\PYG{p}{]}\PYG{l+s+si}{:}\PYG{l+s+s1}{,.0f}\PYG{l+s+si}{\PYGZcb{}}\PYG{l+s+s1}{ Trading Months}\PYG{l+s+s1}{\PYGZsq{}}
\PYG{p}{)}
\PYG{n}{plt}\PYG{o}{.}\PYG{n}{show}\PYG{p}{(}\PYG{p}{)}
\end{sphinxVerbatim}

\end{sphinxuseclass}\end{sphinxVerbatimInput}
\begin{sphinxVerbatimOutput}

\begin{sphinxuseclass}{cell_output}
\noindent\sphinxincludegraphics{{6d78aab7a506358bd5a775b38f3ef6930893fe48e082ff3b0dfee6a188e7bc80}.png}

\end{sphinxuseclass}\end{sphinxVerbatimOutput}

\end{sphinxuseclass}
\begin{sphinxuseclass}{cell}\begin{sphinxVerbatimInput}

\begin{sphinxuseclass}{cell_input}
\begin{sphinxVerbatim}[commandchars=\\\{\}]
\PYG{n}{W\PYGZus{}t}\PYG{o}{.}\PYG{n}{iloc}\PYG{p}{[}\PYG{o}{\PYGZhy{}}\PYG{l+m+mi}{1}\PYG{p}{]}\PYG{o}{.}\PYG{n}{div}\PYG{p}{(}\PYG{l+m+mi}{1\PYGZus{}000\PYGZus{}000}\PYG{p}{)}\PYG{o}{.}\PYG{n}{describe}\PYG{p}{(}\PYG{p}{)}
\end{sphinxVerbatim}

\end{sphinxuseclass}\end{sphinxVerbatimInput}
\begin{sphinxVerbatimOutput}

\begin{sphinxuseclass}{cell_output}
\begin{sphinxVerbatim}[commandchars=\\\{\}]
count   10000.00
mean       69.20
std       106.22
min         0.94
25\PYGZpc{}        19.59
50\PYGZpc{}        39.27
75\PYGZpc{}        79.69
max      3467.92
Name: 2050\PYGZhy{}04, dtype: float64
\end{sphinxVerbatim}

\end{sphinxuseclass}\end{sphinxVerbatimOutput}

\end{sphinxuseclass}
\sphinxAtStartPar
The plot above shows the cumulative distribution of \$W\_T\$, suggesting we expect \$W\_T\$ greater than about \textbackslash{}\$20 million for about 75\% of samples.
Note, even though we deposit \$1,000 per month for almost 40 years, we do not gain much wealth over the previous example with only a lump sum!
Start saving early!

\sphinxstepscope


\chapter{Herron Topic 6 \sphinxhyphen{} Size, Value, and Momentum Investing}
\label{\detokenize{herron_06_lecture:herron-topic-6-size-value-and-momentum-investing}}\label{\detokenize{herron_06_lecture::doc}}
\sphinxAtStartPar
This lecture notebook covers momentum investing, and the practice notebook will apply the tools we learn in this notebook to size and value investing.
Here, we will learn:
\begin{enumerate}
\sphinxsetlistlabels{\arabic}{enumi}{enumii}{}{.}%
\item {} 
\sphinxAtStartPar
What is momentum investing?

\item {} 
\sphinxAtStartPar
How to use Center for Research in Security Prices (CRSP) data, which is survivorship bias free

\item {} 
\sphinxAtStartPar
How to implement and evaluate a momentum strategy

\end{enumerate}

\begin{sphinxuseclass}{cell}\begin{sphinxVerbatimInput}

\begin{sphinxuseclass}{cell_input}
\begin{sphinxVerbatim}[commandchars=\\\{\}]
\PYG{k+kn}{import} \PYG{n+nn}{matplotlib}\PYG{n+nn}{.}\PYG{n+nn}{pyplot} \PYG{k}{as} \PYG{n+nn}{plt}
\PYG{k+kn}{import} \PYG{n+nn}{numpy} \PYG{k}{as} \PYG{n+nn}{np}
\PYG{k+kn}{import} \PYG{n+nn}{pandas} \PYG{k}{as} \PYG{n+nn}{pd}
\end{sphinxVerbatim}

\end{sphinxuseclass}\end{sphinxVerbatimInput}

\end{sphinxuseclass}
\begin{sphinxuseclass}{cell}\begin{sphinxVerbatimInput}

\begin{sphinxuseclass}{cell_input}
\begin{sphinxVerbatim}[commandchars=\\\{\}]
\PYG{o}{\PYGZpc{}}\PYG{k}{config} InlineBackend.figure\PYGZus{}format = \PYGZsq{}retina\PYGZsq{}
\PYG{o}{\PYGZpc{}}\PYG{k}{precision} 4
\PYG{n}{pd}\PYG{o}{.}\PYG{n}{options}\PYG{o}{.}\PYG{n}{display}\PYG{o}{.}\PYG{n}{float\PYGZus{}format} \PYG{o}{=} \PYG{l+s+s1}{\PYGZsq{}}\PYG{l+s+si}{\PYGZob{}:.4f\PYGZcb{}}\PYG{l+s+s1}{\PYGZsq{}}\PYG{o}{.}\PYG{n}{format}
\end{sphinxVerbatim}

\end{sphinxuseclass}\end{sphinxVerbatimInput}

\end{sphinxuseclass}
\begin{sphinxuseclass}{cell}\begin{sphinxVerbatimInput}

\begin{sphinxuseclass}{cell_input}
\begin{sphinxVerbatim}[commandchars=\\\{\}]
\PYG{k+kn}{import} \PYG{n+nn}{yfinance} \PYG{k}{as} \PYG{n+nn}{yf}
\PYG{k+kn}{import} \PYG{n+nn}{pandas\PYGZus{}datareader} \PYG{k}{as} \PYG{n+nn}{pdr}
\PYG{k+kn}{import} \PYG{n+nn}{requests\PYGZus{}cache}
\PYG{n}{session} \PYG{o}{=} \PYG{n}{requests\PYGZus{}cache}\PYG{o}{.}\PYG{n}{CachedSession}\PYG{p}{(}\PYG{p}{)}
\end{sphinxVerbatim}

\end{sphinxuseclass}\end{sphinxVerbatimInput}

\end{sphinxuseclass}

\section{What is momentum investing?}
\label{\detokenize{herron_06_lecture:what-is-momentum-investing}}
\sphinxAtStartPar
From page 13 of \sphinxhref{https://book.ivo-welch.info/read/source5.mba/12-effbehav.pdf}{chapter 12} of Ivo Welch’s free corporate finance textbook (\sphinxstyleemphasis{\sphinxstylestrong{emphasis added}}):
\begin{quote}

\sphinxAtStartPar
The second\sphinxhyphen{}most important factor anomaly was the momentum investing strategy.
\sphinxstyleemphasis{\sphinxstylestrong{Momentum investing strategies require going long in stocks that have increased
greatly over the last year, and going short in stocks that have decreased greatly.}}
(It requires a few more contortions, but this is a reasonable characterization.) As
with value, behavioral finance researchers were quick to adopt momentum as a
consequence of investor psychology. They also developed plenty of theories that
argued about how the psychology of investors could explain momentum.

\sphinxAtStartPar
Yet over the last 17 years, Ken French’s data suggests that the average rate of Momentum has been mostly
return on the momentum investment strategy was — drumroll — 0.03\% with a
standard deviation of 23.8\%. This rate of return is statistically and economically
insignificant. Momentum investing also had the unpleasant aspect of sudden nasty
risk. It lost 83 cents for every dollar invested in 2009
\end{quote}


\section{CRSP Data}
\label{\detokenize{herron_06_lecture:crsp-data}}
\sphinxAtStartPar
We typically use Yahoo! Finance data in class because these data are easy to download and use.
However, Yahoo! Finance only provides data for listed (active) public companies.
When public companies delist, Yahoo! Finance removes their data from its website and application programming interfaces (APIs).
Companies delist for various reasons, including failures, poor performance, buyouts, and acquisitions.
Failures and poor performance are generally associated with large negative returns before delisting.
Buyouts and acquisitions are generally associated with large positive returns before delisting.
Regardless of the reason for delisting, delisted company data are unavailable from Yahoo! Finance.
Because delistings are not randomly assigned and typically related to past performance, we cannot ignore them.
If we ignore delistings, we create a \sphinxhref{https://en.wikipedia.org/wiki/Survivorship\_bias}{survivorship bias}:
\begin{quote}

\sphinxAtStartPar
Survivorship bias, survival bias or immortal time bias is the logical error of concentrating on the people or things that made it past some selection process and overlooking those that did not, typically because of their lack of visibility. This can lead to incorrect conclusions regarding that which (or those who) didn’t make it.

\sphinxAtStartPar
Survivorship bias is a form of selection bias that can lead to overly optimistic beliefs because multiple failures are overlooked, such as when companies that no longer exist are excluded from analyses of financial performance. It can also lead to the false belief that the successes in a group have some special property, rather than just coincidence as in correlation “proves” causality. For example, if 3 of the 5 students with their state’s highest college GPAs went to the same high school, it might lead to the notion (which the institution may even capitalize on through advertising) that their high school offers an excellent education even though it’s actually due to their school being the largest in their state. Therefore, by comparing the average GPA of all of the school’s students — not just the ones who made the top\sphinxhyphen{}five selection process — to state averages, one could better assess the school’s quality (not quantity).

\sphinxAtStartPar
Another kind of survivorship bias would involve thinking that an incident was not all that dangerous because the only people who were involved in the incident who can speak about it are those who survived it. Even if one knew that some people are dead, they would not have their voice to add to the conversation, leading to bias in the conversation.
\end{quote}

\sphinxAtStartPar
We should always be on the lookout for survivorship bias!
Here is my favorite survivorship bias joke:

\sphinxAtStartPar
\sphinxincludegraphics{{survivorship_bias}.png}

\sphinxAtStartPar
To avoid a survivorship bias, we will use survivorship\sphinxhyphen{}bias\sphinxhyphen{}free data from the \sphinxhref{https://www.crsp.org/}{Center for Research in Security Prices (CRSP)}.
CRSP data include delisted stocks and are used by academics and institutional investors to research and backtest trading strategies.
Download the CRSP data file \sphinxcode{\sphinxupquote{crsp.csv}} from Canvas, and put it in the same folder as this notebook.
Then, we can read and clean the CRSP data file as follows.

\begin{sphinxuseclass}{cell}\begin{sphinxVerbatimInput}

\begin{sphinxuseclass}{cell_input}
\begin{sphinxVerbatim}[commandchars=\\\{\}]
\PYG{n}{crsp} \PYG{o}{=} \PYG{p}{(}
    \PYG{n}{pd}\PYG{o}{.}\PYG{n}{read\PYGZus{}csv}\PYG{p}{(}
        \PYG{n}{filepath\PYGZus{}or\PYGZus{}buffer}\PYG{o}{=}\PYG{l+s+s1}{\PYGZsq{}}\PYG{l+s+s1}{crsp.csv}\PYG{l+s+s1}{\PYGZsq{}}\PYG{p}{,}
        \PYG{n}{parse\PYGZus{}dates}\PYG{o}{=}\PYG{p}{[}\PYG{l+s+s1}{\PYGZsq{}}\PYG{l+s+s1}{date}\PYG{l+s+s1}{\PYGZsq{}}\PYG{p}{]}\PYG{p}{,}
        \PYG{n}{na\PYGZus{}values}\PYG{o}{=}\PYG{p}{[}\PYG{l+s+s1}{\PYGZsq{}}\PYG{l+s+s1}{A}\PYG{l+s+s1}{\PYGZsq{}}\PYG{p}{,} \PYG{l+s+s1}{\PYGZsq{}}\PYG{l+s+s1}{B}\PYG{l+s+s1}{\PYGZsq{}}\PYG{p}{,} \PYG{l+s+s1}{\PYGZsq{}}\PYG{l+s+s1}{C}\PYG{l+s+s1}{\PYGZsq{}}\PYG{p}{]} \PYG{c+c1}{\PYGZsh{} CRSP uses letter codes to provide additional information, which we can ignore}
    \PYG{p}{)}
    \PYG{o}{.}\PYG{n}{assign}\PYG{p}{(}\PYG{n}{date}\PYG{o}{=}\PYG{k}{lambda} \PYG{n}{x}\PYG{p}{:} \PYG{n}{x}\PYG{p}{[}\PYG{l+s+s1}{\PYGZsq{}}\PYG{l+s+s1}{date}\PYG{l+s+s1}{\PYGZsq{}}\PYG{p}{]}\PYG{o}{.}\PYG{n}{dt}\PYG{o}{.}\PYG{n}{to\PYGZus{}period}\PYG{p}{(}\PYG{n}{freq}\PYG{o}{=}\PYG{l+s+s1}{\PYGZsq{}}\PYG{l+s+s1}{M}\PYG{l+s+s1}{\PYGZsq{}}\PYG{p}{)}\PYG{p}{)} \PYG{c+c1}{\PYGZsh{} returns span a month, so I prefer to work with periods instead of dates}
    \PYG{o}{.}\PYG{n}{rename\PYGZus{}axis}\PYG{p}{(}\PYG{n}{columns}\PYG{o}{=}\PYG{l+s+s1}{\PYGZsq{}}\PYG{l+s+s1}{Variable}\PYG{l+s+s1}{\PYGZsq{}}\PYG{p}{)}
    \PYG{o}{.}\PYG{n}{set\PYGZus{}index}\PYG{p}{(}\PYG{p}{[}\PYG{l+s+s1}{\PYGZsq{}}\PYG{l+s+s1}{PERMNO}\PYG{l+s+s1}{\PYGZsq{}}\PYG{p}{,} \PYG{l+s+s1}{\PYGZsq{}}\PYG{l+s+s1}{date}\PYG{l+s+s1}{\PYGZsq{}}\PYG{p}{]}\PYG{p}{)}
\PYG{p}{)}
\end{sphinxVerbatim}

\end{sphinxuseclass}\end{sphinxVerbatimInput}

\end{sphinxuseclass}
\sphinxAtStartPar
The CRSP data files contains a small subset of all available CRSP data:
\begin{enumerate}
\sphinxsetlistlabels{\arabic}{enumi}{enumii}{}{.}%
\item {} 
\sphinxAtStartPar
\sphinxcode{\sphinxupquote{PERMNO}} indicates permanent security identifiers, which is more reliable than tickers, which may change or be re\sphinxhyphen{}used by a different firm

\item {} 
\sphinxAtStartPar
\sphinxcode{\sphinxupquote{date}} indicates the last trading day of the month, which we convert to a year\sphinxhyphen{}month “period”

\item {} 
\sphinxAtStartPar
\sphinxcode{\sphinxupquote{SHRCD}} indicates share classes (e.g., A, B, and preferred), which I filtered to values of 10 or 11

\item {} 
\sphinxAtStartPar
\sphinxcode{\sphinxupquote{PRC}} indicates the closing price on the last trading day of the month, and \sphinxcode{\sphinxupquote{PRC}} is negative if it is the mean of the bid and ask prices instead of a price from an observed trade

\item {} 
\sphinxAtStartPar
\sphinxcode{\sphinxupquote{RET}} indicates the holding period return as a simple return, including dividends

\item {} 
\sphinxAtStartPar
\sphinxcode{\sphinxupquote{SHROUT}} indicates the number of shares outstanding in thousands (e.g., a \sphinxcode{\sphinxupquote{SHROUT}} value of 1,000 indicates 1,000,000 shares)

\end{enumerate}

\begin{sphinxuseclass}{cell}\begin{sphinxVerbatimInput}

\begin{sphinxuseclass}{cell_input}
\begin{sphinxVerbatim}[commandchars=\\\{\}]
\PYG{n}{crsp}\PYG{o}{.}\PYG{n}{head}\PYG{p}{(}\PYG{p}{)}
\end{sphinxVerbatim}

\end{sphinxuseclass}\end{sphinxVerbatimInput}
\begin{sphinxVerbatimOutput}

\begin{sphinxuseclass}{cell_output}
\begin{sphinxVerbatim}[commandchars=\\\{\}]
Variable        SHRCD     PRC     RET    SHROUT
PERMNO date                                    
10000  1986\PYGZhy{}01     10 \PYGZhy{}4.3750     NaN 3680.0000
       1986\PYGZhy{}02     10 \PYGZhy{}3.2500 \PYGZhy{}0.2571 3680.0000
       1986\PYGZhy{}03     10 \PYGZhy{}4.4375  0.3654 3680.0000
       1986\PYGZhy{}04     10 \PYGZhy{}4.0000 \PYGZhy{}0.0986 3793.0000
       1986\PYGZhy{}05     10 \PYGZhy{}3.1094 \PYGZhy{}0.2227 3793.0000
\end{sphinxVerbatim}

\end{sphinxuseclass}\end{sphinxVerbatimOutput}

\end{sphinxuseclass}
\sphinxAtStartPar
These CRSP data are de\sphinxhyphen{}duplicated, but we should double\sphinxhyphen{}check for duplicate PERMNO\sphinxhyphen{}date pairs with the \sphinxcode{\sphinxupquote{.duplicated()}} method.

\begin{sphinxuseclass}{cell}\begin{sphinxVerbatimInput}

\begin{sphinxuseclass}{cell_input}
\begin{sphinxVerbatim}[commandchars=\\\{\}]
\PYG{k}{assert} \PYG{o+ow}{not} \PYG{n}{crsp}\PYG{o}{.}\PYG{n}{index}\PYG{o}{.}\PYG{n}{duplicated}\PYG{p}{(}\PYG{p}{)}\PYG{o}{.}\PYG{n}{any}\PYG{p}{(}\PYG{p}{)}
\end{sphinxVerbatim}

\end{sphinxuseclass}\end{sphinxVerbatimInput}

\end{sphinxuseclass}
\sphinxAtStartPar
For fun, we can use these data to count the number of listed stocks each month and plot the trend of publicly listed stocks.

\begin{sphinxuseclass}{cell}\begin{sphinxVerbatimInput}

\begin{sphinxuseclass}{cell_input}
\begin{sphinxVerbatim}[commandchars=\\\{\}]
\PYG{n}{crsp}\PYG{o}{.}\PYG{n}{reset\PYGZus{}index}\PYG{p}{(}\PYG{l+s+s1}{\PYGZsq{}}\PYG{l+s+s1}{PERMNO}\PYG{l+s+s1}{\PYGZsq{}}\PYG{p}{)}\PYG{o}{.}\PYG{n}{resample}\PYG{p}{(}\PYG{l+s+s1}{\PYGZsq{}}\PYG{l+s+s1}{M}\PYG{l+s+s1}{\PYGZsq{}}\PYG{p}{)}\PYG{p}{[}\PYG{l+s+s1}{\PYGZsq{}}\PYG{l+s+s1}{PRC}\PYG{l+s+s1}{\PYGZsq{}}\PYG{p}{]}\PYG{o}{.}\PYG{n}{count}\PYG{p}{(}\PYG{p}{)}\PYG{o}{.}\PYG{n}{plot}\PYG{p}{(}\PYG{p}{)}
\PYG{n}{plt}\PYG{o}{.}\PYG{n}{xlabel}\PYG{p}{(}\PYG{l+s+s1}{\PYGZsq{}}\PYG{l+s+s1}{Date}\PYG{l+s+s1}{\PYGZsq{}}\PYG{p}{)}
\PYG{n}{plt}\PYG{o}{.}\PYG{n}{ylabel}\PYG{p}{(}\PYG{l+s+s1}{\PYGZsq{}}\PYG{l+s+s1}{Number of Publicly Listed Stocks}\PYG{l+s+s1}{\PYGZsq{}}\PYG{p}{)}
\PYG{n}{plt}\PYG{o}{.}\PYG{n}{title}\PYG{p}{(}\PYG{l+s+s1}{\PYGZsq{}}\PYG{l+s+s1}{Time Series of the Number of Publicly Listed Stocks}\PYG{l+s+s1}{\PYGZsq{}}\PYG{p}{)}
\PYG{n}{plt}\PYG{o}{.}\PYG{n}{show}\PYG{p}{(}\PYG{p}{)}
\end{sphinxVerbatim}

\end{sphinxuseclass}\end{sphinxVerbatimInput}
\begin{sphinxVerbatimOutput}

\begin{sphinxuseclass}{cell_output}
\noindent\sphinxincludegraphics{{06d274e77f5598275a25b58eda28b21f1f3fa77afa6386f506d66dfb80ec3216}.png}

\end{sphinxuseclass}\end{sphinxVerbatimOutput}

\end{sphinxuseclass}

\section{Implement a Momentum Investing Strategy}
\label{\detokenize{herron_06_lecture:implement-a-momentum-investing-strategy}}
\sphinxAtStartPar
We will implement a momentum investing strategy with 1\sphinxhyphen{}month holding periods of equal\sphinxhyphen{}weighted portfolios formed on total returns from month \sphinxhyphen{}12 through month \sphinxhyphen{}2.
For example, at the start of January in year \$t\$, we assign stocks to 10 portfolios based on their total returns from January in year \$t\sphinxhyphen{}1\$ through November in year \$t\sphinxhyphen{}1\$.
Portfolio 1 has the lowest trailing returns, and portfolio 10 has the highest trailing returns.
We do not use the returns in month \sphinxhyphen{}1 for portfolio assignment because they would contaminate our portfolio returns with \sphinxhref{https://www.investopedia.com/ask/answers/013015/whats-difference-between-bidask-spread-and-bidask-bounce.asp}{bid\sphinxhyphen{}ask bounce}.
For example, in our example above, we do not use returns from December in year \$t\sphinxhyphen{}1\$ to assign stocks to portfolios at the start of January in year \$t\$.


\subsection{Calculate 1\sphinxhyphen{}Month and 11\sphinxhyphen{}Month Returns}
\label{\detokenize{herron_06_lecture:calculate-1-month-and-11-month-returns}}
\sphinxAtStartPar
We will assign 1\sphinxhyphen{}month returns in month 0 to data frame \sphinxcode{\sphinxupquote{ret\_1m}}.
This calculation is easier in the wide format, but we will later use the long format.

\begin{sphinxuseclass}{cell}\begin{sphinxVerbatimInput}

\begin{sphinxuseclass}{cell_input}
\begin{sphinxVerbatim}[commandchars=\\\{\}]
\PYG{n}{ret\PYGZus{}1m} \PYG{o}{=} \PYG{n}{crsp}\PYG{p}{[}\PYG{l+s+s1}{\PYGZsq{}}\PYG{l+s+s1}{RET}\PYG{l+s+s1}{\PYGZsq{}}\PYG{p}{]}\PYG{o}{.}\PYG{n}{unstack}\PYG{p}{(}\PYG{l+s+s1}{\PYGZsq{}}\PYG{l+s+s1}{PERMNO}\PYG{l+s+s1}{\PYGZsq{}}\PYG{p}{)}

\PYG{n}{ret\PYGZus{}1m}\PYG{o}{.}\PYG{n}{tail}\PYG{p}{(}\PYG{p}{)}
\end{sphinxVerbatim}

\end{sphinxuseclass}\end{sphinxVerbatimInput}
\begin{sphinxVerbatimOutput}

\begin{sphinxuseclass}{cell_output}
\begin{sphinxVerbatim}[commandchars=\\\{\}]
PERMNO   10000  10001  10002  10003  10005  10006  10007  10008  10009  10010  \PYGZbs{}
date                                                                            
2022\PYGZhy{}08    NaN    NaN    NaN    NaN    NaN    NaN    NaN    NaN    NaN    NaN   
2022\PYGZhy{}09    NaN    NaN    NaN    NaN    NaN    NaN    NaN    NaN    NaN    NaN   
2022\PYGZhy{}10    NaN    NaN    NaN    NaN    NaN    NaN    NaN    NaN    NaN    NaN   
2022\PYGZhy{}11    NaN    NaN    NaN    NaN    NaN    NaN    NaN    NaN    NaN    NaN   
2022\PYGZhy{}12    NaN    NaN    NaN    NaN    NaN    NaN    NaN    NaN    NaN    NaN   

PERMNO   ...   93423   93426  93428   93429  93430  93432  93433   93434  \PYGZbs{}
date     ...                                                               
2022\PYGZhy{}08  ... \PYGZhy{}0.0229  0.1004    NaN \PYGZhy{}0.0400    NaN    NaN    NaN  0.3474   
2022\PYGZhy{}09  ... \PYGZhy{}0.2009 \PYGZhy{}0.1376    NaN \PYGZhy{}0.0051    NaN    NaN    NaN \PYGZhy{}0.4297   
2022\PYGZhy{}10  ...  0.2599  0.1406    NaN  0.0607    NaN    NaN    NaN  0.3562   
2022\PYGZhy{}11  ...  0.0803  0.2021    NaN  0.0228    NaN    NaN    NaN  0.1212   
2022\PYGZhy{}12  ... \PYGZhy{}0.0349 \PYGZhy{}0.0473    NaN \PYGZhy{}0.0108    NaN    NaN    NaN  0.3423   

PERMNO   93435   93436  
date                    
2022\PYGZhy{}08    NaN \PYGZhy{}0.0725  
2022\PYGZhy{}09    NaN \PYGZhy{}0.0376  
2022\PYGZhy{}10    NaN \PYGZhy{}0.1422  
2022\PYGZhy{}11    NaN \PYGZhy{}0.1443  
2022\PYGZhy{}12    NaN \PYGZhy{}0.3673  

[5 rows x 26480 columns]
\end{sphinxVerbatim}

\end{sphinxuseclass}\end{sphinxVerbatimOutput}

\end{sphinxuseclass}
\sphinxAtStartPar
We will assign 11\sphinxhyphen{}month returns from month \sphinxhyphen{}12 to month \sphinxhyphen{}2 to data frame \sphinxcode{\sphinxupquote{ret\_11m}}.
The \sphinxcode{\sphinxupquote{.rolling()}} method does not have a \sphinxcode{\sphinxupquote{.prod()}} method, so we will sum log returns instead of compounding simple returns, which speeds up our calculation.

\begin{sphinxuseclass}{cell}\begin{sphinxVerbatimInput}

\begin{sphinxuseclass}{cell_input}
\begin{sphinxVerbatim}[commandchars=\\\{\}]
\PYG{n}{ret\PYGZus{}11m} \PYG{o}{=} \PYG{n}{ret\PYGZus{}1m}\PYG{o}{.}\PYG{n}{pipe}\PYG{p}{(}\PYG{n}{np}\PYG{o}{.}\PYG{n}{log1p}\PYG{p}{)}\PYG{o}{.}\PYG{n}{rolling}\PYG{p}{(}\PYG{l+m+mi}{11}\PYG{p}{)}\PYG{o}{.}\PYG{n}{sum}\PYG{p}{(}\PYG{p}{)}\PYG{o}{.}\PYG{n}{pipe}\PYG{p}{(}\PYG{n}{np}\PYG{o}{.}\PYG{n}{expm1}\PYG{p}{)}

\PYG{n}{ret\PYGZus{}11m}\PYG{o}{.}\PYG{n}{tail}\PYG{p}{(}\PYG{p}{)}
\end{sphinxVerbatim}

\end{sphinxuseclass}\end{sphinxVerbatimInput}
\begin{sphinxVerbatimOutput}

\begin{sphinxuseclass}{cell_output}
\begin{sphinxVerbatim}[commandchars=\\\{\}]
PERMNO   10000  10001  10002  10003  10005  10006  10007  10008  10009  10010  \PYGZbs{}
date                                                                            
2022\PYGZhy{}08    NaN    NaN    NaN    NaN    NaN    NaN    NaN    NaN    NaN    NaN   
2022\PYGZhy{}09    NaN    NaN    NaN    NaN    NaN    NaN    NaN    NaN    NaN    NaN   
2022\PYGZhy{}10    NaN    NaN    NaN    NaN    NaN    NaN    NaN    NaN    NaN    NaN   
2022\PYGZhy{}11    NaN    NaN    NaN    NaN    NaN    NaN    NaN    NaN    NaN    NaN   
2022\PYGZhy{}12    NaN    NaN    NaN    NaN    NaN    NaN    NaN    NaN    NaN    NaN   

PERMNO   ...   93423   93426  93428   93429  93430  93432  93433   93434  \PYGZbs{}
date     ...                                                               
2022\PYGZhy{}08  ... \PYGZhy{}0.4788 \PYGZhy{}0.0132    NaN \PYGZhy{}0.0321    NaN    NaN    NaN \PYGZhy{}0.5058   
2022\PYGZhy{}09  ... \PYGZhy{}0.5697 \PYGZhy{}0.1320    NaN \PYGZhy{}0.0960    NaN    NaN    NaN \PYGZhy{}0.8298   
2022\PYGZhy{}10  ... \PYGZhy{}0.3902 \PYGZhy{}0.0169    NaN \PYGZhy{}0.0224    NaN    NaN    NaN \PYGZhy{}0.6610   
2022\PYGZhy{}11  ... \PYGZhy{}0.4342  0.0929    NaN \PYGZhy{}0.0113    NaN    NaN    NaN \PYGZhy{}0.5934   
2022\PYGZhy{}12  ... \PYGZhy{}0.4112  0.2067    NaN  0.0760    NaN    NaN    NaN \PYGZhy{}0.3578   

PERMNO   93435   93436  
date                    
2022\PYGZhy{}08    NaN  0.0662  
2022\PYGZhy{}09    NaN \PYGZhy{}0.2857  
2022\PYGZhy{}10    NaN \PYGZhy{}0.4037  
2022\PYGZhy{}11    NaN \PYGZhy{}0.4473  
2022\PYGZhy{}12    NaN \PYGZhy{}0.6055  

[5 rows x 26480 columns]
\end{sphinxVerbatim}

\end{sphinxuseclass}\end{sphinxVerbatimOutput}

\end{sphinxuseclass}
\sphinxAtStartPar
We should always check our work!
If we were to use this code in a production environment (i.e., run it frequently to make decisions), we would add \sphinxhref{https://en.wikipedia.org/wiki/Unit\_testing}{unit tests} to check our work.
We can use \sphinxcode{\sphinxupquote{PERMNO}} 10010 to (very lightly) check our work.

\begin{sphinxuseclass}{cell}\begin{sphinxVerbatimInput}

\begin{sphinxuseclass}{cell_input}
\begin{sphinxVerbatim}[commandchars=\\\{\}]
\PYG{k}{assert} \PYG{n}{np}\PYG{o}{.}\PYG{n}{allclose}\PYG{p}{(}
    \PYG{n}{a}\PYG{o}{=}\PYG{n}{ret\PYGZus{}1m}\PYG{p}{[}\PYG{l+m+mi}{10010}\PYG{p}{]}\PYG{o}{.}\PYG{n}{rolling}\PYG{p}{(}\PYG{l+m+mi}{11}\PYG{p}{)}\PYG{o}{.}\PYG{n}{apply}\PYG{p}{(}\PYG{k}{lambda} \PYG{n}{x}\PYG{p}{:} \PYG{p}{(}\PYG{l+m+mi}{1} \PYG{o}{+} \PYG{n}{x}\PYG{p}{)}\PYG{o}{.}\PYG{n}{prod}\PYG{p}{(}\PYG{p}{)} \PYG{o}{\PYGZhy{}} \PYG{l+m+mi}{1}\PYG{p}{)}\PYG{p}{,}
    \PYG{n}{b}\PYG{o}{=}\PYG{n}{ret\PYGZus{}11m}\PYG{p}{[}\PYG{l+m+mi}{10010}\PYG{p}{]}\PYG{p}{,}
    \PYG{n}{equal\PYGZus{}nan}\PYG{o}{=}\PYG{k+kc}{True}
\PYG{p}{)}
\end{sphinxVerbatim}

\end{sphinxuseclass}\end{sphinxVerbatimInput}

\end{sphinxuseclass}

\subsection{Assign Stocks to Portfolios Based on 11\sphinxhyphen{}Month Returns}
\label{\detokenize{herron_06_lecture:assign-stocks-to-portfolios-based-on-11-month-returns}}
\sphinxAtStartPar
We will use \sphinxcode{\sphinxupquote{pd.qcut()}} to assign stocks to portfolios based on their 11\sphinxhyphen{}month trailing returns.
Here is a simple example that uses \sphinxcode{\sphinxupquote{pd.qcut()}} to assign the values 0 to 24 to 10 portfolios.

\begin{sphinxuseclass}{cell}\begin{sphinxVerbatimInput}

\begin{sphinxuseclass}{cell_input}
\begin{sphinxVerbatim}[commandchars=\\\{\}]
\PYG{n}{pd}\PYG{o}{.}\PYG{n}{qcut}\PYG{p}{(}\PYG{n}{np}\PYG{o}{.}\PYG{n}{arange}\PYG{p}{(}\PYG{l+m+mi}{25}\PYG{p}{)}\PYG{p}{,} \PYG{n}{q}\PYG{o}{=}\PYG{l+m+mi}{10}\PYG{p}{,} \PYG{n}{labels}\PYG{o}{=}\PYG{k+kc}{False}\PYG{p}{)} \PYG{o}{+} \PYG{l+m+mi}{1}
\end{sphinxVerbatim}

\end{sphinxuseclass}\end{sphinxVerbatimInput}
\begin{sphinxVerbatimOutput}

\begin{sphinxuseclass}{cell_output}
\begin{sphinxVerbatim}[commandchars=\\\{\}]
array([ 1,  1,  1,  2,  2,  3,  3,  3,  4,  4,  5,  5,  5,  6,  6,  7,  7,
        8,  8,  8,  9,  9, 10, 10, 10])
\end{sphinxVerbatim}

\end{sphinxuseclass}\end{sphinxVerbatimOutput}

\end{sphinxuseclass}
\sphinxAtStartPar
We will save these portfolio assignments to data frame \sphinxcode{\sphinxupquote{port\_11m}}.
The \sphinxcode{\sphinxupquote{pd.qcut()}} function errs if we try to cut an array of all missing values, so we will use \sphinxcode{\sphinxupquote{.dropna(how='all')}} to drop rows with all missing values.

\begin{sphinxuseclass}{cell}\begin{sphinxVerbatimInput}

\begin{sphinxuseclass}{cell_input}
\begin{sphinxVerbatim}[commandchars=\\\{\}]
\PYG{n}{port\PYGZus{}11m} \PYG{o}{=} \PYG{n}{ret\PYGZus{}11m}\PYG{o}{.}\PYG{n}{dropna}\PYG{p}{(}\PYG{n}{how}\PYG{o}{=}\PYG{l+s+s1}{\PYGZsq{}}\PYG{l+s+s1}{all}\PYG{l+s+s1}{\PYGZsq{}}\PYG{p}{)}\PYG{o}{.}\PYG{n}{apply}\PYG{p}{(}\PYG{n}{pd}\PYG{o}{.}\PYG{n}{qcut}\PYG{p}{,} \PYG{n}{q}\PYG{o}{=}\PYG{l+m+mi}{10}\PYG{p}{,} \PYG{n}{labels}\PYG{o}{=}\PYG{k+kc}{False}\PYG{p}{,} \PYG{n}{axis}\PYG{o}{=}\PYG{l+m+mi}{1}\PYG{p}{)} \PYG{o}{+} \PYG{l+m+mi}{1}

\PYG{n}{port\PYGZus{}11m}\PYG{o}{.}\PYG{n}{tail}\PYG{p}{(}\PYG{p}{)}
\end{sphinxVerbatim}

\end{sphinxuseclass}\end{sphinxVerbatimInput}
\begin{sphinxVerbatimOutput}

\begin{sphinxuseclass}{cell_output}
\begin{sphinxVerbatim}[commandchars=\\\{\}]
PERMNO   10000  10001  10002  10003  10005  10006  10007  10008  10009  10010  \PYGZbs{}
date                                                                            
2022\PYGZhy{}08    NaN    NaN    NaN    NaN    NaN    NaN    NaN    NaN    NaN    NaN   
2022\PYGZhy{}09    NaN    NaN    NaN    NaN    NaN    NaN    NaN    NaN    NaN    NaN   
2022\PYGZhy{}10    NaN    NaN    NaN    NaN    NaN    NaN    NaN    NaN    NaN    NaN   
2022\PYGZhy{}11    NaN    NaN    NaN    NaN    NaN    NaN    NaN    NaN    NaN    NaN   
2022\PYGZhy{}12    NaN    NaN    NaN    NaN    NaN    NaN    NaN    NaN    NaN    NaN   

PERMNO   ...  93423  93426  93428  93429  93430  93432  93433  93434  93435  \PYGZbs{}
date     ...                                                                  
2022\PYGZhy{}08  ... 3.0000 7.0000    NaN 7.0000    NaN    NaN    NaN 3.0000    NaN   
2022\PYGZhy{}09  ... 3.0000 7.0000    NaN 8.0000    NaN    NaN    NaN 1.0000    NaN   
2022\PYGZhy{}10  ... 4.0000 7.0000    NaN 7.0000    NaN    NaN    NaN 2.0000    NaN   
2022\PYGZhy{}11  ... 4.0000 9.0000    NaN 7.0000    NaN    NaN    NaN 3.0000    NaN   
2022\PYGZhy{}12  ... 4.0000 9.0000    NaN 9.0000    NaN    NaN    NaN 4.0000    NaN   

PERMNO   93436  
date            
2022\PYGZhy{}08 9.0000  
2022\PYGZhy{}09 6.0000  
2022\PYGZhy{}10 4.0000  
2022\PYGZhy{}11 4.0000  
2022\PYGZhy{}12 2.0000  

[5 rows x 26480 columns]
\end{sphinxVerbatim}

\end{sphinxuseclass}\end{sphinxVerbatimOutput}

\end{sphinxuseclass}
\sphinxAtStartPar
Again. we should check our output, here with the last row of data.

\begin{sphinxuseclass}{cell}\begin{sphinxVerbatimInput}

\begin{sphinxuseclass}{cell_input}
\begin{sphinxVerbatim}[commandchars=\\\{\}]
\PYG{k}{assert} \PYG{n}{np}\PYG{o}{.}\PYG{n}{allclose}\PYG{p}{(}
    \PYG{n}{a}\PYG{o}{=}\PYG{n}{port\PYGZus{}11m}\PYG{o}{.}\PYG{n}{iloc}\PYG{p}{[}\PYG{o}{\PYGZhy{}}\PYG{l+m+mi}{1}\PYG{p}{]}\PYG{p}{,}
    \PYG{n}{b}\PYG{o}{=}\PYG{n}{pd}\PYG{o}{.}\PYG{n}{qcut}\PYG{p}{(}\PYG{n}{x}\PYG{o}{=}\PYG{n}{ret\PYGZus{}11m}\PYG{o}{.}\PYG{n}{iloc}\PYG{p}{[}\PYG{o}{\PYGZhy{}}\PYG{l+m+mi}{1}\PYG{p}{]}\PYG{p}{,} \PYG{n}{q}\PYG{o}{=}\PYG{l+m+mi}{10}\PYG{p}{,} \PYG{n}{labels}\PYG{o}{=}\PYG{k+kc}{False}\PYG{p}{)} \PYG{o}{+} \PYG{l+m+mi}{1}\PYG{p}{,}
    \PYG{n}{equal\PYGZus{}nan}\PYG{o}{=}\PYG{k+kc}{True}
\PYG{p}{)}
\end{sphinxVerbatim}

\end{sphinxuseclass}\end{sphinxVerbatimInput}

\end{sphinxuseclass}

\subsection{Combine Returns and Portfolio Assignments}
\label{\detokenize{herron_06_lecture:combine-returns-and-portfolio-assignments}}
\sphinxAtStartPar
Next, we will use \sphinxcode{\sphinxupquote{pd.concat(axis=1)}} to match returns and portfolio assignments.
We must \sphinxcode{\sphinxupquote{.shift(2)}} the 11\sphinxhyphen{}month returns and portfolios assignments to avoid a look\sphinxhyphen{}ahead bias and drop month \sphinxhyphen{}1 returns.
\begin{enumerate}
\sphinxsetlistlabels{\arabic}{enumi}{enumii}{}{.}%
\item {} 
\sphinxAtStartPar
The first shift makes sure we do not use contemporaneous returns to assign stocks to portfolios (i.e., otherwise the portfolio ranking and portfolio returns would overlap).

\item {} 
\sphinxAtStartPar
The second shift avoids mechanical correlations between returns one month and the next (i.e., bid\sphinxhyphen{}ask bounce and market microstructure noise).

\end{enumerate}

\begin{sphinxuseclass}{cell}\begin{sphinxVerbatimInput}

\begin{sphinxuseclass}{cell_input}
\begin{sphinxVerbatim}[commandchars=\\\{\}]
\PYG{n}{mom\PYGZus{}0} \PYG{o}{=} \PYG{p}{(}
    \PYG{n}{pd}\PYG{o}{.}\PYG{n}{concat}\PYG{p}{(}
        \PYG{n}{objs}\PYG{o}{=}\PYG{p}{[}\PYG{n}{ret\PYGZus{}1m}\PYG{p}{,} \PYG{n}{ret\PYGZus{}11m}\PYG{o}{.}\PYG{n}{shift}\PYG{p}{(}\PYG{l+m+mi}{2}\PYG{p}{)}\PYG{p}{,} \PYG{n}{port\PYGZus{}11m}\PYG{o}{.}\PYG{n}{shift}\PYG{p}{(}\PYG{l+m+mi}{2}\PYG{p}{)}\PYG{p}{]}\PYG{p}{,}
        \PYG{n}{axis}\PYG{o}{=}\PYG{l+m+mi}{1}\PYG{p}{,} 
        \PYG{n}{keys}\PYG{o}{=}\PYG{p}{[}\PYG{l+s+s1}{\PYGZsq{}}\PYG{l+s+s1}{Return}\PYG{l+s+s1}{\PYGZsq{}}\PYG{p}{,} \PYG{l+s+s1}{\PYGZsq{}}\PYG{l+s+s1}{Return\PYGZus{}Trailing}\PYG{l+s+s1}{\PYGZsq{}}\PYG{p}{,} \PYG{l+s+s1}{\PYGZsq{}}\PYG{l+s+s1}{Portfolio}\PYG{l+s+s1}{\PYGZsq{}}\PYG{p}{]}\PYG{p}{,}
        \PYG{n}{names}\PYG{o}{=}\PYG{p}{[}\PYG{l+s+s1}{\PYGZsq{}}\PYG{l+s+s1}{Variable}\PYG{l+s+s1}{\PYGZsq{}}\PYG{p}{,} \PYG{l+s+s1}{\PYGZsq{}}\PYG{l+s+s1}{PERMNO}\PYG{l+s+s1}{\PYGZsq{}}\PYG{p}{]}
    \PYG{p}{)}
    \PYG{o}{.}\PYG{n}{stack}\PYG{p}{(}\PYG{l+s+s1}{\PYGZsq{}}\PYG{l+s+s1}{PERMNO}\PYG{l+s+s1}{\PYGZsq{}}\PYG{p}{)}
    \PYG{o}{.}\PYG{n}{dropna}\PYG{p}{(}\PYG{p}{)}
    \PYG{o}{.}\PYG{n}{assign}\PYG{p}{(}\PYG{n}{Portfolio}\PYG{o}{=}\PYG{k}{lambda} \PYG{n}{x}\PYG{p}{:} \PYG{n}{x}\PYG{p}{[}\PYG{l+s+s1}{\PYGZsq{}}\PYG{l+s+s1}{Portfolio}\PYG{l+s+s1}{\PYGZsq{}}\PYG{p}{]}\PYG{o}{.}\PYG{n}{astype}\PYG{p}{(}\PYG{n+nb}{int}\PYG{p}{)}\PYG{p}{)}
\PYG{p}{)}
\end{sphinxVerbatim}

\end{sphinxuseclass}\end{sphinxVerbatimInput}

\end{sphinxuseclass}
\begin{sphinxuseclass}{cell}\begin{sphinxVerbatimInput}

\begin{sphinxuseclass}{cell_input}
\begin{sphinxVerbatim}[commandchars=\\\{\}]
\PYG{n}{mom\PYGZus{}0}\PYG{o}{.}\PYG{n}{head}\PYG{p}{(}\PYG{p}{)}
\end{sphinxVerbatim}

\end{sphinxuseclass}\end{sphinxVerbatimInput}
\begin{sphinxVerbatimOutput}

\begin{sphinxuseclass}{cell_output}
\begin{sphinxVerbatim}[commandchars=\\\{\}]
Variable        Return  Return\PYGZus{}Trailing  Portfolio
date    PERMNO                                    
1927\PYGZhy{}01 10022  \PYGZhy{}0.0759           0.1182          8
        10030   0.0095           0.0093          6
        10057  \PYGZhy{}0.0510          \PYGZhy{}0.5918          1
        10073   0.0946          \PYGZhy{}0.4286          2
        10081  \PYGZhy{}0.0750          \PYGZhy{}0.4194          2
\end{sphinxVerbatim}

\end{sphinxuseclass}\end{sphinxVerbatimOutput}

\end{sphinxuseclass}

\subsection{Evaluate Performance}
\label{\detokenize{herron_06_lecture:evaluate-performance}}
\sphinxAtStartPar
Is there a relation between returns from month \sphinxhyphen{}12 to month \sphinxhyphen{}2 and return in month 0?
We can use the \sphinxcode{\sphinxupquote{.corr()}} to estimate this.

\begin{sphinxuseclass}{cell}\begin{sphinxVerbatimInput}

\begin{sphinxuseclass}{cell_input}
\begin{sphinxVerbatim}[commandchars=\\\{\}]
\PYG{n}{mom\PYGZus{}0}\PYG{o}{.}\PYG{n}{filter}\PYG{p}{(}\PYG{n}{regex}\PYG{o}{=}\PYG{l+s+s1}{\PYGZsq{}}\PYG{l+s+s1}{Return}\PYG{l+s+s1}{\PYGZsq{}}\PYG{p}{)}\PYG{o}{.}\PYG{n}{corr}\PYG{p}{(}\PYG{p}{)}
\end{sphinxVerbatim}

\end{sphinxuseclass}\end{sphinxVerbatimInput}
\begin{sphinxVerbatimOutput}

\begin{sphinxuseclass}{cell_output}
\begin{sphinxVerbatim}[commandchars=\\\{\}]
Variable         Return  Return\PYGZus{}Trailing
Variable                                
Return           1.0000           0.0036
Return\PYGZus{}Trailing  0.0036           1.0000
\end{sphinxVerbatim}

\end{sphinxuseclass}\end{sphinxVerbatimOutput}

\end{sphinxuseclass}
\sphinxAtStartPar
This correlation is very weak!
But we expect we correlations based on pervious course topics.
We can try again with log returns, which may reduce the noise from outliers.
Recall \$R\_\{Log\} = log(1 + R\_\{Simple\})\$, which we can quickly implement with \sphinxcode{\sphinxupquote{.pipe(np.log1p)}}.

\begin{sphinxuseclass}{cell}\begin{sphinxVerbatimInput}

\begin{sphinxuseclass}{cell_input}
\begin{sphinxVerbatim}[commandchars=\\\{\}]
\PYG{n}{mom\PYGZus{}0}\PYG{o}{.}\PYG{n}{filter}\PYG{p}{(}\PYG{n}{regex}\PYG{o}{=}\PYG{l+s+s1}{\PYGZsq{}}\PYG{l+s+s1}{Return}\PYG{l+s+s1}{\PYGZsq{}}\PYG{p}{)}\PYG{o}{.}\PYG{n}{pipe}\PYG{p}{(}\PYG{n}{np}\PYG{o}{.}\PYG{n}{log1p}\PYG{p}{)}\PYG{o}{.}\PYG{n}{corr}\PYG{p}{(}\PYG{p}{)}
\end{sphinxVerbatim}

\end{sphinxuseclass}\end{sphinxVerbatimInput}
\begin{sphinxVerbatimOutput}

\begin{sphinxuseclass}{cell_output}
\begin{sphinxVerbatim}[commandchars=\\\{\}]
Variable         Return  Return\PYGZus{}Trailing
Variable                                
Return           1.0000           0.0486
Return\PYGZus{}Trailing  0.0486           1.0000
\end{sphinxVerbatim}

\end{sphinxuseclass}\end{sphinxVerbatimOutput}

\end{sphinxuseclass}
\sphinxAtStartPar
The correlation of log returns is larger, but still low because single\sphinxhyphen{}stock returns are noisy!
We can reduce this noise with portfolios formed on trailing returns.
We will equally weight the returns in each portfolio using \sphinxcode{\sphinxupquote{.mean()}}.

\begin{sphinxuseclass}{cell}\begin{sphinxVerbatimInput}

\begin{sphinxuseclass}{cell_input}
\begin{sphinxVerbatim}[commandchars=\\\{\}]
\PYG{n}{mom\PYGZus{}ew} \PYG{o}{=} \PYG{p}{(}
    \PYG{n}{mom\PYGZus{}0}
    \PYG{o}{.}\PYG{n}{groupby}\PYG{p}{(}\PYG{n}{by}\PYG{o}{=}\PYG{p}{[}\PYG{l+s+s1}{\PYGZsq{}}\PYG{l+s+s1}{date}\PYG{l+s+s1}{\PYGZsq{}}\PYG{p}{,} \PYG{l+s+s1}{\PYGZsq{}}\PYG{l+s+s1}{Portfolio}\PYG{l+s+s1}{\PYGZsq{}}\PYG{p}{]}\PYG{p}{)}
    \PYG{p}{[}\PYG{l+s+s1}{\PYGZsq{}}\PYG{l+s+s1}{Return}\PYG{l+s+s1}{\PYGZsq{}}\PYG{p}{]}
    \PYG{o}{.}\PYG{n}{mean}\PYG{p}{(}\PYG{p}{)}
    \PYG{o}{.}\PYG{n}{unstack}\PYG{p}{(}\PYG{l+s+s1}{\PYGZsq{}}\PYG{l+s+s1}{Portfolio}\PYG{l+s+s1}{\PYGZsq{}}\PYG{p}{)}
\PYG{p}{)}
\end{sphinxVerbatim}

\end{sphinxuseclass}\end{sphinxVerbatimInput}

\end{sphinxuseclass}
\begin{sphinxuseclass}{cell}\begin{sphinxVerbatimInput}

\begin{sphinxuseclass}{cell_input}
\begin{sphinxVerbatim}[commandchars=\\\{\}]
\PYG{n}{mom\PYGZus{}ew}\PYG{o}{.}\PYG{n}{head}\PYG{p}{(}\PYG{p}{)}
\end{sphinxVerbatim}

\end{sphinxuseclass}\end{sphinxVerbatimInput}
\begin{sphinxVerbatimOutput}

\begin{sphinxuseclass}{cell_output}
\begin{sphinxVerbatim}[commandchars=\\\{\}]
Portfolio      1       2       3       4       5       6      7       8   \PYGZbs{}
date                                                                       
1927\PYGZhy{}01   \PYGZhy{}0.0006  0.0149  0.0407  0.0378  0.0009  0.0150 0.0124  0.0076   
1927\PYGZhy{}02    0.0641  0.0484  0.0866  0.0745  0.0480  0.0547 0.0561  0.0403   
1927\PYGZhy{}03   \PYGZhy{}0.0530 \PYGZhy{}0.0381 \PYGZhy{}0.0243 \PYGZhy{}0.0408 \PYGZhy{}0.0192 \PYGZhy{}0.0201 0.0108 \PYGZhy{}0.0084   
1927\PYGZhy{}04    0.0207 \PYGZhy{}0.0085 \PYGZhy{}0.0054 \PYGZhy{}0.0061  0.0002 \PYGZhy{}0.0032 0.0137  0.0030   
1927\PYGZhy{}05    0.0069  0.0353  0.0604  0.0635  0.0700  0.0950 0.0796  0.0838   

Portfolio      9      10  
date                      
1927\PYGZhy{}01   \PYGZhy{}0.0044 0.0126  
1927\PYGZhy{}02    0.0579 0.0541  
1927\PYGZhy{}03   \PYGZhy{}0.0058 0.0012  
1927\PYGZhy{}04    0.0307 0.0421  
1927\PYGZhy{}05    0.0803 0.0886  
\end{sphinxVerbatim}

\end{sphinxuseclass}\end{sphinxVerbatimOutput}

\end{sphinxuseclass}
\sphinxAtStartPar
Next, we will plot the mean return for each portfolio.
We will convert these equal\sphinxhyphen{}weighted portfolio returns to percent, but leave them as monthly values.

\begin{sphinxuseclass}{cell}\begin{sphinxVerbatimInput}

\begin{sphinxuseclass}{cell_input}
\begin{sphinxVerbatim}[commandchars=\\\{\}]
\PYG{n}{mom\PYGZus{}ew}\PYG{o}{.}\PYG{n}{mean}\PYG{p}{(}\PYG{p}{)}\PYG{o}{.}\PYG{n}{mul}\PYG{p}{(}\PYG{l+m+mi}{100}\PYG{p}{)}\PYG{o}{.}\PYG{n}{plot}\PYG{p}{(}\PYG{n}{kind}\PYG{o}{=}\PYG{l+s+s1}{\PYGZsq{}}\PYG{l+s+s1}{bar}\PYG{l+s+s1}{\PYGZsq{}}\PYG{p}{)}
\PYG{n}{plt}\PYG{o}{.}\PYG{n}{ylabel}\PYG{p}{(}\PYG{l+s+s1}{\PYGZsq{}}\PYG{l+s+s1}{Mean Monthly Return (}\PYG{l+s+s1}{\PYGZpc{}}\PYG{l+s+s1}{)}\PYG{l+s+s1}{\PYGZsq{}}\PYG{p}{)}
\PYG{n}{plt}\PYG{o}{.}\PYG{n}{xlabel}\PYG{p}{(}\PYG{l+s+s1}{\PYGZsq{}}\PYG{l+s+s1}{Momentum Portfolio}\PYG{l+s+s1}{\PYGZsq{}}\PYG{p}{)}
\PYG{n}{plt}\PYG{o}{.}\PYG{n}{title}\PYG{p}{(}
    \PYG{l+s+s1}{\PYGZsq{}}\PYG{l+s+s1}{Performance of Momentum Investing}\PYG{l+s+s1}{\PYGZsq{}} \PYG{o}{+}
    \PYG{l+s+s1}{\PYGZsq{}}\PYG{l+s+se}{\PYGZbs{}n}\PYG{l+s+s1}{Mean Monthly Returns on Equal\PYGZhy{}Weighted Portfolios}\PYG{l+s+s1}{\PYGZsq{}} \PYG{o}{+}
    \PYG{l+s+s1}{\PYGZsq{}}\PYG{l+s+se}{\PYGZbs{}n}\PYG{l+s+s1}{Formed on Months \PYGZhy{}12 to Month \PYGZhy{}2 Returns}\PYG{l+s+s1}{\PYGZsq{}}
\PYG{p}{)}
\PYG{n}{plt}\PYG{o}{.}\PYG{n}{show}\PYG{p}{(}\PYG{p}{)}
\end{sphinxVerbatim}

\end{sphinxuseclass}\end{sphinxVerbatimInput}
\begin{sphinxVerbatimOutput}

\begin{sphinxuseclass}{cell_output}
\noindent\sphinxincludegraphics{{45567aafbd0029384da805cd24632f5baa386cc58327d94e276e33fda75dad94}.png}

\end{sphinxuseclass}\end{sphinxVerbatimOutput}

\end{sphinxuseclass}
\sphinxAtStartPar
Above, we see about a 75 basis point spread between the returns on portfolios 1 and 10, suggesting that momemtum investing generates excess returns.
We should also consider cumulative returns over holding periods longer than one month.
Next, we will plot the values of \$1 invested in each portfolio.

\begin{sphinxuseclass}{cell}\begin{sphinxVerbatimInput}

\begin{sphinxuseclass}{cell_input}
\begin{sphinxVerbatim}[commandchars=\\\{\}]
\PYG{n}{mom\PYGZus{}ew}\PYG{o}{.}\PYG{n}{add}\PYG{p}{(}\PYG{l+m+mi}{1}\PYG{p}{)}\PYG{o}{.}\PYG{n}{cumprod}\PYG{p}{(}\PYG{p}{)}\PYG{o}{.}\PYG{n}{plot}\PYG{p}{(}\PYG{p}{)}
\PYG{n}{plt}\PYG{o}{.}\PYG{n}{semilogy}\PYG{p}{(}\PYG{p}{)}
\PYG{n}{plt}\PYG{o}{.}\PYG{n}{ylabel}\PYG{p}{(}\PYG{l+s+s1}{\PYGZsq{}}\PYG{l+s+s1}{Value of }\PYG{l+s+s1}{\PYGZbs{}}\PYG{l+s+s1}{\PYGZdl{}1 Invested (}\PYG{l+s+s1}{\PYGZbs{}}\PYG{l+s+s1}{\PYGZdl{})}\PYG{l+s+s1}{\PYGZsq{}}\PYG{p}{)}
\PYG{n}{plt}\PYG{o}{.}\PYG{n}{xlabel}\PYG{p}{(}\PYG{l+s+s1}{\PYGZsq{}}\PYG{l+s+s1}{Date}\PYG{l+s+s1}{\PYGZsq{}}\PYG{p}{)}
\PYG{n}{plt}\PYG{o}{.}\PYG{n}{title}\PYG{p}{(}
    \PYG{l+s+s1}{\PYGZsq{}}\PYG{l+s+s1}{Performance of Momentum Investing}\PYG{l+s+s1}{\PYGZsq{}} \PYG{o}{+}
    \PYG{l+s+s1}{\PYGZsq{}}\PYG{l+s+se}{\PYGZbs{}n}\PYG{l+s+s1}{Value of \PYGZdl{}1 Invested in Equal\PYGZhy{}Weighted Portfolios}\PYG{l+s+s1}{\PYGZsq{}} \PYG{o}{+}
    \PYG{l+s+s1}{\PYGZsq{}}\PYG{l+s+se}{\PYGZbs{}n}\PYG{l+s+s1}{Formed on Months \PYGZhy{}12 to Month \PYGZhy{}2 Returns}\PYG{l+s+s1}{\PYGZsq{}}
\PYG{p}{)}
\PYG{n}{plt}\PYG{o}{.}\PYG{n}{show}\PYG{p}{(}\PYG{p}{)}
\end{sphinxVerbatim}

\end{sphinxuseclass}\end{sphinxVerbatimInput}
\begin{sphinxVerbatimOutput}

\begin{sphinxuseclass}{cell_output}
\noindent\sphinxincludegraphics{{2bfffe7ca96a6b5edd134bc8cf156762b68708f303dae84f9907c26ff8201781}.png}

\end{sphinxuseclass}\end{sphinxVerbatimOutput}

\end{sphinxuseclass}
\sphinxAtStartPar
Finally, we can estimate capital asset pricing model (CAPM) and Fama\sphinxhyphen{}French four\sphinxhyphen{}factor model (FF4) regressions.

\begin{sphinxuseclass}{cell}\begin{sphinxVerbatimInput}

\begin{sphinxuseclass}{cell_input}
\begin{sphinxVerbatim}[commandchars=\\\{\}]
\PYG{n}{ff\PYGZus{}0} \PYG{o}{=} \PYG{n}{pdr}\PYG{o}{.}\PYG{n}{DataReader}\PYG{p}{(}
    \PYG{n}{name}\PYG{o}{=}\PYG{l+s+s1}{\PYGZsq{}}\PYG{l+s+s1}{F\PYGZhy{}F\PYGZus{}Research\PYGZus{}Data\PYGZus{}Factors}\PYG{l+s+s1}{\PYGZsq{}}\PYG{p}{,}
    \PYG{n}{data\PYGZus{}source}\PYG{o}{=}\PYG{l+s+s1}{\PYGZsq{}}\PYG{l+s+s1}{famafrench}\PYG{l+s+s1}{\PYGZsq{}}\PYG{p}{,}
    \PYG{n}{session}\PYG{o}{=}\PYG{n}{session}\PYG{p}{,}
    \PYG{n}{start}\PYG{o}{=}\PYG{l+s+s1}{\PYGZsq{}}\PYG{l+s+s1}{1900}\PYG{l+s+s1}{\PYGZsq{}}
\PYG{p}{)}

\PYG{n}{ff\PYGZus{}0}\PYG{p}{[}\PYG{l+m+mi}{0}\PYG{p}{]}\PYG{o}{.}\PYG{n}{head}\PYG{p}{(}\PYG{p}{)}
\end{sphinxVerbatim}

\end{sphinxuseclass}\end{sphinxVerbatimInput}
\begin{sphinxVerbatimOutput}

\begin{sphinxuseclass}{cell_output}
\begin{sphinxVerbatim}[commandchars=\\\{\}]
         Mkt\PYGZhy{}RF     SMB     HML     RF
Date                                  
1926\PYGZhy{}07  2.9600 \PYGZhy{}2.5600 \PYGZhy{}2.4300 0.2200
1926\PYGZhy{}08  2.6400 \PYGZhy{}1.1700  3.8200 0.2500
1926\PYGZhy{}09  0.3600 \PYGZhy{}1.4000  0.1300 0.2300
1926\PYGZhy{}10 \PYGZhy{}3.2400 \PYGZhy{}0.0900  0.7000 0.3200
1926\PYGZhy{}11  2.5300 \PYGZhy{}0.1000 \PYGZhy{}0.5100 0.3100
\end{sphinxVerbatim}

\end{sphinxuseclass}\end{sphinxVerbatimOutput}

\end{sphinxuseclass}
\begin{sphinxuseclass}{cell}\begin{sphinxVerbatimInput}

\begin{sphinxuseclass}{cell_input}
\begin{sphinxVerbatim}[commandchars=\\\{\}]
\PYG{n}{ff\PYGZus{}mom} \PYG{o}{=} \PYG{n}{pdr}\PYG{o}{.}\PYG{n}{DataReader}\PYG{p}{(}\PYG{n}{name}\PYG{o}{=}\PYG{l+s+s1}{\PYGZsq{}}\PYG{l+s+s1}{F\PYGZhy{}F\PYGZus{}Momentum\PYGZus{}Factor}\PYG{l+s+s1}{\PYGZsq{}}\PYG{p}{,} \PYG{n}{data\PYGZus{}source}\PYG{o}{=}\PYG{l+s+s1}{\PYGZsq{}}\PYG{l+s+s1}{famafrench}\PYG{l+s+s1}{\PYGZsq{}}\PYG{p}{,} \PYG{n}{session}\PYG{o}{=}\PYG{n}{session}\PYG{p}{,} \PYG{n}{start}\PYG{o}{=}\PYG{l+s+s1}{\PYGZsq{}}\PYG{l+s+s1}{1900}\PYG{l+s+s1}{\PYGZsq{}}\PYG{p}{)}
\PYG{n}{ff\PYGZus{}mom}\PYG{p}{[}\PYG{l+m+mi}{0}\PYG{p}{]}\PYG{o}{.}\PYG{n}{columns} \PYG{o}{=} \PYG{p}{[}\PYG{n}{c}\PYG{o}{.}\PYG{n}{strip}\PYG{p}{(}\PYG{p}{)} \PYG{k}{for} \PYG{n}{c} \PYG{o+ow}{in} \PYG{n}{ff\PYGZus{}mom}\PYG{p}{[}\PYG{l+m+mi}{0}\PYG{p}{]}\PYG{o}{.}\PYG{n}{columns}\PYG{p}{]}

\PYG{n}{ff\PYGZus{}mom}\PYG{p}{[}\PYG{l+m+mi}{0}\PYG{p}{]}\PYG{o}{.}\PYG{n}{head}\PYG{p}{(}\PYG{p}{)}
\end{sphinxVerbatim}

\end{sphinxuseclass}\end{sphinxVerbatimInput}
\begin{sphinxVerbatimOutput}

\begin{sphinxuseclass}{cell_output}
\begin{sphinxVerbatim}[commandchars=\\\{\}]
            Mom
Date           
1927\PYGZhy{}01  0.3600
1927\PYGZhy{}02 \PYGZhy{}2.1400
1927\PYGZhy{}03  3.6100
1927\PYGZhy{}04  4.3000
1927\PYGZhy{}05  3.0000
\end{sphinxVerbatim}

\end{sphinxuseclass}\end{sphinxVerbatimOutput}

\end{sphinxuseclass}
\begin{sphinxuseclass}{cell}\begin{sphinxVerbatimInput}

\begin{sphinxuseclass}{cell_input}
\begin{sphinxVerbatim}[commandchars=\\\{\}]
\PYG{k+kn}{import} \PYG{n+nn}{statsmodels}\PYG{n+nn}{.}\PYG{n+nn}{formula}\PYG{n+nn}{.}\PYG{n+nn}{api} \PYG{k}{as} \PYG{n+nn}{smf}
\end{sphinxVerbatim}

\end{sphinxuseclass}\end{sphinxVerbatimInput}

\end{sphinxuseclass}
\begin{sphinxuseclass}{cell}\begin{sphinxVerbatimInput}

\begin{sphinxuseclass}{cell_input}
\begin{sphinxVerbatim}[commandchars=\\\{\}]
\PYG{k}{def} \PYG{n+nf}{capm}\PYG{p}{(}\PYG{n}{c}\PYG{p}{,} \PYG{n}{df}\PYG{p}{)}\PYG{p}{:}
    \PYG{k}{return} \PYG{n}{smf}\PYG{o}{.}\PYG{n}{ols}\PYG{p}{(}\PYG{n}{formula}\PYG{o}{=}\PYG{l+s+sa}{f}\PYG{l+s+s1}{\PYGZsq{}}\PYG{l+s+s1}{I(Q(}\PYG{l+s+si}{\PYGZob{}}\PYG{n}{c}\PYG{l+s+si}{\PYGZcb{}}\PYG{l+s+s1}{) \PYGZhy{} RF) \PYGZti{} Q(}\PYG{l+s+s1}{\PYGZdq{}}\PYG{l+s+s1}{Mkt\PYGZhy{}RF}\PYG{l+s+s1}{\PYGZdq{}}\PYG{l+s+s1}{)}\PYG{l+s+s1}{\PYGZsq{}}\PYG{p}{,} \PYG{n}{data}\PYG{o}{=}\PYG{n}{df}\PYG{p}{)}
\end{sphinxVerbatim}

\end{sphinxuseclass}\end{sphinxVerbatimInput}

\end{sphinxuseclass}
\begin{sphinxuseclass}{cell}\begin{sphinxVerbatimInput}

\begin{sphinxuseclass}{cell_input}
\begin{sphinxVerbatim}[commandchars=\\\{\}]
\PYG{n}{\PYGZus{}} \PYG{o}{=} \PYG{n}{mom\PYGZus{}ew}\PYG{o}{.}\PYG{n}{mul}\PYG{p}{(}\PYG{l+m+mi}{100}\PYG{p}{)}\PYG{o}{.}\PYG{n}{join}\PYG{p}{(}\PYG{n}{ff\PYGZus{}0}\PYG{p}{[}\PYG{l+m+mi}{0}\PYG{p}{]}\PYG{p}{)}
\PYG{n}{models} \PYG{o}{=} \PYG{p}{[}\PYG{n}{capm}\PYG{p}{(}\PYG{n}{c}\PYG{o}{=}\PYG{n}{c}\PYG{p}{,} \PYG{n}{df}\PYG{o}{=}\PYG{n}{\PYGZus{}}\PYG{p}{)} \PYG{k}{for} \PYG{n}{c} \PYG{o+ow}{in} \PYG{n}{mom\PYGZus{}ew}\PYG{o}{.}\PYG{n}{columns}\PYG{p}{]}
\PYG{n}{fits} \PYG{o}{=} \PYG{p}{[}\PYG{n}{m}\PYG{o}{.}\PYG{n}{fit}\PYG{p}{(}\PYG{p}{)} \PYG{k}{for} \PYG{n}{m} \PYG{o+ow}{in} \PYG{n}{models}\PYG{p}{]}
\PYG{n}{params} \PYG{o}{=} \PYG{n}{pd}\PYG{o}{.}\PYG{n}{concat}\PYG{p}{(}\PYG{p}{[}\PYG{n}{f}\PYG{o}{.}\PYG{n}{params} \PYG{k}{for} \PYG{n}{f} \PYG{o+ow}{in} \PYG{n}{fits}\PYG{p}{]}\PYG{p}{,} \PYG{n}{axis}\PYG{o}{=}\PYG{l+m+mi}{1}\PYG{p}{,} \PYG{n}{keys}\PYG{o}{=}\PYG{n}{mom\PYGZus{}ew}\PYG{o}{.}\PYG{n}{columns}\PYG{p}{)}\PYG{o}{.}\PYG{n}{T}
\PYG{n}{bses} \PYG{o}{=} \PYG{n}{pd}\PYG{o}{.}\PYG{n}{concat}\PYG{p}{(}\PYG{p}{[}\PYG{n}{f}\PYG{o}{.}\PYG{n}{bse} \PYG{k}{for} \PYG{n}{f} \PYG{o+ow}{in} \PYG{n}{fits}\PYG{p}{]}\PYG{p}{,} \PYG{n}{axis}\PYG{o}{=}\PYG{l+m+mi}{1}\PYG{p}{,} \PYG{n}{keys}\PYG{o}{=}\PYG{n}{mom\PYGZus{}ew}\PYG{o}{.}\PYG{n}{columns}\PYG{p}{)}\PYG{o}{.}\PYG{n}{T}

\PYG{n}{plt}\PYG{o}{.}\PYG{n}{bar}\PYG{p}{(}
    \PYG{n}{x}\PYG{o}{=}\PYG{n}{params}\PYG{o}{.}\PYG{n}{index}\PYG{p}{,}
    \PYG{n}{height}\PYG{o}{=}\PYG{n}{params}\PYG{p}{[}\PYG{l+s+s1}{\PYGZsq{}}\PYG{l+s+s1}{Intercept}\PYG{l+s+s1}{\PYGZsq{}}\PYG{p}{]}\PYG{p}{,}
    \PYG{n}{yerr}\PYG{o}{=}\PYG{n}{bses}\PYG{p}{[}\PYG{l+s+s1}{\PYGZsq{}}\PYG{l+s+s1}{Intercept}\PYG{l+s+s1}{\PYGZsq{}}\PYG{p}{]}
\PYG{p}{)}

\PYG{n}{plt}\PYG{o}{.}\PYG{n}{ylabel}\PYG{p}{(}\PYG{l+s+s1}{\PYGZsq{}}\PYG{l+s+s1}{Mean Equal\PYGZhy{}Weighted Monthly Alpha (}\PYG{l+s+s1}{\PYGZpc{}}\PYG{l+s+s1}{)}\PYG{l+s+s1}{\PYGZsq{}}\PYG{p}{)}
\PYG{n}{plt}\PYG{o}{.}\PYG{n}{xlabel}\PYG{p}{(}\PYG{l+s+s1}{\PYGZsq{}}\PYG{l+s+s1}{Momentum Portfolio}\PYG{l+s+s1}{\PYGZsq{}}\PYG{p}{)}
\PYG{n}{plt}\PYG{o}{.}\PYG{n}{title}\PYG{p}{(}
    \PYG{l+s+s1}{\PYGZsq{}}\PYG{l+s+s1}{Performance of Momentum Investing}\PYG{l+s+s1}{\PYGZsq{}} \PYG{o}{+}
    \PYG{l+s+s1}{\PYGZsq{}}\PYG{l+s+se}{\PYGZbs{}n}\PYG{l+s+s1}{CAPM Alphas for Equal\PYGZhy{}Weighted Portfolios}\PYG{l+s+s1}{\PYGZsq{}} \PYG{o}{+}
    \PYG{l+s+s1}{\PYGZsq{}}\PYG{l+s+se}{\PYGZbs{}n}\PYG{l+s+s1}{Formed on Months \PYGZhy{}12 to Month \PYGZhy{}2 Returns}\PYG{l+s+s1}{\PYGZsq{}}
\PYG{p}{)}
\PYG{n}{plt}\PYG{o}{.}\PYG{n}{show}\PYG{p}{(}\PYG{p}{)}
\end{sphinxVerbatim}

\end{sphinxuseclass}\end{sphinxVerbatimInput}
\begin{sphinxVerbatimOutput}

\begin{sphinxuseclass}{cell_output}
\noindent\sphinxincludegraphics{{5a425b8630149006931e0cdfa23505979e83b45ef4f1ab66db93b19bb3105618}.png}

\end{sphinxuseclass}\end{sphinxVerbatimOutput}

\end{sphinxuseclass}
\begin{sphinxuseclass}{cell}\begin{sphinxVerbatimInput}

\begin{sphinxuseclass}{cell_input}
\begin{sphinxVerbatim}[commandchars=\\\{\}]
\PYG{k}{def} \PYG{n+nf}{ff4}\PYG{p}{(}\PYG{n}{c}\PYG{p}{,} \PYG{n}{df}\PYG{p}{)}\PYG{p}{:}
    \PYG{k}{return} \PYG{n}{smf}\PYG{o}{.}\PYG{n}{ols}\PYG{p}{(}\PYG{n}{formula}\PYG{o}{=}\PYG{l+s+sa}{f}\PYG{l+s+s1}{\PYGZsq{}}\PYG{l+s+s1}{I(Q(}\PYG{l+s+si}{\PYGZob{}}\PYG{n}{c}\PYG{l+s+si}{\PYGZcb{}}\PYG{l+s+s1}{) \PYGZhy{} RF) \PYGZti{} Q(}\PYG{l+s+s1}{\PYGZdq{}}\PYG{l+s+s1}{Mkt\PYGZhy{}RF}\PYG{l+s+s1}{\PYGZdq{}}\PYG{l+s+s1}{) + SMB + HML + Mom}\PYG{l+s+s1}{\PYGZsq{}}\PYG{p}{,} \PYG{n}{data}\PYG{o}{=}\PYG{n}{df}\PYG{p}{)}
\end{sphinxVerbatim}

\end{sphinxuseclass}\end{sphinxVerbatimInput}

\end{sphinxuseclass}
\begin{sphinxuseclass}{cell}\begin{sphinxVerbatimInput}

\begin{sphinxuseclass}{cell_input}
\begin{sphinxVerbatim}[commandchars=\\\{\}]
\PYG{n}{\PYGZus{}} \PYG{o}{=} \PYG{n}{mom\PYGZus{}ew}\PYG{o}{.}\PYG{n}{mul}\PYG{p}{(}\PYG{l+m+mi}{100}\PYG{p}{)}\PYG{o}{.}\PYG{n}{join}\PYG{p}{(}\PYG{p}{[}\PYG{n}{ff\PYGZus{}0}\PYG{p}{[}\PYG{l+m+mi}{0}\PYG{p}{]}\PYG{p}{,} \PYG{n}{ff\PYGZus{}mom}\PYG{p}{[}\PYG{l+m+mi}{0}\PYG{p}{]}\PYG{p}{]}\PYG{p}{)}
\PYG{n}{models} \PYG{o}{=} \PYG{p}{[}\PYG{n}{ff4}\PYG{p}{(}\PYG{n}{c}\PYG{o}{=}\PYG{n}{c}\PYG{p}{,} \PYG{n}{df}\PYG{o}{=}\PYG{n}{\PYGZus{}}\PYG{p}{)} \PYG{k}{for} \PYG{n}{c} \PYG{o+ow}{in} \PYG{n}{mom\PYGZus{}ew}\PYG{o}{.}\PYG{n}{columns}\PYG{p}{]}
\PYG{n}{fits} \PYG{o}{=} \PYG{p}{[}\PYG{n}{m}\PYG{o}{.}\PYG{n}{fit}\PYG{p}{(}\PYG{p}{)} \PYG{k}{for} \PYG{n}{m} \PYG{o+ow}{in} \PYG{n}{models}\PYG{p}{]}
\PYG{n}{params} \PYG{o}{=} \PYG{n}{pd}\PYG{o}{.}\PYG{n}{concat}\PYG{p}{(}\PYG{p}{[}\PYG{n}{f}\PYG{o}{.}\PYG{n}{params} \PYG{k}{for} \PYG{n}{f} \PYG{o+ow}{in} \PYG{n}{fits}\PYG{p}{]}\PYG{p}{,} \PYG{n}{axis}\PYG{o}{=}\PYG{l+m+mi}{1}\PYG{p}{,} \PYG{n}{keys}\PYG{o}{=}\PYG{n}{mom\PYGZus{}ew}\PYG{o}{.}\PYG{n}{columns}\PYG{p}{)}\PYG{o}{.}\PYG{n}{T}
\PYG{n}{bses} \PYG{o}{=} \PYG{n}{pd}\PYG{o}{.}\PYG{n}{concat}\PYG{p}{(}\PYG{p}{[}\PYG{n}{f}\PYG{o}{.}\PYG{n}{bse} \PYG{k}{for} \PYG{n}{f} \PYG{o+ow}{in} \PYG{n}{fits}\PYG{p}{]}\PYG{p}{,} \PYG{n}{axis}\PYG{o}{=}\PYG{l+m+mi}{1}\PYG{p}{,} \PYG{n}{keys}\PYG{o}{=}\PYG{n}{mom\PYGZus{}ew}\PYG{o}{.}\PYG{n}{columns}\PYG{p}{)}\PYG{o}{.}\PYG{n}{T}

\PYG{n}{plt}\PYG{o}{.}\PYG{n}{bar}\PYG{p}{(}
    \PYG{n}{x}\PYG{o}{=}\PYG{n}{params}\PYG{o}{.}\PYG{n}{index}\PYG{p}{,}
    \PYG{n}{height}\PYG{o}{=}\PYG{n}{params}\PYG{p}{[}\PYG{l+s+s1}{\PYGZsq{}}\PYG{l+s+s1}{Intercept}\PYG{l+s+s1}{\PYGZsq{}}\PYG{p}{]}\PYG{p}{,}
    \PYG{n}{yerr}\PYG{o}{=}\PYG{n}{bses}\PYG{p}{[}\PYG{l+s+s1}{\PYGZsq{}}\PYG{l+s+s1}{Intercept}\PYG{l+s+s1}{\PYGZsq{}}\PYG{p}{]}
\PYG{p}{)}

\PYG{n}{plt}\PYG{o}{.}\PYG{n}{ylabel}\PYG{p}{(}\PYG{l+s+s1}{\PYGZsq{}}\PYG{l+s+s1}{Mean Equal\PYGZhy{}Weighted Monthly Alpha (}\PYG{l+s+s1}{\PYGZpc{}}\PYG{l+s+s1}{)}\PYG{l+s+s1}{\PYGZsq{}}\PYG{p}{)}
\PYG{n}{plt}\PYG{o}{.}\PYG{n}{xlabel}\PYG{p}{(}\PYG{l+s+s1}{\PYGZsq{}}\PYG{l+s+s1}{Momentum Portfolio}\PYG{l+s+s1}{\PYGZsq{}}\PYG{p}{)}
\PYG{n}{plt}\PYG{o}{.}\PYG{n}{title}\PYG{p}{(}
    \PYG{l+s+s1}{\PYGZsq{}}\PYG{l+s+s1}{Performance of Momentum Investing}\PYG{l+s+s1}{\PYGZsq{}} \PYG{o}{+}
    \PYG{l+s+s1}{\PYGZsq{}}\PYG{l+s+se}{\PYGZbs{}n}\PYG{l+s+s1}{FF4 Alphas for Equal\PYGZhy{}Weighted Portfolios}\PYG{l+s+s1}{\PYGZsq{}} \PYG{o}{+}
    \PYG{l+s+s1}{\PYGZsq{}}\PYG{l+s+se}{\PYGZbs{}n}\PYG{l+s+s1}{Formed on Months \PYGZhy{}12 to Month \PYGZhy{}2 Returns}\PYG{l+s+s1}{\PYGZsq{}}
\PYG{p}{)}
\PYG{n}{plt}\PYG{o}{.}\PYG{n}{show}\PYG{p}{(}\PYG{p}{)}
\end{sphinxVerbatim}

\end{sphinxuseclass}\end{sphinxVerbatimInput}
\begin{sphinxVerbatimOutput}

\begin{sphinxuseclass}{cell_output}
\noindent\sphinxincludegraphics{{515b74f596ee84d7c5796e7d3fec3d27c9044122c882246431d59e97623e1d54}.png}

\end{sphinxuseclass}\end{sphinxVerbatimOutput}

\end{sphinxuseclass}
\sphinxAtStartPar
In the FF4 model above, the alphas are abnormally large because we use equal\sphinxhyphen{}weighted portfolios, which overweight small stocks.
Because these portfolios overweight small stocks, these alphas make be associated with liquidity and difficult to earn at scale.
In the practice notebook, we will explore value\sphinxhyphen{}weighted portfolios and size investing strategies.

\sphinxstepscope


\section{Herron Topic 6 \sphinxhyphen{} Practice (Blank)}
\label{\detokenize{herron_06_practice:herron-topic-6-practice-blank}}\label{\detokenize{herron_06_practice::doc}}

\subsection{Announcements}
\label{\detokenize{herron_06_practice:announcements}}

\subsection{Practice}
\label{\detokenize{herron_06_practice:practice}}
\begin{sphinxuseclass}{cell}\begin{sphinxVerbatimInput}

\begin{sphinxuseclass}{cell_input}
\begin{sphinxVerbatim}[commandchars=\\\{\}]
\PYG{k+kn}{import} \PYG{n+nn}{matplotlib}\PYG{n+nn}{.}\PYG{n+nn}{pyplot} \PYG{k}{as} \PYG{n+nn}{plt}
\PYG{k+kn}{import} \PYG{n+nn}{numpy} \PYG{k}{as} \PYG{n+nn}{np}
\PYG{k+kn}{import} \PYG{n+nn}{pandas} \PYG{k}{as} \PYG{n+nn}{pd}
\end{sphinxVerbatim}

\end{sphinxuseclass}\end{sphinxVerbatimInput}

\end{sphinxuseclass}
\begin{sphinxuseclass}{cell}\begin{sphinxVerbatimInput}

\begin{sphinxuseclass}{cell_input}
\begin{sphinxVerbatim}[commandchars=\\\{\}]
\PYG{o}{\PYGZpc{}}\PYG{k}{config} InlineBackend.figure\PYGZus{}format = \PYGZsq{}retina\PYGZsq{}
\PYG{o}{\PYGZpc{}}\PYG{k}{precision} 4
\PYG{n}{pd}\PYG{o}{.}\PYG{n}{options}\PYG{o}{.}\PYG{n}{display}\PYG{o}{.}\PYG{n}{float\PYGZus{}format} \PYG{o}{=} \PYG{l+s+s1}{\PYGZsq{}}\PYG{l+s+si}{\PYGZob{}:.4f\PYGZcb{}}\PYG{l+s+s1}{\PYGZsq{}}\PYG{o}{.}\PYG{n}{format}
\end{sphinxVerbatim}

\end{sphinxuseclass}\end{sphinxVerbatimInput}

\end{sphinxuseclass}
\begin{sphinxuseclass}{cell}\begin{sphinxVerbatimInput}

\begin{sphinxuseclass}{cell_input}
\begin{sphinxVerbatim}[commandchars=\\\{\}]
\PYG{k+kn}{import} \PYG{n+nn}{yfinance} \PYG{k}{as} \PYG{n+nn}{yf}
\PYG{k+kn}{import} \PYG{n+nn}{pandas\PYGZus{}datareader} \PYG{k}{as} \PYG{n+nn}{pdr}
\PYG{k+kn}{import} \PYG{n+nn}{requests\PYGZus{}cache}
\PYG{n}{session} \PYG{o}{=} \PYG{n}{requests\PYGZus{}cache}\PYG{o}{.}\PYG{n}{CachedSession}\PYG{p}{(}\PYG{p}{)}
\end{sphinxVerbatim}

\end{sphinxuseclass}\end{sphinxVerbatimInput}

\end{sphinxuseclass}

\subsubsection{Reimplement the equal\sphinxhyphen{}weighted momentum investing strategy from the lecture notebook}
\label{\detokenize{herron_06_practice:reimplement-the-equal-weighted-momentum-investing-strategy-from-the-lecture-notebook}}
\sphinxAtStartPar
Try to use a few cells and temporary variables as you can (i.e., perform calculations inside \sphinxcode{\sphinxupquote{pd.concat()}}).


\subsubsection{Add a long\sphinxhyphen{}short portfolio that is long portfolio 10 and short portfolio 1}
\label{\detokenize{herron_06_practice:add-a-long-short-portfolio-that-is-long-portfolio-10-and-short-portfolio-1}}
\sphinxAtStartPar
Call this long\sphinxhyphen{}short portfolio UMD.
What are the best and worst months for portfolios 1, 10, and UMD?


\subsubsection{What are the Sharpe Ratios on these 11 portfolios?}
\label{\detokenize{herron_06_practice:what-are-the-sharpe-ratios-on-these-11-portfolios}}

\subsubsection{Implement a value\sphinxhyphen{}weighted momentum investing strategy}
\label{\detokenize{herron_06_practice:implement-a-value-weighted-momentum-investing-strategy}}
\sphinxAtStartPar
Assign this strategy to data frame \sphinxcode{\sphinxupquote{mom\_vw}}, and include long\sphinxhyphen{}short portfolio UMD


\subsubsection{What are the CAPM and FF4 alphas for these value\sphinxhyphen{}weighted portfolios?}
\label{\detokenize{herron_06_practice:what-are-the-capm-and-ff4-alphas-for-these-value-weighted-portfolios}}

\subsubsection{What are the Sharpe Ratios for these value\sphinxhyphen{}weighted portfolios?}
\label{\detokenize{herron_06_practice:what-are-the-sharpe-ratios-for-these-value-weighted-portfolios}}

\subsubsection{Implement an equal\sphinxhyphen{}weighted size investing strategy based on market capitalization at the start of each month}
\label{\detokenize{herron_06_practice:implement-an-equal-weighted-size-investing-strategy-based-on-market-capitalization-at-the-start-of-each-month}}

\subsubsection{Implement a value\sphinxhyphen{}weighted size investing strategy based on market capitalization at the start of each month}
\label{\detokenize{herron_06_practice:implement-a-value-weighted-size-investing-strategy-based-on-market-capitalization-at-the-start-of-each-month}}
\sphinxstepscope


\section{Herron Topic 6 \sphinxhyphen{} Practice (Monday 2:45 PM, Section 3)}
\label{\detokenize{herron_06_practice_03:herron-topic-6-practice-monday-2-45-pm-section-3}}\label{\detokenize{herron_06_practice_03::doc}}

\subsection{Announcements}
\label{\detokenize{herron_06_practice_03:announcements}}\begin{itemize}
\item {} 
\sphinxAtStartPar
Quiz 6 mean and median were about \textasciitilde{}\textasciitilde{}90\% and 96\%\textasciitilde{}\textasciitilde{}  92\% and 100\%
\begin{itemize}
\item {} 
\sphinxAtStartPar
Please see the solution on Canvas and let me know if you have any questions

\item {} 
\sphinxAtStartPar
Gradescope shows which hidden tests you missed

\item {} 
\sphinxAtStartPar
I recently widened the tolerance on two hidden tests to give full credit if you minimized variance instead of volatility

\end{itemize}

\item {} 
\sphinxAtStartPar
I posted {\hyperref[\detokenize{project_02::doc}]{\sphinxcrossref{\DUrole{doc,std,std-doc}{project 2}}}} about Bitcoin and gold as inflation and market risk hedges

\item {} 
\sphinxAtStartPar
Next week (week of 4/10) is the assessment exam
\begin{itemize}
\item {} 
\sphinxAtStartPar
MSFQ students must take it for 5\% of overall course grade

\item {} 
\sphinxAtStartPar
Non\sphinxhyphen{}MSQF students do not take it and weight their grades by only 95\%

\item {} 
\sphinxAtStartPar
20 multiple choice questions on the 6 core courses (corporate finance, investments, math, data analytics, empirical methods, and derivatives)

\item {} 
\sphinxAtStartPar
\sphinxstyleemphasis{\sphinxstylestrong{You must be in the classroom during your assigned date and time to take the MSFQ assessment exam}}

\item {} 
\sphinxAtStartPar
If there is interest, we will discuss five stylized facts of asset returns after the MSFQ assessment exam

\item {} 
\sphinxAtStartPar
We will also have our final quiz, quiz 7

\end{itemize}

\item {} 
\sphinxAtStartPar
The following week (week of 4/17) we will discuss Herron topic 5 (simulations)
\begin{itemize}
\item {} 
\sphinxAtStartPar
I will record a lecture video and complete a practice notebook

\item {} 
\sphinxAtStartPar
But we will reserve class time for group work

\item {} 
\sphinxAtStartPar
The class voted about 2\sphinxhyphen{}to\sphinxhyphen{}1 not to drop a topic, so this is a compromise, given that Monday is Patriot’s Day holiday

\end{itemize}

\item {} 
\sphinxAtStartPar
The following week (week of 4/24) we will reserve class time for group work
\begin{itemize}
\item {} 
\sphinxAtStartPar
Project 2 is due Wednesday, 4/26, at 11:59 PM

\item {} 
\sphinxAtStartPar
Teammates Review 2 is due Wednesday, 4/26, at 11:59 PM

\item {} 
\sphinxAtStartPar
30,000 DataCamp XP are due Friday, 4/28, at 11:59 PM

\end{itemize}

\item {} 
\sphinxAtStartPar
\sphinxstyleemphasis{\sphinxstylestrong{Somewhere in there, please reserve 10 minutes to complete a TRACE review for this course}}
\begin{itemize}
\item {} 
\sphinxAtStartPar
I cannot make you complete TRACE reviews

\item {} 
\sphinxAtStartPar
But they are very helpful

\item {} 
\sphinxAtStartPar
I change my courses every semester, hopefully for the better, based on TRACE reviews

\end{itemize}

\end{itemize}


\subsection{Practice}
\label{\detokenize{herron_06_practice_03:practice}}
\begin{sphinxuseclass}{cell}\begin{sphinxVerbatimInput}

\begin{sphinxuseclass}{cell_input}
\begin{sphinxVerbatim}[commandchars=\\\{\}]
\PYG{k+kn}{import} \PYG{n+nn}{matplotlib}\PYG{n+nn}{.}\PYG{n+nn}{pyplot} \PYG{k}{as} \PYG{n+nn}{plt}
\PYG{k+kn}{import} \PYG{n+nn}{numpy} \PYG{k}{as} \PYG{n+nn}{np}
\PYG{k+kn}{import} \PYG{n+nn}{pandas} \PYG{k}{as} \PYG{n+nn}{pd}
\end{sphinxVerbatim}

\end{sphinxuseclass}\end{sphinxVerbatimInput}

\end{sphinxuseclass}
\begin{sphinxuseclass}{cell}\begin{sphinxVerbatimInput}

\begin{sphinxuseclass}{cell_input}
\begin{sphinxVerbatim}[commandchars=\\\{\}]
\PYG{o}{\PYGZpc{}}\PYG{k}{config} InlineBackend.figure\PYGZus{}format = \PYGZsq{}retina\PYGZsq{}
\PYG{o}{\PYGZpc{}}\PYG{k}{precision} 4
\PYG{n}{pd}\PYG{o}{.}\PYG{n}{options}\PYG{o}{.}\PYG{n}{display}\PYG{o}{.}\PYG{n}{float\PYGZus{}format} \PYG{o}{=} \PYG{l+s+s1}{\PYGZsq{}}\PYG{l+s+si}{\PYGZob{}:.4f\PYGZcb{}}\PYG{l+s+s1}{\PYGZsq{}}\PYG{o}{.}\PYG{n}{format}
\end{sphinxVerbatim}

\end{sphinxuseclass}\end{sphinxVerbatimInput}

\end{sphinxuseclass}
\begin{sphinxuseclass}{cell}\begin{sphinxVerbatimInput}

\begin{sphinxuseclass}{cell_input}
\begin{sphinxVerbatim}[commandchars=\\\{\}]
\PYG{k+kn}{import} \PYG{n+nn}{yfinance} \PYG{k}{as} \PYG{n+nn}{yf}
\PYG{k+kn}{import} \PYG{n+nn}{pandas\PYGZus{}datareader} \PYG{k}{as} \PYG{n+nn}{pdr}
\PYG{k+kn}{import} \PYG{n+nn}{requests\PYGZus{}cache}
\PYG{n}{session} \PYG{o}{=} \PYG{n}{requests\PYGZus{}cache}\PYG{o}{.}\PYG{n}{CachedSession}\PYG{p}{(}\PYG{p}{)}
\end{sphinxVerbatim}

\end{sphinxuseclass}\end{sphinxVerbatimInput}

\end{sphinxuseclass}

\subsubsection{Reimplement the equal\sphinxhyphen{}weighted momentum investing strategy from the lecture notebook}
\label{\detokenize{herron_06_practice_03:reimplement-the-equal-weighted-momentum-investing-strategy-from-the-lecture-notebook}}
\sphinxAtStartPar
Try to use a few cells and temporary variables as you can (i.e., perform calculations inside \sphinxcode{\sphinxupquote{pd.concat()}}).

\begin{sphinxuseclass}{cell}\begin{sphinxVerbatimInput}

\begin{sphinxuseclass}{cell_input}
\begin{sphinxVerbatim}[commandchars=\\\{\}]
\PYG{n}{crsp} \PYG{o}{=} \PYG{p}{(}
    \PYG{n}{pd}\PYG{o}{.}\PYG{n}{read\PYGZus{}csv}\PYG{p}{(}
        \PYG{n}{filepath\PYGZus{}or\PYGZus{}buffer}\PYG{o}{=}\PYG{l+s+s1}{\PYGZsq{}}\PYG{l+s+s1}{crsp.csv}\PYG{l+s+s1}{\PYGZsq{}}\PYG{p}{,}
        \PYG{n}{parse\PYGZus{}dates}\PYG{o}{=}\PYG{p}{[}\PYG{l+s+s1}{\PYGZsq{}}\PYG{l+s+s1}{date}\PYG{l+s+s1}{\PYGZsq{}}\PYG{p}{]}\PYG{p}{,}
        \PYG{n}{na\PYGZus{}values}\PYG{o}{=}\PYG{p}{[}\PYG{l+s+s1}{\PYGZsq{}}\PYG{l+s+s1}{A}\PYG{l+s+s1}{\PYGZsq{}}\PYG{p}{,} \PYG{l+s+s1}{\PYGZsq{}}\PYG{l+s+s1}{B}\PYG{l+s+s1}{\PYGZsq{}}\PYG{p}{,} \PYG{l+s+s1}{\PYGZsq{}}\PYG{l+s+s1}{C}\PYG{l+s+s1}{\PYGZsq{}}\PYG{p}{]} \PYG{c+c1}{\PYGZsh{} we can ignore returns coded as A, B, or C}
    \PYG{p}{)}
    \PYG{o}{.}\PYG{n}{assign}\PYG{p}{(}
        \PYG{n}{date}\PYG{o}{=}\PYG{k}{lambda} \PYG{n}{x}\PYG{p}{:} \PYG{n}{x}\PYG{p}{[}\PYG{l+s+s1}{\PYGZsq{}}\PYG{l+s+s1}{date}\PYG{l+s+s1}{\PYGZsq{}}\PYG{p}{]}\PYG{o}{.}\PYG{n}{dt}\PYG{o}{.}\PYG{n}{to\PYGZus{}period}\PYG{p}{(}\PYG{n}{freq}\PYG{o}{=}\PYG{l+s+s1}{\PYGZsq{}}\PYG{l+s+s1}{M}\PYG{l+s+s1}{\PYGZsq{}}\PYG{p}{)}\PYG{p}{,} \PYG{c+c1}{\PYGZsh{} I prefer periods to dates for monthly data}
        \PYG{n}{ME}\PYG{o}{=}\PYG{k}{lambda} \PYG{n}{x}\PYG{p}{:} \PYG{n}{x}\PYG{p}{[}\PYG{l+s+s1}{\PYGZsq{}}\PYG{l+s+s1}{PRC}\PYG{l+s+s1}{\PYGZsq{}}\PYG{p}{]}\PYG{o}{.}\PYG{n}{abs}\PYG{p}{(}\PYG{p}{)} \PYG{o}{*} \PYG{n}{x}\PYG{p}{[}\PYG{l+s+s1}{\PYGZsq{}}\PYG{l+s+s1}{SHROUT}\PYG{l+s+s1}{\PYGZsq{}}\PYG{p}{]} \PYG{c+c1}{\PYGZsh{} market value of equity in thousands of dollars}
    \PYG{p}{)}
    \PYG{o}{.}\PYG{n}{rename\PYGZus{}axis}\PYG{p}{(}\PYG{n}{columns}\PYG{o}{=}\PYG{l+s+s1}{\PYGZsq{}}\PYG{l+s+s1}{Variable}\PYG{l+s+s1}{\PYGZsq{}}\PYG{p}{)}
    \PYG{o}{.}\PYG{n}{set\PYGZus{}index}\PYG{p}{(}\PYG{p}{[}\PYG{l+s+s1}{\PYGZsq{}}\PYG{l+s+s1}{PERMNO}\PYG{l+s+s1}{\PYGZsq{}}\PYG{p}{,} \PYG{l+s+s1}{\PYGZsq{}}\PYG{l+s+s1}{date}\PYG{l+s+s1}{\PYGZsq{}}\PYG{p}{]}\PYG{p}{)}
\PYG{p}{)}
\end{sphinxVerbatim}

\end{sphinxuseclass}\end{sphinxVerbatimInput}

\end{sphinxuseclass}
\sphinxAtStartPar
Please the {\hyperref[\detokenize{herron_06_lecture::doc}]{\sphinxcrossref{\DUrole{doc,std,std-doc}{Herron Topic 6 lecture notebook}}}} for variable definitions.

\begin{sphinxuseclass}{cell}\begin{sphinxVerbatimInput}

\begin{sphinxuseclass}{cell_input}
\begin{sphinxVerbatim}[commandchars=\\\{\}]
\PYG{n}{crsp}\PYG{o}{.}\PYG{n}{head}\PYG{p}{(}\PYG{p}{)}
\end{sphinxVerbatim}

\end{sphinxuseclass}\end{sphinxVerbatimInput}
\begin{sphinxVerbatimOutput}

\begin{sphinxuseclass}{cell_output}
\begin{sphinxVerbatim}[commandchars=\\\{\}]
Variable        SHRCD     PRC     RET    SHROUT         ME
PERMNO date                                               
10000  1986\PYGZhy{}01     10 \PYGZhy{}4.3750     NaN 3680.0000 16100.0000
       1986\PYGZhy{}02     10 \PYGZhy{}3.2500 \PYGZhy{}0.2571 3680.0000 11960.0000
       1986\PYGZhy{}03     10 \PYGZhy{}4.4375  0.3654 3680.0000 16330.0000
       1986\PYGZhy{}04     10 \PYGZhy{}4.0000 \PYGZhy{}0.0986 3793.0000 15172.0000
       1986\PYGZhy{}05     10 \PYGZhy{}3.1094 \PYGZhy{}0.2227 3793.0000 11793.8783
\end{sphinxVerbatim}

\end{sphinxuseclass}\end{sphinxVerbatimOutput}

\end{sphinxuseclass}
\begin{sphinxuseclass}{cell}\begin{sphinxVerbatimInput}

\begin{sphinxuseclass}{cell_input}
\begin{sphinxVerbatim}[commandchars=\\\{\}]
\PYG{n}{crsp}\PYG{o}{.}\PYG{n}{describe}\PYG{p}{(}\PYG{p}{)}
\end{sphinxVerbatim}

\end{sphinxuseclass}\end{sphinxVerbatimInput}
\begin{sphinxVerbatimOutput}

\begin{sphinxuseclass}{cell_output}
\begin{sphinxVerbatim}[commandchars=\\\{\}]
Variable        SHRCD          PRC          RET        SHROUT              ME
count    3794898.0000 3692827.0000 3664978.0000  3791120.0000    3692827.0000
mean          10.7392      31.8232       0.0115    44254.7485    1822175.8177
std            0.4390    1835.1071       0.1799   249579.2094   16811017.1404
min           10.0000   \PYGZhy{}1832.5000      \PYGZhy{}0.9957        0.0000          0.0000
25\PYGZpc{}           10.0000       1.8750      \PYGZhy{}0.0648     2288.0000      16949.2500
50\PYGZpc{}           11.0000      11.0000       0.0000     7336.0000      73164.0000
75\PYGZpc{}           11.0000      25.8000       0.0700    25600.0000     403080.1875
max           11.0000  528921.0000      24.0000 29206400.0000 2902368140.5592
\end{sphinxVerbatim}

\end{sphinxuseclass}\end{sphinxVerbatimOutput}

\end{sphinxuseclass}
\sphinxAtStartPar
To implement a momentum strategy, we need:
\begin{enumerate}
\sphinxsetlistlabels{\arabic}{enumi}{enumii}{}{.}%
\item {} 
\sphinxAtStartPar
1\sphinxhyphen{}month returns, which are the returns we receive for each holding period

\item {} 
\sphinxAtStartPar
11\sphinxhyphen{}month returns, which are the returns (from month \sphinxhyphen{}12 to month \sphinxhyphen{}2) that we use to rank stocks and assign to momentum portfolios

\item {} 
\sphinxAtStartPar
Momentum portfolio assignments based on 11\sphinxhyphen{}month trailing returns

\end{enumerate}

\begin{sphinxuseclass}{cell}\begin{sphinxVerbatimInput}

\begin{sphinxuseclass}{cell_input}
\begin{sphinxVerbatim}[commandchars=\\\{\}]
\PYG{n}{ret\PYGZus{}1m} \PYG{o}{=} \PYG{n}{crsp}\PYG{p}{[}\PYG{l+s+s1}{\PYGZsq{}}\PYG{l+s+s1}{RET}\PYG{l+s+s1}{\PYGZsq{}}\PYG{p}{]}\PYG{o}{.}\PYG{n}{unstack}\PYG{p}{(}\PYG{l+s+s1}{\PYGZsq{}}\PYG{l+s+s1}{PERMNO}\PYG{l+s+s1}{\PYGZsq{}}\PYG{p}{)}
\end{sphinxVerbatim}

\end{sphinxuseclass}\end{sphinxVerbatimInput}

\end{sphinxuseclass}
\sphinxAtStartPar
There is not a \sphinxcode{\sphinxupquote{.prod()}} method for \sphinxcode{\sphinxupquote{.rolling()}}, so the fastest way to calculate 11\sphinxhyphen{}month rolling returns is to:
\begin{enumerate}
\sphinxsetlistlabels{\arabic}{enumi}{enumii}{}{.}%
\item {} 
\sphinxAtStartPar
Convert simple returns to log returns with \sphinxcode{\sphinxupquote{.pipe(np.log1p)}}

\item {} 
\sphinxAtStartPar
Calculate 11\sphinxhyphen{}month rolling \sphinxstyleemphasis{log} returns with \sphinxcode{\sphinxupquote{.rolling(11).sum()}}

\item {} 
\sphinxAtStartPar
Convert log returns to simple returns with \sphinxcode{\sphinxupquote{.pipe(np.expm1)}}

\end{enumerate}

\begin{sphinxuseclass}{cell}\begin{sphinxVerbatimInput}

\begin{sphinxuseclass}{cell_input}
\begin{sphinxVerbatim}[commandchars=\\\{\}]
\PYG{n}{ret\PYGZus{}11m} \PYG{o}{=} \PYG{n}{ret\PYGZus{}1m}\PYG{o}{.}\PYG{n}{pipe}\PYG{p}{(}\PYG{n}{np}\PYG{o}{.}\PYG{n}{log1p}\PYG{p}{)}\PYG{o}{.}\PYG{n}{rolling}\PYG{p}{(}\PYG{l+m+mi}{11}\PYG{p}{)}\PYG{o}{.}\PYG{n}{sum}\PYG{p}{(}\PYG{p}{)}\PYG{o}{.}\PYG{n}{pipe}\PYG{p}{(}\PYG{n}{np}\PYG{o}{.}\PYG{n}{expm1}\PYG{p}{)}
\end{sphinxVerbatim}

\end{sphinxuseclass}\end{sphinxVerbatimInput}

\end{sphinxuseclass}
\sphinxAtStartPar
Then we use \sphinxcode{\sphinxupquote{pd.qcut()}} to assign \sphinxcode{\sphinxupquote{ret\_11m}} to ten momentum portfolios.
Here is a simple example:

\begin{sphinxuseclass}{cell}\begin{sphinxVerbatimInput}

\begin{sphinxuseclass}{cell_input}
\begin{sphinxVerbatim}[commandchars=\\\{\}]
\PYG{n}{pd}\PYG{o}{.}\PYG{n}{qcut}\PYG{p}{(}\PYG{n}{x}\PYG{o}{=}\PYG{n}{np}\PYG{o}{.}\PYG{n}{arange}\PYG{p}{(}\PYG{l+m+mi}{20}\PYG{p}{)}\PYG{p}{,} \PYG{n}{q}\PYG{o}{=}\PYG{l+m+mi}{10}\PYG{p}{,} \PYG{n}{labels}\PYG{o}{=}\PYG{k+kc}{False}\PYG{p}{)}
\end{sphinxVerbatim}

\end{sphinxuseclass}\end{sphinxVerbatimInput}
\begin{sphinxVerbatimOutput}

\begin{sphinxuseclass}{cell_output}
\begin{sphinxVerbatim}[commandchars=\\\{\}]
array([0, 0, 1, 1, 2, 2, 3, 3, 4, 4, 5, 5, 6, 6, 7, 7, 8, 8, 9, 9])
\end{sphinxVerbatim}

\end{sphinxuseclass}\end{sphinxVerbatimOutput}

\end{sphinxuseclass}
\sphinxAtStartPar
Here are the momentum portfolios:

\begin{sphinxuseclass}{cell}\begin{sphinxVerbatimInput}

\begin{sphinxuseclass}{cell_input}
\begin{sphinxVerbatim}[commandchars=\\\{\}]
\PYG{n}{port\PYGZus{}11m} \PYG{o}{=} \PYG{n}{ret\PYGZus{}11m}\PYG{o}{.}\PYG{n}{dropna}\PYG{p}{(}\PYG{n}{how}\PYG{o}{=}\PYG{l+s+s1}{\PYGZsq{}}\PYG{l+s+s1}{all}\PYG{l+s+s1}{\PYGZsq{}}\PYG{p}{)}\PYG{o}{.}\PYG{n}{apply}\PYG{p}{(}\PYG{n}{pd}\PYG{o}{.}\PYG{n}{qcut}\PYG{p}{,} \PYG{n}{q}\PYG{o}{=}\PYG{l+m+mi}{10}\PYG{p}{,} \PYG{n}{labels}\PYG{o}{=}\PYG{k+kc}{False}\PYG{p}{,} \PYG{n}{axis}\PYG{o}{=}\PYG{l+m+mi}{1}\PYG{p}{)}
\end{sphinxVerbatim}

\end{sphinxuseclass}\end{sphinxVerbatimInput}

\end{sphinxuseclass}
\sphinxAtStartPar
Short\sphinxhyphen{}term reversal, momentum, and long\sphinxhyphen{}term reversal represent distinct aspects of asset price behavior that have been observed and documented in the financial markets. As a researcher, I am particularly interested in understanding the underlying factors that contribute to these market anomalies and their implications for trading strategies and portfolio management.

\sphinxAtStartPar
Short\sphinxhyphen{}term reversal captures the tendency of asset prices to reverse direction in a brief period, typically days to weeks. This phenomenon can be associated with market overreactions to news events, temporary liquidity constraints, or other transient factors. Astute investors may capitalize on these short\sphinxhyphen{}term reversals by adopting a contrarian approach, buying assets that have recently underperformed and selling those that have outperformed.

\sphinxAtStartPar
Momentum represents the persistence of asset price trends over a short to medium\sphinxhyphen{}term horizon, typically ranging from 3 to 12 months. This market anomaly suggests that assets with strong recent performance continue to outperform, while poorly performing assets continue to underperform. Momentum can be linked to factors such as investors’ behavioral biases, positive feedback trading, and herd mentality. To exploit momentum, investors may adopt a trend\sphinxhyphen{}following strategy, buying assets with strong recent performance and selling those with weak performance.

\sphinxAtStartPar
Long\sphinxhyphen{}term reversal, often referred to as mean reversion, is the tendency of asset prices to revert to their long\sphinxhyphen{}term average or fundamental value over an extended period, typically several years. This phenomenon is thought to result from market participants gradually recognizing and correcting mispricing, leading to price adjustments. Investors following a long\sphinxhyphen{}term reversal strategy may focus on assets with extreme valuation levels, buying undervalued assets and selling overvalued ones, with the expectation that prices will eventually revert to their long\sphinxhyphen{}term mean.

\sphinxAtStartPar
\sphinxstyleemphasis{\sphinxstylestrong{To avoid short\sphinxhyphen{}term reversal, we skip one month between 11\sphinxhyphen{}month returns and portfolio formation.}}

\begin{sphinxuseclass}{cell}\begin{sphinxVerbatimInput}

\begin{sphinxuseclass}{cell_input}
\begin{sphinxVerbatim}[commandchars=\\\{\}]
\PYG{n}{mom} \PYG{o}{=} \PYG{p}{(}
    \PYG{n}{pd}\PYG{o}{.}\PYG{n}{concat}\PYG{p}{(}
        \PYG{n}{objs}\PYG{o}{=}\PYG{p}{[}
            \PYG{n}{ret\PYGZus{}1m}\PYG{p}{,} \PYG{c+c1}{\PYGZsh{} holding period return}
            \PYG{n}{ret\PYGZus{}11m}\PYG{o}{.}\PYG{n}{shift}\PYG{p}{(}\PYG{l+m+mi}{2}\PYG{p}{)}\PYG{p}{,} \PYG{c+c1}{\PYGZsh{} trailing returns}
            \PYG{n}{port\PYGZus{}11m}\PYG{o}{.}\PYG{n}{shift}\PYG{p}{(}\PYG{l+m+mi}{2}\PYG{p}{)}\PYG{p}{,} \PYG{c+c1}{\PYGZsh{} portfolio assignment based on trailing return}
            \PYG{n}{crsp}\PYG{p}{[}\PYG{l+s+s1}{\PYGZsq{}}\PYG{l+s+s1}{ME}\PYG{l+s+s1}{\PYGZsq{}}\PYG{p}{]}\PYG{o}{.}\PYG{n}{unstack}\PYG{p}{(}\PYG{l+s+s1}{\PYGZsq{}}\PYG{l+s+s1}{PERMNO}\PYG{l+s+s1}{\PYGZsq{}}\PYG{p}{)}\PYG{o}{.}\PYG{n}{shift}\PYG{p}{(}\PYG{l+m+mi}{1}\PYG{p}{)} \PYG{c+c1}{\PYGZsh{} beginning\PYGZhy{}of\PYGZhy{}month market value of equity}
        \PYG{p}{]}\PYG{p}{,}
        \PYG{n}{keys}\PYG{o}{=}\PYG{p}{[}\PYG{l+s+s1}{\PYGZsq{}}\PYG{l+s+s1}{Return}\PYG{l+s+s1}{\PYGZsq{}}\PYG{p}{,} \PYG{l+s+s1}{\PYGZsq{}}\PYG{l+s+s1}{Trailing Return}\PYG{l+s+s1}{\PYGZsq{}}\PYG{p}{,} \PYG{l+s+s1}{\PYGZsq{}}\PYG{l+s+s1}{Portfolio}\PYG{l+s+s1}{\PYGZsq{}}\PYG{p}{,} \PYG{l+s+s1}{\PYGZsq{}}\PYG{l+s+s1}{Trailing ME}\PYG{l+s+s1}{\PYGZsq{}}\PYG{p}{]}\PYG{p}{,}
        \PYG{n}{names}\PYG{o}{=}\PYG{p}{[}\PYG{l+s+s1}{\PYGZsq{}}\PYG{l+s+s1}{Variable}\PYG{l+s+s1}{\PYGZsq{}}\PYG{p}{]}\PYG{p}{,}
        \PYG{n}{axis}\PYG{o}{=}\PYG{l+m+mi}{1}
    \PYG{p}{)}
    \PYG{o}{.}\PYG{n}{stack}\PYG{p}{(}\PYG{l+s+s1}{\PYGZsq{}}\PYG{l+s+s1}{PERMNO}\PYG{l+s+s1}{\PYGZsq{}}\PYG{p}{)}
    \PYG{o}{.}\PYG{n}{dropna}\PYG{p}{(}\PYG{p}{)}
    \PYG{o}{.}\PYG{n}{assign}\PYG{p}{(}\PYG{n}{Portfolio}\PYG{o}{=}\PYG{k}{lambda} \PYG{n}{x}\PYG{p}{:} \PYG{n}{x}\PYG{p}{[}\PYG{l+s+s1}{\PYGZsq{}}\PYG{l+s+s1}{Portfolio}\PYG{l+s+s1}{\PYGZsq{}}\PYG{p}{]}\PYG{o}{.}\PYG{n}{astype}\PYG{p}{(}\PYG{n+nb}{int}\PYG{p}{)} \PYG{o}{+} \PYG{l+m+mi}{1}\PYG{p}{)}
\PYG{p}{)}
\end{sphinxVerbatim}

\end{sphinxuseclass}\end{sphinxVerbatimInput}

\end{sphinxuseclass}
\begin{sphinxuseclass}{cell}\begin{sphinxVerbatimInput}

\begin{sphinxuseclass}{cell_input}
\begin{sphinxVerbatim}[commandchars=\\\{\}]
\PYG{n}{mom\PYGZus{}ew} \PYG{o}{=} \PYG{p}{(}
    \PYG{n}{mom}
    \PYG{o}{.}\PYG{n}{groupby}\PYG{p}{(}\PYG{p}{[}\PYG{l+s+s1}{\PYGZsq{}}\PYG{l+s+s1}{date}\PYG{l+s+s1}{\PYGZsq{}}\PYG{p}{,} \PYG{l+s+s1}{\PYGZsq{}}\PYG{l+s+s1}{Portfolio}\PYG{l+s+s1}{\PYGZsq{}}\PYG{p}{]}\PYG{p}{)}
    \PYG{p}{[}\PYG{l+s+s1}{\PYGZsq{}}\PYG{l+s+s1}{Return}\PYG{l+s+s1}{\PYGZsq{}}\PYG{p}{]}
    \PYG{o}{.}\PYG{n}{mean}\PYG{p}{(}\PYG{p}{)} \PYG{c+c1}{\PYGZsh{} simple mean is an equal\PYGZhy{}weighted portfolio return}
    \PYG{o}{.}\PYG{n}{unstack}\PYG{p}{(}\PYG{l+s+s1}{\PYGZsq{}}\PYG{l+s+s1}{Portfolio}\PYG{l+s+s1}{\PYGZsq{}}\PYG{p}{)}
    \PYG{o}{.}\PYG{n}{add\PYGZus{}prefix}\PYG{p}{(}\PYG{l+s+s1}{\PYGZsq{}}\PYG{l+s+s1}{Mom }\PYG{l+s+s1}{\PYGZsq{}}\PYG{p}{)} \PYG{c+c1}{\PYGZsh{} simplifies regression specifications below}
\PYG{p}{)}
\end{sphinxVerbatim}

\end{sphinxuseclass}\end{sphinxVerbatimInput}

\end{sphinxuseclass}
\begin{sphinxuseclass}{cell}\begin{sphinxVerbatimInput}

\begin{sphinxuseclass}{cell_input}
\begin{sphinxVerbatim}[commandchars=\\\{\}]
\PYG{n}{mom\PYGZus{}ew}\PYG{o}{.}\PYG{n}{head}\PYG{p}{(}\PYG{p}{)}
\end{sphinxVerbatim}

\end{sphinxuseclass}\end{sphinxVerbatimInput}
\begin{sphinxVerbatimOutput}

\begin{sphinxuseclass}{cell_output}
\begin{sphinxVerbatim}[commandchars=\\\{\}]
Portfolio   Mom 1   Mom 2   Mom 3   Mom 4   Mom 5   Mom 6  Mom 7   Mom 8  \PYGZbs{}
date                                                                       
1927\PYGZhy{}01   \PYGZhy{}0.0006  0.0149  0.0425  0.0378  0.0009  0.0150 0.0124  0.0076   
1927\PYGZhy{}02    0.0641  0.0484  0.0866  0.0745  0.0436  0.0547 0.0561  0.0403   
1927\PYGZhy{}03   \PYGZhy{}0.0530 \PYGZhy{}0.0381 \PYGZhy{}0.0243 \PYGZhy{}0.0408 \PYGZhy{}0.0192 \PYGZhy{}0.0201 0.0097 \PYGZhy{}0.0084   
1927\PYGZhy{}04    0.0230 \PYGZhy{}0.0085 \PYGZhy{}0.0054 \PYGZhy{}0.0061  0.0002 \PYGZhy{}0.0032 0.0138  0.0030   
1927\PYGZhy{}05    0.0069  0.0353  0.0604  0.0536  0.0700  0.0950 0.0796  0.0838   

Portfolio   Mom 9  Mom 10  
date                       
1927\PYGZhy{}01   \PYGZhy{}0.0044  0.0126  
1927\PYGZhy{}02    0.0579  0.0541  
1927\PYGZhy{}03   \PYGZhy{}0.0058  0.0012  
1927\PYGZhy{}04    0.0323  0.0421  
1927\PYGZhy{}05    0.0803  0.0886  
\end{sphinxVerbatim}

\end{sphinxuseclass}\end{sphinxVerbatimOutput}

\end{sphinxuseclass}
\sphinxAtStartPar
We see a (nearly) monotonic relation between momentum portfolios (based on trailing 11\sphinxhyphen{}month returns) and holding period returns!

\begin{sphinxuseclass}{cell}\begin{sphinxVerbatimInput}

\begin{sphinxuseclass}{cell_input}
\begin{sphinxVerbatim}[commandchars=\\\{\}]
\PYG{n}{mom\PYGZus{}ew}\PYG{o}{.}\PYG{n}{mean}\PYG{p}{(}\PYG{p}{)}\PYG{o}{.}\PYG{n}{mul}\PYG{p}{(}\PYG{l+m+mi}{100}\PYG{p}{)}\PYG{o}{.}\PYG{n}{plot}\PYG{p}{(}\PYG{n}{kind}\PYG{o}{=}\PYG{l+s+s1}{\PYGZsq{}}\PYG{l+s+s1}{bar}\PYG{l+s+s1}{\PYGZsq{}}\PYG{p}{)}
\PYG{n}{plt}\PYG{o}{.}\PYG{n}{ylabel}\PYG{p}{(}\PYG{l+s+s1}{\PYGZsq{}}\PYG{l+s+s1}{Mean Monthly Return (}\PYG{l+s+s1}{\PYGZpc{}}\PYG{l+s+s1}{)}\PYG{l+s+s1}{\PYGZsq{}}\PYG{p}{)}
\PYG{n}{plt}\PYG{o}{.}\PYG{n}{title}\PYG{p}{(}\PYG{l+s+s1}{\PYGZsq{}}\PYG{l+s+s1}{Momentum Strategy}\PYG{l+s+se}{\PYGZbs{}n}\PYG{l+s+s1}{ Equal\PYGZhy{}Weighted Portfolios}\PYG{l+s+se}{\PYGZbs{}n}\PYG{l+s+s1}{Formed on Months \PYGZhy{}12 to \PYGZhy{}2}\PYG{l+s+s1}{\PYGZsq{}}\PYG{p}{)}
\PYG{n}{plt}\PYG{o}{.}\PYG{n}{show}\PYG{p}{(}\PYG{p}{)}
\end{sphinxVerbatim}

\end{sphinxuseclass}\end{sphinxVerbatimInput}
\begin{sphinxVerbatimOutput}

\begin{sphinxuseclass}{cell_output}
\noindent\sphinxincludegraphics{{dcee11a00fc596db568bc2b9d9a59c1ad3c930ab1393a44d9ab81f82af9fe5b0}.png}

\end{sphinxuseclass}\end{sphinxVerbatimOutput}

\end{sphinxuseclass}
\sphinxAtStartPar
This relation is stronger when we consider long\sphinxhyphen{}term, buy\sphinxhyphen{}and\sphinxhyphen{}hold returns!

\begin{sphinxuseclass}{cell}\begin{sphinxVerbatimInput}

\begin{sphinxuseclass}{cell_input}
\begin{sphinxVerbatim}[commandchars=\\\{\}]
\PYG{n}{mom\PYGZus{}ew}\PYG{o}{.}\PYG{n}{add}\PYG{p}{(}\PYG{l+m+mi}{1}\PYG{p}{)}\PYG{o}{.}\PYG{n}{cumprod}\PYG{p}{(}\PYG{p}{)}\PYG{o}{.}\PYG{n}{plot}\PYG{p}{(}\PYG{p}{)}
\PYG{n}{plt}\PYG{o}{.}\PYG{n}{semilogy}\PYG{p}{(}\PYG{p}{)}
\PYG{n}{plt}\PYG{o}{.}\PYG{n}{ylabel}\PYG{p}{(}\PYG{l+s+s1}{\PYGZsq{}}\PYG{l+s+s1}{Value of \PYGZdl{}1 Investment (\PYGZdl{})}\PYG{l+s+s1}{\PYGZsq{}}\PYG{p}{)}
\PYG{n}{plt}\PYG{o}{.}\PYG{n}{title}\PYG{p}{(}\PYG{l+s+s1}{\PYGZsq{}}\PYG{l+s+s1}{Momentum Strategy}\PYG{l+s+se}{\PYGZbs{}n}\PYG{l+s+s1}{ Equal\PYGZhy{}Weighted Portfolios}\PYG{l+s+se}{\PYGZbs{}n}\PYG{l+s+s1}{Formed on Months \PYGZhy{}12 to \PYGZhy{}2}\PYG{l+s+s1}{\PYGZsq{}}\PYG{p}{)}
\PYG{n}{plt}\PYG{o}{.}\PYG{n}{show}\PYG{p}{(}\PYG{p}{)}
\end{sphinxVerbatim}

\end{sphinxuseclass}\end{sphinxVerbatimInput}
\begin{sphinxVerbatimOutput}

\begin{sphinxuseclass}{cell_output}
\noindent\sphinxincludegraphics{{926694910f626f471321b9624464058f45fc236a7e4ba5d6cd066f7a37832bd1}.png}

\end{sphinxuseclass}\end{sphinxVerbatimOutput}

\end{sphinxuseclass}

\subsubsection{Add a long\sphinxhyphen{}short portfolio that is long portfolio 10 and short portfolio 1}
\label{\detokenize{herron_06_practice_03:add-a-long-short-portfolio-that-is-long-portfolio-10-and-short-portfolio-1}}
\sphinxAtStartPar
Call this long\sphinxhyphen{}short portfolio UMD.
What are the best and worst months for portfolios 1, 10, and UMD?

\begin{sphinxuseclass}{cell}\begin{sphinxVerbatimInput}

\begin{sphinxuseclass}{cell_input}
\begin{sphinxVerbatim}[commandchars=\\\{\}]
\PYG{n}{mom\PYGZus{}ew} \PYG{o}{=} \PYG{n}{mom\PYGZus{}ew}\PYG{o}{.}\PYG{n}{assign}\PYG{p}{(}\PYG{n}{UMD}\PYG{o}{=}\PYG{k}{lambda} \PYG{n}{x}\PYG{p}{:} \PYG{n}{x}\PYG{p}{[}\PYG{l+s+s1}{\PYGZsq{}}\PYG{l+s+s1}{Mom 10}\PYG{l+s+s1}{\PYGZsq{}}\PYG{p}{]} \PYG{o}{\PYGZhy{}} \PYG{n}{x}\PYG{p}{[}\PYG{l+s+s1}{\PYGZsq{}}\PYG{l+s+s1}{Mom 1}\PYG{l+s+s1}{\PYGZsq{}}\PYG{p}{]}\PYG{p}{)}
\end{sphinxVerbatim}

\end{sphinxuseclass}\end{sphinxVerbatimInput}

\end{sphinxuseclass}
\sphinxAtStartPar
The worst month for the long\sphinxhyphen{}short portfolio is during the Great Depression.

\begin{sphinxuseclass}{cell}\begin{sphinxVerbatimInput}

\begin{sphinxuseclass}{cell_input}
\begin{sphinxVerbatim}[commandchars=\\\{\}]
\PYG{n}{mom\PYGZus{}ew}\PYG{p}{[}\PYG{l+s+s1}{\PYGZsq{}}\PYG{l+s+s1}{UMD}\PYG{l+s+s1}{\PYGZsq{}}\PYG{p}{]}\PYG{o}{.}\PYG{n}{idxmin}\PYG{p}{(}\PYG{p}{)}
\end{sphinxVerbatim}

\end{sphinxuseclass}\end{sphinxVerbatimInput}
\begin{sphinxVerbatimOutput}

\begin{sphinxuseclass}{cell_output}
\begin{sphinxVerbatim}[commandchars=\\\{\}]
Period(\PYGZsq{}1932\PYGZhy{}08\PYGZsq{}, \PYGZsq{}M\PYGZsq{})
\end{sphinxVerbatim}

\end{sphinxuseclass}\end{sphinxVerbatimOutput}

\end{sphinxuseclass}
\sphinxAtStartPar
Momentum strategies appear to generate large absolute returns.
However, long\sphinxhyphen{}short momentum strategies are occasionally wiped out by large market recoveries, that the short side of the portfolio does not survive in practice.
During downturns, the short side of the portfolio (i.e., the loser stocks) has a high beta.
When the market unexpectedly and quickly recovers, the exaggerated recovery in the high\sphinxhyphen{}beta stocks wipes out the long\sphinxhyphen{}short portfolio.
For example, if we invest \$1 in the long\sphinxhyphen{}short portfolio at the end of 1929, the long\sphinxhyphen{}short portfolio is effectively wiped out during August 1932.

\begin{sphinxuseclass}{cell}\begin{sphinxVerbatimInput}

\begin{sphinxuseclass}{cell_input}
\begin{sphinxVerbatim}[commandchars=\\\{\}]
\PYG{n}{mom\PYGZus{}ew}\PYG{o}{.}\PYG{n}{loc}\PYG{p}{[}\PYG{l+s+s1}{\PYGZsq{}}\PYG{l+s+s1}{1930}\PYG{l+s+s1}{\PYGZsq{}}\PYG{p}{:}\PYG{l+s+s1}{\PYGZsq{}}\PYG{l+s+s1}{1932}\PYG{l+s+s1}{\PYGZsq{}}\PYG{p}{,} \PYG{p}{[}\PYG{l+s+s1}{\PYGZsq{}}\PYG{l+s+s1}{Mom 1}\PYG{l+s+s1}{\PYGZsq{}}\PYG{p}{,} \PYG{l+s+s1}{\PYGZsq{}}\PYG{l+s+s1}{Mom 10}\PYG{l+s+s1}{\PYGZsq{}}\PYG{p}{,} \PYG{l+s+s1}{\PYGZsq{}}\PYG{l+s+s1}{UMD}\PYG{l+s+s1}{\PYGZsq{}}\PYG{p}{]}\PYG{p}{]}\PYG{o}{.}\PYG{n}{add}\PYG{p}{(}\PYG{l+m+mi}{1}\PYG{p}{)}\PYG{o}{.}\PYG{n}{cumprod}\PYG{p}{(}\PYG{p}{)}\PYG{o}{.}\PYG{n}{plot}\PYG{p}{(}\PYG{p}{)}
\PYG{n}{plt}\PYG{o}{.}\PYG{n}{ylabel}\PYG{p}{(}\PYG{l+s+s1}{\PYGZsq{}}\PYG{l+s+s1}{Value of \PYGZdl{}1 Investment (\PYGZdl{})}\PYG{l+s+s1}{\PYGZsq{}}\PYG{p}{)}
\PYG{n}{plt}\PYG{o}{.}\PYG{n}{title}\PYG{p}{(}\PYG{l+s+s1}{\PYGZsq{}}\PYG{l+s+s1}{Momentum Strategy}\PYG{l+s+se}{\PYGZbs{}n}\PYG{l+s+s1}{ Equal\PYGZhy{}Weighted Portfolios}\PYG{l+s+se}{\PYGZbs{}n}\PYG{l+s+s1}{Formed on Months \PYGZhy{}12 to \PYGZhy{}2}\PYG{l+s+s1}{\PYGZsq{}}\PYG{p}{)}
\PYG{n}{plt}\PYG{o}{.}\PYG{n}{show}\PYG{p}{(}\PYG{p}{)}
\end{sphinxVerbatim}

\end{sphinxuseclass}\end{sphinxVerbatimInput}
\begin{sphinxVerbatimOutput}

\begin{sphinxuseclass}{cell_output}
\noindent\sphinxincludegraphics{{4162e1d75dbe30fb44095151c1aa1734e90e085b17d4b8236c0f127dd64e09f2}.png}

\end{sphinxuseclass}\end{sphinxVerbatimOutput}

\end{sphinxuseclass}

\subsubsection{What are the CAPM and FF4 alphas for these equal\sphinxhyphen{}weighted portfolios?}
\label{\detokenize{herron_06_practice_03:what-are-the-capm-and-ff4-alphas-for-these-equal-weighted-portfolios}}
\begin{sphinxuseclass}{cell}\begin{sphinxVerbatimInput}

\begin{sphinxuseclass}{cell_input}
\begin{sphinxVerbatim}[commandchars=\\\{\}]
\PYG{k+kn}{import} \PYG{n+nn}{statsmodels}\PYG{n+nn}{.}\PYG{n+nn}{formula}\PYG{n+nn}{.}\PYG{n+nn}{api} \PYG{k}{as} \PYG{n+nn}{smf}
\end{sphinxVerbatim}

\end{sphinxuseclass}\end{sphinxVerbatimInput}

\end{sphinxuseclass}
\sphinxAtStartPar
\sphinxstyleemphasis{\sphinxstylestrong{After class, I made the following code more repeatable.}}

\begin{sphinxuseclass}{cell}\begin{sphinxVerbatimInput}

\begin{sphinxuseclass}{cell_input}
\begin{sphinxVerbatim}[commandchars=\\\{\}]
\PYG{k}{def} \PYG{n+nf}{capm}\PYG{p}{(}\PYG{n}{df}\PYG{p}{,} \PYG{n}{c}\PYG{p}{)}\PYG{p}{:}
    \PYG{k}{return} \PYG{n}{smf}\PYG{o}{.}\PYG{n}{ols}\PYG{p}{(}\PYG{n}{formula}\PYG{o}{=}\PYG{l+s+sa}{f}\PYG{l+s+s1}{\PYGZsq{}}\PYG{l+s+s1}{I(Q(}\PYG{l+s+s1}{\PYGZdq{}}\PYG{l+s+si}{\PYGZob{}}\PYG{n}{c}\PYG{l+s+si}{\PYGZcb{}}\PYG{l+s+s1}{\PYGZdq{}}\PYG{l+s+s1}{)\PYGZhy{}RF) \PYGZti{} Q(}\PYG{l+s+s1}{\PYGZdq{}}\PYG{l+s+s1}{Mkt\PYGZhy{}RF}\PYG{l+s+s1}{\PYGZdq{}}\PYG{l+s+s1}{)}\PYG{l+s+s1}{\PYGZsq{}}\PYG{p}{,} \PYG{n}{data}\PYG{o}{=}\PYG{n}{df}\PYG{p}{)}
\end{sphinxVerbatim}

\end{sphinxuseclass}\end{sphinxVerbatimInput}

\end{sphinxuseclass}
\begin{sphinxuseclass}{cell}\begin{sphinxVerbatimInput}

\begin{sphinxuseclass}{cell_input}
\begin{sphinxVerbatim}[commandchars=\\\{\}]
\PYG{n}{ff\PYGZus{}0} \PYG{o}{=} \PYG{n}{pdr}\PYG{o}{.}\PYG{n}{DataReader}\PYG{p}{(}
    \PYG{n}{name}\PYG{o}{=}\PYG{l+s+s1}{\PYGZsq{}}\PYG{l+s+s1}{F\PYGZhy{}F\PYGZus{}Research\PYGZus{}Data\PYGZus{}Factors}\PYG{l+s+s1}{\PYGZsq{}}\PYG{p}{,}
    \PYG{n}{data\PYGZus{}source}\PYG{o}{=}\PYG{l+s+s1}{\PYGZsq{}}\PYG{l+s+s1}{famafrench}\PYG{l+s+s1}{\PYGZsq{}}\PYG{p}{,}
    \PYG{n}{start}\PYG{o}{=}\PYG{l+s+s1}{\PYGZsq{}}\PYG{l+s+s1}{1900}\PYG{l+s+s1}{\PYGZsq{}}\PYG{p}{,}
    \PYG{n}{session}\PYG{o}{=}\PYG{n}{session}
\PYG{p}{)}
\end{sphinxVerbatim}

\end{sphinxuseclass}\end{sphinxVerbatimInput}

\end{sphinxuseclass}
\begin{sphinxuseclass}{cell}\begin{sphinxVerbatimInput}

\begin{sphinxuseclass}{cell_input}
\begin{sphinxVerbatim}[commandchars=\\\{\}]
\PYG{n}{ff\PYGZus{}mom} \PYG{o}{=} \PYG{n}{pdr}\PYG{o}{.}\PYG{n}{DataReader}\PYG{p}{(}
    \PYG{n}{name}\PYG{o}{=}\PYG{l+s+s1}{\PYGZsq{}}\PYG{l+s+s1}{F\PYGZhy{}F\PYGZus{}Momentum\PYGZus{}Factor}\PYG{l+s+s1}{\PYGZsq{}}\PYG{p}{,}
    \PYG{n}{data\PYGZus{}source}\PYG{o}{=}\PYG{l+s+s1}{\PYGZsq{}}\PYG{l+s+s1}{famafrench}\PYG{l+s+s1}{\PYGZsq{}}\PYG{p}{,}
    \PYG{n}{start}\PYG{o}{=}\PYG{l+s+s1}{\PYGZsq{}}\PYG{l+s+s1}{1900}\PYG{l+s+s1}{\PYGZsq{}}\PYG{p}{,}
    \PYG{n}{session}\PYG{o}{=}\PYG{n}{session}
\PYG{p}{)}

\PYG{n}{ff\PYGZus{}mom}\PYG{p}{[}\PYG{l+m+mi}{0}\PYG{p}{]}\PYG{o}{.}\PYG{n}{columns} \PYG{o}{=} \PYG{p}{[}\PYG{l+s+s1}{\PYGZsq{}}\PYG{l+s+s1}{Mom}\PYG{l+s+s1}{\PYGZsq{}}\PYG{p}{]} \PYG{c+c1}{\PYGZsh{} we need to rename the Mom factor to remove leading and trailing whitespace}
\end{sphinxVerbatim}

\end{sphinxuseclass}\end{sphinxVerbatimInput}

\end{sphinxuseclass}
\sphinxAtStartPar
The \sphinxcode{\sphinxupquote{get\_coefs()}} function accepts a data frame of returns and factors data, a factor model regression function name, and number \$n\$ for the first \$n\$ columns in the data frame.
The factor model regression function must return a statsmodel model.
The \sphinxcode{\sphinxupquote{get\_coefs()}} function returns a data frame of coefficient estimates and standard errors.
We can use the \sphinxcode{\sphinxupquote{get\_coefs()}} function to quickly apply a factor model regression function.

\begin{sphinxuseclass}{cell}\begin{sphinxVerbatimInput}

\begin{sphinxuseclass}{cell_input}
\begin{sphinxVerbatim}[commandchars=\\\{\}]
\PYG{k}{def} \PYG{n+nf}{get\PYGZus{}coefs}\PYG{p}{(}\PYG{n}{df}\PYG{p}{,} \PYG{n}{fun}\PYG{p}{,} \PYG{n}{n}\PYG{p}{)}\PYG{p}{:}
    \PYG{n}{models} \PYG{o}{=} \PYG{p}{[}\PYG{n}{df}\PYG{o}{.}\PYG{n}{pipe}\PYG{p}{(}\PYG{n}{fun}\PYG{p}{,} \PYG{n}{c}\PYG{p}{)} \PYG{k}{for} \PYG{n}{c} \PYG{o+ow}{in} \PYG{n}{df}\PYG{o}{.}\PYG{n}{columns}\PYG{p}{[}\PYG{p}{:}\PYG{n}{n}\PYG{p}{]}\PYG{p}{]}
    \PYG{n}{fits} \PYG{o}{=} \PYG{p}{[}\PYG{n}{m}\PYG{o}{.}\PYG{n}{fit}\PYG{p}{(}\PYG{p}{)} \PYG{k}{for} \PYG{n}{m} \PYG{o+ow}{in} \PYG{n}{models}\PYG{p}{]}
    \PYG{n}{coefs} \PYG{o}{=} \PYG{p}{(}
        \PYG{n}{pd}\PYG{o}{.}\PYG{n}{concat}\PYG{p}{(}
            \PYG{n}{objs}\PYG{o}{=}\PYG{p}{[}\PYG{n}{f}\PYG{o}{.}\PYG{n}{params} \PYG{k}{for} \PYG{n}{f} \PYG{o+ow}{in} \PYG{n}{fits}\PYG{p}{]}\PYG{p}{,}
            \PYG{n}{axis}\PYG{o}{=}\PYG{l+m+mi}{1}\PYG{p}{,}
            \PYG{n}{keys}\PYG{o}{=}\PYG{n}{df}\PYG{o}{.}\PYG{n}{columns}\PYG{p}{[}\PYG{p}{:}\PYG{l+m+mi}{10}\PYG{p}{]}
        \PYG{p}{)}
        \PYG{o}{.}\PYG{n}{transpose}\PYG{p}{(}\PYG{p}{)}
        \PYG{o}{.}\PYG{n}{rename\PYGZus{}axis}\PYG{p}{(}\PYG{n}{index}\PYG{o}{=}\PYG{l+s+s1}{\PYGZsq{}}\PYG{l+s+s1}{Portfolio}\PYG{l+s+s1}{\PYGZsq{}}\PYG{p}{,} \PYG{n}{columns}\PYG{o}{=}\PYG{l+s+s1}{\PYGZsq{}}\PYG{l+s+s1}{Variable}\PYG{l+s+s1}{\PYGZsq{}}\PYG{p}{)}
    \PYG{p}{)}

    \PYG{n}{ses} \PYG{o}{=} \PYG{p}{(}
        \PYG{n}{pd}\PYG{o}{.}\PYG{n}{concat}\PYG{p}{(}
            \PYG{n}{objs}\PYG{o}{=}\PYG{p}{[}\PYG{n}{f}\PYG{o}{.}\PYG{n}{bse} \PYG{k}{for} \PYG{n}{f} \PYG{o+ow}{in} \PYG{n}{fits}\PYG{p}{]}\PYG{p}{,}
            \PYG{n}{axis}\PYG{o}{=}\PYG{l+m+mi}{1}\PYG{p}{,}
            \PYG{n}{keys}\PYG{o}{=}\PYG{n}{df}\PYG{o}{.}\PYG{n}{columns}\PYG{p}{[}\PYG{p}{:}\PYG{l+m+mi}{10}\PYG{p}{]}
        \PYG{p}{)}
        \PYG{o}{.}\PYG{n}{transpose}\PYG{p}{(}\PYG{p}{)}
        \PYG{o}{.}\PYG{n}{rename\PYGZus{}axis}\PYG{p}{(}\PYG{n}{index}\PYG{o}{=}\PYG{l+s+s1}{\PYGZsq{}}\PYG{l+s+s1}{Portfolio}\PYG{l+s+s1}{\PYGZsq{}}\PYG{p}{,} \PYG{n}{columns}\PYG{o}{=}\PYG{l+s+s1}{\PYGZsq{}}\PYG{l+s+s1}{Variable}\PYG{l+s+s1}{\PYGZsq{}}\PYG{p}{)}
    \PYG{p}{)}

    \PYG{k}{return} \PYG{n}{pd}\PYG{o}{.}\PYG{n}{concat}\PYG{p}{(}
        \PYG{n}{objs}\PYG{o}{=}\PYG{p}{[}\PYG{n}{coefs}\PYG{p}{,} \PYG{n}{ses}\PYG{p}{]}\PYG{p}{,} 
        \PYG{n}{keys}\PYG{o}{=}\PYG{p}{[}\PYG{l+s+s1}{\PYGZsq{}}\PYG{l+s+s1}{Coef}\PYG{l+s+s1}{\PYGZsq{}}\PYG{p}{,} \PYG{l+s+s1}{\PYGZsq{}}\PYG{l+s+s1}{SE}\PYG{l+s+s1}{\PYGZsq{}}\PYG{p}{]}\PYG{p}{,} 
        \PYG{n}{names}\PYG{o}{=}\PYG{p}{[}\PYG{l+s+s1}{\PYGZsq{}}\PYG{l+s+s1}{Statistic}\PYG{l+s+s1}{\PYGZsq{}}\PYG{p}{]}\PYG{p}{,} 
        \PYG{n}{axis}\PYG{o}{=}\PYG{l+m+mi}{1}
    \PYG{p}{)}
\end{sphinxVerbatim}

\end{sphinxuseclass}\end{sphinxVerbatimInput}

\end{sphinxuseclass}
\sphinxAtStartPar
The \sphinxcode{\sphinxupquote{plot\_alpha()}} function accepts the output of the \sphinxcode{\sphinxupquote{get\_coefs()}} function and plots portfolio alphas.

\begin{sphinxuseclass}{cell}\begin{sphinxVerbatimInput}

\begin{sphinxuseclass}{cell_input}
\begin{sphinxVerbatim}[commandchars=\\\{\}]
\PYG{k}{def} \PYG{n+nf}{plot\PYGZus{}alpha}\PYG{p}{(}\PYG{n}{df}\PYG{p}{)}\PYG{p}{:}
    \PYG{n}{\PYGZus{}} \PYG{o}{=} \PYG{n}{df}\PYG{o}{.}\PYG{n}{swaplevel}\PYG{p}{(}\PYG{n}{axis}\PYG{o}{=}\PYG{l+m+mi}{1}\PYG{p}{)}\PYG{p}{[}\PYG{l+s+s1}{\PYGZsq{}}\PYG{l+s+s1}{Intercept}\PYG{l+s+s1}{\PYGZsq{}}\PYG{p}{]}
    \PYG{n}{\PYGZus{}}\PYG{p}{[}\PYG{l+s+s1}{\PYGZsq{}}\PYG{l+s+s1}{Coef}\PYG{l+s+s1}{\PYGZsq{}}\PYG{p}{]}\PYG{o}{.}\PYG{n}{plot}\PYG{p}{(}\PYG{n}{kind}\PYG{o}{=}\PYG{l+s+s1}{\PYGZsq{}}\PYG{l+s+s1}{bar}\PYG{l+s+s1}{\PYGZsq{}}\PYG{p}{,} \PYG{n}{yerr}\PYG{o}{=}\PYG{n}{\PYGZus{}}\PYG{p}{[}\PYG{l+s+s1}{\PYGZsq{}}\PYG{l+s+s1}{SE}\PYG{l+s+s1}{\PYGZsq{}}\PYG{p}{]}\PYG{p}{)}
    \PYG{n}{plt}\PYG{o}{.}\PYG{n}{ylabel}\PYG{p}{(}\PYG{l+s+s1}{\PYGZsq{}}\PYG{l+s+s1}{Monthly Alpha (}\PYG{l+s+s1}{\PYGZpc{}}\PYG{l+s+s1}{)}\PYG{l+s+s1}{\PYGZsq{}}\PYG{p}{)}
\end{sphinxVerbatim}

\end{sphinxuseclass}\end{sphinxVerbatimInput}

\end{sphinxuseclass}
\sphinxAtStartPar
We can look at the output of \sphinxcode{\sphinxupquote{get\_coefs()}}, but I typically will chain these functions.

\begin{sphinxuseclass}{cell}\begin{sphinxVerbatimInput}

\begin{sphinxuseclass}{cell_input}
\begin{sphinxVerbatim}[commandchars=\\\{\}]
\PYG{n}{mom\PYGZus{}ew}\PYG{o}{.}\PYG{n}{mul}\PYG{p}{(}\PYG{l+m+mi}{100}\PYG{p}{)}\PYG{o}{.}\PYG{n}{join}\PYG{p}{(}\PYG{n}{ff\PYGZus{}0}\PYG{p}{[}\PYG{l+m+mi}{0}\PYG{p}{]}\PYG{p}{)}\PYG{o}{.}\PYG{n}{pipe}\PYG{p}{(}\PYG{n}{get\PYGZus{}coefs}\PYG{p}{,} \PYG{n}{fun}\PYG{o}{=}\PYG{n}{capm}\PYG{p}{,} \PYG{n}{n}\PYG{o}{=}\PYG{l+m+mi}{10}\PYG{p}{)}
\end{sphinxVerbatim}

\end{sphinxuseclass}\end{sphinxVerbatimInput}
\begin{sphinxVerbatimOutput}

\begin{sphinxuseclass}{cell_output}
\begin{sphinxVerbatim}[commandchars=\\\{\}]
Statistic      Coef                    SE            
Variable  Intercept Q(\PYGZdq{}Mkt\PYGZhy{}RF\PYGZdq{}) Intercept Q(\PYGZdq{}Mkt\PYGZhy{}RF\PYGZdq{})
Portfolio                                            
Mom 1       \PYGZhy{}0.2642      1.6488    0.2364      0.0438
Mom 2       \PYGZhy{}0.2485      1.4625    0.1550      0.0287
Mom 3       \PYGZhy{}0.1499      1.3130    0.1170      0.0216
Mom 4        0.0747      1.2729    0.1071      0.0198
Mom 5        0.1002      1.1657    0.0874      0.0162
Mom 6        0.2728      1.1267    0.0786      0.0146
Mom 7        0.3420      1.0899    0.0778      0.0144
Mom 8        0.4860      1.0698    0.0794      0.0147
Mom 9        0.5828      1.0881    0.0882      0.0163
Mom 10       0.7466      1.1711    0.1273      0.0236
\end{sphinxVerbatim}

\end{sphinxuseclass}\end{sphinxVerbatimOutput}

\end{sphinxuseclass}
\sphinxAtStartPar
The beauty of chaining these functions is:
\begin{enumerate}
\sphinxsetlistlabels{\arabic}{enumi}{enumii}{}{.}%
\item {} 
\sphinxAtStartPar
We have all calculations in one cell, eliminating the need to track many “off screen” calculations in other cells

\item {} 
\sphinxAtStartPar
We do not create intermediate data frames, eliminating the need to name and track many data frames

\end{enumerate}

\begin{sphinxuseclass}{cell}\begin{sphinxVerbatimInput}

\begin{sphinxuseclass}{cell_input}
\begin{sphinxVerbatim}[commandchars=\\\{\}]
\PYG{n}{mom\PYGZus{}ew}\PYG{o}{.}\PYG{n}{mul}\PYG{p}{(}\PYG{l+m+mi}{100}\PYG{p}{)}\PYG{o}{.}\PYG{n}{join}\PYG{p}{(}\PYG{n}{ff\PYGZus{}0}\PYG{p}{[}\PYG{l+m+mi}{0}\PYG{p}{]}\PYG{p}{)}\PYG{o}{.}\PYG{n}{pipe}\PYG{p}{(}\PYG{n}{get\PYGZus{}coefs}\PYG{p}{,} \PYG{n}{fun}\PYG{o}{=}\PYG{n}{capm}\PYG{p}{,} \PYG{n}{n}\PYG{o}{=}\PYG{l+m+mi}{10}\PYG{p}{)}\PYG{o}{.}\PYG{n}{pipe}\PYG{p}{(}\PYG{n}{plot\PYGZus{}alpha}\PYG{p}{)}
\PYG{n}{plt}\PYG{o}{.}\PYG{n}{title}\PYG{p}{(}
    \PYG{l+s+s1}{\PYGZsq{}}\PYG{l+s+s1}{CAPM Tests of Momentum Strategy}\PYG{l+s+s1}{\PYGZsq{}} \PYG{o}{+}
    \PYG{l+s+s1}{\PYGZsq{}}\PYG{l+s+se}{\PYGZbs{}n}\PYG{l+s+s1}{Equal\PYGZhy{}Weighted Portfolios Formed on Months \PYGZhy{}12 to \PYGZhy{}2}\PYG{l+s+s1}{\PYGZsq{}} \PYG{o}{+}
    \PYG{l+s+s1}{\PYGZsq{}}\PYG{l+s+se}{\PYGZbs{}n}\PYG{l+s+s1}{Black Vertical Bars Indicate Standard Errors}\PYG{l+s+s1}{\PYGZsq{}}
\PYG{p}{)}
\PYG{n}{plt}\PYG{o}{.}\PYG{n}{show}\PYG{p}{(}\PYG{p}{)}
\end{sphinxVerbatim}

\end{sphinxuseclass}\end{sphinxVerbatimInput}
\begin{sphinxVerbatimOutput}

\begin{sphinxuseclass}{cell_output}
\noindent\sphinxincludegraphics{{5e1e913e1de9ed0d78aede753ac097b65b1979e747744da61ed4e7ca405d0adc}.png}

\end{sphinxuseclass}\end{sphinxVerbatimOutput}

\end{sphinxuseclass}
\sphinxAtStartPar
The momentum strategy has a large, positive, statistically significant alpha, suggesting it is a source of risk\sphinxhyphen{}adjusted returns (i.e., returns not associated with risk).
However, we saw above that momentum occasionally crashes, suggesting that the momentum strategy has risk.
We will investigate this with the Fama\sphinxhyphen{}French Four\sphinxhyphen{}Factor model, which add size, value, and momentum factors.
Once we consider SMB, HML, and Mom risk factors, we see:
\begin{enumerate}
\sphinxsetlistlabels{\arabic}{enumi}{enumii}{}{.}%
\item {} 
\sphinxAtStartPar
The relation between returns and momentum portfolios is much weaker and not monotonic

\item {} 
\sphinxAtStartPar
The statistical significance of the alpha coefficient estimates is much lower.

\end{enumerate}

\begin{sphinxuseclass}{cell}\begin{sphinxVerbatimInput}

\begin{sphinxuseclass}{cell_input}
\begin{sphinxVerbatim}[commandchars=\\\{\}]
\PYG{k}{def} \PYG{n+nf}{ff4}\PYG{p}{(}\PYG{n}{df}\PYG{p}{,} \PYG{n}{c}\PYG{p}{)}\PYG{p}{:}
    \PYG{k}{return} \PYG{n}{smf}\PYG{o}{.}\PYG{n}{ols}\PYG{p}{(}\PYG{n}{formula}\PYG{o}{=}\PYG{l+s+sa}{f}\PYG{l+s+s1}{\PYGZsq{}}\PYG{l+s+s1}{I(Q(}\PYG{l+s+s1}{\PYGZdq{}}\PYG{l+s+si}{\PYGZob{}}\PYG{n}{c}\PYG{l+s+si}{\PYGZcb{}}\PYG{l+s+s1}{\PYGZdq{}}\PYG{l+s+s1}{)\PYGZhy{}RF) \PYGZti{} Q(}\PYG{l+s+s1}{\PYGZdq{}}\PYG{l+s+s1}{Mkt\PYGZhy{}RF}\PYG{l+s+s1}{\PYGZdq{}}\PYG{l+s+s1}{) + SMB + HML + Mom}\PYG{l+s+s1}{\PYGZsq{}}\PYG{p}{,} \PYG{n}{data}\PYG{o}{=}\PYG{n}{df}\PYG{p}{)}
\end{sphinxVerbatim}

\end{sphinxuseclass}\end{sphinxVerbatimInput}

\end{sphinxuseclass}
\begin{sphinxuseclass}{cell}\begin{sphinxVerbatimInput}

\begin{sphinxuseclass}{cell_input}
\begin{sphinxVerbatim}[commandchars=\\\{\}]
\PYG{n}{mom\PYGZus{}ew}\PYG{o}{.}\PYG{n}{mul}\PYG{p}{(}\PYG{l+m+mi}{100}\PYG{p}{)}\PYG{o}{.}\PYG{n}{join}\PYG{p}{(}\PYG{p}{[}\PYG{n}{ff\PYGZus{}0}\PYG{p}{[}\PYG{l+m+mi}{0}\PYG{p}{]}\PYG{p}{,} \PYG{n}{ff\PYGZus{}mom}\PYG{p}{[}\PYG{l+m+mi}{0}\PYG{p}{]}\PYG{p}{]}\PYG{p}{)}\PYG{o}{.}\PYG{n}{pipe}\PYG{p}{(}\PYG{n}{get\PYGZus{}coefs}\PYG{p}{,} \PYG{n}{fun}\PYG{o}{=}\PYG{n}{ff4}\PYG{p}{,} \PYG{n}{n}\PYG{o}{=}\PYG{l+m+mi}{10}\PYG{p}{)}\PYG{o}{.}\PYG{n}{pipe}\PYG{p}{(}\PYG{n}{plot\PYGZus{}alpha}\PYG{p}{)}
\PYG{n}{plt}\PYG{o}{.}\PYG{n}{title}\PYG{p}{(}
    \PYG{l+s+s1}{\PYGZsq{}}\PYG{l+s+s1}{FF4 Tests of Momentum Strategy}\PYG{l+s+s1}{\PYGZsq{}} \PYG{o}{+}
    \PYG{l+s+s1}{\PYGZsq{}}\PYG{l+s+se}{\PYGZbs{}n}\PYG{l+s+s1}{Equal\PYGZhy{}Weighted Portfolios Formed on Months \PYGZhy{}12 to \PYGZhy{}2}\PYG{l+s+s1}{\PYGZsq{}} \PYG{o}{+}
    \PYG{l+s+s1}{\PYGZsq{}}\PYG{l+s+se}{\PYGZbs{}n}\PYG{l+s+s1}{Black Vertical Bars Indicate Standard Errors}\PYG{l+s+s1}{\PYGZsq{}}
\PYG{p}{)}
\PYG{n}{plt}\PYG{o}{.}\PYG{n}{show}\PYG{p}{(}\PYG{p}{)}
\end{sphinxVerbatim}

\end{sphinxuseclass}\end{sphinxVerbatimInput}
\begin{sphinxVerbatimOutput}

\begin{sphinxuseclass}{cell_output}
\noindent\sphinxincludegraphics{{49d58b0e8f4a15b5f64c08669380a620a5477c654dbd2bd39d8fe61718b6edb4}.png}

\end{sphinxuseclass}\end{sphinxVerbatimOutput}

\end{sphinxuseclass}
\sphinxAtStartPar
A more extreme test is to only consider the last 20 years of returns.

\begin{sphinxuseclass}{cell}\begin{sphinxVerbatimInput}

\begin{sphinxuseclass}{cell_input}
\begin{sphinxVerbatim}[commandchars=\\\{\}]
\PYG{n}{mom\PYGZus{}ew}\PYG{o}{.}\PYG{n}{mul}\PYG{p}{(}\PYG{l+m+mi}{100}\PYG{p}{)}\PYG{o}{.}\PYG{n}{join}\PYG{p}{(}\PYG{p}{[}\PYG{n}{ff\PYGZus{}0}\PYG{p}{[}\PYG{l+m+mi}{0}\PYG{p}{]}\PYG{p}{,} \PYG{n}{ff\PYGZus{}mom}\PYG{p}{[}\PYG{l+m+mi}{0}\PYG{p}{]}\PYG{p}{]}\PYG{p}{)}\PYG{o}{.}\PYG{n}{iloc}\PYG{p}{[}\PYG{o}{\PYGZhy{}}\PYG{l+m+mi}{240}\PYG{p}{:}\PYG{p}{]}\PYG{o}{.}\PYG{n}{pipe}\PYG{p}{(}\PYG{n}{get\PYGZus{}coefs}\PYG{p}{,} \PYG{n}{fun}\PYG{o}{=}\PYG{n}{ff4}\PYG{p}{,} \PYG{n}{n}\PYG{o}{=}\PYG{l+m+mi}{10}\PYG{p}{)}\PYG{o}{.}\PYG{n}{pipe}\PYG{p}{(}\PYG{n}{plot\PYGZus{}alpha}\PYG{p}{)}
\PYG{n}{plt}\PYG{o}{.}\PYG{n}{title}\PYG{p}{(}
    \PYG{l+s+s1}{\PYGZsq{}}\PYG{l+s+s1}{FF4 Tests of Momentum Strategy}\PYG{l+s+s1}{\PYGZsq{}} \PYG{o}{+}
    \PYG{l+s+s1}{\PYGZsq{}}\PYG{l+s+se}{\PYGZbs{}n}\PYG{l+s+s1}{Equal\PYGZhy{}Weighted Portfolios Formed on Months \PYGZhy{}12 to \PYGZhy{}2}\PYG{l+s+s1}{\PYGZsq{}} \PYG{o}{+}
    \PYG{l+s+s1}{\PYGZsq{}}\PYG{l+s+se}{\PYGZbs{}n}\PYG{l+s+s1}{Black Vertical Bars Indicate Standard Errors}\PYG{l+s+s1}{\PYGZsq{}}
\PYG{p}{)}
\PYG{n}{plt}\PYG{o}{.}\PYG{n}{show}\PYG{p}{(}\PYG{p}{)}
\end{sphinxVerbatim}

\end{sphinxuseclass}\end{sphinxVerbatimInput}
\begin{sphinxVerbatimOutput}

\begin{sphinxuseclass}{cell_output}
\noindent\sphinxincludegraphics{{61e94dbb15479ee32c7ad18ef05875f96952d508778d361dacb6f9ec0d32fdaa}.png}

\end{sphinxuseclass}\end{sphinxVerbatimOutput}

\end{sphinxuseclass}

\subsubsection{What are the Sharpe Ratios on these 11 portfolios?}
\label{\detokenize{herron_06_practice_03:what-are-the-sharpe-ratios-on-these-11-portfolios}}
\begin{sphinxuseclass}{cell}\begin{sphinxVerbatimInput}

\begin{sphinxuseclass}{cell_input}
\begin{sphinxVerbatim}[commandchars=\\\{\}]
\PYG{k}{def} \PYG{n+nf}{sharpe}\PYG{p}{(}\PYG{n}{r}\PYG{p}{,} \PYG{n}{tgt}\PYG{p}{,} \PYG{n}{ppy}\PYG{p}{)}\PYG{p}{:}
    \PYG{n}{er} \PYG{o}{=} \PYG{n}{r}\PYG{o}{.}\PYG{n}{sub}\PYG{p}{(}\PYG{n}{tgt}\PYG{p}{)}\PYG{o}{.}\PYG{n}{dropna}\PYG{p}{(}\PYG{p}{)}
    \PYG{k}{return} \PYG{n}{np}\PYG{o}{.}\PYG{n}{sqrt}\PYG{p}{(}\PYG{n}{ppy}\PYG{p}{)} \PYG{o}{*} \PYG{n}{er}\PYG{o}{.}\PYG{n}{mean}\PYG{p}{(}\PYG{p}{)} \PYG{o}{/} \PYG{n}{er}\PYG{o}{.}\PYG{n}{std}\PYG{p}{(}\PYG{p}{)}
\end{sphinxVerbatim}

\end{sphinxuseclass}\end{sphinxVerbatimInput}

\end{sphinxuseclass}
\begin{sphinxuseclass}{cell}\begin{sphinxVerbatimInput}

\begin{sphinxuseclass}{cell_input}
\begin{sphinxVerbatim}[commandchars=\\\{\}]
\PYG{n}{mom\PYGZus{}ew}\PYG{o}{.}\PYG{n}{mul}\PYG{p}{(}\PYG{l+m+mi}{100}\PYG{p}{)}\PYG{o}{.}\PYG{n}{apply}\PYG{p}{(}\PYG{n}{sharpe}\PYG{p}{,} \PYG{n}{tgt}\PYG{o}{=}\PYG{n}{ff\PYGZus{}0}\PYG{p}{[}\PYG{l+m+mi}{0}\PYG{p}{]}\PYG{p}{[}\PYG{l+s+s1}{\PYGZsq{}}\PYG{l+s+s1}{RF}\PYG{l+s+s1}{\PYGZsq{}}\PYG{p}{]}\PYG{p}{,} \PYG{n}{ppy}\PYG{o}{=}\PYG{l+m+mi}{12}\PYG{p}{)}\PYG{o}{.}\PYG{n}{plot}\PYG{p}{(}\PYG{n}{kind}\PYG{o}{=}\PYG{l+s+s1}{\PYGZsq{}}\PYG{l+s+s1}{bar}\PYG{l+s+s1}{\PYGZsq{}}\PYG{p}{)}
\PYG{n}{plt}\PYG{o}{.}\PYG{n}{ylabel}\PYG{p}{(}\PYG{l+s+s1}{\PYGZsq{}}\PYG{l+s+s1}{Sharpe Ratio}\PYG{l+s+s1}{\PYGZsq{}}\PYG{p}{)}
\PYG{n}{plt}\PYG{o}{.}\PYG{n}{title}\PYG{p}{(}
    \PYG{l+s+s1}{\PYGZsq{}}\PYG{l+s+s1}{Sharpe Ratios for Momentum Strategy Relative to RF}\PYG{l+s+s1}{\PYGZsq{}} \PYG{o}{+}
    \PYG{l+s+s1}{\PYGZsq{}}\PYG{l+s+se}{\PYGZbs{}n}\PYG{l+s+s1}{Equal\PYGZhy{}Weighted Portfolios Formed on Months \PYGZhy{}12 to \PYGZhy{}2}\PYG{l+s+s1}{\PYGZsq{}}
\PYG{p}{)}
\PYG{n}{plt}\PYG{o}{.}\PYG{n}{show}\PYG{p}{(}\PYG{p}{)}
\end{sphinxVerbatim}

\end{sphinxuseclass}\end{sphinxVerbatimInput}
\begin{sphinxVerbatimOutput}

\begin{sphinxuseclass}{cell_output}
\noindent\sphinxincludegraphics{{e3f863714a8690aab96231d2252fbc1266e1eb8df9a55c92216420269382a89e}.png}

\end{sphinxuseclass}\end{sphinxVerbatimOutput}

\end{sphinxuseclass}

\subsubsection{Implement a value\sphinxhyphen{}weighted momentum investing strategy}
\label{\detokenize{herron_06_practice_03:implement-a-value-weighted-momentum-investing-strategy}}
\sphinxAtStartPar
Assign this strategy to data frame \sphinxcode{\sphinxupquote{mom\_vw}}, and include long\sphinxhyphen{}short portfolio UMD

\sphinxAtStartPar
We can replace the \sphinxcode{\sphinxupquote{.mean()}} method, which calculates simple means (i.e., equal\sphinxhyphen{}weighted means), with the \sphinxcode{\sphinxupquote{.apply()}} method and \sphinxcode{\sphinxupquote{np.average()}} function, which has a \sphinxcode{\sphinxupquote{weights=}} argument and calculates weighted means.
We weight portfolio returns by the beginning\sphinxhyphen{}of\sphinxhyphen{}month marker value of equity in the \sphinxcode{\sphinxupquote{Trailing ME}} column.

\begin{sphinxuseclass}{cell}\begin{sphinxVerbatimInput}

\begin{sphinxuseclass}{cell_input}
\begin{sphinxVerbatim}[commandchars=\\\{\}]
\PYG{n}{mom\PYGZus{}vw} \PYG{o}{=} \PYG{p}{(}
    \PYG{n}{mom}
    \PYG{o}{.}\PYG{n}{groupby}\PYG{p}{(}\PYG{p}{[}\PYG{l+s+s1}{\PYGZsq{}}\PYG{l+s+s1}{date}\PYG{l+s+s1}{\PYGZsq{}}\PYG{p}{,} \PYG{l+s+s1}{\PYGZsq{}}\PYG{l+s+s1}{Portfolio}\PYG{l+s+s1}{\PYGZsq{}}\PYG{p}{]}\PYG{p}{)}
    \PYG{o}{.}\PYG{n}{apply}\PYG{p}{(}\PYG{k}{lambda} \PYG{n}{x}\PYG{p}{:} \PYG{n}{np}\PYG{o}{.}\PYG{n}{average}\PYG{p}{(}\PYG{n}{a}\PYG{o}{=}\PYG{n}{x}\PYG{p}{[}\PYG{l+s+s1}{\PYGZsq{}}\PYG{l+s+s1}{Return}\PYG{l+s+s1}{\PYGZsq{}}\PYG{p}{]}\PYG{p}{,} \PYG{n}{weights}\PYG{o}{=}\PYG{n}{x}\PYG{p}{[}\PYG{l+s+s1}{\PYGZsq{}}\PYG{l+s+s1}{Trailing ME}\PYG{l+s+s1}{\PYGZsq{}}\PYG{p}{]}\PYG{p}{)}\PYG{p}{)}
    \PYG{o}{.}\PYG{n}{unstack}\PYG{p}{(}\PYG{l+s+s1}{\PYGZsq{}}\PYG{l+s+s1}{Portfolio}\PYG{l+s+s1}{\PYGZsq{}}\PYG{p}{)}
    \PYG{o}{.}\PYG{n}{add\PYGZus{}prefix}\PYG{p}{(}\PYG{l+s+s1}{\PYGZsq{}}\PYG{l+s+s1}{Mom }\PYG{l+s+s1}{\PYGZsq{}}\PYG{p}{)}
\PYG{p}{)}
\end{sphinxVerbatim}

\end{sphinxuseclass}\end{sphinxVerbatimInput}

\end{sphinxuseclass}

\subsubsection{What are the CAPM and FF4 alphas for these value\sphinxhyphen{}weighted portfolios?}
\label{\detokenize{herron_06_practice_03:what-are-the-capm-and-ff4-alphas-for-these-value-weighted-portfolios}}
\sphinxAtStartPar
The alphas for the value\sphinxhyphen{}weighted momentum portfolios are generally smaller in magnitude than for the equal\sphinxhyphen{}weighted momentum portfolios.
Equal\sphinxhyphen{}weighted portfolios over\sphinxhyphen{}weight small stocks relative to value\sphinxhyphen{}weighted portfolios, and we expect most anomalies to be stronger in small stocks because small stocks are more difficult for institutional investors to  invest in at scale.

\begin{sphinxuseclass}{cell}\begin{sphinxVerbatimInput}

\begin{sphinxuseclass}{cell_input}
\begin{sphinxVerbatim}[commandchars=\\\{\}]
\PYG{n}{mom\PYGZus{}vw}\PYG{o}{.}\PYG{n}{mul}\PYG{p}{(}\PYG{l+m+mi}{100}\PYG{p}{)}\PYG{o}{.}\PYG{n}{join}\PYG{p}{(}\PYG{n}{ff\PYGZus{}0}\PYG{p}{[}\PYG{l+m+mi}{0}\PYG{p}{]}\PYG{p}{)}\PYG{o}{.}\PYG{n}{pipe}\PYG{p}{(}\PYG{n}{get\PYGZus{}coefs}\PYG{p}{,} \PYG{n}{fun}\PYG{o}{=}\PYG{n}{capm}\PYG{p}{,} \PYG{n}{n}\PYG{o}{=}\PYG{l+m+mi}{10}\PYG{p}{)}\PYG{o}{.}\PYG{n}{pipe}\PYG{p}{(}\PYG{n}{plot\PYGZus{}alpha}\PYG{p}{)}
\PYG{n}{plt}\PYG{o}{.}\PYG{n}{title}\PYG{p}{(}
    \PYG{l+s+s1}{\PYGZsq{}}\PYG{l+s+s1}{CAPM Tests of Momentum Strategy}\PYG{l+s+s1}{\PYGZsq{}} \PYG{o}{+}
    \PYG{l+s+s1}{\PYGZsq{}}\PYG{l+s+se}{\PYGZbs{}n}\PYG{l+s+s1}{Value\PYGZhy{}Weighted Portfolios Formed on Months \PYGZhy{}12 to \PYGZhy{}2}\PYG{l+s+s1}{\PYGZsq{}} \PYG{o}{+}
    \PYG{l+s+s1}{\PYGZsq{}}\PYG{l+s+se}{\PYGZbs{}n}\PYG{l+s+s1}{Black Vertical Bars Indicate Standard Errors}\PYG{l+s+s1}{\PYGZsq{}}
\PYG{p}{)}
\PYG{n}{plt}\PYG{o}{.}\PYG{n}{show}\PYG{p}{(}\PYG{p}{)}
\end{sphinxVerbatim}

\end{sphinxuseclass}\end{sphinxVerbatimInput}
\begin{sphinxVerbatimOutput}

\begin{sphinxuseclass}{cell_output}
\noindent\sphinxincludegraphics{{ced34c2a1d5da1a44cb646030a067d3e0a7d3f3faa565e7aa4f78e1952078890}.png}

\end{sphinxuseclass}\end{sphinxVerbatimOutput}

\end{sphinxuseclass}
\sphinxAtStartPar
The momentum strategy falls apart once we consider the Fama\sphinxhyphen{}French Four\sphinxhyphen{}Factor model with value\sphinxhyphen{}weighted portfolios.

\begin{sphinxuseclass}{cell}\begin{sphinxVerbatimInput}

\begin{sphinxuseclass}{cell_input}
\begin{sphinxVerbatim}[commandchars=\\\{\}]
\PYG{n}{mom\PYGZus{}vw}\PYG{o}{.}\PYG{n}{mul}\PYG{p}{(}\PYG{l+m+mi}{100}\PYG{p}{)}\PYG{o}{.}\PYG{n}{join}\PYG{p}{(}\PYG{p}{[}\PYG{n}{ff\PYGZus{}0}\PYG{p}{[}\PYG{l+m+mi}{0}\PYG{p}{]}\PYG{p}{,} \PYG{n}{ff\PYGZus{}mom}\PYG{p}{[}\PYG{l+m+mi}{0}\PYG{p}{]}\PYG{p}{]}\PYG{p}{)}\PYG{o}{.}\PYG{n}{pipe}\PYG{p}{(}\PYG{n}{get\PYGZus{}coefs}\PYG{p}{,} \PYG{n}{fun}\PYG{o}{=}\PYG{n}{ff4}\PYG{p}{,} \PYG{n}{n}\PYG{o}{=}\PYG{l+m+mi}{10}\PYG{p}{)}\PYG{o}{.}\PYG{n}{pipe}\PYG{p}{(}\PYG{n}{plot\PYGZus{}alpha}\PYG{p}{)}
\PYG{n}{plt}\PYG{o}{.}\PYG{n}{title}\PYG{p}{(}
    \PYG{l+s+s1}{\PYGZsq{}}\PYG{l+s+s1}{FF4 Tests of Momentum Strategy}\PYG{l+s+s1}{\PYGZsq{}} \PYG{o}{+}
    \PYG{l+s+s1}{\PYGZsq{}}\PYG{l+s+se}{\PYGZbs{}n}\PYG{l+s+s1}{Value\PYGZhy{}Weighted Portfolios Formed on Months \PYGZhy{}12 to \PYGZhy{}2}\PYG{l+s+s1}{\PYGZsq{}} \PYG{o}{+}
    \PYG{l+s+s1}{\PYGZsq{}}\PYG{l+s+se}{\PYGZbs{}n}\PYG{l+s+s1}{Black Vertical Bars Indicate Standard Errors}\PYG{l+s+s1}{\PYGZsq{}}
\PYG{p}{)}
\PYG{n}{plt}\PYG{o}{.}\PYG{n}{show}\PYG{p}{(}\PYG{p}{)}
\end{sphinxVerbatim}

\end{sphinxuseclass}\end{sphinxVerbatimInput}
\begin{sphinxVerbatimOutput}

\begin{sphinxuseclass}{cell_output}
\noindent\sphinxincludegraphics{{66e40b5dd92b6085cded9eeb7d2c959f470ca8560f6630174d8dc1235085e3d4}.png}

\end{sphinxuseclass}\end{sphinxVerbatimOutput}

\end{sphinxuseclass}

\subsubsection{What are the Sharpe Ratios for these value\sphinxhyphen{}weighted portfolios?}
\label{\detokenize{herron_06_practice_03:what-are-the-sharpe-ratios-for-these-value-weighted-portfolios}}
\begin{sphinxuseclass}{cell}\begin{sphinxVerbatimInput}

\begin{sphinxuseclass}{cell_input}
\begin{sphinxVerbatim}[commandchars=\\\{\}]
\PYG{n}{mom\PYGZus{}vw}\PYG{o}{.}\PYG{n}{mul}\PYG{p}{(}\PYG{l+m+mi}{100}\PYG{p}{)}\PYG{o}{.}\PYG{n}{apply}\PYG{p}{(}\PYG{n}{sharpe}\PYG{p}{,} \PYG{n}{tgt}\PYG{o}{=}\PYG{n}{ff\PYGZus{}0}\PYG{p}{[}\PYG{l+m+mi}{0}\PYG{p}{]}\PYG{p}{[}\PYG{l+s+s1}{\PYGZsq{}}\PYG{l+s+s1}{RF}\PYG{l+s+s1}{\PYGZsq{}}\PYG{p}{]}\PYG{p}{,} \PYG{n}{ppy}\PYG{o}{=}\PYG{l+m+mi}{12}\PYG{p}{)}\PYG{o}{.}\PYG{n}{plot}\PYG{p}{(}\PYG{n}{kind}\PYG{o}{=}\PYG{l+s+s1}{\PYGZsq{}}\PYG{l+s+s1}{bar}\PYG{l+s+s1}{\PYGZsq{}}\PYG{p}{)}
\PYG{n}{plt}\PYG{o}{.}\PYG{n}{ylabel}\PYG{p}{(}\PYG{l+s+s1}{\PYGZsq{}}\PYG{l+s+s1}{Sharpe Ratio}\PYG{l+s+s1}{\PYGZsq{}}\PYG{p}{)}
\PYG{n}{plt}\PYG{o}{.}\PYG{n}{title}\PYG{p}{(}
    \PYG{l+s+s1}{\PYGZsq{}}\PYG{l+s+s1}{Sharpe Ratios for Momentum Strategy Relative to RF}\PYG{l+s+s1}{\PYGZsq{}} \PYG{o}{+}
    \PYG{l+s+s1}{\PYGZsq{}}\PYG{l+s+se}{\PYGZbs{}n}\PYG{l+s+s1}{Value\PYGZhy{}Weighted Portfolios Formed on Months \PYGZhy{}12 to \PYGZhy{}2}\PYG{l+s+s1}{\PYGZsq{}}
\PYG{p}{)}
\PYG{n}{plt}\PYG{o}{.}\PYG{n}{show}\PYG{p}{(}\PYG{p}{)}
\end{sphinxVerbatim}

\end{sphinxuseclass}\end{sphinxVerbatimInput}
\begin{sphinxVerbatimOutput}

\begin{sphinxuseclass}{cell_output}
\noindent\sphinxincludegraphics{{f7a171cb8d4193690488883353b1ad5d1c1208f33f2581c2d2f6b793b9526043}.png}

\end{sphinxuseclass}\end{sphinxVerbatimOutput}

\end{sphinxuseclass}

\subsubsection{Implement an equal\sphinxhyphen{}weighted size investing strategy based on market capitalization \textasciitilde{}\textasciitilde{}at the start of each month\textasciitilde{}\textasciitilde{} from the previous June}
\label{\detokenize{herron_06_practice_03:implement-an-equal-weighted-size-investing-strategy-based-on-market-capitalization-at-the-start-of-each-month-from-the-previous-june}}
\sphinxAtStartPar
\sphinxstyleemphasis{\sphinxstylestrong{To show how to form portfolios at lower frequencies, I change this task to form portfolios once a year based on market capitalization from the previous June.}}
We will assign stocks to portfolios based on market capitalization (i.e., size or the market value of equity) each June and use these portfolio assignments from July through the following June.
That is, we rank in June of year \$t\$, then hold them in portfolios from July in year \$t\$ through June in year \$t+1\$.

\begin{sphinxuseclass}{cell}\begin{sphinxVerbatimInput}

\begin{sphinxuseclass}{cell_input}
\begin{sphinxVerbatim}[commandchars=\\\{\}]
\PYG{n}{me\PYGZus{}june} \PYG{o}{=} \PYG{n}{crsp}\PYG{p}{[}\PYG{l+s+s1}{\PYGZsq{}}\PYG{l+s+s1}{ME}\PYG{l+s+s1}{\PYGZsq{}}\PYG{p}{]}\PYG{o}{.}\PYG{n}{unstack}\PYG{p}{(}\PYG{l+s+s1}{\PYGZsq{}}\PYG{l+s+s1}{PERMNO}\PYG{l+s+s1}{\PYGZsq{}}\PYG{p}{)}\PYG{o}{.}\PYG{n}{pipe}\PYG{p}{(}\PYG{k}{lambda} \PYG{n}{x}\PYG{p}{:} \PYG{n}{x}\PYG{o}{.}\PYG{n}{loc}\PYG{p}{[}\PYG{n}{x}\PYG{o}{.}\PYG{n}{index}\PYG{o}{.}\PYG{n}{month} \PYG{o}{==} \PYG{l+m+mi}{6}\PYG{p}{]}\PYG{p}{)}
\end{sphinxVerbatim}

\end{sphinxuseclass}\end{sphinxVerbatimInput}

\end{sphinxuseclass}
\begin{sphinxuseclass}{cell}\begin{sphinxVerbatimInput}

\begin{sphinxuseclass}{cell_input}
\begin{sphinxVerbatim}[commandchars=\\\{\}]
\PYG{n}{me\PYGZus{}june}\PYG{o}{.}\PYG{n}{head}\PYG{p}{(}\PYG{p}{)}
\end{sphinxVerbatim}

\end{sphinxuseclass}\end{sphinxVerbatimInput}
\begin{sphinxVerbatimOutput}

\begin{sphinxuseclass}{cell_output}
\begin{sphinxVerbatim}[commandchars=\\\{\}]
PERMNO   10000  10001  10002  10003  10005      10006  10007  10008  10009  \PYGZbs{}
date                                                                         
1926\PYGZhy{}06    NaN    NaN    NaN    NaN    NaN 59400.0000    NaN    NaN    NaN   
1927\PYGZhy{}06    NaN    NaN    NaN    NaN    NaN 59400.0000    NaN    NaN    NaN   
1928\PYGZhy{}06    NaN    NaN    NaN    NaN    NaN 57675.0000    NaN    NaN    NaN   
1929\PYGZhy{}06    NaN    NaN    NaN    NaN    NaN 57900.0000    NaN    NaN    NaN   
1930\PYGZhy{}06    NaN    NaN    NaN    NaN    NaN 30900.0000    NaN    NaN    NaN   

PERMNO   10010  ...  93423  93426  93428  93429  93430  93432  93433  93434  \PYGZbs{}
date            ...                                                           
1926\PYGZhy{}06    NaN  ...    NaN    NaN    NaN    NaN    NaN    NaN    NaN    NaN   
1927\PYGZhy{}06    NaN  ...    NaN    NaN    NaN    NaN    NaN    NaN    NaN    NaN   
1928\PYGZhy{}06    NaN  ...    NaN    NaN    NaN    NaN    NaN    NaN    NaN    NaN   
1929\PYGZhy{}06    NaN  ...    NaN    NaN    NaN    NaN    NaN    NaN    NaN    NaN   
1930\PYGZhy{}06    NaN  ...    NaN    NaN    NaN    NaN    NaN    NaN    NaN    NaN   

PERMNO   93435  93436  
date                   
1926\PYGZhy{}06    NaN    NaN  
1927\PYGZhy{}06    NaN    NaN  
1928\PYGZhy{}06    NaN    NaN  
1929\PYGZhy{}06    NaN    NaN  
1930\PYGZhy{}06    NaN    NaN  

[5 rows x 26480 columns]
\end{sphinxVerbatim}

\end{sphinxuseclass}\end{sphinxVerbatimOutput}

\end{sphinxuseclass}
\sphinxAtStartPar
Next we assign portfolios on these June market capitalizations.

\begin{sphinxuseclass}{cell}\begin{sphinxVerbatimInput}

\begin{sphinxuseclass}{cell_input}
\begin{sphinxVerbatim}[commandchars=\\\{\}]
\PYG{n}{me\PYGZus{}june}\PYG{o}{.}\PYG{n}{apply}\PYG{p}{(}\PYG{n}{pd}\PYG{o}{.}\PYG{n}{qcut}\PYG{p}{,} \PYG{n}{q}\PYG{o}{=}\PYG{l+m+mi}{10}\PYG{p}{,} \PYG{n}{labels}\PYG{o}{=}\PYG{k+kc}{False}\PYG{p}{,} \PYG{n}{axis}\PYG{o}{=}\PYG{l+m+mi}{1}\PYG{p}{)}\PYG{o}{.}\PYG{n}{head}\PYG{p}{(}\PYG{p}{)}
\end{sphinxVerbatim}

\end{sphinxuseclass}\end{sphinxVerbatimInput}
\begin{sphinxVerbatimOutput}

\begin{sphinxuseclass}{cell_output}
\begin{sphinxVerbatim}[commandchars=\\\{\}]
PERMNO   10000  10001  10002  10003  10005  10006  10007  10008  10009  10010  \PYGZbs{}
date                                                                            
1926\PYGZhy{}06    NaN    NaN    NaN    NaN    NaN 7.0000    NaN    NaN    NaN    NaN   
1927\PYGZhy{}06    NaN    NaN    NaN    NaN    NaN 7.0000    NaN    NaN    NaN    NaN   
1928\PYGZhy{}06    NaN    NaN    NaN    NaN    NaN 7.0000    NaN    NaN    NaN    NaN   
1929\PYGZhy{}06    NaN    NaN    NaN    NaN    NaN 6.0000    NaN    NaN    NaN    NaN   
1930\PYGZhy{}06    NaN    NaN    NaN    NaN    NaN 6.0000    NaN    NaN    NaN    NaN   

PERMNO   ...  93423  93426  93428  93429  93430  93432  93433  93434  93435  \PYGZbs{}
date     ...                                                                  
1926\PYGZhy{}06  ...    NaN    NaN    NaN    NaN    NaN    NaN    NaN    NaN    NaN   
1927\PYGZhy{}06  ...    NaN    NaN    NaN    NaN    NaN    NaN    NaN    NaN    NaN   
1928\PYGZhy{}06  ...    NaN    NaN    NaN    NaN    NaN    NaN    NaN    NaN    NaN   
1929\PYGZhy{}06  ...    NaN    NaN    NaN    NaN    NaN    NaN    NaN    NaN    NaN   
1930\PYGZhy{}06  ...    NaN    NaN    NaN    NaN    NaN    NaN    NaN    NaN    NaN   

PERMNO   93436  
date            
1926\PYGZhy{}06    NaN  
1927\PYGZhy{}06    NaN  
1928\PYGZhy{}06    NaN  
1929\PYGZhy{}06    NaN  
1930\PYGZhy{}06    NaN  

[5 rows x 26480 columns]
\end{sphinxVerbatim}

\end{sphinxuseclass}\end{sphinxVerbatimOutput}

\end{sphinxuseclass}
\sphinxAtStartPar
Note, we want to use these portfolio assignments from July through June, so we need to \sphinxcode{\sphinxupquote{.shift()}} with \sphinxcode{\sphinxupquote{freq='1M'}}, which shifts the index one month instead of shifting the values one row.
Because we only have June data, a one\sphinxhyphen{}row shift is a one\sphinxhyphen{}year shift instead of a one\sphinxhyphen{}month shift.
We also need to up sample our June data to monthly date with \sphinxcode{\sphinxupquote{.resample('M')}} and the \sphinxcode{\sphinxupquote{.ffill()}} method.
Because we only want to fill forward 12 months, we use \sphinxcode{\sphinxupquote{.ffill(limit=12)}}.

\begin{sphinxuseclass}{cell}\begin{sphinxVerbatimInput}

\begin{sphinxuseclass}{cell_input}
\begin{sphinxVerbatim}[commandchars=\\\{\}]
\PYG{n}{me\PYGZus{}june}\PYG{o}{.}\PYG{n}{apply}\PYG{p}{(}\PYG{n}{pd}\PYG{o}{.}\PYG{n}{qcut}\PYG{p}{,} \PYG{n}{q}\PYG{o}{=}\PYG{l+m+mi}{10}\PYG{p}{,} \PYG{n}{labels}\PYG{o}{=}\PYG{k+kc}{False}\PYG{p}{,} \PYG{n}{axis}\PYG{o}{=}\PYG{l+m+mi}{1}\PYG{p}{)}\PYG{o}{.}\PYG{n}{shift}\PYG{p}{(}\PYG{n}{freq}\PYG{o}{=}\PYG{l+s+s1}{\PYGZsq{}}\PYG{l+s+s1}{1M}\PYG{l+s+s1}{\PYGZsq{}}\PYG{p}{)}\PYG{o}{.}\PYG{n}{resample}\PYG{p}{(}\PYG{l+s+s1}{\PYGZsq{}}\PYG{l+s+s1}{M}\PYG{l+s+s1}{\PYGZsq{}}\PYG{p}{)}\PYG{o}{.}\PYG{n}{ffill}\PYG{p}{(}\PYG{n}{limit}\PYG{o}{=}\PYG{l+m+mi}{12}\PYG{p}{)}\PYG{o}{.}\PYG{n}{tail}\PYG{p}{(}\PYG{p}{)}
\end{sphinxVerbatim}

\end{sphinxuseclass}\end{sphinxVerbatimInput}
\begin{sphinxVerbatimOutput}

\begin{sphinxuseclass}{cell_output}
\begin{sphinxVerbatim}[commandchars=\\\{\}]
PERMNO   10000  10001  10002  10003  10005  10006  10007  10008  10009  10010  \PYGZbs{}
date                                                                            
2022\PYGZhy{}03    NaN    NaN    NaN    NaN    NaN    NaN    NaN    NaN    NaN    NaN   
2022\PYGZhy{}04    NaN    NaN    NaN    NaN    NaN    NaN    NaN    NaN    NaN    NaN   
2022\PYGZhy{}05    NaN    NaN    NaN    NaN    NaN    NaN    NaN    NaN    NaN    NaN   
2022\PYGZhy{}06    NaN    NaN    NaN    NaN    NaN    NaN    NaN    NaN    NaN    NaN   
2022\PYGZhy{}07    NaN    NaN    NaN    NaN    NaN    NaN    NaN    NaN    NaN    NaN   

PERMNO   ...  93423  93426  93428  93429  93430  93432  93433  93434  93435  \PYGZbs{}
date     ...                                                                  
2022\PYGZhy{}03  ... 7.0000 3.0000    NaN 8.0000    NaN    NaN    NaN 1.0000    NaN   
2022\PYGZhy{}04  ... 7.0000 3.0000    NaN 8.0000    NaN    NaN    NaN 1.0000    NaN   
2022\PYGZhy{}05  ... 7.0000 3.0000    NaN 8.0000    NaN    NaN    NaN 1.0000    NaN   
2022\PYGZhy{}06  ... 7.0000 3.0000    NaN 8.0000    NaN    NaN    NaN 1.0000    NaN   
2022\PYGZhy{}07  ... 6.0000 4.0000    NaN 8.0000    NaN    NaN    NaN 1.0000    NaN   

PERMNO   93436  
date            
2022\PYGZhy{}03 9.0000  
2022\PYGZhy{}04 9.0000  
2022\PYGZhy{}05 9.0000  
2022\PYGZhy{}06 9.0000  
2022\PYGZhy{}07 9.0000  

[5 rows x 26480 columns]
\end{sphinxVerbatim}

\end{sphinxuseclass}\end{sphinxVerbatimOutput}

\end{sphinxuseclass}
\sphinxAtStartPar
We can combine the \sphinxcode{\sphinxupquote{.apply()}}, \sphinxcode{\sphinxupquote{.shift()}}, \sphinxcode{\sphinxupquote{.resample()}}, and \sphinxcode{\sphinxupquote{.fill()}} methods into one operation when we combine the returns, trailing market capitalization, and portfolio assignments.

\begin{sphinxuseclass}{cell}\begin{sphinxVerbatimInput}

\begin{sphinxuseclass}{cell_input}
\begin{sphinxVerbatim}[commandchars=\\\{\}]
\PYG{n}{me} \PYG{o}{=} \PYG{p}{(}
    \PYG{n}{pd}\PYG{o}{.}\PYG{n}{concat}\PYG{p}{(}
        \PYG{n}{objs}\PYG{o}{=}\PYG{p}{[}
            \PYG{n}{ret\PYGZus{}1m}\PYG{p}{,}
            \PYG{n}{crsp}\PYG{p}{[}\PYG{l+s+s1}{\PYGZsq{}}\PYG{l+s+s1}{ME}\PYG{l+s+s1}{\PYGZsq{}}\PYG{p}{]}\PYG{o}{.}\PYG{n}{unstack}\PYG{p}{(}\PYG{l+s+s1}{\PYGZsq{}}\PYG{l+s+s1}{PERMNO}\PYG{l+s+s1}{\PYGZsq{}}\PYG{p}{)}\PYG{o}{.}\PYG{n}{shift}\PYG{p}{(}\PYG{l+m+mi}{1}\PYG{p}{)}\PYG{p}{,}
            \PYG{p}{(}
                \PYG{n}{me\PYGZus{}june}
                \PYG{o}{.}\PYG{n}{apply}\PYG{p}{(}\PYG{n}{pd}\PYG{o}{.}\PYG{n}{qcut}\PYG{p}{,} \PYG{n}{q}\PYG{o}{=}\PYG{l+m+mi}{10}\PYG{p}{,} \PYG{n}{labels}\PYG{o}{=}\PYG{k+kc}{False}\PYG{p}{,} \PYG{n}{axis}\PYG{o}{=}\PYG{l+m+mi}{1}\PYG{p}{)}
                \PYG{o}{.}\PYG{n}{shift}\PYG{p}{(}\PYG{n}{freq}\PYG{o}{=}\PYG{l+s+s1}{\PYGZsq{}}\PYG{l+s+s1}{1M}\PYG{l+s+s1}{\PYGZsq{}}\PYG{p}{)}
                \PYG{o}{.}\PYG{n}{resample}\PYG{p}{(}\PYG{l+s+s1}{\PYGZsq{}}\PYG{l+s+s1}{M}\PYG{l+s+s1}{\PYGZsq{}}\PYG{p}{)}
                \PYG{o}{.}\PYG{n}{ffill}\PYG{p}{(}\PYG{n}{limit}\PYG{o}{=}\PYG{l+m+mi}{12}\PYG{p}{)}
            \PYG{p}{)}
        \PYG{p}{]}\PYG{p}{,}
        \PYG{n}{keys}\PYG{o}{=}\PYG{p}{[}\PYG{l+s+s1}{\PYGZsq{}}\PYG{l+s+s1}{Return}\PYG{l+s+s1}{\PYGZsq{}}\PYG{p}{,} \PYG{l+s+s1}{\PYGZsq{}}\PYG{l+s+s1}{Trailing ME}\PYG{l+s+s1}{\PYGZsq{}}\PYG{p}{,} \PYG{l+s+s1}{\PYGZsq{}}\PYG{l+s+s1}{Portfolio}\PYG{l+s+s1}{\PYGZsq{}}\PYG{p}{]}\PYG{p}{,}
        \PYG{n}{names}\PYG{o}{=}\PYG{p}{[}\PYG{l+s+s1}{\PYGZsq{}}\PYG{l+s+s1}{Variable}\PYG{l+s+s1}{\PYGZsq{}}\PYG{p}{]}\PYG{p}{,}
        \PYG{n}{axis}\PYG{o}{=}\PYG{l+m+mi}{1}
    \PYG{p}{)}
    \PYG{o}{.}\PYG{n}{stack}\PYG{p}{(}\PYG{l+s+s1}{\PYGZsq{}}\PYG{l+s+s1}{PERMNO}\PYG{l+s+s1}{\PYGZsq{}}\PYG{p}{)}
    \PYG{o}{.}\PYG{n}{dropna}\PYG{p}{(}\PYG{p}{)}
    \PYG{o}{.}\PYG{n}{assign}\PYG{p}{(}\PYG{n}{Portfolio}\PYG{o}{=}\PYG{k}{lambda} \PYG{n}{x}\PYG{p}{:} \PYG{n}{x}\PYG{p}{[}\PYG{l+s+s1}{\PYGZsq{}}\PYG{l+s+s1}{Portfolio}\PYG{l+s+s1}{\PYGZsq{}}\PYG{p}{]}\PYG{o}{.}\PYG{n}{astype}\PYG{p}{(}\PYG{n+nb}{int}\PYG{p}{)} \PYG{o}{+} \PYG{l+m+mi}{1}\PYG{p}{)}
\PYG{p}{)}
\end{sphinxVerbatim}

\end{sphinxuseclass}\end{sphinxVerbatimInput}

\end{sphinxuseclass}
\begin{sphinxuseclass}{cell}\begin{sphinxVerbatimInput}

\begin{sphinxuseclass}{cell_input}
\begin{sphinxVerbatim}[commandchars=\\\{\}]
\PYG{n}{me}\PYG{o}{.}\PYG{n}{head}\PYG{p}{(}\PYG{p}{)}
\end{sphinxVerbatim}

\end{sphinxuseclass}\end{sphinxVerbatimInput}
\begin{sphinxVerbatimOutput}

\begin{sphinxuseclass}{cell_output}
\begin{sphinxVerbatim}[commandchars=\\\{\}]
Variable        Return  Trailing ME  Portfolio
date    PERMNO                                
1926\PYGZhy{}07 10022   0.2400   10000.0000          5
        10030   0.0211   19402.5000          6
        10057   0.0078    4031.2500          2
        10073   0.0729    1656.0000          1
        10081   0.0156    9536.0000          4
\end{sphinxVerbatim}

\end{sphinxuseclass}\end{sphinxVerbatimOutput}

\end{sphinxuseclass}
\begin{sphinxuseclass}{cell}\begin{sphinxVerbatimInput}

\begin{sphinxuseclass}{cell_input}
\begin{sphinxVerbatim}[commandchars=\\\{\}]
\PYG{n}{me\PYGZus{}ew} \PYG{o}{=} \PYG{p}{(}
    \PYG{n}{me}
    \PYG{o}{.}\PYG{n}{groupby}\PYG{p}{(}\PYG{p}{[}\PYG{l+s+s1}{\PYGZsq{}}\PYG{l+s+s1}{date}\PYG{l+s+s1}{\PYGZsq{}}\PYG{p}{,} \PYG{l+s+s1}{\PYGZsq{}}\PYG{l+s+s1}{Portfolio}\PYG{l+s+s1}{\PYGZsq{}}\PYG{p}{]}\PYG{p}{)}
    \PYG{p}{[}\PYG{l+s+s1}{\PYGZsq{}}\PYG{l+s+s1}{Return}\PYG{l+s+s1}{\PYGZsq{}}\PYG{p}{]}
    \PYG{o}{.}\PYG{n}{mean}\PYG{p}{(}\PYG{p}{)}
    \PYG{o}{.}\PYG{n}{unstack}\PYG{p}{(}\PYG{l+s+s1}{\PYGZsq{}}\PYG{l+s+s1}{Portfolio}\PYG{l+s+s1}{\PYGZsq{}}\PYG{p}{)}
    \PYG{o}{.}\PYG{n}{add\PYGZus{}prefix}\PYG{p}{(}\PYG{l+s+s1}{\PYGZsq{}}\PYG{l+s+s1}{ME }\PYG{l+s+s1}{\PYGZsq{}}\PYG{p}{)}
\PYG{p}{)}
\end{sphinxVerbatim}

\end{sphinxuseclass}\end{sphinxVerbatimInput}

\end{sphinxuseclass}
\begin{sphinxuseclass}{cell}\begin{sphinxVerbatimInput}

\begin{sphinxuseclass}{cell_input}
\begin{sphinxVerbatim}[commandchars=\\\{\}]
\PYG{n}{me\PYGZus{}ew}\PYG{o}{.}\PYG{n}{mul}\PYG{p}{(}\PYG{l+m+mi}{100}\PYG{p}{)}\PYG{o}{.}\PYG{n}{join}\PYG{p}{(}\PYG{p}{[}\PYG{n}{ff\PYGZus{}0}\PYG{p}{[}\PYG{l+m+mi}{0}\PYG{p}{]}\PYG{p}{,} \PYG{n}{ff\PYGZus{}mom}\PYG{p}{[}\PYG{l+m+mi}{0}\PYG{p}{]}\PYG{p}{]}\PYG{p}{)}\PYG{o}{.}\PYG{n}{pipe}\PYG{p}{(}\PYG{n}{get\PYGZus{}coefs}\PYG{p}{,} \PYG{n}{fun}\PYG{o}{=}\PYG{n}{capm}\PYG{p}{,} \PYG{n}{n}\PYG{o}{=}\PYG{l+m+mi}{10}\PYG{p}{)}\PYG{o}{.}\PYG{n}{pipe}\PYG{p}{(}\PYG{n}{plot\PYGZus{}alpha}\PYG{p}{)}
\PYG{n}{plt}\PYG{o}{.}\PYG{n}{title}\PYG{p}{(}
    \PYG{l+s+s1}{\PYGZsq{}}\PYG{l+s+s1}{CAPM Tests of Size Strategy}\PYG{l+s+s1}{\PYGZsq{}} \PYG{o}{+}
    \PYG{l+s+s1}{\PYGZsq{}}\PYG{l+s+se}{\PYGZbs{}n}\PYG{l+s+s1}{Equal\PYGZhy{}Weighted Portfolios Formed June Market Value of Equity}\PYG{l+s+s1}{\PYGZsq{}} \PYG{o}{+}
    \PYG{l+s+s1}{\PYGZsq{}}\PYG{l+s+se}{\PYGZbs{}n}\PYG{l+s+s1}{Black Vertical Bars Indicate Standard Errors}\PYG{l+s+s1}{\PYGZsq{}}
\PYG{p}{)}
\PYG{n}{plt}\PYG{o}{.}\PYG{n}{show}\PYG{p}{(}\PYG{p}{)}
\end{sphinxVerbatim}

\end{sphinxuseclass}\end{sphinxVerbatimInput}
\begin{sphinxVerbatimOutput}

\begin{sphinxuseclass}{cell_output}
\noindent\sphinxincludegraphics{{8872db5b249404c8e14edbab82f5358ec876d78ae3ef3d65305b1190711b891a}.png}

\end{sphinxuseclass}\end{sphinxVerbatimOutput}

\end{sphinxuseclass}
\begin{sphinxuseclass}{cell}\begin{sphinxVerbatimInput}

\begin{sphinxuseclass}{cell_input}
\begin{sphinxVerbatim}[commandchars=\\\{\}]
\PYG{n}{me\PYGZus{}ew}\PYG{o}{.}\PYG{n}{mul}\PYG{p}{(}\PYG{l+m+mi}{100}\PYG{p}{)}\PYG{o}{.}\PYG{n}{join}\PYG{p}{(}\PYG{p}{[}\PYG{n}{ff\PYGZus{}0}\PYG{p}{[}\PYG{l+m+mi}{0}\PYG{p}{]}\PYG{p}{,} \PYG{n}{ff\PYGZus{}mom}\PYG{p}{[}\PYG{l+m+mi}{0}\PYG{p}{]}\PYG{p}{]}\PYG{p}{)}\PYG{o}{.}\PYG{n}{pipe}\PYG{p}{(}\PYG{n}{get\PYGZus{}coefs}\PYG{p}{,} \PYG{n}{fun}\PYG{o}{=}\PYG{n}{ff4}\PYG{p}{,} \PYG{n}{n}\PYG{o}{=}\PYG{l+m+mi}{10}\PYG{p}{)}\PYG{o}{.}\PYG{n}{pipe}\PYG{p}{(}\PYG{n}{plot\PYGZus{}alpha}\PYG{p}{)}
\PYG{n}{plt}\PYG{o}{.}\PYG{n}{title}\PYG{p}{(}
    \PYG{l+s+s1}{\PYGZsq{}}\PYG{l+s+s1}{FF4 Tests of Size Strategy}\PYG{l+s+s1}{\PYGZsq{}} \PYG{o}{+}
    \PYG{l+s+s1}{\PYGZsq{}}\PYG{l+s+se}{\PYGZbs{}n}\PYG{l+s+s1}{Equal\PYGZhy{}Weighted Portfolios Formed June Market Value of Equity}\PYG{l+s+s1}{\PYGZsq{}} \PYG{o}{+}
    \PYG{l+s+s1}{\PYGZsq{}}\PYG{l+s+se}{\PYGZbs{}n}\PYG{l+s+s1}{Black Vertical Bars Indicate Standard Errors}\PYG{l+s+s1}{\PYGZsq{}}
\PYG{p}{)}
\PYG{n}{plt}\PYG{o}{.}\PYG{n}{show}\PYG{p}{(}\PYG{p}{)}
\end{sphinxVerbatim}

\end{sphinxuseclass}\end{sphinxVerbatimInput}
\begin{sphinxVerbatimOutput}

\begin{sphinxuseclass}{cell_output}
\noindent\sphinxincludegraphics{{da1a5df44d3d573abd9b0e06810182e2faf2d6075d7ba52d9720be73ab245739}.png}

\end{sphinxuseclass}\end{sphinxVerbatimOutput}

\end{sphinxuseclass}
\sphinxAtStartPar
If we had more time to dig into the size investing strategy, we would find:
\begin{enumerate}
\sphinxsetlistlabels{\arabic}{enumi}{enumii}{}{.}%
\item {} 
\sphinxAtStartPar
The alpha is concentrated in January due to tax loss harvesting, where investors sell their losers in December and buy them back in January to earn a tax deduction

\item {} 
\sphinxAtStartPar
The smallest stocks in the ME 1 portfolio are tiny, making them impractical for institutional investors and allowing the anomaly to persist (because these stocks are small and illiquid, the size factor does not try to explain away their alpha)

\end{enumerate}

\sphinxAtStartPar
We can quickly see the point 2 above by plotting the mean marker capitalization for each portfolio.
This analysis is “quick and dirty” because we have not inflation adjusted these values.
Still, we quickly see that the ME 10 stocks are 1,000 times larger than the ME 1 stocks, on average.

\begin{sphinxuseclass}{cell}\begin{sphinxVerbatimInput}

\begin{sphinxuseclass}{cell_input}
\begin{sphinxVerbatim}[commandchars=\\\{\}]
\PYG{p}{(}
    \PYG{n}{me}
    \PYG{o}{.}\PYG{n}{groupby}\PYG{p}{(}\PYG{p}{[}\PYG{l+s+s1}{\PYGZsq{}}\PYG{l+s+s1}{Portfolio}\PYG{l+s+s1}{\PYGZsq{}}\PYG{p}{,} \PYG{l+s+s1}{\PYGZsq{}}\PYG{l+s+s1}{date}\PYG{l+s+s1}{\PYGZsq{}}\PYG{p}{]}\PYG{p}{)}
    \PYG{p}{[}\PYG{l+s+s1}{\PYGZsq{}}\PYG{l+s+s1}{Trailing ME}\PYG{l+s+s1}{\PYGZsq{}}\PYG{p}{]}
    \PYG{o}{.}\PYG{n}{mean}\PYG{p}{(}\PYG{p}{)}
    \PYG{o}{.}\PYG{n}{unstack}\PYG{p}{(}\PYG{l+s+s1}{\PYGZsq{}}\PYG{l+s+s1}{Portfolio}\PYG{l+s+s1}{\PYGZsq{}}\PYG{p}{)}
    \PYG{o}{.}\PYG{n}{add\PYGZus{}prefix}\PYG{p}{(}\PYG{l+s+s1}{\PYGZsq{}}\PYG{l+s+s1}{ME }\PYG{l+s+s1}{\PYGZsq{}}\PYG{p}{)}
    \PYG{o}{.}\PYG{n}{mean}\PYG{p}{(}\PYG{p}{)}
    \PYG{o}{.}\PYG{n}{plot}\PYG{p}{(}\PYG{n}{kind}\PYG{o}{=}\PYG{l+s+s1}{\PYGZsq{}}\PYG{l+s+s1}{bar}\PYG{l+s+s1}{\PYGZsq{}}\PYG{p}{)}
\PYG{p}{)}
\PYG{n}{plt}\PYG{o}{.}\PYG{n}{semilogy}\PYG{p}{(}\PYG{p}{)}
\PYG{n}{plt}\PYG{o}{.}\PYG{n}{ylabel}\PYG{p}{(}\PYG{l+s+s1}{\PYGZsq{}}\PYG{l+s+s1}{Mean Market Value of Equity (Nominal U.S. Dollars)}\PYG{l+s+s1}{\PYGZsq{}}\PYG{p}{)}
\PYG{n}{plt}\PYG{o}{.}\PYG{n}{title}\PYG{p}{(}\PYG{l+s+s1}{\PYGZsq{}}\PYG{l+s+s1}{Mean Market Value of Stocks in ME Portfolios}\PYG{l+s+s1}{\PYGZsq{}}\PYG{p}{)}
\PYG{n}{plt}\PYG{o}{.}\PYG{n}{show}\PYG{p}{(}\PYG{p}{)}
\end{sphinxVerbatim}

\end{sphinxuseclass}\end{sphinxVerbatimInput}
\begin{sphinxVerbatimOutput}

\begin{sphinxuseclass}{cell_output}
\noindent\sphinxincludegraphics{{9912623f731b1538e1491a458598c59c8be346a294d6d950902f235fd10a843d}.png}

\end{sphinxuseclass}\end{sphinxVerbatimOutput}

\end{sphinxuseclass}

\subsubsection{Implement a value\sphinxhyphen{}weighted size investing strategy based on market capitalization \textasciitilde{}\textasciitilde{}at the start of each month\textasciitilde{}\textasciitilde{} from the previous June}
\label{\detokenize{herron_06_practice_03:implement-a-value-weighted-size-investing-strategy-based-on-market-capitalization-at-the-start-of-each-month-from-the-previous-june}}
\sphinxAtStartPar
With value\sphinxhyphen{}weighted portfolios, the size investing strategy alphas disappear!
The alpha disappears because value\sphinxhyphen{}weighted portfolios down weight the smallest of the smallest stocks that made the size investing strategy appear to be returns not explained by risk.

\begin{sphinxuseclass}{cell}\begin{sphinxVerbatimInput}

\begin{sphinxuseclass}{cell_input}
\begin{sphinxVerbatim}[commandchars=\\\{\}]
\PYG{n}{me\PYGZus{}vw} \PYG{o}{=} \PYG{p}{(}
    \PYG{n}{me}
    \PYG{o}{.}\PYG{n}{groupby}\PYG{p}{(}\PYG{p}{[}\PYG{l+s+s1}{\PYGZsq{}}\PYG{l+s+s1}{date}\PYG{l+s+s1}{\PYGZsq{}}\PYG{p}{,} \PYG{l+s+s1}{\PYGZsq{}}\PYG{l+s+s1}{Portfolio}\PYG{l+s+s1}{\PYGZsq{}}\PYG{p}{]}\PYG{p}{)}
    \PYG{o}{.}\PYG{n}{apply}\PYG{p}{(}\PYG{k}{lambda} \PYG{n}{x}\PYG{p}{:} \PYG{n}{np}\PYG{o}{.}\PYG{n}{average}\PYG{p}{(}\PYG{n}{a}\PYG{o}{=}\PYG{n}{x}\PYG{p}{[}\PYG{l+s+s1}{\PYGZsq{}}\PYG{l+s+s1}{Return}\PYG{l+s+s1}{\PYGZsq{}}\PYG{p}{]}\PYG{p}{,} \PYG{n}{weights}\PYG{o}{=}\PYG{n}{x}\PYG{p}{[}\PYG{l+s+s1}{\PYGZsq{}}\PYG{l+s+s1}{Trailing ME}\PYG{l+s+s1}{\PYGZsq{}}\PYG{p}{]}\PYG{p}{)}\PYG{p}{)}
    \PYG{o}{.}\PYG{n}{unstack}\PYG{p}{(}\PYG{l+s+s1}{\PYGZsq{}}\PYG{l+s+s1}{Portfolio}\PYG{l+s+s1}{\PYGZsq{}}\PYG{p}{)}
    \PYG{o}{.}\PYG{n}{add\PYGZus{}prefix}\PYG{p}{(}\PYG{l+s+s1}{\PYGZsq{}}\PYG{l+s+s1}{ME }\PYG{l+s+s1}{\PYGZsq{}}\PYG{p}{)}
\PYG{p}{)}
\end{sphinxVerbatim}

\end{sphinxuseclass}\end{sphinxVerbatimInput}

\end{sphinxuseclass}
\begin{sphinxuseclass}{cell}\begin{sphinxVerbatimInput}

\begin{sphinxuseclass}{cell_input}
\begin{sphinxVerbatim}[commandchars=\\\{\}]
\PYG{n}{me\PYGZus{}vw}\PYG{o}{.}\PYG{n}{mul}\PYG{p}{(}\PYG{l+m+mi}{100}\PYG{p}{)}\PYG{o}{.}\PYG{n}{join}\PYG{p}{(}\PYG{p}{[}\PYG{n}{ff\PYGZus{}0}\PYG{p}{[}\PYG{l+m+mi}{0}\PYG{p}{]}\PYG{p}{,} \PYG{n}{ff\PYGZus{}mom}\PYG{p}{[}\PYG{l+m+mi}{0}\PYG{p}{]}\PYG{p}{]}\PYG{p}{)}\PYG{o}{.}\PYG{n}{pipe}\PYG{p}{(}\PYG{n}{get\PYGZus{}coefs}\PYG{p}{,} \PYG{n}{fun}\PYG{o}{=}\PYG{n}{capm}\PYG{p}{,} \PYG{n}{n}\PYG{o}{=}\PYG{l+m+mi}{10}\PYG{p}{)}\PYG{o}{.}\PYG{n}{pipe}\PYG{p}{(}\PYG{n}{plot\PYGZus{}alpha}\PYG{p}{)}
\PYG{n}{plt}\PYG{o}{.}\PYG{n}{title}\PYG{p}{(}
    \PYG{l+s+s1}{\PYGZsq{}}\PYG{l+s+s1}{CAPM Tests of Size Strategy}\PYG{l+s+s1}{\PYGZsq{}} \PYG{o}{+}
    \PYG{l+s+s1}{\PYGZsq{}}\PYG{l+s+se}{\PYGZbs{}n}\PYG{l+s+s1}{Value\PYGZhy{}Weighted Portfolios Formed June Market Value of Equity}\PYG{l+s+s1}{\PYGZsq{}} \PYG{o}{+}
    \PYG{l+s+s1}{\PYGZsq{}}\PYG{l+s+se}{\PYGZbs{}n}\PYG{l+s+s1}{Black Vertical Bars Indicate Standard Errors}\PYG{l+s+s1}{\PYGZsq{}}
\PYG{p}{)}
\PYG{n}{plt}\PYG{o}{.}\PYG{n}{show}\PYG{p}{(}\PYG{p}{)}
\end{sphinxVerbatim}

\end{sphinxuseclass}\end{sphinxVerbatimInput}
\begin{sphinxVerbatimOutput}

\begin{sphinxuseclass}{cell_output}
\noindent\sphinxincludegraphics{{3f1ab9c29b78dcf3f973dbae288263b3138d8bf9faea23b8554377f0023b8978}.png}

\end{sphinxuseclass}\end{sphinxVerbatimOutput}

\end{sphinxuseclass}
\begin{sphinxuseclass}{cell}\begin{sphinxVerbatimInput}

\begin{sphinxuseclass}{cell_input}
\begin{sphinxVerbatim}[commandchars=\\\{\}]
\PYG{n}{me\PYGZus{}vw}\PYG{o}{.}\PYG{n}{mul}\PYG{p}{(}\PYG{l+m+mi}{100}\PYG{p}{)}\PYG{o}{.}\PYG{n}{join}\PYG{p}{(}\PYG{p}{[}\PYG{n}{ff\PYGZus{}0}\PYG{p}{[}\PYG{l+m+mi}{0}\PYG{p}{]}\PYG{p}{,} \PYG{n}{ff\PYGZus{}mom}\PYG{p}{[}\PYG{l+m+mi}{0}\PYG{p}{]}\PYG{p}{]}\PYG{p}{)}\PYG{o}{.}\PYG{n}{pipe}\PYG{p}{(}\PYG{n}{get\PYGZus{}coefs}\PYG{p}{,} \PYG{n}{fun}\PYG{o}{=}\PYG{n}{ff4}\PYG{p}{,} \PYG{n}{n}\PYG{o}{=}\PYG{l+m+mi}{10}\PYG{p}{)}\PYG{o}{.}\PYG{n}{pipe}\PYG{p}{(}\PYG{n}{plot\PYGZus{}alpha}\PYG{p}{)}
\PYG{n}{plt}\PYG{o}{.}\PYG{n}{title}\PYG{p}{(}
    \PYG{l+s+s1}{\PYGZsq{}}\PYG{l+s+s1}{FF4 Tests of Size Strategy}\PYG{l+s+s1}{\PYGZsq{}} \PYG{o}{+}
    \PYG{l+s+s1}{\PYGZsq{}}\PYG{l+s+se}{\PYGZbs{}n}\PYG{l+s+s1}{Value\PYGZhy{}Weighted Portfolios Formed June Market Value of Equity}\PYG{l+s+s1}{\PYGZsq{}} \PYG{o}{+}
    \PYG{l+s+s1}{\PYGZsq{}}\PYG{l+s+se}{\PYGZbs{}n}\PYG{l+s+s1}{Black Vertical Bars Indicate Standard Errors}\PYG{l+s+s1}{\PYGZsq{}}
\PYG{p}{)}
\PYG{n}{plt}\PYG{o}{.}\PYG{n}{show}\PYG{p}{(}\PYG{p}{)}
\end{sphinxVerbatim}

\end{sphinxuseclass}\end{sphinxVerbatimInput}
\begin{sphinxVerbatimOutput}

\begin{sphinxuseclass}{cell_output}
\noindent\sphinxincludegraphics{{8e1d15ddfe810ce7dbb2eed7e8f77fe665b0dbd55e16bed2bd257f7354616cac}.png}

\end{sphinxuseclass}\end{sphinxVerbatimOutput}

\end{sphinxuseclass}
\sphinxstepscope


\section{Herron Topic 6 \sphinxhyphen{} Practice (Wednesday 11:45 AM, Section 4)}
\label{\detokenize{herron_06_practice_04:herron-topic-6-practice-wednesday-11-45-am-section-4}}\label{\detokenize{herron_06_practice_04::doc}}

\subsection{Announcements}
\label{\detokenize{herron_06_practice_04:announcements}}\begin{itemize}
\item {} 
\sphinxAtStartPar
Quiz 6 mean and median were about \textasciitilde{}\textasciitilde{}90\% and 96\%\textasciitilde{}\textasciitilde{}  92\% and 100\%
\begin{itemize}
\item {} 
\sphinxAtStartPar
Please see the solution on Canvas and let me know if you have any questions

\item {} 
\sphinxAtStartPar
Gradescope shows which hidden tests you missed

\item {} 
\sphinxAtStartPar
I recently widened the tolerance on two hidden tests to give full credit if you minimized variance instead of volatility

\end{itemize}

\item {} 
\sphinxAtStartPar
I posted {\hyperref[\detokenize{project_02::doc}]{\sphinxcrossref{\DUrole{doc,std,std-doc}{project 2}}}} about Bitcoin and gold as inflation and market risk hedges

\item {} 
\sphinxAtStartPar
Next week (week of 4/10) is the assessment exam
\begin{itemize}
\item {} 
\sphinxAtStartPar
MSFQ students must take it for 5\% of overall course grade

\item {} 
\sphinxAtStartPar
Non\sphinxhyphen{}MSQF students do not take it and weight their grades by only 95\%

\item {} 
\sphinxAtStartPar
20 multiple choice questions on the 6 core courses (corporate finance, investments, math, data analytics, empirical methods, and derivatives)

\item {} 
\sphinxAtStartPar
\sphinxstyleemphasis{\sphinxstylestrong{You must be in the classroom during your assigned date and time to take the MSFQ assessment exam}}

\item {} 
\sphinxAtStartPar
If there is interest, we will discuss five stylized facts of asset returns after the MSFQ assessment exam

\item {} 
\sphinxAtStartPar
We will also have our final quiz, quiz 7

\end{itemize}

\item {} 
\sphinxAtStartPar
The following week (week of 4/17) we will discuss Herron topic 5 (simulations)
\begin{itemize}
\item {} 
\sphinxAtStartPar
I will record a lecture video and complete a practice notebook

\item {} 
\sphinxAtStartPar
But we will reserve class time for group work

\item {} 
\sphinxAtStartPar
The class voted about 2\sphinxhyphen{}to\sphinxhyphen{}1 not to drop a topic, so this is a compromise, given that Monday is Patriot’s Day holiday

\end{itemize}

\item {} 
\sphinxAtStartPar
The following week (week of 4/24) we will reserve class time for group work
\begin{itemize}
\item {} 
\sphinxAtStartPar
Project 2 is due Wednesday, 4/26, at 11:59 PM

\item {} 
\sphinxAtStartPar
Teammates Review 2 is due Wednesday, 4/26, at 11:59 PM

\item {} 
\sphinxAtStartPar
30,000 DataCamp XP are due Friday, 4/28, at 11:59 PM

\end{itemize}

\item {} 
\sphinxAtStartPar
\sphinxstyleemphasis{\sphinxstylestrong{Somewhere in there, please reserve 10 minutes to complete a TRACE review for this course}}
\begin{itemize}
\item {} 
\sphinxAtStartPar
I cannot make you complete TRACE reviews

\item {} 
\sphinxAtStartPar
But they are very helpful

\item {} 
\sphinxAtStartPar
I change my courses every semester, hopefully for the better, based on TRACE reviews

\end{itemize}

\end{itemize}


\subsection{Practice}
\label{\detokenize{herron_06_practice_04:practice}}
\begin{sphinxuseclass}{cell}\begin{sphinxVerbatimInput}

\begin{sphinxuseclass}{cell_input}
\begin{sphinxVerbatim}[commandchars=\\\{\}]
\PYG{k+kn}{import} \PYG{n+nn}{matplotlib}\PYG{n+nn}{.}\PYG{n+nn}{pyplot} \PYG{k}{as} \PYG{n+nn}{plt}
\PYG{k+kn}{import} \PYG{n+nn}{numpy} \PYG{k}{as} \PYG{n+nn}{np}
\PYG{k+kn}{import} \PYG{n+nn}{pandas} \PYG{k}{as} \PYG{n+nn}{pd}
\end{sphinxVerbatim}

\end{sphinxuseclass}\end{sphinxVerbatimInput}

\end{sphinxuseclass}
\begin{sphinxuseclass}{cell}\begin{sphinxVerbatimInput}

\begin{sphinxuseclass}{cell_input}
\begin{sphinxVerbatim}[commandchars=\\\{\}]
\PYG{o}{\PYGZpc{}}\PYG{k}{config} InlineBackend.figure\PYGZus{}format = \PYGZsq{}retina\PYGZsq{}
\PYG{o}{\PYGZpc{}}\PYG{k}{precision} 4
\PYG{n}{pd}\PYG{o}{.}\PYG{n}{options}\PYG{o}{.}\PYG{n}{display}\PYG{o}{.}\PYG{n}{float\PYGZus{}format} \PYG{o}{=} \PYG{l+s+s1}{\PYGZsq{}}\PYG{l+s+si}{\PYGZob{}:.4f\PYGZcb{}}\PYG{l+s+s1}{\PYGZsq{}}\PYG{o}{.}\PYG{n}{format}
\end{sphinxVerbatim}

\end{sphinxuseclass}\end{sphinxVerbatimInput}

\end{sphinxuseclass}
\begin{sphinxuseclass}{cell}\begin{sphinxVerbatimInput}

\begin{sphinxuseclass}{cell_input}
\begin{sphinxVerbatim}[commandchars=\\\{\}]
\PYG{k+kn}{import} \PYG{n+nn}{yfinance} \PYG{k}{as} \PYG{n+nn}{yf}
\PYG{k+kn}{import} \PYG{n+nn}{pandas\PYGZus{}datareader} \PYG{k}{as} \PYG{n+nn}{pdr}
\PYG{k+kn}{import} \PYG{n+nn}{requests\PYGZus{}cache}
\PYG{n}{session} \PYG{o}{=} \PYG{n}{requests\PYGZus{}cache}\PYG{o}{.}\PYG{n}{CachedSession}\PYG{p}{(}\PYG{p}{)}
\end{sphinxVerbatim}

\end{sphinxuseclass}\end{sphinxVerbatimInput}

\end{sphinxuseclass}

\subsubsection{Reimplement the equal\sphinxhyphen{}weighted momentum investing strategy from the lecture notebook}
\label{\detokenize{herron_06_practice_04:reimplement-the-equal-weighted-momentum-investing-strategy-from-the-lecture-notebook}}
\sphinxAtStartPar
Try to use a few cells and temporary variables as you can (i.e., perform calculations inside \sphinxcode{\sphinxupquote{pd.concat()}}).

\begin{sphinxuseclass}{cell}\begin{sphinxVerbatimInput}

\begin{sphinxuseclass}{cell_input}
\begin{sphinxVerbatim}[commandchars=\\\{\}]
\PYG{n}{crsp} \PYG{o}{=} \PYG{p}{(}
    \PYG{n}{pd}\PYG{o}{.}\PYG{n}{read\PYGZus{}csv}\PYG{p}{(}
        \PYG{n}{filepath\PYGZus{}or\PYGZus{}buffer}\PYG{o}{=}\PYG{l+s+s1}{\PYGZsq{}}\PYG{l+s+s1}{crsp.csv}\PYG{l+s+s1}{\PYGZsq{}}\PYG{p}{,}
        \PYG{n}{parse\PYGZus{}dates}\PYG{o}{=}\PYG{p}{[}\PYG{l+s+s1}{\PYGZsq{}}\PYG{l+s+s1}{date}\PYG{l+s+s1}{\PYGZsq{}}\PYG{p}{]}\PYG{p}{,}
        \PYG{n}{na\PYGZus{}values}\PYG{o}{=}\PYG{p}{[}\PYG{l+s+s1}{\PYGZsq{}}\PYG{l+s+s1}{A}\PYG{l+s+s1}{\PYGZsq{}}\PYG{p}{,} \PYG{l+s+s1}{\PYGZsq{}}\PYG{l+s+s1}{B}\PYG{l+s+s1}{\PYGZsq{}}\PYG{p}{,} \PYG{l+s+s1}{\PYGZsq{}}\PYG{l+s+s1}{C}\PYG{l+s+s1}{\PYGZsq{}}\PYG{p}{]} \PYG{c+c1}{\PYGZsh{} CRSP uses letter codes to provide additional information, which we can ignore}
    \PYG{p}{)}
    \PYG{o}{.}\PYG{n}{assign}\PYG{p}{(}
        \PYG{n}{date}\PYG{o}{=}\PYG{k}{lambda} \PYG{n}{x}\PYG{p}{:} \PYG{n}{x}\PYG{p}{[}\PYG{l+s+s1}{\PYGZsq{}}\PYG{l+s+s1}{date}\PYG{l+s+s1}{\PYGZsq{}}\PYG{p}{]}\PYG{o}{.}\PYG{n}{dt}\PYG{o}{.}\PYG{n}{to\PYGZus{}period}\PYG{p}{(}\PYG{n}{freq}\PYG{o}{=}\PYG{l+s+s1}{\PYGZsq{}}\PYG{l+s+s1}{M}\PYG{l+s+s1}{\PYGZsq{}}\PYG{p}{)}\PYG{p}{,} \PYG{c+c1}{\PYGZsh{} returns span a month, so I prefer to work with periods instead of dates}
        \PYG{n}{ME}\PYG{o}{=}\PYG{k}{lambda} \PYG{n}{x}\PYG{p}{:} \PYG{n}{x}\PYG{p}{[}\PYG{l+s+s1}{\PYGZsq{}}\PYG{l+s+s1}{PRC}\PYG{l+s+s1}{\PYGZsq{}}\PYG{p}{]}\PYG{o}{.}\PYG{n}{abs}\PYG{p}{(}\PYG{p}{)} \PYG{o}{*} \PYG{n}{x}\PYG{p}{[}\PYG{l+s+s1}{\PYGZsq{}}\PYG{l+s+s1}{SHROUT}\PYG{l+s+s1}{\PYGZsq{}}\PYG{p}{]} \PYG{c+c1}{\PYGZsh{} market value of equity in thousands of dollars}
    \PYG{p}{)}
    \PYG{o}{.}\PYG{n}{rename\PYGZus{}axis}\PYG{p}{(}\PYG{n}{columns}\PYG{o}{=}\PYG{l+s+s1}{\PYGZsq{}}\PYG{l+s+s1}{Variable}\PYG{l+s+s1}{\PYGZsq{}}\PYG{p}{)}
    \PYG{o}{.}\PYG{n}{set\PYGZus{}index}\PYG{p}{(}\PYG{p}{[}\PYG{l+s+s1}{\PYGZsq{}}\PYG{l+s+s1}{PERMNO}\PYG{l+s+s1}{\PYGZsq{}}\PYG{p}{,} \PYG{l+s+s1}{\PYGZsq{}}\PYG{l+s+s1}{date}\PYG{l+s+s1}{\PYGZsq{}}\PYG{p}{]}\PYG{p}{)}
\PYG{p}{)}
\end{sphinxVerbatim}

\end{sphinxuseclass}\end{sphinxVerbatimInput}

\end{sphinxuseclass}
\sphinxAtStartPar
Please the {\hyperref[\detokenize{herron_06_lecture::doc}]{\sphinxcrossref{\DUrole{doc,std,std-doc}{Herron Topic 6 lecture notebook}}}} for variable definitions.

\begin{sphinxuseclass}{cell}\begin{sphinxVerbatimInput}

\begin{sphinxuseclass}{cell_input}
\begin{sphinxVerbatim}[commandchars=\\\{\}]
\PYG{n}{crsp}\PYG{o}{.}\PYG{n}{head}\PYG{p}{(}\PYG{p}{)}
\end{sphinxVerbatim}

\end{sphinxuseclass}\end{sphinxVerbatimInput}
\begin{sphinxVerbatimOutput}

\begin{sphinxuseclass}{cell_output}
\begin{sphinxVerbatim}[commandchars=\\\{\}]
Variable        SHRCD     PRC     RET    SHROUT         ME
PERMNO date                                               
10000  1986\PYGZhy{}01     10 \PYGZhy{}4.3750     NaN 3680.0000 16100.0000
       1986\PYGZhy{}02     10 \PYGZhy{}3.2500 \PYGZhy{}0.2571 3680.0000 11960.0000
       1986\PYGZhy{}03     10 \PYGZhy{}4.4375  0.3654 3680.0000 16330.0000
       1986\PYGZhy{}04     10 \PYGZhy{}4.0000 \PYGZhy{}0.0986 3793.0000 15172.0000
       1986\PYGZhy{}05     10 \PYGZhy{}3.1094 \PYGZhy{}0.2227 3793.0000 11793.8783
\end{sphinxVerbatim}

\end{sphinxuseclass}\end{sphinxVerbatimOutput}

\end{sphinxuseclass}
\sphinxAtStartPar
To implement a momentum strategy, we need:
\begin{enumerate}
\sphinxsetlistlabels{\arabic}{enumi}{enumii}{}{.}%
\item {} 
\sphinxAtStartPar
1\sphinxhyphen{}month returns, which are the returns we receive for each holding period

\item {} 
\sphinxAtStartPar
11\sphinxhyphen{}month returns, which are the returns (from month \sphinxhyphen{}12 to month \sphinxhyphen{}2) that we use to rank stocks and assign to momentum portfolios

\item {} 
\sphinxAtStartPar
Momentum portfolio assignments based on 11\sphinxhyphen{}month trailing returns

\end{enumerate}

\begin{sphinxuseclass}{cell}\begin{sphinxVerbatimInput}

\begin{sphinxuseclass}{cell_input}
\begin{sphinxVerbatim}[commandchars=\\\{\}]
\PYG{n}{ret\PYGZus{}1m} \PYG{o}{=} \PYG{n}{crsp}\PYG{p}{[}\PYG{l+s+s1}{\PYGZsq{}}\PYG{l+s+s1}{RET}\PYG{l+s+s1}{\PYGZsq{}}\PYG{p}{]}\PYG{o}{.}\PYG{n}{unstack}\PYG{p}{(}\PYG{l+s+s1}{\PYGZsq{}}\PYG{l+s+s1}{PERMNO}\PYG{l+s+s1}{\PYGZsq{}}\PYG{p}{)}
\end{sphinxVerbatim}

\end{sphinxuseclass}\end{sphinxVerbatimInput}

\end{sphinxuseclass}
\sphinxAtStartPar
There is not a \sphinxcode{\sphinxupquote{.prod()}} method for \sphinxcode{\sphinxupquote{.rolling()}}, so the fastest way to calculate 11\sphinxhyphen{}month rolling returns is to:
\begin{enumerate}
\sphinxsetlistlabels{\arabic}{enumi}{enumii}{}{.}%
\item {} 
\sphinxAtStartPar
Convert simple returns to log returns with \sphinxcode{\sphinxupquote{.pipe(np.log1p)}}

\item {} 
\sphinxAtStartPar
Calculate 11\sphinxhyphen{}month rolling \sphinxstyleemphasis{log} returns with \sphinxcode{\sphinxupquote{.rolling(11).sum()}}

\item {} 
\sphinxAtStartPar
Convert log returns to simple returns with \sphinxcode{\sphinxupquote{.pipe(np.expm1)}}

\end{enumerate}

\begin{sphinxuseclass}{cell}\begin{sphinxVerbatimInput}

\begin{sphinxuseclass}{cell_input}
\begin{sphinxVerbatim}[commandchars=\\\{\}]
\PYG{n}{ret\PYGZus{}11m} \PYG{o}{=} \PYG{n}{ret\PYGZus{}1m}\PYG{o}{.}\PYG{n}{pipe}\PYG{p}{(}\PYG{n}{np}\PYG{o}{.}\PYG{n}{log1p}\PYG{p}{)}\PYG{o}{.}\PYG{n}{rolling}\PYG{p}{(}\PYG{l+m+mi}{11}\PYG{p}{)}\PYG{o}{.}\PYG{n}{sum}\PYG{p}{(}\PYG{p}{)}\PYG{o}{.}\PYG{n}{pipe}\PYG{p}{(}\PYG{n}{np}\PYG{o}{.}\PYG{n}{expm1}\PYG{p}{)}
\end{sphinxVerbatim}

\end{sphinxuseclass}\end{sphinxVerbatimInput}

\end{sphinxuseclass}
\sphinxAtStartPar
Then we use \sphinxcode{\sphinxupquote{pd.qcut()}} to assign \sphinxcode{\sphinxupquote{ret\_11m}} to ten momentum portfolios.
Here is a simple example:

\begin{sphinxuseclass}{cell}\begin{sphinxVerbatimInput}

\begin{sphinxuseclass}{cell_input}
\begin{sphinxVerbatim}[commandchars=\\\{\}]
\PYG{n}{pd}\PYG{o}{.}\PYG{n}{qcut}\PYG{p}{(}\PYG{n}{x}\PYG{o}{=}\PYG{n}{np}\PYG{o}{.}\PYG{n}{arange}\PYG{p}{(}\PYG{l+m+mi}{20}\PYG{p}{)}\PYG{p}{,} \PYG{n}{q}\PYG{o}{=}\PYG{l+m+mi}{2}\PYG{p}{,} \PYG{n}{labels}\PYG{o}{=}\PYG{k+kc}{False}\PYG{p}{)} \PYG{o}{+} \PYG{l+m+mi}{1}
\end{sphinxVerbatim}

\end{sphinxuseclass}\end{sphinxVerbatimInput}
\begin{sphinxVerbatimOutput}

\begin{sphinxuseclass}{cell_output}
\begin{sphinxVerbatim}[commandchars=\\\{\}]
array([1, 1, 1, 1, 1, 1, 1, 1, 1, 1, 2, 2, 2, 2, 2, 2, 2, 2, 2, 2],
      dtype=int64)
\end{sphinxVerbatim}

\end{sphinxuseclass}\end{sphinxVerbatimOutput}

\end{sphinxuseclass}
\sphinxAtStartPar
Here are the momentum portfolios:

\begin{sphinxuseclass}{cell}\begin{sphinxVerbatimInput}

\begin{sphinxuseclass}{cell_input}
\begin{sphinxVerbatim}[commandchars=\\\{\}]
\PYG{n}{port\PYGZus{}11m} \PYG{o}{=} \PYG{n}{ret\PYGZus{}11m}\PYG{o}{.}\PYG{n}{dropna}\PYG{p}{(}\PYG{n}{how}\PYG{o}{=}\PYG{l+s+s1}{\PYGZsq{}}\PYG{l+s+s1}{all}\PYG{l+s+s1}{\PYGZsq{}}\PYG{p}{)}\PYG{o}{.}\PYG{n}{apply}\PYG{p}{(}\PYG{n}{pd}\PYG{o}{.}\PYG{n}{qcut}\PYG{p}{,} \PYG{n}{q}\PYG{o}{=}\PYG{l+m+mi}{10}\PYG{p}{,} \PYG{n}{labels}\PYG{o}{=}\PYG{k+kc}{False}\PYG{p}{,} \PYG{n}{axis}\PYG{o}{=}\PYG{l+m+mi}{1}\PYG{p}{)}
\end{sphinxVerbatim}

\end{sphinxuseclass}\end{sphinxVerbatimInput}

\end{sphinxuseclass}
\sphinxAtStartPar
Short\sphinxhyphen{}term reversal, momentum, and long\sphinxhyphen{}term reversal represent distinct aspects of asset price behavior that have been observed and documented in the financial markets. As a researcher, I am particularly interested in understanding the underlying factors that contribute to these market anomalies and their implications for trading strategies and portfolio management.

\sphinxAtStartPar
Short\sphinxhyphen{}term reversal captures the tendency of asset prices to reverse direction in a brief period, typically days to weeks. This phenomenon can be associated with market overreactions to news events, temporary liquidity constraints, or other transient factors. Astute investors may capitalize on these short\sphinxhyphen{}term reversals by adopting a contrarian approach, buying assets that have recently underperformed and selling those that have outperformed.

\sphinxAtStartPar
Momentum represents the persistence of asset price trends over a short to medium\sphinxhyphen{}term horizon, typically ranging from 3 to 12 months. This market anomaly suggests that assets with strong recent performance continue to outperform, while poorly performing assets continue to underperform. Momentum can be linked to factors such as investors’ behavioral biases, positive feedback trading, and herd mentality. To exploit momentum, investors may adopt a trend\sphinxhyphen{}following strategy, buying assets with strong recent performance and selling those with weak performance.

\sphinxAtStartPar
Long\sphinxhyphen{}term reversal, often referred to as mean reversion, is the tendency of asset prices to revert to their long\sphinxhyphen{}term average or fundamental value over an extended period, typically several years. This phenomenon is thought to result from market participants gradually recognizing and correcting mispricing, leading to price adjustments. Investors following a long\sphinxhyphen{}term reversal strategy may focus on assets with extreme valuation levels, buying undervalued assets and selling overvalued ones, with the expectation that prices will eventually revert to their long\sphinxhyphen{}term mean.

\sphinxAtStartPar
\sphinxstyleemphasis{\sphinxstylestrong{To avoid short\sphinxhyphen{}term reversal, we skip one month between 11\sphinxhyphen{}month returns and portfolio formation.}}

\begin{sphinxuseclass}{cell}\begin{sphinxVerbatimInput}

\begin{sphinxuseclass}{cell_input}
\begin{sphinxVerbatim}[commandchars=\\\{\}]
\PYG{n}{mom} \PYG{o}{=} \PYG{p}{(}
    \PYG{n}{pd}\PYG{o}{.}\PYG{n}{concat}\PYG{p}{(}
        \PYG{n}{objs}\PYG{o}{=}\PYG{p}{[}
            \PYG{n}{ret\PYGZus{}1m}\PYG{p}{,}
            \PYG{n}{ret\PYGZus{}11m}\PYG{o}{.}\PYG{n}{shift}\PYG{p}{(}\PYG{l+m+mi}{2}\PYG{p}{)}\PYG{p}{,}
            \PYG{n}{port\PYGZus{}11m}\PYG{o}{.}\PYG{n}{shift}\PYG{p}{(}\PYG{l+m+mi}{2}\PYG{p}{)}\PYG{p}{,}
            \PYG{n}{crsp}\PYG{p}{[}\PYG{l+s+s1}{\PYGZsq{}}\PYG{l+s+s1}{ME}\PYG{l+s+s1}{\PYGZsq{}}\PYG{p}{]}\PYG{o}{.}\PYG{n}{unstack}\PYG{p}{(}\PYG{l+s+s1}{\PYGZsq{}}\PYG{l+s+s1}{PERMNO}\PYG{l+s+s1}{\PYGZsq{}}\PYG{p}{)}\PYG{o}{.}\PYG{n}{shift}\PYG{p}{(}\PYG{l+m+mi}{1}\PYG{p}{)}
        \PYG{p}{]}\PYG{p}{,}
        \PYG{n}{axis}\PYG{o}{=}\PYG{l+m+mi}{1}\PYG{p}{,}
        \PYG{n}{keys}\PYG{o}{=}\PYG{p}{[}\PYG{l+s+s1}{\PYGZsq{}}\PYG{l+s+s1}{Return}\PYG{l+s+s1}{\PYGZsq{}}\PYG{p}{,} \PYG{l+s+s1}{\PYGZsq{}}\PYG{l+s+s1}{Trailing Return}\PYG{l+s+s1}{\PYGZsq{}}\PYG{p}{,} \PYG{l+s+s1}{\PYGZsq{}}\PYG{l+s+s1}{Portfolio}\PYG{l+s+s1}{\PYGZsq{}}\PYG{p}{,} \PYG{l+s+s1}{\PYGZsq{}}\PYG{l+s+s1}{Trailing ME}\PYG{l+s+s1}{\PYGZsq{}}\PYG{p}{]}\PYG{p}{,}
        \PYG{n}{names}\PYG{o}{=}\PYG{p}{[}\PYG{l+s+s1}{\PYGZsq{}}\PYG{l+s+s1}{Variable}\PYG{l+s+s1}{\PYGZsq{}}\PYG{p}{]}
    \PYG{p}{)}
    \PYG{o}{.}\PYG{n}{stack}\PYG{p}{(}\PYG{l+s+s1}{\PYGZsq{}}\PYG{l+s+s1}{PERMNO}\PYG{l+s+s1}{\PYGZsq{}}\PYG{p}{)}
    \PYG{o}{.}\PYG{n}{dropna}\PYG{p}{(}\PYG{p}{)}
    \PYG{o}{.}\PYG{n}{assign}\PYG{p}{(}\PYG{n}{Portfolio} \PYG{o}{=} \PYG{k}{lambda} \PYG{n}{x}\PYG{p}{:} \PYG{l+m+mi}{1} \PYG{o}{+} \PYG{n}{x}\PYG{p}{[}\PYG{l+s+s1}{\PYGZsq{}}\PYG{l+s+s1}{Portfolio}\PYG{l+s+s1}{\PYGZsq{}}\PYG{p}{]}\PYG{o}{.}\PYG{n}{astype}\PYG{p}{(}\PYG{n+nb}{int}\PYG{p}{)}\PYG{p}{)}
\PYG{p}{)}
\end{sphinxVerbatim}

\end{sphinxuseclass}\end{sphinxVerbatimInput}

\end{sphinxuseclass}
\begin{sphinxuseclass}{cell}\begin{sphinxVerbatimInput}

\begin{sphinxuseclass}{cell_input}
\begin{sphinxVerbatim}[commandchars=\\\{\}]
\PYG{n}{mom\PYGZus{}ew} \PYG{o}{=} \PYG{p}{(}
    \PYG{n}{mom}
    \PYG{o}{.}\PYG{n}{groupby}\PYG{p}{(}\PYG{p}{[}\PYG{l+s+s1}{\PYGZsq{}}\PYG{l+s+s1}{date}\PYG{l+s+s1}{\PYGZsq{}}\PYG{p}{,} \PYG{l+s+s1}{\PYGZsq{}}\PYG{l+s+s1}{Portfolio}\PYG{l+s+s1}{\PYGZsq{}}\PYG{p}{]}\PYG{p}{)}
    \PYG{p}{[}\PYG{l+s+s1}{\PYGZsq{}}\PYG{l+s+s1}{Return}\PYG{l+s+s1}{\PYGZsq{}}\PYG{p}{]}
    \PYG{o}{.}\PYG{n}{mean}\PYG{p}{(}\PYG{p}{)}
    \PYG{o}{.}\PYG{n}{unstack}\PYG{p}{(}\PYG{l+s+s1}{\PYGZsq{}}\PYG{l+s+s1}{Portfolio}\PYG{l+s+s1}{\PYGZsq{}}\PYG{p}{)}
    \PYG{o}{.}\PYG{n}{add\PYGZus{}prefix}\PYG{p}{(}\PYG{l+s+s1}{\PYGZsq{}}\PYG{l+s+s1}{Mom }\PYG{l+s+s1}{\PYGZsq{}}\PYG{p}{)}
\PYG{p}{)}
\end{sphinxVerbatim}

\end{sphinxuseclass}\end{sphinxVerbatimInput}

\end{sphinxuseclass}
\begin{sphinxuseclass}{cell}\begin{sphinxVerbatimInput}

\begin{sphinxuseclass}{cell_input}
\begin{sphinxVerbatim}[commandchars=\\\{\}]
\PYG{n}{mom\PYGZus{}ew}\PYG{o}{.}\PYG{n}{head}\PYG{p}{(}\PYG{p}{)}
\end{sphinxVerbatim}

\end{sphinxuseclass}\end{sphinxVerbatimInput}
\begin{sphinxVerbatimOutput}

\begin{sphinxuseclass}{cell_output}
\begin{sphinxVerbatim}[commandchars=\\\{\}]
Portfolio   Mom 1   Mom 2   Mom 3   Mom 4   Mom 5   Mom 6  Mom 7   Mom 8  \PYGZbs{}
date                                                                       
1927\PYGZhy{}01   \PYGZhy{}0.0006  0.0149  0.0425  0.0378  0.0009  0.0150 0.0124  0.0076   
1927\PYGZhy{}02    0.0641  0.0484  0.0866  0.0745  0.0436  0.0547 0.0561  0.0403   
1927\PYGZhy{}03   \PYGZhy{}0.0530 \PYGZhy{}0.0381 \PYGZhy{}0.0243 \PYGZhy{}0.0408 \PYGZhy{}0.0192 \PYGZhy{}0.0201 0.0097 \PYGZhy{}0.0084   
1927\PYGZhy{}04    0.0230 \PYGZhy{}0.0085 \PYGZhy{}0.0054 \PYGZhy{}0.0061  0.0002 \PYGZhy{}0.0032 0.0138  0.0030   
1927\PYGZhy{}05    0.0069  0.0353  0.0604  0.0536  0.0700  0.0950 0.0796  0.0838   

Portfolio   Mom 9  Mom 10  
date                       
1927\PYGZhy{}01   \PYGZhy{}0.0044  0.0126  
1927\PYGZhy{}02    0.0579  0.0541  
1927\PYGZhy{}03   \PYGZhy{}0.0058  0.0012  
1927\PYGZhy{}04    0.0323  0.0421  
1927\PYGZhy{}05    0.0803  0.0886  
\end{sphinxVerbatim}

\end{sphinxuseclass}\end{sphinxVerbatimOutput}

\end{sphinxuseclass}
\sphinxAtStartPar
We see a (nearly) monotonic relation between momentum portfolios (based on trailing 11\sphinxhyphen{}month returns) and holding period returns!

\begin{sphinxuseclass}{cell}\begin{sphinxVerbatimInput}

\begin{sphinxuseclass}{cell_input}
\begin{sphinxVerbatim}[commandchars=\\\{\}]
\PYG{n}{mom\PYGZus{}ew}\PYG{o}{.}\PYG{n}{mean}\PYG{p}{(}\PYG{p}{)}\PYG{o}{.}\PYG{n}{mul}\PYG{p}{(}\PYG{l+m+mi}{100}\PYG{p}{)}\PYG{o}{.}\PYG{n}{plot}\PYG{p}{(}\PYG{n}{kind}\PYG{o}{=}\PYG{l+s+s1}{\PYGZsq{}}\PYG{l+s+s1}{bar}\PYG{l+s+s1}{\PYGZsq{}}\PYG{p}{)}
\PYG{n}{plt}\PYG{o}{.}\PYG{n}{ylabel}\PYG{p}{(}\PYG{l+s+s1}{\PYGZsq{}}\PYG{l+s+s1}{Mean Monthly Return (}\PYG{l+s+s1}{\PYGZpc{}}\PYG{l+s+s1}{)}\PYG{l+s+s1}{\PYGZsq{}}\PYG{p}{)}
\PYG{n}{plt}\PYG{o}{.}\PYG{n}{title}\PYG{p}{(}\PYG{l+s+s1}{\PYGZsq{}}\PYG{l+s+s1}{Momentum Strategy}\PYG{l+s+se}{\PYGZbs{}n}\PYG{l+s+s1}{ Equal\PYGZhy{}Weighted Portfolios}\PYG{l+s+se}{\PYGZbs{}n}\PYG{l+s+s1}{Formed on Months \PYGZhy{}12 to \PYGZhy{}2}\PYG{l+s+s1}{\PYGZsq{}}\PYG{p}{)}
\PYG{n}{plt}\PYG{o}{.}\PYG{n}{show}\PYG{p}{(}\PYG{p}{)}
\end{sphinxVerbatim}

\end{sphinxuseclass}\end{sphinxVerbatimInput}
\begin{sphinxVerbatimOutput}

\begin{sphinxuseclass}{cell_output}
\begin{sphinxVerbatim}[commandchars=\\\{\}]
\PYGZlt{}AxesSubplot:xlabel=\PYGZsq{}Portfolio\PYGZsq{}\PYGZgt{}
\end{sphinxVerbatim}

\noindent\sphinxincludegraphics{{9a2633b50c7a88523b27b196fefcfafd510d1f41f67efa8cf4263f504fe84fbe}.png}

\end{sphinxuseclass}\end{sphinxVerbatimOutput}

\end{sphinxuseclass}
\sphinxAtStartPar
This relation is stronger when we consider long\sphinxhyphen{}term, buy\sphinxhyphen{}and\sphinxhyphen{}hold returns!

\begin{sphinxuseclass}{cell}\begin{sphinxVerbatimInput}

\begin{sphinxuseclass}{cell_input}
\begin{sphinxVerbatim}[commandchars=\\\{\}]
\PYG{n}{mom\PYGZus{}ew}\PYG{o}{.}\PYG{n}{add}\PYG{p}{(}\PYG{l+m+mi}{1}\PYG{p}{)}\PYG{o}{.}\PYG{n}{cumprod}\PYG{p}{(}\PYG{p}{)}\PYG{o}{.}\PYG{n}{plot}\PYG{p}{(}\PYG{p}{)}
\PYG{n}{plt}\PYG{o}{.}\PYG{n}{semilogy}\PYG{p}{(}\PYG{p}{)}
\PYG{n}{plt}\PYG{o}{.}\PYG{n}{ylabel}\PYG{p}{(}\PYG{l+s+s1}{\PYGZsq{}}\PYG{l+s+s1}{Value of \PYGZdl{}1 Investment (\PYGZdl{})}\PYG{l+s+s1}{\PYGZsq{}}\PYG{p}{)}
\PYG{n}{plt}\PYG{o}{.}\PYG{n}{title}\PYG{p}{(}\PYG{l+s+s1}{\PYGZsq{}}\PYG{l+s+s1}{Momentum Strategy}\PYG{l+s+se}{\PYGZbs{}n}\PYG{l+s+s1}{ Equal\PYGZhy{}Weighted Portfolios}\PYG{l+s+se}{\PYGZbs{}n}\PYG{l+s+s1}{Formed on Months \PYGZhy{}12 to \PYGZhy{}2}\PYG{l+s+s1}{\PYGZsq{}}\PYG{p}{)}
\PYG{n}{plt}\PYG{o}{.}\PYG{n}{show}\PYG{p}{(}\PYG{p}{)}
\end{sphinxVerbatim}

\end{sphinxuseclass}\end{sphinxVerbatimInput}
\begin{sphinxVerbatimOutput}

\begin{sphinxuseclass}{cell_output}
\begin{sphinxVerbatim}[commandchars=\\\{\}]
[]
\end{sphinxVerbatim}

\noindent\sphinxincludegraphics{{ccabedf7af49b0007b47bea1b17251766a377ee7ca50d2f5a6cd203be10e48ac}.png}

\end{sphinxuseclass}\end{sphinxVerbatimOutput}

\end{sphinxuseclass}
\begin{sphinxuseclass}{cell}\begin{sphinxVerbatimInput}

\begin{sphinxuseclass}{cell_input}
\begin{sphinxVerbatim}[commandchars=\\\{\}]
\PYG{n}{ff\PYGZus{}0} \PYG{o}{=} \PYG{n}{pdr}\PYG{o}{.}\PYG{n}{DataReader}\PYG{p}{(}
    \PYG{n}{name}\PYG{o}{=}\PYG{l+s+s1}{\PYGZsq{}}\PYG{l+s+s1}{F\PYGZhy{}F\PYGZus{}Research\PYGZus{}Data\PYGZus{}Factors}\PYG{l+s+s1}{\PYGZsq{}}\PYG{p}{,}
    \PYG{n}{data\PYGZus{}source}\PYG{o}{=}\PYG{l+s+s1}{\PYGZsq{}}\PYG{l+s+s1}{famafrench}\PYG{l+s+s1}{\PYGZsq{}}\PYG{p}{,}
    \PYG{n}{start}\PYG{o}{=}\PYG{l+s+s1}{\PYGZsq{}}\PYG{l+s+s1}{1900}\PYG{l+s+s1}{\PYGZsq{}}\PYG{p}{,}
    \PYG{n}{session}\PYG{o}{=}\PYG{n}{session}
\PYG{p}{)}
\end{sphinxVerbatim}

\end{sphinxuseclass}\end{sphinxVerbatimInput}

\end{sphinxuseclass}
\begin{sphinxuseclass}{cell}\begin{sphinxVerbatimInput}

\begin{sphinxuseclass}{cell_input}
\begin{sphinxVerbatim}[commandchars=\\\{\}]
\PYG{n}{ff\PYGZus{}mom} \PYG{o}{=} \PYG{n}{pdr}\PYG{o}{.}\PYG{n}{DataReader}\PYG{p}{(}
    \PYG{n}{name}\PYG{o}{=}\PYG{l+s+s1}{\PYGZsq{}}\PYG{l+s+s1}{F\PYGZhy{}F\PYGZus{}Momentum\PYGZus{}Factor}\PYG{l+s+s1}{\PYGZsq{}}\PYG{p}{,}
    \PYG{n}{data\PYGZus{}source}\PYG{o}{=}\PYG{l+s+s1}{\PYGZsq{}}\PYG{l+s+s1}{famafrench}\PYG{l+s+s1}{\PYGZsq{}}\PYG{p}{,}
    \PYG{n}{start}\PYG{o}{=}\PYG{l+s+s1}{\PYGZsq{}}\PYG{l+s+s1}{1900}\PYG{l+s+s1}{\PYGZsq{}}\PYG{p}{,}
    \PYG{n}{session}\PYG{o}{=}\PYG{n}{session}
\PYG{p}{)}

\PYG{n}{ff\PYGZus{}mom}\PYG{p}{[}\PYG{l+m+mi}{0}\PYG{p}{]}\PYG{o}{.}\PYG{n}{columns} \PYG{o}{=} \PYG{p}{[}\PYG{n}{c}\PYG{o}{.}\PYG{n}{strip}\PYG{p}{(}\PYG{p}{)} \PYG{k}{for} \PYG{n}{c} \PYG{o+ow}{in} \PYG{n}{ff\PYGZus{}mom}\PYG{p}{[}\PYG{l+m+mi}{0}\PYG{p}{]}\PYG{o}{.}\PYG{n}{columns}\PYG{p}{]}
\end{sphinxVerbatim}

\end{sphinxuseclass}\end{sphinxVerbatimInput}

\end{sphinxuseclass}

\subsubsection{Add a long\sphinxhyphen{}short portfolio that is long portfolio 10 and short portfolio 1}
\label{\detokenize{herron_06_practice_04:add-a-long-short-portfolio-that-is-long-portfolio-10-and-short-portfolio-1}}
\sphinxAtStartPar
Call this long\sphinxhyphen{}short portfolio UMD.
What are the best and worst months for portfolios 1, 10, and UMD?

\begin{sphinxuseclass}{cell}\begin{sphinxVerbatimInput}

\begin{sphinxuseclass}{cell_input}
\begin{sphinxVerbatim}[commandchars=\\\{\}]
\PYG{n}{mom\PYGZus{}ew} \PYG{o}{=} \PYG{n}{mom\PYGZus{}ew}\PYG{o}{.}\PYG{n}{assign}\PYG{p}{(}\PYG{n}{UMD}\PYG{o}{=}\PYG{k}{lambda} \PYG{n}{x}\PYG{p}{:} \PYG{n}{x}\PYG{p}{[}\PYG{l+s+s1}{\PYGZsq{}}\PYG{l+s+s1}{Mom 10}\PYG{l+s+s1}{\PYGZsq{}}\PYG{p}{]} \PYG{o}{\PYGZhy{}} \PYG{n}{x}\PYG{p}{[}\PYG{l+s+s1}{\PYGZsq{}}\PYG{l+s+s1}{Mom 1}\PYG{l+s+s1}{\PYGZsq{}}\PYG{p}{]}\PYG{p}{)}
\end{sphinxVerbatim}

\end{sphinxuseclass}\end{sphinxVerbatimInput}

\end{sphinxuseclass}
\sphinxAtStartPar
The worst month for the long\sphinxhyphen{}short portfolio is during the Great Depression.

\begin{sphinxuseclass}{cell}\begin{sphinxVerbatimInput}

\begin{sphinxuseclass}{cell_input}
\begin{sphinxVerbatim}[commandchars=\\\{\}]
\PYG{n}{mom\PYGZus{}ew}\PYG{p}{[}\PYG{l+s+s1}{\PYGZsq{}}\PYG{l+s+s1}{UMD}\PYG{l+s+s1}{\PYGZsq{}}\PYG{p}{]}\PYG{o}{.}\PYG{n}{idxmin}\PYG{p}{(}\PYG{p}{)}
\end{sphinxVerbatim}

\end{sphinxuseclass}\end{sphinxVerbatimInput}
\begin{sphinxVerbatimOutput}

\begin{sphinxuseclass}{cell_output}
\begin{sphinxVerbatim}[commandchars=\\\{\}]
Period(\PYGZsq{}1932\PYGZhy{}08\PYGZsq{}, \PYGZsq{}M\PYGZsq{})
\end{sphinxVerbatim}

\end{sphinxuseclass}\end{sphinxVerbatimOutput}

\end{sphinxuseclass}
\sphinxAtStartPar
Momentum strategies appear to generate large absolute returns.
However, long\sphinxhyphen{}short momentum strategies are occasionally wiped out by large market recoveries, that the short side of the portfolio does not survive in practice.
During downturns, the short side of the portfolio (i.e., the loser stocks) has a high beta.
When the market unexpectedly and quickly recovers, the exaggerated recovery in the high\sphinxhyphen{}beta stocks wipes out the long\sphinxhyphen{}short portfolio.
For example, if we invest \$1 in the long\sphinxhyphen{}short portfolio at the end of 1929, the long\sphinxhyphen{}short portfolio is effectively wiped out during August 1932.

\begin{sphinxuseclass}{cell}\begin{sphinxVerbatimInput}

\begin{sphinxuseclass}{cell_input}
\begin{sphinxVerbatim}[commandchars=\\\{\}]
\PYG{n}{mom\PYGZus{}ew}\PYG{o}{.}\PYG{n}{loc}\PYG{p}{[}\PYG{l+s+s1}{\PYGZsq{}}\PYG{l+s+s1}{1930}\PYG{l+s+s1}{\PYGZsq{}}\PYG{p}{:}\PYG{l+s+s1}{\PYGZsq{}}\PYG{l+s+s1}{1932}\PYG{l+s+s1}{\PYGZsq{}}\PYG{p}{,} \PYG{p}{[}\PYG{l+s+s1}{\PYGZsq{}}\PYG{l+s+s1}{Mom 1}\PYG{l+s+s1}{\PYGZsq{}}\PYG{p}{,} \PYG{l+s+s1}{\PYGZsq{}}\PYG{l+s+s1}{Mom 10}\PYG{l+s+s1}{\PYGZsq{}}\PYG{p}{,} \PYG{l+s+s1}{\PYGZsq{}}\PYG{l+s+s1}{UMD}\PYG{l+s+s1}{\PYGZsq{}}\PYG{p}{]}\PYG{p}{]}\PYG{o}{.}\PYG{n}{add}\PYG{p}{(}\PYG{l+m+mi}{1}\PYG{p}{)}\PYG{o}{.}\PYG{n}{cumprod}\PYG{p}{(}\PYG{p}{)}\PYG{o}{.}\PYG{n}{plot}\PYG{p}{(}\PYG{p}{)}
\PYG{n}{plt}\PYG{o}{.}\PYG{n}{ylabel}\PYG{p}{(}\PYG{l+s+s1}{\PYGZsq{}}\PYG{l+s+s1}{Value of \PYGZdl{}1 Investment (\PYGZdl{})}\PYG{l+s+s1}{\PYGZsq{}}\PYG{p}{)}
\PYG{n}{plt}\PYG{o}{.}\PYG{n}{title}\PYG{p}{(}\PYG{l+s+s1}{\PYGZsq{}}\PYG{l+s+s1}{Momentum Strategy}\PYG{l+s+se}{\PYGZbs{}n}\PYG{l+s+s1}{ Equal\PYGZhy{}Weighted Portfolios}\PYG{l+s+se}{\PYGZbs{}n}\PYG{l+s+s1}{Formed on Months \PYGZhy{}12 to \PYGZhy{}2}\PYG{l+s+s1}{\PYGZsq{}}\PYG{p}{)}
\PYG{n}{plt}\PYG{o}{.}\PYG{n}{show}\PYG{p}{(}\PYG{p}{)}
\end{sphinxVerbatim}

\end{sphinxuseclass}\end{sphinxVerbatimInput}
\begin{sphinxVerbatimOutput}

\begin{sphinxuseclass}{cell_output}
\noindent\sphinxincludegraphics{{4162e1d75dbe30fb44095151c1aa1734e90e085b17d4b8236c0f127dd64e09f2}.png}

\end{sphinxuseclass}\end{sphinxVerbatimOutput}

\end{sphinxuseclass}

\subsubsection{What are the CAPM and FF4 alphas for these equal\sphinxhyphen{}weighted portfolios?}
\label{\detokenize{herron_06_practice_04:what-are-the-capm-and-ff4-alphas-for-these-equal-weighted-portfolios}}
\begin{sphinxuseclass}{cell}\begin{sphinxVerbatimInput}

\begin{sphinxuseclass}{cell_input}
\begin{sphinxVerbatim}[commandchars=\\\{\}]
\PYG{k+kn}{import} \PYG{n+nn}{statsmodels}\PYG{n+nn}{.}\PYG{n+nn}{formula}\PYG{n+nn}{.}\PYG{n+nn}{api} \PYG{k}{as} \PYG{n+nn}{smf}
\end{sphinxVerbatim}

\end{sphinxuseclass}\end{sphinxVerbatimInput}

\end{sphinxuseclass}
\sphinxAtStartPar
\sphinxstyleemphasis{\sphinxstylestrong{After class, I made the following code more repeatable.}}

\begin{sphinxuseclass}{cell}\begin{sphinxVerbatimInput}

\begin{sphinxuseclass}{cell_input}
\begin{sphinxVerbatim}[commandchars=\\\{\}]
\PYG{k}{def} \PYG{n+nf}{capm}\PYG{p}{(}\PYG{n}{df}\PYG{p}{,} \PYG{n}{c}\PYG{p}{)}\PYG{p}{:}
    \PYG{k}{return} \PYG{n}{smf}\PYG{o}{.}\PYG{n}{ols}\PYG{p}{(}\PYG{n}{formula}\PYG{o}{=}\PYG{l+s+sa}{f}\PYG{l+s+s1}{\PYGZsq{}}\PYG{l+s+s1}{I(Q(}\PYG{l+s+s1}{\PYGZdq{}}\PYG{l+s+si}{\PYGZob{}}\PYG{n}{c}\PYG{l+s+si}{\PYGZcb{}}\PYG{l+s+s1}{\PYGZdq{}}\PYG{l+s+s1}{)\PYGZhy{}RF) \PYGZti{} Q(}\PYG{l+s+s1}{\PYGZdq{}}\PYG{l+s+s1}{Mkt\PYGZhy{}RF}\PYG{l+s+s1}{\PYGZdq{}}\PYG{l+s+s1}{)}\PYG{l+s+s1}{\PYGZsq{}}\PYG{p}{,} \PYG{n}{data}\PYG{o}{=}\PYG{n}{df}\PYG{p}{)}
\end{sphinxVerbatim}

\end{sphinxuseclass}\end{sphinxVerbatimInput}

\end{sphinxuseclass}
\begin{sphinxuseclass}{cell}\begin{sphinxVerbatimInput}

\begin{sphinxuseclass}{cell_input}
\begin{sphinxVerbatim}[commandchars=\\\{\}]
\PYG{n}{ff\PYGZus{}0} \PYG{o}{=} \PYG{n}{pdr}\PYG{o}{.}\PYG{n}{DataReader}\PYG{p}{(}
    \PYG{n}{name}\PYG{o}{=}\PYG{l+s+s1}{\PYGZsq{}}\PYG{l+s+s1}{F\PYGZhy{}F\PYGZus{}Research\PYGZus{}Data\PYGZus{}Factors}\PYG{l+s+s1}{\PYGZsq{}}\PYG{p}{,}
    \PYG{n}{data\PYGZus{}source}\PYG{o}{=}\PYG{l+s+s1}{\PYGZsq{}}\PYG{l+s+s1}{famafrench}\PYG{l+s+s1}{\PYGZsq{}}\PYG{p}{,}
    \PYG{n}{start}\PYG{o}{=}\PYG{l+s+s1}{\PYGZsq{}}\PYG{l+s+s1}{1900}\PYG{l+s+s1}{\PYGZsq{}}\PYG{p}{,}
    \PYG{n}{session}\PYG{o}{=}\PYG{n}{session}
\PYG{p}{)}
\end{sphinxVerbatim}

\end{sphinxuseclass}\end{sphinxVerbatimInput}

\end{sphinxuseclass}
\begin{sphinxuseclass}{cell}\begin{sphinxVerbatimInput}

\begin{sphinxuseclass}{cell_input}
\begin{sphinxVerbatim}[commandchars=\\\{\}]
\PYG{n}{ff\PYGZus{}mom} \PYG{o}{=} \PYG{n}{pdr}\PYG{o}{.}\PYG{n}{DataReader}\PYG{p}{(}
    \PYG{n}{name}\PYG{o}{=}\PYG{l+s+s1}{\PYGZsq{}}\PYG{l+s+s1}{F\PYGZhy{}F\PYGZus{}Momentum\PYGZus{}Factor}\PYG{l+s+s1}{\PYGZsq{}}\PYG{p}{,}
    \PYG{n}{data\PYGZus{}source}\PYG{o}{=}\PYG{l+s+s1}{\PYGZsq{}}\PYG{l+s+s1}{famafrench}\PYG{l+s+s1}{\PYGZsq{}}\PYG{p}{,}
    \PYG{n}{start}\PYG{o}{=}\PYG{l+s+s1}{\PYGZsq{}}\PYG{l+s+s1}{1900}\PYG{l+s+s1}{\PYGZsq{}}\PYG{p}{,}
    \PYG{n}{session}\PYG{o}{=}\PYG{n}{session}
\PYG{p}{)}

\PYG{n}{ff\PYGZus{}mom}\PYG{p}{[}\PYG{l+m+mi}{0}\PYG{p}{]}\PYG{o}{.}\PYG{n}{columns} \PYG{o}{=} \PYG{p}{[}\PYG{l+s+s1}{\PYGZsq{}}\PYG{l+s+s1}{Mom}\PYG{l+s+s1}{\PYGZsq{}}\PYG{p}{]} \PYG{c+c1}{\PYGZsh{} we need to rename the Mom factor to remove leading and trailing whitespace}
\end{sphinxVerbatim}

\end{sphinxuseclass}\end{sphinxVerbatimInput}

\end{sphinxuseclass}
\sphinxAtStartPar
The \sphinxcode{\sphinxupquote{get\_coefs()}} function accepts a data frame of returns and factors data, a factor model regression function name, and number \$n\$ for the first \$n\$ columns in the data frame.
The factor model regression function must return a statsmodel model.
The \sphinxcode{\sphinxupquote{get\_coefs()}} function returns a data frame of coefficient estimates and standard errors.
We can use the \sphinxcode{\sphinxupquote{get\_coefs()}} function to quickly apply a factor model regression function.

\begin{sphinxuseclass}{cell}\begin{sphinxVerbatimInput}

\begin{sphinxuseclass}{cell_input}
\begin{sphinxVerbatim}[commandchars=\\\{\}]
\PYG{k}{def} \PYG{n+nf}{get\PYGZus{}coefs}\PYG{p}{(}\PYG{n}{df}\PYG{p}{,} \PYG{n}{fun}\PYG{p}{,} \PYG{n}{n}\PYG{p}{)}\PYG{p}{:}
    \PYG{n}{models} \PYG{o}{=} \PYG{p}{[}\PYG{n}{df}\PYG{o}{.}\PYG{n}{pipe}\PYG{p}{(}\PYG{n}{fun}\PYG{p}{,} \PYG{n}{c}\PYG{p}{)} \PYG{k}{for} \PYG{n}{c} \PYG{o+ow}{in} \PYG{n}{df}\PYG{o}{.}\PYG{n}{columns}\PYG{p}{[}\PYG{p}{:}\PYG{n}{n}\PYG{p}{]}\PYG{p}{]}
    \PYG{n}{fits} \PYG{o}{=} \PYG{p}{[}\PYG{n}{m}\PYG{o}{.}\PYG{n}{fit}\PYG{p}{(}\PYG{p}{)} \PYG{k}{for} \PYG{n}{m} \PYG{o+ow}{in} \PYG{n}{models}\PYG{p}{]}
    \PYG{n}{coefs} \PYG{o}{=} \PYG{p}{(}
        \PYG{n}{pd}\PYG{o}{.}\PYG{n}{concat}\PYG{p}{(}
            \PYG{n}{objs}\PYG{o}{=}\PYG{p}{[}\PYG{n}{f}\PYG{o}{.}\PYG{n}{params} \PYG{k}{for} \PYG{n}{f} \PYG{o+ow}{in} \PYG{n}{fits}\PYG{p}{]}\PYG{p}{,}
            \PYG{n}{axis}\PYG{o}{=}\PYG{l+m+mi}{1}\PYG{p}{,}
            \PYG{n}{keys}\PYG{o}{=}\PYG{n}{df}\PYG{o}{.}\PYG{n}{columns}\PYG{p}{[}\PYG{p}{:}\PYG{l+m+mi}{10}\PYG{p}{]}
        \PYG{p}{)}
        \PYG{o}{.}\PYG{n}{transpose}\PYG{p}{(}\PYG{p}{)}
        \PYG{o}{.}\PYG{n}{rename\PYGZus{}axis}\PYG{p}{(}\PYG{n}{index}\PYG{o}{=}\PYG{l+s+s1}{\PYGZsq{}}\PYG{l+s+s1}{Portfolio}\PYG{l+s+s1}{\PYGZsq{}}\PYG{p}{,} \PYG{n}{columns}\PYG{o}{=}\PYG{l+s+s1}{\PYGZsq{}}\PYG{l+s+s1}{Variable}\PYG{l+s+s1}{\PYGZsq{}}\PYG{p}{)}
    \PYG{p}{)}

    \PYG{n}{ses} \PYG{o}{=} \PYG{p}{(}
        \PYG{n}{pd}\PYG{o}{.}\PYG{n}{concat}\PYG{p}{(}
            \PYG{n}{objs}\PYG{o}{=}\PYG{p}{[}\PYG{n}{f}\PYG{o}{.}\PYG{n}{bse} \PYG{k}{for} \PYG{n}{f} \PYG{o+ow}{in} \PYG{n}{fits}\PYG{p}{]}\PYG{p}{,}
            \PYG{n}{axis}\PYG{o}{=}\PYG{l+m+mi}{1}\PYG{p}{,}
            \PYG{n}{keys}\PYG{o}{=}\PYG{n}{df}\PYG{o}{.}\PYG{n}{columns}\PYG{p}{[}\PYG{p}{:}\PYG{l+m+mi}{10}\PYG{p}{]}
        \PYG{p}{)}
        \PYG{o}{.}\PYG{n}{transpose}\PYG{p}{(}\PYG{p}{)}
        \PYG{o}{.}\PYG{n}{rename\PYGZus{}axis}\PYG{p}{(}\PYG{n}{index}\PYG{o}{=}\PYG{l+s+s1}{\PYGZsq{}}\PYG{l+s+s1}{Portfolio}\PYG{l+s+s1}{\PYGZsq{}}\PYG{p}{,} \PYG{n}{columns}\PYG{o}{=}\PYG{l+s+s1}{\PYGZsq{}}\PYG{l+s+s1}{Variable}\PYG{l+s+s1}{\PYGZsq{}}\PYG{p}{)}
    \PYG{p}{)}

    \PYG{k}{return} \PYG{n}{pd}\PYG{o}{.}\PYG{n}{concat}\PYG{p}{(}
        \PYG{n}{objs}\PYG{o}{=}\PYG{p}{[}\PYG{n}{coefs}\PYG{p}{,} \PYG{n}{ses}\PYG{p}{]}\PYG{p}{,} 
        \PYG{n}{keys}\PYG{o}{=}\PYG{p}{[}\PYG{l+s+s1}{\PYGZsq{}}\PYG{l+s+s1}{Coef}\PYG{l+s+s1}{\PYGZsq{}}\PYG{p}{,} \PYG{l+s+s1}{\PYGZsq{}}\PYG{l+s+s1}{SE}\PYG{l+s+s1}{\PYGZsq{}}\PYG{p}{]}\PYG{p}{,} 
        \PYG{n}{names}\PYG{o}{=}\PYG{p}{[}\PYG{l+s+s1}{\PYGZsq{}}\PYG{l+s+s1}{Statistic}\PYG{l+s+s1}{\PYGZsq{}}\PYG{p}{]}\PYG{p}{,} 
        \PYG{n}{axis}\PYG{o}{=}\PYG{l+m+mi}{1}
    \PYG{p}{)}
\end{sphinxVerbatim}

\end{sphinxuseclass}\end{sphinxVerbatimInput}

\end{sphinxuseclass}
\sphinxAtStartPar
The \sphinxcode{\sphinxupquote{plot\_alpha()}} function accepts the output of the \sphinxcode{\sphinxupquote{get\_coefs()}} function and plots portfolio alphas.

\begin{sphinxuseclass}{cell}\begin{sphinxVerbatimInput}

\begin{sphinxuseclass}{cell_input}
\begin{sphinxVerbatim}[commandchars=\\\{\}]
\PYG{k}{def} \PYG{n+nf}{plot\PYGZus{}alpha}\PYG{p}{(}\PYG{n}{df}\PYG{p}{)}\PYG{p}{:}
    \PYG{n}{\PYGZus{}} \PYG{o}{=} \PYG{n}{df}\PYG{o}{.}\PYG{n}{swaplevel}\PYG{p}{(}\PYG{n}{axis}\PYG{o}{=}\PYG{l+m+mi}{1}\PYG{p}{)}\PYG{p}{[}\PYG{l+s+s1}{\PYGZsq{}}\PYG{l+s+s1}{Intercept}\PYG{l+s+s1}{\PYGZsq{}}\PYG{p}{]}
    \PYG{n}{\PYGZus{}}\PYG{p}{[}\PYG{l+s+s1}{\PYGZsq{}}\PYG{l+s+s1}{Coef}\PYG{l+s+s1}{\PYGZsq{}}\PYG{p}{]}\PYG{o}{.}\PYG{n}{plot}\PYG{p}{(}\PYG{n}{kind}\PYG{o}{=}\PYG{l+s+s1}{\PYGZsq{}}\PYG{l+s+s1}{bar}\PYG{l+s+s1}{\PYGZsq{}}\PYG{p}{,} \PYG{n}{yerr}\PYG{o}{=}\PYG{n}{\PYGZus{}}\PYG{p}{[}\PYG{l+s+s1}{\PYGZsq{}}\PYG{l+s+s1}{SE}\PYG{l+s+s1}{\PYGZsq{}}\PYG{p}{]}\PYG{p}{)}
    \PYG{n}{plt}\PYG{o}{.}\PYG{n}{ylabel}\PYG{p}{(}\PYG{l+s+s1}{\PYGZsq{}}\PYG{l+s+s1}{Monthly Alpha (}\PYG{l+s+s1}{\PYGZpc{}}\PYG{l+s+s1}{)}\PYG{l+s+s1}{\PYGZsq{}}\PYG{p}{)}
\end{sphinxVerbatim}

\end{sphinxuseclass}\end{sphinxVerbatimInput}

\end{sphinxuseclass}
\sphinxAtStartPar
We can look at the output of \sphinxcode{\sphinxupquote{get\_coefs()}}, but I typically will chain these functions.

\begin{sphinxuseclass}{cell}\begin{sphinxVerbatimInput}

\begin{sphinxuseclass}{cell_input}
\begin{sphinxVerbatim}[commandchars=\\\{\}]
\PYG{n}{mom\PYGZus{}ew}\PYG{o}{.}\PYG{n}{mul}\PYG{p}{(}\PYG{l+m+mi}{100}\PYG{p}{)}\PYG{o}{.}\PYG{n}{join}\PYG{p}{(}\PYG{n}{ff\PYGZus{}0}\PYG{p}{[}\PYG{l+m+mi}{0}\PYG{p}{]}\PYG{p}{)}\PYG{o}{.}\PYG{n}{pipe}\PYG{p}{(}\PYG{n}{get\PYGZus{}coefs}\PYG{p}{,} \PYG{n}{fun}\PYG{o}{=}\PYG{n}{capm}\PYG{p}{,} \PYG{n}{n}\PYG{o}{=}\PYG{l+m+mi}{10}\PYG{p}{)}
\end{sphinxVerbatim}

\end{sphinxuseclass}\end{sphinxVerbatimInput}
\begin{sphinxVerbatimOutput}

\begin{sphinxuseclass}{cell_output}
\begin{sphinxVerbatim}[commandchars=\\\{\}]
Statistic      Coef                    SE            
Variable  Intercept Q(\PYGZdq{}Mkt\PYGZhy{}RF\PYGZdq{}) Intercept Q(\PYGZdq{}Mkt\PYGZhy{}RF\PYGZdq{})
Portfolio                                            
Mom 1       \PYGZhy{}0.2642      1.6488    0.2364      0.0438
Mom 2       \PYGZhy{}0.2485      1.4625    0.1550      0.0287
Mom 3       \PYGZhy{}0.1499      1.3130    0.1170      0.0216
Mom 4        0.0747      1.2729    0.1071      0.0198
Mom 5        0.1002      1.1657    0.0874      0.0162
Mom 6        0.2728      1.1267    0.0786      0.0146
Mom 7        0.3420      1.0899    0.0778      0.0144
Mom 8        0.4860      1.0698    0.0794      0.0147
Mom 9        0.5828      1.0881    0.0882      0.0163
Mom 10       0.7466      1.1711    0.1273      0.0236
\end{sphinxVerbatim}

\end{sphinxuseclass}\end{sphinxVerbatimOutput}

\end{sphinxuseclass}
\sphinxAtStartPar
The beauty of chaining these functions is:
\begin{enumerate}
\sphinxsetlistlabels{\arabic}{enumi}{enumii}{}{.}%
\item {} 
\sphinxAtStartPar
We have all calculations in one cell, eliminating the need to track many “off screen” calculations in other cells

\item {} 
\sphinxAtStartPar
We do not create intermediate data frames, eliminating the need to name and track many data frames

\end{enumerate}

\begin{sphinxuseclass}{cell}\begin{sphinxVerbatimInput}

\begin{sphinxuseclass}{cell_input}
\begin{sphinxVerbatim}[commandchars=\\\{\}]
\PYG{n}{mom\PYGZus{}ew}\PYG{o}{.}\PYG{n}{mul}\PYG{p}{(}\PYG{l+m+mi}{100}\PYG{p}{)}\PYG{o}{.}\PYG{n}{join}\PYG{p}{(}\PYG{n}{ff\PYGZus{}0}\PYG{p}{[}\PYG{l+m+mi}{0}\PYG{p}{]}\PYG{p}{)}\PYG{o}{.}\PYG{n}{pipe}\PYG{p}{(}\PYG{n}{get\PYGZus{}coefs}\PYG{p}{,} \PYG{n}{fun}\PYG{o}{=}\PYG{n}{capm}\PYG{p}{,} \PYG{n}{n}\PYG{o}{=}\PYG{l+m+mi}{10}\PYG{p}{)}\PYG{o}{.}\PYG{n}{pipe}\PYG{p}{(}\PYG{n}{plot\PYGZus{}alpha}\PYG{p}{)}
\PYG{n}{plt}\PYG{o}{.}\PYG{n}{title}\PYG{p}{(}
    \PYG{l+s+s1}{\PYGZsq{}}\PYG{l+s+s1}{CAPM Tests of Momentum Strategy}\PYG{l+s+s1}{\PYGZsq{}} \PYG{o}{+}
    \PYG{l+s+s1}{\PYGZsq{}}\PYG{l+s+se}{\PYGZbs{}n}\PYG{l+s+s1}{Equal\PYGZhy{}Weighted Portfolios Formed on Months \PYGZhy{}12 to \PYGZhy{}2}\PYG{l+s+s1}{\PYGZsq{}} \PYG{o}{+}
    \PYG{l+s+s1}{\PYGZsq{}}\PYG{l+s+se}{\PYGZbs{}n}\PYG{l+s+s1}{Black Vertical Bars Indicate Standard Errors}\PYG{l+s+s1}{\PYGZsq{}}
\PYG{p}{)}
\PYG{n}{plt}\PYG{o}{.}\PYG{n}{show}\PYG{p}{(}\PYG{p}{)}
\end{sphinxVerbatim}

\end{sphinxuseclass}\end{sphinxVerbatimInput}
\begin{sphinxVerbatimOutput}

\begin{sphinxuseclass}{cell_output}
\noindent\sphinxincludegraphics{{5e1e913e1de9ed0d78aede753ac097b65b1979e747744da61ed4e7ca405d0adc}.png}

\end{sphinxuseclass}\end{sphinxVerbatimOutput}

\end{sphinxuseclass}
\sphinxAtStartPar
The momentum strategy has a large, positive, statistically significant alpha, suggesting it is a source of risk\sphinxhyphen{}adjusted returns (i.e., returns not associated with risk).
However, we saw above that momentum occasionally crashes, suggesting that the momentum strategy has risk.
We will investigate this with the Fama\sphinxhyphen{}French Four\sphinxhyphen{}Factor model, which add size, value, and momentum factors.
Once we consider SMB, HML, and Mom risk factors, we see:
\begin{enumerate}
\sphinxsetlistlabels{\arabic}{enumi}{enumii}{}{.}%
\item {} 
\sphinxAtStartPar
The relation between returns and momentum portfolios is much weaker and not monotonic

\item {} 
\sphinxAtStartPar
The statistical significance of the alpha coefficient estimates is much lower.

\end{enumerate}

\begin{sphinxuseclass}{cell}\begin{sphinxVerbatimInput}

\begin{sphinxuseclass}{cell_input}
\begin{sphinxVerbatim}[commandchars=\\\{\}]
\PYG{k}{def} \PYG{n+nf}{ff4}\PYG{p}{(}\PYG{n}{df}\PYG{p}{,} \PYG{n}{c}\PYG{p}{)}\PYG{p}{:}
    \PYG{k}{return} \PYG{n}{smf}\PYG{o}{.}\PYG{n}{ols}\PYG{p}{(}\PYG{n}{formula}\PYG{o}{=}\PYG{l+s+sa}{f}\PYG{l+s+s1}{\PYGZsq{}}\PYG{l+s+s1}{I(Q(}\PYG{l+s+s1}{\PYGZdq{}}\PYG{l+s+si}{\PYGZob{}}\PYG{n}{c}\PYG{l+s+si}{\PYGZcb{}}\PYG{l+s+s1}{\PYGZdq{}}\PYG{l+s+s1}{)\PYGZhy{}RF) \PYGZti{} Q(}\PYG{l+s+s1}{\PYGZdq{}}\PYG{l+s+s1}{Mkt\PYGZhy{}RF}\PYG{l+s+s1}{\PYGZdq{}}\PYG{l+s+s1}{) + SMB + HML + Mom}\PYG{l+s+s1}{\PYGZsq{}}\PYG{p}{,} \PYG{n}{data}\PYG{o}{=}\PYG{n}{df}\PYG{p}{)}
\end{sphinxVerbatim}

\end{sphinxuseclass}\end{sphinxVerbatimInput}

\end{sphinxuseclass}
\begin{sphinxuseclass}{cell}\begin{sphinxVerbatimInput}

\begin{sphinxuseclass}{cell_input}
\begin{sphinxVerbatim}[commandchars=\\\{\}]
\PYG{n}{mom\PYGZus{}ew}\PYG{o}{.}\PYG{n}{mul}\PYG{p}{(}\PYG{l+m+mi}{100}\PYG{p}{)}\PYG{o}{.}\PYG{n}{join}\PYG{p}{(}\PYG{p}{[}\PYG{n}{ff\PYGZus{}0}\PYG{p}{[}\PYG{l+m+mi}{0}\PYG{p}{]}\PYG{p}{,} \PYG{n}{ff\PYGZus{}mom}\PYG{p}{[}\PYG{l+m+mi}{0}\PYG{p}{]}\PYG{p}{]}\PYG{p}{)}\PYG{o}{.}\PYG{n}{pipe}\PYG{p}{(}\PYG{n}{get\PYGZus{}coefs}\PYG{p}{,} \PYG{n}{fun}\PYG{o}{=}\PYG{n}{ff4}\PYG{p}{,} \PYG{n}{n}\PYG{o}{=}\PYG{l+m+mi}{10}\PYG{p}{)}\PYG{o}{.}\PYG{n}{pipe}\PYG{p}{(}\PYG{n}{plot\PYGZus{}alpha}\PYG{p}{)}
\PYG{n}{plt}\PYG{o}{.}\PYG{n}{title}\PYG{p}{(}
    \PYG{l+s+s1}{\PYGZsq{}}\PYG{l+s+s1}{FF4 Tests of Momentum Strategy}\PYG{l+s+s1}{\PYGZsq{}} \PYG{o}{+}
    \PYG{l+s+s1}{\PYGZsq{}}\PYG{l+s+se}{\PYGZbs{}n}\PYG{l+s+s1}{Equal\PYGZhy{}Weighted Portfolios Formed on Months \PYGZhy{}12 to \PYGZhy{}2}\PYG{l+s+s1}{\PYGZsq{}} \PYG{o}{+}
    \PYG{l+s+s1}{\PYGZsq{}}\PYG{l+s+se}{\PYGZbs{}n}\PYG{l+s+s1}{Black Vertical Bars Indicate Standard Errors}\PYG{l+s+s1}{\PYGZsq{}}
\PYG{p}{)}
\PYG{n}{plt}\PYG{o}{.}\PYG{n}{show}\PYG{p}{(}\PYG{p}{)}
\end{sphinxVerbatim}

\end{sphinxuseclass}\end{sphinxVerbatimInput}
\begin{sphinxVerbatimOutput}

\begin{sphinxuseclass}{cell_output}
\noindent\sphinxincludegraphics{{49d58b0e8f4a15b5f64c08669380a620a5477c654dbd2bd39d8fe61718b6edb4}.png}

\end{sphinxuseclass}\end{sphinxVerbatimOutput}

\end{sphinxuseclass}
\sphinxAtStartPar
A more extreme test is to only consider the last 20 years of returns.

\begin{sphinxuseclass}{cell}\begin{sphinxVerbatimInput}

\begin{sphinxuseclass}{cell_input}
\begin{sphinxVerbatim}[commandchars=\\\{\}]
\PYG{n}{mom\PYGZus{}ew}\PYG{o}{.}\PYG{n}{mul}\PYG{p}{(}\PYG{l+m+mi}{100}\PYG{p}{)}\PYG{o}{.}\PYG{n}{join}\PYG{p}{(}\PYG{p}{[}\PYG{n}{ff\PYGZus{}0}\PYG{p}{[}\PYG{l+m+mi}{0}\PYG{p}{]}\PYG{p}{,} \PYG{n}{ff\PYGZus{}mom}\PYG{p}{[}\PYG{l+m+mi}{0}\PYG{p}{]}\PYG{p}{]}\PYG{p}{)}\PYG{o}{.}\PYG{n}{iloc}\PYG{p}{[}\PYG{o}{\PYGZhy{}}\PYG{l+m+mi}{240}\PYG{p}{:}\PYG{p}{]}\PYG{o}{.}\PYG{n}{pipe}\PYG{p}{(}\PYG{n}{get\PYGZus{}coefs}\PYG{p}{,} \PYG{n}{fun}\PYG{o}{=}\PYG{n}{ff4}\PYG{p}{,} \PYG{n}{n}\PYG{o}{=}\PYG{l+m+mi}{10}\PYG{p}{)}\PYG{o}{.}\PYG{n}{pipe}\PYG{p}{(}\PYG{n}{plot\PYGZus{}alpha}\PYG{p}{)}
\PYG{n}{plt}\PYG{o}{.}\PYG{n}{title}\PYG{p}{(}
    \PYG{l+s+s1}{\PYGZsq{}}\PYG{l+s+s1}{FF4 Tests of Momentum Strategy}\PYG{l+s+s1}{\PYGZsq{}} \PYG{o}{+}
    \PYG{l+s+s1}{\PYGZsq{}}\PYG{l+s+se}{\PYGZbs{}n}\PYG{l+s+s1}{Equal\PYGZhy{}Weighted Portfolios Formed on Months \PYGZhy{}12 to \PYGZhy{}2}\PYG{l+s+s1}{\PYGZsq{}} \PYG{o}{+}
    \PYG{l+s+s1}{\PYGZsq{}}\PYG{l+s+se}{\PYGZbs{}n}\PYG{l+s+s1}{Black Vertical Bars Indicate Standard Errors}\PYG{l+s+s1}{\PYGZsq{}}
\PYG{p}{)}
\PYG{n}{plt}\PYG{o}{.}\PYG{n}{show}\PYG{p}{(}\PYG{p}{)}
\end{sphinxVerbatim}

\end{sphinxuseclass}\end{sphinxVerbatimInput}
\begin{sphinxVerbatimOutput}

\begin{sphinxuseclass}{cell_output}
\noindent\sphinxincludegraphics{{61e94dbb15479ee32c7ad18ef05875f96952d508778d361dacb6f9ec0d32fdaa}.png}

\end{sphinxuseclass}\end{sphinxVerbatimOutput}

\end{sphinxuseclass}

\subsubsection{What are the Sharpe Ratios on these 11 portfolios?}
\label{\detokenize{herron_06_practice_04:what-are-the-sharpe-ratios-on-these-11-portfolios}}
\begin{sphinxuseclass}{cell}\begin{sphinxVerbatimInput}

\begin{sphinxuseclass}{cell_input}
\begin{sphinxVerbatim}[commandchars=\\\{\}]
\PYG{k}{def} \PYG{n+nf}{sharpe}\PYG{p}{(}\PYG{n}{r}\PYG{p}{,} \PYG{n}{tgt}\PYG{p}{,} \PYG{n}{ppy}\PYG{p}{)}\PYG{p}{:}
    \PYG{n}{er} \PYG{o}{=} \PYG{n}{r}\PYG{o}{.}\PYG{n}{sub}\PYG{p}{(}\PYG{n}{tgt}\PYG{p}{)}\PYG{o}{.}\PYG{n}{dropna}\PYG{p}{(}\PYG{p}{)}
    \PYG{k}{return} \PYG{n}{np}\PYG{o}{.}\PYG{n}{sqrt}\PYG{p}{(}\PYG{n}{ppy}\PYG{p}{)} \PYG{o}{*} \PYG{n}{er}\PYG{o}{.}\PYG{n}{mean}\PYG{p}{(}\PYG{p}{)} \PYG{o}{/} \PYG{n}{er}\PYG{o}{.}\PYG{n}{std}\PYG{p}{(}\PYG{p}{)}
\end{sphinxVerbatim}

\end{sphinxuseclass}\end{sphinxVerbatimInput}

\end{sphinxuseclass}
\begin{sphinxuseclass}{cell}\begin{sphinxVerbatimInput}

\begin{sphinxuseclass}{cell_input}
\begin{sphinxVerbatim}[commandchars=\\\{\}]
\PYG{n}{mom\PYGZus{}ew}\PYG{o}{.}\PYG{n}{mul}\PYG{p}{(}\PYG{l+m+mi}{100}\PYG{p}{)}\PYG{o}{.}\PYG{n}{apply}\PYG{p}{(}\PYG{n}{sharpe}\PYG{p}{,} \PYG{n}{tgt}\PYG{o}{=}\PYG{n}{ff\PYGZus{}0}\PYG{p}{[}\PYG{l+m+mi}{0}\PYG{p}{]}\PYG{p}{[}\PYG{l+s+s1}{\PYGZsq{}}\PYG{l+s+s1}{RF}\PYG{l+s+s1}{\PYGZsq{}}\PYG{p}{]}\PYG{p}{,} \PYG{n}{ppy}\PYG{o}{=}\PYG{l+m+mi}{12}\PYG{p}{)}\PYG{o}{.}\PYG{n}{plot}\PYG{p}{(}\PYG{n}{kind}\PYG{o}{=}\PYG{l+s+s1}{\PYGZsq{}}\PYG{l+s+s1}{bar}\PYG{l+s+s1}{\PYGZsq{}}\PYG{p}{)}
\PYG{n}{plt}\PYG{o}{.}\PYG{n}{ylabel}\PYG{p}{(}\PYG{l+s+s1}{\PYGZsq{}}\PYG{l+s+s1}{Sharpe Ratio}\PYG{l+s+s1}{\PYGZsq{}}\PYG{p}{)}
\PYG{n}{plt}\PYG{o}{.}\PYG{n}{title}\PYG{p}{(}
    \PYG{l+s+s1}{\PYGZsq{}}\PYG{l+s+s1}{Sharpe Ratios for Momentum Strategy Relative to RF}\PYG{l+s+s1}{\PYGZsq{}} \PYG{o}{+}
    \PYG{l+s+s1}{\PYGZsq{}}\PYG{l+s+se}{\PYGZbs{}n}\PYG{l+s+s1}{Equal\PYGZhy{}Weighted Portfolios Formed on Months \PYGZhy{}12 to \PYGZhy{}2}\PYG{l+s+s1}{\PYGZsq{}}
\PYG{p}{)}
\PYG{n}{plt}\PYG{o}{.}\PYG{n}{show}\PYG{p}{(}\PYG{p}{)}
\end{sphinxVerbatim}

\end{sphinxuseclass}\end{sphinxVerbatimInput}
\begin{sphinxVerbatimOutput}

\begin{sphinxuseclass}{cell_output}
\noindent\sphinxincludegraphics{{e3f863714a8690aab96231d2252fbc1266e1eb8df9a55c92216420269382a89e}.png}

\end{sphinxuseclass}\end{sphinxVerbatimOutput}

\end{sphinxuseclass}

\subsubsection{Implement a value\sphinxhyphen{}weighted momentum investing strategy}
\label{\detokenize{herron_06_practice_04:implement-a-value-weighted-momentum-investing-strategy}}
\sphinxAtStartPar
Assign this strategy to data frame \sphinxcode{\sphinxupquote{mom\_vw}}, and include long\sphinxhyphen{}short portfolio UMD

\sphinxAtStartPar
We can replace the \sphinxcode{\sphinxupquote{.mean()}} method, which calculates simple means (i.e., equal\sphinxhyphen{}weighted means), with the \sphinxcode{\sphinxupquote{.apply()}} method and \sphinxcode{\sphinxupquote{np.average()}} function, which has a \sphinxcode{\sphinxupquote{weights=}} argument and calculates weighted means.
We weight portfolio returns by the beginning\sphinxhyphen{}of\sphinxhyphen{}month marker value of equity in the \sphinxcode{\sphinxupquote{Trailing ME}} column.

\begin{sphinxuseclass}{cell}\begin{sphinxVerbatimInput}

\begin{sphinxuseclass}{cell_input}
\begin{sphinxVerbatim}[commandchars=\\\{\}]
\PYG{n}{mom\PYGZus{}vw} \PYG{o}{=} \PYG{p}{(}
    \PYG{n}{mom}
    \PYG{o}{.}\PYG{n}{groupby}\PYG{p}{(}\PYG{p}{[}\PYG{l+s+s1}{\PYGZsq{}}\PYG{l+s+s1}{date}\PYG{l+s+s1}{\PYGZsq{}}\PYG{p}{,} \PYG{l+s+s1}{\PYGZsq{}}\PYG{l+s+s1}{Portfolio}\PYG{l+s+s1}{\PYGZsq{}}\PYG{p}{]}\PYG{p}{)}
    \PYG{o}{.}\PYG{n}{apply}\PYG{p}{(}\PYG{k}{lambda} \PYG{n}{x}\PYG{p}{:} \PYG{n}{np}\PYG{o}{.}\PYG{n}{average}\PYG{p}{(}\PYG{n}{a}\PYG{o}{=}\PYG{n}{x}\PYG{p}{[}\PYG{l+s+s1}{\PYGZsq{}}\PYG{l+s+s1}{Return}\PYG{l+s+s1}{\PYGZsq{}}\PYG{p}{]}\PYG{p}{,} \PYG{n}{weights}\PYG{o}{=}\PYG{n}{x}\PYG{p}{[}\PYG{l+s+s1}{\PYGZsq{}}\PYG{l+s+s1}{Trailing ME}\PYG{l+s+s1}{\PYGZsq{}}\PYG{p}{]}\PYG{p}{)}\PYG{p}{)}
    \PYG{o}{.}\PYG{n}{unstack}\PYG{p}{(}\PYG{l+s+s1}{\PYGZsq{}}\PYG{l+s+s1}{Portfolio}\PYG{l+s+s1}{\PYGZsq{}}\PYG{p}{)}
    \PYG{o}{.}\PYG{n}{add\PYGZus{}prefix}\PYG{p}{(}\PYG{l+s+s1}{\PYGZsq{}}\PYG{l+s+s1}{Mom }\PYG{l+s+s1}{\PYGZsq{}}\PYG{p}{)}
\PYG{p}{)}
\end{sphinxVerbatim}

\end{sphinxuseclass}\end{sphinxVerbatimInput}

\end{sphinxuseclass}

\subsubsection{What are the CAPM and FF4 alphas for these value\sphinxhyphen{}weighted portfolios?}
\label{\detokenize{herron_06_practice_04:what-are-the-capm-and-ff4-alphas-for-these-value-weighted-portfolios}}
\sphinxAtStartPar
The alphas for the value\sphinxhyphen{}weighted momentum portfolios are generally smaller in magnitude than for the equal\sphinxhyphen{}weighted momentum portfolios.
Equal\sphinxhyphen{}weighted portfolios over\sphinxhyphen{}weight small stocks relative to value\sphinxhyphen{}weighted portfolios, and we expect most anomalies to be stronger in small stocks because small stocks are more difficult for institutional investors to  invest in at scale.

\begin{sphinxuseclass}{cell}\begin{sphinxVerbatimInput}

\begin{sphinxuseclass}{cell_input}
\begin{sphinxVerbatim}[commandchars=\\\{\}]
\PYG{n}{mom\PYGZus{}vw}\PYG{o}{.}\PYG{n}{mul}\PYG{p}{(}\PYG{l+m+mi}{100}\PYG{p}{)}\PYG{o}{.}\PYG{n}{join}\PYG{p}{(}\PYG{n}{ff\PYGZus{}0}\PYG{p}{[}\PYG{l+m+mi}{0}\PYG{p}{]}\PYG{p}{)}\PYG{o}{.}\PYG{n}{pipe}\PYG{p}{(}\PYG{n}{get\PYGZus{}coefs}\PYG{p}{,} \PYG{n}{fun}\PYG{o}{=}\PYG{n}{capm}\PYG{p}{,} \PYG{n}{n}\PYG{o}{=}\PYG{l+m+mi}{10}\PYG{p}{)}\PYG{o}{.}\PYG{n}{pipe}\PYG{p}{(}\PYG{n}{plot\PYGZus{}alpha}\PYG{p}{)}
\PYG{n}{plt}\PYG{o}{.}\PYG{n}{title}\PYG{p}{(}
    \PYG{l+s+s1}{\PYGZsq{}}\PYG{l+s+s1}{CAPM Tests of Momentum Strategy}\PYG{l+s+s1}{\PYGZsq{}} \PYG{o}{+}
    \PYG{l+s+s1}{\PYGZsq{}}\PYG{l+s+se}{\PYGZbs{}n}\PYG{l+s+s1}{Value\PYGZhy{}Weighted Portfolios Formed on Months \PYGZhy{}12 to \PYGZhy{}2}\PYG{l+s+s1}{\PYGZsq{}} \PYG{o}{+}
    \PYG{l+s+s1}{\PYGZsq{}}\PYG{l+s+se}{\PYGZbs{}n}\PYG{l+s+s1}{Black Vertical Bars Indicate Standard Errors}\PYG{l+s+s1}{\PYGZsq{}}
\PYG{p}{)}
\PYG{n}{plt}\PYG{o}{.}\PYG{n}{show}\PYG{p}{(}\PYG{p}{)}
\end{sphinxVerbatim}

\end{sphinxuseclass}\end{sphinxVerbatimInput}
\begin{sphinxVerbatimOutput}

\begin{sphinxuseclass}{cell_output}
\noindent\sphinxincludegraphics{{ced34c2a1d5da1a44cb646030a067d3e0a7d3f3faa565e7aa4f78e1952078890}.png}

\end{sphinxuseclass}\end{sphinxVerbatimOutput}

\end{sphinxuseclass}
\sphinxAtStartPar
The momentum strategy falls apart once we consider the Fama\sphinxhyphen{}French Four\sphinxhyphen{}Factor model with value\sphinxhyphen{}weighted portfolios.

\begin{sphinxuseclass}{cell}\begin{sphinxVerbatimInput}

\begin{sphinxuseclass}{cell_input}
\begin{sphinxVerbatim}[commandchars=\\\{\}]
\PYG{n}{mom\PYGZus{}vw}\PYG{o}{.}\PYG{n}{mul}\PYG{p}{(}\PYG{l+m+mi}{100}\PYG{p}{)}\PYG{o}{.}\PYG{n}{join}\PYG{p}{(}\PYG{p}{[}\PYG{n}{ff\PYGZus{}0}\PYG{p}{[}\PYG{l+m+mi}{0}\PYG{p}{]}\PYG{p}{,} \PYG{n}{ff\PYGZus{}mom}\PYG{p}{[}\PYG{l+m+mi}{0}\PYG{p}{]}\PYG{p}{]}\PYG{p}{)}\PYG{o}{.}\PYG{n}{pipe}\PYG{p}{(}\PYG{n}{get\PYGZus{}coefs}\PYG{p}{,} \PYG{n}{fun}\PYG{o}{=}\PYG{n}{ff4}\PYG{p}{,} \PYG{n}{n}\PYG{o}{=}\PYG{l+m+mi}{10}\PYG{p}{)}\PYG{o}{.}\PYG{n}{pipe}\PYG{p}{(}\PYG{n}{plot\PYGZus{}alpha}\PYG{p}{)}
\PYG{n}{plt}\PYG{o}{.}\PYG{n}{title}\PYG{p}{(}
    \PYG{l+s+s1}{\PYGZsq{}}\PYG{l+s+s1}{FF4 Tests of Momentum Strategy}\PYG{l+s+s1}{\PYGZsq{}} \PYG{o}{+}
    \PYG{l+s+s1}{\PYGZsq{}}\PYG{l+s+se}{\PYGZbs{}n}\PYG{l+s+s1}{Value\PYGZhy{}Weighted Portfolios Formed on Months \PYGZhy{}12 to \PYGZhy{}2}\PYG{l+s+s1}{\PYGZsq{}} \PYG{o}{+}
    \PYG{l+s+s1}{\PYGZsq{}}\PYG{l+s+se}{\PYGZbs{}n}\PYG{l+s+s1}{Black Vertical Bars Indicate Standard Errors}\PYG{l+s+s1}{\PYGZsq{}}
\PYG{p}{)}
\PYG{n}{plt}\PYG{o}{.}\PYG{n}{show}\PYG{p}{(}\PYG{p}{)}
\end{sphinxVerbatim}

\end{sphinxuseclass}\end{sphinxVerbatimInput}
\begin{sphinxVerbatimOutput}

\begin{sphinxuseclass}{cell_output}
\noindent\sphinxincludegraphics{{66e40b5dd92b6085cded9eeb7d2c959f470ca8560f6630174d8dc1235085e3d4}.png}

\end{sphinxuseclass}\end{sphinxVerbatimOutput}

\end{sphinxuseclass}

\subsubsection{What are the Sharpe Ratios for these value\sphinxhyphen{}weighted portfolios?}
\label{\detokenize{herron_06_practice_04:what-are-the-sharpe-ratios-for-these-value-weighted-portfolios}}
\begin{sphinxuseclass}{cell}\begin{sphinxVerbatimInput}

\begin{sphinxuseclass}{cell_input}
\begin{sphinxVerbatim}[commandchars=\\\{\}]
\PYG{n}{mom\PYGZus{}vw}\PYG{o}{.}\PYG{n}{mul}\PYG{p}{(}\PYG{l+m+mi}{100}\PYG{p}{)}\PYG{o}{.}\PYG{n}{apply}\PYG{p}{(}\PYG{n}{sharpe}\PYG{p}{,} \PYG{n}{tgt}\PYG{o}{=}\PYG{n}{ff\PYGZus{}0}\PYG{p}{[}\PYG{l+m+mi}{0}\PYG{p}{]}\PYG{p}{[}\PYG{l+s+s1}{\PYGZsq{}}\PYG{l+s+s1}{RF}\PYG{l+s+s1}{\PYGZsq{}}\PYG{p}{]}\PYG{p}{,} \PYG{n}{ppy}\PYG{o}{=}\PYG{l+m+mi}{12}\PYG{p}{)}\PYG{o}{.}\PYG{n}{plot}\PYG{p}{(}\PYG{n}{kind}\PYG{o}{=}\PYG{l+s+s1}{\PYGZsq{}}\PYG{l+s+s1}{bar}\PYG{l+s+s1}{\PYGZsq{}}\PYG{p}{)}
\PYG{n}{plt}\PYG{o}{.}\PYG{n}{ylabel}\PYG{p}{(}\PYG{l+s+s1}{\PYGZsq{}}\PYG{l+s+s1}{Sharpe Ratio}\PYG{l+s+s1}{\PYGZsq{}}\PYG{p}{)}
\PYG{n}{plt}\PYG{o}{.}\PYG{n}{title}\PYG{p}{(}
    \PYG{l+s+s1}{\PYGZsq{}}\PYG{l+s+s1}{Sharpe Ratios for Momentum Strategy Relative to RF}\PYG{l+s+s1}{\PYGZsq{}} \PYG{o}{+}
    \PYG{l+s+s1}{\PYGZsq{}}\PYG{l+s+se}{\PYGZbs{}n}\PYG{l+s+s1}{Value\PYGZhy{}Weighted Portfolios Formed on Months \PYGZhy{}12 to \PYGZhy{}2}\PYG{l+s+s1}{\PYGZsq{}}
\PYG{p}{)}
\PYG{n}{plt}\PYG{o}{.}\PYG{n}{show}\PYG{p}{(}\PYG{p}{)}
\end{sphinxVerbatim}

\end{sphinxuseclass}\end{sphinxVerbatimInput}
\begin{sphinxVerbatimOutput}

\begin{sphinxuseclass}{cell_output}
\noindent\sphinxincludegraphics{{f7a171cb8d4193690488883353b1ad5d1c1208f33f2581c2d2f6b793b9526043}.png}

\end{sphinxuseclass}\end{sphinxVerbatimOutput}

\end{sphinxuseclass}

\subsubsection{Implement an equal\sphinxhyphen{}weighted size investing strategy based on market capitalization \textasciitilde{}\textasciitilde{}at the start of each month\textasciitilde{}\textasciitilde{} from the previous June}
\label{\detokenize{herron_06_practice_04:implement-an-equal-weighted-size-investing-strategy-based-on-market-capitalization-at-the-start-of-each-month-from-the-previous-june}}
\sphinxAtStartPar
\sphinxstyleemphasis{\sphinxstylestrong{To show how to form portfolios at lower frequencies, I change this task to form portfolios once a year based on market capitalization from the previous June.}}
We will assign stocks to portfolios based on market capitalization (i.e., size or the market value of equity) each June and use these portfolio assignments from July through the following June.
That is, we rank in June of year \$t\$, then hold them in portfolios from July in year \$t\$ through June in year \$t+1\$.

\begin{sphinxuseclass}{cell}\begin{sphinxVerbatimInput}

\begin{sphinxuseclass}{cell_input}
\begin{sphinxVerbatim}[commandchars=\\\{\}]
\PYG{n}{me\PYGZus{}june} \PYG{o}{=} \PYG{n}{crsp}\PYG{p}{[}\PYG{l+s+s1}{\PYGZsq{}}\PYG{l+s+s1}{ME}\PYG{l+s+s1}{\PYGZsq{}}\PYG{p}{]}\PYG{o}{.}\PYG{n}{unstack}\PYG{p}{(}\PYG{l+s+s1}{\PYGZsq{}}\PYG{l+s+s1}{PERMNO}\PYG{l+s+s1}{\PYGZsq{}}\PYG{p}{)}\PYG{o}{.}\PYG{n}{pipe}\PYG{p}{(}\PYG{k}{lambda} \PYG{n}{x}\PYG{p}{:} \PYG{n}{x}\PYG{o}{.}\PYG{n}{loc}\PYG{p}{[}\PYG{n}{x}\PYG{o}{.}\PYG{n}{index}\PYG{o}{.}\PYG{n}{month} \PYG{o}{==} \PYG{l+m+mi}{6}\PYG{p}{]}\PYG{p}{)}
\end{sphinxVerbatim}

\end{sphinxuseclass}\end{sphinxVerbatimInput}

\end{sphinxuseclass}
\begin{sphinxuseclass}{cell}\begin{sphinxVerbatimInput}

\begin{sphinxuseclass}{cell_input}
\begin{sphinxVerbatim}[commandchars=\\\{\}]
\PYG{n}{me\PYGZus{}june}\PYG{o}{.}\PYG{n}{head}\PYG{p}{(}\PYG{p}{)}
\end{sphinxVerbatim}

\end{sphinxuseclass}\end{sphinxVerbatimInput}
\begin{sphinxVerbatimOutput}

\begin{sphinxuseclass}{cell_output}
\begin{sphinxVerbatim}[commandchars=\\\{\}]
PERMNO   10000  10001  10002  10003  10005      10006  10007  10008  10009  \PYGZbs{}
date                                                                         
1926\PYGZhy{}06    NaN    NaN    NaN    NaN    NaN 59400.0000    NaN    NaN    NaN   
1927\PYGZhy{}06    NaN    NaN    NaN    NaN    NaN 59400.0000    NaN    NaN    NaN   
1928\PYGZhy{}06    NaN    NaN    NaN    NaN    NaN 57675.0000    NaN    NaN    NaN   
1929\PYGZhy{}06    NaN    NaN    NaN    NaN    NaN 57900.0000    NaN    NaN    NaN   
1930\PYGZhy{}06    NaN    NaN    NaN    NaN    NaN 30900.0000    NaN    NaN    NaN   

PERMNO   10010  ...  93423  93426  93428  93429  93430  93432  93433  93434  \PYGZbs{}
date            ...                                                           
1926\PYGZhy{}06    NaN  ...    NaN    NaN    NaN    NaN    NaN    NaN    NaN    NaN   
1927\PYGZhy{}06    NaN  ...    NaN    NaN    NaN    NaN    NaN    NaN    NaN    NaN   
1928\PYGZhy{}06    NaN  ...    NaN    NaN    NaN    NaN    NaN    NaN    NaN    NaN   
1929\PYGZhy{}06    NaN  ...    NaN    NaN    NaN    NaN    NaN    NaN    NaN    NaN   
1930\PYGZhy{}06    NaN  ...    NaN    NaN    NaN    NaN    NaN    NaN    NaN    NaN   

PERMNO   93435  93436  
date                   
1926\PYGZhy{}06    NaN    NaN  
1927\PYGZhy{}06    NaN    NaN  
1928\PYGZhy{}06    NaN    NaN  
1929\PYGZhy{}06    NaN    NaN  
1930\PYGZhy{}06    NaN    NaN  

[5 rows x 26480 columns]
\end{sphinxVerbatim}

\end{sphinxuseclass}\end{sphinxVerbatimOutput}

\end{sphinxuseclass}
\sphinxAtStartPar
Next we assign portfolios on these June market capitalizations.

\begin{sphinxuseclass}{cell}\begin{sphinxVerbatimInput}

\begin{sphinxuseclass}{cell_input}
\begin{sphinxVerbatim}[commandchars=\\\{\}]
\PYG{n}{me\PYGZus{}june}\PYG{o}{.}\PYG{n}{apply}\PYG{p}{(}\PYG{n}{pd}\PYG{o}{.}\PYG{n}{qcut}\PYG{p}{,} \PYG{n}{q}\PYG{o}{=}\PYG{l+m+mi}{10}\PYG{p}{,} \PYG{n}{labels}\PYG{o}{=}\PYG{k+kc}{False}\PYG{p}{,} \PYG{n}{axis}\PYG{o}{=}\PYG{l+m+mi}{1}\PYG{p}{)}\PYG{o}{.}\PYG{n}{head}\PYG{p}{(}\PYG{p}{)}
\end{sphinxVerbatim}

\end{sphinxuseclass}\end{sphinxVerbatimInput}
\begin{sphinxVerbatimOutput}

\begin{sphinxuseclass}{cell_output}
\begin{sphinxVerbatim}[commandchars=\\\{\}]
PERMNO   10000  10001  10002  10003  10005  10006  10007  10008  10009  10010  \PYGZbs{}
date                                                                            
1926\PYGZhy{}06    NaN    NaN    NaN    NaN    NaN 7.0000    NaN    NaN    NaN    NaN   
1927\PYGZhy{}06    NaN    NaN    NaN    NaN    NaN 7.0000    NaN    NaN    NaN    NaN   
1928\PYGZhy{}06    NaN    NaN    NaN    NaN    NaN 7.0000    NaN    NaN    NaN    NaN   
1929\PYGZhy{}06    NaN    NaN    NaN    NaN    NaN 6.0000    NaN    NaN    NaN    NaN   
1930\PYGZhy{}06    NaN    NaN    NaN    NaN    NaN 6.0000    NaN    NaN    NaN    NaN   

PERMNO   ...  93423  93426  93428  93429  93430  93432  93433  93434  93435  \PYGZbs{}
date     ...                                                                  
1926\PYGZhy{}06  ...    NaN    NaN    NaN    NaN    NaN    NaN    NaN    NaN    NaN   
1927\PYGZhy{}06  ...    NaN    NaN    NaN    NaN    NaN    NaN    NaN    NaN    NaN   
1928\PYGZhy{}06  ...    NaN    NaN    NaN    NaN    NaN    NaN    NaN    NaN    NaN   
1929\PYGZhy{}06  ...    NaN    NaN    NaN    NaN    NaN    NaN    NaN    NaN    NaN   
1930\PYGZhy{}06  ...    NaN    NaN    NaN    NaN    NaN    NaN    NaN    NaN    NaN   

PERMNO   93436  
date            
1926\PYGZhy{}06    NaN  
1927\PYGZhy{}06    NaN  
1928\PYGZhy{}06    NaN  
1929\PYGZhy{}06    NaN  
1930\PYGZhy{}06    NaN  

[5 rows x 26480 columns]
\end{sphinxVerbatim}

\end{sphinxuseclass}\end{sphinxVerbatimOutput}

\end{sphinxuseclass}
\sphinxAtStartPar
Note, we want to use these portfolio assignments from July through June, so we need to \sphinxcode{\sphinxupquote{.shift()}} with \sphinxcode{\sphinxupquote{freq='1M'}}, which shifts the index one month instead of shifting the values one row.
Because we only have June data, a one\sphinxhyphen{}row shift is a one\sphinxhyphen{}year shift instead of a one\sphinxhyphen{}month shift.
We also need to up sample our June data to monthly date with \sphinxcode{\sphinxupquote{.resample('M')}} and the \sphinxcode{\sphinxupquote{.ffill()}} method.
Because we only want to fill forward 12 months, we use \sphinxcode{\sphinxupquote{.ffill(limit=12)}}.

\begin{sphinxuseclass}{cell}\begin{sphinxVerbatimInput}

\begin{sphinxuseclass}{cell_input}
\begin{sphinxVerbatim}[commandchars=\\\{\}]
\PYG{n}{me\PYGZus{}june}\PYG{o}{.}\PYG{n}{apply}\PYG{p}{(}\PYG{n}{pd}\PYG{o}{.}\PYG{n}{qcut}\PYG{p}{,} \PYG{n}{q}\PYG{o}{=}\PYG{l+m+mi}{10}\PYG{p}{,} \PYG{n}{labels}\PYG{o}{=}\PYG{k+kc}{False}\PYG{p}{,} \PYG{n}{axis}\PYG{o}{=}\PYG{l+m+mi}{1}\PYG{p}{)}\PYG{o}{.}\PYG{n}{shift}\PYG{p}{(}\PYG{n}{freq}\PYG{o}{=}\PYG{l+s+s1}{\PYGZsq{}}\PYG{l+s+s1}{1M}\PYG{l+s+s1}{\PYGZsq{}}\PYG{p}{)}\PYG{o}{.}\PYG{n}{resample}\PYG{p}{(}\PYG{l+s+s1}{\PYGZsq{}}\PYG{l+s+s1}{M}\PYG{l+s+s1}{\PYGZsq{}}\PYG{p}{)}\PYG{o}{.}\PYG{n}{ffill}\PYG{p}{(}\PYG{n}{limit}\PYG{o}{=}\PYG{l+m+mi}{12}\PYG{p}{)}\PYG{o}{.}\PYG{n}{tail}\PYG{p}{(}\PYG{p}{)}
\end{sphinxVerbatim}

\end{sphinxuseclass}\end{sphinxVerbatimInput}
\begin{sphinxVerbatimOutput}

\begin{sphinxuseclass}{cell_output}
\begin{sphinxVerbatim}[commandchars=\\\{\}]
PERMNO   10000  10001  10002  10003  10005  10006  10007  10008  10009  10010  \PYGZbs{}
date                                                                            
2022\PYGZhy{}03    NaN    NaN    NaN    NaN    NaN    NaN    NaN    NaN    NaN    NaN   
2022\PYGZhy{}04    NaN    NaN    NaN    NaN    NaN    NaN    NaN    NaN    NaN    NaN   
2022\PYGZhy{}05    NaN    NaN    NaN    NaN    NaN    NaN    NaN    NaN    NaN    NaN   
2022\PYGZhy{}06    NaN    NaN    NaN    NaN    NaN    NaN    NaN    NaN    NaN    NaN   
2022\PYGZhy{}07    NaN    NaN    NaN    NaN    NaN    NaN    NaN    NaN    NaN    NaN   

PERMNO   ...  93423  93426  93428  93429  93430  93432  93433  93434  93435  \PYGZbs{}
date     ...                                                                  
2022\PYGZhy{}03  ... 7.0000 3.0000    NaN 8.0000    NaN    NaN    NaN 1.0000    NaN   
2022\PYGZhy{}04  ... 7.0000 3.0000    NaN 8.0000    NaN    NaN    NaN 1.0000    NaN   
2022\PYGZhy{}05  ... 7.0000 3.0000    NaN 8.0000    NaN    NaN    NaN 1.0000    NaN   
2022\PYGZhy{}06  ... 7.0000 3.0000    NaN 8.0000    NaN    NaN    NaN 1.0000    NaN   
2022\PYGZhy{}07  ... 6.0000 4.0000    NaN 8.0000    NaN    NaN    NaN 1.0000    NaN   

PERMNO   93436  
date            
2022\PYGZhy{}03 9.0000  
2022\PYGZhy{}04 9.0000  
2022\PYGZhy{}05 9.0000  
2022\PYGZhy{}06 9.0000  
2022\PYGZhy{}07 9.0000  

[5 rows x 26480 columns]
\end{sphinxVerbatim}

\end{sphinxuseclass}\end{sphinxVerbatimOutput}

\end{sphinxuseclass}
\sphinxAtStartPar
We can combine the \sphinxcode{\sphinxupquote{.apply()}}, \sphinxcode{\sphinxupquote{.shift()}}, \sphinxcode{\sphinxupquote{.resample()}}, and \sphinxcode{\sphinxupquote{.fill()}} methods into one operation when we combine the returns, trailing market capitalization, and portfolio assignments.

\begin{sphinxuseclass}{cell}\begin{sphinxVerbatimInput}

\begin{sphinxuseclass}{cell_input}
\begin{sphinxVerbatim}[commandchars=\\\{\}]
\PYG{n}{me} \PYG{o}{=} \PYG{p}{(}
    \PYG{n}{pd}\PYG{o}{.}\PYG{n}{concat}\PYG{p}{(}
        \PYG{n}{objs}\PYG{o}{=}\PYG{p}{[}
            \PYG{n}{ret\PYGZus{}1m}\PYG{p}{,}
            \PYG{n}{crsp}\PYG{p}{[}\PYG{l+s+s1}{\PYGZsq{}}\PYG{l+s+s1}{ME}\PYG{l+s+s1}{\PYGZsq{}}\PYG{p}{]}\PYG{o}{.}\PYG{n}{unstack}\PYG{p}{(}\PYG{l+s+s1}{\PYGZsq{}}\PYG{l+s+s1}{PERMNO}\PYG{l+s+s1}{\PYGZsq{}}\PYG{p}{)}\PYG{o}{.}\PYG{n}{shift}\PYG{p}{(}\PYG{l+m+mi}{1}\PYG{p}{)}\PYG{p}{,}
            \PYG{p}{(}
                \PYG{n}{me\PYGZus{}june}
                \PYG{o}{.}\PYG{n}{apply}\PYG{p}{(}\PYG{n}{pd}\PYG{o}{.}\PYG{n}{qcut}\PYG{p}{,} \PYG{n}{q}\PYG{o}{=}\PYG{l+m+mi}{10}\PYG{p}{,} \PYG{n}{labels}\PYG{o}{=}\PYG{k+kc}{False}\PYG{p}{,} \PYG{n}{axis}\PYG{o}{=}\PYG{l+m+mi}{1}\PYG{p}{)}
                \PYG{o}{.}\PYG{n}{shift}\PYG{p}{(}\PYG{n}{freq}\PYG{o}{=}\PYG{l+s+s1}{\PYGZsq{}}\PYG{l+s+s1}{1M}\PYG{l+s+s1}{\PYGZsq{}}\PYG{p}{)}
                \PYG{o}{.}\PYG{n}{resample}\PYG{p}{(}\PYG{l+s+s1}{\PYGZsq{}}\PYG{l+s+s1}{M}\PYG{l+s+s1}{\PYGZsq{}}\PYG{p}{)}
                \PYG{o}{.}\PYG{n}{ffill}\PYG{p}{(}\PYG{n}{limit}\PYG{o}{=}\PYG{l+m+mi}{12}\PYG{p}{)}
            \PYG{p}{)}
        \PYG{p}{]}\PYG{p}{,}
        \PYG{n}{keys}\PYG{o}{=}\PYG{p}{[}\PYG{l+s+s1}{\PYGZsq{}}\PYG{l+s+s1}{Return}\PYG{l+s+s1}{\PYGZsq{}}\PYG{p}{,} \PYG{l+s+s1}{\PYGZsq{}}\PYG{l+s+s1}{Trailing ME}\PYG{l+s+s1}{\PYGZsq{}}\PYG{p}{,} \PYG{l+s+s1}{\PYGZsq{}}\PYG{l+s+s1}{Portfolio}\PYG{l+s+s1}{\PYGZsq{}}\PYG{p}{]}\PYG{p}{,}
        \PYG{n}{names}\PYG{o}{=}\PYG{p}{[}\PYG{l+s+s1}{\PYGZsq{}}\PYG{l+s+s1}{Variable}\PYG{l+s+s1}{\PYGZsq{}}\PYG{p}{]}\PYG{p}{,}
        \PYG{n}{axis}\PYG{o}{=}\PYG{l+m+mi}{1}
    \PYG{p}{)}
    \PYG{o}{.}\PYG{n}{stack}\PYG{p}{(}\PYG{l+s+s1}{\PYGZsq{}}\PYG{l+s+s1}{PERMNO}\PYG{l+s+s1}{\PYGZsq{}}\PYG{p}{)}
    \PYG{o}{.}\PYG{n}{dropna}\PYG{p}{(}\PYG{p}{)}
    \PYG{o}{.}\PYG{n}{assign}\PYG{p}{(}\PYG{n}{Portfolio}\PYG{o}{=}\PYG{k}{lambda} \PYG{n}{x}\PYG{p}{:} \PYG{n}{x}\PYG{p}{[}\PYG{l+s+s1}{\PYGZsq{}}\PYG{l+s+s1}{Portfolio}\PYG{l+s+s1}{\PYGZsq{}}\PYG{p}{]}\PYG{o}{.}\PYG{n}{astype}\PYG{p}{(}\PYG{n+nb}{int}\PYG{p}{)} \PYG{o}{+} \PYG{l+m+mi}{1}\PYG{p}{)}
\PYG{p}{)}
\end{sphinxVerbatim}

\end{sphinxuseclass}\end{sphinxVerbatimInput}

\end{sphinxuseclass}
\begin{sphinxuseclass}{cell}\begin{sphinxVerbatimInput}

\begin{sphinxuseclass}{cell_input}
\begin{sphinxVerbatim}[commandchars=\\\{\}]
\PYG{n}{me}\PYG{o}{.}\PYG{n}{head}\PYG{p}{(}\PYG{p}{)}
\end{sphinxVerbatim}

\end{sphinxuseclass}\end{sphinxVerbatimInput}
\begin{sphinxVerbatimOutput}

\begin{sphinxuseclass}{cell_output}
\begin{sphinxVerbatim}[commandchars=\\\{\}]
Variable        Return  Trailing ME  Portfolio
date    PERMNO                                
1926\PYGZhy{}07 10022   0.2400   10000.0000          5
        10030   0.0211   19402.5000          6
        10057   0.0078    4031.2500          2
        10073   0.0729    1656.0000          1
        10081   0.0156    9536.0000          4
\end{sphinxVerbatim}

\end{sphinxuseclass}\end{sphinxVerbatimOutput}

\end{sphinxuseclass}
\begin{sphinxuseclass}{cell}\begin{sphinxVerbatimInput}

\begin{sphinxuseclass}{cell_input}
\begin{sphinxVerbatim}[commandchars=\\\{\}]
\PYG{n}{me\PYGZus{}ew} \PYG{o}{=} \PYG{p}{(}
    \PYG{n}{me}
    \PYG{o}{.}\PYG{n}{groupby}\PYG{p}{(}\PYG{p}{[}\PYG{l+s+s1}{\PYGZsq{}}\PYG{l+s+s1}{date}\PYG{l+s+s1}{\PYGZsq{}}\PYG{p}{,} \PYG{l+s+s1}{\PYGZsq{}}\PYG{l+s+s1}{Portfolio}\PYG{l+s+s1}{\PYGZsq{}}\PYG{p}{]}\PYG{p}{)}
    \PYG{p}{[}\PYG{l+s+s1}{\PYGZsq{}}\PYG{l+s+s1}{Return}\PYG{l+s+s1}{\PYGZsq{}}\PYG{p}{]}
    \PYG{o}{.}\PYG{n}{mean}\PYG{p}{(}\PYG{p}{)}
    \PYG{o}{.}\PYG{n}{unstack}\PYG{p}{(}\PYG{l+s+s1}{\PYGZsq{}}\PYG{l+s+s1}{Portfolio}\PYG{l+s+s1}{\PYGZsq{}}\PYG{p}{)}
    \PYG{o}{.}\PYG{n}{add\PYGZus{}prefix}\PYG{p}{(}\PYG{l+s+s1}{\PYGZsq{}}\PYG{l+s+s1}{ME }\PYG{l+s+s1}{\PYGZsq{}}\PYG{p}{)}
\PYG{p}{)}
\end{sphinxVerbatim}

\end{sphinxuseclass}\end{sphinxVerbatimInput}

\end{sphinxuseclass}
\begin{sphinxuseclass}{cell}\begin{sphinxVerbatimInput}

\begin{sphinxuseclass}{cell_input}
\begin{sphinxVerbatim}[commandchars=\\\{\}]
\PYG{n}{me\PYGZus{}ew}\PYG{o}{.}\PYG{n}{mul}\PYG{p}{(}\PYG{l+m+mi}{100}\PYG{p}{)}\PYG{o}{.}\PYG{n}{join}\PYG{p}{(}\PYG{p}{[}\PYG{n}{ff\PYGZus{}0}\PYG{p}{[}\PYG{l+m+mi}{0}\PYG{p}{]}\PYG{p}{,} \PYG{n}{ff\PYGZus{}mom}\PYG{p}{[}\PYG{l+m+mi}{0}\PYG{p}{]}\PYG{p}{]}\PYG{p}{)}\PYG{o}{.}\PYG{n}{pipe}\PYG{p}{(}\PYG{n}{get\PYGZus{}coefs}\PYG{p}{,} \PYG{n}{fun}\PYG{o}{=}\PYG{n}{capm}\PYG{p}{,} \PYG{n}{n}\PYG{o}{=}\PYG{l+m+mi}{10}\PYG{p}{)}\PYG{o}{.}\PYG{n}{pipe}\PYG{p}{(}\PYG{n}{plot\PYGZus{}alpha}\PYG{p}{)}
\PYG{n}{plt}\PYG{o}{.}\PYG{n}{title}\PYG{p}{(}
    \PYG{l+s+s1}{\PYGZsq{}}\PYG{l+s+s1}{CAPM Tests of Size Strategy}\PYG{l+s+s1}{\PYGZsq{}} \PYG{o}{+}
    \PYG{l+s+s1}{\PYGZsq{}}\PYG{l+s+se}{\PYGZbs{}n}\PYG{l+s+s1}{Equal\PYGZhy{}Weighted Portfolios Formed June Market Value of Equity}\PYG{l+s+s1}{\PYGZsq{}} \PYG{o}{+}
    \PYG{l+s+s1}{\PYGZsq{}}\PYG{l+s+se}{\PYGZbs{}n}\PYG{l+s+s1}{Black Vertical Bars Indicate Standard Errors}\PYG{l+s+s1}{\PYGZsq{}}
\PYG{p}{)}
\PYG{n}{plt}\PYG{o}{.}\PYG{n}{show}\PYG{p}{(}\PYG{p}{)}
\end{sphinxVerbatim}

\end{sphinxuseclass}\end{sphinxVerbatimInput}
\begin{sphinxVerbatimOutput}

\begin{sphinxuseclass}{cell_output}
\noindent\sphinxincludegraphics{{8872db5b249404c8e14edbab82f5358ec876d78ae3ef3d65305b1190711b891a}.png}

\end{sphinxuseclass}\end{sphinxVerbatimOutput}

\end{sphinxuseclass}
\begin{sphinxuseclass}{cell}\begin{sphinxVerbatimInput}

\begin{sphinxuseclass}{cell_input}
\begin{sphinxVerbatim}[commandchars=\\\{\}]
\PYG{n}{me\PYGZus{}ew}\PYG{o}{.}\PYG{n}{mul}\PYG{p}{(}\PYG{l+m+mi}{100}\PYG{p}{)}\PYG{o}{.}\PYG{n}{join}\PYG{p}{(}\PYG{p}{[}\PYG{n}{ff\PYGZus{}0}\PYG{p}{[}\PYG{l+m+mi}{0}\PYG{p}{]}\PYG{p}{,} \PYG{n}{ff\PYGZus{}mom}\PYG{p}{[}\PYG{l+m+mi}{0}\PYG{p}{]}\PYG{p}{]}\PYG{p}{)}\PYG{o}{.}\PYG{n}{pipe}\PYG{p}{(}\PYG{n}{get\PYGZus{}coefs}\PYG{p}{,} \PYG{n}{fun}\PYG{o}{=}\PYG{n}{ff4}\PYG{p}{,} \PYG{n}{n}\PYG{o}{=}\PYG{l+m+mi}{10}\PYG{p}{)}\PYG{o}{.}\PYG{n}{pipe}\PYG{p}{(}\PYG{n}{plot\PYGZus{}alpha}\PYG{p}{)}
\PYG{n}{plt}\PYG{o}{.}\PYG{n}{title}\PYG{p}{(}
    \PYG{l+s+s1}{\PYGZsq{}}\PYG{l+s+s1}{FF4 Tests of Size Strategy}\PYG{l+s+s1}{\PYGZsq{}} \PYG{o}{+}
    \PYG{l+s+s1}{\PYGZsq{}}\PYG{l+s+se}{\PYGZbs{}n}\PYG{l+s+s1}{Equal\PYGZhy{}Weighted Portfolios Formed June Market Value of Equity}\PYG{l+s+s1}{\PYGZsq{}} \PYG{o}{+}
    \PYG{l+s+s1}{\PYGZsq{}}\PYG{l+s+se}{\PYGZbs{}n}\PYG{l+s+s1}{Black Vertical Bars Indicate Standard Errors}\PYG{l+s+s1}{\PYGZsq{}}
\PYG{p}{)}
\PYG{n}{plt}\PYG{o}{.}\PYG{n}{show}\PYG{p}{(}\PYG{p}{)}
\end{sphinxVerbatim}

\end{sphinxuseclass}\end{sphinxVerbatimInput}
\begin{sphinxVerbatimOutput}

\begin{sphinxuseclass}{cell_output}
\noindent\sphinxincludegraphics{{da1a5df44d3d573abd9b0e06810182e2faf2d6075d7ba52d9720be73ab245739}.png}

\end{sphinxuseclass}\end{sphinxVerbatimOutput}

\end{sphinxuseclass}
\sphinxAtStartPar
If we had more time to dig into the size investing strategy, we would find:
\begin{enumerate}
\sphinxsetlistlabels{\arabic}{enumi}{enumii}{}{.}%
\item {} 
\sphinxAtStartPar
The alpha is concentrated in January due to tax loss harvesting, where investors sell their losers in December and buy them back in January to earn a tax deduction

\item {} 
\sphinxAtStartPar
The smallest stocks in the ME 1 portfolio are tiny, making them impractical for institutional investors and allowing the anomaly to persist (because these stocks are small and illiquid, the size factor does not try to explain away their alpha)

\end{enumerate}

\sphinxAtStartPar
We can quickly see the point 2 above by plotting the mean marker capitalization for each portfolio.
This analysis is “quick and dirty” because we have not inflation adjusted these values.
Still, we quickly see that the ME 10 stocks are 1,000 times larger than the ME 1 stocks, on average.

\begin{sphinxuseclass}{cell}\begin{sphinxVerbatimInput}

\begin{sphinxuseclass}{cell_input}
\begin{sphinxVerbatim}[commandchars=\\\{\}]
\PYG{p}{(}
    \PYG{n}{me}
    \PYG{o}{.}\PYG{n}{groupby}\PYG{p}{(}\PYG{p}{[}\PYG{l+s+s1}{\PYGZsq{}}\PYG{l+s+s1}{Portfolio}\PYG{l+s+s1}{\PYGZsq{}}\PYG{p}{,} \PYG{l+s+s1}{\PYGZsq{}}\PYG{l+s+s1}{date}\PYG{l+s+s1}{\PYGZsq{}}\PYG{p}{]}\PYG{p}{)}
    \PYG{p}{[}\PYG{l+s+s1}{\PYGZsq{}}\PYG{l+s+s1}{Trailing ME}\PYG{l+s+s1}{\PYGZsq{}}\PYG{p}{]}
    \PYG{o}{.}\PYG{n}{mean}\PYG{p}{(}\PYG{p}{)}
    \PYG{o}{.}\PYG{n}{unstack}\PYG{p}{(}\PYG{l+s+s1}{\PYGZsq{}}\PYG{l+s+s1}{Portfolio}\PYG{l+s+s1}{\PYGZsq{}}\PYG{p}{)}
    \PYG{o}{.}\PYG{n}{add\PYGZus{}prefix}\PYG{p}{(}\PYG{l+s+s1}{\PYGZsq{}}\PYG{l+s+s1}{ME }\PYG{l+s+s1}{\PYGZsq{}}\PYG{p}{)}
    \PYG{o}{.}\PYG{n}{mean}\PYG{p}{(}\PYG{p}{)}
    \PYG{o}{.}\PYG{n}{plot}\PYG{p}{(}\PYG{n}{kind}\PYG{o}{=}\PYG{l+s+s1}{\PYGZsq{}}\PYG{l+s+s1}{bar}\PYG{l+s+s1}{\PYGZsq{}}\PYG{p}{)}
\PYG{p}{)}
\PYG{n}{plt}\PYG{o}{.}\PYG{n}{semilogy}\PYG{p}{(}\PYG{p}{)}
\PYG{n}{plt}\PYG{o}{.}\PYG{n}{ylabel}\PYG{p}{(}\PYG{l+s+s1}{\PYGZsq{}}\PYG{l+s+s1}{Mean Market Value of Equity (Nominal U.S. Dollars)}\PYG{l+s+s1}{\PYGZsq{}}\PYG{p}{)}
\PYG{n}{plt}\PYG{o}{.}\PYG{n}{title}\PYG{p}{(}\PYG{l+s+s1}{\PYGZsq{}}\PYG{l+s+s1}{Mean Market Value of Stocks in ME Portfolios}\PYG{l+s+s1}{\PYGZsq{}}\PYG{p}{)}
\PYG{n}{plt}\PYG{o}{.}\PYG{n}{show}\PYG{p}{(}\PYG{p}{)}
\end{sphinxVerbatim}

\end{sphinxuseclass}\end{sphinxVerbatimInput}
\begin{sphinxVerbatimOutput}

\begin{sphinxuseclass}{cell_output}
\noindent\sphinxincludegraphics{{9912623f731b1538e1491a458598c59c8be346a294d6d950902f235fd10a843d}.png}

\end{sphinxuseclass}\end{sphinxVerbatimOutput}

\end{sphinxuseclass}

\subsubsection{Implement a value\sphinxhyphen{}weighted size investing strategy based on market capitalization \textasciitilde{}\textasciitilde{}at the start of each month\textasciitilde{}\textasciitilde{} from the previous June}
\label{\detokenize{herron_06_practice_04:implement-a-value-weighted-size-investing-strategy-based-on-market-capitalization-at-the-start-of-each-month-from-the-previous-june}}
\sphinxAtStartPar
With value\sphinxhyphen{}weighted portfolios, the size investing strategy alphas disappear!
The alpha disappears because value\sphinxhyphen{}weighted portfolios down weight the smallest of the smallest stocks that made the size investing strategy appear to be returns not explained by risk.

\begin{sphinxuseclass}{cell}\begin{sphinxVerbatimInput}

\begin{sphinxuseclass}{cell_input}
\begin{sphinxVerbatim}[commandchars=\\\{\}]
\PYG{n}{me\PYGZus{}vw} \PYG{o}{=} \PYG{p}{(}
    \PYG{n}{me}
    \PYG{o}{.}\PYG{n}{groupby}\PYG{p}{(}\PYG{p}{[}\PYG{l+s+s1}{\PYGZsq{}}\PYG{l+s+s1}{date}\PYG{l+s+s1}{\PYGZsq{}}\PYG{p}{,} \PYG{l+s+s1}{\PYGZsq{}}\PYG{l+s+s1}{Portfolio}\PYG{l+s+s1}{\PYGZsq{}}\PYG{p}{]}\PYG{p}{)}
    \PYG{o}{.}\PYG{n}{apply}\PYG{p}{(}\PYG{k}{lambda} \PYG{n}{x}\PYG{p}{:} \PYG{n}{np}\PYG{o}{.}\PYG{n}{average}\PYG{p}{(}\PYG{n}{a}\PYG{o}{=}\PYG{n}{x}\PYG{p}{[}\PYG{l+s+s1}{\PYGZsq{}}\PYG{l+s+s1}{Return}\PYG{l+s+s1}{\PYGZsq{}}\PYG{p}{]}\PYG{p}{,} \PYG{n}{weights}\PYG{o}{=}\PYG{n}{x}\PYG{p}{[}\PYG{l+s+s1}{\PYGZsq{}}\PYG{l+s+s1}{Trailing ME}\PYG{l+s+s1}{\PYGZsq{}}\PYG{p}{]}\PYG{p}{)}\PYG{p}{)}
    \PYG{o}{.}\PYG{n}{unstack}\PYG{p}{(}\PYG{l+s+s1}{\PYGZsq{}}\PYG{l+s+s1}{Portfolio}\PYG{l+s+s1}{\PYGZsq{}}\PYG{p}{)}
    \PYG{o}{.}\PYG{n}{add\PYGZus{}prefix}\PYG{p}{(}\PYG{l+s+s1}{\PYGZsq{}}\PYG{l+s+s1}{ME }\PYG{l+s+s1}{\PYGZsq{}}\PYG{p}{)}
\PYG{p}{)}
\end{sphinxVerbatim}

\end{sphinxuseclass}\end{sphinxVerbatimInput}

\end{sphinxuseclass}
\begin{sphinxuseclass}{cell}\begin{sphinxVerbatimInput}

\begin{sphinxuseclass}{cell_input}
\begin{sphinxVerbatim}[commandchars=\\\{\}]
\PYG{n}{me\PYGZus{}vw}\PYG{o}{.}\PYG{n}{mul}\PYG{p}{(}\PYG{l+m+mi}{100}\PYG{p}{)}\PYG{o}{.}\PYG{n}{join}\PYG{p}{(}\PYG{p}{[}\PYG{n}{ff\PYGZus{}0}\PYG{p}{[}\PYG{l+m+mi}{0}\PYG{p}{]}\PYG{p}{,} \PYG{n}{ff\PYGZus{}mom}\PYG{p}{[}\PYG{l+m+mi}{0}\PYG{p}{]}\PYG{p}{]}\PYG{p}{)}\PYG{o}{.}\PYG{n}{pipe}\PYG{p}{(}\PYG{n}{get\PYGZus{}coefs}\PYG{p}{,} \PYG{n}{fun}\PYG{o}{=}\PYG{n}{capm}\PYG{p}{,} \PYG{n}{n}\PYG{o}{=}\PYG{l+m+mi}{10}\PYG{p}{)}\PYG{o}{.}\PYG{n}{pipe}\PYG{p}{(}\PYG{n}{plot\PYGZus{}alpha}\PYG{p}{)}
\PYG{n}{plt}\PYG{o}{.}\PYG{n}{title}\PYG{p}{(}
    \PYG{l+s+s1}{\PYGZsq{}}\PYG{l+s+s1}{CAPM Tests of Size Strategy}\PYG{l+s+s1}{\PYGZsq{}} \PYG{o}{+}
    \PYG{l+s+s1}{\PYGZsq{}}\PYG{l+s+se}{\PYGZbs{}n}\PYG{l+s+s1}{Value\PYGZhy{}Weighted Portfolios Formed June Market Value of Equity}\PYG{l+s+s1}{\PYGZsq{}} \PYG{o}{+}
    \PYG{l+s+s1}{\PYGZsq{}}\PYG{l+s+se}{\PYGZbs{}n}\PYG{l+s+s1}{Black Vertical Bars Indicate Standard Errors}\PYG{l+s+s1}{\PYGZsq{}}
\PYG{p}{)}
\PYG{n}{plt}\PYG{o}{.}\PYG{n}{show}\PYG{p}{(}\PYG{p}{)}
\end{sphinxVerbatim}

\end{sphinxuseclass}\end{sphinxVerbatimInput}
\begin{sphinxVerbatimOutput}

\begin{sphinxuseclass}{cell_output}
\noindent\sphinxincludegraphics{{3f1ab9c29b78dcf3f973dbae288263b3138d8bf9faea23b8554377f0023b8978}.png}

\end{sphinxuseclass}\end{sphinxVerbatimOutput}

\end{sphinxuseclass}
\begin{sphinxuseclass}{cell}\begin{sphinxVerbatimInput}

\begin{sphinxuseclass}{cell_input}
\begin{sphinxVerbatim}[commandchars=\\\{\}]
\PYG{n}{me\PYGZus{}vw}\PYG{o}{.}\PYG{n}{mul}\PYG{p}{(}\PYG{l+m+mi}{100}\PYG{p}{)}\PYG{o}{.}\PYG{n}{join}\PYG{p}{(}\PYG{p}{[}\PYG{n}{ff\PYGZus{}0}\PYG{p}{[}\PYG{l+m+mi}{0}\PYG{p}{]}\PYG{p}{,} \PYG{n}{ff\PYGZus{}mom}\PYG{p}{[}\PYG{l+m+mi}{0}\PYG{p}{]}\PYG{p}{]}\PYG{p}{)}\PYG{o}{.}\PYG{n}{pipe}\PYG{p}{(}\PYG{n}{get\PYGZus{}coefs}\PYG{p}{,} \PYG{n}{fun}\PYG{o}{=}\PYG{n}{ff4}\PYG{p}{,} \PYG{n}{n}\PYG{o}{=}\PYG{l+m+mi}{10}\PYG{p}{)}\PYG{o}{.}\PYG{n}{pipe}\PYG{p}{(}\PYG{n}{plot\PYGZus{}alpha}\PYG{p}{)}
\PYG{n}{plt}\PYG{o}{.}\PYG{n}{title}\PYG{p}{(}
    \PYG{l+s+s1}{\PYGZsq{}}\PYG{l+s+s1}{FF4 Tests of Size Strategy}\PYG{l+s+s1}{\PYGZsq{}} \PYG{o}{+}
    \PYG{l+s+s1}{\PYGZsq{}}\PYG{l+s+se}{\PYGZbs{}n}\PYG{l+s+s1}{Value\PYGZhy{}Weighted Portfolios Formed June Market Value of Equity}\PYG{l+s+s1}{\PYGZsq{}} \PYG{o}{+}
    \PYG{l+s+s1}{\PYGZsq{}}\PYG{l+s+se}{\PYGZbs{}n}\PYG{l+s+s1}{Black Vertical Bars Indicate Standard Errors}\PYG{l+s+s1}{\PYGZsq{}}
\PYG{p}{)}
\PYG{n}{plt}\PYG{o}{.}\PYG{n}{show}\PYG{p}{(}\PYG{p}{)}
\end{sphinxVerbatim}

\end{sphinxuseclass}\end{sphinxVerbatimInput}
\begin{sphinxVerbatimOutput}

\begin{sphinxuseclass}{cell_output}
\noindent\sphinxincludegraphics{{8e1d15ddfe810ce7dbb2eed7e8f77fe665b0dbd55e16bed2bd257f7354616cac}.png}

\end{sphinxuseclass}\end{sphinxVerbatimOutput}

\end{sphinxuseclass}
\sphinxstepscope


\section{Herron Topic 6 \sphinxhyphen{} Practice (Wednesday 2:45 PM, Section 2)}
\label{\detokenize{herron_06_practice_02:herron-topic-6-practice-wednesday-2-45-pm-section-2}}\label{\detokenize{herron_06_practice_02::doc}}

\subsection{Announcements}
\label{\detokenize{herron_06_practice_02:announcements}}\begin{itemize}
\item {} 
\sphinxAtStartPar
Quiz 6 mean and median were about \textasciitilde{}\textasciitilde{}90\% and 96\%\textasciitilde{}\textasciitilde{}  92\% and 100\%
\begin{itemize}
\item {} 
\sphinxAtStartPar
Please see the solution on Canvas and let me know if you have any questions

\item {} 
\sphinxAtStartPar
Gradescope shows which hidden tests you missed

\item {} 
\sphinxAtStartPar
I recently widened the tolerance on two hidden tests to give full credit if you minimized variance instead of volatility

\end{itemize}

\item {} 
\sphinxAtStartPar
I posted {\hyperref[\detokenize{project_02::doc}]{\sphinxcrossref{\DUrole{doc,std,std-doc}{project 2}}}} about Bitcoin and gold as inflation and market risk hedges

\item {} 
\sphinxAtStartPar
Next week (week of 4/10) is the assessment exam
\begin{itemize}
\item {} 
\sphinxAtStartPar
MSFQ students must take it for 5\% of overall course grade

\item {} 
\sphinxAtStartPar
Non\sphinxhyphen{}MSQF students do not take it and weight their grades by only 95\%

\item {} 
\sphinxAtStartPar
20 multiple choice questions on the 6 core courses (corporate finance, investments, math, data analytics, empirical methods, and derivatives)

\item {} 
\sphinxAtStartPar
\sphinxstyleemphasis{\sphinxstylestrong{You must be in the classroom during your assigned date and time to take the MSFQ assessment exam}}

\item {} 
\sphinxAtStartPar
If there is interest, we will discuss five stylized facts of asset returns after the MSFQ assessment exam

\item {} 
\sphinxAtStartPar
We will also have our final quiz, quiz 7

\end{itemize}

\item {} 
\sphinxAtStartPar
The following week (week of 4/17) we will discuss Herron topic 5 (simulations)
\begin{itemize}
\item {} 
\sphinxAtStartPar
I will record a lecture video and complete a practice notebook

\item {} 
\sphinxAtStartPar
But we will reserve class time for group work

\item {} 
\sphinxAtStartPar
The class voted about 2\sphinxhyphen{}to\sphinxhyphen{}1 not to drop a topic, so this is a compromise, given that Monday is Patriot’s Day holiday

\end{itemize}

\item {} 
\sphinxAtStartPar
The following week (week of 4/24) we will reserve class time for group work
\begin{itemize}
\item {} 
\sphinxAtStartPar
Project 2 is due Wednesday, 4/26, at 11:59 PM

\item {} 
\sphinxAtStartPar
Teammates Review 2 is due Wednesday, 4/26, at 11:59 PM

\item {} 
\sphinxAtStartPar
30,000 DataCamp XP are due Friday, 4/28, at 11:59 PM

\end{itemize}

\item {} 
\sphinxAtStartPar
\sphinxstyleemphasis{\sphinxstylestrong{Somewhere in there, please reserve 10 minutes to complete a TRACE review for this course}}
\begin{itemize}
\item {} 
\sphinxAtStartPar
I cannot make you complete TRACE reviews

\item {} 
\sphinxAtStartPar
But they are very helpful

\item {} 
\sphinxAtStartPar
I change my courses every semester, hopefully for the better, based on TRACE reviews

\end{itemize}

\end{itemize}


\subsection{Practice}
\label{\detokenize{herron_06_practice_02:practice}}
\begin{sphinxuseclass}{cell}\begin{sphinxVerbatimInput}

\begin{sphinxuseclass}{cell_input}
\begin{sphinxVerbatim}[commandchars=\\\{\}]
\PYG{k+kn}{import} \PYG{n+nn}{matplotlib}\PYG{n+nn}{.}\PYG{n+nn}{pyplot} \PYG{k}{as} \PYG{n+nn}{plt}
\PYG{k+kn}{import} \PYG{n+nn}{numpy} \PYG{k}{as} \PYG{n+nn}{np}
\PYG{k+kn}{import} \PYG{n+nn}{pandas} \PYG{k}{as} \PYG{n+nn}{pd}
\end{sphinxVerbatim}

\end{sphinxuseclass}\end{sphinxVerbatimInput}

\end{sphinxuseclass}
\begin{sphinxuseclass}{cell}\begin{sphinxVerbatimInput}

\begin{sphinxuseclass}{cell_input}
\begin{sphinxVerbatim}[commandchars=\\\{\}]
\PYG{o}{\PYGZpc{}}\PYG{k}{config} InlineBackend.figure\PYGZus{}format = \PYGZsq{}retina\PYGZsq{}
\PYG{o}{\PYGZpc{}}\PYG{k}{precision} 4
\PYG{n}{pd}\PYG{o}{.}\PYG{n}{options}\PYG{o}{.}\PYG{n}{display}\PYG{o}{.}\PYG{n}{float\PYGZus{}format} \PYG{o}{=} \PYG{l+s+s1}{\PYGZsq{}}\PYG{l+s+si}{\PYGZob{}:.4f\PYGZcb{}}\PYG{l+s+s1}{\PYGZsq{}}\PYG{o}{.}\PYG{n}{format}
\end{sphinxVerbatim}

\end{sphinxuseclass}\end{sphinxVerbatimInput}

\end{sphinxuseclass}
\begin{sphinxuseclass}{cell}\begin{sphinxVerbatimInput}

\begin{sphinxuseclass}{cell_input}
\begin{sphinxVerbatim}[commandchars=\\\{\}]
\PYG{k+kn}{import} \PYG{n+nn}{yfinance} \PYG{k}{as} \PYG{n+nn}{yf}
\PYG{k+kn}{import} \PYG{n+nn}{pandas\PYGZus{}datareader} \PYG{k}{as} \PYG{n+nn}{pdr}
\PYG{k+kn}{import} \PYG{n+nn}{requests\PYGZus{}cache}
\PYG{n}{session} \PYG{o}{=} \PYG{n}{requests\PYGZus{}cache}\PYG{o}{.}\PYG{n}{CachedSession}\PYG{p}{(}\PYG{p}{)}
\end{sphinxVerbatim}

\end{sphinxuseclass}\end{sphinxVerbatimInput}

\end{sphinxuseclass}

\subsubsection{Reimplement the equal\sphinxhyphen{}weighted momentum investing strategy from the lecture notebook}
\label{\detokenize{herron_06_practice_02:reimplement-the-equal-weighted-momentum-investing-strategy-from-the-lecture-notebook}}
\sphinxAtStartPar
Try to use a few cells and temporary variables as you can (i.e., perform calculations inside \sphinxcode{\sphinxupquote{pd.concat()}}).

\begin{sphinxuseclass}{cell}\begin{sphinxVerbatimInput}

\begin{sphinxuseclass}{cell_input}
\begin{sphinxVerbatim}[commandchars=\\\{\}]
\PYG{n}{crsp} \PYG{o}{=} \PYG{p}{(}
    \PYG{n}{pd}\PYG{o}{.}\PYG{n}{read\PYGZus{}csv}\PYG{p}{(}
        \PYG{n}{filepath\PYGZus{}or\PYGZus{}buffer}\PYG{o}{=}\PYG{l+s+s1}{\PYGZsq{}}\PYG{l+s+s1}{crsp.csv}\PYG{l+s+s1}{\PYGZsq{}}\PYG{p}{,}
        \PYG{n}{parse\PYGZus{}dates}\PYG{o}{=}\PYG{p}{[}\PYG{l+s+s1}{\PYGZsq{}}\PYG{l+s+s1}{date}\PYG{l+s+s1}{\PYGZsq{}}\PYG{p}{]}\PYG{p}{,}
        \PYG{n}{na\PYGZus{}values}\PYG{o}{=}\PYG{p}{[}\PYG{l+s+s1}{\PYGZsq{}}\PYG{l+s+s1}{A}\PYG{l+s+s1}{\PYGZsq{}}\PYG{p}{,} \PYG{l+s+s1}{\PYGZsq{}}\PYG{l+s+s1}{B}\PYG{l+s+s1}{\PYGZsq{}}\PYG{p}{,} \PYG{l+s+s1}{\PYGZsq{}}\PYG{l+s+s1}{C}\PYG{l+s+s1}{\PYGZsq{}}\PYG{p}{]} \PYG{c+c1}{\PYGZsh{} CRSP uses letter codes to provide additional information, which we can ignore}
    \PYG{p}{)}
    \PYG{o}{.}\PYG{n}{assign}\PYG{p}{(}
        \PYG{n}{date}\PYG{o}{=}\PYG{k}{lambda} \PYG{n}{x}\PYG{p}{:} \PYG{n}{x}\PYG{p}{[}\PYG{l+s+s1}{\PYGZsq{}}\PYG{l+s+s1}{date}\PYG{l+s+s1}{\PYGZsq{}}\PYG{p}{]}\PYG{o}{.}\PYG{n}{dt}\PYG{o}{.}\PYG{n}{to\PYGZus{}period}\PYG{p}{(}\PYG{n}{freq}\PYG{o}{=}\PYG{l+s+s1}{\PYGZsq{}}\PYG{l+s+s1}{M}\PYG{l+s+s1}{\PYGZsq{}}\PYG{p}{)}\PYG{p}{,} \PYG{c+c1}{\PYGZsh{} returns span a month, so I prefer to work with periods instead of dates}
        \PYG{n}{ME}\PYG{o}{=}\PYG{k}{lambda} \PYG{n}{x}\PYG{p}{:} \PYG{n}{x}\PYG{p}{[}\PYG{l+s+s1}{\PYGZsq{}}\PYG{l+s+s1}{PRC}\PYG{l+s+s1}{\PYGZsq{}}\PYG{p}{]}\PYG{o}{.}\PYG{n}{abs}\PYG{p}{(}\PYG{p}{)} \PYG{o}{*} \PYG{n}{x}\PYG{p}{[}\PYG{l+s+s1}{\PYGZsq{}}\PYG{l+s+s1}{SHROUT}\PYG{l+s+s1}{\PYGZsq{}}\PYG{p}{]}
    \PYG{p}{)}
    \PYG{o}{.}\PYG{n}{rename\PYGZus{}axis}\PYG{p}{(}\PYG{n}{columns}\PYG{o}{=}\PYG{l+s+s1}{\PYGZsq{}}\PYG{l+s+s1}{Variable}\PYG{l+s+s1}{\PYGZsq{}}\PYG{p}{)}
    \PYG{o}{.}\PYG{n}{set\PYGZus{}index}\PYG{p}{(}\PYG{p}{[}\PYG{l+s+s1}{\PYGZsq{}}\PYG{l+s+s1}{PERMNO}\PYG{l+s+s1}{\PYGZsq{}}\PYG{p}{,} \PYG{l+s+s1}{\PYGZsq{}}\PYG{l+s+s1}{date}\PYG{l+s+s1}{\PYGZsq{}}\PYG{p}{]}\PYG{p}{)}
\PYG{p}{)}
\end{sphinxVerbatim}

\end{sphinxuseclass}\end{sphinxVerbatimInput}

\end{sphinxuseclass}
\sphinxAtStartPar
Please the {\hyperref[\detokenize{herron_06_lecture::doc}]{\sphinxcrossref{\DUrole{doc,std,std-doc}{Herron Topic 6 lecture notebook}}}} for variable definitions.

\begin{sphinxuseclass}{cell}\begin{sphinxVerbatimInput}

\begin{sphinxuseclass}{cell_input}
\begin{sphinxVerbatim}[commandchars=\\\{\}]
\PYG{n}{crsp}\PYG{o}{.}\PYG{n}{head}\PYG{p}{(}\PYG{p}{)}
\end{sphinxVerbatim}

\end{sphinxuseclass}\end{sphinxVerbatimInput}
\begin{sphinxVerbatimOutput}

\begin{sphinxuseclass}{cell_output}
\begin{sphinxVerbatim}[commandchars=\\\{\}]
Variable        SHRCD     PRC     RET    SHROUT         ME
PERMNO date                                               
10000  1986\PYGZhy{}01     10 \PYGZhy{}4.3750     NaN 3680.0000 16100.0000
       1986\PYGZhy{}02     10 \PYGZhy{}3.2500 \PYGZhy{}0.2571 3680.0000 11960.0000
       1986\PYGZhy{}03     10 \PYGZhy{}4.4375  0.3654 3680.0000 16330.0000
       1986\PYGZhy{}04     10 \PYGZhy{}4.0000 \PYGZhy{}0.0986 3793.0000 15172.0000
       1986\PYGZhy{}05     10 \PYGZhy{}3.1094 \PYGZhy{}0.2227 3793.0000 11793.8783
\end{sphinxVerbatim}

\end{sphinxuseclass}\end{sphinxVerbatimOutput}

\end{sphinxuseclass}
\sphinxAtStartPar
To implement a momentum strategy, we need:
\begin{enumerate}
\sphinxsetlistlabels{\arabic}{enumi}{enumii}{}{.}%
\item {} 
\sphinxAtStartPar
1\sphinxhyphen{}month returns, which are the returns we receive for each holding period

\item {} 
\sphinxAtStartPar
11\sphinxhyphen{}month returns, which are the returns (from month \sphinxhyphen{}12 to month \sphinxhyphen{}2) that we use to rank stocks and assign to momentum portfolios

\item {} 
\sphinxAtStartPar
Momentum portfolio assignments based on 11\sphinxhyphen{}month trailing returns

\end{enumerate}

\begin{sphinxuseclass}{cell}\begin{sphinxVerbatimInput}

\begin{sphinxuseclass}{cell_input}
\begin{sphinxVerbatim}[commandchars=\\\{\}]
\PYG{n}{ret\PYGZus{}1m} \PYG{o}{=} \PYG{n}{crsp}\PYG{p}{[}\PYG{l+s+s1}{\PYGZsq{}}\PYG{l+s+s1}{RET}\PYG{l+s+s1}{\PYGZsq{}}\PYG{p}{]}\PYG{o}{.}\PYG{n}{unstack}\PYG{p}{(}\PYG{l+s+s1}{\PYGZsq{}}\PYG{l+s+s1}{PERMNO}\PYG{l+s+s1}{\PYGZsq{}}\PYG{p}{)}
\end{sphinxVerbatim}

\end{sphinxuseclass}\end{sphinxVerbatimInput}

\end{sphinxuseclass}
\sphinxAtStartPar
There is not a \sphinxcode{\sphinxupquote{.prod()}} method for \sphinxcode{\sphinxupquote{.rolling()}}, so the fastest way to calculate 11\sphinxhyphen{}month rolling returns is to:
\begin{enumerate}
\sphinxsetlistlabels{\arabic}{enumi}{enumii}{}{.}%
\item {} 
\sphinxAtStartPar
Convert simple returns to log returns with \sphinxcode{\sphinxupquote{.pipe(np.log1p)}}

\item {} 
\sphinxAtStartPar
Calculate 11\sphinxhyphen{}month rolling \sphinxstyleemphasis{log} returns with \sphinxcode{\sphinxupquote{.rolling(11).sum()}}

\item {} 
\sphinxAtStartPar
Convert log returns to simple returns with \sphinxcode{\sphinxupquote{.pipe(np.expm1)}}

\end{enumerate}

\begin{sphinxuseclass}{cell}\begin{sphinxVerbatimInput}

\begin{sphinxuseclass}{cell_input}
\begin{sphinxVerbatim}[commandchars=\\\{\}]
\PYG{n}{ret\PYGZus{}11m} \PYG{o}{=} \PYG{n}{ret\PYGZus{}1m}\PYG{o}{.}\PYG{n}{pipe}\PYG{p}{(}\PYG{n}{np}\PYG{o}{.}\PYG{n}{log1p}\PYG{p}{)}\PYG{o}{.}\PYG{n}{rolling}\PYG{p}{(}\PYG{l+m+mi}{11}\PYG{p}{)}\PYG{o}{.}\PYG{n}{sum}\PYG{p}{(}\PYG{p}{)}\PYG{o}{.}\PYG{n}{pipe}\PYG{p}{(}\PYG{n}{np}\PYG{o}{.}\PYG{n}{expm1}\PYG{p}{)}
\end{sphinxVerbatim}

\end{sphinxuseclass}\end{sphinxVerbatimInput}

\end{sphinxuseclass}
\sphinxAtStartPar
Then we use \sphinxcode{\sphinxupquote{pd.qcut()}} to assign \sphinxcode{\sphinxupquote{ret\_11m}} to ten momentum portfolios.
Here is a simple example:

\begin{sphinxuseclass}{cell}\begin{sphinxVerbatimInput}

\begin{sphinxuseclass}{cell_input}
\begin{sphinxVerbatim}[commandchars=\\\{\}]
\PYG{n}{pd}\PYG{o}{.}\PYG{n}{qcut}\PYG{p}{(}\PYG{n}{x}\PYG{o}{=}\PYG{n}{np}\PYG{o}{.}\PYG{n}{arange}\PYG{p}{(}\PYG{l+m+mi}{10}\PYG{p}{)}\PYG{p}{,} \PYG{n}{q}\PYG{o}{=}\PYG{l+m+mi}{2}\PYG{p}{,} \PYG{n}{labels}\PYG{o}{=}\PYG{k+kc}{False}\PYG{p}{)}
\end{sphinxVerbatim}

\end{sphinxuseclass}\end{sphinxVerbatimInput}
\begin{sphinxVerbatimOutput}

\begin{sphinxuseclass}{cell_output}
\begin{sphinxVerbatim}[commandchars=\\\{\}]
array([0, 0, 0, 0, 0, 1, 1, 1, 1, 1], dtype=int64)
\end{sphinxVerbatim}

\end{sphinxuseclass}\end{sphinxVerbatimOutput}

\end{sphinxuseclass}
\sphinxAtStartPar
Here are the momentum portfolios:

\begin{sphinxuseclass}{cell}\begin{sphinxVerbatimInput}

\begin{sphinxuseclass}{cell_input}
\begin{sphinxVerbatim}[commandchars=\\\{\}]
\PYG{n}{port\PYGZus{}11m} \PYG{o}{=} \PYG{n}{ret\PYGZus{}11m}\PYG{o}{.}\PYG{n}{dropna}\PYG{p}{(}\PYG{n}{how}\PYG{o}{=}\PYG{l+s+s1}{\PYGZsq{}}\PYG{l+s+s1}{all}\PYG{l+s+s1}{\PYGZsq{}}\PYG{p}{)}\PYG{o}{.}\PYG{n}{apply}\PYG{p}{(}\PYG{n}{pd}\PYG{o}{.}\PYG{n}{qcut}\PYG{p}{,} \PYG{n}{q}\PYG{o}{=}\PYG{l+m+mi}{10}\PYG{p}{,} \PYG{n}{labels}\PYG{o}{=}\PYG{k+kc}{False}\PYG{p}{,} \PYG{n}{axis}\PYG{o}{=}\PYG{l+m+mi}{1}\PYG{p}{)}
\end{sphinxVerbatim}

\end{sphinxuseclass}\end{sphinxVerbatimInput}

\end{sphinxuseclass}
\sphinxAtStartPar
Short\sphinxhyphen{}term reversal, momentum, and long\sphinxhyphen{}term reversal represent distinct aspects of asset price behavior that have been observed and documented in the financial markets. As a researcher, I am particularly interested in understanding the underlying factors that contribute to these market anomalies and their implications for trading strategies and portfolio management.

\sphinxAtStartPar
Short\sphinxhyphen{}term reversal captures the tendency of asset prices to reverse direction in a brief period, typically days to weeks. This phenomenon can be associated with market overreactions to news events, temporary liquidity constraints, or other transient factors. Astute investors may capitalize on these short\sphinxhyphen{}term reversals by adopting a contrarian approach, buying assets that have recently underperformed and selling those that have outperformed.

\sphinxAtStartPar
Momentum represents the persistence of asset price trends over a short to medium\sphinxhyphen{}term horizon, typically ranging from 3 to 12 months. This market anomaly suggests that assets with strong recent performance continue to outperform, while poorly performing assets continue to underperform. Momentum can be linked to factors such as investors’ behavioral biases, positive feedback trading, and herd mentality. To exploit momentum, investors may adopt a trend\sphinxhyphen{}following strategy, buying assets with strong recent performance and selling those with weak performance.

\sphinxAtStartPar
Long\sphinxhyphen{}term reversal, often referred to as mean reversion, is the tendency of asset prices to revert to their long\sphinxhyphen{}term average or fundamental value over an extended period, typically several years. This phenomenon is thought to result from market participants gradually recognizing and correcting mispricing, leading to price adjustments. Investors following a long\sphinxhyphen{}term reversal strategy may focus on assets with extreme valuation levels, buying undervalued assets and selling overvalued ones, with the expectation that prices will eventually revert to their long\sphinxhyphen{}term mean.

\sphinxAtStartPar
\sphinxstyleemphasis{\sphinxstylestrong{To avoid short\sphinxhyphen{}term reversal, we skip one month between 11\sphinxhyphen{}month returns and portfolio formation.}}

\begin{sphinxuseclass}{cell}\begin{sphinxVerbatimInput}

\begin{sphinxuseclass}{cell_input}
\begin{sphinxVerbatim}[commandchars=\\\{\}]
\PYG{n}{mom} \PYG{o}{=} \PYG{p}{(}
    \PYG{n}{pd}\PYG{o}{.}\PYG{n}{concat}\PYG{p}{(}
        \PYG{n}{objs}\PYG{o}{=}\PYG{p}{[}
            \PYG{n}{ret\PYGZus{}1m}\PYG{p}{,} 
            \PYG{n}{ret\PYGZus{}11m}\PYG{o}{.}\PYG{n}{shift}\PYG{p}{(}\PYG{l+m+mi}{2}\PYG{p}{)}\PYG{p}{,}
            \PYG{n}{port\PYGZus{}11m}\PYG{o}{.}\PYG{n}{shift}\PYG{p}{(}\PYG{l+m+mi}{2}\PYG{p}{)}\PYG{p}{,} \PYG{c+c1}{\PYGZsh{} base portfolio assignments on avail. info. and avoid short\PYGZhy{}term reversal}
            \PYG{n}{crsp}\PYG{p}{[}\PYG{l+s+s1}{\PYGZsq{}}\PYG{l+s+s1}{ME}\PYG{l+s+s1}{\PYGZsq{}}\PYG{p}{]}\PYG{o}{.}\PYG{n}{unstack}\PYG{p}{(}\PYG{l+s+s1}{\PYGZsq{}}\PYG{l+s+s1}{PERMNO}\PYG{l+s+s1}{\PYGZsq{}}\PYG{p}{)}\PYG{o}{.}\PYG{n}{shift}\PYG{p}{(}\PYG{l+m+mi}{1}\PYG{p}{)}
        \PYG{p}{]}\PYG{p}{,}
        \PYG{n}{axis}\PYG{o}{=}\PYG{l+m+mi}{1}\PYG{p}{,}
        \PYG{n}{keys}\PYG{o}{=}\PYG{p}{[}\PYG{l+s+s1}{\PYGZsq{}}\PYG{l+s+s1}{Return}\PYG{l+s+s1}{\PYGZsq{}}\PYG{p}{,} \PYG{l+s+s1}{\PYGZsq{}}\PYG{l+s+s1}{Trailing Return}\PYG{l+s+s1}{\PYGZsq{}}\PYG{p}{,} \PYG{l+s+s1}{\PYGZsq{}}\PYG{l+s+s1}{Portfolio}\PYG{l+s+s1}{\PYGZsq{}}\PYG{p}{,} \PYG{l+s+s1}{\PYGZsq{}}\PYG{l+s+s1}{Trailing ME}\PYG{l+s+s1}{\PYGZsq{}}\PYG{p}{]}\PYG{p}{,}
        \PYG{n}{names}\PYG{o}{=}\PYG{p}{[}\PYG{l+s+s1}{\PYGZsq{}}\PYG{l+s+s1}{Variable}\PYG{l+s+s1}{\PYGZsq{}}\PYG{p}{]}
    \PYG{p}{)}
    \PYG{o}{.}\PYG{n}{stack}\PYG{p}{(}\PYG{l+s+s1}{\PYGZsq{}}\PYG{l+s+s1}{PERMNO}\PYG{l+s+s1}{\PYGZsq{}}\PYG{p}{)}
    \PYG{o}{.}\PYG{n}{dropna}\PYG{p}{(}\PYG{p}{)}
    \PYG{o}{.}\PYG{n}{assign}\PYG{p}{(}\PYG{n}{Portfolio}\PYG{o}{=}\PYG{k}{lambda} \PYG{n}{x}\PYG{p}{:} \PYG{n}{x}\PYG{p}{[}\PYG{l+s+s1}{\PYGZsq{}}\PYG{l+s+s1}{Portfolio}\PYG{l+s+s1}{\PYGZsq{}}\PYG{p}{]}\PYG{o}{.}\PYG{n}{astype}\PYG{p}{(}\PYG{n+nb}{int}\PYG{p}{)} \PYG{o}{+} \PYG{l+m+mi}{1}\PYG{p}{)}
\PYG{p}{)}
\end{sphinxVerbatim}

\end{sphinxuseclass}\end{sphinxVerbatimInput}

\end{sphinxuseclass}
\begin{sphinxuseclass}{cell}\begin{sphinxVerbatimInput}

\begin{sphinxuseclass}{cell_input}
\begin{sphinxVerbatim}[commandchars=\\\{\}]
\PYG{n}{mom}\PYG{o}{.}\PYG{n}{head}\PYG{p}{(}\PYG{p}{)}
\end{sphinxVerbatim}

\end{sphinxuseclass}\end{sphinxVerbatimInput}
\begin{sphinxVerbatimOutput}

\begin{sphinxuseclass}{cell_output}
\begin{sphinxVerbatim}[commandchars=\\\{\}]
Variable        Return  Trailing Return  Portfolio  Trailing ME
date    PERMNO                                                 
1927\PYGZhy{}01 10022  \PYGZhy{}0.0759           0.1182          8   11424.0000
        10030   0.0095           0.0093          6   21450.0000
        10057  \PYGZhy{}0.0510          \PYGZhy{}0.5918          1    3062.5000
        10073   0.0946          \PYGZhy{}0.4286          2    1276.5000
        10081  \PYGZhy{}0.0750          \PYGZhy{}0.4194          2    5960.0000
\end{sphinxVerbatim}

\end{sphinxuseclass}\end{sphinxVerbatimOutput}

\end{sphinxuseclass}
\begin{sphinxuseclass}{cell}\begin{sphinxVerbatimInput}

\begin{sphinxuseclass}{cell_input}
\begin{sphinxVerbatim}[commandchars=\\\{\}]
\PYG{n}{mom\PYGZus{}ew} \PYG{o}{=} \PYG{p}{(}
    \PYG{n}{mom}
    \PYG{o}{.}\PYG{n}{groupby}\PYG{p}{(}\PYG{p}{[}\PYG{l+s+s1}{\PYGZsq{}}\PYG{l+s+s1}{date}\PYG{l+s+s1}{\PYGZsq{}}\PYG{p}{,} \PYG{l+s+s1}{\PYGZsq{}}\PYG{l+s+s1}{Portfolio}\PYG{l+s+s1}{\PYGZsq{}}\PYG{p}{]}\PYG{p}{)}
    \PYG{p}{[}\PYG{l+s+s1}{\PYGZsq{}}\PYG{l+s+s1}{Return}\PYG{l+s+s1}{\PYGZsq{}}\PYG{p}{]}
    \PYG{o}{.}\PYG{n}{mean}\PYG{p}{(}\PYG{p}{)}
    \PYG{o}{.}\PYG{n}{unstack}\PYG{p}{(}\PYG{l+s+s1}{\PYGZsq{}}\PYG{l+s+s1}{Portfolio}\PYG{l+s+s1}{\PYGZsq{}}\PYG{p}{)}
    \PYG{o}{.}\PYG{n}{add\PYGZus{}prefix}\PYG{p}{(}\PYG{l+s+s1}{\PYGZsq{}}\PYG{l+s+s1}{Mom }\PYG{l+s+s1}{\PYGZsq{}}\PYG{p}{)}
\PYG{p}{)}
\end{sphinxVerbatim}

\end{sphinxuseclass}\end{sphinxVerbatimInput}

\end{sphinxuseclass}
\sphinxAtStartPar
We see a (nearly) monotonic relation between momentum portfolios (based on trailing 11\sphinxhyphen{}month returns) and holding period returns!

\begin{sphinxuseclass}{cell}\begin{sphinxVerbatimInput}

\begin{sphinxuseclass}{cell_input}
\begin{sphinxVerbatim}[commandchars=\\\{\}]
\PYG{n}{mom\PYGZus{}ew}\PYG{o}{.}\PYG{n}{head}\PYG{p}{(}\PYG{p}{)}
\end{sphinxVerbatim}

\end{sphinxuseclass}\end{sphinxVerbatimInput}
\begin{sphinxVerbatimOutput}

\begin{sphinxuseclass}{cell_output}
\begin{sphinxVerbatim}[commandchars=\\\{\}]
Portfolio   Mom 1   Mom 2   Mom 3   Mom 4   Mom 5   Mom 6  Mom 7   Mom 8  \PYGZbs{}
date                                                                       
1927\PYGZhy{}01   \PYGZhy{}0.0006  0.0149  0.0425  0.0378  0.0009  0.0150 0.0124  0.0076   
1927\PYGZhy{}02    0.0641  0.0484  0.0866  0.0745  0.0436  0.0547 0.0561  0.0403   
1927\PYGZhy{}03   \PYGZhy{}0.0530 \PYGZhy{}0.0381 \PYGZhy{}0.0243 \PYGZhy{}0.0408 \PYGZhy{}0.0192 \PYGZhy{}0.0201 0.0097 \PYGZhy{}0.0084   
1927\PYGZhy{}04    0.0230 \PYGZhy{}0.0085 \PYGZhy{}0.0054 \PYGZhy{}0.0061  0.0002 \PYGZhy{}0.0032 0.0138  0.0030   
1927\PYGZhy{}05    0.0069  0.0353  0.0604  0.0536  0.0700  0.0950 0.0796  0.0838   

Portfolio   Mom 9  Mom 10  
date                       
1927\PYGZhy{}01   \PYGZhy{}0.0044  0.0126  
1927\PYGZhy{}02    0.0579  0.0541  
1927\PYGZhy{}03   \PYGZhy{}0.0058  0.0012  
1927\PYGZhy{}04    0.0323  0.0421  
1927\PYGZhy{}05    0.0803  0.0886  
\end{sphinxVerbatim}

\end{sphinxuseclass}\end{sphinxVerbatimOutput}

\end{sphinxuseclass}
\begin{sphinxuseclass}{cell}\begin{sphinxVerbatimInput}

\begin{sphinxuseclass}{cell_input}
\begin{sphinxVerbatim}[commandchars=\\\{\}]
\PYG{n}{mom\PYGZus{}ew}\PYG{o}{.}\PYG{n}{mean}\PYG{p}{(}\PYG{p}{)}\PYG{o}{.}\PYG{n}{mul}\PYG{p}{(}\PYG{l+m+mi}{100}\PYG{p}{)}\PYG{o}{.}\PYG{n}{plot}\PYG{p}{(}\PYG{n}{kind}\PYG{o}{=}\PYG{l+s+s1}{\PYGZsq{}}\PYG{l+s+s1}{bar}\PYG{l+s+s1}{\PYGZsq{}}\PYG{p}{)}
\PYG{n}{plt}\PYG{o}{.}\PYG{n}{ylabel}\PYG{p}{(}\PYG{l+s+s1}{\PYGZsq{}}\PYG{l+s+s1}{Mean Monthly Return (}\PYG{l+s+s1}{\PYGZpc{}}\PYG{l+s+s1}{)}\PYG{l+s+s1}{\PYGZsq{}}\PYG{p}{)}
\PYG{n}{plt}\PYG{o}{.}\PYG{n}{title}\PYG{p}{(}\PYG{l+s+s1}{\PYGZsq{}}\PYG{l+s+s1}{Momentum Strategy}\PYG{l+s+se}{\PYGZbs{}n}\PYG{l+s+s1}{ Equal\PYGZhy{}Weighted Portfolios}\PYG{l+s+se}{\PYGZbs{}n}\PYG{l+s+s1}{Formed on Months \PYGZhy{}12 to \PYGZhy{}2}\PYG{l+s+s1}{\PYGZsq{}}\PYG{p}{)}
\PYG{n}{plt}\PYG{o}{.}\PYG{n}{show}\PYG{p}{(}\PYG{p}{)}
\end{sphinxVerbatim}

\end{sphinxuseclass}\end{sphinxVerbatimInput}
\begin{sphinxVerbatimOutput}

\begin{sphinxuseclass}{cell_output}
\begin{sphinxVerbatim}[commandchars=\\\{\}]
\PYGZlt{}AxesSubplot:xlabel=\PYGZsq{}Portfolio\PYGZsq{}\PYGZgt{}
\end{sphinxVerbatim}

\noindent\sphinxincludegraphics{{9a2633b50c7a88523b27b196fefcfafd510d1f41f67efa8cf4263f504fe84fbe}.png}

\end{sphinxuseclass}\end{sphinxVerbatimOutput}

\end{sphinxuseclass}
\sphinxAtStartPar
This relation is stronger when we consider long\sphinxhyphen{}term, buy\sphinxhyphen{}and\sphinxhyphen{}hold returns!

\begin{sphinxuseclass}{cell}\begin{sphinxVerbatimInput}

\begin{sphinxuseclass}{cell_input}
\begin{sphinxVerbatim}[commandchars=\\\{\}]
\PYG{n}{mom\PYGZus{}ew}\PYG{o}{.}\PYG{n}{add}\PYG{p}{(}\PYG{l+m+mi}{1}\PYG{p}{)}\PYG{o}{.}\PYG{n}{cumprod}\PYG{p}{(}\PYG{p}{)}\PYG{o}{.}\PYG{n}{plot}\PYG{p}{(}\PYG{p}{)}
\PYG{n}{plt}\PYG{o}{.}\PYG{n}{semilogy}\PYG{p}{(}\PYG{p}{)}
\PYG{n}{plt}\PYG{o}{.}\PYG{n}{ylabel}\PYG{p}{(}\PYG{l+s+s1}{\PYGZsq{}}\PYG{l+s+s1}{Value of \PYGZdl{}1 Investment (\PYGZdl{})}\PYG{l+s+s1}{\PYGZsq{}}\PYG{p}{)}
\PYG{n}{plt}\PYG{o}{.}\PYG{n}{title}\PYG{p}{(}\PYG{l+s+s1}{\PYGZsq{}}\PYG{l+s+s1}{Momentum Strategy}\PYG{l+s+se}{\PYGZbs{}n}\PYG{l+s+s1}{ Equal\PYGZhy{}Weighted Portfolios}\PYG{l+s+se}{\PYGZbs{}n}\PYG{l+s+s1}{Formed on Months \PYGZhy{}12 to \PYGZhy{}2}\PYG{l+s+s1}{\PYGZsq{}}\PYG{p}{)}
\PYG{n}{plt}\PYG{o}{.}\PYG{n}{show}\PYG{p}{(}\PYG{p}{)}
\end{sphinxVerbatim}

\end{sphinxuseclass}\end{sphinxVerbatimInput}
\begin{sphinxVerbatimOutput}

\begin{sphinxuseclass}{cell_output}
\begin{sphinxVerbatim}[commandchars=\\\{\}]
[]
\end{sphinxVerbatim}

\noindent\sphinxincludegraphics{{ccabedf7af49b0007b47bea1b17251766a377ee7ca50d2f5a6cd203be10e48ac}.png}

\end{sphinxuseclass}\end{sphinxVerbatimOutput}

\end{sphinxuseclass}

\subsubsection{Add a long\sphinxhyphen{}short portfolio that is long portfolio 10 and short portfolio 1}
\label{\detokenize{herron_06_practice_02:add-a-long-short-portfolio-that-is-long-portfolio-10-and-short-portfolio-1}}
\sphinxAtStartPar
Call this long\sphinxhyphen{}short portfolio UMD.
What are the best and worst months for portfolios 1, 10, and UMD?

\begin{sphinxuseclass}{cell}\begin{sphinxVerbatimInput}

\begin{sphinxuseclass}{cell_input}
\begin{sphinxVerbatim}[commandchars=\\\{\}]
\PYG{n}{mom\PYGZus{}ew} \PYG{o}{=} \PYG{n}{mom\PYGZus{}ew}\PYG{o}{.}\PYG{n}{assign}\PYG{p}{(}\PYG{n}{UMD}\PYG{o}{=}\PYG{k}{lambda} \PYG{n}{x}\PYG{p}{:} \PYG{n}{x}\PYG{p}{[}\PYG{l+s+s1}{\PYGZsq{}}\PYG{l+s+s1}{Mom 10}\PYG{l+s+s1}{\PYGZsq{}}\PYG{p}{]} \PYG{o}{\PYGZhy{}} \PYG{n}{x}\PYG{p}{[}\PYG{l+s+s1}{\PYGZsq{}}\PYG{l+s+s1}{Mom 1}\PYG{l+s+s1}{\PYGZsq{}}\PYG{p}{]}\PYG{p}{)}
\end{sphinxVerbatim}

\end{sphinxuseclass}\end{sphinxVerbatimInput}

\end{sphinxuseclass}
\sphinxAtStartPar
The worst month for the long\sphinxhyphen{}short portfolio is during the Great Depression.

\begin{sphinxuseclass}{cell}\begin{sphinxVerbatimInput}

\begin{sphinxuseclass}{cell_input}
\begin{sphinxVerbatim}[commandchars=\\\{\}]
\PYG{n}{mom\PYGZus{}ew}\PYG{p}{[}\PYG{l+s+s1}{\PYGZsq{}}\PYG{l+s+s1}{UMD}\PYG{l+s+s1}{\PYGZsq{}}\PYG{p}{]}\PYG{o}{.}\PYG{n}{idxmin}\PYG{p}{(}\PYG{p}{)}
\end{sphinxVerbatim}

\end{sphinxuseclass}\end{sphinxVerbatimInput}
\begin{sphinxVerbatimOutput}

\begin{sphinxuseclass}{cell_output}
\begin{sphinxVerbatim}[commandchars=\\\{\}]
Period(\PYGZsq{}1932\PYGZhy{}08\PYGZsq{}, \PYGZsq{}M\PYGZsq{})
\end{sphinxVerbatim}

\end{sphinxuseclass}\end{sphinxVerbatimOutput}

\end{sphinxuseclass}
\sphinxAtStartPar
Momentum strategies appear to generate large absolute returns.
However, long\sphinxhyphen{}short momentum strategies are occasionally wiped out by large market recoveries, that the short side of the portfolio does not survive in practice.
During downturns, the short side of the portfolio (i.e., the loser stocks) has a high beta.
When the market unexpectedly and quickly recovers, the exaggerated recovery in the high\sphinxhyphen{}beta stocks wipes out the long\sphinxhyphen{}short portfolio.
For example, if we invest \$1 in the long\sphinxhyphen{}short portfolio at the end of 1929, the long\sphinxhyphen{}short portfolio is effectively wiped out during August 1932.

\begin{sphinxuseclass}{cell}\begin{sphinxVerbatimInput}

\begin{sphinxuseclass}{cell_input}
\begin{sphinxVerbatim}[commandchars=\\\{\}]
\PYG{n}{mom\PYGZus{}ew}\PYG{o}{.}\PYG{n}{loc}\PYG{p}{[}\PYG{l+s+s1}{\PYGZsq{}}\PYG{l+s+s1}{1930}\PYG{l+s+s1}{\PYGZsq{}}\PYG{p}{:}\PYG{l+s+s1}{\PYGZsq{}}\PYG{l+s+s1}{1932}\PYG{l+s+s1}{\PYGZsq{}}\PYG{p}{,} \PYG{p}{[}\PYG{l+s+s1}{\PYGZsq{}}\PYG{l+s+s1}{Mom 1}\PYG{l+s+s1}{\PYGZsq{}}\PYG{p}{,} \PYG{l+s+s1}{\PYGZsq{}}\PYG{l+s+s1}{Mom 10}\PYG{l+s+s1}{\PYGZsq{}}\PYG{p}{,} \PYG{l+s+s1}{\PYGZsq{}}\PYG{l+s+s1}{UMD}\PYG{l+s+s1}{\PYGZsq{}}\PYG{p}{]}\PYG{p}{]}\PYG{o}{.}\PYG{n}{add}\PYG{p}{(}\PYG{l+m+mi}{1}\PYG{p}{)}\PYG{o}{.}\PYG{n}{cumprod}\PYG{p}{(}\PYG{p}{)}\PYG{o}{.}\PYG{n}{plot}\PYG{p}{(}\PYG{p}{)}
\PYG{n}{plt}\PYG{o}{.}\PYG{n}{ylabel}\PYG{p}{(}\PYG{l+s+s1}{\PYGZsq{}}\PYG{l+s+s1}{Value of \PYGZdl{}1 Investment (\PYGZdl{})}\PYG{l+s+s1}{\PYGZsq{}}\PYG{p}{)}
\PYG{n}{plt}\PYG{o}{.}\PYG{n}{title}\PYG{p}{(}\PYG{l+s+s1}{\PYGZsq{}}\PYG{l+s+s1}{Momentum Strategy}\PYG{l+s+se}{\PYGZbs{}n}\PYG{l+s+s1}{ Equal\PYGZhy{}Weighted Portfolios}\PYG{l+s+se}{\PYGZbs{}n}\PYG{l+s+s1}{Formed on Months \PYGZhy{}12 to \PYGZhy{}2}\PYG{l+s+s1}{\PYGZsq{}}\PYG{p}{)}
\PYG{n}{plt}\PYG{o}{.}\PYG{n}{show}\PYG{p}{(}\PYG{p}{)}
\end{sphinxVerbatim}

\end{sphinxuseclass}\end{sphinxVerbatimInput}
\begin{sphinxVerbatimOutput}

\begin{sphinxuseclass}{cell_output}
\noindent\sphinxincludegraphics{{4162e1d75dbe30fb44095151c1aa1734e90e085b17d4b8236c0f127dd64e09f2}.png}

\end{sphinxuseclass}\end{sphinxVerbatimOutput}

\end{sphinxuseclass}

\subsubsection{What are the CAPM and FF4 alphas for these equal\sphinxhyphen{}weighted portfolios?}
\label{\detokenize{herron_06_practice_02:what-are-the-capm-and-ff4-alphas-for-these-equal-weighted-portfolios}}
\begin{sphinxuseclass}{cell}\begin{sphinxVerbatimInput}

\begin{sphinxuseclass}{cell_input}
\begin{sphinxVerbatim}[commandchars=\\\{\}]
\PYG{k+kn}{import} \PYG{n+nn}{statsmodels}\PYG{n+nn}{.}\PYG{n+nn}{formula}\PYG{n+nn}{.}\PYG{n+nn}{api} \PYG{k}{as} \PYG{n+nn}{smf}
\end{sphinxVerbatim}

\end{sphinxuseclass}\end{sphinxVerbatimInput}

\end{sphinxuseclass}
\sphinxAtStartPar
\sphinxstyleemphasis{\sphinxstylestrong{After class, I made the following code more repeatable.}}

\begin{sphinxuseclass}{cell}\begin{sphinxVerbatimInput}

\begin{sphinxuseclass}{cell_input}
\begin{sphinxVerbatim}[commandchars=\\\{\}]
\PYG{k}{def} \PYG{n+nf}{capm}\PYG{p}{(}\PYG{n}{df}\PYG{p}{,} \PYG{n}{c}\PYG{p}{)}\PYG{p}{:}
    \PYG{k}{return} \PYG{n}{smf}\PYG{o}{.}\PYG{n}{ols}\PYG{p}{(}\PYG{n}{formula}\PYG{o}{=}\PYG{l+s+sa}{f}\PYG{l+s+s1}{\PYGZsq{}}\PYG{l+s+s1}{I(Q(}\PYG{l+s+s1}{\PYGZdq{}}\PYG{l+s+si}{\PYGZob{}}\PYG{n}{c}\PYG{l+s+si}{\PYGZcb{}}\PYG{l+s+s1}{\PYGZdq{}}\PYG{l+s+s1}{)\PYGZhy{}RF) \PYGZti{} Q(}\PYG{l+s+s1}{\PYGZdq{}}\PYG{l+s+s1}{Mkt\PYGZhy{}RF}\PYG{l+s+s1}{\PYGZdq{}}\PYG{l+s+s1}{)}\PYG{l+s+s1}{\PYGZsq{}}\PYG{p}{,} \PYG{n}{data}\PYG{o}{=}\PYG{n}{df}\PYG{p}{)}
\end{sphinxVerbatim}

\end{sphinxuseclass}\end{sphinxVerbatimInput}

\end{sphinxuseclass}
\begin{sphinxuseclass}{cell}\begin{sphinxVerbatimInput}

\begin{sphinxuseclass}{cell_input}
\begin{sphinxVerbatim}[commandchars=\\\{\}]
\PYG{n}{ff\PYGZus{}0} \PYG{o}{=} \PYG{n}{pdr}\PYG{o}{.}\PYG{n}{DataReader}\PYG{p}{(}
    \PYG{n}{name}\PYG{o}{=}\PYG{l+s+s1}{\PYGZsq{}}\PYG{l+s+s1}{F\PYGZhy{}F\PYGZus{}Research\PYGZus{}Data\PYGZus{}Factors}\PYG{l+s+s1}{\PYGZsq{}}\PYG{p}{,}
    \PYG{n}{data\PYGZus{}source}\PYG{o}{=}\PYG{l+s+s1}{\PYGZsq{}}\PYG{l+s+s1}{famafrench}\PYG{l+s+s1}{\PYGZsq{}}\PYG{p}{,}
    \PYG{n}{start}\PYG{o}{=}\PYG{l+s+s1}{\PYGZsq{}}\PYG{l+s+s1}{1900}\PYG{l+s+s1}{\PYGZsq{}}\PYG{p}{,}
    \PYG{n}{session}\PYG{o}{=}\PYG{n}{session}
\PYG{p}{)}
\end{sphinxVerbatim}

\end{sphinxuseclass}\end{sphinxVerbatimInput}

\end{sphinxuseclass}
\begin{sphinxuseclass}{cell}\begin{sphinxVerbatimInput}

\begin{sphinxuseclass}{cell_input}
\begin{sphinxVerbatim}[commandchars=\\\{\}]
\PYG{n}{ff\PYGZus{}mom} \PYG{o}{=} \PYG{n}{pdr}\PYG{o}{.}\PYG{n}{DataReader}\PYG{p}{(}
    \PYG{n}{name}\PYG{o}{=}\PYG{l+s+s1}{\PYGZsq{}}\PYG{l+s+s1}{F\PYGZhy{}F\PYGZus{}Momentum\PYGZus{}Factor}\PYG{l+s+s1}{\PYGZsq{}}\PYG{p}{,}
    \PYG{n}{data\PYGZus{}source}\PYG{o}{=}\PYG{l+s+s1}{\PYGZsq{}}\PYG{l+s+s1}{famafrench}\PYG{l+s+s1}{\PYGZsq{}}\PYG{p}{,}
    \PYG{n}{start}\PYG{o}{=}\PYG{l+s+s1}{\PYGZsq{}}\PYG{l+s+s1}{1900}\PYG{l+s+s1}{\PYGZsq{}}\PYG{p}{,}
    \PYG{n}{session}\PYG{o}{=}\PYG{n}{session}
\PYG{p}{)}

\PYG{n}{ff\PYGZus{}mom}\PYG{p}{[}\PYG{l+m+mi}{0}\PYG{p}{]}\PYG{o}{.}\PYG{n}{columns} \PYG{o}{=} \PYG{p}{[}\PYG{l+s+s1}{\PYGZsq{}}\PYG{l+s+s1}{Mom}\PYG{l+s+s1}{\PYGZsq{}}\PYG{p}{]} \PYG{c+c1}{\PYGZsh{} we need to rename the Mom factor to remove leading and trailing whitespace}
\end{sphinxVerbatim}

\end{sphinxuseclass}\end{sphinxVerbatimInput}

\end{sphinxuseclass}
\sphinxAtStartPar
The \sphinxcode{\sphinxupquote{get\_coefs()}} function accepts a data frame of returns and factors data, a factor model regression function name, and number \$n\$ for the first \$n\$ columns in the data frame.
The factor model regression function must return a statsmodel model.
The \sphinxcode{\sphinxupquote{get\_coefs()}} function returns a data frame of coefficient estimates and standard errors.
We can use the \sphinxcode{\sphinxupquote{get\_coefs()}} function to quickly apply a factor model regression function.

\begin{sphinxuseclass}{cell}\begin{sphinxVerbatimInput}

\begin{sphinxuseclass}{cell_input}
\begin{sphinxVerbatim}[commandchars=\\\{\}]
\PYG{k}{def} \PYG{n+nf}{get\PYGZus{}coefs}\PYG{p}{(}\PYG{n}{df}\PYG{p}{,} \PYG{n}{fun}\PYG{p}{,} \PYG{n}{n}\PYG{p}{)}\PYG{p}{:}
    \PYG{n}{models} \PYG{o}{=} \PYG{p}{[}\PYG{n}{df}\PYG{o}{.}\PYG{n}{pipe}\PYG{p}{(}\PYG{n}{fun}\PYG{p}{,} \PYG{n}{c}\PYG{p}{)} \PYG{k}{for} \PYG{n}{c} \PYG{o+ow}{in} \PYG{n}{df}\PYG{o}{.}\PYG{n}{columns}\PYG{p}{[}\PYG{p}{:}\PYG{n}{n}\PYG{p}{]}\PYG{p}{]}
    \PYG{n}{fits} \PYG{o}{=} \PYG{p}{[}\PYG{n}{m}\PYG{o}{.}\PYG{n}{fit}\PYG{p}{(}\PYG{p}{)} \PYG{k}{for} \PYG{n}{m} \PYG{o+ow}{in} \PYG{n}{models}\PYG{p}{]}
    \PYG{n}{coefs} \PYG{o}{=} \PYG{p}{(}
        \PYG{n}{pd}\PYG{o}{.}\PYG{n}{concat}\PYG{p}{(}
            \PYG{n}{objs}\PYG{o}{=}\PYG{p}{[}\PYG{n}{f}\PYG{o}{.}\PYG{n}{params} \PYG{k}{for} \PYG{n}{f} \PYG{o+ow}{in} \PYG{n}{fits}\PYG{p}{]}\PYG{p}{,}
            \PYG{n}{axis}\PYG{o}{=}\PYG{l+m+mi}{1}\PYG{p}{,}
            \PYG{n}{keys}\PYG{o}{=}\PYG{n}{df}\PYG{o}{.}\PYG{n}{columns}\PYG{p}{[}\PYG{p}{:}\PYG{l+m+mi}{10}\PYG{p}{]}
        \PYG{p}{)}
        \PYG{o}{.}\PYG{n}{transpose}\PYG{p}{(}\PYG{p}{)}
        \PYG{o}{.}\PYG{n}{rename\PYGZus{}axis}\PYG{p}{(}\PYG{n}{index}\PYG{o}{=}\PYG{l+s+s1}{\PYGZsq{}}\PYG{l+s+s1}{Portfolio}\PYG{l+s+s1}{\PYGZsq{}}\PYG{p}{,} \PYG{n}{columns}\PYG{o}{=}\PYG{l+s+s1}{\PYGZsq{}}\PYG{l+s+s1}{Variable}\PYG{l+s+s1}{\PYGZsq{}}\PYG{p}{)}
    \PYG{p}{)}

    \PYG{n}{ses} \PYG{o}{=} \PYG{p}{(}
        \PYG{n}{pd}\PYG{o}{.}\PYG{n}{concat}\PYG{p}{(}
            \PYG{n}{objs}\PYG{o}{=}\PYG{p}{[}\PYG{n}{f}\PYG{o}{.}\PYG{n}{bse} \PYG{k}{for} \PYG{n}{f} \PYG{o+ow}{in} \PYG{n}{fits}\PYG{p}{]}\PYG{p}{,}
            \PYG{n}{axis}\PYG{o}{=}\PYG{l+m+mi}{1}\PYG{p}{,}
            \PYG{n}{keys}\PYG{o}{=}\PYG{n}{df}\PYG{o}{.}\PYG{n}{columns}\PYG{p}{[}\PYG{p}{:}\PYG{l+m+mi}{10}\PYG{p}{]}
        \PYG{p}{)}
        \PYG{o}{.}\PYG{n}{transpose}\PYG{p}{(}\PYG{p}{)}
        \PYG{o}{.}\PYG{n}{rename\PYGZus{}axis}\PYG{p}{(}\PYG{n}{index}\PYG{o}{=}\PYG{l+s+s1}{\PYGZsq{}}\PYG{l+s+s1}{Portfolio}\PYG{l+s+s1}{\PYGZsq{}}\PYG{p}{,} \PYG{n}{columns}\PYG{o}{=}\PYG{l+s+s1}{\PYGZsq{}}\PYG{l+s+s1}{Variable}\PYG{l+s+s1}{\PYGZsq{}}\PYG{p}{)}
    \PYG{p}{)}

    \PYG{k}{return} \PYG{n}{pd}\PYG{o}{.}\PYG{n}{concat}\PYG{p}{(}
        \PYG{n}{objs}\PYG{o}{=}\PYG{p}{[}\PYG{n}{coefs}\PYG{p}{,} \PYG{n}{ses}\PYG{p}{]}\PYG{p}{,} 
        \PYG{n}{keys}\PYG{o}{=}\PYG{p}{[}\PYG{l+s+s1}{\PYGZsq{}}\PYG{l+s+s1}{Coef}\PYG{l+s+s1}{\PYGZsq{}}\PYG{p}{,} \PYG{l+s+s1}{\PYGZsq{}}\PYG{l+s+s1}{SE}\PYG{l+s+s1}{\PYGZsq{}}\PYG{p}{]}\PYG{p}{,} 
        \PYG{n}{names}\PYG{o}{=}\PYG{p}{[}\PYG{l+s+s1}{\PYGZsq{}}\PYG{l+s+s1}{Statistic}\PYG{l+s+s1}{\PYGZsq{}}\PYG{p}{]}\PYG{p}{,} 
        \PYG{n}{axis}\PYG{o}{=}\PYG{l+m+mi}{1}
    \PYG{p}{)}
\end{sphinxVerbatim}

\end{sphinxuseclass}\end{sphinxVerbatimInput}

\end{sphinxuseclass}
\sphinxAtStartPar
The \sphinxcode{\sphinxupquote{plot\_alpha()}} function accepts the output of the \sphinxcode{\sphinxupquote{get\_coefs()}} function and plots portfolio alphas.

\begin{sphinxuseclass}{cell}\begin{sphinxVerbatimInput}

\begin{sphinxuseclass}{cell_input}
\begin{sphinxVerbatim}[commandchars=\\\{\}]
\PYG{k}{def} \PYG{n+nf}{plot\PYGZus{}alpha}\PYG{p}{(}\PYG{n}{df}\PYG{p}{)}\PYG{p}{:}
    \PYG{n}{\PYGZus{}} \PYG{o}{=} \PYG{n}{df}\PYG{o}{.}\PYG{n}{swaplevel}\PYG{p}{(}\PYG{n}{axis}\PYG{o}{=}\PYG{l+m+mi}{1}\PYG{p}{)}\PYG{p}{[}\PYG{l+s+s1}{\PYGZsq{}}\PYG{l+s+s1}{Intercept}\PYG{l+s+s1}{\PYGZsq{}}\PYG{p}{]}
    \PYG{n}{\PYGZus{}}\PYG{p}{[}\PYG{l+s+s1}{\PYGZsq{}}\PYG{l+s+s1}{Coef}\PYG{l+s+s1}{\PYGZsq{}}\PYG{p}{]}\PYG{o}{.}\PYG{n}{plot}\PYG{p}{(}\PYG{n}{kind}\PYG{o}{=}\PYG{l+s+s1}{\PYGZsq{}}\PYG{l+s+s1}{bar}\PYG{l+s+s1}{\PYGZsq{}}\PYG{p}{,} \PYG{n}{yerr}\PYG{o}{=}\PYG{n}{\PYGZus{}}\PYG{p}{[}\PYG{l+s+s1}{\PYGZsq{}}\PYG{l+s+s1}{SE}\PYG{l+s+s1}{\PYGZsq{}}\PYG{p}{]}\PYG{p}{)}
    \PYG{n}{plt}\PYG{o}{.}\PYG{n}{ylabel}\PYG{p}{(}\PYG{l+s+s1}{\PYGZsq{}}\PYG{l+s+s1}{Monthly Alpha (}\PYG{l+s+s1}{\PYGZpc{}}\PYG{l+s+s1}{)}\PYG{l+s+s1}{\PYGZsq{}}\PYG{p}{)}
\end{sphinxVerbatim}

\end{sphinxuseclass}\end{sphinxVerbatimInput}

\end{sphinxuseclass}
\sphinxAtStartPar
We can look at the output of \sphinxcode{\sphinxupquote{get\_coefs()}}, but I typically will chain these functions.

\begin{sphinxuseclass}{cell}\begin{sphinxVerbatimInput}

\begin{sphinxuseclass}{cell_input}
\begin{sphinxVerbatim}[commandchars=\\\{\}]
\PYG{n}{mom\PYGZus{}ew}\PYG{o}{.}\PYG{n}{mul}\PYG{p}{(}\PYG{l+m+mi}{100}\PYG{p}{)}\PYG{o}{.}\PYG{n}{join}\PYG{p}{(}\PYG{n}{ff\PYGZus{}0}\PYG{p}{[}\PYG{l+m+mi}{0}\PYG{p}{]}\PYG{p}{)}\PYG{o}{.}\PYG{n}{pipe}\PYG{p}{(}\PYG{n}{get\PYGZus{}coefs}\PYG{p}{,} \PYG{n}{fun}\PYG{o}{=}\PYG{n}{capm}\PYG{p}{,} \PYG{n}{n}\PYG{o}{=}\PYG{l+m+mi}{10}\PYG{p}{)}
\end{sphinxVerbatim}

\end{sphinxuseclass}\end{sphinxVerbatimInput}
\begin{sphinxVerbatimOutput}

\begin{sphinxuseclass}{cell_output}
\begin{sphinxVerbatim}[commandchars=\\\{\}]
Statistic      Coef                    SE            
Variable  Intercept Q(\PYGZdq{}Mkt\PYGZhy{}RF\PYGZdq{}) Intercept Q(\PYGZdq{}Mkt\PYGZhy{}RF\PYGZdq{})
Portfolio                                            
Mom 1       \PYGZhy{}0.2642      1.6488    0.2364      0.0438
Mom 2       \PYGZhy{}0.2485      1.4625    0.1550      0.0287
Mom 3       \PYGZhy{}0.1499      1.3130    0.1170      0.0216
Mom 4        0.0747      1.2729    0.1071      0.0198
Mom 5        0.1002      1.1657    0.0874      0.0162
Mom 6        0.2728      1.1267    0.0786      0.0146
Mom 7        0.3420      1.0899    0.0778      0.0144
Mom 8        0.4860      1.0698    0.0794      0.0147
Mom 9        0.5828      1.0881    0.0882      0.0163
Mom 10       0.7466      1.1711    0.1273      0.0236
\end{sphinxVerbatim}

\end{sphinxuseclass}\end{sphinxVerbatimOutput}

\end{sphinxuseclass}
\sphinxAtStartPar
The beauty of chaining these functions is:
\begin{enumerate}
\sphinxsetlistlabels{\arabic}{enumi}{enumii}{}{.}%
\item {} 
\sphinxAtStartPar
We have all calculations in one cell, eliminating the need to track many “off screen” calculations in other cells

\item {} 
\sphinxAtStartPar
We do not create intermediate data frames, eliminating the need to name and track many data frames

\end{enumerate}

\begin{sphinxuseclass}{cell}\begin{sphinxVerbatimInput}

\begin{sphinxuseclass}{cell_input}
\begin{sphinxVerbatim}[commandchars=\\\{\}]
\PYG{n}{mom\PYGZus{}ew}\PYG{o}{.}\PYG{n}{mul}\PYG{p}{(}\PYG{l+m+mi}{100}\PYG{p}{)}\PYG{o}{.}\PYG{n}{join}\PYG{p}{(}\PYG{n}{ff\PYGZus{}0}\PYG{p}{[}\PYG{l+m+mi}{0}\PYG{p}{]}\PYG{p}{)}\PYG{o}{.}\PYG{n}{pipe}\PYG{p}{(}\PYG{n}{get\PYGZus{}coefs}\PYG{p}{,} \PYG{n}{fun}\PYG{o}{=}\PYG{n}{capm}\PYG{p}{,} \PYG{n}{n}\PYG{o}{=}\PYG{l+m+mi}{10}\PYG{p}{)}\PYG{o}{.}\PYG{n}{pipe}\PYG{p}{(}\PYG{n}{plot\PYGZus{}alpha}\PYG{p}{)}
\PYG{n}{plt}\PYG{o}{.}\PYG{n}{title}\PYG{p}{(}
    \PYG{l+s+s1}{\PYGZsq{}}\PYG{l+s+s1}{CAPM Tests of Momentum Strategy}\PYG{l+s+s1}{\PYGZsq{}} \PYG{o}{+}
    \PYG{l+s+s1}{\PYGZsq{}}\PYG{l+s+se}{\PYGZbs{}n}\PYG{l+s+s1}{Equal\PYGZhy{}Weighted Portfolios Formed on Months \PYGZhy{}12 to \PYGZhy{}2}\PYG{l+s+s1}{\PYGZsq{}} \PYG{o}{+}
    \PYG{l+s+s1}{\PYGZsq{}}\PYG{l+s+se}{\PYGZbs{}n}\PYG{l+s+s1}{Black Vertical Bars Indicate Standard Errors}\PYG{l+s+s1}{\PYGZsq{}}
\PYG{p}{)}
\PYG{n}{plt}\PYG{o}{.}\PYG{n}{show}\PYG{p}{(}\PYG{p}{)}
\end{sphinxVerbatim}

\end{sphinxuseclass}\end{sphinxVerbatimInput}
\begin{sphinxVerbatimOutput}

\begin{sphinxuseclass}{cell_output}
\noindent\sphinxincludegraphics{{5e1e913e1de9ed0d78aede753ac097b65b1979e747744da61ed4e7ca405d0adc}.png}

\end{sphinxuseclass}\end{sphinxVerbatimOutput}

\end{sphinxuseclass}
\sphinxAtStartPar
The momentum strategy has a large, positive, statistically significant alpha, suggesting it is a source of risk\sphinxhyphen{}adjusted returns (i.e., returns not associated with risk).
However, we saw above that momentum occasionally crashes, suggesting that the momentum strategy has risk.
We will investigate this with the Fama\sphinxhyphen{}French Four\sphinxhyphen{}Factor model, which add size, value, and momentum factors.
Once we consider SMB, HML, and Mom risk factors, we see:
\begin{enumerate}
\sphinxsetlistlabels{\arabic}{enumi}{enumii}{}{.}%
\item {} 
\sphinxAtStartPar
The relation between returns and momentum portfolios is much weaker and not monotonic

\item {} 
\sphinxAtStartPar
The statistical significance of the alpha coefficient estimates is much lower.

\end{enumerate}

\begin{sphinxuseclass}{cell}\begin{sphinxVerbatimInput}

\begin{sphinxuseclass}{cell_input}
\begin{sphinxVerbatim}[commandchars=\\\{\}]
\PYG{k}{def} \PYG{n+nf}{ff4}\PYG{p}{(}\PYG{n}{df}\PYG{p}{,} \PYG{n}{c}\PYG{p}{)}\PYG{p}{:}
    \PYG{k}{return} \PYG{n}{smf}\PYG{o}{.}\PYG{n}{ols}\PYG{p}{(}\PYG{n}{formula}\PYG{o}{=}\PYG{l+s+sa}{f}\PYG{l+s+s1}{\PYGZsq{}}\PYG{l+s+s1}{I(Q(}\PYG{l+s+s1}{\PYGZdq{}}\PYG{l+s+si}{\PYGZob{}}\PYG{n}{c}\PYG{l+s+si}{\PYGZcb{}}\PYG{l+s+s1}{\PYGZdq{}}\PYG{l+s+s1}{)\PYGZhy{}RF) \PYGZti{} Q(}\PYG{l+s+s1}{\PYGZdq{}}\PYG{l+s+s1}{Mkt\PYGZhy{}RF}\PYG{l+s+s1}{\PYGZdq{}}\PYG{l+s+s1}{) + SMB + HML + Mom}\PYG{l+s+s1}{\PYGZsq{}}\PYG{p}{,} \PYG{n}{data}\PYG{o}{=}\PYG{n}{df}\PYG{p}{)}
\end{sphinxVerbatim}

\end{sphinxuseclass}\end{sphinxVerbatimInput}

\end{sphinxuseclass}
\begin{sphinxuseclass}{cell}\begin{sphinxVerbatimInput}

\begin{sphinxuseclass}{cell_input}
\begin{sphinxVerbatim}[commandchars=\\\{\}]
\PYG{n}{mom\PYGZus{}ew}\PYG{o}{.}\PYG{n}{mul}\PYG{p}{(}\PYG{l+m+mi}{100}\PYG{p}{)}\PYG{o}{.}\PYG{n}{join}\PYG{p}{(}\PYG{p}{[}\PYG{n}{ff\PYGZus{}0}\PYG{p}{[}\PYG{l+m+mi}{0}\PYG{p}{]}\PYG{p}{,} \PYG{n}{ff\PYGZus{}mom}\PYG{p}{[}\PYG{l+m+mi}{0}\PYG{p}{]}\PYG{p}{]}\PYG{p}{)}\PYG{o}{.}\PYG{n}{pipe}\PYG{p}{(}\PYG{n}{get\PYGZus{}coefs}\PYG{p}{,} \PYG{n}{fun}\PYG{o}{=}\PYG{n}{ff4}\PYG{p}{,} \PYG{n}{n}\PYG{o}{=}\PYG{l+m+mi}{10}\PYG{p}{)}\PYG{o}{.}\PYG{n}{pipe}\PYG{p}{(}\PYG{n}{plot\PYGZus{}alpha}\PYG{p}{)}
\PYG{n}{plt}\PYG{o}{.}\PYG{n}{title}\PYG{p}{(}
    \PYG{l+s+s1}{\PYGZsq{}}\PYG{l+s+s1}{FF4 Tests of Momentum Strategy}\PYG{l+s+s1}{\PYGZsq{}} \PYG{o}{+}
    \PYG{l+s+s1}{\PYGZsq{}}\PYG{l+s+se}{\PYGZbs{}n}\PYG{l+s+s1}{Equal\PYGZhy{}Weighted Portfolios Formed on Months \PYGZhy{}12 to \PYGZhy{}2}\PYG{l+s+s1}{\PYGZsq{}} \PYG{o}{+}
    \PYG{l+s+s1}{\PYGZsq{}}\PYG{l+s+se}{\PYGZbs{}n}\PYG{l+s+s1}{Black Vertical Bars Indicate Standard Errors}\PYG{l+s+s1}{\PYGZsq{}}
\PYG{p}{)}
\PYG{n}{plt}\PYG{o}{.}\PYG{n}{show}\PYG{p}{(}\PYG{p}{)}
\end{sphinxVerbatim}

\end{sphinxuseclass}\end{sphinxVerbatimInput}
\begin{sphinxVerbatimOutput}

\begin{sphinxuseclass}{cell_output}
\noindent\sphinxincludegraphics{{49d58b0e8f4a15b5f64c08669380a620a5477c654dbd2bd39d8fe61718b6edb4}.png}

\end{sphinxuseclass}\end{sphinxVerbatimOutput}

\end{sphinxuseclass}
\sphinxAtStartPar
A more extreme test is to only consider the last 20 years of returns.

\begin{sphinxuseclass}{cell}\begin{sphinxVerbatimInput}

\begin{sphinxuseclass}{cell_input}
\begin{sphinxVerbatim}[commandchars=\\\{\}]
\PYG{n}{mom\PYGZus{}ew}\PYG{o}{.}\PYG{n}{mul}\PYG{p}{(}\PYG{l+m+mi}{100}\PYG{p}{)}\PYG{o}{.}\PYG{n}{join}\PYG{p}{(}\PYG{p}{[}\PYG{n}{ff\PYGZus{}0}\PYG{p}{[}\PYG{l+m+mi}{0}\PYG{p}{]}\PYG{p}{,} \PYG{n}{ff\PYGZus{}mom}\PYG{p}{[}\PYG{l+m+mi}{0}\PYG{p}{]}\PYG{p}{]}\PYG{p}{)}\PYG{o}{.}\PYG{n}{iloc}\PYG{p}{[}\PYG{o}{\PYGZhy{}}\PYG{l+m+mi}{240}\PYG{p}{:}\PYG{p}{]}\PYG{o}{.}\PYG{n}{pipe}\PYG{p}{(}\PYG{n}{get\PYGZus{}coefs}\PYG{p}{,} \PYG{n}{fun}\PYG{o}{=}\PYG{n}{ff4}\PYG{p}{,} \PYG{n}{n}\PYG{o}{=}\PYG{l+m+mi}{10}\PYG{p}{)}\PYG{o}{.}\PYG{n}{pipe}\PYG{p}{(}\PYG{n}{plot\PYGZus{}alpha}\PYG{p}{)}
\PYG{n}{plt}\PYG{o}{.}\PYG{n}{title}\PYG{p}{(}
    \PYG{l+s+s1}{\PYGZsq{}}\PYG{l+s+s1}{FF4 Tests of Momentum Strategy}\PYG{l+s+s1}{\PYGZsq{}} \PYG{o}{+}
    \PYG{l+s+s1}{\PYGZsq{}}\PYG{l+s+se}{\PYGZbs{}n}\PYG{l+s+s1}{Equal\PYGZhy{}Weighted Portfolios Formed on Months \PYGZhy{}12 to \PYGZhy{}2}\PYG{l+s+s1}{\PYGZsq{}} \PYG{o}{+}
    \PYG{l+s+s1}{\PYGZsq{}}\PYG{l+s+se}{\PYGZbs{}n}\PYG{l+s+s1}{Black Vertical Bars Indicate Standard Errors}\PYG{l+s+s1}{\PYGZsq{}}
\PYG{p}{)}
\PYG{n}{plt}\PYG{o}{.}\PYG{n}{show}\PYG{p}{(}\PYG{p}{)}
\end{sphinxVerbatim}

\end{sphinxuseclass}\end{sphinxVerbatimInput}
\begin{sphinxVerbatimOutput}

\begin{sphinxuseclass}{cell_output}
\noindent\sphinxincludegraphics{{61e94dbb15479ee32c7ad18ef05875f96952d508778d361dacb6f9ec0d32fdaa}.png}

\end{sphinxuseclass}\end{sphinxVerbatimOutput}

\end{sphinxuseclass}

\subsubsection{What are the Sharpe Ratios on these 11 portfolios?}
\label{\detokenize{herron_06_practice_02:what-are-the-sharpe-ratios-on-these-11-portfolios}}
\begin{sphinxuseclass}{cell}\begin{sphinxVerbatimInput}

\begin{sphinxuseclass}{cell_input}
\begin{sphinxVerbatim}[commandchars=\\\{\}]
\PYG{k}{def} \PYG{n+nf}{sharpe}\PYG{p}{(}\PYG{n}{r}\PYG{p}{,} \PYG{n}{tgt}\PYG{p}{,} \PYG{n}{ppy}\PYG{p}{)}\PYG{p}{:}
    \PYG{n}{er} \PYG{o}{=} \PYG{n}{r}\PYG{o}{.}\PYG{n}{sub}\PYG{p}{(}\PYG{n}{tgt}\PYG{p}{)}\PYG{o}{.}\PYG{n}{dropna}\PYG{p}{(}\PYG{p}{)}
    \PYG{k}{return} \PYG{n}{np}\PYG{o}{.}\PYG{n}{sqrt}\PYG{p}{(}\PYG{n}{ppy}\PYG{p}{)} \PYG{o}{*} \PYG{n}{er}\PYG{o}{.}\PYG{n}{mean}\PYG{p}{(}\PYG{p}{)} \PYG{o}{/} \PYG{n}{er}\PYG{o}{.}\PYG{n}{std}\PYG{p}{(}\PYG{p}{)}
\end{sphinxVerbatim}

\end{sphinxuseclass}\end{sphinxVerbatimInput}

\end{sphinxuseclass}
\begin{sphinxuseclass}{cell}\begin{sphinxVerbatimInput}

\begin{sphinxuseclass}{cell_input}
\begin{sphinxVerbatim}[commandchars=\\\{\}]
\PYG{n}{mom\PYGZus{}ew}\PYG{o}{.}\PYG{n}{mul}\PYG{p}{(}\PYG{l+m+mi}{100}\PYG{p}{)}\PYG{o}{.}\PYG{n}{apply}\PYG{p}{(}\PYG{n}{sharpe}\PYG{p}{,} \PYG{n}{tgt}\PYG{o}{=}\PYG{n}{ff\PYGZus{}0}\PYG{p}{[}\PYG{l+m+mi}{0}\PYG{p}{]}\PYG{p}{[}\PYG{l+s+s1}{\PYGZsq{}}\PYG{l+s+s1}{RF}\PYG{l+s+s1}{\PYGZsq{}}\PYG{p}{]}\PYG{p}{,} \PYG{n}{ppy}\PYG{o}{=}\PYG{l+m+mi}{12}\PYG{p}{)}\PYG{o}{.}\PYG{n}{plot}\PYG{p}{(}\PYG{n}{kind}\PYG{o}{=}\PYG{l+s+s1}{\PYGZsq{}}\PYG{l+s+s1}{bar}\PYG{l+s+s1}{\PYGZsq{}}\PYG{p}{)}
\PYG{n}{plt}\PYG{o}{.}\PYG{n}{ylabel}\PYG{p}{(}\PYG{l+s+s1}{\PYGZsq{}}\PYG{l+s+s1}{Sharpe Ratio}\PYG{l+s+s1}{\PYGZsq{}}\PYG{p}{)}
\PYG{n}{plt}\PYG{o}{.}\PYG{n}{title}\PYG{p}{(}
    \PYG{l+s+s1}{\PYGZsq{}}\PYG{l+s+s1}{Sharpe Ratios for Momentum Strategy Relative to RF}\PYG{l+s+s1}{\PYGZsq{}} \PYG{o}{+}
    \PYG{l+s+s1}{\PYGZsq{}}\PYG{l+s+se}{\PYGZbs{}n}\PYG{l+s+s1}{Equal\PYGZhy{}Weighted Portfolios Formed on Months \PYGZhy{}12 to \PYGZhy{}2}\PYG{l+s+s1}{\PYGZsq{}}
\PYG{p}{)}
\PYG{n}{plt}\PYG{o}{.}\PYG{n}{show}\PYG{p}{(}\PYG{p}{)}
\end{sphinxVerbatim}

\end{sphinxuseclass}\end{sphinxVerbatimInput}
\begin{sphinxVerbatimOutput}

\begin{sphinxuseclass}{cell_output}
\noindent\sphinxincludegraphics{{e3f863714a8690aab96231d2252fbc1266e1eb8df9a55c92216420269382a89e}.png}

\end{sphinxuseclass}\end{sphinxVerbatimOutput}

\end{sphinxuseclass}

\subsubsection{Implement a value\sphinxhyphen{}weighted momentum investing strategy}
\label{\detokenize{herron_06_practice_02:implement-a-value-weighted-momentum-investing-strategy}}
\sphinxAtStartPar
Assign this strategy to data frame \sphinxcode{\sphinxupquote{mom\_vw}}, and include long\sphinxhyphen{}short portfolio UMD

\sphinxAtStartPar
We can replace the \sphinxcode{\sphinxupquote{.mean()}} method, which calculates simple means (i.e., equal\sphinxhyphen{}weighted means), with the \sphinxcode{\sphinxupquote{.apply()}} method and \sphinxcode{\sphinxupquote{np.average()}} function, which has a \sphinxcode{\sphinxupquote{weights=}} argument and calculates weighted means.
We weight portfolio returns by the beginning\sphinxhyphen{}of\sphinxhyphen{}month marker value of equity in the \sphinxcode{\sphinxupquote{Trailing ME}} column.

\begin{sphinxuseclass}{cell}\begin{sphinxVerbatimInput}

\begin{sphinxuseclass}{cell_input}
\begin{sphinxVerbatim}[commandchars=\\\{\}]
\PYG{n}{mom\PYGZus{}vw} \PYG{o}{=} \PYG{p}{(}
    \PYG{n}{mom}
    \PYG{o}{.}\PYG{n}{groupby}\PYG{p}{(}\PYG{p}{[}\PYG{l+s+s1}{\PYGZsq{}}\PYG{l+s+s1}{date}\PYG{l+s+s1}{\PYGZsq{}}\PYG{p}{,} \PYG{l+s+s1}{\PYGZsq{}}\PYG{l+s+s1}{Portfolio}\PYG{l+s+s1}{\PYGZsq{}}\PYG{p}{]}\PYG{p}{)}
    \PYG{o}{.}\PYG{n}{apply}\PYG{p}{(}\PYG{k}{lambda} \PYG{n}{x}\PYG{p}{:} \PYG{n}{np}\PYG{o}{.}\PYG{n}{average}\PYG{p}{(}\PYG{n}{a}\PYG{o}{=}\PYG{n}{x}\PYG{p}{[}\PYG{l+s+s1}{\PYGZsq{}}\PYG{l+s+s1}{Return}\PYG{l+s+s1}{\PYGZsq{}}\PYG{p}{]}\PYG{p}{,} \PYG{n}{weights}\PYG{o}{=}\PYG{n}{x}\PYG{p}{[}\PYG{l+s+s1}{\PYGZsq{}}\PYG{l+s+s1}{Trailing ME}\PYG{l+s+s1}{\PYGZsq{}}\PYG{p}{]}\PYG{p}{)}\PYG{p}{)}
    \PYG{o}{.}\PYG{n}{unstack}\PYG{p}{(}\PYG{l+s+s1}{\PYGZsq{}}\PYG{l+s+s1}{Portfolio}\PYG{l+s+s1}{\PYGZsq{}}\PYG{p}{)}
    \PYG{o}{.}\PYG{n}{add\PYGZus{}prefix}\PYG{p}{(}\PYG{l+s+s1}{\PYGZsq{}}\PYG{l+s+s1}{Mom }\PYG{l+s+s1}{\PYGZsq{}}\PYG{p}{)}
\PYG{p}{)}
\end{sphinxVerbatim}

\end{sphinxuseclass}\end{sphinxVerbatimInput}

\end{sphinxuseclass}

\subsubsection{What are the CAPM and FF4 alphas for these value\sphinxhyphen{}weighted portfolios?}
\label{\detokenize{herron_06_practice_02:what-are-the-capm-and-ff4-alphas-for-these-value-weighted-portfolios}}
\sphinxAtStartPar
The alphas for the value\sphinxhyphen{}weighted momentum portfolios are generally smaller in magnitude than for the equal\sphinxhyphen{}weighted momentum portfolios.
Equal\sphinxhyphen{}weighted portfolios over\sphinxhyphen{}weight small stocks relative to value\sphinxhyphen{}weighted portfolios, and we expect most anomalies to be stronger in small stocks because small stocks are more difficult for institutional investors to  invest in at scale.

\begin{sphinxuseclass}{cell}\begin{sphinxVerbatimInput}

\begin{sphinxuseclass}{cell_input}
\begin{sphinxVerbatim}[commandchars=\\\{\}]
\PYG{n}{mom\PYGZus{}vw}\PYG{o}{.}\PYG{n}{mul}\PYG{p}{(}\PYG{l+m+mi}{100}\PYG{p}{)}\PYG{o}{.}\PYG{n}{join}\PYG{p}{(}\PYG{n}{ff\PYGZus{}0}\PYG{p}{[}\PYG{l+m+mi}{0}\PYG{p}{]}\PYG{p}{)}\PYG{o}{.}\PYG{n}{pipe}\PYG{p}{(}\PYG{n}{get\PYGZus{}coefs}\PYG{p}{,} \PYG{n}{fun}\PYG{o}{=}\PYG{n}{capm}\PYG{p}{,} \PYG{n}{n}\PYG{o}{=}\PYG{l+m+mi}{10}\PYG{p}{)}\PYG{o}{.}\PYG{n}{pipe}\PYG{p}{(}\PYG{n}{plot\PYGZus{}alpha}\PYG{p}{)}
\PYG{n}{plt}\PYG{o}{.}\PYG{n}{title}\PYG{p}{(}
    \PYG{l+s+s1}{\PYGZsq{}}\PYG{l+s+s1}{CAPM Tests of Momentum Strategy}\PYG{l+s+s1}{\PYGZsq{}} \PYG{o}{+}
    \PYG{l+s+s1}{\PYGZsq{}}\PYG{l+s+se}{\PYGZbs{}n}\PYG{l+s+s1}{Value\PYGZhy{}Weighted Portfolios Formed on Months \PYGZhy{}12 to \PYGZhy{}2}\PYG{l+s+s1}{\PYGZsq{}} \PYG{o}{+}
    \PYG{l+s+s1}{\PYGZsq{}}\PYG{l+s+se}{\PYGZbs{}n}\PYG{l+s+s1}{Black Vertical Bars Indicate Standard Errors}\PYG{l+s+s1}{\PYGZsq{}}
\PYG{p}{)}
\PYG{n}{plt}\PYG{o}{.}\PYG{n}{show}\PYG{p}{(}\PYG{p}{)}
\end{sphinxVerbatim}

\end{sphinxuseclass}\end{sphinxVerbatimInput}
\begin{sphinxVerbatimOutput}

\begin{sphinxuseclass}{cell_output}
\noindent\sphinxincludegraphics{{ced34c2a1d5da1a44cb646030a067d3e0a7d3f3faa565e7aa4f78e1952078890}.png}

\end{sphinxuseclass}\end{sphinxVerbatimOutput}

\end{sphinxuseclass}
\sphinxAtStartPar
The momentum strategy falls apart once we consider the Fama\sphinxhyphen{}French Four\sphinxhyphen{}Factor model with value\sphinxhyphen{}weighted portfolios.

\begin{sphinxuseclass}{cell}\begin{sphinxVerbatimInput}

\begin{sphinxuseclass}{cell_input}
\begin{sphinxVerbatim}[commandchars=\\\{\}]
\PYG{n}{mom\PYGZus{}vw}\PYG{o}{.}\PYG{n}{mul}\PYG{p}{(}\PYG{l+m+mi}{100}\PYG{p}{)}\PYG{o}{.}\PYG{n}{join}\PYG{p}{(}\PYG{p}{[}\PYG{n}{ff\PYGZus{}0}\PYG{p}{[}\PYG{l+m+mi}{0}\PYG{p}{]}\PYG{p}{,} \PYG{n}{ff\PYGZus{}mom}\PYG{p}{[}\PYG{l+m+mi}{0}\PYG{p}{]}\PYG{p}{]}\PYG{p}{)}\PYG{o}{.}\PYG{n}{pipe}\PYG{p}{(}\PYG{n}{get\PYGZus{}coefs}\PYG{p}{,} \PYG{n}{fun}\PYG{o}{=}\PYG{n}{ff4}\PYG{p}{,} \PYG{n}{n}\PYG{o}{=}\PYG{l+m+mi}{10}\PYG{p}{)}\PYG{o}{.}\PYG{n}{pipe}\PYG{p}{(}\PYG{n}{plot\PYGZus{}alpha}\PYG{p}{)}
\PYG{n}{plt}\PYG{o}{.}\PYG{n}{title}\PYG{p}{(}
    \PYG{l+s+s1}{\PYGZsq{}}\PYG{l+s+s1}{FF4 Tests of Momentum Strategy}\PYG{l+s+s1}{\PYGZsq{}} \PYG{o}{+}
    \PYG{l+s+s1}{\PYGZsq{}}\PYG{l+s+se}{\PYGZbs{}n}\PYG{l+s+s1}{Value\PYGZhy{}Weighted Portfolios Formed on Months \PYGZhy{}12 to \PYGZhy{}2}\PYG{l+s+s1}{\PYGZsq{}} \PYG{o}{+}
    \PYG{l+s+s1}{\PYGZsq{}}\PYG{l+s+se}{\PYGZbs{}n}\PYG{l+s+s1}{Black Vertical Bars Indicate Standard Errors}\PYG{l+s+s1}{\PYGZsq{}}
\PYG{p}{)}
\PYG{n}{plt}\PYG{o}{.}\PYG{n}{show}\PYG{p}{(}\PYG{p}{)}
\end{sphinxVerbatim}

\end{sphinxuseclass}\end{sphinxVerbatimInput}
\begin{sphinxVerbatimOutput}

\begin{sphinxuseclass}{cell_output}
\noindent\sphinxincludegraphics{{66e40b5dd92b6085cded9eeb7d2c959f470ca8560f6630174d8dc1235085e3d4}.png}

\end{sphinxuseclass}\end{sphinxVerbatimOutput}

\end{sphinxuseclass}

\subsubsection{What are the Sharpe Ratios for these value\sphinxhyphen{}weighted portfolios?}
\label{\detokenize{herron_06_practice_02:what-are-the-sharpe-ratios-for-these-value-weighted-portfolios}}
\begin{sphinxuseclass}{cell}\begin{sphinxVerbatimInput}

\begin{sphinxuseclass}{cell_input}
\begin{sphinxVerbatim}[commandchars=\\\{\}]
\PYG{n}{mom\PYGZus{}vw}\PYG{o}{.}\PYG{n}{mul}\PYG{p}{(}\PYG{l+m+mi}{100}\PYG{p}{)}\PYG{o}{.}\PYG{n}{apply}\PYG{p}{(}\PYG{n}{sharpe}\PYG{p}{,} \PYG{n}{tgt}\PYG{o}{=}\PYG{n}{ff\PYGZus{}0}\PYG{p}{[}\PYG{l+m+mi}{0}\PYG{p}{]}\PYG{p}{[}\PYG{l+s+s1}{\PYGZsq{}}\PYG{l+s+s1}{RF}\PYG{l+s+s1}{\PYGZsq{}}\PYG{p}{]}\PYG{p}{,} \PYG{n}{ppy}\PYG{o}{=}\PYG{l+m+mi}{12}\PYG{p}{)}\PYG{o}{.}\PYG{n}{plot}\PYG{p}{(}\PYG{n}{kind}\PYG{o}{=}\PYG{l+s+s1}{\PYGZsq{}}\PYG{l+s+s1}{bar}\PYG{l+s+s1}{\PYGZsq{}}\PYG{p}{)}
\PYG{n}{plt}\PYG{o}{.}\PYG{n}{ylabel}\PYG{p}{(}\PYG{l+s+s1}{\PYGZsq{}}\PYG{l+s+s1}{Sharpe Ratio}\PYG{l+s+s1}{\PYGZsq{}}\PYG{p}{)}
\PYG{n}{plt}\PYG{o}{.}\PYG{n}{title}\PYG{p}{(}
    \PYG{l+s+s1}{\PYGZsq{}}\PYG{l+s+s1}{Sharpe Ratios for Momentum Strategy Relative to RF}\PYG{l+s+s1}{\PYGZsq{}} \PYG{o}{+}
    \PYG{l+s+s1}{\PYGZsq{}}\PYG{l+s+se}{\PYGZbs{}n}\PYG{l+s+s1}{Value\PYGZhy{}Weighted Portfolios Formed on Months \PYGZhy{}12 to \PYGZhy{}2}\PYG{l+s+s1}{\PYGZsq{}}
\PYG{p}{)}
\PYG{n}{plt}\PYG{o}{.}\PYG{n}{show}\PYG{p}{(}\PYG{p}{)}
\end{sphinxVerbatim}

\end{sphinxuseclass}\end{sphinxVerbatimInput}
\begin{sphinxVerbatimOutput}

\begin{sphinxuseclass}{cell_output}
\noindent\sphinxincludegraphics{{f7a171cb8d4193690488883353b1ad5d1c1208f33f2581c2d2f6b793b9526043}.png}

\end{sphinxuseclass}\end{sphinxVerbatimOutput}

\end{sphinxuseclass}

\subsubsection{Implement an equal\sphinxhyphen{}weighted size investing strategy based on market capitalization \textasciitilde{}\textasciitilde{}at the start of each month\textasciitilde{}\textasciitilde{} from the previous June}
\label{\detokenize{herron_06_practice_02:implement-an-equal-weighted-size-investing-strategy-based-on-market-capitalization-at-the-start-of-each-month-from-the-previous-june}}
\sphinxAtStartPar
\sphinxstyleemphasis{\sphinxstylestrong{To show how to form portfolios at lower frequencies, I change this task to form portfolios once a year based on market capitalization from the previous June.}}
We will assign stocks to portfolios based on market capitalization (i.e., size or the market value of equity) each June and use these portfolio assignments from July through the following June.
That is, we rank in June of year \$t\$, then hold them in portfolios from July in year \$t\$ through June in year \$t+1\$.

\begin{sphinxuseclass}{cell}\begin{sphinxVerbatimInput}

\begin{sphinxuseclass}{cell_input}
\begin{sphinxVerbatim}[commandchars=\\\{\}]
\PYG{n}{me\PYGZus{}june} \PYG{o}{=} \PYG{n}{crsp}\PYG{p}{[}\PYG{l+s+s1}{\PYGZsq{}}\PYG{l+s+s1}{ME}\PYG{l+s+s1}{\PYGZsq{}}\PYG{p}{]}\PYG{o}{.}\PYG{n}{unstack}\PYG{p}{(}\PYG{l+s+s1}{\PYGZsq{}}\PYG{l+s+s1}{PERMNO}\PYG{l+s+s1}{\PYGZsq{}}\PYG{p}{)}\PYG{o}{.}\PYG{n}{pipe}\PYG{p}{(}\PYG{k}{lambda} \PYG{n}{x}\PYG{p}{:} \PYG{n}{x}\PYG{o}{.}\PYG{n}{loc}\PYG{p}{[}\PYG{n}{x}\PYG{o}{.}\PYG{n}{index}\PYG{o}{.}\PYG{n}{month} \PYG{o}{==} \PYG{l+m+mi}{6}\PYG{p}{]}\PYG{p}{)}
\end{sphinxVerbatim}

\end{sphinxuseclass}\end{sphinxVerbatimInput}

\end{sphinxuseclass}
\begin{sphinxuseclass}{cell}\begin{sphinxVerbatimInput}

\begin{sphinxuseclass}{cell_input}
\begin{sphinxVerbatim}[commandchars=\\\{\}]
\PYG{n}{me\PYGZus{}june}\PYG{o}{.}\PYG{n}{head}\PYG{p}{(}\PYG{p}{)}
\end{sphinxVerbatim}

\end{sphinxuseclass}\end{sphinxVerbatimInput}
\begin{sphinxVerbatimOutput}

\begin{sphinxuseclass}{cell_output}
\begin{sphinxVerbatim}[commandchars=\\\{\}]
PERMNO   10000  10001  10002  10003  10005      10006  10007  10008  10009  \PYGZbs{}
date                                                                         
1926\PYGZhy{}06    NaN    NaN    NaN    NaN    NaN 59400.0000    NaN    NaN    NaN   
1927\PYGZhy{}06    NaN    NaN    NaN    NaN    NaN 59400.0000    NaN    NaN    NaN   
1928\PYGZhy{}06    NaN    NaN    NaN    NaN    NaN 57675.0000    NaN    NaN    NaN   
1929\PYGZhy{}06    NaN    NaN    NaN    NaN    NaN 57900.0000    NaN    NaN    NaN   
1930\PYGZhy{}06    NaN    NaN    NaN    NaN    NaN 30900.0000    NaN    NaN    NaN   

PERMNO   10010  ...  93423  93426  93428  93429  93430  93432  93433  93434  \PYGZbs{}
date            ...                                                           
1926\PYGZhy{}06    NaN  ...    NaN    NaN    NaN    NaN    NaN    NaN    NaN    NaN   
1927\PYGZhy{}06    NaN  ...    NaN    NaN    NaN    NaN    NaN    NaN    NaN    NaN   
1928\PYGZhy{}06    NaN  ...    NaN    NaN    NaN    NaN    NaN    NaN    NaN    NaN   
1929\PYGZhy{}06    NaN  ...    NaN    NaN    NaN    NaN    NaN    NaN    NaN    NaN   
1930\PYGZhy{}06    NaN  ...    NaN    NaN    NaN    NaN    NaN    NaN    NaN    NaN   

PERMNO   93435  93436  
date                   
1926\PYGZhy{}06    NaN    NaN  
1927\PYGZhy{}06    NaN    NaN  
1928\PYGZhy{}06    NaN    NaN  
1929\PYGZhy{}06    NaN    NaN  
1930\PYGZhy{}06    NaN    NaN  

[5 rows x 26480 columns]
\end{sphinxVerbatim}

\end{sphinxuseclass}\end{sphinxVerbatimOutput}

\end{sphinxuseclass}
\sphinxAtStartPar
Next we assign portfolios on these June market capitalizations.

\begin{sphinxuseclass}{cell}\begin{sphinxVerbatimInput}

\begin{sphinxuseclass}{cell_input}
\begin{sphinxVerbatim}[commandchars=\\\{\}]
\PYG{n}{me\PYGZus{}june}\PYG{o}{.}\PYG{n}{apply}\PYG{p}{(}\PYG{n}{pd}\PYG{o}{.}\PYG{n}{qcut}\PYG{p}{,} \PYG{n}{q}\PYG{o}{=}\PYG{l+m+mi}{10}\PYG{p}{,} \PYG{n}{labels}\PYG{o}{=}\PYG{k+kc}{False}\PYG{p}{,} \PYG{n}{axis}\PYG{o}{=}\PYG{l+m+mi}{1}\PYG{p}{)}\PYG{o}{.}\PYG{n}{head}\PYG{p}{(}\PYG{p}{)}
\end{sphinxVerbatim}

\end{sphinxuseclass}\end{sphinxVerbatimInput}
\begin{sphinxVerbatimOutput}

\begin{sphinxuseclass}{cell_output}
\begin{sphinxVerbatim}[commandchars=\\\{\}]
PERMNO   10000  10001  10002  10003  10005  10006  10007  10008  10009  10010  \PYGZbs{}
date                                                                            
1926\PYGZhy{}06    NaN    NaN    NaN    NaN    NaN 7.0000    NaN    NaN    NaN    NaN   
1927\PYGZhy{}06    NaN    NaN    NaN    NaN    NaN 7.0000    NaN    NaN    NaN    NaN   
1928\PYGZhy{}06    NaN    NaN    NaN    NaN    NaN 7.0000    NaN    NaN    NaN    NaN   
1929\PYGZhy{}06    NaN    NaN    NaN    NaN    NaN 6.0000    NaN    NaN    NaN    NaN   
1930\PYGZhy{}06    NaN    NaN    NaN    NaN    NaN 6.0000    NaN    NaN    NaN    NaN   

PERMNO   ...  93423  93426  93428  93429  93430  93432  93433  93434  93435  \PYGZbs{}
date     ...                                                                  
1926\PYGZhy{}06  ...    NaN    NaN    NaN    NaN    NaN    NaN    NaN    NaN    NaN   
1927\PYGZhy{}06  ...    NaN    NaN    NaN    NaN    NaN    NaN    NaN    NaN    NaN   
1928\PYGZhy{}06  ...    NaN    NaN    NaN    NaN    NaN    NaN    NaN    NaN    NaN   
1929\PYGZhy{}06  ...    NaN    NaN    NaN    NaN    NaN    NaN    NaN    NaN    NaN   
1930\PYGZhy{}06  ...    NaN    NaN    NaN    NaN    NaN    NaN    NaN    NaN    NaN   

PERMNO   93436  
date            
1926\PYGZhy{}06    NaN  
1927\PYGZhy{}06    NaN  
1928\PYGZhy{}06    NaN  
1929\PYGZhy{}06    NaN  
1930\PYGZhy{}06    NaN  

[5 rows x 26480 columns]
\end{sphinxVerbatim}

\end{sphinxuseclass}\end{sphinxVerbatimOutput}

\end{sphinxuseclass}
\sphinxAtStartPar
Note, we want to use these portfolio assignments from July through June, so we need to \sphinxcode{\sphinxupquote{.shift()}} with \sphinxcode{\sphinxupquote{freq='1M'}}, which shifts the index one month instead of shifting the values one row.
Because we only have June data, a one\sphinxhyphen{}row shift is a one\sphinxhyphen{}year shift instead of a one\sphinxhyphen{}month shift.
We also need to up sample our June data to monthly date with \sphinxcode{\sphinxupquote{.resample('M')}} and the \sphinxcode{\sphinxupquote{.ffill()}} method.
Because we only want to fill forward 12 months, we use \sphinxcode{\sphinxupquote{.ffill(limit=12)}}.

\begin{sphinxuseclass}{cell}\begin{sphinxVerbatimInput}

\begin{sphinxuseclass}{cell_input}
\begin{sphinxVerbatim}[commandchars=\\\{\}]
\PYG{n}{me\PYGZus{}june}\PYG{o}{.}\PYG{n}{apply}\PYG{p}{(}\PYG{n}{pd}\PYG{o}{.}\PYG{n}{qcut}\PYG{p}{,} \PYG{n}{q}\PYG{o}{=}\PYG{l+m+mi}{10}\PYG{p}{,} \PYG{n}{labels}\PYG{o}{=}\PYG{k+kc}{False}\PYG{p}{,} \PYG{n}{axis}\PYG{o}{=}\PYG{l+m+mi}{1}\PYG{p}{)}\PYG{o}{.}\PYG{n}{shift}\PYG{p}{(}\PYG{n}{freq}\PYG{o}{=}\PYG{l+s+s1}{\PYGZsq{}}\PYG{l+s+s1}{1M}\PYG{l+s+s1}{\PYGZsq{}}\PYG{p}{)}\PYG{o}{.}\PYG{n}{resample}\PYG{p}{(}\PYG{l+s+s1}{\PYGZsq{}}\PYG{l+s+s1}{M}\PYG{l+s+s1}{\PYGZsq{}}\PYG{p}{)}\PYG{o}{.}\PYG{n}{ffill}\PYG{p}{(}\PYG{n}{limit}\PYG{o}{=}\PYG{l+m+mi}{12}\PYG{p}{)}\PYG{o}{.}\PYG{n}{tail}\PYG{p}{(}\PYG{p}{)}
\end{sphinxVerbatim}

\end{sphinxuseclass}\end{sphinxVerbatimInput}
\begin{sphinxVerbatimOutput}

\begin{sphinxuseclass}{cell_output}
\begin{sphinxVerbatim}[commandchars=\\\{\}]
PERMNO   10000  10001  10002  10003  10005  10006  10007  10008  10009  10010  \PYGZbs{}
date                                                                            
2022\PYGZhy{}03    NaN    NaN    NaN    NaN    NaN    NaN    NaN    NaN    NaN    NaN   
2022\PYGZhy{}04    NaN    NaN    NaN    NaN    NaN    NaN    NaN    NaN    NaN    NaN   
2022\PYGZhy{}05    NaN    NaN    NaN    NaN    NaN    NaN    NaN    NaN    NaN    NaN   
2022\PYGZhy{}06    NaN    NaN    NaN    NaN    NaN    NaN    NaN    NaN    NaN    NaN   
2022\PYGZhy{}07    NaN    NaN    NaN    NaN    NaN    NaN    NaN    NaN    NaN    NaN   

PERMNO   ...  93423  93426  93428  93429  93430  93432  93433  93434  93435  \PYGZbs{}
date     ...                                                                  
2022\PYGZhy{}03  ... 7.0000 3.0000    NaN 8.0000    NaN    NaN    NaN 1.0000    NaN   
2022\PYGZhy{}04  ... 7.0000 3.0000    NaN 8.0000    NaN    NaN    NaN 1.0000    NaN   
2022\PYGZhy{}05  ... 7.0000 3.0000    NaN 8.0000    NaN    NaN    NaN 1.0000    NaN   
2022\PYGZhy{}06  ... 7.0000 3.0000    NaN 8.0000    NaN    NaN    NaN 1.0000    NaN   
2022\PYGZhy{}07  ... 6.0000 4.0000    NaN 8.0000    NaN    NaN    NaN 1.0000    NaN   

PERMNO   93436  
date            
2022\PYGZhy{}03 9.0000  
2022\PYGZhy{}04 9.0000  
2022\PYGZhy{}05 9.0000  
2022\PYGZhy{}06 9.0000  
2022\PYGZhy{}07 9.0000  

[5 rows x 26480 columns]
\end{sphinxVerbatim}

\end{sphinxuseclass}\end{sphinxVerbatimOutput}

\end{sphinxuseclass}
\sphinxAtStartPar
We can combine the \sphinxcode{\sphinxupquote{.apply()}}, \sphinxcode{\sphinxupquote{.shift()}}, \sphinxcode{\sphinxupquote{.resample()}}, and \sphinxcode{\sphinxupquote{.fill()}} methods into one operation when we combine the returns, trailing market capitalization, and portfolio assignments.

\begin{sphinxuseclass}{cell}\begin{sphinxVerbatimInput}

\begin{sphinxuseclass}{cell_input}
\begin{sphinxVerbatim}[commandchars=\\\{\}]
\PYG{n}{me} \PYG{o}{=} \PYG{p}{(}
    \PYG{n}{pd}\PYG{o}{.}\PYG{n}{concat}\PYG{p}{(}
        \PYG{n}{objs}\PYG{o}{=}\PYG{p}{[}
            \PYG{n}{ret\PYGZus{}1m}\PYG{p}{,}
            \PYG{n}{crsp}\PYG{p}{[}\PYG{l+s+s1}{\PYGZsq{}}\PYG{l+s+s1}{ME}\PYG{l+s+s1}{\PYGZsq{}}\PYG{p}{]}\PYG{o}{.}\PYG{n}{unstack}\PYG{p}{(}\PYG{l+s+s1}{\PYGZsq{}}\PYG{l+s+s1}{PERMNO}\PYG{l+s+s1}{\PYGZsq{}}\PYG{p}{)}\PYG{o}{.}\PYG{n}{shift}\PYG{p}{(}\PYG{l+m+mi}{1}\PYG{p}{)}\PYG{p}{,}
            \PYG{p}{(}
                \PYG{n}{me\PYGZus{}june}
                \PYG{o}{.}\PYG{n}{apply}\PYG{p}{(}\PYG{n}{pd}\PYG{o}{.}\PYG{n}{qcut}\PYG{p}{,} \PYG{n}{q}\PYG{o}{=}\PYG{l+m+mi}{10}\PYG{p}{,} \PYG{n}{labels}\PYG{o}{=}\PYG{k+kc}{False}\PYG{p}{,} \PYG{n}{axis}\PYG{o}{=}\PYG{l+m+mi}{1}\PYG{p}{)}
                \PYG{o}{.}\PYG{n}{shift}\PYG{p}{(}\PYG{n}{freq}\PYG{o}{=}\PYG{l+s+s1}{\PYGZsq{}}\PYG{l+s+s1}{1M}\PYG{l+s+s1}{\PYGZsq{}}\PYG{p}{)}
                \PYG{o}{.}\PYG{n}{resample}\PYG{p}{(}\PYG{l+s+s1}{\PYGZsq{}}\PYG{l+s+s1}{M}\PYG{l+s+s1}{\PYGZsq{}}\PYG{p}{)}
                \PYG{o}{.}\PYG{n}{ffill}\PYG{p}{(}\PYG{n}{limit}\PYG{o}{=}\PYG{l+m+mi}{12}\PYG{p}{)}
            \PYG{p}{)}
        \PYG{p}{]}\PYG{p}{,}
        \PYG{n}{keys}\PYG{o}{=}\PYG{p}{[}\PYG{l+s+s1}{\PYGZsq{}}\PYG{l+s+s1}{Return}\PYG{l+s+s1}{\PYGZsq{}}\PYG{p}{,} \PYG{l+s+s1}{\PYGZsq{}}\PYG{l+s+s1}{Trailing ME}\PYG{l+s+s1}{\PYGZsq{}}\PYG{p}{,} \PYG{l+s+s1}{\PYGZsq{}}\PYG{l+s+s1}{Portfolio}\PYG{l+s+s1}{\PYGZsq{}}\PYG{p}{]}\PYG{p}{,}
        \PYG{n}{names}\PYG{o}{=}\PYG{p}{[}\PYG{l+s+s1}{\PYGZsq{}}\PYG{l+s+s1}{Variable}\PYG{l+s+s1}{\PYGZsq{}}\PYG{p}{]}\PYG{p}{,}
        \PYG{n}{axis}\PYG{o}{=}\PYG{l+m+mi}{1}
    \PYG{p}{)}
    \PYG{o}{.}\PYG{n}{stack}\PYG{p}{(}\PYG{l+s+s1}{\PYGZsq{}}\PYG{l+s+s1}{PERMNO}\PYG{l+s+s1}{\PYGZsq{}}\PYG{p}{)}
    \PYG{o}{.}\PYG{n}{dropna}\PYG{p}{(}\PYG{p}{)}
    \PYG{o}{.}\PYG{n}{assign}\PYG{p}{(}\PYG{n}{Portfolio}\PYG{o}{=}\PYG{k}{lambda} \PYG{n}{x}\PYG{p}{:} \PYG{n}{x}\PYG{p}{[}\PYG{l+s+s1}{\PYGZsq{}}\PYG{l+s+s1}{Portfolio}\PYG{l+s+s1}{\PYGZsq{}}\PYG{p}{]}\PYG{o}{.}\PYG{n}{astype}\PYG{p}{(}\PYG{n+nb}{int}\PYG{p}{)} \PYG{o}{+} \PYG{l+m+mi}{1}\PYG{p}{)}
\PYG{p}{)}
\end{sphinxVerbatim}

\end{sphinxuseclass}\end{sphinxVerbatimInput}

\end{sphinxuseclass}
\begin{sphinxuseclass}{cell}\begin{sphinxVerbatimInput}

\begin{sphinxuseclass}{cell_input}
\begin{sphinxVerbatim}[commandchars=\\\{\}]
\PYG{n}{me}\PYG{o}{.}\PYG{n}{head}\PYG{p}{(}\PYG{p}{)}
\end{sphinxVerbatim}

\end{sphinxuseclass}\end{sphinxVerbatimInput}
\begin{sphinxVerbatimOutput}

\begin{sphinxuseclass}{cell_output}
\begin{sphinxVerbatim}[commandchars=\\\{\}]
Variable        Return  Trailing ME  Portfolio
date    PERMNO                                
1926\PYGZhy{}07 10022   0.2400   10000.0000          5
        10030   0.0211   19402.5000          6
        10057   0.0078    4031.2500          2
        10073   0.0729    1656.0000          1
        10081   0.0156    9536.0000          4
\end{sphinxVerbatim}

\end{sphinxuseclass}\end{sphinxVerbatimOutput}

\end{sphinxuseclass}
\begin{sphinxuseclass}{cell}\begin{sphinxVerbatimInput}

\begin{sphinxuseclass}{cell_input}
\begin{sphinxVerbatim}[commandchars=\\\{\}]
\PYG{n}{me\PYGZus{}ew} \PYG{o}{=} \PYG{p}{(}
    \PYG{n}{me}
    \PYG{o}{.}\PYG{n}{groupby}\PYG{p}{(}\PYG{p}{[}\PYG{l+s+s1}{\PYGZsq{}}\PYG{l+s+s1}{date}\PYG{l+s+s1}{\PYGZsq{}}\PYG{p}{,} \PYG{l+s+s1}{\PYGZsq{}}\PYG{l+s+s1}{Portfolio}\PYG{l+s+s1}{\PYGZsq{}}\PYG{p}{]}\PYG{p}{)}
    \PYG{p}{[}\PYG{l+s+s1}{\PYGZsq{}}\PYG{l+s+s1}{Return}\PYG{l+s+s1}{\PYGZsq{}}\PYG{p}{]}
    \PYG{o}{.}\PYG{n}{mean}\PYG{p}{(}\PYG{p}{)}
    \PYG{o}{.}\PYG{n}{unstack}\PYG{p}{(}\PYG{l+s+s1}{\PYGZsq{}}\PYG{l+s+s1}{Portfolio}\PYG{l+s+s1}{\PYGZsq{}}\PYG{p}{)}
    \PYG{o}{.}\PYG{n}{add\PYGZus{}prefix}\PYG{p}{(}\PYG{l+s+s1}{\PYGZsq{}}\PYG{l+s+s1}{ME }\PYG{l+s+s1}{\PYGZsq{}}\PYG{p}{)}
\PYG{p}{)}
\end{sphinxVerbatim}

\end{sphinxuseclass}\end{sphinxVerbatimInput}

\end{sphinxuseclass}
\begin{sphinxuseclass}{cell}\begin{sphinxVerbatimInput}

\begin{sphinxuseclass}{cell_input}
\begin{sphinxVerbatim}[commandchars=\\\{\}]
\PYG{n}{me\PYGZus{}ew}\PYG{o}{.}\PYG{n}{mul}\PYG{p}{(}\PYG{l+m+mi}{100}\PYG{p}{)}\PYG{o}{.}\PYG{n}{join}\PYG{p}{(}\PYG{p}{[}\PYG{n}{ff\PYGZus{}0}\PYG{p}{[}\PYG{l+m+mi}{0}\PYG{p}{]}\PYG{p}{,} \PYG{n}{ff\PYGZus{}mom}\PYG{p}{[}\PYG{l+m+mi}{0}\PYG{p}{]}\PYG{p}{]}\PYG{p}{)}\PYG{o}{.}\PYG{n}{pipe}\PYG{p}{(}\PYG{n}{get\PYGZus{}coefs}\PYG{p}{,} \PYG{n}{fun}\PYG{o}{=}\PYG{n}{capm}\PYG{p}{,} \PYG{n}{n}\PYG{o}{=}\PYG{l+m+mi}{10}\PYG{p}{)}\PYG{o}{.}\PYG{n}{pipe}\PYG{p}{(}\PYG{n}{plot\PYGZus{}alpha}\PYG{p}{)}
\PYG{n}{plt}\PYG{o}{.}\PYG{n}{title}\PYG{p}{(}
    \PYG{l+s+s1}{\PYGZsq{}}\PYG{l+s+s1}{CAPM Tests of Size Strategy}\PYG{l+s+s1}{\PYGZsq{}} \PYG{o}{+}
    \PYG{l+s+s1}{\PYGZsq{}}\PYG{l+s+se}{\PYGZbs{}n}\PYG{l+s+s1}{Equal\PYGZhy{}Weighted Portfolios Formed June Market Value of Equity}\PYG{l+s+s1}{\PYGZsq{}} \PYG{o}{+}
    \PYG{l+s+s1}{\PYGZsq{}}\PYG{l+s+se}{\PYGZbs{}n}\PYG{l+s+s1}{Black Vertical Bars Indicate Standard Errors}\PYG{l+s+s1}{\PYGZsq{}}
\PYG{p}{)}
\PYG{n}{plt}\PYG{o}{.}\PYG{n}{show}\PYG{p}{(}\PYG{p}{)}
\end{sphinxVerbatim}

\end{sphinxuseclass}\end{sphinxVerbatimInput}
\begin{sphinxVerbatimOutput}

\begin{sphinxuseclass}{cell_output}
\noindent\sphinxincludegraphics{{8872db5b249404c8e14edbab82f5358ec876d78ae3ef3d65305b1190711b891a}.png}

\end{sphinxuseclass}\end{sphinxVerbatimOutput}

\end{sphinxuseclass}
\begin{sphinxuseclass}{cell}\begin{sphinxVerbatimInput}

\begin{sphinxuseclass}{cell_input}
\begin{sphinxVerbatim}[commandchars=\\\{\}]
\PYG{n}{me\PYGZus{}ew}\PYG{o}{.}\PYG{n}{mul}\PYG{p}{(}\PYG{l+m+mi}{100}\PYG{p}{)}\PYG{o}{.}\PYG{n}{join}\PYG{p}{(}\PYG{p}{[}\PYG{n}{ff\PYGZus{}0}\PYG{p}{[}\PYG{l+m+mi}{0}\PYG{p}{]}\PYG{p}{,} \PYG{n}{ff\PYGZus{}mom}\PYG{p}{[}\PYG{l+m+mi}{0}\PYG{p}{]}\PYG{p}{]}\PYG{p}{)}\PYG{o}{.}\PYG{n}{pipe}\PYG{p}{(}\PYG{n}{get\PYGZus{}coefs}\PYG{p}{,} \PYG{n}{fun}\PYG{o}{=}\PYG{n}{ff4}\PYG{p}{,} \PYG{n}{n}\PYG{o}{=}\PYG{l+m+mi}{10}\PYG{p}{)}\PYG{o}{.}\PYG{n}{pipe}\PYG{p}{(}\PYG{n}{plot\PYGZus{}alpha}\PYG{p}{)}
\PYG{n}{plt}\PYG{o}{.}\PYG{n}{title}\PYG{p}{(}
    \PYG{l+s+s1}{\PYGZsq{}}\PYG{l+s+s1}{FF4 Tests of Size Strategy}\PYG{l+s+s1}{\PYGZsq{}} \PYG{o}{+}
    \PYG{l+s+s1}{\PYGZsq{}}\PYG{l+s+se}{\PYGZbs{}n}\PYG{l+s+s1}{Equal\PYGZhy{}Weighted Portfolios Formed June Market Value of Equity}\PYG{l+s+s1}{\PYGZsq{}} \PYG{o}{+}
    \PYG{l+s+s1}{\PYGZsq{}}\PYG{l+s+se}{\PYGZbs{}n}\PYG{l+s+s1}{Black Vertical Bars Indicate Standard Errors}\PYG{l+s+s1}{\PYGZsq{}}
\PYG{p}{)}
\PYG{n}{plt}\PYG{o}{.}\PYG{n}{show}\PYG{p}{(}\PYG{p}{)}
\end{sphinxVerbatim}

\end{sphinxuseclass}\end{sphinxVerbatimInput}
\begin{sphinxVerbatimOutput}

\begin{sphinxuseclass}{cell_output}
\noindent\sphinxincludegraphics{{da1a5df44d3d573abd9b0e06810182e2faf2d6075d7ba52d9720be73ab245739}.png}

\end{sphinxuseclass}\end{sphinxVerbatimOutput}

\end{sphinxuseclass}
\sphinxAtStartPar
If we had more time to dig into the size investing strategy, we would find:
\begin{enumerate}
\sphinxsetlistlabels{\arabic}{enumi}{enumii}{}{.}%
\item {} 
\sphinxAtStartPar
The alpha is concentrated in January due to tax loss harvesting, where investors sell their losers in December and buy them back in January to earn a tax deduction

\item {} 
\sphinxAtStartPar
The smallest stocks in the ME 1 portfolio are tiny, making them impractical for institutional investors and allowing the anomaly to persist (because these stocks are small and illiquid, the size factor does not try to explain away their alpha)

\end{enumerate}

\sphinxAtStartPar
We can quickly see the point 2 above by plotting the mean marker capitalization for each portfolio.
This analysis is “quick and dirty” because we have not inflation adjusted these values.
Still, we quickly see that the ME 10 stocks are 1,000 times larger than the ME 1 stocks, on average.

\begin{sphinxuseclass}{cell}\begin{sphinxVerbatimInput}

\begin{sphinxuseclass}{cell_input}
\begin{sphinxVerbatim}[commandchars=\\\{\}]
\PYG{p}{(}
    \PYG{n}{me}
    \PYG{o}{.}\PYG{n}{groupby}\PYG{p}{(}\PYG{p}{[}\PYG{l+s+s1}{\PYGZsq{}}\PYG{l+s+s1}{Portfolio}\PYG{l+s+s1}{\PYGZsq{}}\PYG{p}{,} \PYG{l+s+s1}{\PYGZsq{}}\PYG{l+s+s1}{date}\PYG{l+s+s1}{\PYGZsq{}}\PYG{p}{]}\PYG{p}{)}
    \PYG{p}{[}\PYG{l+s+s1}{\PYGZsq{}}\PYG{l+s+s1}{Trailing ME}\PYG{l+s+s1}{\PYGZsq{}}\PYG{p}{]}
    \PYG{o}{.}\PYG{n}{mean}\PYG{p}{(}\PYG{p}{)}
    \PYG{o}{.}\PYG{n}{unstack}\PYG{p}{(}\PYG{l+s+s1}{\PYGZsq{}}\PYG{l+s+s1}{Portfolio}\PYG{l+s+s1}{\PYGZsq{}}\PYG{p}{)}
    \PYG{o}{.}\PYG{n}{add\PYGZus{}prefix}\PYG{p}{(}\PYG{l+s+s1}{\PYGZsq{}}\PYG{l+s+s1}{ME }\PYG{l+s+s1}{\PYGZsq{}}\PYG{p}{)}
    \PYG{o}{.}\PYG{n}{mean}\PYG{p}{(}\PYG{p}{)}
    \PYG{o}{.}\PYG{n}{plot}\PYG{p}{(}\PYG{n}{kind}\PYG{o}{=}\PYG{l+s+s1}{\PYGZsq{}}\PYG{l+s+s1}{bar}\PYG{l+s+s1}{\PYGZsq{}}\PYG{p}{)}
\PYG{p}{)}
\PYG{n}{plt}\PYG{o}{.}\PYG{n}{semilogy}\PYG{p}{(}\PYG{p}{)}
\PYG{n}{plt}\PYG{o}{.}\PYG{n}{ylabel}\PYG{p}{(}\PYG{l+s+s1}{\PYGZsq{}}\PYG{l+s+s1}{Mean Market Value of Equity (Nominal U.S. Dollars)}\PYG{l+s+s1}{\PYGZsq{}}\PYG{p}{)}
\PYG{n}{plt}\PYG{o}{.}\PYG{n}{title}\PYG{p}{(}\PYG{l+s+s1}{\PYGZsq{}}\PYG{l+s+s1}{Mean Market Value of Stocks in ME Portfolios}\PYG{l+s+s1}{\PYGZsq{}}\PYG{p}{)}
\PYG{n}{plt}\PYG{o}{.}\PYG{n}{show}\PYG{p}{(}\PYG{p}{)}
\end{sphinxVerbatim}

\end{sphinxuseclass}\end{sphinxVerbatimInput}
\begin{sphinxVerbatimOutput}

\begin{sphinxuseclass}{cell_output}
\noindent\sphinxincludegraphics{{9912623f731b1538e1491a458598c59c8be346a294d6d950902f235fd10a843d}.png}

\end{sphinxuseclass}\end{sphinxVerbatimOutput}

\end{sphinxuseclass}

\subsubsection{Implement a value\sphinxhyphen{}weighted size investing strategy based on market capitalization \textasciitilde{}\textasciitilde{}at the start of each month\textasciitilde{}\textasciitilde{} from the previous June}
\label{\detokenize{herron_06_practice_02:implement-a-value-weighted-size-investing-strategy-based-on-market-capitalization-at-the-start-of-each-month-from-the-previous-june}}
\sphinxAtStartPar
With value\sphinxhyphen{}weighted portfolios, the size investing strategy alphas disappear!
The alpha disappears because value\sphinxhyphen{}weighted portfolios down weight the smallest of the smallest stocks that made the size investing strategy appear to be returns not explained by risk.

\begin{sphinxuseclass}{cell}\begin{sphinxVerbatimInput}

\begin{sphinxuseclass}{cell_input}
\begin{sphinxVerbatim}[commandchars=\\\{\}]
\PYG{n}{me\PYGZus{}vw} \PYG{o}{=} \PYG{p}{(}
    \PYG{n}{me}
    \PYG{o}{.}\PYG{n}{groupby}\PYG{p}{(}\PYG{p}{[}\PYG{l+s+s1}{\PYGZsq{}}\PYG{l+s+s1}{date}\PYG{l+s+s1}{\PYGZsq{}}\PYG{p}{,} \PYG{l+s+s1}{\PYGZsq{}}\PYG{l+s+s1}{Portfolio}\PYG{l+s+s1}{\PYGZsq{}}\PYG{p}{]}\PYG{p}{)}
    \PYG{o}{.}\PYG{n}{apply}\PYG{p}{(}\PYG{k}{lambda} \PYG{n}{x}\PYG{p}{:} \PYG{n}{np}\PYG{o}{.}\PYG{n}{average}\PYG{p}{(}\PYG{n}{a}\PYG{o}{=}\PYG{n}{x}\PYG{p}{[}\PYG{l+s+s1}{\PYGZsq{}}\PYG{l+s+s1}{Return}\PYG{l+s+s1}{\PYGZsq{}}\PYG{p}{]}\PYG{p}{,} \PYG{n}{weights}\PYG{o}{=}\PYG{n}{x}\PYG{p}{[}\PYG{l+s+s1}{\PYGZsq{}}\PYG{l+s+s1}{Trailing ME}\PYG{l+s+s1}{\PYGZsq{}}\PYG{p}{]}\PYG{p}{)}\PYG{p}{)}
    \PYG{o}{.}\PYG{n}{unstack}\PYG{p}{(}\PYG{l+s+s1}{\PYGZsq{}}\PYG{l+s+s1}{Portfolio}\PYG{l+s+s1}{\PYGZsq{}}\PYG{p}{)}
    \PYG{o}{.}\PYG{n}{add\PYGZus{}prefix}\PYG{p}{(}\PYG{l+s+s1}{\PYGZsq{}}\PYG{l+s+s1}{ME }\PYG{l+s+s1}{\PYGZsq{}}\PYG{p}{)}
\PYG{p}{)}
\end{sphinxVerbatim}

\end{sphinxuseclass}\end{sphinxVerbatimInput}

\end{sphinxuseclass}
\begin{sphinxuseclass}{cell}\begin{sphinxVerbatimInput}

\begin{sphinxuseclass}{cell_input}
\begin{sphinxVerbatim}[commandchars=\\\{\}]
\PYG{n}{me\PYGZus{}vw}\PYG{o}{.}\PYG{n}{mul}\PYG{p}{(}\PYG{l+m+mi}{100}\PYG{p}{)}\PYG{o}{.}\PYG{n}{join}\PYG{p}{(}\PYG{p}{[}\PYG{n}{ff\PYGZus{}0}\PYG{p}{[}\PYG{l+m+mi}{0}\PYG{p}{]}\PYG{p}{,} \PYG{n}{ff\PYGZus{}mom}\PYG{p}{[}\PYG{l+m+mi}{0}\PYG{p}{]}\PYG{p}{]}\PYG{p}{)}\PYG{o}{.}\PYG{n}{pipe}\PYG{p}{(}\PYG{n}{get\PYGZus{}coefs}\PYG{p}{,} \PYG{n}{fun}\PYG{o}{=}\PYG{n}{capm}\PYG{p}{,} \PYG{n}{n}\PYG{o}{=}\PYG{l+m+mi}{10}\PYG{p}{)}\PYG{o}{.}\PYG{n}{pipe}\PYG{p}{(}\PYG{n}{plot\PYGZus{}alpha}\PYG{p}{)}
\PYG{n}{plt}\PYG{o}{.}\PYG{n}{title}\PYG{p}{(}
    \PYG{l+s+s1}{\PYGZsq{}}\PYG{l+s+s1}{CAPM Tests of Size Strategy}\PYG{l+s+s1}{\PYGZsq{}} \PYG{o}{+}
    \PYG{l+s+s1}{\PYGZsq{}}\PYG{l+s+se}{\PYGZbs{}n}\PYG{l+s+s1}{Value\PYGZhy{}Weighted Portfolios Formed June Market Value of Equity}\PYG{l+s+s1}{\PYGZsq{}} \PYG{o}{+}
    \PYG{l+s+s1}{\PYGZsq{}}\PYG{l+s+se}{\PYGZbs{}n}\PYG{l+s+s1}{Black Vertical Bars Indicate Standard Errors}\PYG{l+s+s1}{\PYGZsq{}}
\PYG{p}{)}
\PYG{n}{plt}\PYG{o}{.}\PYG{n}{show}\PYG{p}{(}\PYG{p}{)}
\end{sphinxVerbatim}

\end{sphinxuseclass}\end{sphinxVerbatimInput}
\begin{sphinxVerbatimOutput}

\begin{sphinxuseclass}{cell_output}
\noindent\sphinxincludegraphics{{3f1ab9c29b78dcf3f973dbae288263b3138d8bf9faea23b8554377f0023b8978}.png}

\end{sphinxuseclass}\end{sphinxVerbatimOutput}

\end{sphinxuseclass}
\begin{sphinxuseclass}{cell}\begin{sphinxVerbatimInput}

\begin{sphinxuseclass}{cell_input}
\begin{sphinxVerbatim}[commandchars=\\\{\}]
\PYG{n}{me\PYGZus{}vw}\PYG{o}{.}\PYG{n}{mul}\PYG{p}{(}\PYG{l+m+mi}{100}\PYG{p}{)}\PYG{o}{.}\PYG{n}{join}\PYG{p}{(}\PYG{p}{[}\PYG{n}{ff\PYGZus{}0}\PYG{p}{[}\PYG{l+m+mi}{0}\PYG{p}{]}\PYG{p}{,} \PYG{n}{ff\PYGZus{}mom}\PYG{p}{[}\PYG{l+m+mi}{0}\PYG{p}{]}\PYG{p}{]}\PYG{p}{)}\PYG{o}{.}\PYG{n}{pipe}\PYG{p}{(}\PYG{n}{get\PYGZus{}coefs}\PYG{p}{,} \PYG{n}{fun}\PYG{o}{=}\PYG{n}{ff4}\PYG{p}{,} \PYG{n}{n}\PYG{o}{=}\PYG{l+m+mi}{10}\PYG{p}{)}\PYG{o}{.}\PYG{n}{pipe}\PYG{p}{(}\PYG{n}{plot\PYGZus{}alpha}\PYG{p}{)}
\PYG{n}{plt}\PYG{o}{.}\PYG{n}{title}\PYG{p}{(}
    \PYG{l+s+s1}{\PYGZsq{}}\PYG{l+s+s1}{FF4 Tests of Size Strategy}\PYG{l+s+s1}{\PYGZsq{}} \PYG{o}{+}
    \PYG{l+s+s1}{\PYGZsq{}}\PYG{l+s+se}{\PYGZbs{}n}\PYG{l+s+s1}{Value\PYGZhy{}Weighted Portfolios Formed June Market Value of Equity}\PYG{l+s+s1}{\PYGZsq{}} \PYG{o}{+}
    \PYG{l+s+s1}{\PYGZsq{}}\PYG{l+s+se}{\PYGZbs{}n}\PYG{l+s+s1}{Black Vertical Bars Indicate Standard Errors}\PYG{l+s+s1}{\PYGZsq{}}
\PYG{p}{)}
\PYG{n}{plt}\PYG{o}{.}\PYG{n}{show}\PYG{p}{(}\PYG{p}{)}
\end{sphinxVerbatim}

\end{sphinxuseclass}\end{sphinxVerbatimInput}
\begin{sphinxVerbatimOutput}

\begin{sphinxuseclass}{cell_output}
\noindent\sphinxincludegraphics{{8e1d15ddfe810ce7dbb2eed7e8f77fe665b0dbd55e16bed2bd257f7354616cac}.png}

\end{sphinxuseclass}\end{sphinxVerbatimOutput}

\end{sphinxuseclass}
\sphinxstepscope


\chapter{Herron Topic 7 \sphinxhyphen{} Five Stylized Facts of Asset Returns}
\label{\detokenize{herron_07_lecture:herron-topic-7-five-stylized-facts-of-asset-returns}}\label{\detokenize{herron_07_lecture::doc}}
\sphinxAtStartPar
\sphinxhref{https://www.tandfonline.com/doi/abs/10.1080/713665670}{Rama Cont (2001)} provides five stylized facts of asset returns, which we will investigate with the daily market factor data from the French Data Library.

\begin{sphinxuseclass}{cell}\begin{sphinxVerbatimInput}

\begin{sphinxuseclass}{cell_input}
\begin{sphinxVerbatim}[commandchars=\\\{\}]
\PYG{k+kn}{import} \PYG{n+nn}{matplotlib}\PYG{n+nn}{.}\PYG{n+nn}{pyplot} \PYG{k}{as} \PYG{n+nn}{plt}
\PYG{k+kn}{import} \PYG{n+nn}{numpy} \PYG{k}{as} \PYG{n+nn}{np}
\PYG{k+kn}{import} \PYG{n+nn}{pandas} \PYG{k}{as} \PYG{n+nn}{pd}
\end{sphinxVerbatim}

\end{sphinxuseclass}\end{sphinxVerbatimInput}

\end{sphinxuseclass}
\begin{sphinxuseclass}{cell}\begin{sphinxVerbatimInput}

\begin{sphinxuseclass}{cell_input}
\begin{sphinxVerbatim}[commandchars=\\\{\}]
\PYG{o}{\PYGZpc{}}\PYG{k}{config} InlineBackend.figure\PYGZus{}format = \PYGZsq{}retina\PYGZsq{}
\PYG{o}{\PYGZpc{}}\PYG{k}{precision} 4
\PYG{n}{pd}\PYG{o}{.}\PYG{n}{options}\PYG{o}{.}\PYG{n}{display}\PYG{o}{.}\PYG{n}{float\PYGZus{}format} \PYG{o}{=} \PYG{l+s+s1}{\PYGZsq{}}\PYG{l+s+si}{\PYGZob{}:.4f\PYGZcb{}}\PYG{l+s+s1}{\PYGZsq{}}\PYG{o}{.}\PYG{n}{format}
\end{sphinxVerbatim}

\end{sphinxuseclass}\end{sphinxVerbatimInput}

\end{sphinxuseclass}
\begin{sphinxuseclass}{cell}\begin{sphinxVerbatimInput}

\begin{sphinxuseclass}{cell_input}
\begin{sphinxVerbatim}[commandchars=\\\{\}]
\PYG{k+kn}{import} \PYG{n+nn}{requests\PYGZus{}cache}
\PYG{n}{session} \PYG{o}{=} \PYG{n}{requests\PYGZus{}cache}\PYG{o}{.}\PYG{n}{CachedSession}\PYG{p}{(}\PYG{p}{)}
\PYG{k+kn}{import} \PYG{n+nn}{yfinance} \PYG{k}{as} \PYG{n+nn}{yf}
\PYG{k+kn}{import} \PYG{n+nn}{pandas\PYGZus{}datareader} \PYG{k}{as} \PYG{n+nn}{pdr}
\end{sphinxVerbatim}

\end{sphinxuseclass}\end{sphinxVerbatimInput}

\end{sphinxuseclass}

\section{Returns are \sphinxstyleemphasis{not} normally distributed}
\label{\detokenize{herron_07_lecture:returns-are-not-normally-distributed}}
\sphinxAtStartPar
Daily stock returns are not normally distributed (i.e., are non\sphinxhyphen{}Gaussian).
Daily stock returns have:
\begin{enumerate}
\sphinxsetlistlabels{\arabic}{enumi}{enumii}{}{.}%
\item {} 
\sphinxAtStartPar
negative skewness (large negative returns occur more often than large positive ones)

\item {} 
\sphinxAtStartPar
excess kurtosis (large magnitude returns, positive or negative, occur more often than if returns were normally distributed)

\end{enumerate}

\sphinxAtStartPar
We can show the non\sphinxhyphen{}normality of daily stock returns in at least two ways:
\begin{enumerate}
\sphinxsetlistlabels{\arabic}{enumi}{enumii}{}{.}%
\item {} 
\sphinxAtStartPar
descriptive statistics (also known as summary statistics)

\item {} 
\sphinxAtStartPar
histograms

\end{enumerate}


\subsection{Log and Simple Returns}
\label{\detokenize{herron_07_lecture:log-and-simple-returns}}
\sphinxAtStartPar
We will use log returns to explore this first stylized fact only.
Simple returns cannot be less than \sphinxhyphen{}100\%, so simple returns cannot be normally distributed.
However, log returns can approach positive and negative infinity.
Note that \sphinxcode{\sphinxupquote{np.log1p(X)}} is equivalent to \sphinxcode{\sphinxupquote{np.log(1 + X)}}.

\begin{sphinxuseclass}{cell}\begin{sphinxVerbatimInput}

\begin{sphinxuseclass}{cell_input}
\begin{sphinxVerbatim}[commandchars=\\\{\}]
\PYG{n}{mkt} \PYG{o}{=} \PYG{p}{(}
    \PYG{n}{pdr}\PYG{o}{.}\PYG{n}{DataReader}\PYG{p}{(}
        \PYG{n}{name}\PYG{o}{=}\PYG{l+s+s1}{\PYGZsq{}}\PYG{l+s+s1}{F\PYGZhy{}F\PYGZus{}Research\PYGZus{}Data\PYGZus{}Factors\PYGZus{}daily}\PYG{l+s+s1}{\PYGZsq{}}\PYG{p}{,}
        \PYG{n}{data\PYGZus{}source}\PYG{o}{=}\PYG{l+s+s1}{\PYGZsq{}}\PYG{l+s+s1}{famafrench}\PYG{l+s+s1}{\PYGZsq{}}\PYG{p}{,}
        \PYG{n}{start}\PYG{o}{=}\PYG{l+s+s1}{\PYGZsq{}}\PYG{l+s+s1}{1900}\PYG{l+s+s1}{\PYGZsq{}}\PYG{p}{,}
        \PYG{n}{session}\PYG{o}{=}\PYG{n}{session}
    \PYG{p}{)}
    \PYG{p}{[}\PYG{l+m+mi}{0}\PYG{p}{]}
    \PYG{o}{.}\PYG{n}{assign}\PYG{p}{(}
        \PYG{n}{R}\PYG{o}{=}\PYG{k}{lambda} \PYG{n}{x}\PYG{p}{:} \PYG{p}{(}\PYG{n}{x}\PYG{p}{[}\PYG{l+s+s1}{\PYGZsq{}}\PYG{l+s+s1}{Mkt\PYGZhy{}RF}\PYG{l+s+s1}{\PYGZsq{}}\PYG{p}{]} \PYG{o}{+} \PYG{n}{x}\PYG{p}{[}\PYG{l+s+s1}{\PYGZsq{}}\PYG{l+s+s1}{RF}\PYG{l+s+s1}{\PYGZsq{}}\PYG{p}{]}\PYG{p}{)} \PYG{o}{/} \PYG{l+m+mi}{100}\PYG{p}{,}
        \PYG{n}{logR}\PYG{o}{=}\PYG{k}{lambda} \PYG{n}{x}\PYG{p}{:} \PYG{n}{np}\PYG{o}{.}\PYG{n}{log1p}\PYG{p}{(}\PYG{n}{x}\PYG{p}{[}\PYG{l+s+s1}{\PYGZsq{}}\PYG{l+s+s1}{R}\PYG{l+s+s1}{\PYGZsq{}}\PYG{p}{]}\PYG{p}{)}
    \PYG{p}{)}
    \PYG{p}{[}\PYG{p}{[}\PYG{l+s+s1}{\PYGZsq{}}\PYG{l+s+s1}{R}\PYG{l+s+s1}{\PYGZsq{}}\PYG{p}{,} \PYG{l+s+s1}{\PYGZsq{}}\PYG{l+s+s1}{logR}\PYG{l+s+s1}{\PYGZsq{}}\PYG{p}{]}\PYG{p}{]}
\PYG{p}{)}
\end{sphinxVerbatim}

\end{sphinxuseclass}\end{sphinxVerbatimInput}

\end{sphinxuseclass}
\begin{sphinxuseclass}{cell}\begin{sphinxVerbatimInput}

\begin{sphinxuseclass}{cell_input}
\begin{sphinxVerbatim}[commandchars=\\\{\}]
\PYG{n}{mkt}
\end{sphinxVerbatim}

\end{sphinxuseclass}\end{sphinxVerbatimInput}
\begin{sphinxVerbatimOutput}

\begin{sphinxuseclass}{cell_output}
\begin{sphinxVerbatim}[commandchars=\\\{\}]
                 R    logR
Date                      
1926\PYGZhy{}07\PYGZhy{}01  0.0011  0.0011
1926\PYGZhy{}07\PYGZhy{}02  0.0046  0.0046
1926\PYGZhy{}07\PYGZhy{}06  0.0018  0.0018
1926\PYGZhy{}07\PYGZhy{}07  0.0010  0.0010
1926\PYGZhy{}07\PYGZhy{}08  0.0022  0.0022
...            ...     ...
2022\PYGZhy{}12\PYGZhy{}23  0.0053  0.0052
2022\PYGZhy{}12\PYGZhy{}27 \PYGZhy{}0.0049 \PYGZhy{}0.0050
2022\PYGZhy{}12\PYGZhy{}28 \PYGZhy{}0.0121 \PYGZhy{}0.0122
2022\PYGZhy{}12\PYGZhy{}29  0.0189  0.0187
2022\PYGZhy{}12\PYGZhy{}30 \PYGZhy{}0.0020 \PYGZhy{}0.0020

[25399 rows x 2 columns]
\end{sphinxVerbatim}

\end{sphinxuseclass}\end{sphinxVerbatimOutput}

\end{sphinxuseclass}
\begin{sphinxuseclass}{cell}\begin{sphinxVerbatimInput}

\begin{sphinxuseclass}{cell_input}
\begin{sphinxVerbatim}[commandchars=\\\{\}]
\PYG{n}{labels} \PYG{o}{=} \PYG{p}{\PYGZob{}}
    \PYG{l+s+s1}{\PYGZsq{}}\PYG{l+s+s1}{R}\PYG{l+s+s1}{\PYGZsq{}}\PYG{p}{:} \PYG{l+s+s1}{\PYGZsq{}}\PYG{l+s+s1}{Simple Return}\PYG{l+s+s1}{\PYGZsq{}}\PYG{p}{,}
    \PYG{l+s+s1}{\PYGZsq{}}\PYG{l+s+s1}{logR}\PYG{l+s+s1}{\PYGZsq{}}\PYG{p}{:} \PYG{l+s+s1}{\PYGZsq{}}\PYG{l+s+s1}{Log Return}\PYG{l+s+s1}{\PYGZsq{}}
\PYG{p}{\PYGZcb{}}
\end{sphinxVerbatim}

\end{sphinxuseclass}\end{sphinxVerbatimInput}

\end{sphinxuseclass}
\begin{sphinxuseclass}{cell}\begin{sphinxVerbatimInput}

\begin{sphinxuseclass}{cell_input}
\begin{sphinxVerbatim}[commandchars=\\\{\}]
\PYG{n}{mkt}\PYG{o}{.}\PYG{n}{rename}\PYG{p}{(}\PYG{n}{columns}\PYG{o}{=}\PYG{n}{labels}\PYG{p}{)}\PYG{o}{.}\PYG{n}{corr}\PYG{p}{(}\PYG{p}{)}
\end{sphinxVerbatim}

\end{sphinxuseclass}\end{sphinxVerbatimInput}
\begin{sphinxVerbatimOutput}

\begin{sphinxuseclass}{cell_output}
\begin{sphinxVerbatim}[commandchars=\\\{\}]
               Simple Return  Log Return
Simple Return         1.0000      0.9997
Log Return            0.9997      1.0000
\end{sphinxVerbatim}

\end{sphinxuseclass}\end{sphinxVerbatimOutput}

\end{sphinxuseclass}
\begin{sphinxuseclass}{cell}\begin{sphinxVerbatimInput}

\begin{sphinxuseclass}{cell_input}
\begin{sphinxVerbatim}[commandchars=\\\{\}]
\PYG{n}{mkt}\PYG{o}{.}\PYG{n}{plot}\PYG{p}{(}\PYG{n}{x}\PYG{o}{=}\PYG{l+s+s1}{\PYGZsq{}}\PYG{l+s+s1}{R}\PYG{l+s+s1}{\PYGZsq{}}\PYG{p}{,} \PYG{n}{y}\PYG{o}{=}\PYG{l+s+s1}{\PYGZsq{}}\PYG{l+s+s1}{logR}\PYG{l+s+s1}{\PYGZsq{}}\PYG{p}{,} \PYG{n}{kind}\PYG{o}{=}\PYG{l+s+s1}{\PYGZsq{}}\PYG{l+s+s1}{scatter}\PYG{l+s+s1}{\PYGZsq{}}\PYG{p}{)}
\PYG{n}{plt}\PYG{o}{.}\PYG{n}{title}\PYG{p}{(}\PYG{l+s+s1}{\PYGZsq{}}\PYG{l+s+s1}{Comparison of Daily Log and Simple Returns}\PYG{l+s+s1}{\PYGZsq{}}\PYG{p}{)}
\PYG{n}{plt}\PYG{o}{.}\PYG{n}{xlabel}\PYG{p}{(}\PYG{n}{labels}\PYG{p}{[}\PYG{l+s+s1}{\PYGZsq{}}\PYG{l+s+s1}{R}\PYG{l+s+s1}{\PYGZsq{}}\PYG{p}{]}\PYG{p}{)}
\PYG{n}{plt}\PYG{o}{.}\PYG{n}{ylabel}\PYG{p}{(}\PYG{n}{labels}\PYG{p}{[}\PYG{l+s+s1}{\PYGZsq{}}\PYG{l+s+s1}{logR}\PYG{l+s+s1}{\PYGZsq{}}\PYG{p}{]}\PYG{p}{)}
\PYG{n}{plt}\PYG{o}{.}\PYG{n}{show}\PYG{p}{(}\PYG{p}{)}
\end{sphinxVerbatim}

\end{sphinxuseclass}\end{sphinxVerbatimInput}
\begin{sphinxVerbatimOutput}

\begin{sphinxuseclass}{cell_output}
\noindent\sphinxincludegraphics{{0ee11fba4ceb6cfaaa007b81e4b9aafb97ec204de3d92a4d6dcb2380b9190c79}.png}

\end{sphinxuseclass}\end{sphinxVerbatimOutput}

\end{sphinxuseclass}

\subsection{Summary statistics}
\label{\detokenize{herron_07_lecture:summary-statistics}}
\sphinxAtStartPar
The \sphinxcode{\sphinxupquote{.describe()}} method reports mean and standard deviation (the first and second moments of the distribution) but does not report skewness and kurtosis (the third and fourth moments).
However, we can use the \sphinxcode{\sphinxupquote{.skew()}} and \sphinxcode{\sphinxupquote{.kurt()}} methods.

\begin{sphinxuseclass}{cell}\begin{sphinxVerbatimInput}

\begin{sphinxuseclass}{cell_input}
\begin{sphinxVerbatim}[commandchars=\\\{\}]
\PYG{n}{mkt}\PYG{o}{.}\PYG{n}{rename}\PYG{p}{(}\PYG{n}{columns}\PYG{o}{=}\PYG{n}{labels}\PYG{p}{)}\PYG{o}{.}\PYG{n}{describe}\PYG{p}{(}\PYG{p}{)}
\end{sphinxVerbatim}

\end{sphinxuseclass}\end{sphinxVerbatimInput}
\begin{sphinxVerbatimOutput}

\begin{sphinxuseclass}{cell_output}
\begin{sphinxVerbatim}[commandchars=\\\{\}]
       Simple Return  Log Return
count     25399.0000  25399.0000
mean          0.0004      0.0004
std           0.0108      0.0108
min          \PYGZhy{}0.1741     \PYGZhy{}0.1913
25\PYGZpc{}          \PYGZhy{}0.0039     \PYGZhy{}0.0039
50\PYGZpc{}           0.0008      0.0007
75\PYGZpc{}           0.0051      0.0051
max           0.1576      0.1464
\end{sphinxVerbatim}

\end{sphinxuseclass}\end{sphinxVerbatimOutput}

\end{sphinxuseclass}
\begin{sphinxuseclass}{cell}\begin{sphinxVerbatimInput}

\begin{sphinxuseclass}{cell_input}
\begin{sphinxVerbatim}[commandchars=\\\{\}]
\PYG{n+nb}{print}\PYG{p}{(}\PYG{l+s+sa}{f}\PYG{l+s+s1}{\PYGZsq{}}\PYG{l+s+s1}{Skewness:        }\PYG{l+s+si}{\PYGZob{}}\PYG{n}{mkt}\PYG{p}{[}\PYG{l+s+s2}{\PYGZdq{}}\PYG{l+s+s2}{logR}\PYG{l+s+s2}{\PYGZdq{}}\PYG{p}{]}\PYG{o}{.}\PYG{n}{skew}\PYG{p}{(}\PYG{p}{)}\PYG{l+s+si}{:}\PYG{l+s+s1}{.2f}\PYG{l+s+si}{\PYGZcb{}}\PYG{l+s+s1}{\PYGZsq{}}\PYG{p}{)}
\PYG{n+nb}{print}\PYG{p}{(}\PYG{l+s+sa}{f}\PYG{l+s+s1}{\PYGZsq{}}\PYG{l+s+s1}{Excess Kurtosis: }\PYG{l+s+si}{\PYGZob{}}\PYG{n}{mkt}\PYG{p}{[}\PYG{l+s+s2}{\PYGZdq{}}\PYG{l+s+s2}{logR}\PYG{l+s+s2}{\PYGZdq{}}\PYG{p}{]}\PYG{o}{.}\PYG{n}{kurt}\PYG{p}{(}\PYG{p}{)}\PYG{l+s+si}{:}\PYG{l+s+s1}{.2f}\PYG{l+s+si}{\PYGZcb{}}\PYG{l+s+s1}{\PYGZsq{}}\PYG{p}{)}
\end{sphinxVerbatim}

\end{sphinxuseclass}\end{sphinxVerbatimInput}
\begin{sphinxVerbatimOutput}

\begin{sphinxuseclass}{cell_output}
\begin{sphinxVerbatim}[commandchars=\\\{\}]
Skewness:        \PYGZhy{}0.47
Excess Kurtosis: 17.32
\end{sphinxVerbatim}

\end{sphinxuseclass}\end{sphinxVerbatimOutput}

\end{sphinxuseclass}
\sphinxAtStartPar
Both values are zero for the normal distribution.
Kurtosis is three for the normal distribution.
However, the \sphinxcode{\sphinxupquote{.kurt()}} method reports \sphinxstyleemphasis{excess} kurtosis, which is kurtosis minus three.


\subsection{Histograms and density plots}
\label{\detokenize{herron_07_lecture:histograms-and-density-plots}}
\sphinxAtStartPar
Histograms provide another way to see the skewness and kurtosis of daily stock returns.
We can overlay a normal distribution with the same mean and standard deviation to highlight negative skewness and excess kurtosis.

\begin{sphinxuseclass}{cell}\begin{sphinxVerbatimInput}

\begin{sphinxuseclass}{cell_input}
\begin{sphinxVerbatim}[commandchars=\\\{\}]
\PYG{k+kn}{import} \PYG{n+nn}{scipy}\PYG{n+nn}{.}\PYG{n+nn}{stats} \PYG{k}{as} \PYG{n+nn}{scs}
\end{sphinxVerbatim}

\end{sphinxuseclass}\end{sphinxVerbatimInput}

\end{sphinxuseclass}
\begin{sphinxuseclass}{cell}\begin{sphinxVerbatimInput}

\begin{sphinxuseclass}{cell_input}
\begin{sphinxVerbatim}[commandchars=\\\{\}]
\PYG{n}{mkt}\PYG{p}{[}\PYG{l+s+s1}{\PYGZsq{}}\PYG{l+s+s1}{logR}\PYG{l+s+s1}{\PYGZsq{}}\PYG{p}{]}\PYG{o}{.}\PYG{n}{plot}\PYG{p}{(}\PYG{n}{kind}\PYG{o}{=}\PYG{l+s+s1}{\PYGZsq{}}\PYG{l+s+s1}{hist}\PYG{l+s+s1}{\PYGZsq{}}\PYG{p}{,} \PYG{n}{bins}\PYG{o}{=}\PYG{l+m+mi}{100}\PYG{p}{,} \PYG{n}{density}\PYG{o}{=}\PYG{k+kc}{True}\PYG{p}{,} \PYG{n}{label}\PYG{o}{=}\PYG{l+s+s1}{\PYGZsq{}}\PYG{l+s+s1}{Observed}\PYG{l+s+s1}{\PYGZsq{}}\PYG{p}{)}
\PYG{n}{xs} \PYG{o}{=} \PYG{n}{np}\PYG{o}{.}\PYG{n}{linspace}\PYG{p}{(}\PYG{n}{mkt}\PYG{p}{[}\PYG{l+s+s1}{\PYGZsq{}}\PYG{l+s+s1}{logR}\PYG{l+s+s1}{\PYGZsq{}}\PYG{p}{]}\PYG{o}{.}\PYG{n}{min}\PYG{p}{(}\PYG{p}{)}\PYG{p}{,} \PYG{n}{mkt}\PYG{p}{[}\PYG{l+s+s1}{\PYGZsq{}}\PYG{l+s+s1}{logR}\PYG{l+s+s1}{\PYGZsq{}}\PYG{p}{]}\PYG{o}{.}\PYG{n}{max}\PYG{p}{(}\PYG{p}{)}\PYG{p}{,} \PYG{l+m+mi}{100}\PYG{p}{)}
\PYG{n}{ys} \PYG{o}{=} \PYG{n}{scs}\PYG{o}{.}\PYG{n}{norm}\PYG{o}{.}\PYG{n}{pdf}\PYG{p}{(}\PYG{n}{x}\PYG{o}{=}\PYG{n}{xs}\PYG{p}{,} \PYG{n}{loc}\PYG{o}{=}\PYG{n}{mkt}\PYG{p}{[}\PYG{l+s+s1}{\PYGZsq{}}\PYG{l+s+s1}{logR}\PYG{l+s+s1}{\PYGZsq{}}\PYG{p}{]}\PYG{o}{.}\PYG{n}{mean}\PYG{p}{(}\PYG{p}{)}\PYG{p}{,} \PYG{n}{scale}\PYG{o}{=}\PYG{n}{mkt}\PYG{p}{[}\PYG{l+s+s1}{\PYGZsq{}}\PYG{l+s+s1}{logR}\PYG{l+s+s1}{\PYGZsq{}}\PYG{p}{]}\PYG{o}{.}\PYG{n}{std}\PYG{p}{(}\PYG{p}{)}\PYG{p}{)}
\PYG{n}{plt}\PYG{o}{.}\PYG{n}{plot}\PYG{p}{(}\PYG{n}{xs}\PYG{p}{,} \PYG{n}{ys}\PYG{p}{,} \PYG{n}{label}\PYG{o}{=}\PYG{l+s+s1}{\PYGZsq{}}\PYG{l+s+s1}{Normal}\PYG{l+s+s1}{\PYGZsq{}}\PYG{p}{)}
\PYG{n}{plt}\PYG{o}{.}\PYG{n}{xlabel}\PYG{p}{(}\PYG{l+s+s1}{\PYGZsq{}}\PYG{l+s+s1}{Daily Log Return}\PYG{l+s+s1}{\PYGZsq{}}\PYG{p}{)}
\PYG{n}{plt}\PYG{o}{.}\PYG{n}{title}\PYG{p}{(}\PYG{l+s+s1}{\PYGZsq{}}\PYG{l+s+s1}{Distribution of Daily Log Returns}\PYG{l+s+s1}{\PYGZsq{}}\PYG{p}{)}
\PYG{n}{plt}\PYG{o}{.}\PYG{n}{legend}\PYG{p}{(}\PYG{n}{title}\PYG{o}{=}\PYG{l+s+s1}{\PYGZsq{}}\PYG{l+s+s1}{Distribution}\PYG{l+s+s1}{\PYGZsq{}}\PYG{p}{)}
\PYG{n}{plt}\PYG{o}{.}\PYG{n}{show}\PYG{p}{(}\PYG{p}{)}
\end{sphinxVerbatim}

\end{sphinxuseclass}\end{sphinxVerbatimInput}
\begin{sphinxVerbatimOutput}

\begin{sphinxuseclass}{cell_output}
\noindent\sphinxincludegraphics{{9aa769fa37663549d5406e5a9c470165f4aada10c6dd1dfe8976897fbfbef4ed}.png}

\end{sphinxuseclass}\end{sphinxVerbatimOutput}

\end{sphinxuseclass}
\sphinxAtStartPar
If we zoom in, we can see that returns in the left tail (large magnitude negative returns) are much more likely than if they were normally distributed.

\begin{sphinxuseclass}{cell}\begin{sphinxVerbatimInput}

\begin{sphinxuseclass}{cell_input}
\begin{sphinxVerbatim}[commandchars=\\\{\}]
\PYG{n}{mkt}\PYG{p}{[}\PYG{l+s+s1}{\PYGZsq{}}\PYG{l+s+s1}{logR}\PYG{l+s+s1}{\PYGZsq{}}\PYG{p}{]}\PYG{o}{.}\PYG{n}{plot}\PYG{p}{(}\PYG{n}{kind}\PYG{o}{=}\PYG{l+s+s1}{\PYGZsq{}}\PYG{l+s+s1}{hist}\PYG{l+s+s1}{\PYGZsq{}}\PYG{p}{,} \PYG{n}{bins}\PYG{o}{=}\PYG{l+m+mi}{100}\PYG{p}{,} \PYG{n}{density}\PYG{o}{=}\PYG{k+kc}{True}\PYG{p}{,} \PYG{n}{label}\PYG{o}{=}\PYG{l+s+s1}{\PYGZsq{}}\PYG{l+s+s1}{Observed}\PYG{l+s+s1}{\PYGZsq{}}\PYG{p}{)}
\PYG{n}{xs} \PYG{o}{=} \PYG{n}{np}\PYG{o}{.}\PYG{n}{linspace}\PYG{p}{(}\PYG{n}{mkt}\PYG{p}{[}\PYG{l+s+s1}{\PYGZsq{}}\PYG{l+s+s1}{logR}\PYG{l+s+s1}{\PYGZsq{}}\PYG{p}{]}\PYG{o}{.}\PYG{n}{min}\PYG{p}{(}\PYG{p}{)}\PYG{p}{,} \PYG{n}{mkt}\PYG{p}{[}\PYG{l+s+s1}{\PYGZsq{}}\PYG{l+s+s1}{logR}\PYG{l+s+s1}{\PYGZsq{}}\PYG{p}{]}\PYG{o}{.}\PYG{n}{max}\PYG{p}{(}\PYG{p}{)}\PYG{p}{,} \PYG{l+m+mi}{100}\PYG{p}{)}
\PYG{n}{ys} \PYG{o}{=} \PYG{n}{scs}\PYG{o}{.}\PYG{n}{norm}\PYG{o}{.}\PYG{n}{pdf}\PYG{p}{(}\PYG{n}{x}\PYG{o}{=}\PYG{n}{xs}\PYG{p}{,} \PYG{n}{loc}\PYG{o}{=}\PYG{n}{mkt}\PYG{p}{[}\PYG{l+s+s1}{\PYGZsq{}}\PYG{l+s+s1}{logR}\PYG{l+s+s1}{\PYGZsq{}}\PYG{p}{]}\PYG{o}{.}\PYG{n}{mean}\PYG{p}{(}\PYG{p}{)}\PYG{p}{,} \PYG{n}{scale}\PYG{o}{=}\PYG{n}{mkt}\PYG{p}{[}\PYG{l+s+s1}{\PYGZsq{}}\PYG{l+s+s1}{logR}\PYG{l+s+s1}{\PYGZsq{}}\PYG{p}{]}\PYG{o}{.}\PYG{n}{std}\PYG{p}{(}\PYG{p}{)}\PYG{p}{)}
\PYG{n}{plt}\PYG{o}{.}\PYG{n}{plot}\PYG{p}{(}\PYG{n}{xs}\PYG{p}{,} \PYG{n}{ys}\PYG{p}{,} \PYG{n}{label}\PYG{o}{=}\PYG{l+s+s1}{\PYGZsq{}}\PYG{l+s+s1}{Normal}\PYG{l+s+s1}{\PYGZsq{}}\PYG{p}{)}
\PYG{n}{plt}\PYG{o}{.}\PYG{n}{xlabel}\PYG{p}{(}\PYG{l+s+s1}{\PYGZsq{}}\PYG{l+s+s1}{Daily Log Return}\PYG{l+s+s1}{\PYGZsq{}}\PYG{p}{)}
\PYG{n}{plt}\PYG{o}{.}\PYG{n}{title}\PYG{p}{(}\PYG{l+s+s1}{\PYGZsq{}}\PYG{l+s+s1}{Distribution of Daily Log Returns}\PYG{l+s+s1}{\PYGZsq{}}\PYG{p}{)}
\PYG{n}{plt}\PYG{o}{.}\PYG{n}{legend}\PYG{p}{(}\PYG{n}{title}\PYG{o}{=}\PYG{l+s+s1}{\PYGZsq{}}\PYG{l+s+s1}{Distribution}\PYG{l+s+s1}{\PYGZsq{}}\PYG{p}{)}
\PYG{n}{plt}\PYG{o}{.}\PYG{n}{xlim}\PYG{p}{(}\PYG{p}{[}\PYG{o}{\PYGZhy{}}\PYG{l+m+mf}{0.12}\PYG{p}{,} \PYG{o}{\PYGZhy{}}\PYG{l+m+mf}{0.07}\PYG{p}{]}\PYG{p}{)}
\PYG{n}{plt}\PYG{o}{.}\PYG{n}{ylim}\PYG{p}{(}\PYG{p}{[}\PYG{l+m+mi}{0}\PYG{p}{,} \PYG{l+m+mf}{0.1}\PYG{p}{]}\PYG{p}{)}
\PYG{n}{plt}\PYG{o}{.}\PYG{n}{show}\PYG{p}{(}\PYG{p}{)}
\end{sphinxVerbatim}

\end{sphinxuseclass}\end{sphinxVerbatimInput}
\begin{sphinxVerbatimOutput}

\begin{sphinxuseclass}{cell_output}
\noindent\sphinxincludegraphics{{de26ff2b36337183dca69fe81c8f09696352ec8c260beb0192c02c24acf6bbd2}.png}

\end{sphinxuseclass}\end{sphinxVerbatimOutput}

\end{sphinxuseclass}

\section{Return volatility clusters in time}
\label{\detokenize{herron_07_lecture:return-volatility-clusters-in-time}}
\sphinxAtStartPar
Stock return volatility varies over time, with alternating periods of low and high volatility.
We can visualize volatility clustering by plotting the time series of returns and volatility.
To reduce noise, we can plot the monthly means and standard deviations of daily returns.

\begin{sphinxuseclass}{cell}\begin{sphinxVerbatimInput}

\begin{sphinxuseclass}{cell_input}
\begin{sphinxVerbatim}[commandchars=\\\{\}]
\PYG{k}{def} \PYG{n+nf}{totret}\PYG{p}{(}\PYG{n}{x}\PYG{p}{)}\PYG{p}{:}
    \PYG{k}{return} \PYG{p}{(}\PYG{l+m+mi}{1} \PYG{o}{+} \PYG{n}{x}\PYG{p}{)}\PYG{o}{.}\PYG{n}{prod}\PYG{p}{(}\PYG{p}{)} \PYG{o}{\PYGZhy{}} \PYG{l+m+mi}{1}
\end{sphinxVerbatim}

\end{sphinxuseclass}\end{sphinxVerbatimInput}

\end{sphinxuseclass}
\begin{sphinxuseclass}{cell}\begin{sphinxVerbatimInput}

\begin{sphinxuseclass}{cell_input}
\begin{sphinxVerbatim}[commandchars=\\\{\}]
\PYG{n}{mkt\PYGZus{}m} \PYG{o}{=} \PYG{p}{(}
    \PYG{n}{mkt}\PYG{p}{[}\PYG{l+s+s1}{\PYGZsq{}}\PYG{l+s+s1}{R}\PYG{l+s+s1}{\PYGZsq{}}\PYG{p}{]}
    \PYG{o}{.}\PYG{n}{resample}\PYG{p}{(}\PYG{n}{rule}\PYG{o}{=}\PYG{l+s+s1}{\PYGZsq{}}\PYG{l+s+s1}{M}\PYG{l+s+s1}{\PYGZsq{}}\PYG{p}{,} \PYG{n}{kind}\PYG{o}{=}\PYG{l+s+s1}{\PYGZsq{}}\PYG{l+s+s1}{period}\PYG{l+s+s1}{\PYGZsq{}}\PYG{p}{)} \PYG{c+c1}{\PYGZsh{} kind=\PYGZsq{}period\PYGZsq{} gives dates in year\PYGZhy{}months instead of year\PYGZhy{}month\PYGZhy{}days}
    \PYG{o}{.}\PYG{n}{agg}\PYG{p}{(}\PYG{p}{[}\PYG{n}{totret}\PYG{p}{,} \PYG{l+s+s1}{\PYGZsq{}}\PYG{l+s+s1}{std}\PYG{l+s+s1}{\PYGZsq{}}\PYG{p}{]}\PYG{p}{)}
\PYG{p}{)}
\end{sphinxVerbatim}

\end{sphinxuseclass}\end{sphinxVerbatimInput}

\end{sphinxuseclass}
\begin{sphinxuseclass}{cell}\begin{sphinxVerbatimInput}

\begin{sphinxuseclass}{cell_input}
\begin{sphinxVerbatim}[commandchars=\\\{\}]
\PYG{n}{mkt\PYGZus{}m}\PYG{o}{.}\PYG{n}{head}\PYG{p}{(}\PYG{p}{)}
\end{sphinxVerbatim}

\end{sphinxuseclass}\end{sphinxVerbatimInput}
\begin{sphinxVerbatimOutput}

\begin{sphinxuseclass}{cell_output}
\begin{sphinxVerbatim}[commandchars=\\\{\}]
         totret    std
Date                  
1926\PYGZhy{}07  0.0312 0.0045
1926\PYGZhy{}08  0.0294 0.0059
1926\PYGZhy{}09  0.0059 0.0050
1926\PYGZhy{}10 \PYGZhy{}0.0299 0.0084
1926\PYGZhy{}11  0.0290 0.0038
\end{sphinxVerbatim}

\end{sphinxuseclass}\end{sphinxVerbatimOutput}

\end{sphinxuseclass}
\sphinxAtStartPar
In the top panel, we see that there are alternating periods of low magnitude and high magnitude mean daily returns.
In the bottom panel, we see that there are alternating periods of low and high volatility.

\begin{sphinxuseclass}{cell}\begin{sphinxVerbatimInput}

\begin{sphinxuseclass}{cell_input}
\begin{sphinxVerbatim}[commandchars=\\\{\}]
\PYG{n}{axes} \PYG{o}{=} \PYG{n}{mkt\PYGZus{}m}\PYG{o}{.}\PYG{n}{mul}\PYG{p}{(}\PYG{l+m+mi}{100}\PYG{p}{)}\PYG{o}{.}\PYG{n}{plot}\PYG{p}{(}\PYG{n}{subplots}\PYG{o}{=}\PYG{k+kc}{True}\PYG{p}{,} \PYG{n}{legend}\PYG{o}{=}\PYG{k+kc}{False}\PYG{p}{)}
\PYG{n}{axes}\PYG{p}{[}\PYG{l+m+mi}{0}\PYG{p}{]}\PYG{o}{.}\PYG{n}{set\PYGZus{}ylabel}\PYG{p}{(}\PYG{l+s+s1}{\PYGZsq{}}\PYG{l+s+s1}{Tot. Ret. (}\PYG{l+s+s1}{\PYGZpc{}}\PYG{l+s+s1}{)}\PYG{l+s+s1}{\PYGZsq{}}\PYG{p}{)}
\PYG{n}{axes}\PYG{p}{[}\PYG{l+m+mi}{1}\PYG{p}{]}\PYG{o}{.}\PYG{n}{set\PYGZus{}ylabel}\PYG{p}{(}\PYG{l+s+s1}{\PYGZsq{}}\PYG{l+s+s1}{Std. Dev. (}\PYG{l+s+s1}{\PYGZpc{}}\PYG{l+s+s1}{)}\PYG{l+s+s1}{\PYGZsq{}}\PYG{p}{)}
\PYG{n}{plt}\PYG{o}{.}\PYG{n}{suptitle}\PYG{p}{(}
    \PYG{l+s+s1}{\PYGZsq{}}\PYG{l+s+s1}{Return Volatility Clusters in Time}\PYG{l+s+s1}{\PYGZsq{}} \PYG{o}{+} 
    \PYG{l+s+s1}{\PYGZsq{}}\PYG{l+s+se}{\PYGZbs{}n}\PYG{l+s+s1}{Monthly Aggregates of Daily Returns}\PYG{l+s+s1}{\PYGZsq{}}
\PYG{p}{)}
\PYG{n}{plt}\PYG{o}{.}\PYG{n}{show}\PYG{p}{(}\PYG{p}{)}
\end{sphinxVerbatim}

\end{sphinxuseclass}\end{sphinxVerbatimInput}
\begin{sphinxVerbatimOutput}

\begin{sphinxuseclass}{cell_output}
\noindent\sphinxincludegraphics{{2f0a599d1744d1d1cc405429795dc31267b8c445ba871d421752d6dbb810b011}.png}

\end{sphinxuseclass}\end{sphinxVerbatimOutput}

\end{sphinxuseclass}

\section{Returns are \sphinxstyleemphasis{not} autocorrelated}
\label{\detokenize{herron_07_lecture:returns-are-not-autocorrelated}}
\sphinxAtStartPar
\sphinxstyleemphasis{\sphinxstylestrong{Stock returns today do not predict stock returns tomorrow.}}
Stock returns are \sphinxstyleemphasis{not} autocorrelated, and stock returns on one day are not correlated with stock returns on previous days.
Therefore, we cannot predict future returns with past returns.
We can show this with an autocorrelation plot of daily stock returns (autocorrelation and serial correlation are synonyms).
The height of each line indicates the correlation coefficient (\$\textbackslash{}rho\$) between returns on day 0 and lag \$t\$ (i.e., day \$0 \sphinxhyphen{} t\$).

\begin{sphinxuseclass}{cell}\begin{sphinxVerbatimInput}

\begin{sphinxuseclass}{cell_input}
\begin{sphinxVerbatim}[commandchars=\\\{\}]
\PYG{n}{N} \PYG{o}{=} \PYG{l+m+mi}{10}
\PYG{n}{mkt\PYGZus{}lags} \PYG{o}{=} \PYG{n}{pd}\PYG{o}{.}\PYG{n}{concat}\PYG{p}{(}\PYG{n}{objs}\PYG{o}{=}\PYG{p}{[}\PYG{n}{mkt}\PYG{p}{[}\PYG{l+s+s1}{\PYGZsq{}}\PYG{l+s+s1}{R}\PYG{l+s+s1}{\PYGZsq{}}\PYG{p}{]}\PYG{o}{.}\PYG{n}{shift}\PYG{p}{(}\PYG{n}{t}\PYG{p}{)} \PYG{k}{for} \PYG{n}{t} \PYG{o+ow}{in} \PYG{n+nb}{range}\PYG{p}{(}\PYG{n}{N} \PYG{o}{+} \PYG{l+m+mi}{1}\PYG{p}{)}\PYG{p}{]}\PYG{p}{,} \PYG{n}{axis}\PYG{o}{=}\PYG{l+m+mi}{1}\PYG{p}{)}
\PYG{n}{corrs} \PYG{o}{=} \PYG{n}{mkt\PYGZus{}lags}\PYG{o}{.}\PYG{n}{corr}\PYG{p}{(}\PYG{p}{)}\PYG{o}{.}\PYG{n}{iloc}\PYG{p}{[}\PYG{l+m+mi}{0}\PYG{p}{]}
\PYG{n}{serrs} \PYG{o}{=} \PYG{n}{np}\PYG{o}{.}\PYG{n}{sqrt}\PYG{p}{(}\PYG{p}{(}\PYG{l+m+mi}{1} \PYG{o}{\PYGZhy{}} \PYG{n}{corrs}\PYG{o}{*}\PYG{o}{*}\PYG{l+m+mi}{2}\PYG{p}{)} \PYG{o}{/} \PYG{p}{(}\PYG{n}{mkt\PYGZus{}lags}\PYG{o}{.}\PYG{n}{count}\PYG{p}{(}\PYG{p}{)} \PYG{o}{\PYGZhy{}} \PYG{l+m+mi}{2}\PYG{p}{)}\PYG{p}{)}

\PYG{n}{plt}\PYG{o}{.}\PYG{n}{bar}\PYG{p}{(}\PYG{n}{height}\PYG{o}{=}\PYG{n}{corrs}\PYG{p}{,} \PYG{n}{x}\PYG{o}{=}\PYG{n+nb}{range}\PYG{p}{(}\PYG{n}{N} \PYG{o}{+} \PYG{l+m+mi}{1}\PYG{p}{)}\PYG{p}{,} \PYG{n}{yerr}\PYG{o}{=}\PYG{l+m+mi}{2}\PYG{o}{*}\PYG{n}{serrs}\PYG{p}{)}
\PYG{n}{plt}\PYG{o}{.}\PYG{n}{title}\PYG{p}{(}\PYG{l+s+s1}{\PYGZsq{}}\PYG{l+s+s1}{Autocorrelation of Daily Returns}\PYG{l+s+s1}{\PYGZsq{}}\PYG{p}{)}
\PYG{n}{plt}\PYG{o}{.}\PYG{n}{xlabel}\PYG{p}{(}\PYG{l+s+s1}{\PYGZsq{}}\PYG{l+s+s1}{Daily Lag}\PYG{l+s+s1}{\PYGZsq{}}\PYG{p}{)}
\PYG{n}{plt}\PYG{o}{.}\PYG{n}{ylabel}\PYG{p}{(}\PYG{l+s+s1}{\PYGZsq{}}\PYG{l+s+s1}{Autocorrelation Coefficient}\PYG{l+s+s1}{\PYGZsq{}}\PYG{p}{)}
\PYG{n}{plt}\PYG{o}{.}\PYG{n}{show}\PYG{p}{(}\PYG{p}{)}
\end{sphinxVerbatim}

\end{sphinxuseclass}\end{sphinxVerbatimInput}
\begin{sphinxVerbatimOutput}

\begin{sphinxuseclass}{cell_output}
\noindent\sphinxincludegraphics{{66fa62c4057dd5ddbfb62306be1a19adcee1b7ff683abe6f9e6ca52737ea81ca}.png}

\end{sphinxuseclass}\end{sphinxVerbatimOutput}

\end{sphinxuseclass}
\sphinxAtStartPar
Above, we see small autocorrelations at lags 1 and 2, suggesting some return predictability.
However, the magnitudes are small and disappear in monthly data.

\sphinxAtStartPar
As a side note, if we plot autocorrelations for a single security instead of the market, we typically find weak, negative autocorrelations for the 1\sphinxhyphen{}day lags.
However, this is due to bid\sphinxhyphen{}ask bounce.
About half the time, positive returns follow negative returns (and \sphinxstyleemphasis{vice versa}) if closing prices randomly alternate between the bid and ask without changes in true prices.


\section{\sphinxstyleemphasis{Squared} returns \sphinxstyleemphasis{are} autocorrelated with slowly decaying autocorrelation}
\label{\detokenize{herron_07_lecture:squared-returns-are-autocorrelated-with-slowly-decaying-autocorrelation}}
\sphinxAtStartPar
Because volatility clusters in time, squared stock returns (and the absolute values of stock returns) are autocorrelated.

\begin{sphinxuseclass}{cell}\begin{sphinxVerbatimInput}

\begin{sphinxuseclass}{cell_input}
\begin{sphinxVerbatim}[commandchars=\\\{\}]
\PYG{n}{N} \PYG{o}{=} \PYG{l+m+mi}{60}
\PYG{n}{mkt\PYGZus{}lags} \PYG{o}{=} \PYG{n}{pd}\PYG{o}{.}\PYG{n}{concat}\PYG{p}{(}\PYG{n}{objs}\PYG{o}{=}\PYG{p}{[}\PYG{n}{mkt}\PYG{p}{[}\PYG{l+s+s1}{\PYGZsq{}}\PYG{l+s+s1}{R}\PYG{l+s+s1}{\PYGZsq{}}\PYG{p}{]}\PYG{o}{.}\PYG{n}{shift}\PYG{p}{(}\PYG{n}{t}\PYG{p}{)} \PYG{k}{for} \PYG{n}{t} \PYG{o+ow}{in} \PYG{n+nb}{range}\PYG{p}{(}\PYG{n}{N} \PYG{o}{+} \PYG{l+m+mi}{1}\PYG{p}{)}\PYG{p}{]}\PYG{p}{,} \PYG{n}{axis}\PYG{o}{=}\PYG{l+m+mi}{1}\PYG{p}{)}
\PYG{n}{corrs} \PYG{o}{=} \PYG{p}{(}\PYG{n}{mkt\PYGZus{}lags} \PYG{o}{*}\PYG{o}{*} \PYG{l+m+mi}{2}\PYG{p}{)}\PYG{o}{.}\PYG{n}{corr}\PYG{p}{(}\PYG{p}{)}\PYG{o}{.}\PYG{n}{iloc}\PYG{p}{[}\PYG{l+m+mi}{0}\PYG{p}{]}
\PYG{n}{serrs} \PYG{o}{=} \PYG{n}{np}\PYG{o}{.}\PYG{n}{sqrt}\PYG{p}{(}\PYG{p}{(}\PYG{l+m+mi}{1} \PYG{o}{\PYGZhy{}} \PYG{n}{corrs}\PYG{o}{*}\PYG{o}{*}\PYG{l+m+mi}{2}\PYG{p}{)} \PYG{o}{/} \PYG{p}{(}\PYG{n}{mkt\PYGZus{}lags}\PYG{o}{.}\PYG{n}{count}\PYG{p}{(}\PYG{p}{)} \PYG{o}{\PYGZhy{}} \PYG{l+m+mi}{2}\PYG{p}{)}\PYG{p}{)}

\PYG{n}{plt}\PYG{o}{.}\PYG{n}{bar}\PYG{p}{(}\PYG{n}{height}\PYG{o}{=}\PYG{n}{corrs}\PYG{p}{,} \PYG{n}{x}\PYG{o}{=}\PYG{n+nb}{range}\PYG{p}{(}\PYG{n}{N} \PYG{o}{+} \PYG{l+m+mi}{1}\PYG{p}{)}\PYG{p}{,} \PYG{n}{yerr}\PYG{o}{=}\PYG{l+m+mi}{2}\PYG{o}{*}\PYG{n}{serrs}\PYG{p}{)}
\PYG{n}{plt}\PYG{o}{.}\PYG{n}{title}\PYG{p}{(}\PYG{l+s+s1}{\PYGZsq{}}\PYG{l+s+s1}{Autocorrelation of Squared Daily Returns}\PYG{l+s+s1}{\PYGZsq{}}\PYG{p}{)}
\PYG{n}{plt}\PYG{o}{.}\PYG{n}{xlabel}\PYG{p}{(}\PYG{l+s+s1}{\PYGZsq{}}\PYG{l+s+s1}{Daily Lag}\PYG{l+s+s1}{\PYGZsq{}}\PYG{p}{)}
\PYG{n}{plt}\PYG{o}{.}\PYG{n}{ylabel}\PYG{p}{(}\PYG{l+s+s1}{\PYGZsq{}}\PYG{l+s+s1}{Autocorrelation Coefficient}\PYG{l+s+s1}{\PYGZsq{}}\PYG{p}{)}
\PYG{n}{plt}\PYG{o}{.}\PYG{n}{show}\PYG{p}{(}\PYG{p}{)}
\end{sphinxVerbatim}

\end{sphinxuseclass}\end{sphinxVerbatimInput}
\begin{sphinxVerbatimOutput}

\begin{sphinxuseclass}{cell_output}
\noindent\sphinxincludegraphics{{151040f6f4f701cc38a28d2a9131d486755b375b0d522d44a4a63a93fba5aa3e}.png}

\end{sphinxuseclass}\end{sphinxVerbatimOutput}

\end{sphinxuseclass}

\section{Fisher Black’s leverage effect \sphinxhyphen{} volatility and returns are negatively related}
\label{\detokenize{herron_07_lecture:fisher-black-s-leverage-effect-volatility-and-returns-are-negatively-related}}\begin{quote}

\sphinxAtStartPar
One of the most enduring empirical regularities in equity markets is the inverse relationship between stock prices and volatility, first documented by Black (1976) who attributed it to the effects of financial leverage. As a company’s stock price declines, it becomes more highly leveraged given a fixed level of debt outstanding, and this increase in leverage induces a higher equity\sphinxhyphen{}return volatility… \sphinxhref{https://papers.ssrn.com/sol3/papers.cfm?abstract\_id=1762363}{(Hasanhodzic and Lo, 2011)}
\end{quote}

\sphinxAtStartPar
We can visualize the leverage effect by plotting the monthly mean and standard deviation of daily returns.
We typically want to report \sphinxstyleemphasis{annualized} mean and volatility of returns (i.e., multiply the mean and volatility of daily returns by 252 and \$\textbackslash{}sqrt\{252\}\$).
However, we will plot daily values here  because some annualized values would be very large when we estimate means and volatilities using only one month of data.

\begin{sphinxuseclass}{cell}\begin{sphinxVerbatimInput}

\begin{sphinxuseclass}{cell_input}
\begin{sphinxVerbatim}[commandchars=\\\{\}]
\PYG{n}{mkt\PYGZus{}m} \PYG{o}{=} \PYG{p}{(}
    \PYG{n}{mkt}\PYG{p}{[}\PYG{l+s+s1}{\PYGZsq{}}\PYG{l+s+s1}{R}\PYG{l+s+s1}{\PYGZsq{}}\PYG{p}{]}
    \PYG{o}{.}\PYG{n}{resample}\PYG{p}{(}\PYG{n}{rule}\PYG{o}{=}\PYG{l+s+s1}{\PYGZsq{}}\PYG{l+s+s1}{M}\PYG{l+s+s1}{\PYGZsq{}}\PYG{p}{,} \PYG{n}{kind}\PYG{o}{=}\PYG{l+s+s1}{\PYGZsq{}}\PYG{l+s+s1}{period}\PYG{l+s+s1}{\PYGZsq{}}\PYG{p}{)}
    \PYG{o}{.}\PYG{n}{agg}\PYG{p}{(}\PYG{p}{[}\PYG{n}{totret}\PYG{p}{,} \PYG{l+s+s1}{\PYGZsq{}}\PYG{l+s+s1}{std}\PYG{l+s+s1}{\PYGZsq{}}\PYG{p}{]}\PYG{p}{)}
\PYG{p}{)}
\end{sphinxVerbatim}

\end{sphinxuseclass}\end{sphinxVerbatimInput}

\end{sphinxuseclass}
\begin{sphinxuseclass}{cell}\begin{sphinxVerbatimInput}

\begin{sphinxuseclass}{cell_input}
\begin{sphinxVerbatim}[commandchars=\\\{\}]
\PYG{k+kn}{import} \PYG{n+nn}{seaborn} \PYG{k}{as} \PYG{n+nn}{sns}
\end{sphinxVerbatim}

\end{sphinxuseclass}\end{sphinxVerbatimInput}

\end{sphinxuseclass}
\begin{sphinxuseclass}{cell}\begin{sphinxVerbatimInput}

\begin{sphinxuseclass}{cell_input}
\begin{sphinxVerbatim}[commandchars=\\\{\}]
\PYG{n}{sns}\PYG{o}{.}\PYG{n}{regplot}\PYG{p}{(}
    \PYG{n}{x}\PYG{o}{=}\PYG{l+s+s1}{\PYGZsq{}}\PYG{l+s+s1}{totret}\PYG{l+s+s1}{\PYGZsq{}}\PYG{p}{,}
    \PYG{n}{y}\PYG{o}{=}\PYG{l+s+s1}{\PYGZsq{}}\PYG{l+s+s1}{std}\PYG{l+s+s1}{\PYGZsq{}}\PYG{p}{,}
    \PYG{n}{data}\PYG{o}{=}\PYG{n}{mkt\PYGZus{}m}\PYG{o}{.}\PYG{n}{mul}\PYG{p}{(}\PYG{l+m+mi}{100}\PYG{p}{)}\PYG{p}{,}
    \PYG{n}{scatter\PYGZus{}kws}\PYG{o}{=}\PYG{p}{\PYGZob{}}\PYG{l+s+s1}{\PYGZsq{}}\PYG{l+s+s1}{alpha}\PYG{l+s+s1}{\PYGZsq{}}\PYG{p}{:} \PYG{l+m+mf}{0.1}\PYG{p}{\PYGZcb{}}
\PYG{p}{)}
\PYG{n}{plt}\PYG{o}{.}\PYG{n}{xlabel}\PYG{p}{(}\PYG{l+s+s1}{\PYGZsq{}}\PYG{l+s+s1}{Total Return (}\PYG{l+s+s1}{\PYGZpc{}}\PYG{l+s+s1}{)}\PYG{l+s+s1}{\PYGZsq{}}\PYG{p}{)}
\PYG{n}{plt}\PYG{o}{.}\PYG{n}{ylabel}\PYG{p}{(}\PYG{l+s+s1}{\PYGZsq{}}\PYG{l+s+s1}{Volatility of Returns (}\PYG{l+s+s1}{\PYGZpc{}}\PYG{l+s+s1}{)}\PYG{l+s+s1}{\PYGZsq{}}\PYG{p}{)}
\PYG{n}{plt}\PYG{o}{.}\PYG{n}{suptitle}\PYG{p}{(}
    \PYG{l+s+s1}{\PYGZsq{}}\PYG{l+s+s1}{Fisher Black}\PYG{l+s+se}{\PYGZbs{}\PYGZsq{}}\PYG{l+s+s1}{s Leverage Effect}\PYG{l+s+se}{\PYGZbs{}n}\PYG{l+s+s1}{\PYGZsq{}} \PYG{o}{+} 
    \PYG{l+s+s1}{\PYGZsq{}}\PYG{l+s+s1}{(Monthly Aggregates of Daily Returns)}\PYG{l+s+s1}{\PYGZsq{}}
\PYG{p}{)}
\PYG{n}{plt}\PYG{o}{.}\PYG{n}{show}\PYG{p}{(}\PYG{p}{)}
\end{sphinxVerbatim}

\end{sphinxuseclass}\end{sphinxVerbatimInput}
\begin{sphinxVerbatimOutput}

\begin{sphinxuseclass}{cell_output}
\noindent\sphinxincludegraphics{{904cfb22b445746e245d49174e734ba2acda7a86a7f565991ae326849767b67c}.png}

\end{sphinxuseclass}\end{sphinxVerbatimOutput}

\end{sphinxuseclass}

\section{Conclusion}
\label{\detokenize{herron_07_lecture:conclusion}}
\sphinxAtStartPar
Financial data analysts should know that real data do not exactly behave as we learn in finance 101.
\begin{enumerate}
\sphinxsetlistlabels{\arabic}{enumi}{enumii}{}{.}%
\item {} 
\sphinxAtStartPar
Stock returns are non\sphinxhyphen{}Gaussian (not normally distributed) with:
\begin{enumerate}
\sphinxsetlistlabels{\arabic}{enumii}{enumiii}{}{.}%
\item {} 
\sphinxAtStartPar
Negative skew

\item {} 
\sphinxAtStartPar
Excess kurtosis

\end{enumerate}

\item {} 
\sphinxAtStartPar
Stock return volatility clusters in time

\item {} 
\sphinxAtStartPar
Stock returns are \sphinxstyleemphasis{not} autocorrelated

\item {} 
\sphinxAtStartPar
\sphinxstyleemphasis{However,} squared stock returns \sphinxstyleemphasis{are} autocorrelated

\item {} 
\sphinxAtStartPar
Stock volatility and stock returns are inversely related

\end{enumerate}

\sphinxstepscope


\section{Herron Topic 7 \sphinxhyphen{} Practice (Blank)}
\label{\detokenize{herron_07_practice:herron-topic-7-practice-blank}}\label{\detokenize{herron_07_practice::doc}}

\subsection{Announcements}
\label{\detokenize{herron_07_practice:announcements}}

\subsection{Practice}
\label{\detokenize{herron_07_practice:practice}}

\subsubsection{Repeat the comparison of log and simple returns with monthly and annual returns}
\label{\detokenize{herron_07_practice:repeat-the-comparison-of-log-and-simple-returns-with-monthly-and-annual-returns}}

\subsubsection{Repeat the autocorrelation plot with \sphinxstyleemphasis{monthly} returns instead of \sphinxstyleemphasis{daily} returns}
\label{\detokenize{herron_07_practice:repeat-the-autocorrelation-plot-with-monthly-returns-instead-of-daily-returns}}

\subsubsection{Plot monthly returns versus their one\sphinxhyphen{}month lag}
\label{\detokenize{herron_07_practice:plot-monthly-returns-versus-their-one-month-lag}}

\subsubsection{Repeat the autocorrelation plot with the \sphinxstyleemphasis{absolute value} of daily returns}
\label{\detokenize{herron_07_practice:repeat-the-autocorrelation-plot-with-the-absolute-value-of-daily-returns}}

\subsubsection{Repeat the leverage effect plot with lagged returns}
\label{\detokenize{herron_07_practice:repeat-the-leverage-effect-plot-with-lagged-returns}}
\sphinxstepscope


\section{Herron Topic 7 \sphinxhyphen{} Practice (Monday 2:45 PM, Section 3)}
\label{\detokenize{herron_07_practice_03:herron-topic-7-practice-monday-2-45-pm-section-3}}\label{\detokenize{herron_07_practice_03::doc}}

\subsection{Announcements}
\label{\detokenize{herron_07_practice_03:announcements}}\begin{itemize}
\item {} 
\sphinxAtStartPar
Quiz 7 (our last quiz!) is due Friday, 4/14, at 11:59 PM

\item {} 
\sphinxAtStartPar
Week of 4/17
\begin{itemize}
\item {} 
\sphinxAtStartPar
I will record a lecture for Herron topic 5 on simulations and post a practice notebook

\item {} 
\sphinxAtStartPar
However, we will use class time for group work

\item {} 
\sphinxAtStartPar
This is a compromise between survey responses (1/3 wanted to drop simulations, 2/3 did not) and Monday’s holiday (Patriot’s Day)

\end{itemize}

\item {} 
\sphinxAtStartPar
Week of 4/24
\begin{itemize}
\item {} 
\sphinxAtStartPar
No lecture

\item {} 
\sphinxAtStartPar
We will use class time for group work

\end{itemize}

\item {} 
\sphinxAtStartPar
Due dates
\begin{itemize}
\item {} 
\sphinxAtStartPar
Project 2 at 11:59 PM on Wednesday, 4/26

\item {} 
\sphinxAtStartPar
Teammates review 2 at 11:59 PM on Wednesday, 4/26

\item {} 
\sphinxAtStartPar
30,000 DataCamp XP at 11:59 PM on Friday, 4/28

\end{itemize}

\item {} 
\sphinxAtStartPar
\sphinxstyleemphasis{\sphinxstylestrong{Please complete TRACE feedback on what topics, tools, and techniques you want more of}}

\end{itemize}


\subsection{Practice}
\label{\detokenize{herron_07_practice_03:practice}}
\begin{sphinxuseclass}{cell}\begin{sphinxVerbatimInput}

\begin{sphinxuseclass}{cell_input}
\begin{sphinxVerbatim}[commandchars=\\\{\}]
\PYG{k+kn}{import} \PYG{n+nn}{matplotlib}\PYG{n+nn}{.}\PYG{n+nn}{pyplot} \PYG{k}{as} \PYG{n+nn}{plt}
\PYG{k+kn}{import} \PYG{n+nn}{numpy} \PYG{k}{as} \PYG{n+nn}{np}
\PYG{k+kn}{import} \PYG{n+nn}{pandas} \PYG{k}{as} \PYG{n+nn}{pd}
\end{sphinxVerbatim}

\end{sphinxuseclass}\end{sphinxVerbatimInput}

\end{sphinxuseclass}
\begin{sphinxuseclass}{cell}\begin{sphinxVerbatimInput}

\begin{sphinxuseclass}{cell_input}
\begin{sphinxVerbatim}[commandchars=\\\{\}]
\PYG{o}{\PYGZpc{}}\PYG{k}{config} InlineBackend.figure\PYGZus{}format = \PYGZsq{}retina\PYGZsq{}
\PYG{o}{\PYGZpc{}}\PYG{k}{precision} 4
\PYG{n}{pd}\PYG{o}{.}\PYG{n}{options}\PYG{o}{.}\PYG{n}{display}\PYG{o}{.}\PYG{n}{float\PYGZus{}format} \PYG{o}{=} \PYG{l+s+s1}{\PYGZsq{}}\PYG{l+s+si}{\PYGZob{}:.4f\PYGZcb{}}\PYG{l+s+s1}{\PYGZsq{}}\PYG{o}{.}\PYG{n}{format}
\end{sphinxVerbatim}

\end{sphinxuseclass}\end{sphinxVerbatimInput}

\end{sphinxuseclass}
\begin{sphinxuseclass}{cell}\begin{sphinxVerbatimInput}

\begin{sphinxuseclass}{cell_input}
\begin{sphinxVerbatim}[commandchars=\\\{\}]
\PYG{k+kn}{import} \PYG{n+nn}{requests\PYGZus{}cache}
\PYG{n}{session} \PYG{o}{=} \PYG{n}{requests\PYGZus{}cache}\PYG{o}{.}\PYG{n}{CachedSession}\PYG{p}{(}\PYG{p}{)}
\PYG{k+kn}{import} \PYG{n+nn}{yfinance} \PYG{k}{as} \PYG{n+nn}{yf}
\PYG{k+kn}{import} \PYG{n+nn}{pandas\PYGZus{}datareader} \PYG{k}{as} \PYG{n+nn}{pdr}
\end{sphinxVerbatim}

\end{sphinxuseclass}\end{sphinxVerbatimInput}

\end{sphinxuseclass}
\begin{sphinxuseclass}{cell}\begin{sphinxVerbatimInput}

\begin{sphinxuseclass}{cell_input}
\begin{sphinxVerbatim}[commandchars=\\\{\}]
\PYG{n}{mkt} \PYG{o}{=} \PYG{p}{(}
    \PYG{n}{pdr}\PYG{o}{.}\PYG{n}{DataReader}\PYG{p}{(}
        \PYG{n}{name}\PYG{o}{=}\PYG{l+s+s1}{\PYGZsq{}}\PYG{l+s+s1}{F\PYGZhy{}F\PYGZus{}Research\PYGZus{}Data\PYGZus{}Factors\PYGZus{}daily}\PYG{l+s+s1}{\PYGZsq{}}\PYG{p}{,}
        \PYG{n}{data\PYGZus{}source}\PYG{o}{=}\PYG{l+s+s1}{\PYGZsq{}}\PYG{l+s+s1}{famafrench}\PYG{l+s+s1}{\PYGZsq{}}\PYG{p}{,}
        \PYG{n}{start}\PYG{o}{=}\PYG{l+s+s1}{\PYGZsq{}}\PYG{l+s+s1}{1900}\PYG{l+s+s1}{\PYGZsq{}}\PYG{p}{,}
        \PYG{n}{session}\PYG{o}{=}\PYG{n}{session}
    \PYG{p}{)}
    \PYG{p}{[}\PYG{l+m+mi}{0}\PYG{p}{]}
    \PYG{o}{.}\PYG{n}{assign}\PYG{p}{(}
        \PYG{n}{R}\PYG{o}{=}\PYG{k}{lambda} \PYG{n}{x}\PYG{p}{:} \PYG{p}{(}\PYG{n}{x}\PYG{p}{[}\PYG{l+s+s1}{\PYGZsq{}}\PYG{l+s+s1}{Mkt\PYGZhy{}RF}\PYG{l+s+s1}{\PYGZsq{}}\PYG{p}{]} \PYG{o}{+} \PYG{n}{x}\PYG{p}{[}\PYG{l+s+s1}{\PYGZsq{}}\PYG{l+s+s1}{RF}\PYG{l+s+s1}{\PYGZsq{}}\PYG{p}{]}\PYG{p}{)} \PYG{o}{/} \PYG{l+m+mi}{100}\PYG{p}{,}
        \PYG{n}{logR}\PYG{o}{=}\PYG{k}{lambda} \PYG{n}{x}\PYG{p}{:} \PYG{n}{np}\PYG{o}{.}\PYG{n}{log1p}\PYG{p}{(}\PYG{n}{x}\PYG{p}{[}\PYG{l+s+s1}{\PYGZsq{}}\PYG{l+s+s1}{R}\PYG{l+s+s1}{\PYGZsq{}}\PYG{p}{]}\PYG{p}{)}
    \PYG{p}{)}
    \PYG{p}{[}\PYG{p}{[}\PYG{l+s+s1}{\PYGZsq{}}\PYG{l+s+s1}{R}\PYG{l+s+s1}{\PYGZsq{}}\PYG{p}{,} \PYG{l+s+s1}{\PYGZsq{}}\PYG{l+s+s1}{logR}\PYG{l+s+s1}{\PYGZsq{}}\PYG{p}{]}\PYG{p}{]}
\PYG{p}{)}
\end{sphinxVerbatim}

\end{sphinxuseclass}\end{sphinxVerbatimInput}

\end{sphinxuseclass}
\begin{sphinxuseclass}{cell}\begin{sphinxVerbatimInput}

\begin{sphinxuseclass}{cell_input}
\begin{sphinxVerbatim}[commandchars=\\\{\}]
\PYG{n}{labels} \PYG{o}{=} \PYG{p}{\PYGZob{}}
    \PYG{l+s+s1}{\PYGZsq{}}\PYG{l+s+s1}{R}\PYG{l+s+s1}{\PYGZsq{}}\PYG{p}{:} \PYG{l+s+s1}{\PYGZsq{}}\PYG{l+s+s1}{Simple Return}\PYG{l+s+s1}{\PYGZsq{}}\PYG{p}{,}
    \PYG{l+s+s1}{\PYGZsq{}}\PYG{l+s+s1}{logR}\PYG{l+s+s1}{\PYGZsq{}}\PYG{p}{:} \PYG{l+s+s1}{\PYGZsq{}}\PYG{l+s+s1}{Log Return}\PYG{l+s+s1}{\PYGZsq{}}
\PYG{p}{\PYGZcb{}}
\end{sphinxVerbatim}

\end{sphinxuseclass}\end{sphinxVerbatimInput}

\end{sphinxuseclass}

\subsubsection{Repeat the comparison of log and simple returns with monthly and annual returns}
\label{\detokenize{herron_07_practice_03:repeat-the-comparison-of-log-and-simple-returns-with-monthly-and-annual-returns}}
\sphinxAtStartPar
We could download monthly and annual data from the Ken French data library.
However, we can also use the \sphinxcode{\sphinxupquote{.resample()}} method to compound daily returns into monthly and annual returns.
Let us use the \sphinxcode{\sphinxupquote{.resample()}} method (1) for practice and (2) because the \sphinxcode{\sphinxupquote{.resample()}} code is simpler and more compact than the \sphinxcode{\sphinxupquote{pdr.DataReader()}} code.
We need to re\sphinxhyphen{}calculate the log return.
We could instead sum the daily log returns, but combining the \sphinxcode{\sphinxupquote{.resample()}} method with two different operations on two differ columns is complex and easily avoided here.

\begin{sphinxuseclass}{cell}\begin{sphinxVerbatimInput}

\begin{sphinxuseclass}{cell_input}
\begin{sphinxVerbatim}[commandchars=\\\{\}]
\PYG{k}{def} \PYG{n+nf}{totret}\PYG{p}{(}\PYG{n}{x}\PYG{p}{)}\PYG{p}{:}
    \PYG{k}{return} \PYG{p}{(}\PYG{l+m+mi}{1} \PYG{o}{+} \PYG{n}{x}\PYG{p}{)}\PYG{o}{.}\PYG{n}{prod}\PYG{p}{(}\PYG{p}{)} \PYG{o}{\PYGZhy{}} \PYG{l+m+mi}{1}
\end{sphinxVerbatim}

\end{sphinxuseclass}\end{sphinxVerbatimInput}

\end{sphinxuseclass}
\begin{sphinxuseclass}{cell}\begin{sphinxVerbatimInput}

\begin{sphinxuseclass}{cell_input}
\begin{sphinxVerbatim}[commandchars=\\\{\}]
\PYG{n}{mkt\PYGZus{}m} \PYG{o}{=} \PYG{p}{(}
    \PYG{n}{mkt}\PYG{p}{[}\PYG{p}{[}\PYG{l+s+s1}{\PYGZsq{}}\PYG{l+s+s1}{R}\PYG{l+s+s1}{\PYGZsq{}}\PYG{p}{]}\PYG{p}{]}
    \PYG{o}{.}\PYG{n}{resample}\PYG{p}{(}\PYG{n}{rule}\PYG{o}{=}\PYG{l+s+s1}{\PYGZsq{}}\PYG{l+s+s1}{M}\PYG{l+s+s1}{\PYGZsq{}}\PYG{p}{,} \PYG{n}{kind}\PYG{o}{=}\PYG{l+s+s1}{\PYGZsq{}}\PYG{l+s+s1}{period}\PYG{l+s+s1}{\PYGZsq{}}\PYG{p}{)}
    \PYG{o}{.}\PYG{n}{apply}\PYG{p}{(}\PYG{n}{totret}\PYG{p}{)}
    \PYG{o}{.}\PYG{n}{assign}\PYG{p}{(}\PYG{n}{logR}\PYG{o}{=}\PYG{k}{lambda} \PYG{n}{x}\PYG{p}{:} \PYG{n}{np}\PYG{o}{.}\PYG{n}{log1p}\PYG{p}{(}\PYG{n}{x}\PYG{p}{[}\PYG{l+s+s1}{\PYGZsq{}}\PYG{l+s+s1}{R}\PYG{l+s+s1}{\PYGZsq{}}\PYG{p}{]}\PYG{p}{)}\PYG{p}{)}
\PYG{p}{)}
\end{sphinxVerbatim}

\end{sphinxuseclass}\end{sphinxVerbatimInput}

\end{sphinxuseclass}
\begin{sphinxuseclass}{cell}\begin{sphinxVerbatimInput}

\begin{sphinxuseclass}{cell_input}
\begin{sphinxVerbatim}[commandchars=\\\{\}]
\PYG{n}{mkt\PYGZus{}a} \PYG{o}{=} \PYG{p}{(}
    \PYG{n}{mkt}\PYG{p}{[}\PYG{p}{[}\PYG{l+s+s1}{\PYGZsq{}}\PYG{l+s+s1}{R}\PYG{l+s+s1}{\PYGZsq{}}\PYG{p}{]}\PYG{p}{]}
    \PYG{o}{.}\PYG{n}{resample}\PYG{p}{(}\PYG{n}{rule}\PYG{o}{=}\PYG{l+s+s1}{\PYGZsq{}}\PYG{l+s+s1}{A}\PYG{l+s+s1}{\PYGZsq{}}\PYG{p}{,} \PYG{n}{kind}\PYG{o}{=}\PYG{l+s+s1}{\PYGZsq{}}\PYG{l+s+s1}{period}\PYG{l+s+s1}{\PYGZsq{}}\PYG{p}{)}
    \PYG{o}{.}\PYG{n}{apply}\PYG{p}{(}\PYG{n}{totret}\PYG{p}{)}
    \PYG{o}{.}\PYG{n}{assign}\PYG{p}{(}\PYG{n}{logR}\PYG{o}{=}\PYG{k}{lambda} \PYG{n}{x}\PYG{p}{:} \PYG{n}{np}\PYG{o}{.}\PYG{n}{log1p}\PYG{p}{(}\PYG{n}{x}\PYG{p}{[}\PYG{l+s+s1}{\PYGZsq{}}\PYG{l+s+s1}{R}\PYG{l+s+s1}{\PYGZsq{}}\PYG{p}{]}\PYG{p}{)}\PYG{p}{)}
\PYG{p}{)}
\end{sphinxVerbatim}

\end{sphinxuseclass}\end{sphinxVerbatimInput}

\end{sphinxuseclass}
\sphinxAtStartPar
We want to repeat this plot for monthly and annual returns, so we can make out lives easier by writing a function.

\begin{sphinxuseclass}{cell}\begin{sphinxVerbatimInput}

\begin{sphinxuseclass}{cell_input}
\begin{sphinxVerbatim}[commandchars=\\\{\}]
\PYG{k}{def} \PYG{n+nf}{plot\PYGZus{}compare\PYGZus{}returns}\PYG{p}{(}\PYG{n}{x}\PYG{p}{,} \PYG{n}{freq}\PYG{p}{,} \PYG{n}{labs}\PYG{o}{=}\PYG{n}{labels}\PYG{p}{)}\PYG{p}{:}
    \PYG{n}{x}\PYG{o}{.}\PYG{n}{plot}\PYG{p}{(}\PYG{n}{x}\PYG{o}{=}\PYG{l+s+s1}{\PYGZsq{}}\PYG{l+s+s1}{R}\PYG{l+s+s1}{\PYGZsq{}}\PYG{p}{,} \PYG{n}{y}\PYG{o}{=}\PYG{l+s+s1}{\PYGZsq{}}\PYG{l+s+s1}{logR}\PYG{l+s+s1}{\PYGZsq{}}\PYG{p}{,} \PYG{n}{kind}\PYG{o}{=}\PYG{l+s+s1}{\PYGZsq{}}\PYG{l+s+s1}{scatter}\PYG{l+s+s1}{\PYGZsq{}}\PYG{p}{)}
    \PYG{n}{plt}\PYG{o}{.}\PYG{n}{title}\PYG{p}{(}\PYG{l+s+sa}{f}\PYG{l+s+s1}{\PYGZsq{}}\PYG{l+s+s1}{Comparison of }\PYG{l+s+si}{\PYGZob{}}\PYG{n}{freq}\PYG{l+s+si}{\PYGZcb{}}\PYG{l+s+s1}{ Log and Simple Returns}\PYG{l+s+s1}{\PYGZsq{}}\PYG{p}{)}
    \PYG{n}{plt}\PYG{o}{.}\PYG{n}{xlabel}\PYG{p}{(}\PYG{n}{labs}\PYG{p}{[}\PYG{l+s+s1}{\PYGZsq{}}\PYG{l+s+s1}{R}\PYG{l+s+s1}{\PYGZsq{}}\PYG{p}{]}\PYG{p}{)}
    \PYG{n}{plt}\PYG{o}{.}\PYG{n}{ylabel}\PYG{p}{(}\PYG{n}{labs}\PYG{p}{[}\PYG{l+s+s1}{\PYGZsq{}}\PYG{l+s+s1}{logR}\PYG{l+s+s1}{\PYGZsq{}}\PYG{p}{]}\PYG{p}{)}
    \PYG{n}{plt}\PYG{o}{.}\PYG{n}{show}\PYG{p}{(}\PYG{p}{)}
\end{sphinxVerbatim}

\end{sphinxuseclass}\end{sphinxVerbatimInput}

\end{sphinxuseclass}
\sphinxAtStartPar
Log and simple returns are similar for small returns.
We almost always see small daily returns, which rarely exceed 10\% in magnitude.
However, at monthly frequencies, we start to see large outliers, where the similarity of log and simple returns starts to break down.

\sphinxAtStartPar
Recall the \sphinxcode{\sphinxupquote{.pipe()}} method allows us to think “data first”.
Here \sphinxcode{\sphinxupquote{mkt\_m.pipe(plot\_compare\_returns, freq='Monthly')}} is the same as \sphinxcode{\sphinxupquote{plot\_compare\_returns(mkt\_m, freq='Monthly')}}.
One is not better than the other, but I find it useful to think “data, then operation, then operation, …”, instead of “operation, then operation, …, then data”.

\begin{sphinxuseclass}{cell}\begin{sphinxVerbatimInput}

\begin{sphinxuseclass}{cell_input}
\begin{sphinxVerbatim}[commandchars=\\\{\}]
\PYG{n}{mkt\PYGZus{}m}\PYG{o}{.}\PYG{n}{pipe}\PYG{p}{(}\PYG{n}{plot\PYGZus{}compare\PYGZus{}returns}\PYG{p}{,} \PYG{n}{freq}\PYG{o}{=}\PYG{l+s+s1}{\PYGZsq{}}\PYG{l+s+s1}{Monthly}\PYG{l+s+s1}{\PYGZsq{}}\PYG{p}{)}
\end{sphinxVerbatim}

\end{sphinxuseclass}\end{sphinxVerbatimInput}
\begin{sphinxVerbatimOutput}

\begin{sphinxuseclass}{cell_output}
\noindent\sphinxincludegraphics{{9da6a387ba9225585f23ec07e7688582194b74a315970a52fe7835f195a9cae0}.png}

\end{sphinxuseclass}\end{sphinxVerbatimOutput}

\end{sphinxuseclass}
\sphinxAtStartPar
At annual frequencies, log and simple returns can be quite different!
For example, at \$R\_\{simple\} \textbackslash{}approx 0.6\$, we see \$R\_\{log\} \textbackslash{}approx 0.4\$.

\begin{sphinxuseclass}{cell}\begin{sphinxVerbatimInput}

\begin{sphinxuseclass}{cell_input}
\begin{sphinxVerbatim}[commandchars=\\\{\}]
\PYG{n}{mkt\PYGZus{}a}\PYG{o}{.}\PYG{n}{pipe}\PYG{p}{(}\PYG{n}{plot\PYGZus{}compare\PYGZus{}returns}\PYG{p}{,} \PYG{n}{freq}\PYG{o}{=}\PYG{l+s+s1}{\PYGZsq{}}\PYG{l+s+s1}{Annual}\PYG{l+s+s1}{\PYGZsq{}}\PYG{p}{)}
\end{sphinxVerbatim}

\end{sphinxuseclass}\end{sphinxVerbatimInput}
\begin{sphinxVerbatimOutput}

\begin{sphinxuseclass}{cell_output}
\noindent\sphinxincludegraphics{{855454bf37961990bedd4c816260c33de93e0077db52c38663f212a23a1127f2}.png}

\end{sphinxuseclass}\end{sphinxVerbatimOutput}

\end{sphinxuseclass}

\subsubsection{Repeat the autocorrelation plot with \sphinxstyleemphasis{monthly} returns instead of \sphinxstyleemphasis{daily} returns}
\label{\detokenize{herron_07_practice_03:repeat-the-autocorrelation-plot-with-monthly-returns-instead-of-daily-returns}}
\sphinxAtStartPar
Again, we will write a function to plot these autocorrelations.


\subsubsection{Plot monthly returns versus their one\sphinxhyphen{}month lag}
\label{\detokenize{herron_07_practice_03:plot-monthly-returns-versus-their-one-month-lag}}
\begin{sphinxuseclass}{cell}\begin{sphinxVerbatimInput}

\begin{sphinxuseclass}{cell_input}
\begin{sphinxVerbatim}[commandchars=\\\{\}]
\PYG{k}{def} \PYG{n+nf}{plot\PYGZus{}autocorr}\PYG{p}{(}\PYG{n}{x}\PYG{p}{,} \PYG{n}{freq}\PYG{p}{,} \PYG{n}{n}\PYG{o}{=}\PYG{l+m+mi}{10}\PYG{p}{,} \PYG{n}{labs}\PYG{o}{=}\PYG{n}{labels}\PYG{p}{)}\PYG{p}{:}
    \PYG{n}{x\PYGZus{}lags} \PYG{o}{=} \PYG{n}{pd}\PYG{o}{.}\PYG{n}{concat}\PYG{p}{(}\PYG{n}{objs}\PYG{o}{=}\PYG{p}{[}\PYG{n}{x}\PYG{p}{[}\PYG{l+s+s1}{\PYGZsq{}}\PYG{l+s+s1}{R}\PYG{l+s+s1}{\PYGZsq{}}\PYG{p}{]}\PYG{o}{.}\PYG{n}{shift}\PYG{p}{(}\PYG{n}{t}\PYG{p}{)} \PYG{k}{for} \PYG{n}{t} \PYG{o+ow}{in} \PYG{n+nb}{range}\PYG{p}{(}\PYG{n}{n} \PYG{o}{+} \PYG{l+m+mi}{1}\PYG{p}{)}\PYG{p}{]}\PYG{p}{,} \PYG{n}{axis}\PYG{o}{=}\PYG{l+m+mi}{1}\PYG{p}{)}
    \PYG{n}{corrs} \PYG{o}{=} \PYG{n}{x\PYGZus{}lags}\PYG{o}{.}\PYG{n}{corr}\PYG{p}{(}\PYG{p}{)}\PYG{o}{.}\PYG{n}{iloc}\PYG{p}{[}\PYG{l+m+mi}{0}\PYG{p}{]}
    \PYG{n}{serrs} \PYG{o}{=} \PYG{n}{np}\PYG{o}{.}\PYG{n}{sqrt}\PYG{p}{(}\PYG{p}{(}\PYG{l+m+mi}{1} \PYG{o}{\PYGZhy{}} \PYG{n}{corrs}\PYG{o}{*}\PYG{o}{*}\PYG{l+m+mi}{2}\PYG{p}{)} \PYG{o}{/} \PYG{p}{(}\PYG{n}{x\PYGZus{}lags}\PYG{o}{.}\PYG{n}{count}\PYG{p}{(}\PYG{p}{)} \PYG{o}{\PYGZhy{}} \PYG{l+m+mi}{2}\PYG{p}{)}\PYG{p}{)}

    \PYG{n}{plt}\PYG{o}{.}\PYG{n}{bar}\PYG{p}{(}\PYG{n}{height}\PYG{o}{=}\PYG{n}{corrs}\PYG{p}{,} \PYG{n}{x}\PYG{o}{=}\PYG{n+nb}{range}\PYG{p}{(}\PYG{n}{n} \PYG{o}{+} \PYG{l+m+mi}{1}\PYG{p}{)}\PYG{p}{,} \PYG{n}{yerr}\PYG{o}{=}\PYG{l+m+mi}{2}\PYG{o}{*}\PYG{n}{serrs}\PYG{p}{)}
    \PYG{n}{plt}\PYG{o}{.}\PYG{n}{title}\PYG{p}{(}
        \PYG{l+s+sa}{f}\PYG{l+s+s1}{\PYGZsq{}}\PYG{l+s+s1}{Autocorrelation of }\PYG{l+s+si}{\PYGZob{}}\PYG{n}{freq}\PYG{l+s+si}{\PYGZcb{}}\PYG{l+s+s1}{ Returns}\PYG{l+s+s1}{\PYGZsq{}} \PYG{o}{+}
        \PYG{l+s+s1}{\PYGZsq{}}\PYG{l+s+se}{\PYGZbs{}n}\PYG{l+s+s1}{Black Vertical Bars Indicate Standard Errors}\PYG{l+s+s1}{\PYGZsq{}}
    \PYG{p}{)}
    \PYG{n}{plt}\PYG{o}{.}\PYG{n}{xlabel}\PYG{p}{(}\PYG{l+s+sa}{f}\PYG{l+s+s1}{\PYGZsq{}}\PYG{l+s+si}{\PYGZob{}}\PYG{n}{freq}\PYG{l+s+si}{\PYGZcb{}}\PYG{l+s+s1}{ Lag}\PYG{l+s+s1}{\PYGZsq{}}\PYG{p}{)}
    \PYG{n}{plt}\PYG{o}{.}\PYG{n}{ylabel}\PYG{p}{(}\PYG{l+s+s1}{\PYGZsq{}}\PYG{l+s+s1}{Autocorrelation Coefficient}\PYG{l+s+s1}{\PYGZsq{}}\PYG{p}{)}
    \PYG{n}{plt}\PYG{o}{.}\PYG{n}{show}\PYG{p}{(}\PYG{p}{)}
\end{sphinxVerbatim}

\end{sphinxuseclass}\end{sphinxVerbatimInput}

\end{sphinxuseclass}
\sphinxAtStartPar
At the monthly horizon, return autocorrelation weakens!
All autocorrelation coefficient estimates are small and less than two standard errors away from zero.

\begin{sphinxuseclass}{cell}\begin{sphinxVerbatimInput}

\begin{sphinxuseclass}{cell_input}
\begin{sphinxVerbatim}[commandchars=\\\{\}]
\PYG{n}{mkt\PYGZus{}m}\PYG{o}{.}\PYG{n}{pipe}\PYG{p}{(}\PYG{n}{plot\PYGZus{}autocorr}\PYG{p}{,} \PYG{n}{freq}\PYG{o}{=}\PYG{l+s+s1}{\PYGZsq{}}\PYG{l+s+s1}{Monthly}\PYG{l+s+s1}{\PYGZsq{}}\PYG{p}{)}
\end{sphinxVerbatim}

\end{sphinxuseclass}\end{sphinxVerbatimInput}
\begin{sphinxVerbatimOutput}

\begin{sphinxuseclass}{cell_output}
\noindent\sphinxincludegraphics{{577eff5e19e19dc54941bc7a9834ee62adebb90307d7950137937ecbff02aa19}.png}

\end{sphinxuseclass}\end{sphinxVerbatimOutput}

\end{sphinxuseclass}
\sphinxAtStartPar
At the annual horizon, return autocorrelation disappears!

\begin{sphinxuseclass}{cell}\begin{sphinxVerbatimInput}

\begin{sphinxuseclass}{cell_input}
\begin{sphinxVerbatim}[commandchars=\\\{\}]
\PYG{n}{mkt\PYGZus{}a}\PYG{o}{.}\PYG{n}{pipe}\PYG{p}{(}\PYG{n}{plot\PYGZus{}autocorr}\PYG{p}{,} \PYG{n}{freq}\PYG{o}{=}\PYG{l+s+s1}{\PYGZsq{}}\PYG{l+s+s1}{Annual}\PYG{l+s+s1}{\PYGZsq{}}\PYG{p}{)}
\end{sphinxVerbatim}

\end{sphinxuseclass}\end{sphinxVerbatimInput}
\begin{sphinxVerbatimOutput}

\begin{sphinxuseclass}{cell_output}
\noindent\sphinxincludegraphics{{dab8e1e9e00f986ab847e9a1a2c2afdd7231c32adf286c828b84b9f47c7c3308}.png}

\end{sphinxuseclass}\end{sphinxVerbatimOutput}

\end{sphinxuseclass}

\subsubsection{Repeat the autocorrelation plot with the \sphinxstyleemphasis{absolute value} of daily returns}
\label{\detokenize{herron_07_practice_03:repeat-the-autocorrelation-plot-with-the-absolute-value-of-daily-returns}}
\sphinxAtStartPar
We can use our function again!
But first, we have to convert simple returns to the \sphinxstyleemphasis{absolute value} of simple returns.

\begin{sphinxuseclass}{cell}\begin{sphinxVerbatimInput}

\begin{sphinxuseclass}{cell_input}
\begin{sphinxVerbatim}[commandchars=\\\{\}]
\PYG{p}{(}
    \PYG{n}{mkt}
    \PYG{o}{.}\PYG{n}{assign}\PYG{p}{(}\PYG{n}{R}\PYG{o}{=}\PYG{k}{lambda} \PYG{n}{x}\PYG{p}{:} \PYG{n}{x}\PYG{p}{[}\PYG{l+s+s1}{\PYGZsq{}}\PYG{l+s+s1}{R}\PYG{l+s+s1}{\PYGZsq{}}\PYG{p}{]}\PYG{o}{.}\PYG{n}{abs}\PYG{p}{(}\PYG{p}{)}\PYG{p}{)}
    \PYG{o}{.}\PYG{n}{pipe}\PYG{p}{(}\PYG{n}{plot\PYGZus{}autocorr}\PYG{p}{,} \PYG{n}{freq}\PYG{o}{=}\PYG{l+s+s1}{\PYGZsq{}}\PYG{l+s+s1}{Absolute Value of Daily}\PYG{l+s+s1}{\PYGZsq{}}\PYG{p}{,} \PYG{n}{n}\PYG{o}{=}\PYG{l+m+mi}{60}\PYG{p}{)}
\PYG{p}{)}
\end{sphinxVerbatim}

\end{sphinxuseclass}\end{sphinxVerbatimInput}
\begin{sphinxVerbatimOutput}

\begin{sphinxuseclass}{cell_output}
\noindent\sphinxincludegraphics{{d3d7bf9c70dc6c456097b3d68a8ee74b909e4e809e7c5da6b7afc41f17dcc8e3}.png}

\end{sphinxuseclass}\end{sphinxVerbatimOutput}

\end{sphinxuseclass}
\sphinxAtStartPar
Saying that “squared returns” or “the absolute value of returns” is autocorrelated is similar to saying “return magnitudes” are autocorrelated, which is similar to saying “volatility” is autocorrelated or persistent.


\subsubsection{Repeat the leverage effect plot with lagged returns}
\label{\detokenize{herron_07_practice_03:repeat-the-leverage-effect-plot-with-lagged-returns}}
\sphinxAtStartPar
We will only make this plot, but we can still write a function!

\begin{sphinxuseclass}{cell}\begin{sphinxVerbatimInput}

\begin{sphinxuseclass}{cell_input}
\begin{sphinxVerbatim}[commandchars=\\\{\}]
\PYG{k+kn}{import} \PYG{n+nn}{seaborn} \PYG{k}{as} \PYG{n+nn}{sns}
\end{sphinxVerbatim}

\end{sphinxuseclass}\end{sphinxVerbatimInput}

\end{sphinxuseclass}
\begin{sphinxuseclass}{cell}\begin{sphinxVerbatimInput}

\begin{sphinxuseclass}{cell_input}
\begin{sphinxVerbatim}[commandchars=\\\{\}]
\PYG{k}{def} \PYG{n+nf}{plot\PYGZus{}leverage\PYGZus{}effect}\PYG{p}{(}\PYG{n}{x}\PYG{p}{,} \PYG{n}{rule}\PYG{p}{,} \PYG{n}{freq}\PYG{p}{)}\PYG{p}{:}
    \PYG{p}{(}
        \PYG{n}{x}\PYG{p}{[}\PYG{l+s+s1}{\PYGZsq{}}\PYG{l+s+s1}{R}\PYG{l+s+s1}{\PYGZsq{}}\PYG{p}{]}
        \PYG{o}{.}\PYG{n}{resample}\PYG{p}{(}\PYG{n}{rule}\PYG{o}{=}\PYG{n}{rule}\PYG{p}{,} \PYG{n}{kind}\PYG{o}{=}\PYG{l+s+s1}{\PYGZsq{}}\PYG{l+s+s1}{period}\PYG{l+s+s1}{\PYGZsq{}}\PYG{p}{)}
        \PYG{o}{.}\PYG{n}{agg}\PYG{p}{(}\PYG{p}{[}\PYG{n}{totret}\PYG{p}{,} \PYG{l+s+s1}{\PYGZsq{}}\PYG{l+s+s1}{std}\PYG{l+s+s1}{\PYGZsq{}}\PYG{p}{]}\PYG{p}{)}
        \PYG{o}{.}\PYG{n}{assign}\PYG{p}{(}\PYG{n}{totret\PYGZus{}lag}\PYG{o}{=}\PYG{k}{lambda} \PYG{n}{x}\PYG{p}{:} \PYG{n}{x}\PYG{p}{[}\PYG{l+s+s1}{\PYGZsq{}}\PYG{l+s+s1}{totret}\PYG{l+s+s1}{\PYGZsq{}}\PYG{p}{]}\PYG{o}{.}\PYG{n}{shift}\PYG{p}{(}\PYG{p}{)}\PYG{p}{)}
        \PYG{o}{.}\PYG{n}{mul}\PYG{p}{(}\PYG{l+m+mi}{100}\PYG{p}{)}
        \PYG{o}{.}\PYG{n}{pipe}\PYG{p}{(}\PYG{k}{lambda} \PYG{n}{x}\PYG{p}{:}
              \PYG{n}{sns}\PYG{o}{.}\PYG{n}{regplot}\PYG{p}{(}
                  \PYG{n}{x}\PYG{o}{=}\PYG{l+s+s1}{\PYGZsq{}}\PYG{l+s+s1}{totret\PYGZus{}lag}\PYG{l+s+s1}{\PYGZsq{}}\PYG{p}{,} 
                  \PYG{n}{y}\PYG{o}{=}\PYG{l+s+s1}{\PYGZsq{}}\PYG{l+s+s1}{std}\PYG{l+s+s1}{\PYGZsq{}}\PYG{p}{,} 
                  \PYG{n}{data}\PYG{o}{=}\PYG{n}{x}\PYG{p}{,}
                  \PYG{n}{scatter\PYGZus{}kws}\PYG{o}{=}\PYG{p}{\PYGZob{}}\PYG{l+s+s1}{\PYGZsq{}}\PYG{l+s+s1}{alpha}\PYG{l+s+s1}{\PYGZsq{}}\PYG{p}{:} \PYG{l+m+mf}{0.1}\PYG{p}{\PYGZcb{}}
            \PYG{p}{)}
        \PYG{p}{)}
    \PYG{p}{)}
    \PYG{n}{plt}\PYG{o}{.}\PYG{n}{xlabel}\PYG{p}{(}\PYG{l+s+sa}{f}\PYG{l+s+s1}{\PYGZsq{}}\PYG{l+s+s1}{One\PYGZhy{}}\PYG{l+s+si}{\PYGZob{}}\PYG{n}{freq}\PYG{l+s+si}{\PYGZcb{}}\PYG{l+s+s1}{ Lag of Total Return (\PYGZpc{})}\PYG{l+s+s1}{\PYGZsq{}}\PYG{p}{)}
    \PYG{n}{plt}\PYG{o}{.}\PYG{n}{ylabel}\PYG{p}{(}\PYG{l+s+s1}{\PYGZsq{}}\PYG{l+s+s1}{Volatility of Returns (}\PYG{l+s+s1}{\PYGZpc{}}\PYG{l+s+s1}{)}\PYG{l+s+s1}{\PYGZsq{}}\PYG{p}{)}
    \PYG{n}{plt}\PYG{o}{.}\PYG{n}{suptitle}\PYG{p}{(}
        \PYG{l+s+s1}{\PYGZsq{}}\PYG{l+s+s1}{Fisher Black}\PYG{l+s+se}{\PYGZbs{}\PYGZsq{}}\PYG{l+s+s1}{s Leverage Effect}\PYG{l+s+se}{\PYGZbs{}n}\PYG{l+s+s1}{\PYGZsq{}} \PYG{o}{+} 
        \PYG{l+s+sa}{f}\PYG{l+s+s1}{\PYGZsq{}}\PYG{l+s+s1}{(Daily Returns Aggregated by }\PYG{l+s+si}{\PYGZob{}}\PYG{n}{freq}\PYG{l+s+si}{\PYGZcb{}}\PYG{l+s+s1}{)}\PYG{l+s+s1}{\PYGZsq{}}
    \PYG{p}{)}
    \PYG{n}{plt}\PYG{o}{.}\PYG{n}{show}\PYG{p}{(}\PYG{p}{)}
\end{sphinxVerbatim}

\end{sphinxuseclass}\end{sphinxVerbatimInput}

\end{sphinxuseclass}
\begin{sphinxuseclass}{cell}\begin{sphinxVerbatimInput}

\begin{sphinxuseclass}{cell_input}
\begin{sphinxVerbatim}[commandchars=\\\{\}]
\PYG{n}{mkt}\PYG{o}{.}\PYG{n}{pipe}\PYG{p}{(}\PYG{n}{plot\PYGZus{}leverage\PYGZus{}effect}\PYG{p}{,} \PYG{n}{rule}\PYG{o}{=}\PYG{l+s+s1}{\PYGZsq{}}\PYG{l+s+s1}{M}\PYG{l+s+s1}{\PYGZsq{}}\PYG{p}{,} \PYG{n}{freq}\PYG{o}{=}\PYG{l+s+s1}{\PYGZsq{}}\PYG{l+s+s1}{Month}\PYG{l+s+s1}{\PYGZsq{}}\PYG{p}{)}
\end{sphinxVerbatim}

\end{sphinxuseclass}\end{sphinxVerbatimInput}
\begin{sphinxVerbatimOutput}

\begin{sphinxuseclass}{cell_output}
\noindent\sphinxincludegraphics{{62e7001c503c8e60917b5100d02409dcd3ccf9c7e9ece1ec61310b466678b5e4}.png}

\end{sphinxuseclass}\end{sphinxVerbatimOutput}

\end{sphinxuseclass}
\sphinxAtStartPar
Since we have the function, let use a try a few different horizons!

\begin{sphinxuseclass}{cell}\begin{sphinxVerbatimInput}

\begin{sphinxuseclass}{cell_input}
\begin{sphinxVerbatim}[commandchars=\\\{\}]
\PYG{n}{mkt}\PYG{o}{.}\PYG{n}{pipe}\PYG{p}{(}\PYG{n}{plot\PYGZus{}leverage\PYGZus{}effect}\PYG{p}{,} \PYG{n}{rule}\PYG{o}{=}\PYG{l+s+s1}{\PYGZsq{}}\PYG{l+s+s1}{Q}\PYG{l+s+s1}{\PYGZsq{}}\PYG{p}{,} \PYG{n}{freq}\PYG{o}{=}\PYG{l+s+s1}{\PYGZsq{}}\PYG{l+s+s1}{Quarter}\PYG{l+s+s1}{\PYGZsq{}}\PYG{p}{)}
\end{sphinxVerbatim}

\end{sphinxuseclass}\end{sphinxVerbatimInput}
\begin{sphinxVerbatimOutput}

\begin{sphinxuseclass}{cell_output}
\noindent\sphinxincludegraphics{{ff976ba304a97f96d983f02f7485f56eb23bfd23cfbcfd87033ab33a9f5ccaeb}.png}

\end{sphinxuseclass}\end{sphinxVerbatimOutput}

\end{sphinxuseclass}
\begin{sphinxuseclass}{cell}\begin{sphinxVerbatimInput}

\begin{sphinxuseclass}{cell_input}
\begin{sphinxVerbatim}[commandchars=\\\{\}]
\PYG{n}{mkt}\PYG{o}{.}\PYG{n}{pipe}\PYG{p}{(}\PYG{n}{plot\PYGZus{}leverage\PYGZus{}effect}\PYG{p}{,} \PYG{n}{rule}\PYG{o}{=}\PYG{l+s+s1}{\PYGZsq{}}\PYG{l+s+s1}{A}\PYG{l+s+s1}{\PYGZsq{}}\PYG{p}{,} \PYG{n}{freq}\PYG{o}{=}\PYG{l+s+s1}{\PYGZsq{}}\PYG{l+s+s1}{Annual}\PYG{l+s+s1}{\PYGZsq{}}\PYG{p}{)}
\end{sphinxVerbatim}

\end{sphinxuseclass}\end{sphinxVerbatimInput}
\begin{sphinxVerbatimOutput}

\begin{sphinxuseclass}{cell_output}
\noindent\sphinxincludegraphics{{46a58722d1a1c1c4c0fdb996dbfea5e9a3cc031b37dabf1b49a0f8fae607a745}.png}

\end{sphinxuseclass}\end{sphinxVerbatimOutput}

\end{sphinxuseclass}
\sphinxstepscope


\section{Herron Topic 7 \sphinxhyphen{} Practice (Wednesday 11:45 AM, Section 4)}
\label{\detokenize{herron_07_practice_04:herron-topic-7-practice-wednesday-11-45-am-section-4}}\label{\detokenize{herron_07_practice_04::doc}}

\subsection{Announcements}
\label{\detokenize{herron_07_practice_04:announcements}}\begin{itemize}
\item {} 
\sphinxAtStartPar
Quiz 7 (our last quiz!) is due Friday, 4/14, at 11:59 PM

\item {} 
\sphinxAtStartPar
Week of 4/17
\begin{itemize}
\item {} 
\sphinxAtStartPar
I will record a lecture for Herron topic 5 on simulations and post a practice notebook

\item {} 
\sphinxAtStartPar
However, we will use class time for group work

\item {} 
\sphinxAtStartPar
This is a compromise between survey responses (1/3 wanted to drop simulations, 2/3 did not) and Monday’s holiday (Patriot’s Day)

\end{itemize}

\item {} 
\sphinxAtStartPar
Week of 4/24
\begin{itemize}
\item {} 
\sphinxAtStartPar
No lecture

\item {} 
\sphinxAtStartPar
We will use class time for group work

\end{itemize}

\item {} 
\sphinxAtStartPar
Due dates
\begin{itemize}
\item {} 
\sphinxAtStartPar
Project 2 at 11:59 PM on Wednesday, 4/26

\item {} 
\sphinxAtStartPar
Teammates review 2 at 11:59 PM on Wednesday, 4/26

\item {} 
\sphinxAtStartPar
30,000 DataCamp XP at 11:59 PM on Friday, 4/28

\end{itemize}

\item {} 
\sphinxAtStartPar
\sphinxstyleemphasis{\sphinxstylestrong{Please complete TRACE feedback on what topics, tools, and techniques you want more of}}

\end{itemize}


\subsection{Practice}
\label{\detokenize{herron_07_practice_04:practice}}
\begin{sphinxuseclass}{cell}\begin{sphinxVerbatimInput}

\begin{sphinxuseclass}{cell_input}
\begin{sphinxVerbatim}[commandchars=\\\{\}]
\PYG{k+kn}{import} \PYG{n+nn}{matplotlib}\PYG{n+nn}{.}\PYG{n+nn}{pyplot} \PYG{k}{as} \PYG{n+nn}{plt}
\PYG{k+kn}{import} \PYG{n+nn}{numpy} \PYG{k}{as} \PYG{n+nn}{np}
\PYG{k+kn}{import} \PYG{n+nn}{pandas} \PYG{k}{as} \PYG{n+nn}{pd}
\end{sphinxVerbatim}

\end{sphinxuseclass}\end{sphinxVerbatimInput}

\end{sphinxuseclass}
\begin{sphinxuseclass}{cell}\begin{sphinxVerbatimInput}

\begin{sphinxuseclass}{cell_input}
\begin{sphinxVerbatim}[commandchars=\\\{\}]
\PYG{o}{\PYGZpc{}}\PYG{k}{config} InlineBackend.figure\PYGZus{}format = \PYGZsq{}retina\PYGZsq{}
\PYG{o}{\PYGZpc{}}\PYG{k}{precision} 4
\PYG{n}{pd}\PYG{o}{.}\PYG{n}{options}\PYG{o}{.}\PYG{n}{display}\PYG{o}{.}\PYG{n}{float\PYGZus{}format} \PYG{o}{=} \PYG{l+s+s1}{\PYGZsq{}}\PYG{l+s+si}{\PYGZob{}:.4f\PYGZcb{}}\PYG{l+s+s1}{\PYGZsq{}}\PYG{o}{.}\PYG{n}{format}
\end{sphinxVerbatim}

\end{sphinxuseclass}\end{sphinxVerbatimInput}

\end{sphinxuseclass}
\begin{sphinxuseclass}{cell}\begin{sphinxVerbatimInput}

\begin{sphinxuseclass}{cell_input}
\begin{sphinxVerbatim}[commandchars=\\\{\}]
\PYG{k+kn}{import} \PYG{n+nn}{requests\PYGZus{}cache}
\PYG{n}{session} \PYG{o}{=} \PYG{n}{requests\PYGZus{}cache}\PYG{o}{.}\PYG{n}{CachedSession}\PYG{p}{(}\PYG{p}{)}
\PYG{k+kn}{import} \PYG{n+nn}{yfinance} \PYG{k}{as} \PYG{n+nn}{yf}
\PYG{k+kn}{import} \PYG{n+nn}{pandas\PYGZus{}datareader} \PYG{k}{as} \PYG{n+nn}{pdr}
\end{sphinxVerbatim}

\end{sphinxuseclass}\end{sphinxVerbatimInput}

\end{sphinxuseclass}
\begin{sphinxuseclass}{cell}\begin{sphinxVerbatimInput}

\begin{sphinxuseclass}{cell_input}
\begin{sphinxVerbatim}[commandchars=\\\{\}]
\PYG{n}{mkt} \PYG{o}{=} \PYG{p}{(}
    \PYG{n}{pdr}\PYG{o}{.}\PYG{n}{DataReader}\PYG{p}{(}
        \PYG{n}{name}\PYG{o}{=}\PYG{l+s+s1}{\PYGZsq{}}\PYG{l+s+s1}{F\PYGZhy{}F\PYGZus{}Research\PYGZus{}Data\PYGZus{}Factors\PYGZus{}daily}\PYG{l+s+s1}{\PYGZsq{}}\PYG{p}{,}
        \PYG{n}{data\PYGZus{}source}\PYG{o}{=}\PYG{l+s+s1}{\PYGZsq{}}\PYG{l+s+s1}{famafrench}\PYG{l+s+s1}{\PYGZsq{}}\PYG{p}{,}
        \PYG{n}{start}\PYG{o}{=}\PYG{l+s+s1}{\PYGZsq{}}\PYG{l+s+s1}{1900}\PYG{l+s+s1}{\PYGZsq{}}\PYG{p}{,}
        \PYG{n}{session}\PYG{o}{=}\PYG{n}{session}
    \PYG{p}{)}
    \PYG{p}{[}\PYG{l+m+mi}{0}\PYG{p}{]}
    \PYG{o}{.}\PYG{n}{assign}\PYG{p}{(}
        \PYG{n}{R}\PYG{o}{=}\PYG{k}{lambda} \PYG{n}{x}\PYG{p}{:} \PYG{p}{(}\PYG{n}{x}\PYG{p}{[}\PYG{l+s+s1}{\PYGZsq{}}\PYG{l+s+s1}{Mkt\PYGZhy{}RF}\PYG{l+s+s1}{\PYGZsq{}}\PYG{p}{]} \PYG{o}{+} \PYG{n}{x}\PYG{p}{[}\PYG{l+s+s1}{\PYGZsq{}}\PYG{l+s+s1}{RF}\PYG{l+s+s1}{\PYGZsq{}}\PYG{p}{]}\PYG{p}{)} \PYG{o}{/} \PYG{l+m+mi}{100}\PYG{p}{,}
        \PYG{n}{logR}\PYG{o}{=}\PYG{k}{lambda} \PYG{n}{x}\PYG{p}{:} \PYG{n}{np}\PYG{o}{.}\PYG{n}{log1p}\PYG{p}{(}\PYG{n}{x}\PYG{p}{[}\PYG{l+s+s1}{\PYGZsq{}}\PYG{l+s+s1}{R}\PYG{l+s+s1}{\PYGZsq{}}\PYG{p}{]}\PYG{p}{)}
    \PYG{p}{)}
    \PYG{p}{[}\PYG{p}{[}\PYG{l+s+s1}{\PYGZsq{}}\PYG{l+s+s1}{R}\PYG{l+s+s1}{\PYGZsq{}}\PYG{p}{,} \PYG{l+s+s1}{\PYGZsq{}}\PYG{l+s+s1}{logR}\PYG{l+s+s1}{\PYGZsq{}}\PYG{p}{]}\PYG{p}{]}
\PYG{p}{)}
\end{sphinxVerbatim}

\end{sphinxuseclass}\end{sphinxVerbatimInput}

\end{sphinxuseclass}
\begin{sphinxuseclass}{cell}\begin{sphinxVerbatimInput}

\begin{sphinxuseclass}{cell_input}
\begin{sphinxVerbatim}[commandchars=\\\{\}]
\PYG{n}{labels} \PYG{o}{=} \PYG{p}{\PYGZob{}}
    \PYG{l+s+s1}{\PYGZsq{}}\PYG{l+s+s1}{R}\PYG{l+s+s1}{\PYGZsq{}}\PYG{p}{:} \PYG{l+s+s1}{\PYGZsq{}}\PYG{l+s+s1}{Simple Return}\PYG{l+s+s1}{\PYGZsq{}}\PYG{p}{,}
    \PYG{l+s+s1}{\PYGZsq{}}\PYG{l+s+s1}{logR}\PYG{l+s+s1}{\PYGZsq{}}\PYG{p}{:} \PYG{l+s+s1}{\PYGZsq{}}\PYG{l+s+s1}{Log Return}\PYG{l+s+s1}{\PYGZsq{}}
\PYG{p}{\PYGZcb{}}
\end{sphinxVerbatim}

\end{sphinxuseclass}\end{sphinxVerbatimInput}

\end{sphinxuseclass}

\subsubsection{Repeat the comparison of log and simple returns with monthly and annual returns}
\label{\detokenize{herron_07_practice_04:repeat-the-comparison-of-log-and-simple-returns-with-monthly-and-annual-returns}}
\sphinxAtStartPar
We could download monthly and annual data from the Ken French data library.
However, we can also use the \sphinxcode{\sphinxupquote{.resample()}} method to compound daily returns into monthly and annual returns.
Let us use the \sphinxcode{\sphinxupquote{.resample()}} method (1) for practice and (2) because the \sphinxcode{\sphinxupquote{.resample()}} code is simpler and more compact than the \sphinxcode{\sphinxupquote{pdr.DataReader()}} code.
We need to re\sphinxhyphen{}calculate the log return.
We could instead sum the daily log returns, but combining the \sphinxcode{\sphinxupquote{.resample()}} method with two different operations on two differ columns is complex and easily avoided here.

\begin{sphinxuseclass}{cell}\begin{sphinxVerbatimInput}

\begin{sphinxuseclass}{cell_input}
\begin{sphinxVerbatim}[commandchars=\\\{\}]
\PYG{k}{def} \PYG{n+nf}{totret}\PYG{p}{(}\PYG{n}{x}\PYG{p}{)}\PYG{p}{:}
    \PYG{k}{return} \PYG{p}{(}\PYG{l+m+mi}{1} \PYG{o}{+} \PYG{n}{x}\PYG{p}{)}\PYG{o}{.}\PYG{n}{prod}\PYG{p}{(}\PYG{p}{)} \PYG{o}{\PYGZhy{}} \PYG{l+m+mi}{1}
\end{sphinxVerbatim}

\end{sphinxuseclass}\end{sphinxVerbatimInput}

\end{sphinxuseclass}
\begin{sphinxuseclass}{cell}\begin{sphinxVerbatimInput}

\begin{sphinxuseclass}{cell_input}
\begin{sphinxVerbatim}[commandchars=\\\{\}]
\PYG{n}{mkt\PYGZus{}m} \PYG{o}{=} \PYG{p}{(}
    \PYG{n}{mkt}\PYG{p}{[}\PYG{p}{[}\PYG{l+s+s1}{\PYGZsq{}}\PYG{l+s+s1}{R}\PYG{l+s+s1}{\PYGZsq{}}\PYG{p}{]}\PYG{p}{]}
    \PYG{o}{.}\PYG{n}{resample}\PYG{p}{(}\PYG{n}{rule}\PYG{o}{=}\PYG{l+s+s1}{\PYGZsq{}}\PYG{l+s+s1}{M}\PYG{l+s+s1}{\PYGZsq{}}\PYG{p}{,} \PYG{n}{kind}\PYG{o}{=}\PYG{l+s+s1}{\PYGZsq{}}\PYG{l+s+s1}{period}\PYG{l+s+s1}{\PYGZsq{}}\PYG{p}{)}
    \PYG{o}{.}\PYG{n}{apply}\PYG{p}{(}\PYG{n}{totret}\PYG{p}{)}
    \PYG{o}{.}\PYG{n}{assign}\PYG{p}{(}\PYG{n}{logR}\PYG{o}{=}\PYG{k}{lambda} \PYG{n}{x}\PYG{p}{:} \PYG{n}{np}\PYG{o}{.}\PYG{n}{log1p}\PYG{p}{(}\PYG{n}{x}\PYG{p}{[}\PYG{l+s+s1}{\PYGZsq{}}\PYG{l+s+s1}{R}\PYG{l+s+s1}{\PYGZsq{}}\PYG{p}{]}\PYG{p}{)}\PYG{p}{)}
\PYG{p}{)}
\end{sphinxVerbatim}

\end{sphinxuseclass}\end{sphinxVerbatimInput}

\end{sphinxuseclass}
\begin{sphinxuseclass}{cell}\begin{sphinxVerbatimInput}

\begin{sphinxuseclass}{cell_input}
\begin{sphinxVerbatim}[commandchars=\\\{\}]
\PYG{n}{mkt\PYGZus{}a} \PYG{o}{=} \PYG{p}{(}
    \PYG{n}{mkt}\PYG{p}{[}\PYG{p}{[}\PYG{l+s+s1}{\PYGZsq{}}\PYG{l+s+s1}{R}\PYG{l+s+s1}{\PYGZsq{}}\PYG{p}{]}\PYG{p}{]}
    \PYG{o}{.}\PYG{n}{resample}\PYG{p}{(}\PYG{n}{rule}\PYG{o}{=}\PYG{l+s+s1}{\PYGZsq{}}\PYG{l+s+s1}{A}\PYG{l+s+s1}{\PYGZsq{}}\PYG{p}{,} \PYG{n}{kind}\PYG{o}{=}\PYG{l+s+s1}{\PYGZsq{}}\PYG{l+s+s1}{period}\PYG{l+s+s1}{\PYGZsq{}}\PYG{p}{)}
    \PYG{o}{.}\PYG{n}{apply}\PYG{p}{(}\PYG{n}{totret}\PYG{p}{)}
    \PYG{o}{.}\PYG{n}{assign}\PYG{p}{(}\PYG{n}{logR}\PYG{o}{=}\PYG{k}{lambda} \PYG{n}{x}\PYG{p}{:} \PYG{n}{np}\PYG{o}{.}\PYG{n}{log1p}\PYG{p}{(}\PYG{n}{x}\PYG{p}{[}\PYG{l+s+s1}{\PYGZsq{}}\PYG{l+s+s1}{R}\PYG{l+s+s1}{\PYGZsq{}}\PYG{p}{]}\PYG{p}{)}\PYG{p}{)}
\PYG{p}{)}
\end{sphinxVerbatim}

\end{sphinxuseclass}\end{sphinxVerbatimInput}

\end{sphinxuseclass}
\sphinxAtStartPar
We want to repeat this plot for monthly and annual returns, so we can make out lives easier by writing a function.

\begin{sphinxuseclass}{cell}\begin{sphinxVerbatimInput}

\begin{sphinxuseclass}{cell_input}
\begin{sphinxVerbatim}[commandchars=\\\{\}]
\PYG{k}{def} \PYG{n+nf}{plot\PYGZus{}compare\PYGZus{}returns}\PYG{p}{(}\PYG{n}{x}\PYG{p}{,} \PYG{n}{freq}\PYG{p}{,} \PYG{n}{labs}\PYG{o}{=}\PYG{n}{labels}\PYG{p}{)}\PYG{p}{:}
    \PYG{n}{x}\PYG{o}{.}\PYG{n}{plot}\PYG{p}{(}\PYG{n}{x}\PYG{o}{=}\PYG{l+s+s1}{\PYGZsq{}}\PYG{l+s+s1}{R}\PYG{l+s+s1}{\PYGZsq{}}\PYG{p}{,} \PYG{n}{y}\PYG{o}{=}\PYG{l+s+s1}{\PYGZsq{}}\PYG{l+s+s1}{logR}\PYG{l+s+s1}{\PYGZsq{}}\PYG{p}{,} \PYG{n}{kind}\PYG{o}{=}\PYG{l+s+s1}{\PYGZsq{}}\PYG{l+s+s1}{scatter}\PYG{l+s+s1}{\PYGZsq{}}\PYG{p}{)}
    \PYG{n}{plt}\PYG{o}{.}\PYG{n}{title}\PYG{p}{(}\PYG{l+s+sa}{f}\PYG{l+s+s1}{\PYGZsq{}}\PYG{l+s+s1}{Comparison of }\PYG{l+s+si}{\PYGZob{}}\PYG{n}{freq}\PYG{l+s+si}{\PYGZcb{}}\PYG{l+s+s1}{ Log and Simple Returns}\PYG{l+s+s1}{\PYGZsq{}}\PYG{p}{)}
    \PYG{n}{plt}\PYG{o}{.}\PYG{n}{xlabel}\PYG{p}{(}\PYG{n}{labs}\PYG{p}{[}\PYG{l+s+s1}{\PYGZsq{}}\PYG{l+s+s1}{R}\PYG{l+s+s1}{\PYGZsq{}}\PYG{p}{]}\PYG{p}{)}
    \PYG{n}{plt}\PYG{o}{.}\PYG{n}{ylabel}\PYG{p}{(}\PYG{n}{labs}\PYG{p}{[}\PYG{l+s+s1}{\PYGZsq{}}\PYG{l+s+s1}{logR}\PYG{l+s+s1}{\PYGZsq{}}\PYG{p}{]}\PYG{p}{)}
    \PYG{n}{plt}\PYG{o}{.}\PYG{n}{show}\PYG{p}{(}\PYG{p}{)}
\end{sphinxVerbatim}

\end{sphinxuseclass}\end{sphinxVerbatimInput}

\end{sphinxuseclass}
\sphinxAtStartPar
Log and simple returns are similar for small returns.
We almost always see small daily returns, which rarely exceed 10\% in magnitude.
However, at monthly frequencies, we start to see large outliers, where the similarity of log and simple returns starts to break down.

\sphinxAtStartPar
Recall the \sphinxcode{\sphinxupquote{.pipe()}} method allows us to think “data first”.
Here \sphinxcode{\sphinxupquote{mkt\_m.pipe(plot\_compare\_returns, freq='Monthly')}} is the same as \sphinxcode{\sphinxupquote{plot\_compare\_returns(mkt\_m, freq='Monthly')}}.
One is not better than the other, but I find it useful to think “data, then operation, then operation, …”, instead of “operation, then operation, …, then data”.

\begin{sphinxuseclass}{cell}\begin{sphinxVerbatimInput}

\begin{sphinxuseclass}{cell_input}
\begin{sphinxVerbatim}[commandchars=\\\{\}]
\PYG{n}{mkt\PYGZus{}m}\PYG{o}{.}\PYG{n}{pipe}\PYG{p}{(}\PYG{n}{plot\PYGZus{}compare\PYGZus{}returns}\PYG{p}{,} \PYG{n}{freq}\PYG{o}{=}\PYG{l+s+s1}{\PYGZsq{}}\PYG{l+s+s1}{Monthly}\PYG{l+s+s1}{\PYGZsq{}}\PYG{p}{)}
\end{sphinxVerbatim}

\end{sphinxuseclass}\end{sphinxVerbatimInput}
\begin{sphinxVerbatimOutput}

\begin{sphinxuseclass}{cell_output}
\noindent\sphinxincludegraphics{{9da6a387ba9225585f23ec07e7688582194b74a315970a52fe7835f195a9cae0}.png}

\end{sphinxuseclass}\end{sphinxVerbatimOutput}

\end{sphinxuseclass}
\sphinxAtStartPar
At annual frequencies, log and simple returns can be quite different!
For example, at \$R\_\{simple\} \textbackslash{}approx 0.6\$, we see \$R\_\{log\} \textbackslash{}approx 0.4\$.

\begin{sphinxuseclass}{cell}\begin{sphinxVerbatimInput}

\begin{sphinxuseclass}{cell_input}
\begin{sphinxVerbatim}[commandchars=\\\{\}]
\PYG{n}{mkt\PYGZus{}a}\PYG{o}{.}\PYG{n}{pipe}\PYG{p}{(}\PYG{n}{plot\PYGZus{}compare\PYGZus{}returns}\PYG{p}{,} \PYG{n}{freq}\PYG{o}{=}\PYG{l+s+s1}{\PYGZsq{}}\PYG{l+s+s1}{Annual}\PYG{l+s+s1}{\PYGZsq{}}\PYG{p}{)}
\end{sphinxVerbatim}

\end{sphinxuseclass}\end{sphinxVerbatimInput}
\begin{sphinxVerbatimOutput}

\begin{sphinxuseclass}{cell_output}
\noindent\sphinxincludegraphics{{91cdd1a6918dd2338d6e0bca41671f9f1ec0bb9d5d3f929ce8578b058048cf54}.png}

\end{sphinxuseclass}\end{sphinxVerbatimOutput}

\end{sphinxuseclass}

\subsubsection{Repeat the autocorrelation plot with \sphinxstyleemphasis{monthly} returns instead of \sphinxstyleemphasis{daily} returns}
\label{\detokenize{herron_07_practice_04:repeat-the-autocorrelation-plot-with-monthly-returns-instead-of-daily-returns}}
\sphinxAtStartPar
Again, we will write a function to plot these autocorrelations.


\subsubsection{Plot monthly returns versus their one\sphinxhyphen{}month lag}
\label{\detokenize{herron_07_practice_04:plot-monthly-returns-versus-their-one-month-lag}}
\begin{sphinxuseclass}{cell}\begin{sphinxVerbatimInput}

\begin{sphinxuseclass}{cell_input}
\begin{sphinxVerbatim}[commandchars=\\\{\}]
\PYG{k}{def} \PYG{n+nf}{plot\PYGZus{}autocorr}\PYG{p}{(}\PYG{n}{x}\PYG{p}{,} \PYG{n}{freq}\PYG{p}{,} \PYG{n}{n}\PYG{o}{=}\PYG{l+m+mi}{10}\PYG{p}{,} \PYG{n}{labs}\PYG{o}{=}\PYG{n}{labels}\PYG{p}{)}\PYG{p}{:}
    \PYG{n}{x\PYGZus{}lags} \PYG{o}{=} \PYG{n}{pd}\PYG{o}{.}\PYG{n}{concat}\PYG{p}{(}\PYG{n}{objs}\PYG{o}{=}\PYG{p}{[}\PYG{n}{x}\PYG{p}{[}\PYG{l+s+s1}{\PYGZsq{}}\PYG{l+s+s1}{R}\PYG{l+s+s1}{\PYGZsq{}}\PYG{p}{]}\PYG{o}{.}\PYG{n}{shift}\PYG{p}{(}\PYG{n}{t}\PYG{p}{)} \PYG{k}{for} \PYG{n}{t} \PYG{o+ow}{in} \PYG{n+nb}{range}\PYG{p}{(}\PYG{n}{n} \PYG{o}{+} \PYG{l+m+mi}{1}\PYG{p}{)}\PYG{p}{]}\PYG{p}{,} \PYG{n}{axis}\PYG{o}{=}\PYG{l+m+mi}{1}\PYG{p}{)}
    \PYG{n}{corrs} \PYG{o}{=} \PYG{n}{x\PYGZus{}lags}\PYG{o}{.}\PYG{n}{corr}\PYG{p}{(}\PYG{p}{)}\PYG{o}{.}\PYG{n}{iloc}\PYG{p}{[}\PYG{l+m+mi}{0}\PYG{p}{]}
    \PYG{n}{serrs} \PYG{o}{=} \PYG{n}{np}\PYG{o}{.}\PYG{n}{sqrt}\PYG{p}{(}\PYG{p}{(}\PYG{l+m+mi}{1} \PYG{o}{\PYGZhy{}} \PYG{n}{corrs}\PYG{o}{*}\PYG{o}{*}\PYG{l+m+mi}{2}\PYG{p}{)} \PYG{o}{/} \PYG{p}{(}\PYG{n}{x\PYGZus{}lags}\PYG{o}{.}\PYG{n}{count}\PYG{p}{(}\PYG{p}{)} \PYG{o}{\PYGZhy{}} \PYG{l+m+mi}{2}\PYG{p}{)}\PYG{p}{)}

    \PYG{n}{plt}\PYG{o}{.}\PYG{n}{bar}\PYG{p}{(}\PYG{n}{height}\PYG{o}{=}\PYG{n}{corrs}\PYG{p}{,} \PYG{n}{x}\PYG{o}{=}\PYG{n+nb}{range}\PYG{p}{(}\PYG{n}{n} \PYG{o}{+} \PYG{l+m+mi}{1}\PYG{p}{)}\PYG{p}{,} \PYG{n}{yerr}\PYG{o}{=}\PYG{l+m+mi}{2}\PYG{o}{*}\PYG{n}{serrs}\PYG{p}{)}
    \PYG{n}{plt}\PYG{o}{.}\PYG{n}{title}\PYG{p}{(}
        \PYG{l+s+sa}{f}\PYG{l+s+s1}{\PYGZsq{}}\PYG{l+s+s1}{Autocorrelation of }\PYG{l+s+si}{\PYGZob{}}\PYG{n}{freq}\PYG{l+s+si}{\PYGZcb{}}\PYG{l+s+s1}{ Returns}\PYG{l+s+s1}{\PYGZsq{}} \PYG{o}{+}
        \PYG{l+s+s1}{\PYGZsq{}}\PYG{l+s+se}{\PYGZbs{}n}\PYG{l+s+s1}{Black Vertical Bars Indicate Standard Errors}\PYG{l+s+s1}{\PYGZsq{}}
    \PYG{p}{)}
    \PYG{n}{plt}\PYG{o}{.}\PYG{n}{xlabel}\PYG{p}{(}\PYG{l+s+sa}{f}\PYG{l+s+s1}{\PYGZsq{}}\PYG{l+s+si}{\PYGZob{}}\PYG{n}{freq}\PYG{l+s+si}{\PYGZcb{}}\PYG{l+s+s1}{ Lag}\PYG{l+s+s1}{\PYGZsq{}}\PYG{p}{)}
    \PYG{n}{plt}\PYG{o}{.}\PYG{n}{ylabel}\PYG{p}{(}\PYG{l+s+s1}{\PYGZsq{}}\PYG{l+s+s1}{Autocorrelation Coefficient}\PYG{l+s+s1}{\PYGZsq{}}\PYG{p}{)}
    \PYG{n}{plt}\PYG{o}{.}\PYG{n}{show}\PYG{p}{(}\PYG{p}{)}
\end{sphinxVerbatim}

\end{sphinxuseclass}\end{sphinxVerbatimInput}

\end{sphinxuseclass}
\sphinxAtStartPar
At the monthly horizon, return autocorrelation weakens!
All autocorrelation coefficient estimates are small and less than two standard errors away from zero.

\begin{sphinxuseclass}{cell}\begin{sphinxVerbatimInput}

\begin{sphinxuseclass}{cell_input}
\begin{sphinxVerbatim}[commandchars=\\\{\}]
\PYG{n}{mkt\PYGZus{}m}\PYG{o}{.}\PYG{n}{pipe}\PYG{p}{(}\PYG{n}{plot\PYGZus{}autocorr}\PYG{p}{,} \PYG{n}{freq}\PYG{o}{=}\PYG{l+s+s1}{\PYGZsq{}}\PYG{l+s+s1}{Monthly}\PYG{l+s+s1}{\PYGZsq{}}\PYG{p}{)}
\end{sphinxVerbatim}

\end{sphinxuseclass}\end{sphinxVerbatimInput}
\begin{sphinxVerbatimOutput}

\begin{sphinxuseclass}{cell_output}
\noindent\sphinxincludegraphics{{e50eaaa95743b0cfbb3f396f86d62395449865d5d37837b808ec8883ec31cd3f}.png}

\end{sphinxuseclass}\end{sphinxVerbatimOutput}

\end{sphinxuseclass}
\sphinxAtStartPar
At the annual horizon, return autocorrelation disappears!

\begin{sphinxuseclass}{cell}\begin{sphinxVerbatimInput}

\begin{sphinxuseclass}{cell_input}
\begin{sphinxVerbatim}[commandchars=\\\{\}]
\PYG{n}{mkt\PYGZus{}a}\PYG{o}{.}\PYG{n}{pipe}\PYG{p}{(}\PYG{n}{plot\PYGZus{}autocorr}\PYG{p}{,} \PYG{n}{freq}\PYG{o}{=}\PYG{l+s+s1}{\PYGZsq{}}\PYG{l+s+s1}{Annual}\PYG{l+s+s1}{\PYGZsq{}}\PYG{p}{)}
\end{sphinxVerbatim}

\end{sphinxuseclass}\end{sphinxVerbatimInput}
\begin{sphinxVerbatimOutput}

\begin{sphinxuseclass}{cell_output}
\noindent\sphinxincludegraphics{{464841c0b36ed50158550c2c63776c49d41933e9b37ec4d568cdf332d6b0ed28}.png}

\end{sphinxuseclass}\end{sphinxVerbatimOutput}

\end{sphinxuseclass}

\subsubsection{Repeat the autocorrelation plot with the \sphinxstyleemphasis{absolute value} of daily returns}
\label{\detokenize{herron_07_practice_04:repeat-the-autocorrelation-plot-with-the-absolute-value-of-daily-returns}}
\sphinxAtStartPar
We can use our function again!
But first, we have to convert simple returns to the \sphinxstyleemphasis{absolute value} of simple returns.

\begin{sphinxuseclass}{cell}\begin{sphinxVerbatimInput}

\begin{sphinxuseclass}{cell_input}
\begin{sphinxVerbatim}[commandchars=\\\{\}]
\PYG{p}{(}
    \PYG{n}{mkt}
    \PYG{o}{.}\PYG{n}{assign}\PYG{p}{(}\PYG{n}{R}\PYG{o}{=}\PYG{k}{lambda} \PYG{n}{x}\PYG{p}{:} \PYG{n}{x}\PYG{p}{[}\PYG{l+s+s1}{\PYGZsq{}}\PYG{l+s+s1}{R}\PYG{l+s+s1}{\PYGZsq{}}\PYG{p}{]}\PYG{o}{.}\PYG{n}{abs}\PYG{p}{(}\PYG{p}{)}\PYG{p}{)}
    \PYG{o}{.}\PYG{n}{pipe}\PYG{p}{(}\PYG{n}{plot\PYGZus{}autocorr}\PYG{p}{,} \PYG{n}{freq}\PYG{o}{=}\PYG{l+s+s1}{\PYGZsq{}}\PYG{l+s+s1}{Absolute Value of Daily}\PYG{l+s+s1}{\PYGZsq{}}\PYG{p}{,} \PYG{n}{n}\PYG{o}{=}\PYG{l+m+mi}{60}\PYG{p}{)}
\PYG{p}{)}
\end{sphinxVerbatim}

\end{sphinxuseclass}\end{sphinxVerbatimInput}
\begin{sphinxVerbatimOutput}

\begin{sphinxuseclass}{cell_output}
\noindent\sphinxincludegraphics{{cc8f241a78be63b790360cc93abd6d31ac40c11d0865e6e2dd7c866cd2d56c71}.png}

\end{sphinxuseclass}\end{sphinxVerbatimOutput}

\end{sphinxuseclass}
\sphinxAtStartPar
Saying that “squared returns” or “the absolute value of returns” is autocorrelated is similar to saying “return magnitudes” are autocorrelated, which is similar to saying “volatility” is autocorrelated or persistent.


\subsubsection{Repeat the leverage effect plot with lagged returns}
\label{\detokenize{herron_07_practice_04:repeat-the-leverage-effect-plot-with-lagged-returns}}
\sphinxAtStartPar
We will only make this plot, but we can still write a function!

\begin{sphinxuseclass}{cell}\begin{sphinxVerbatimInput}

\begin{sphinxuseclass}{cell_input}
\begin{sphinxVerbatim}[commandchars=\\\{\}]
\PYG{k+kn}{import} \PYG{n+nn}{seaborn} \PYG{k}{as} \PYG{n+nn}{sns}
\end{sphinxVerbatim}

\end{sphinxuseclass}\end{sphinxVerbatimInput}

\end{sphinxuseclass}
\begin{sphinxuseclass}{cell}\begin{sphinxVerbatimInput}

\begin{sphinxuseclass}{cell_input}
\begin{sphinxVerbatim}[commandchars=\\\{\}]
\PYG{k}{def} \PYG{n+nf}{plot\PYGZus{}leverage\PYGZus{}effect}\PYG{p}{(}\PYG{n}{x}\PYG{p}{,} \PYG{n}{rule}\PYG{p}{,} \PYG{n}{freq}\PYG{p}{)}\PYG{p}{:}
    \PYG{p}{(}
        \PYG{n}{x}\PYG{p}{[}\PYG{l+s+s1}{\PYGZsq{}}\PYG{l+s+s1}{R}\PYG{l+s+s1}{\PYGZsq{}}\PYG{p}{]}
        \PYG{o}{.}\PYG{n}{resample}\PYG{p}{(}\PYG{n}{rule}\PYG{o}{=}\PYG{n}{rule}\PYG{p}{,} \PYG{n}{kind}\PYG{o}{=}\PYG{l+s+s1}{\PYGZsq{}}\PYG{l+s+s1}{period}\PYG{l+s+s1}{\PYGZsq{}}\PYG{p}{)}
        \PYG{o}{.}\PYG{n}{agg}\PYG{p}{(}\PYG{p}{[}\PYG{n}{totret}\PYG{p}{,} \PYG{l+s+s1}{\PYGZsq{}}\PYG{l+s+s1}{std}\PYG{l+s+s1}{\PYGZsq{}}\PYG{p}{]}\PYG{p}{)}
        \PYG{o}{.}\PYG{n}{assign}\PYG{p}{(}\PYG{n}{totret\PYGZus{}lag}\PYG{o}{=}\PYG{k}{lambda} \PYG{n}{x}\PYG{p}{:} \PYG{n}{x}\PYG{p}{[}\PYG{l+s+s1}{\PYGZsq{}}\PYG{l+s+s1}{totret}\PYG{l+s+s1}{\PYGZsq{}}\PYG{p}{]}\PYG{o}{.}\PYG{n}{shift}\PYG{p}{(}\PYG{p}{)}\PYG{p}{)}
        \PYG{o}{.}\PYG{n}{mul}\PYG{p}{(}\PYG{l+m+mi}{100}\PYG{p}{)}
        \PYG{o}{.}\PYG{n}{pipe}\PYG{p}{(}\PYG{k}{lambda} \PYG{n}{x}\PYG{p}{:}
              \PYG{n}{sns}\PYG{o}{.}\PYG{n}{regplot}\PYG{p}{(}
                  \PYG{n}{x}\PYG{o}{=}\PYG{l+s+s1}{\PYGZsq{}}\PYG{l+s+s1}{totret\PYGZus{}lag}\PYG{l+s+s1}{\PYGZsq{}}\PYG{p}{,} 
                  \PYG{n}{y}\PYG{o}{=}\PYG{l+s+s1}{\PYGZsq{}}\PYG{l+s+s1}{std}\PYG{l+s+s1}{\PYGZsq{}}\PYG{p}{,} 
                  \PYG{n}{data}\PYG{o}{=}\PYG{n}{x}\PYG{p}{,}
                  \PYG{n}{scatter\PYGZus{}kws}\PYG{o}{=}\PYG{p}{\PYGZob{}}\PYG{l+s+s1}{\PYGZsq{}}\PYG{l+s+s1}{alpha}\PYG{l+s+s1}{\PYGZsq{}}\PYG{p}{:} \PYG{l+m+mf}{0.1}\PYG{p}{\PYGZcb{}}
            \PYG{p}{)}
        \PYG{p}{)}
    \PYG{p}{)}
    \PYG{n}{plt}\PYG{o}{.}\PYG{n}{xlabel}\PYG{p}{(}\PYG{l+s+sa}{f}\PYG{l+s+s1}{\PYGZsq{}}\PYG{l+s+s1}{One\PYGZhy{}}\PYG{l+s+si}{\PYGZob{}}\PYG{n}{freq}\PYG{l+s+si}{\PYGZcb{}}\PYG{l+s+s1}{ Lag of Total Return (\PYGZpc{})}\PYG{l+s+s1}{\PYGZsq{}}\PYG{p}{)}
    \PYG{n}{plt}\PYG{o}{.}\PYG{n}{ylabel}\PYG{p}{(}\PYG{l+s+s1}{\PYGZsq{}}\PYG{l+s+s1}{Volatility of Returns (}\PYG{l+s+s1}{\PYGZpc{}}\PYG{l+s+s1}{)}\PYG{l+s+s1}{\PYGZsq{}}\PYG{p}{)}
    \PYG{n}{plt}\PYG{o}{.}\PYG{n}{suptitle}\PYG{p}{(}
        \PYG{l+s+s1}{\PYGZsq{}}\PYG{l+s+s1}{Fisher Black}\PYG{l+s+se}{\PYGZbs{}\PYGZsq{}}\PYG{l+s+s1}{s Leverage Effect}\PYG{l+s+se}{\PYGZbs{}n}\PYG{l+s+s1}{\PYGZsq{}} \PYG{o}{+} 
        \PYG{l+s+sa}{f}\PYG{l+s+s1}{\PYGZsq{}}\PYG{l+s+s1}{(Daily Returns Aggregated by }\PYG{l+s+si}{\PYGZob{}}\PYG{n}{freq}\PYG{l+s+si}{\PYGZcb{}}\PYG{l+s+s1}{)}\PYG{l+s+s1}{\PYGZsq{}}
    \PYG{p}{)}
    \PYG{n}{plt}\PYG{o}{.}\PYG{n}{show}\PYG{p}{(}\PYG{p}{)}
\end{sphinxVerbatim}

\end{sphinxuseclass}\end{sphinxVerbatimInput}

\end{sphinxuseclass}
\begin{sphinxuseclass}{cell}\begin{sphinxVerbatimInput}

\begin{sphinxuseclass}{cell_input}
\begin{sphinxVerbatim}[commandchars=\\\{\}]
\PYG{n}{mkt}\PYG{o}{.}\PYG{n}{pipe}\PYG{p}{(}\PYG{n}{plot\PYGZus{}leverage\PYGZus{}effect}\PYG{p}{,} \PYG{n}{rule}\PYG{o}{=}\PYG{l+s+s1}{\PYGZsq{}}\PYG{l+s+s1}{M}\PYG{l+s+s1}{\PYGZsq{}}\PYG{p}{,} \PYG{n}{freq}\PYG{o}{=}\PYG{l+s+s1}{\PYGZsq{}}\PYG{l+s+s1}{Month}\PYG{l+s+s1}{\PYGZsq{}}\PYG{p}{)}
\end{sphinxVerbatim}

\end{sphinxuseclass}\end{sphinxVerbatimInput}
\begin{sphinxVerbatimOutput}

\begin{sphinxuseclass}{cell_output}
\noindent\sphinxincludegraphics{{dc5d4ab1c17eac3a39783c8593ac95e7e21f3fef4a1db544511381b01cb0aac5}.png}

\end{sphinxuseclass}\end{sphinxVerbatimOutput}

\end{sphinxuseclass}
\sphinxAtStartPar
Since we have the function, let use a try a few different horizons!

\begin{sphinxuseclass}{cell}\begin{sphinxVerbatimInput}

\begin{sphinxuseclass}{cell_input}
\begin{sphinxVerbatim}[commandchars=\\\{\}]
\PYG{n}{mkt}\PYG{o}{.}\PYG{n}{pipe}\PYG{p}{(}\PYG{n}{plot\PYGZus{}leverage\PYGZus{}effect}\PYG{p}{,} \PYG{n}{rule}\PYG{o}{=}\PYG{l+s+s1}{\PYGZsq{}}\PYG{l+s+s1}{Q}\PYG{l+s+s1}{\PYGZsq{}}\PYG{p}{,} \PYG{n}{freq}\PYG{o}{=}\PYG{l+s+s1}{\PYGZsq{}}\PYG{l+s+s1}{Quarter}\PYG{l+s+s1}{\PYGZsq{}}\PYG{p}{)}
\end{sphinxVerbatim}

\end{sphinxuseclass}\end{sphinxVerbatimInput}
\begin{sphinxVerbatimOutput}

\begin{sphinxuseclass}{cell_output}
\noindent\sphinxincludegraphics{{ed8fcea3fa622d1091b742e5696426cd6e1126d699f885265956cd9275417d2f}.png}

\end{sphinxuseclass}\end{sphinxVerbatimOutput}

\end{sphinxuseclass}
\begin{sphinxuseclass}{cell}\begin{sphinxVerbatimInput}

\begin{sphinxuseclass}{cell_input}
\begin{sphinxVerbatim}[commandchars=\\\{\}]
\PYG{n}{mkt}\PYG{o}{.}\PYG{n}{pipe}\PYG{p}{(}\PYG{n}{plot\PYGZus{}leverage\PYGZus{}effect}\PYG{p}{,} \PYG{n}{rule}\PYG{o}{=}\PYG{l+s+s1}{\PYGZsq{}}\PYG{l+s+s1}{A}\PYG{l+s+s1}{\PYGZsq{}}\PYG{p}{,} \PYG{n}{freq}\PYG{o}{=}\PYG{l+s+s1}{\PYGZsq{}}\PYG{l+s+s1}{Annual}\PYG{l+s+s1}{\PYGZsq{}}\PYG{p}{)}
\end{sphinxVerbatim}

\end{sphinxuseclass}\end{sphinxVerbatimInput}
\begin{sphinxVerbatimOutput}

\begin{sphinxuseclass}{cell_output}
\noindent\sphinxincludegraphics{{54ee36ce8bcd7250990a5d43a889d720ffe64dc13e9132e142b04b10d0be6274}.png}

\end{sphinxuseclass}\end{sphinxVerbatimOutput}

\end{sphinxuseclass}
\sphinxstepscope


\section{Herron Topic 7 \sphinxhyphen{} Practice (Wednesday 2:45 PM, Section 2)}
\label{\detokenize{herron_07_practice_02:herron-topic-7-practice-wednesday-2-45-pm-section-2}}\label{\detokenize{herron_07_practice_02::doc}}

\subsection{Announcements}
\label{\detokenize{herron_07_practice_02:announcements}}\begin{itemize}
\item {} 
\sphinxAtStartPar
Quiz 7 (our last quiz!) is due Friday, 4/14, at 11:59 PM

\item {} 
\sphinxAtStartPar
Week of 4/17
\begin{itemize}
\item {} 
\sphinxAtStartPar
I will record a lecture for Herron topic 5 on simulations and post a practice notebook

\item {} 
\sphinxAtStartPar
However, we will use class time for group work

\item {} 
\sphinxAtStartPar
This is a compromise between survey responses (1/3 wanted to drop simulations, 2/3 did not) and Monday’s holiday (Patriot’s Day)

\end{itemize}

\item {} 
\sphinxAtStartPar
Week of 4/24
\begin{itemize}
\item {} 
\sphinxAtStartPar
No lecture

\item {} 
\sphinxAtStartPar
We will use class time for group work

\end{itemize}

\item {} 
\sphinxAtStartPar
Due dates
\begin{itemize}
\item {} 
\sphinxAtStartPar
Project 2 at 11:59 PM on Wednesday, 4/26

\item {} 
\sphinxAtStartPar
Teammates review 2 at 11:59 PM on Wednesday, 4/26

\item {} 
\sphinxAtStartPar
30,000 DataCamp XP at 11:59 PM on Friday, 4/28

\end{itemize}

\item {} 
\sphinxAtStartPar
\sphinxstyleemphasis{\sphinxstylestrong{Please complete TRACE feedback on what topics, tools, and techniques you want more of}}

\end{itemize}


\subsection{Practice}
\label{\detokenize{herron_07_practice_02:practice}}
\begin{sphinxuseclass}{cell}\begin{sphinxVerbatimInput}

\begin{sphinxuseclass}{cell_input}
\begin{sphinxVerbatim}[commandchars=\\\{\}]
\PYG{k+kn}{import} \PYG{n+nn}{matplotlib}\PYG{n+nn}{.}\PYG{n+nn}{pyplot} \PYG{k}{as} \PYG{n+nn}{plt}
\PYG{k+kn}{import} \PYG{n+nn}{numpy} \PYG{k}{as} \PYG{n+nn}{np}
\PYG{k+kn}{import} \PYG{n+nn}{pandas} \PYG{k}{as} \PYG{n+nn}{pd}
\end{sphinxVerbatim}

\end{sphinxuseclass}\end{sphinxVerbatimInput}

\end{sphinxuseclass}
\begin{sphinxuseclass}{cell}\begin{sphinxVerbatimInput}

\begin{sphinxuseclass}{cell_input}
\begin{sphinxVerbatim}[commandchars=\\\{\}]
\PYG{o}{\PYGZpc{}}\PYG{k}{config} InlineBackend.figure\PYGZus{}format = \PYGZsq{}retina\PYGZsq{}
\PYG{o}{\PYGZpc{}}\PYG{k}{precision} 4
\PYG{n}{pd}\PYG{o}{.}\PYG{n}{options}\PYG{o}{.}\PYG{n}{display}\PYG{o}{.}\PYG{n}{float\PYGZus{}format} \PYG{o}{=} \PYG{l+s+s1}{\PYGZsq{}}\PYG{l+s+si}{\PYGZob{}:.4f\PYGZcb{}}\PYG{l+s+s1}{\PYGZsq{}}\PYG{o}{.}\PYG{n}{format}
\end{sphinxVerbatim}

\end{sphinxuseclass}\end{sphinxVerbatimInput}

\end{sphinxuseclass}
\begin{sphinxuseclass}{cell}\begin{sphinxVerbatimInput}

\begin{sphinxuseclass}{cell_input}
\begin{sphinxVerbatim}[commandchars=\\\{\}]
\PYG{k+kn}{import} \PYG{n+nn}{requests\PYGZus{}cache}
\PYG{n}{session} \PYG{o}{=} \PYG{n}{requests\PYGZus{}cache}\PYG{o}{.}\PYG{n}{CachedSession}\PYG{p}{(}\PYG{p}{)}
\PYG{k+kn}{import} \PYG{n+nn}{yfinance} \PYG{k}{as} \PYG{n+nn}{yf}
\PYG{k+kn}{import} \PYG{n+nn}{pandas\PYGZus{}datareader} \PYG{k}{as} \PYG{n+nn}{pdr}
\end{sphinxVerbatim}

\end{sphinxuseclass}\end{sphinxVerbatimInput}

\end{sphinxuseclass}
\begin{sphinxuseclass}{cell}\begin{sphinxVerbatimInput}

\begin{sphinxuseclass}{cell_input}
\begin{sphinxVerbatim}[commandchars=\\\{\}]
\PYG{n}{mkt} \PYG{o}{=} \PYG{p}{(}
    \PYG{n}{pdr}\PYG{o}{.}\PYG{n}{DataReader}\PYG{p}{(}
        \PYG{n}{name}\PYG{o}{=}\PYG{l+s+s1}{\PYGZsq{}}\PYG{l+s+s1}{F\PYGZhy{}F\PYGZus{}Research\PYGZus{}Data\PYGZus{}Factors\PYGZus{}daily}\PYG{l+s+s1}{\PYGZsq{}}\PYG{p}{,}
        \PYG{n}{data\PYGZus{}source}\PYG{o}{=}\PYG{l+s+s1}{\PYGZsq{}}\PYG{l+s+s1}{famafrench}\PYG{l+s+s1}{\PYGZsq{}}\PYG{p}{,}
        \PYG{n}{start}\PYG{o}{=}\PYG{l+s+s1}{\PYGZsq{}}\PYG{l+s+s1}{1900}\PYG{l+s+s1}{\PYGZsq{}}\PYG{p}{,}
        \PYG{n}{session}\PYG{o}{=}\PYG{n}{session}
    \PYG{p}{)}
    \PYG{p}{[}\PYG{l+m+mi}{0}\PYG{p}{]}
    \PYG{o}{.}\PYG{n}{assign}\PYG{p}{(}
        \PYG{n}{R}\PYG{o}{=}\PYG{k}{lambda} \PYG{n}{x}\PYG{p}{:} \PYG{p}{(}\PYG{n}{x}\PYG{p}{[}\PYG{l+s+s1}{\PYGZsq{}}\PYG{l+s+s1}{Mkt\PYGZhy{}RF}\PYG{l+s+s1}{\PYGZsq{}}\PYG{p}{]} \PYG{o}{+} \PYG{n}{x}\PYG{p}{[}\PYG{l+s+s1}{\PYGZsq{}}\PYG{l+s+s1}{RF}\PYG{l+s+s1}{\PYGZsq{}}\PYG{p}{]}\PYG{p}{)} \PYG{o}{/} \PYG{l+m+mi}{100}\PYG{p}{,}
        \PYG{n}{logR}\PYG{o}{=}\PYG{k}{lambda} \PYG{n}{x}\PYG{p}{:} \PYG{n}{np}\PYG{o}{.}\PYG{n}{log1p}\PYG{p}{(}\PYG{n}{x}\PYG{p}{[}\PYG{l+s+s1}{\PYGZsq{}}\PYG{l+s+s1}{R}\PYG{l+s+s1}{\PYGZsq{}}\PYG{p}{]}\PYG{p}{)}
    \PYG{p}{)}
    \PYG{p}{[}\PYG{p}{[}\PYG{l+s+s1}{\PYGZsq{}}\PYG{l+s+s1}{R}\PYG{l+s+s1}{\PYGZsq{}}\PYG{p}{,} \PYG{l+s+s1}{\PYGZsq{}}\PYG{l+s+s1}{logR}\PYG{l+s+s1}{\PYGZsq{}}\PYG{p}{]}\PYG{p}{]}
\PYG{p}{)}
\end{sphinxVerbatim}

\end{sphinxuseclass}\end{sphinxVerbatimInput}

\end{sphinxuseclass}
\begin{sphinxuseclass}{cell}\begin{sphinxVerbatimInput}

\begin{sphinxuseclass}{cell_input}
\begin{sphinxVerbatim}[commandchars=\\\{\}]
\PYG{n}{labels} \PYG{o}{=} \PYG{p}{\PYGZob{}}
    \PYG{l+s+s1}{\PYGZsq{}}\PYG{l+s+s1}{R}\PYG{l+s+s1}{\PYGZsq{}}\PYG{p}{:} \PYG{l+s+s1}{\PYGZsq{}}\PYG{l+s+s1}{Simple Return}\PYG{l+s+s1}{\PYGZsq{}}\PYG{p}{,}
    \PYG{l+s+s1}{\PYGZsq{}}\PYG{l+s+s1}{logR}\PYG{l+s+s1}{\PYGZsq{}}\PYG{p}{:} \PYG{l+s+s1}{\PYGZsq{}}\PYG{l+s+s1}{Log Return}\PYG{l+s+s1}{\PYGZsq{}}
\PYG{p}{\PYGZcb{}}
\end{sphinxVerbatim}

\end{sphinxuseclass}\end{sphinxVerbatimInput}

\end{sphinxuseclass}

\subsubsection{Repeat the comparison of log and simple returns with monthly and annual returns}
\label{\detokenize{herron_07_practice_02:repeat-the-comparison-of-log-and-simple-returns-with-monthly-and-annual-returns}}
\sphinxAtStartPar
We could download monthly and annual data from the Ken French data library.
However, we can also use the \sphinxcode{\sphinxupquote{.resample()}} method to compound daily returns into monthly and annual returns.
Let us use the \sphinxcode{\sphinxupquote{.resample()}} method (1) for practice and (2) because the \sphinxcode{\sphinxupquote{.resample()}} code is simpler and more compact than the \sphinxcode{\sphinxupquote{pdr.DataReader()}} code.
We need to re\sphinxhyphen{}calculate the log return.
We could instead sum the daily log returns, but combining the \sphinxcode{\sphinxupquote{.resample()}} method with two different operations on two differ columns is complex and easily avoided here.

\begin{sphinxuseclass}{cell}\begin{sphinxVerbatimInput}

\begin{sphinxuseclass}{cell_input}
\begin{sphinxVerbatim}[commandchars=\\\{\}]
\PYG{k}{def} \PYG{n+nf}{totret}\PYG{p}{(}\PYG{n}{x}\PYG{p}{)}\PYG{p}{:}
    \PYG{k}{return} \PYG{p}{(}\PYG{l+m+mi}{1} \PYG{o}{+} \PYG{n}{x}\PYG{p}{)}\PYG{o}{.}\PYG{n}{prod}\PYG{p}{(}\PYG{p}{)} \PYG{o}{\PYGZhy{}} \PYG{l+m+mi}{1}
\end{sphinxVerbatim}

\end{sphinxuseclass}\end{sphinxVerbatimInput}

\end{sphinxuseclass}
\begin{sphinxuseclass}{cell}\begin{sphinxVerbatimInput}

\begin{sphinxuseclass}{cell_input}
\begin{sphinxVerbatim}[commandchars=\\\{\}]
\PYG{n}{mkt\PYGZus{}m} \PYG{o}{=} \PYG{p}{(}
    \PYG{n}{mkt}\PYG{p}{[}\PYG{p}{[}\PYG{l+s+s1}{\PYGZsq{}}\PYG{l+s+s1}{R}\PYG{l+s+s1}{\PYGZsq{}}\PYG{p}{]}\PYG{p}{]}
    \PYG{o}{.}\PYG{n}{resample}\PYG{p}{(}\PYG{n}{rule}\PYG{o}{=}\PYG{l+s+s1}{\PYGZsq{}}\PYG{l+s+s1}{M}\PYG{l+s+s1}{\PYGZsq{}}\PYG{p}{,} \PYG{n}{kind}\PYG{o}{=}\PYG{l+s+s1}{\PYGZsq{}}\PYG{l+s+s1}{period}\PYG{l+s+s1}{\PYGZsq{}}\PYG{p}{)}
    \PYG{o}{.}\PYG{n}{apply}\PYG{p}{(}\PYG{n}{totret}\PYG{p}{)}
    \PYG{o}{.}\PYG{n}{assign}\PYG{p}{(}\PYG{n}{logR}\PYG{o}{=}\PYG{k}{lambda} \PYG{n}{x}\PYG{p}{:} \PYG{n}{np}\PYG{o}{.}\PYG{n}{log1p}\PYG{p}{(}\PYG{n}{x}\PYG{p}{[}\PYG{l+s+s1}{\PYGZsq{}}\PYG{l+s+s1}{R}\PYG{l+s+s1}{\PYGZsq{}}\PYG{p}{]}\PYG{p}{)}\PYG{p}{)}
\PYG{p}{)}
\end{sphinxVerbatim}

\end{sphinxuseclass}\end{sphinxVerbatimInput}

\end{sphinxuseclass}
\begin{sphinxuseclass}{cell}\begin{sphinxVerbatimInput}

\begin{sphinxuseclass}{cell_input}
\begin{sphinxVerbatim}[commandchars=\\\{\}]
\PYG{n}{mkt\PYGZus{}a} \PYG{o}{=} \PYG{p}{(}
    \PYG{n}{mkt}\PYG{p}{[}\PYG{p}{[}\PYG{l+s+s1}{\PYGZsq{}}\PYG{l+s+s1}{R}\PYG{l+s+s1}{\PYGZsq{}}\PYG{p}{]}\PYG{p}{]}
    \PYG{o}{.}\PYG{n}{resample}\PYG{p}{(}\PYG{n}{rule}\PYG{o}{=}\PYG{l+s+s1}{\PYGZsq{}}\PYG{l+s+s1}{A}\PYG{l+s+s1}{\PYGZsq{}}\PYG{p}{,} \PYG{n}{kind}\PYG{o}{=}\PYG{l+s+s1}{\PYGZsq{}}\PYG{l+s+s1}{period}\PYG{l+s+s1}{\PYGZsq{}}\PYG{p}{)}
    \PYG{o}{.}\PYG{n}{apply}\PYG{p}{(}\PYG{n}{totret}\PYG{p}{)}
    \PYG{o}{.}\PYG{n}{assign}\PYG{p}{(}\PYG{n}{logR}\PYG{o}{=}\PYG{k}{lambda} \PYG{n}{x}\PYG{p}{:} \PYG{n}{np}\PYG{o}{.}\PYG{n}{log1p}\PYG{p}{(}\PYG{n}{x}\PYG{p}{[}\PYG{l+s+s1}{\PYGZsq{}}\PYG{l+s+s1}{R}\PYG{l+s+s1}{\PYGZsq{}}\PYG{p}{]}\PYG{p}{)}\PYG{p}{)}
\PYG{p}{)}
\end{sphinxVerbatim}

\end{sphinxuseclass}\end{sphinxVerbatimInput}

\end{sphinxuseclass}
\sphinxAtStartPar
We want to repeat this plot for monthly and annual returns, so we can make out lives easier by writing a function.

\begin{sphinxuseclass}{cell}\begin{sphinxVerbatimInput}

\begin{sphinxuseclass}{cell_input}
\begin{sphinxVerbatim}[commandchars=\\\{\}]
\PYG{k}{def} \PYG{n+nf}{plot\PYGZus{}compare\PYGZus{}returns}\PYG{p}{(}\PYG{n}{x}\PYG{p}{,} \PYG{n}{freq}\PYG{p}{,} \PYG{n}{labs}\PYG{o}{=}\PYG{n}{labels}\PYG{p}{)}\PYG{p}{:}
    \PYG{n}{x}\PYG{o}{.}\PYG{n}{plot}\PYG{p}{(}\PYG{n}{x}\PYG{o}{=}\PYG{l+s+s1}{\PYGZsq{}}\PYG{l+s+s1}{R}\PYG{l+s+s1}{\PYGZsq{}}\PYG{p}{,} \PYG{n}{y}\PYG{o}{=}\PYG{l+s+s1}{\PYGZsq{}}\PYG{l+s+s1}{logR}\PYG{l+s+s1}{\PYGZsq{}}\PYG{p}{,} \PYG{n}{kind}\PYG{o}{=}\PYG{l+s+s1}{\PYGZsq{}}\PYG{l+s+s1}{scatter}\PYG{l+s+s1}{\PYGZsq{}}\PYG{p}{)}
    \PYG{n}{plt}\PYG{o}{.}\PYG{n}{title}\PYG{p}{(}\PYG{l+s+sa}{f}\PYG{l+s+s1}{\PYGZsq{}}\PYG{l+s+s1}{Comparison of }\PYG{l+s+si}{\PYGZob{}}\PYG{n}{freq}\PYG{l+s+si}{\PYGZcb{}}\PYG{l+s+s1}{ Log and Simple Returns}\PYG{l+s+s1}{\PYGZsq{}}\PYG{p}{)}
    \PYG{n}{plt}\PYG{o}{.}\PYG{n}{xlabel}\PYG{p}{(}\PYG{n}{labs}\PYG{p}{[}\PYG{l+s+s1}{\PYGZsq{}}\PYG{l+s+s1}{R}\PYG{l+s+s1}{\PYGZsq{}}\PYG{p}{]}\PYG{p}{)}
    \PYG{n}{plt}\PYG{o}{.}\PYG{n}{ylabel}\PYG{p}{(}\PYG{n}{labs}\PYG{p}{[}\PYG{l+s+s1}{\PYGZsq{}}\PYG{l+s+s1}{logR}\PYG{l+s+s1}{\PYGZsq{}}\PYG{p}{]}\PYG{p}{)}
    \PYG{n}{plt}\PYG{o}{.}\PYG{n}{show}\PYG{p}{(}\PYG{p}{)}
\end{sphinxVerbatim}

\end{sphinxuseclass}\end{sphinxVerbatimInput}

\end{sphinxuseclass}
\sphinxAtStartPar
Log and simple returns are similar for small returns.
We almost always see small daily returns, which rarely exceed 10\% in magnitude.
However, at monthly frequencies, we start to see large outliers, where the similarity of log and simple returns starts to break down.

\sphinxAtStartPar
Recall the \sphinxcode{\sphinxupquote{.pipe()}} method allows us to think “data first”.
Here \sphinxcode{\sphinxupquote{mkt\_m.pipe(plot\_compare\_returns, freq='Monthly')}} is the same as \sphinxcode{\sphinxupquote{plot\_compare\_returns(mkt\_m, freq='Monthly')}}.
One is not better than the other, but I find it useful to think “data, then operation, then operation, …”, instead of “operation, then operation, …, then data”.

\begin{sphinxuseclass}{cell}\begin{sphinxVerbatimInput}

\begin{sphinxuseclass}{cell_input}
\begin{sphinxVerbatim}[commandchars=\\\{\}]
\PYG{n}{mkt\PYGZus{}m}\PYG{o}{.}\PYG{n}{pipe}\PYG{p}{(}\PYG{n}{plot\PYGZus{}compare\PYGZus{}returns}\PYG{p}{,} \PYG{n}{freq}\PYG{o}{=}\PYG{l+s+s1}{\PYGZsq{}}\PYG{l+s+s1}{Monthly}\PYG{l+s+s1}{\PYGZsq{}}\PYG{p}{)}
\end{sphinxVerbatim}

\end{sphinxuseclass}\end{sphinxVerbatimInput}
\begin{sphinxVerbatimOutput}

\begin{sphinxuseclass}{cell_output}
\noindent\sphinxincludegraphics{{9da6a387ba9225585f23ec07e7688582194b74a315970a52fe7835f195a9cae0}.png}

\end{sphinxuseclass}\end{sphinxVerbatimOutput}

\end{sphinxuseclass}
\sphinxAtStartPar
At annual frequencies, log and simple returns can be quite different!
For example, at \$R\_\{simple\} \textbackslash{}approx 0.6\$, we see \$R\_\{log\} \textbackslash{}approx 0.4\$.

\begin{sphinxuseclass}{cell}\begin{sphinxVerbatimInput}

\begin{sphinxuseclass}{cell_input}
\begin{sphinxVerbatim}[commandchars=\\\{\}]
\PYG{n}{mkt\PYGZus{}a}\PYG{o}{.}\PYG{n}{pipe}\PYG{p}{(}\PYG{n}{plot\PYGZus{}compare\PYGZus{}returns}\PYG{p}{,} \PYG{n}{freq}\PYG{o}{=}\PYG{l+s+s1}{\PYGZsq{}}\PYG{l+s+s1}{Annual}\PYG{l+s+s1}{\PYGZsq{}}\PYG{p}{)}
\end{sphinxVerbatim}

\end{sphinxuseclass}\end{sphinxVerbatimInput}
\begin{sphinxVerbatimOutput}

\begin{sphinxuseclass}{cell_output}
\noindent\sphinxincludegraphics{{91cdd1a6918dd2338d6e0bca41671f9f1ec0bb9d5d3f929ce8578b058048cf54}.png}

\end{sphinxuseclass}\end{sphinxVerbatimOutput}

\end{sphinxuseclass}

\subsubsection{Repeat the autocorrelation plot with \sphinxstyleemphasis{monthly} returns instead of \sphinxstyleemphasis{daily} returns}
\label{\detokenize{herron_07_practice_02:repeat-the-autocorrelation-plot-with-monthly-returns-instead-of-daily-returns}}
\sphinxAtStartPar
Again, we will write a function to plot these autocorrelations.


\subsubsection{Plot monthly returns versus their one\sphinxhyphen{}month lag}
\label{\detokenize{herron_07_practice_02:plot-monthly-returns-versus-their-one-month-lag}}
\begin{sphinxuseclass}{cell}\begin{sphinxVerbatimInput}

\begin{sphinxuseclass}{cell_input}
\begin{sphinxVerbatim}[commandchars=\\\{\}]
\PYG{k}{def} \PYG{n+nf}{plot\PYGZus{}autocorr}\PYG{p}{(}\PYG{n}{x}\PYG{p}{,} \PYG{n}{freq}\PYG{p}{,} \PYG{n}{n}\PYG{o}{=}\PYG{l+m+mi}{10}\PYG{p}{,} \PYG{n}{labs}\PYG{o}{=}\PYG{n}{labels}\PYG{p}{)}\PYG{p}{:}
    \PYG{n}{x\PYGZus{}lags} \PYG{o}{=} \PYG{n}{pd}\PYG{o}{.}\PYG{n}{concat}\PYG{p}{(}\PYG{n}{objs}\PYG{o}{=}\PYG{p}{[}\PYG{n}{x}\PYG{p}{[}\PYG{l+s+s1}{\PYGZsq{}}\PYG{l+s+s1}{R}\PYG{l+s+s1}{\PYGZsq{}}\PYG{p}{]}\PYG{o}{.}\PYG{n}{shift}\PYG{p}{(}\PYG{n}{t}\PYG{p}{)} \PYG{k}{for} \PYG{n}{t} \PYG{o+ow}{in} \PYG{n+nb}{range}\PYG{p}{(}\PYG{n}{n} \PYG{o}{+} \PYG{l+m+mi}{1}\PYG{p}{)}\PYG{p}{]}\PYG{p}{,} \PYG{n}{axis}\PYG{o}{=}\PYG{l+m+mi}{1}\PYG{p}{)}
    \PYG{n}{corrs} \PYG{o}{=} \PYG{n}{x\PYGZus{}lags}\PYG{o}{.}\PYG{n}{corr}\PYG{p}{(}\PYG{p}{)}\PYG{o}{.}\PYG{n}{iloc}\PYG{p}{[}\PYG{l+m+mi}{0}\PYG{p}{]}
    \PYG{n}{serrs} \PYG{o}{=} \PYG{n}{np}\PYG{o}{.}\PYG{n}{sqrt}\PYG{p}{(}\PYG{p}{(}\PYG{l+m+mi}{1} \PYG{o}{\PYGZhy{}} \PYG{n}{corrs}\PYG{o}{*}\PYG{o}{*}\PYG{l+m+mi}{2}\PYG{p}{)} \PYG{o}{/} \PYG{p}{(}\PYG{n}{x\PYGZus{}lags}\PYG{o}{.}\PYG{n}{count}\PYG{p}{(}\PYG{p}{)} \PYG{o}{\PYGZhy{}} \PYG{l+m+mi}{2}\PYG{p}{)}\PYG{p}{)}

    \PYG{n}{plt}\PYG{o}{.}\PYG{n}{bar}\PYG{p}{(}\PYG{n}{height}\PYG{o}{=}\PYG{n}{corrs}\PYG{p}{,} \PYG{n}{x}\PYG{o}{=}\PYG{n+nb}{range}\PYG{p}{(}\PYG{n}{n} \PYG{o}{+} \PYG{l+m+mi}{1}\PYG{p}{)}\PYG{p}{,} \PYG{n}{yerr}\PYG{o}{=}\PYG{l+m+mi}{2}\PYG{o}{*}\PYG{n}{serrs}\PYG{p}{)}
    \PYG{n}{plt}\PYG{o}{.}\PYG{n}{title}\PYG{p}{(}
        \PYG{l+s+sa}{f}\PYG{l+s+s1}{\PYGZsq{}}\PYG{l+s+s1}{Autocorrelation of }\PYG{l+s+si}{\PYGZob{}}\PYG{n}{freq}\PYG{l+s+si}{\PYGZcb{}}\PYG{l+s+s1}{ Returns}\PYG{l+s+s1}{\PYGZsq{}} \PYG{o}{+}
        \PYG{l+s+s1}{\PYGZsq{}}\PYG{l+s+se}{\PYGZbs{}n}\PYG{l+s+s1}{Black Vertical Bars Indicate Standard Errors}\PYG{l+s+s1}{\PYGZsq{}}
    \PYG{p}{)}
    \PYG{n}{plt}\PYG{o}{.}\PYG{n}{xlabel}\PYG{p}{(}\PYG{l+s+sa}{f}\PYG{l+s+s1}{\PYGZsq{}}\PYG{l+s+si}{\PYGZob{}}\PYG{n}{freq}\PYG{l+s+si}{\PYGZcb{}}\PYG{l+s+s1}{ Lag}\PYG{l+s+s1}{\PYGZsq{}}\PYG{p}{)}
    \PYG{n}{plt}\PYG{o}{.}\PYG{n}{ylabel}\PYG{p}{(}\PYG{l+s+s1}{\PYGZsq{}}\PYG{l+s+s1}{Autocorrelation Coefficient}\PYG{l+s+s1}{\PYGZsq{}}\PYG{p}{)}
    \PYG{n}{plt}\PYG{o}{.}\PYG{n}{show}\PYG{p}{(}\PYG{p}{)}
\end{sphinxVerbatim}

\end{sphinxuseclass}\end{sphinxVerbatimInput}

\end{sphinxuseclass}
\sphinxAtStartPar
At the monthly horizon, return autocorrelation weakens!
All autocorrelation coefficient estimates are small and less than two standard errors away from zero.

\begin{sphinxuseclass}{cell}\begin{sphinxVerbatimInput}

\begin{sphinxuseclass}{cell_input}
\begin{sphinxVerbatim}[commandchars=\\\{\}]
\PYG{n}{mkt\PYGZus{}m}\PYG{o}{.}\PYG{n}{pipe}\PYG{p}{(}\PYG{n}{plot\PYGZus{}autocorr}\PYG{p}{,} \PYG{n}{freq}\PYG{o}{=}\PYG{l+s+s1}{\PYGZsq{}}\PYG{l+s+s1}{Monthly}\PYG{l+s+s1}{\PYGZsq{}}\PYG{p}{)}
\end{sphinxVerbatim}

\end{sphinxuseclass}\end{sphinxVerbatimInput}
\begin{sphinxVerbatimOutput}

\begin{sphinxuseclass}{cell_output}
\noindent\sphinxincludegraphics{{e50eaaa95743b0cfbb3f396f86d62395449865d5d37837b808ec8883ec31cd3f}.png}

\end{sphinxuseclass}\end{sphinxVerbatimOutput}

\end{sphinxuseclass}
\sphinxAtStartPar
At the annual horizon, return autocorrelation disappears!

\begin{sphinxuseclass}{cell}\begin{sphinxVerbatimInput}

\begin{sphinxuseclass}{cell_input}
\begin{sphinxVerbatim}[commandchars=\\\{\}]
\PYG{n}{mkt\PYGZus{}a}\PYG{o}{.}\PYG{n}{pipe}\PYG{p}{(}\PYG{n}{plot\PYGZus{}autocorr}\PYG{p}{,} \PYG{n}{freq}\PYG{o}{=}\PYG{l+s+s1}{\PYGZsq{}}\PYG{l+s+s1}{Annual}\PYG{l+s+s1}{\PYGZsq{}}\PYG{p}{)}
\end{sphinxVerbatim}

\end{sphinxuseclass}\end{sphinxVerbatimInput}
\begin{sphinxVerbatimOutput}

\begin{sphinxuseclass}{cell_output}
\noindent\sphinxincludegraphics{{464841c0b36ed50158550c2c63776c49d41933e9b37ec4d568cdf332d6b0ed28}.png}

\end{sphinxuseclass}\end{sphinxVerbatimOutput}

\end{sphinxuseclass}

\subsubsection{Repeat the autocorrelation plot with the \sphinxstyleemphasis{absolute value} of daily returns}
\label{\detokenize{herron_07_practice_02:repeat-the-autocorrelation-plot-with-the-absolute-value-of-daily-returns}}
\sphinxAtStartPar
We can use our function again!
But first, we have to convert simple returns to the \sphinxstyleemphasis{absolute value} of simple returns.

\begin{sphinxuseclass}{cell}\begin{sphinxVerbatimInput}

\begin{sphinxuseclass}{cell_input}
\begin{sphinxVerbatim}[commandchars=\\\{\}]
\PYG{p}{(}
    \PYG{n}{mkt}
    \PYG{o}{.}\PYG{n}{assign}\PYG{p}{(}\PYG{n}{R}\PYG{o}{=}\PYG{k}{lambda} \PYG{n}{x}\PYG{p}{:} \PYG{n}{x}\PYG{p}{[}\PYG{l+s+s1}{\PYGZsq{}}\PYG{l+s+s1}{R}\PYG{l+s+s1}{\PYGZsq{}}\PYG{p}{]}\PYG{o}{.}\PYG{n}{abs}\PYG{p}{(}\PYG{p}{)}\PYG{p}{)}
    \PYG{o}{.}\PYG{n}{pipe}\PYG{p}{(}\PYG{n}{plot\PYGZus{}autocorr}\PYG{p}{,} \PYG{n}{freq}\PYG{o}{=}\PYG{l+s+s1}{\PYGZsq{}}\PYG{l+s+s1}{Absolute Value of Daily}\PYG{l+s+s1}{\PYGZsq{}}\PYG{p}{,} \PYG{n}{n}\PYG{o}{=}\PYG{l+m+mi}{60}\PYG{p}{)}
\PYG{p}{)}
\end{sphinxVerbatim}

\end{sphinxuseclass}\end{sphinxVerbatimInput}
\begin{sphinxVerbatimOutput}

\begin{sphinxuseclass}{cell_output}
\noindent\sphinxincludegraphics{{cc8f241a78be63b790360cc93abd6d31ac40c11d0865e6e2dd7c866cd2d56c71}.png}

\end{sphinxuseclass}\end{sphinxVerbatimOutput}

\end{sphinxuseclass}
\sphinxAtStartPar
Saying that “squared returns” or “the absolute value of returns” is autocorrelated is similar to saying “return magnitudes” are autocorrelated, which is similar to saying “volatility” is autocorrelated or persistent.


\subsubsection{Repeat the leverage effect plot with lagged returns}
\label{\detokenize{herron_07_practice_02:repeat-the-leverage-effect-plot-with-lagged-returns}}
\sphinxAtStartPar
We will only make this plot, but we can still write a function!

\begin{sphinxuseclass}{cell}\begin{sphinxVerbatimInput}

\begin{sphinxuseclass}{cell_input}
\begin{sphinxVerbatim}[commandchars=\\\{\}]
\PYG{k+kn}{import} \PYG{n+nn}{seaborn} \PYG{k}{as} \PYG{n+nn}{sns}
\end{sphinxVerbatim}

\end{sphinxuseclass}\end{sphinxVerbatimInput}

\end{sphinxuseclass}
\begin{sphinxuseclass}{cell}\begin{sphinxVerbatimInput}

\begin{sphinxuseclass}{cell_input}
\begin{sphinxVerbatim}[commandchars=\\\{\}]
\PYG{k}{def} \PYG{n+nf}{plot\PYGZus{}leverage\PYGZus{}effect}\PYG{p}{(}\PYG{n}{x}\PYG{p}{,} \PYG{n}{rule}\PYG{p}{,} \PYG{n}{freq}\PYG{p}{)}\PYG{p}{:}
    \PYG{p}{(}
        \PYG{n}{x}\PYG{p}{[}\PYG{l+s+s1}{\PYGZsq{}}\PYG{l+s+s1}{R}\PYG{l+s+s1}{\PYGZsq{}}\PYG{p}{]}
        \PYG{o}{.}\PYG{n}{resample}\PYG{p}{(}\PYG{n}{rule}\PYG{o}{=}\PYG{n}{rule}\PYG{p}{,} \PYG{n}{kind}\PYG{o}{=}\PYG{l+s+s1}{\PYGZsq{}}\PYG{l+s+s1}{period}\PYG{l+s+s1}{\PYGZsq{}}\PYG{p}{)}
        \PYG{o}{.}\PYG{n}{agg}\PYG{p}{(}\PYG{p}{[}\PYG{n}{totret}\PYG{p}{,} \PYG{l+s+s1}{\PYGZsq{}}\PYG{l+s+s1}{std}\PYG{l+s+s1}{\PYGZsq{}}\PYG{p}{]}\PYG{p}{)}
        \PYG{o}{.}\PYG{n}{assign}\PYG{p}{(}\PYG{n}{totret\PYGZus{}lag}\PYG{o}{=}\PYG{k}{lambda} \PYG{n}{x}\PYG{p}{:} \PYG{n}{x}\PYG{p}{[}\PYG{l+s+s1}{\PYGZsq{}}\PYG{l+s+s1}{totret}\PYG{l+s+s1}{\PYGZsq{}}\PYG{p}{]}\PYG{o}{.}\PYG{n}{shift}\PYG{p}{(}\PYG{p}{)}\PYG{p}{)}
        \PYG{o}{.}\PYG{n}{mul}\PYG{p}{(}\PYG{l+m+mi}{100}\PYG{p}{)}
        \PYG{o}{.}\PYG{n}{pipe}\PYG{p}{(}\PYG{k}{lambda} \PYG{n}{x}\PYG{p}{:}
              \PYG{n}{sns}\PYG{o}{.}\PYG{n}{regplot}\PYG{p}{(}
                  \PYG{n}{x}\PYG{o}{=}\PYG{l+s+s1}{\PYGZsq{}}\PYG{l+s+s1}{totret\PYGZus{}lag}\PYG{l+s+s1}{\PYGZsq{}}\PYG{p}{,} 
                  \PYG{n}{y}\PYG{o}{=}\PYG{l+s+s1}{\PYGZsq{}}\PYG{l+s+s1}{std}\PYG{l+s+s1}{\PYGZsq{}}\PYG{p}{,} 
                  \PYG{n}{data}\PYG{o}{=}\PYG{n}{x}\PYG{p}{,}
                  \PYG{n}{scatter\PYGZus{}kws}\PYG{o}{=}\PYG{p}{\PYGZob{}}\PYG{l+s+s1}{\PYGZsq{}}\PYG{l+s+s1}{alpha}\PYG{l+s+s1}{\PYGZsq{}}\PYG{p}{:} \PYG{l+m+mf}{0.1}\PYG{p}{\PYGZcb{}}
            \PYG{p}{)}
        \PYG{p}{)}
    \PYG{p}{)}
    \PYG{n}{plt}\PYG{o}{.}\PYG{n}{xlabel}\PYG{p}{(}\PYG{l+s+sa}{f}\PYG{l+s+s1}{\PYGZsq{}}\PYG{l+s+s1}{One\PYGZhy{}}\PYG{l+s+si}{\PYGZob{}}\PYG{n}{freq}\PYG{l+s+si}{\PYGZcb{}}\PYG{l+s+s1}{ Lag of Total Return (\PYGZpc{})}\PYG{l+s+s1}{\PYGZsq{}}\PYG{p}{)}
    \PYG{n}{plt}\PYG{o}{.}\PYG{n}{ylabel}\PYG{p}{(}\PYG{l+s+s1}{\PYGZsq{}}\PYG{l+s+s1}{Volatility of Returns (}\PYG{l+s+s1}{\PYGZpc{}}\PYG{l+s+s1}{)}\PYG{l+s+s1}{\PYGZsq{}}\PYG{p}{)}
    \PYG{n}{plt}\PYG{o}{.}\PYG{n}{suptitle}\PYG{p}{(}
        \PYG{l+s+s1}{\PYGZsq{}}\PYG{l+s+s1}{Fisher Black}\PYG{l+s+se}{\PYGZbs{}\PYGZsq{}}\PYG{l+s+s1}{s Leverage Effect}\PYG{l+s+se}{\PYGZbs{}n}\PYG{l+s+s1}{\PYGZsq{}} \PYG{o}{+} 
        \PYG{l+s+sa}{f}\PYG{l+s+s1}{\PYGZsq{}}\PYG{l+s+s1}{(Daily Returns Aggregated by }\PYG{l+s+si}{\PYGZob{}}\PYG{n}{freq}\PYG{l+s+si}{\PYGZcb{}}\PYG{l+s+s1}{)}\PYG{l+s+s1}{\PYGZsq{}}
    \PYG{p}{)}
    \PYG{n}{plt}\PYG{o}{.}\PYG{n}{show}\PYG{p}{(}\PYG{p}{)}
\end{sphinxVerbatim}

\end{sphinxuseclass}\end{sphinxVerbatimInput}

\end{sphinxuseclass}
\begin{sphinxuseclass}{cell}\begin{sphinxVerbatimInput}

\begin{sphinxuseclass}{cell_input}
\begin{sphinxVerbatim}[commandchars=\\\{\}]
\PYG{n}{mkt}\PYG{o}{.}\PYG{n}{pipe}\PYG{p}{(}\PYG{n}{plot\PYGZus{}leverage\PYGZus{}effect}\PYG{p}{,} \PYG{n}{rule}\PYG{o}{=}\PYG{l+s+s1}{\PYGZsq{}}\PYG{l+s+s1}{M}\PYG{l+s+s1}{\PYGZsq{}}\PYG{p}{,} \PYG{n}{freq}\PYG{o}{=}\PYG{l+s+s1}{\PYGZsq{}}\PYG{l+s+s1}{Month}\PYG{l+s+s1}{\PYGZsq{}}\PYG{p}{)}
\end{sphinxVerbatim}

\end{sphinxuseclass}\end{sphinxVerbatimInput}
\begin{sphinxVerbatimOutput}

\begin{sphinxuseclass}{cell_output}
\noindent\sphinxincludegraphics{{b23fd71fc13e9a18e65dcc695f8b94000cb9cfdc1ea9fbd84b5559b54166ed22}.png}

\end{sphinxuseclass}\end{sphinxVerbatimOutput}

\end{sphinxuseclass}
\sphinxAtStartPar
Since we have the function, let use a try a few different horizons!

\begin{sphinxuseclass}{cell}\begin{sphinxVerbatimInput}

\begin{sphinxuseclass}{cell_input}
\begin{sphinxVerbatim}[commandchars=\\\{\}]
\PYG{n}{mkt}\PYG{o}{.}\PYG{n}{pipe}\PYG{p}{(}\PYG{n}{plot\PYGZus{}leverage\PYGZus{}effect}\PYG{p}{,} \PYG{n}{rule}\PYG{o}{=}\PYG{l+s+s1}{\PYGZsq{}}\PYG{l+s+s1}{Q}\PYG{l+s+s1}{\PYGZsq{}}\PYG{p}{,} \PYG{n}{freq}\PYG{o}{=}\PYG{l+s+s1}{\PYGZsq{}}\PYG{l+s+s1}{Quarter}\PYG{l+s+s1}{\PYGZsq{}}\PYG{p}{)}
\end{sphinxVerbatim}

\end{sphinxuseclass}\end{sphinxVerbatimInput}
\begin{sphinxVerbatimOutput}

\begin{sphinxuseclass}{cell_output}
\noindent\sphinxincludegraphics{{4aea303060f95e8ebc6fa066317af74180a96d10a65d6ec7eb41c225bb93b389}.png}

\end{sphinxuseclass}\end{sphinxVerbatimOutput}

\end{sphinxuseclass}
\begin{sphinxuseclass}{cell}\begin{sphinxVerbatimInput}

\begin{sphinxuseclass}{cell_input}
\begin{sphinxVerbatim}[commandchars=\\\{\}]
\PYG{n}{mkt}\PYG{o}{.}\PYG{n}{pipe}\PYG{p}{(}\PYG{n}{plot\PYGZus{}leverage\PYGZus{}effect}\PYG{p}{,} \PYG{n}{rule}\PYG{o}{=}\PYG{l+s+s1}{\PYGZsq{}}\PYG{l+s+s1}{A}\PYG{l+s+s1}{\PYGZsq{}}\PYG{p}{,} \PYG{n}{freq}\PYG{o}{=}\PYG{l+s+s1}{\PYGZsq{}}\PYG{l+s+s1}{Annual}\PYG{l+s+s1}{\PYGZsq{}}\PYG{p}{)}
\end{sphinxVerbatim}

\end{sphinxuseclass}\end{sphinxVerbatimInput}
\begin{sphinxVerbatimOutput}

\begin{sphinxuseclass}{cell_output}
\noindent\sphinxincludegraphics{{b558f57c6bf8364ad9d969e983e09b83213e07b01e821eea53c09352d4f9c20d}.png}

\end{sphinxuseclass}\end{sphinxVerbatimOutput}

\end{sphinxuseclass}
\sphinxstepscope


\part{Quizzes}

\sphinxstepscope

\begin{sphinxuseclass}{cell}\begin{sphinxVerbatimInput}

\begin{sphinxuseclass}{cell_input}
\begin{sphinxVerbatim}[commandchars=\\\{\}]
\PYG{c+c1}{\PYGZsh{} Initialize Otter}
\PYG{k+kn}{import} \PYG{n+nn}{otter}
\PYG{n}{grader} \PYG{o}{=} \PYG{n}{otter}\PYG{o}{.}\PYG{n}{Notebook}\PYG{p}{(}\PYG{l+s+s2}{\PYGZdq{}}\PYG{l+s+s2}{quiz\PYGZus{}01.ipynb}\PYG{l+s+s2}{\PYGZdq{}}\PYG{p}{)}
\end{sphinxVerbatim}

\end{sphinxuseclass}\end{sphinxVerbatimInput}

\end{sphinxuseclass}

\chapter{Quiz 1}
\label{\detokenize{quiz_01:quiz-1}}\label{\detokenize{quiz_01::doc}}

\section{Instructions}
\label{\detokenize{quiz_01:instructions}}\begin{enumerate}
\sphinxsetlistlabels{\arabic}{enumi}{enumii}{}{.}%
\item {} 
\sphinxAtStartPar
After you answer a question, you should run its public tests.

\item {} 
\sphinxAtStartPar
After you answer every question, you should:
\begin{enumerate}
\sphinxsetlistlabels{\arabic}{enumii}{enumiii}{}{.}%
\item {} 
\sphinxAtStartPar
Restart your kernel and clear all output

\item {} 
\sphinxAtStartPar
Run up to the last cell

\item {} 
\sphinxAtStartPar
Save your notebook

\item {} 
\sphinxAtStartPar
Run the last cell to create a .zip file for Gradescope

\item {} 
\sphinxAtStartPar
Upload this .zip file to Gradescope

\item {} 
\sphinxAtStartPar
\sphinxstyleemphasis{\sphinxstylestrong{Make sure your local autograder results match your Gradescope autograder results}}

\end{enumerate}

\item {} 
\sphinxAtStartPar
This quiz has public and hidden tests:
\begin{enumerate}
\sphinxsetlistlabels{\arabic}{enumii}{enumiii}{}{.}%
\item {} 
\sphinxAtStartPar
Public tests check if your answers are the correct types and shapes, but they may not check if your answers are exactly correct

\item {} 
\sphinxAtStartPar
Hidden tests check if your answers are exactly correct, but their results are not available until after the due date

\end{enumerate}

\item {} 
\sphinxAtStartPar
You may ask technical questions on Canvas Discussions, but the quiz is an individual effort.

\end{enumerate}


\section{Questions}
\label{\detokenize{quiz_01:questions}}

\subsection{Question 1}
\label{\detokenize{quiz_01:question-1}}
\sphinxAtStartPar
Create a list named \sphinxcode{\sphinxupquote{list\_1}} that contains the squares for integers from 1 to 100.
The first and last values in \sphinxcode{\sphinxupquote{list\_1}} should be \$1\textasciicircum{}2 = 1\$ and \$100\textasciicircum{}2 = 10,000\$, respectively.

\sphinxAtStartPar
\sphinxstyleemphasis{Points:} 30

\begin{sphinxuseclass}{cell}\begin{sphinxVerbatimInput}

\begin{sphinxuseclass}{cell_input}
\begin{sphinxVerbatim}[commandchars=\\\{\}]
\PYG{o}{.}\PYG{o}{.}\PYG{o}{.}
\end{sphinxVerbatim}

\end{sphinxuseclass}\end{sphinxVerbatimInput}

\end{sphinxuseclass}
\begin{sphinxuseclass}{cell}\begin{sphinxVerbatimInput}

\begin{sphinxuseclass}{cell_input}
\begin{sphinxVerbatim}[commandchars=\\\{\}]
\PYG{n}{grader}\PYG{o}{.}\PYG{n}{check}\PYG{p}{(}\PYG{l+s+s2}{\PYGZdq{}}\PYG{l+s+s2}{q1}\PYG{l+s+s2}{\PYGZdq{}}\PYG{p}{)}
\end{sphinxVerbatim}

\end{sphinxuseclass}\end{sphinxVerbatimInput}

\end{sphinxuseclass}

\subsection{Question 2}
\label{\detokenize{quiz_01:question-2}}
\sphinxAtStartPar
Create an integer \sphinxcode{\sphinxupquote{int\_2}} that contains the sum of squared integers from 1 to 100 inclusive.
This \sphinxcode{\sphinxupquote{int\_2}} contains the sum of the values in \sphinxcode{\sphinxupquote{list\_1}} from question 1.

\sphinxAtStartPar
\sphinxstyleemphasis{Points:} 30

\begin{sphinxuseclass}{cell}\begin{sphinxVerbatimInput}

\begin{sphinxuseclass}{cell_input}
\begin{sphinxVerbatim}[commandchars=\\\{\}]
\PYG{o}{.}\PYG{o}{.}\PYG{o}{.}
\end{sphinxVerbatim}

\end{sphinxuseclass}\end{sphinxVerbatimInput}

\end{sphinxuseclass}
\begin{sphinxuseclass}{cell}\begin{sphinxVerbatimInput}

\begin{sphinxuseclass}{cell_input}
\begin{sphinxVerbatim}[commandchars=\\\{\}]
\PYG{n}{grader}\PYG{o}{.}\PYG{n}{check}\PYG{p}{(}\PYG{l+s+s2}{\PYGZdq{}}\PYG{l+s+s2}{q2}\PYG{l+s+s2}{\PYGZdq{}}\PYG{p}{)}
\end{sphinxVerbatim}

\end{sphinxuseclass}\end{sphinxVerbatimInput}

\end{sphinxuseclass}

\subsection{Question 3}
\label{\detokenize{quiz_01:question-3}}
\sphinxAtStartPar
Write a function \sphinxcode{\sphinxupquote{fun\_3}} that sums that squares of integers between its arguments \sphinxcode{\sphinxupquote{start}} and \sphinxcode{\sphinxupquote{stop}} inclusive.
If \sphinxcode{\sphinxupquote{start=1}} and \sphinxcode{\sphinxupquote{stop=2}}, \sphinxcode{\sphinxupquote{fun\_3}} should return \$1\textasciicircum{}2 + 2\textasciicircum{}2 = 1 + 4 = 5\$.
If \sphinxcode{\sphinxupquote{start=1}} and \sphinxcode{\sphinxupquote{stop=100}}, \sphinxcode{\sphinxupquote{fun\_3}} should return your answer from question 2.
I will test your function at several values of \sphinxcode{\sphinxupquote{start}} and \sphinxcode{\sphinxupquote{stop}}, you may assume that \sphinxcode{\sphinxupquote{stop}} is always greater than \sphinxcode{\sphinxupquote{start}}.

\sphinxAtStartPar
\sphinxstyleemphasis{\sphinxstylestrong{Note:}}
This question has 1 hidden test worth 10 points.
The autograder will not show you hidden test results until after the due date, and Gradescope will show your quiz score as \sphinxcode{\sphinxupquote{\textasciitilde{}/100}} until after the due date.

\sphinxAtStartPar
\sphinxstyleemphasis{Points:} 40

\begin{sphinxuseclass}{cell}\begin{sphinxVerbatimInput}

\begin{sphinxuseclass}{cell_input}
\begin{sphinxVerbatim}[commandchars=\\\{\}]
\PYG{k}{def} \PYG{n+nf}{fun\PYGZus{}3}\PYG{p}{(}\PYG{n}{start}\PYG{p}{,} \PYG{n}{stop}\PYG{p}{)}\PYG{p}{:}
\PYG{o}{.}\PYG{o}{.}\PYG{o}{.}
\end{sphinxVerbatim}

\end{sphinxuseclass}\end{sphinxVerbatimInput}

\end{sphinxuseclass}
\begin{sphinxuseclass}{cell}\begin{sphinxVerbatimInput}

\begin{sphinxuseclass}{cell_input}
\begin{sphinxVerbatim}[commandchars=\\\{\}]
\PYG{n}{grader}\PYG{o}{.}\PYG{n}{check}\PYG{p}{(}\PYG{l+s+s2}{\PYGZdq{}}\PYG{l+s+s2}{q3}\PYG{l+s+s2}{\PYGZdq{}}\PYG{p}{)}
\end{sphinxVerbatim}

\end{sphinxuseclass}\end{sphinxVerbatimInput}

\end{sphinxuseclass}

\section{Submission}
\label{\detokenize{quiz_01:submission}}
\sphinxAtStartPar
Make sure you have run all cells in your notebook in order before running the cell below, so that all images/graphs appear in the output. The cell below will generate a zip file for you to submit. \sphinxstylestrong{Please save before exporting!}

\begin{sphinxuseclass}{cell}\begin{sphinxVerbatimInput}

\begin{sphinxuseclass}{cell_input}
\begin{sphinxVerbatim}[commandchars=\\\{\}]
\PYG{c+c1}{\PYGZsh{} Save your notebook first, then run this cell to export your submission.}
\PYG{n}{grader}\PYG{o}{.}\PYG{n}{export}\PYG{p}{(}\PYG{n}{pdf}\PYG{o}{=}\PYG{k+kc}{False}\PYG{p}{,} \PYG{n}{run\PYGZus{}tests}\PYG{o}{=}\PYG{k+kc}{True}\PYG{p}{)}
\end{sphinxVerbatim}

\end{sphinxuseclass}\end{sphinxVerbatimInput}

\end{sphinxuseclass}
\sphinxstepscope

\begin{sphinxuseclass}{cell}\begin{sphinxVerbatimInput}

\begin{sphinxuseclass}{cell_input}
\begin{sphinxVerbatim}[commandchars=\\\{\}]
\PYG{c+c1}{\PYGZsh{} Initialize Otter}
\PYG{k+kn}{import} \PYG{n+nn}{otter}
\PYG{n}{grader} \PYG{o}{=} \PYG{n}{otter}\PYG{o}{.}\PYG{n}{Notebook}\PYG{p}{(}\PYG{l+s+s2}{\PYGZdq{}}\PYG{l+s+s2}{quiz\PYGZus{}02.ipynb}\PYG{l+s+s2}{\PYGZdq{}}\PYG{p}{)}
\end{sphinxVerbatim}

\end{sphinxuseclass}\end{sphinxVerbatimInput}

\end{sphinxuseclass}

\chapter{Quiz 2}
\label{\detokenize{quiz_02:quiz-2}}\label{\detokenize{quiz_02::doc}}

\section{Instructions}
\label{\detokenize{quiz_02:instructions}}\begin{enumerate}
\sphinxsetlistlabels{\arabic}{enumi}{enumii}{}{.}%
\item {} 
\sphinxAtStartPar
After you answer a question, you should run its public tests.

\item {} 
\sphinxAtStartPar
After you answer every question, you should:
\begin{enumerate}
\sphinxsetlistlabels{\arabic}{enumii}{enumiii}{}{.}%
\item {} 
\sphinxAtStartPar
Restart your kernel and clear all output

\item {} 
\sphinxAtStartPar
Run up to the last cell

\item {} 
\sphinxAtStartPar
Save your notebook

\item {} 
\sphinxAtStartPar
Run the last cell to create a .zip file for Gradescope

\item {} 
\sphinxAtStartPar
Upload this .zip file to Gradescope

\item {} 
\sphinxAtStartPar
\sphinxstyleemphasis{\sphinxstylestrong{Make sure your local autograder results match your Gradescope autograder results}}

\end{enumerate}

\item {} 
\sphinxAtStartPar
This quiz has public and hidden tests:
\begin{enumerate}
\sphinxsetlistlabels{\arabic}{enumii}{enumiii}{}{.}%
\item {} 
\sphinxAtStartPar
Public tests check your answers for correct types and shapes but may not completely check your answers

\item {} 
\sphinxAtStartPar
Hidden tests completely check your answers but will not be available until after the due date

\end{enumerate}

\item {} 
\sphinxAtStartPar
You may ask technical questions on Canvas Discussions, but the quiz is an individual effort.

\end{enumerate}


\section{Packages and Settings}
\label{\detokenize{quiz_02:packages-and-settings}}
\begin{sphinxuseclass}{cell}\begin{sphinxVerbatimInput}

\begin{sphinxuseclass}{cell_input}
\begin{sphinxVerbatim}[commandchars=\\\{\}]
\PYG{k+kn}{import} \PYG{n+nn}{numpy} \PYG{k}{as} \PYG{n+nn}{np}
\PYG{o}{\PYGZpc{}}\PYG{k}{precision} 4
\end{sphinxVerbatim}

\end{sphinxuseclass}\end{sphinxVerbatimInput}

\end{sphinxuseclass}

\section{Questions}
\label{\detokenize{quiz_02:questions}}

\subsection{Question 1}
\label{\detokenize{quiz_02:question-1}}
\sphinxAtStartPar
Write a function \sphinxcode{\sphinxupquote{npv()}} that calculates the net present value of a NumPy array of cashflows given a discount rate.
Assume the following:
\begin{enumerate}
\sphinxsetlistlabels{\arabic}{enumi}{enumii}{}{.}%
\item {} 
\sphinxAtStartPar
Argument \sphinxcode{\sphinxupquote{c}} is a NumPy array of annual cash flows, starting at \$t=0\$

\item {} 
\sphinxAtStartPar
Argument \sphinxcode{\sphinxupquote{r}} is a float of the annual discount rate as a decimal (i.e., \sphinxcode{\sphinxupquote{r=0.1}} indicates \$r=10\%\$)

\end{enumerate}

\sphinxAtStartPar
\sphinxstyleemphasis{\sphinxstylestrong{Note: This question has 1 hidden test worth a total of 20 points.}}

\sphinxAtStartPar
\sphinxstyleemphasis{Points:} 50

\begin{sphinxuseclass}{cell}\begin{sphinxVerbatimInput}

\begin{sphinxuseclass}{cell_input}
\begin{sphinxVerbatim}[commandchars=\\\{\}]
\PYG{k}{def} \PYG{n+nf}{npv}\PYG{p}{(}\PYG{n}{c}\PYG{p}{,} \PYG{n}{r}\PYG{p}{)}\PYG{p}{:}
    \PYG{o}{.}\PYG{o}{.}\PYG{o}{.}
\end{sphinxVerbatim}

\end{sphinxuseclass}\end{sphinxVerbatimInput}

\end{sphinxuseclass}
\begin{sphinxuseclass}{cell}\begin{sphinxVerbatimInput}

\begin{sphinxuseclass}{cell_input}
\begin{sphinxVerbatim}[commandchars=\\\{\}]
\PYG{n}{grader}\PYG{o}{.}\PYG{n}{check}\PYG{p}{(}\PYG{l+s+s2}{\PYGZdq{}}\PYG{l+s+s2}{q1}\PYG{l+s+s2}{\PYGZdq{}}\PYG{p}{)}
\end{sphinxVerbatim}

\end{sphinxuseclass}\end{sphinxVerbatimInput}

\end{sphinxuseclass}

\subsection{Question 2}
\label{\detokenize{quiz_02:question-2}}
\sphinxAtStartPar
Write a function \sphinxcode{\sphinxupquote{totret()}} that calculates the total return of a NumPy array of returns.
Assume the following:
\begin{enumerate}
\sphinxsetlistlabels{\arabic}{enumi}{enumii}{}{.}%
\item {} 
\sphinxAtStartPar
Argument \sphinxcode{\sphinxupquote{r}} is a NumPy array of decimal returns (i.e., \sphinxcode{\sphinxupquote{r}}=0.1 indicates \$r=10\%\$ and \$r > \sphinxhyphen{}100\%\$ for all \$r\$)

\item {} 
\sphinxAtStartPar
Function \sphinxcode{\sphinxupquote{totret()}} should return total return as a decimal

\end{enumerate}

\sphinxAtStartPar
\sphinxstyleemphasis{\sphinxstylestrong{Note: This question has 2 hidden tests worth a total of 20 points.}}

\sphinxAtStartPar
\sphinxstyleemphasis{Points:} 50

\begin{sphinxuseclass}{cell}\begin{sphinxVerbatimInput}

\begin{sphinxuseclass}{cell_input}
\begin{sphinxVerbatim}[commandchars=\\\{\}]
\PYG{k}{def} \PYG{n+nf}{totret}\PYG{p}{(}\PYG{n}{r}\PYG{p}{)}\PYG{p}{:}
    \PYG{o}{.}\PYG{o}{.}\PYG{o}{.}
\end{sphinxVerbatim}

\end{sphinxuseclass}\end{sphinxVerbatimInput}

\end{sphinxuseclass}
\begin{sphinxuseclass}{cell}\begin{sphinxVerbatimInput}

\begin{sphinxuseclass}{cell_input}
\begin{sphinxVerbatim}[commandchars=\\\{\}]
\PYG{n}{grader}\PYG{o}{.}\PYG{n}{check}\PYG{p}{(}\PYG{l+s+s2}{\PYGZdq{}}\PYG{l+s+s2}{q2}\PYG{l+s+s2}{\PYGZdq{}}\PYG{p}{)}
\end{sphinxVerbatim}

\end{sphinxuseclass}\end{sphinxVerbatimInput}

\end{sphinxuseclass}

\section{Submission}
\label{\detokenize{quiz_02:submission}}
\sphinxAtStartPar
Make sure you have run all cells in your notebook in order before running the cell below, so that all images/graphs appear in the output. The cell below will generate a zip file for you to submit. \sphinxstylestrong{Please save before exporting!}

\begin{sphinxuseclass}{cell}\begin{sphinxVerbatimInput}

\begin{sphinxuseclass}{cell_input}
\begin{sphinxVerbatim}[commandchars=\\\{\}]
\PYG{c+c1}{\PYGZsh{} Save your notebook first, then run this cell to export your submission.}
\PYG{n}{grader}\PYG{o}{.}\PYG{n}{export}\PYG{p}{(}\PYG{n}{pdf}\PYG{o}{=}\PYG{k+kc}{False}\PYG{p}{,} \PYG{n}{run\PYGZus{}tests}\PYG{o}{=}\PYG{k+kc}{True}\PYG{p}{)}
\end{sphinxVerbatim}

\end{sphinxuseclass}\end{sphinxVerbatimInput}

\end{sphinxuseclass}
\sphinxstepscope


\chapter{Quiz 3}
\label{\detokenize{quiz_03:quiz-3}}\label{\detokenize{quiz_03::doc}}
\sphinxAtStartPar
Quiz 3 involves two data files.
I zipped the notebook and two data files into one file, but I cannot include zip files on this website.
Therefore, I posted this file to Canvas under \sphinxhref{https://northeastern.instructure.com/courses/137565/files/folder/Quiz\%20Zips}{Files/Quiz Zips}.
I linked this file from the Canvas assignment, too.
Download and extract this file, leaving the file names and relative positions as\sphinxhyphen{}is.

\sphinxstepscope


\chapter{Quiz 4}
\label{\detokenize{quiz_04:quiz-4}}\label{\detokenize{quiz_04::doc}}
\sphinxAtStartPar
Quiz 4 involves one data file.
I zipped the notebook and data file into one file, but I cannot include zip files on this website.
Therefore, I posted this file to Canvas under \sphinxhref{https://northeastern.instructure.com/courses/137565/files/folder/Quiz\%20Zips}{Files/Quiz Zips}.
I linked this file from the Canvas assignment, too.
Download and extract this file, leaving the file names and relative positions as\sphinxhyphen{}is.

\sphinxstepscope


\chapter{Quiz 5}
\label{\detokenize{quiz_05:quiz-5}}\label{\detokenize{quiz_05::doc}}
\sphinxAtStartPar
Quiz 5 involves one data file.
I zipped the notebook and data file into one file, but I cannot include zip files on this website.
Therefore, I posted this file to Canvas under \sphinxhref{https://northeastern.instructure.com/courses/137565/files/folder/Quiz\%20Zips}{Files/Quiz Zips}.
I linked this file from the Canvas assignment, too.
Download and extract this file, leaving the file names and relative positions as\sphinxhyphen{}is.

\sphinxstepscope


\chapter{Quiz 6}
\label{\detokenize{quiz_06:quiz-6}}\label{\detokenize{quiz_06::doc}}
\sphinxAtStartPar
Quiz 6 involves one data file.
I zipped the notebook and data file into one file, but I cannot include zip files on this website.
Therefore, I posted this zip file to Canvas under \sphinxhref{https://northeastern.instructure.com/courses/137565/files/folder/Quiz\%20Zips}{Files/Quiz Zips}.
I linked this file from the Canvas assignment, too.
Download and extract this file, leaving any file names and relative locations as\sphinxhyphen{}is.

\sphinxstepscope


\chapter{Quiz 7}
\label{\detokenize{quiz_07:quiz-7}}\label{\detokenize{quiz_07::doc}}
\sphinxAtStartPar
Quiz 7 involves one data file.
I zipped the notebook and data file into one file, but I cannot include zip files on this website.
Therefore, I posted this zip file to Canvas under \sphinxhref{https://northeastern.instructure.com/courses/137565/files/folder/Quiz\%20Zips}{Files/Quiz Zips}.
I linked this file from the Canvas assignment, too.
Download and extract this file, leaving any file names and relative locations as\sphinxhyphen{}is.

\sphinxstepscope


\part{Projects}

\sphinxstepscope


\chapter{Project 1}
\label{\detokenize{project_01:project-1}}\label{\detokenize{project_01::doc}}
\sphinxAtStartPar
Project 1 involves two data files.
I zipped the notebook and two data files into one file, but I cannot include zip files on this website.
Therefore, I posted this file to Canvas under \sphinxhref{https://northeastern.instructure.com/courses/137565/files/folder/Project\%20Zips}{Files/Project Zips/project\_01.zip}.
I linked this file from the Canvas assignment, too.
Download and extract this file, leaving the file names and relative positions as\sphinxhyphen{}is.

\sphinxstepscope


\section{Project 1 Solution}
\label{\detokenize{project_01_solution:project-1-solution}}\label{\detokenize{project_01_solution::doc}}

\section{Purpose}
\label{\detokenize{project_01_solution:purpose}}
\sphinxAtStartPar
I have two goals for this project:
\begin{enumerate}
\sphinxsetlistlabels{\arabic}{enumi}{enumii}{}{.}%
\item {} 
\sphinxAtStartPar
To help you master data manipulation and visualization

\item {} 
\sphinxAtStartPar
To help you understand the risk\sphinxhyphen{}return tradeoff for several measures of risk

\end{enumerate}


\section{Tasks}
\label{\detokenize{project_01_solution:tasks}}

\subsection{Packages and Settings}
\label{\detokenize{project_01_solution:packages-and-settings}}
\begin{sphinxuseclass}{cell}\begin{sphinxVerbatimInput}

\begin{sphinxuseclass}{cell_input}
\begin{sphinxVerbatim}[commandchars=\\\{\}]
\PYG{k+kn}{import} \PYG{n+nn}{matplotlib}\PYG{n+nn}{.}\PYG{n+nn}{pyplot} \PYG{k}{as} \PYG{n+nn}{plt}
\PYG{k+kn}{import} \PYG{n+nn}{numpy} \PYG{k}{as} \PYG{n+nn}{np}
\PYG{k+kn}{import} \PYG{n+nn}{pandas} \PYG{k}{as} \PYG{n+nn}{pd}
\end{sphinxVerbatim}

\end{sphinxuseclass}\end{sphinxVerbatimInput}

\end{sphinxuseclass}
\begin{sphinxuseclass}{cell}\begin{sphinxVerbatimInput}

\begin{sphinxuseclass}{cell_input}
\begin{sphinxVerbatim}[commandchars=\\\{\}]
\PYG{o}{\PYGZpc{}}\PYG{k}{config} InlineBackend.figure\PYGZus{}format = \PYGZsq{}retina\PYGZsq{}
\PYG{o}{\PYGZpc{}}\PYG{k}{precision} 4
\PYG{n}{pd}\PYG{o}{.}\PYG{n}{options}\PYG{o}{.}\PYG{n}{display}\PYG{o}{.}\PYG{n}{float\PYGZus{}format} \PYG{o}{=} \PYG{l+s+s1}{\PYGZsq{}}\PYG{l+s+si}{\PYGZob{}:.4f\PYGZcb{}}\PYG{l+s+s1}{\PYGZsq{}}\PYG{o}{.}\PYG{n}{format}
\end{sphinxVerbatim}

\end{sphinxuseclass}\end{sphinxVerbatimInput}

\end{sphinxuseclass}

\subsection{Data}
\label{\detokenize{project_01_solution:data}}
\sphinxAtStartPar
I used the following code cell to download the data for this project.
Leave this code cell commented out and use the CSV files I provided with this notebook.

\begin{sphinxuseclass}{cell}\begin{sphinxVerbatimInput}

\begin{sphinxuseclass}{cell_input}
\begin{sphinxVerbatim}[commandchars=\\\{\}]
\PYG{c+c1}{\PYGZsh{} import yfinance as yf}
\PYG{c+c1}{\PYGZsh{} import pandas\PYGZus{}datareader as pdr}
\PYG{c+c1}{\PYGZsh{} import requests\PYGZus{}cache}
\PYG{c+c1}{\PYGZsh{} session = requests\PYGZus{}cache.CachedSession(expire\PYGZus{}after=1)}
\end{sphinxVerbatim}

\end{sphinxuseclass}\end{sphinxVerbatimInput}

\end{sphinxuseclass}
\begin{sphinxuseclass}{cell}\begin{sphinxVerbatimInput}

\begin{sphinxuseclass}{cell_input}
\begin{sphinxVerbatim}[commandchars=\\\{\}]
\PYG{c+c1}{\PYGZsh{} wiki = pd.read\PYGZus{}html(\PYGZsq{}https://en.wikipedia.org/wiki/Russell\PYGZus{}1000\PYGZus{}Index\PYGZsq{})}
\end{sphinxVerbatim}

\end{sphinxuseclass}\end{sphinxVerbatimInput}

\end{sphinxuseclass}
\begin{sphinxuseclass}{cell}\begin{sphinxVerbatimInput}

\begin{sphinxuseclass}{cell_input}
\begin{sphinxVerbatim}[commandchars=\\\{\}]
\PYG{c+c1}{\PYGZsh{} (}
\PYG{c+c1}{\PYGZsh{}     yf.Tickers(}
\PYG{c+c1}{\PYGZsh{}         tickers=wiki[2][\PYGZsq{}Ticker\PYGZsq{}].str.replace(pat=\PYGZsq{}.\PYGZsq{}, repl=\PYGZsq{}\PYGZhy{}\PYGZsq{}, regex=False).to\PYGZus{}list(),}
\PYG{c+c1}{\PYGZsh{}         session=session}
\PYG{c+c1}{\PYGZsh{}     )}
\PYG{c+c1}{\PYGZsh{}     .history(period=\PYGZsq{}max\PYGZsq{}, auto\PYGZus{}adjust=False)}
\PYG{c+c1}{\PYGZsh{}     .assign(Date = lambda x: x.index.tz\PYGZus{}localize(None))}
\PYG{c+c1}{\PYGZsh{}     .set\PYGZus{}index(\PYGZsq{}Date\PYGZsq{})}
\PYG{c+c1}{\PYGZsh{}     .rename\PYGZus{}axis(columns=[\PYGZsq{}Variable\PYGZsq{}, \PYGZsq{}Ticker\PYGZsq{}])}
\PYG{c+c1}{\PYGZsh{}     [\PYGZsq{}Adj Close\PYGZsq{}]}
\PYG{c+c1}{\PYGZsh{}     .pct\PYGZus{}change()}
\PYG{c+c1}{\PYGZsh{}     .loc[\PYGZsq{}1962\PYGZsq{}:\PYGZsq{}2022\PYGZsq{}]}
\PYG{c+c1}{\PYGZsh{}     .to\PYGZus{}csv(\PYGZsq{}returns.csv\PYGZsq{})}
\PYG{c+c1}{\PYGZsh{} )}
\end{sphinxVerbatim}

\end{sphinxuseclass}\end{sphinxVerbatimInput}

\end{sphinxuseclass}
\begin{sphinxuseclass}{cell}\begin{sphinxVerbatimInput}

\begin{sphinxuseclass}{cell_input}
\begin{sphinxVerbatim}[commandchars=\\\{\}]
\PYG{c+c1}{\PYGZsh{} (}
\PYG{c+c1}{\PYGZsh{}     pdr.DataReader(}
\PYG{c+c1}{\PYGZsh{}         name=\PYGZsq{}F\PYGZhy{}F\PYGZus{}Research\PYGZus{}Data\PYGZus{}Factors\PYGZus{}daily\PYGZsq{},}
\PYG{c+c1}{\PYGZsh{}         data\PYGZus{}source=\PYGZsq{}famafrench\PYGZsq{},}
\PYG{c+c1}{\PYGZsh{}         start=\PYGZsq{}1900\PYGZsq{},}
\PYG{c+c1}{\PYGZsh{}         session=session}
\PYG{c+c1}{\PYGZsh{}     )}
\PYG{c+c1}{\PYGZsh{}     [0]}
\PYG{c+c1}{\PYGZsh{}     .rename\PYGZus{}axis(columns=\PYGZsq{}Variable\PYGZsq{})}
\PYG{c+c1}{\PYGZsh{}     .div(100)}
\PYG{c+c1}{\PYGZsh{}     .loc[\PYGZsq{}1962\PYGZsq{}:\PYGZsq{}2022\PYGZsq{}]}
\PYG{c+c1}{\PYGZsh{}     .to\PYGZus{}csv(\PYGZsq{}ff.csv\PYGZsq{})}
\PYG{c+c1}{\PYGZsh{} )}
\end{sphinxVerbatim}

\end{sphinxuseclass}\end{sphinxVerbatimInput}

\end{sphinxuseclass}
\sphinxAtStartPar
Run the following code cell to read the data for this project.
The \sphinxcode{\sphinxupquote{returns.csv}} file contains daily returns for the Russell 1000 stocks from 1962 through 2022, and the \sphinxcode{\sphinxupquote{ff.csv}} contains daily Fama and French factors from 1962 through 2022.

\begin{sphinxuseclass}{cell}\begin{sphinxVerbatimInput}

\begin{sphinxuseclass}{cell_input}
\begin{sphinxVerbatim}[commandchars=\\\{\}]
\PYG{n}{returns} \PYG{o}{=} \PYG{n}{pd}\PYG{o}{.}\PYG{n}{read\PYGZus{}csv}\PYG{p}{(}\PYG{l+s+s1}{\PYGZsq{}}\PYG{l+s+s1}{returns.csv}\PYG{l+s+s1}{\PYGZsq{}}\PYG{p}{,} \PYG{n}{index\PYGZus{}col}\PYG{o}{=}\PYG{l+s+s1}{\PYGZsq{}}\PYG{l+s+s1}{Date}\PYG{l+s+s1}{\PYGZsq{}}\PYG{p}{,} \PYG{n}{parse\PYGZus{}dates}\PYG{o}{=}\PYG{k+kc}{True}\PYG{p}{)}
\PYG{n}{ff} \PYG{o}{=} \PYG{n}{pd}\PYG{o}{.}\PYG{n}{read\PYGZus{}csv}\PYG{p}{(}\PYG{l+s+s1}{\PYGZsq{}}\PYG{l+s+s1}{ff.csv}\PYG{l+s+s1}{\PYGZsq{}}\PYG{p}{,} \PYG{n}{index\PYGZus{}col}\PYG{o}{=}\PYG{l+s+s1}{\PYGZsq{}}\PYG{l+s+s1}{Date}\PYG{l+s+s1}{\PYGZsq{}}\PYG{p}{,} \PYG{n}{parse\PYGZus{}dates}\PYG{o}{=}\PYG{k+kc}{True}\PYG{p}{)}
\end{sphinxVerbatim}

\end{sphinxuseclass}\end{sphinxVerbatimInput}

\end{sphinxuseclass}

\subsection{Single Stocks}
\label{\detokenize{project_01_solution:single-stocks}}
\sphinxAtStartPar
For this section, use the single stock returns in \sphinxcode{\sphinxupquote{returns}}.
You may select years \$t\$ and \$t+1\$, but only use stocks with complete returns data for years \$t\$ and \$t+1\$.


\subsubsection{Task 1: Do mean returns in year \$t\$ predict mean returns in year \$t+1\$?}
\label{\detokenize{project_01_solution:task-1-do-mean-returns-in-year-t-predict-mean-returns-in-year-t-1}}
\begin{sphinxuseclass}{cell}\begin{sphinxVerbatimInput}

\begin{sphinxuseclass}{cell_input}
\begin{sphinxVerbatim}[commandchars=\\\{\}]
\PYG{n}{time} \PYG{o}{=} \PYG{l+m+mi}{2018}
\PYG{n}{time}\PYG{p}{,} \PYG{n}{timep1} \PYG{o}{=} \PYG{n+nb}{str}\PYG{p}{(}\PYG{n}{time}\PYG{p}{)}\PYG{p}{,} \PYG{n+nb}{str}\PYG{p}{(}\PYG{n}{time} \PYG{o}{+} \PYG{l+m+mi}{1}\PYG{p}{)}
\end{sphinxVerbatim}

\end{sphinxuseclass}\end{sphinxVerbatimInput}

\end{sphinxuseclass}
\begin{sphinxuseclass}{cell}\begin{sphinxVerbatimInput}

\begin{sphinxuseclass}{cell_input}
\begin{sphinxVerbatim}[commandchars=\\\{\}]
\PYG{k}{def} \PYG{n+nf}{mean}\PYG{p}{(}\PYG{n}{ri}\PYG{p}{,} \PYG{n}{ann}\PYG{o}{=}\PYG{l+m+mi}{252}\PYG{p}{,} \PYG{n}{mul}\PYG{o}{=}\PYG{l+m+mi}{100}\PYG{p}{)}\PYG{p}{:}
    \PYG{k}{return} \PYG{n}{mul} \PYG{o}{*} \PYG{n}{ann} \PYG{o}{*} \PYG{n}{ri}\PYG{o}{.}\PYG{n}{mean}\PYG{p}{(}\PYG{p}{)}
\end{sphinxVerbatim}

\end{sphinxuseclass}\end{sphinxVerbatimInput}

\end{sphinxuseclass}
\begin{sphinxuseclass}{cell}\begin{sphinxVerbatimInput}

\begin{sphinxuseclass}{cell_input}
\begin{sphinxVerbatim}[commandchars=\\\{\}]
\PYG{k}{def} \PYG{n+nf}{std}\PYG{p}{(}\PYG{n}{ri}\PYG{p}{,} \PYG{n}{ann}\PYG{o}{=}\PYG{n}{np}\PYG{o}{.}\PYG{n}{sqrt}\PYG{p}{(}\PYG{l+m+mi}{252}\PYG{p}{)}\PYG{p}{,} \PYG{n}{mul}\PYG{o}{=}\PYG{l+m+mi}{100}\PYG{p}{)}\PYG{p}{:}
    \PYG{k}{return} \PYG{n}{mul} \PYG{o}{*} \PYG{n}{ann} \PYG{o}{*} \PYG{n}{ri}\PYG{o}{.}\PYG{n}{std}\PYG{p}{(}\PYG{p}{)}
\end{sphinxVerbatim}

\end{sphinxuseclass}\end{sphinxVerbatimInput}

\end{sphinxuseclass}
\begin{sphinxuseclass}{cell}\begin{sphinxVerbatimInput}

\begin{sphinxuseclass}{cell_input}
\begin{sphinxVerbatim}[commandchars=\\\{\}]
\PYG{k}{def} \PYG{n+nf}{sharpe}\PYG{p}{(}\PYG{n}{ri}\PYG{p}{,} \PYG{n}{rf}\PYG{o}{=}\PYG{n}{ff}\PYG{p}{[}\PYG{l+s+s1}{\PYGZsq{}}\PYG{l+s+s1}{RF}\PYG{l+s+s1}{\PYGZsq{}}\PYG{p}{]}\PYG{p}{,} \PYG{n}{ann}\PYG{o}{=}\PYG{n}{np}\PYG{o}{.}\PYG{n}{sqrt}\PYG{p}{(}\PYG{l+m+mi}{252}\PYG{p}{)}\PYG{p}{)}\PYG{p}{:}
    \PYG{n}{ri\PYGZus{}rf} \PYG{o}{=} \PYG{n}{ri}\PYG{o}{.}\PYG{n}{sub}\PYG{p}{(}\PYG{n}{rf}\PYG{p}{)}\PYG{o}{.}\PYG{n}{dropna}\PYG{p}{(}\PYG{p}{)}
    \PYG{k}{return} \PYG{n}{ann} \PYG{o}{*} \PYG{n}{ri\PYGZus{}rf}\PYG{o}{.}\PYG{n}{mean}\PYG{p}{(}\PYG{p}{)} \PYG{o}{/} \PYG{n}{ri\PYGZus{}rf}\PYG{o}{.}\PYG{n}{std}\PYG{p}{(}\PYG{p}{)}
\end{sphinxVerbatim}

\end{sphinxuseclass}\end{sphinxVerbatimInput}

\end{sphinxuseclass}
\begin{sphinxuseclass}{cell}\begin{sphinxVerbatimInput}

\begin{sphinxuseclass}{cell_input}
\begin{sphinxVerbatim}[commandchars=\\\{\}]
\PYG{k}{def} \PYG{n+nf}{beta}\PYG{p}{(}\PYG{n}{ri}\PYG{p}{,} \PYG{n}{rf}\PYG{o}{=}\PYG{n}{ff}\PYG{p}{[}\PYG{l+s+s1}{\PYGZsq{}}\PYG{l+s+s1}{RF}\PYG{l+s+s1}{\PYGZsq{}}\PYG{p}{]}\PYG{p}{,} \PYG{n}{rm\PYGZus{}rf}\PYG{o}{=}\PYG{n}{ff}\PYG{p}{[}\PYG{l+s+s1}{\PYGZsq{}}\PYG{l+s+s1}{Mkt\PYGZhy{}RF}\PYG{l+s+s1}{\PYGZsq{}}\PYG{p}{]}\PYG{p}{)}\PYG{p}{:}
    \PYG{n}{ri\PYGZus{}rf} \PYG{o}{=} \PYG{n}{ri}\PYG{o}{.}\PYG{n}{sub}\PYG{p}{(}\PYG{n}{rf}\PYG{p}{)}\PYG{o}{.}\PYG{n}{dropna}\PYG{p}{(}\PYG{p}{)}
    \PYG{k}{return} \PYG{n}{ri\PYGZus{}rf}\PYG{o}{.}\PYG{n}{cov}\PYG{p}{(}\PYG{n}{rm\PYGZus{}rf}\PYG{p}{)} \PYG{o}{/} \PYG{n}{rm\PYGZus{}rf}\PYG{o}{.}\PYG{n}{loc}\PYG{p}{[}\PYG{n}{ri\PYGZus{}rf}\PYG{o}{.}\PYG{n}{index}\PYG{p}{]}\PYG{o}{.}\PYG{n}{var}\PYG{p}{(}\PYG{p}{)}
\end{sphinxVerbatim}

\end{sphinxuseclass}\end{sphinxVerbatimInput}

\end{sphinxuseclass}
\begin{sphinxuseclass}{cell}\begin{sphinxVerbatimInput}

\begin{sphinxuseclass}{cell_input}
\begin{sphinxVerbatim}[commandchars=\\\{\}]
\PYG{k}{def} \PYG{n+nf}{stats\PYGZus{}all\PYGZus{}ann}\PYG{p}{(}\PYG{n}{x}\PYG{p}{)}\PYG{p}{:}
    \PYG{k}{return} \PYG{p}{(}
        \PYG{n}{x}
        \PYG{o}{.}\PYG{n}{resample}\PYG{p}{(}\PYG{n}{rule}\PYG{o}{=}\PYG{l+s+s1}{\PYGZsq{}}\PYG{l+s+s1}{A}\PYG{l+s+s1}{\PYGZsq{}}\PYG{p}{,} \PYG{n}{kind}\PYG{o}{=}\PYG{l+s+s1}{\PYGZsq{}}\PYG{l+s+s1}{period}\PYG{l+s+s1}{\PYGZsq{}}\PYG{p}{)}
        \PYG{o}{.}\PYG{n}{agg}\PYG{p}{(}\PYG{p}{[}\PYG{n}{mean}\PYG{p}{,} \PYG{n}{std}\PYG{p}{,} \PYG{n}{sharpe}\PYG{p}{,} \PYG{n}{beta}\PYG{p}{]}\PYG{p}{)}
        \PYG{o}{.}\PYG{n}{rename\PYGZus{}axis}\PYG{p}{(}\PYG{p}{[}\PYG{l+s+s1}{\PYGZsq{}}\PYG{l+s+s1}{Ticker}\PYG{l+s+s1}{\PYGZsq{}}\PYG{p}{,} \PYG{l+s+s1}{\PYGZsq{}}\PYG{l+s+s1}{Statistic}\PYG{l+s+s1}{\PYGZsq{}}\PYG{p}{]}\PYG{p}{,} \PYG{n}{axis}\PYG{o}{=}\PYG{l+m+mi}{1}\PYG{p}{)}
        \PYG{o}{.}\PYG{n}{stack}\PYG{p}{(}\PYG{l+s+s1}{\PYGZsq{}}\PYG{l+s+s1}{Ticker}\PYG{l+s+s1}{\PYGZsq{}}\PYG{p}{)}
        \PYG{o}{.}\PYG{n}{unstack}\PYG{p}{(}\PYG{l+s+s1}{\PYGZsq{}}\PYG{l+s+s1}{Date}\PYG{l+s+s1}{\PYGZsq{}}\PYG{p}{)}
    \PYG{p}{)}
\end{sphinxVerbatim}

\end{sphinxuseclass}\end{sphinxVerbatimInput}

\end{sphinxuseclass}
\begin{sphinxuseclass}{cell}\begin{sphinxVerbatimInput}

\begin{sphinxuseclass}{cell_input}
\begin{sphinxVerbatim}[commandchars=\\\{\}]
\PYG{k+kn}{import} \PYG{n+nn}{seaborn} \PYG{k}{as} \PYG{n+nn}{sns}

\PYG{k}{def} \PYG{n+nf}{scatter}\PYG{p}{(}\PYG{n}{df}\PYG{p}{,} \PYG{o}{*}\PYG{o}{*}\PYG{n}{kwargs}\PYG{p}{)}\PYG{p}{:}
    \PYG{n}{sns}\PYG{o}{.}\PYG{n}{regplot}\PYG{p}{(}\PYG{n}{data}\PYG{o}{=}\PYG{n}{df}\PYG{p}{,} \PYG{n}{scatter\PYGZus{}kws}\PYG{o}{=}\PYG{p}{\PYGZob{}}\PYG{l+s+s1}{\PYGZsq{}}\PYG{l+s+s1}{alpha}\PYG{l+s+s1}{\PYGZsq{}}\PYG{p}{:} \PYG{l+m+mf}{0.1}\PYG{p}{\PYGZcb{}}\PYG{p}{,} \PYG{o}{*}\PYG{o}{*}\PYG{n}{kwargs}\PYG{p}{)}
    \PYG{k}{return} \PYG{k+kc}{None}
\end{sphinxVerbatim}

\end{sphinxuseclass}\end{sphinxVerbatimInput}

\end{sphinxuseclass}
\begin{sphinxuseclass}{cell}\begin{sphinxVerbatimInput}

\begin{sphinxuseclass}{cell_input}
\begin{sphinxVerbatim}[commandchars=\\\{\}]
\PYG{k}{def} \PYG{n+nf}{stats\PYGZus{}corr}\PYG{p}{(}\PYG{n}{x}\PYG{p}{,} \PYG{n}{sub}\PYG{p}{,} \PYG{n}{pct}\PYG{o}{=}\PYG{l+m+mi}{5}\PYG{p}{)}\PYG{p}{:}
    \PYG{n}{n} \PYG{o}{=} \PYG{n+nb}{int}\PYG{p}{(}\PYG{n}{pct} \PYG{o}{*} \PYG{n}{x}\PYG{o}{.}\PYG{n}{shape}\PYG{p}{[}\PYG{l+m+mi}{0}\PYG{p}{]} \PYG{o}{/} \PYG{l+m+mi}{100}\PYG{p}{)}
    \PYG{k}{return} \PYG{p}{(}
        \PYG{n}{pd}\PYG{o}{.}\PYG{n}{DataFrame}\PYG{p}{(}
            \PYG{n}{data}\PYG{o}{=}\PYG{p}{\PYGZob{}}\PYG{l+s+s1}{\PYGZsq{}}\PYG{l+s+s1}{rho}\PYG{l+s+s1}{\PYGZsq{}}\PYG{p}{:} \PYG{p}{[}
                \PYG{n}{x}\PYG{o}{.}\PYG{n}{corr}\PYG{p}{(}\PYG{p}{)}\PYG{o}{.}\PYG{n}{iloc}\PYG{p}{[}\PYG{l+m+mi}{0}\PYG{p}{,} \PYG{l+m+mi}{1}\PYG{p}{]}\PYG{p}{,} 
                \PYG{n}{x}\PYG{o}{.}\PYG{n}{sort\PYGZus{}values}\PYG{p}{(}\PYG{n}{x}\PYG{o}{.}\PYG{n}{columns}\PYG{p}{[}\PYG{l+m+mi}{0}\PYG{p}{]}\PYG{p}{)}\PYG{o}{.}\PYG{n}{iloc}\PYG{p}{[}\PYG{n}{n}\PYG{p}{:}\PYG{o}{\PYGZhy{}}\PYG{n}{n}\PYG{p}{]}\PYG{o}{.}\PYG{n}{corr}\PYG{p}{(}\PYG{p}{)}\PYG{o}{.}\PYG{n}{iloc}\PYG{p}{[}\PYG{l+m+mi}{0}\PYG{p}{,} \PYG{l+m+mi}{1}\PYG{p}{]}
            \PYG{p}{]}\PYG{p}{\PYGZcb{}}\PYG{p}{,}
            \PYG{n}{index}\PYG{o}{=}\PYG{p}{[}
                \PYG{l+s+sa}{f}\PYG{l+s+s1}{\PYGZsq{}}\PYG{l+s+s1}{All Values}\PYG{l+s+s1}{\PYGZsq{}}\PYG{p}{,}
                \PYG{l+s+sa}{f}\PYG{l+s+s1}{\PYGZsq{}}\PYG{l+s+s1}{Trim }\PYG{l+s+si}{\PYGZob{}}\PYG{n}{pct}\PYG{l+s+si}{\PYGZcb{}}\PYG{l+s+s1}{\PYGZpc{} Outliers}\PYG{l+s+s1}{\PYGZsq{}}
            \PYG{p}{]}
        \PYG{p}{)}
        \PYG{o}{.}\PYG{n}{rename\PYGZus{}axis}\PYG{p}{(}\PYG{n}{index}\PYG{o}{=}\PYG{l+s+s1}{\PYGZsq{}}\PYG{l+s+s1}{Sample}\PYG{l+s+s1}{\PYGZsq{}}\PYG{p}{,} \PYG{n}{columns}\PYG{o}{=}\PYG{l+s+s1}{\PYGZsq{}}\PYG{l+s+s1}{Statistic}\PYG{l+s+s1}{\PYGZsq{}}\PYG{p}{)}
        \PYG{o}{.}\PYG{n}{style}\PYG{o}{.}\PYG{n}{set\PYGZus{}caption}\PYG{p}{(}\PYG{n}{sub}\PYG{p}{)}
    \PYG{p}{)}
\end{sphinxVerbatim}

\end{sphinxuseclass}\end{sphinxVerbatimInput}

\end{sphinxuseclass}
\begin{sphinxuseclass}{cell}\begin{sphinxVerbatimInput}

\begin{sphinxuseclass}{cell_input}
\begin{sphinxVerbatim}[commandchars=\\\{\}]
\PYG{n}{df\PYGZus{}1} \PYG{o}{=} \PYG{p}{(}
    \PYG{n}{returns}
    \PYG{o}{.}\PYG{n}{loc}\PYG{p}{[}\PYG{n}{time}\PYG{p}{:}\PYG{n}{timep1}\PYG{p}{]}
    \PYG{o}{.}\PYG{n}{dropna}\PYG{p}{(}\PYG{n}{axis}\PYG{o}{=}\PYG{l+m+mi}{1}\PYG{p}{)}
    \PYG{o}{.}\PYG{n}{pipe}\PYG{p}{(}\PYG{n}{stats\PYGZus{}all\PYGZus{}ann}\PYG{p}{)}
\PYG{p}{)}
\end{sphinxVerbatim}

\end{sphinxuseclass}\end{sphinxVerbatimInput}

\end{sphinxuseclass}
\begin{sphinxuseclass}{cell}\begin{sphinxVerbatimInput}

\begin{sphinxuseclass}{cell_input}
\begin{sphinxVerbatim}[commandchars=\\\{\}]
\PYG{n}{df\PYGZus{}1}\PYG{o}{.}\PYG{n}{head}\PYG{p}{(}\PYG{p}{)}
\end{sphinxVerbatim}

\end{sphinxuseclass}\end{sphinxVerbatimInput}
\begin{sphinxVerbatimOutput}

\begin{sphinxuseclass}{cell_output}
\begin{sphinxVerbatim}[commandchars=\\\{\}]
Statistic   beta            mean           sharpe             std        
Date        2018   2019     2018     2019    2018    2019    2018    2019
Ticker                                                                   
A         1.0995 1.2094   5.3336  27.3691  0.1305  1.0312 27.0755 24.4681
AA        1.2903 1.7639 \PYGZhy{}60.6016 \PYGZhy{}13.4767 \PYGZhy{}1.3938 \PYGZhy{}0.3970 44.7694 39.3309
AAL       1.3264 1.5513 \PYGZhy{}39.4041  \PYGZhy{}3.7435 \PYGZhy{}1.0287 \PYGZhy{}0.1655 40.0530 35.5337
AAP       0.8528 0.6426  51.1971   5.3710  1.5344  0.1220 32.1921 26.5124
AAPL      1.2410 1.4751  \PYGZhy{}1.4430  67.1478 \PYGZhy{}0.1128  2.4871 28.7421 26.1383
\end{sphinxVerbatim}

\end{sphinxuseclass}\end{sphinxVerbatimOutput}

\end{sphinxuseclass}
\begin{sphinxuseclass}{cell}\begin{sphinxVerbatimInput}

\begin{sphinxuseclass}{cell_input}
\begin{sphinxVerbatim}[commandchars=\\\{\}]
\PYG{n}{df\PYGZus{}1}\PYG{p}{[}\PYG{l+s+s1}{\PYGZsq{}}\PYG{l+s+s1}{mean}\PYG{l+s+s1}{\PYGZsq{}}\PYG{p}{]}\PYG{o}{.}\PYG{n}{pipe}\PYG{p}{(}\PYG{n}{scatter}\PYG{p}{,} \PYG{n}{x}\PYG{o}{=}\PYG{n}{time}\PYG{p}{,} \PYG{n}{y}\PYG{o}{=}\PYG{n}{timep1}\PYG{p}{)}
\PYG{n}{plt}\PYG{o}{.}\PYG{n}{xlabel}\PYG{p}{(}\PYG{l+s+sa}{f}\PYG{l+s+s1}{\PYGZsq{}}\PYG{l+s+s1}{Ann. Mean of Daily Returns (\PYGZpc{}) for }\PYG{l+s+si}{\PYGZob{}}\PYG{n}{time}\PYG{l+s+si}{\PYGZcb{}}\PYG{l+s+s1}{\PYGZsq{}}\PYG{p}{)}
\PYG{n}{plt}\PYG{o}{.}\PYG{n}{ylabel}\PYG{p}{(}\PYG{l+s+sa}{f}\PYG{l+s+s1}{\PYGZsq{}}\PYG{l+s+s1}{Ann. Mean of Daily Returns (\PYGZpc{}) for }\PYG{l+s+si}{\PYGZob{}}\PYG{n}{timep1}\PYG{l+s+si}{\PYGZcb{}}\PYG{l+s+s1}{\PYGZsq{}}\PYG{p}{)}
\PYG{n}{plt}\PYG{o}{.}\PYG{n}{title}\PYG{p}{(}\PYG{l+s+s1}{\PYGZsq{}}\PYG{l+s+s1}{Return Predictability for Russell 1000 Stocks}\PYG{l+s+s1}{\PYGZsq{}}\PYG{p}{)}
\PYG{n}{plt}\PYG{o}{.}\PYG{n}{show}\PYG{p}{(}\PYG{p}{)}
\end{sphinxVerbatim}

\end{sphinxuseclass}\end{sphinxVerbatimInput}
\begin{sphinxVerbatimOutput}

\begin{sphinxuseclass}{cell_output}
\noindent\sphinxincludegraphics{{30c30daf910fac43b106df83e10b554fde67f1d93e91fc833d9bec5dde2465b2}.png}

\end{sphinxuseclass}\end{sphinxVerbatimOutput}

\end{sphinxuseclass}
\begin{sphinxuseclass}{cell}\begin{sphinxVerbatimInput}

\begin{sphinxuseclass}{cell_input}
\begin{sphinxVerbatim}[commandchars=\\\{\}]
\PYG{n}{df\PYGZus{}1}\PYG{p}{[}\PYG{l+s+s1}{\PYGZsq{}}\PYG{l+s+s1}{mean}\PYG{l+s+s1}{\PYGZsq{}}\PYG{p}{]}\PYG{o}{.}\PYG{n}{pipe}\PYG{p}{(}
    \PYG{n}{stats\PYGZus{}corr}\PYG{p}{,} 
    \PYG{n}{sub}\PYG{o}{=}\PYG{l+s+sa}{f}\PYG{l+s+s1}{\PYGZsq{}}\PYG{l+s+s1}{Ann. Means of Daily Returns (\PYGZpc{}) for }\PYG{l+s+si}{\PYGZob{}}\PYG{n}{time}\PYG{l+s+si}{\PYGZcb{}}\PYG{l+s+s1}{ and }\PYG{l+s+si}{\PYGZob{}}\PYG{n}{timep1}\PYG{l+s+si}{\PYGZcb{}}\PYG{l+s+s1}{\PYGZsq{}}\PYG{p}{,}
\PYG{p}{)}
\end{sphinxVerbatim}

\end{sphinxuseclass}\end{sphinxVerbatimInput}
\begin{sphinxVerbatimOutput}

\begin{sphinxuseclass}{cell_output}
\begin{sphinxVerbatim}[commandchars=\\\{\}]
\PYGZlt{}pandas.io.formats.style.Styler at 0x182014734c0\PYGZgt{}
\end{sphinxVerbatim}

\end{sphinxuseclass}\end{sphinxVerbatimOutput}

\end{sphinxuseclass}

\subsubsection{Task 2: Does volatility in year \$t\$ predict volatility in year \$t+1\$?}
\label{\detokenize{project_01_solution:task-2-does-volatility-in-year-t-predict-volatility-in-year-t-1}}
\begin{sphinxuseclass}{cell}\begin{sphinxVerbatimInput}

\begin{sphinxuseclass}{cell_input}
\begin{sphinxVerbatim}[commandchars=\\\{\}]
\PYG{n}{df\PYGZus{}1}\PYG{p}{[}\PYG{l+s+s1}{\PYGZsq{}}\PYG{l+s+s1}{std}\PYG{l+s+s1}{\PYGZsq{}}\PYG{p}{]}\PYG{o}{.}\PYG{n}{pipe}\PYG{p}{(}\PYG{n}{scatter}\PYG{p}{,} \PYG{n}{x}\PYG{o}{=}\PYG{n}{time}\PYG{p}{,} \PYG{n}{y}\PYG{o}{=}\PYG{n}{timep1}\PYG{p}{)}
\PYG{n}{plt}\PYG{o}{.}\PYG{n}{xlabel}\PYG{p}{(}\PYG{l+s+sa}{f}\PYG{l+s+s1}{\PYGZsq{}}\PYG{l+s+s1}{Ann. Vol. of Daily Returns (\PYGZpc{}) for }\PYG{l+s+si}{\PYGZob{}}\PYG{n}{time}\PYG{l+s+si}{\PYGZcb{}}\PYG{l+s+s1}{\PYGZsq{}}\PYG{p}{)}
\PYG{n}{plt}\PYG{o}{.}\PYG{n}{ylabel}\PYG{p}{(}\PYG{l+s+sa}{f}\PYG{l+s+s1}{\PYGZsq{}}\PYG{l+s+s1}{Ann. Vol. of Daily Returns (\PYGZpc{}) for }\PYG{l+s+si}{\PYGZob{}}\PYG{n}{timep1}\PYG{l+s+si}{\PYGZcb{}}\PYG{l+s+s1}{\PYGZsq{}}\PYG{p}{)}
\PYG{n}{plt}\PYG{o}{.}\PYG{n}{title}\PYG{p}{(}\PYG{l+s+s1}{\PYGZsq{}}\PYG{l+s+s1}{Volatility Predictability for Russell 1000 Stocks}\PYG{l+s+s1}{\PYGZsq{}}\PYG{p}{)}
\PYG{n}{plt}\PYG{o}{.}\PYG{n}{show}\PYG{p}{(}\PYG{p}{)}
\end{sphinxVerbatim}

\end{sphinxuseclass}\end{sphinxVerbatimInput}
\begin{sphinxVerbatimOutput}

\begin{sphinxuseclass}{cell_output}
\noindent\sphinxincludegraphics{{334211fe16f96af2c93a1a01a48f1972c13a1b91839d07b70f2f98bf30f7b358}.png}

\end{sphinxuseclass}\end{sphinxVerbatimOutput}

\end{sphinxuseclass}
\begin{sphinxuseclass}{cell}\begin{sphinxVerbatimInput}

\begin{sphinxuseclass}{cell_input}
\begin{sphinxVerbatim}[commandchars=\\\{\}]
\PYG{n}{df\PYGZus{}1}\PYG{p}{[}\PYG{l+s+s1}{\PYGZsq{}}\PYG{l+s+s1}{std}\PYG{l+s+s1}{\PYGZsq{}}\PYG{p}{]}\PYG{o}{.}\PYG{n}{pipe}\PYG{p}{(}
    \PYG{n}{stats\PYGZus{}corr}\PYG{p}{,}
    \PYG{n}{sub}\PYG{o}{=}\PYG{l+s+sa}{f}\PYG{l+s+s1}{\PYGZsq{}}\PYG{l+s+s1}{Ann. Vol. of Daily Returns (\PYGZpc{}) for }\PYG{l+s+si}{\PYGZob{}}\PYG{n}{time}\PYG{l+s+si}{\PYGZcb{}}\PYG{l+s+s1}{ and }\PYG{l+s+si}{\PYGZob{}}\PYG{n}{timep1}\PYG{l+s+si}{\PYGZcb{}}\PYG{l+s+s1}{\PYGZsq{}}
\PYG{p}{)}
\end{sphinxVerbatim}

\end{sphinxuseclass}\end{sphinxVerbatimInput}
\begin{sphinxVerbatimOutput}

\begin{sphinxuseclass}{cell_output}
\begin{sphinxVerbatim}[commandchars=\\\{\}]
\PYGZlt{}pandas.io.formats.style.Styler at 0x182029001c0\PYGZgt{}
\end{sphinxVerbatim}

\end{sphinxuseclass}\end{sphinxVerbatimOutput}

\end{sphinxuseclass}

\subsubsection{Task 3: Do Sharpe Ratios in year \$t\$ predict Sharpe Ratios in year \$t+1\$?}
\label{\detokenize{project_01_solution:task-3-do-sharpe-ratios-in-year-t-predict-sharpe-ratios-in-year-t-1}}
\begin{sphinxuseclass}{cell}\begin{sphinxVerbatimInput}

\begin{sphinxuseclass}{cell_input}
\begin{sphinxVerbatim}[commandchars=\\\{\}]
\PYG{n}{df\PYGZus{}1}\PYG{p}{[}\PYG{l+s+s1}{\PYGZsq{}}\PYG{l+s+s1}{sharpe}\PYG{l+s+s1}{\PYGZsq{}}\PYG{p}{]}\PYG{o}{.}\PYG{n}{pipe}\PYG{p}{(}\PYG{n}{scatter}\PYG{p}{,} \PYG{n}{x}\PYG{o}{=}\PYG{n}{time}\PYG{p}{,} \PYG{n}{y}\PYG{o}{=}\PYG{n}{timep1}\PYG{p}{)}
\PYG{n}{plt}\PYG{o}{.}\PYG{n}{xlabel}\PYG{p}{(}\PYG{l+s+sa}{f}\PYG{l+s+s1}{\PYGZsq{}}\PYG{l+s+s1}{Sharpe Ratios of Daily Returns for }\PYG{l+s+si}{\PYGZob{}}\PYG{n}{time}\PYG{l+s+si}{\PYGZcb{}}\PYG{l+s+s1}{\PYGZsq{}}\PYG{p}{)}
\PYG{n}{plt}\PYG{o}{.}\PYG{n}{ylabel}\PYG{p}{(}\PYG{l+s+sa}{f}\PYG{l+s+s1}{\PYGZsq{}}\PYG{l+s+s1}{Sharpe Ratios of Daily Returns for }\PYG{l+s+si}{\PYGZob{}}\PYG{n}{timep1}\PYG{l+s+si}{\PYGZcb{}}\PYG{l+s+s1}{\PYGZsq{}}\PYG{p}{)}
\PYG{n}{plt}\PYG{o}{.}\PYG{n}{title}\PYG{p}{(}\PYG{l+s+s1}{\PYGZsq{}}\PYG{l+s+s1}{Sharpe Ratio Predictability for Russell 1000 Stocks}\PYG{l+s+s1}{\PYGZsq{}}\PYG{p}{)}
\PYG{n}{plt}\PYG{o}{.}\PYG{n}{show}\PYG{p}{(}\PYG{p}{)}
\end{sphinxVerbatim}

\end{sphinxuseclass}\end{sphinxVerbatimInput}
\begin{sphinxVerbatimOutput}

\begin{sphinxuseclass}{cell_output}
\noindent\sphinxincludegraphics{{e3dc9052925f43561886a55396ac67bbe6c012b3e8a8169e53ac4dc7cfbd7cec}.png}

\end{sphinxuseclass}\end{sphinxVerbatimOutput}

\end{sphinxuseclass}
\begin{sphinxuseclass}{cell}\begin{sphinxVerbatimInput}

\begin{sphinxuseclass}{cell_input}
\begin{sphinxVerbatim}[commandchars=\\\{\}]
\PYG{n}{df\PYGZus{}1}\PYG{p}{[}\PYG{l+s+s1}{\PYGZsq{}}\PYG{l+s+s1}{sharpe}\PYG{l+s+s1}{\PYGZsq{}}\PYG{p}{]}\PYG{o}{.}\PYG{n}{pipe}\PYG{p}{(}
    \PYG{n}{stats\PYGZus{}corr}\PYG{p}{,} 
    \PYG{n}{sub}\PYG{o}{=}\PYG{l+s+sa}{f}\PYG{l+s+s1}{\PYGZsq{}}\PYG{l+s+s1}{Sharpe Ratios of Daily Returns for }\PYG{l+s+si}{\PYGZob{}}\PYG{n}{time}\PYG{l+s+si}{\PYGZcb{}}\PYG{l+s+s1}{ and }\PYG{l+s+si}{\PYGZob{}}\PYG{n}{timep1}\PYG{l+s+si}{\PYGZcb{}}\PYG{l+s+s1}{\PYGZsq{}}
\PYG{p}{)}
\end{sphinxVerbatim}

\end{sphinxuseclass}\end{sphinxVerbatimInput}
\begin{sphinxVerbatimOutput}

\begin{sphinxuseclass}{cell_output}
\begin{sphinxVerbatim}[commandchars=\\\{\}]
\PYGZlt{}pandas.io.formats.style.Styler at 0x1820288f730\PYGZgt{}
\end{sphinxVerbatim}

\end{sphinxuseclass}\end{sphinxVerbatimOutput}

\end{sphinxuseclass}

\subsubsection{Task 4: Do CAPM betas in year \$t\$ predict CAPM betas in year \$t+1\$?}
\label{\detokenize{project_01_solution:task-4-do-capm-betas-in-year-t-predict-capm-betas-in-year-t-1}}
\begin{sphinxuseclass}{cell}\begin{sphinxVerbatimInput}

\begin{sphinxuseclass}{cell_input}
\begin{sphinxVerbatim}[commandchars=\\\{\}]
\PYG{n}{df\PYGZus{}1}\PYG{p}{[}\PYG{l+s+s1}{\PYGZsq{}}\PYG{l+s+s1}{beta}\PYG{l+s+s1}{\PYGZsq{}}\PYG{p}{]}\PYG{o}{.}\PYG{n}{pipe}\PYG{p}{(}\PYG{n}{scatter}\PYG{p}{,} \PYG{n}{x}\PYG{o}{=}\PYG{n}{time}\PYG{p}{,} \PYG{n}{y}\PYG{o}{=}\PYG{n}{timep1}\PYG{p}{)}
\PYG{n}{plt}\PYG{o}{.}\PYG{n}{xlabel}\PYG{p}{(}\PYG{l+s+sa}{f}\PYG{l+s+s1}{\PYGZsq{}}\PYG{l+s+s1}{Beta of Daily Returns for }\PYG{l+s+si}{\PYGZob{}}\PYG{n}{time}\PYG{l+s+si}{\PYGZcb{}}\PYG{l+s+s1}{\PYGZsq{}}\PYG{p}{)}
\PYG{n}{plt}\PYG{o}{.}\PYG{n}{ylabel}\PYG{p}{(}\PYG{l+s+sa}{f}\PYG{l+s+s1}{\PYGZsq{}}\PYG{l+s+s1}{Beta of Daily Returns for }\PYG{l+s+si}{\PYGZob{}}\PYG{n}{timep1}\PYG{l+s+si}{\PYGZcb{}}\PYG{l+s+s1}{\PYGZsq{}}\PYG{p}{)}
\PYG{n}{plt}\PYG{o}{.}\PYG{n}{title}\PYG{p}{(}\PYG{l+s+s1}{\PYGZsq{}}\PYG{l+s+s1}{Beta Predictability for Russell 1000 Stocks}\PYG{l+s+s1}{\PYGZsq{}}\PYG{p}{)}
\PYG{n}{plt}\PYG{o}{.}\PYG{n}{show}\PYG{p}{(}\PYG{p}{)}
\end{sphinxVerbatim}

\end{sphinxuseclass}\end{sphinxVerbatimInput}
\begin{sphinxVerbatimOutput}

\begin{sphinxuseclass}{cell_output}
\noindent\sphinxincludegraphics{{d2a2b7cd58b52766b7602be32ab3f9c97e0047ffa20d782aff6acc954085d755}.png}

\end{sphinxuseclass}\end{sphinxVerbatimOutput}

\end{sphinxuseclass}
\begin{sphinxuseclass}{cell}\begin{sphinxVerbatimInput}

\begin{sphinxuseclass}{cell_input}
\begin{sphinxVerbatim}[commandchars=\\\{\}]
\PYG{n}{df\PYGZus{}1}\PYG{p}{[}\PYG{l+s+s1}{\PYGZsq{}}\PYG{l+s+s1}{beta}\PYG{l+s+s1}{\PYGZsq{}}\PYG{p}{]}\PYG{o}{.}\PYG{n}{pipe}\PYG{p}{(}
    \PYG{n}{stats\PYGZus{}corr}\PYG{p}{,} 
    \PYG{n}{sub}\PYG{o}{=}\PYG{l+s+sa}{f}\PYG{l+s+s1}{\PYGZsq{}}\PYG{l+s+s1}{Betas of Daily Returns for }\PYG{l+s+si}{\PYGZob{}}\PYG{n}{time}\PYG{l+s+si}{\PYGZcb{}}\PYG{l+s+s1}{ and }\PYG{l+s+si}{\PYGZob{}}\PYG{n}{timep1}\PYG{l+s+si}{\PYGZcb{}}\PYG{l+s+s1}{\PYGZsq{}}
\PYG{p}{)}
\end{sphinxVerbatim}

\end{sphinxuseclass}\end{sphinxVerbatimInput}
\begin{sphinxVerbatimOutput}

\begin{sphinxuseclass}{cell_output}
\begin{sphinxVerbatim}[commandchars=\\\{\}]
\PYGZlt{}pandas.io.formats.style.Styler at 0x182027f5f70\PYGZgt{}
\end{sphinxVerbatim}

\end{sphinxuseclass}\end{sphinxVerbatimOutput}

\end{sphinxuseclass}

\subsubsection{Task 5: Does volatility in year \$t\$ predict \sphinxstyleemphasis{mean returns} in year \$t+1\$?}
\label{\detokenize{project_01_solution:task-5-does-volatility-in-year-t-predict-mean-returns-in-year-t-1}}
\begin{sphinxuseclass}{cell}\begin{sphinxVerbatimInput}

\begin{sphinxuseclass}{cell_input}
\begin{sphinxVerbatim}[commandchars=\\\{\}]
\PYG{p}{(}
    \PYG{n}{df\PYGZus{}1}
    \PYG{p}{[}\PYG{p}{[}\PYG{p}{(}\PYG{l+s+s1}{\PYGZsq{}}\PYG{l+s+s1}{std}\PYG{l+s+s1}{\PYGZsq{}}\PYG{p}{,} \PYG{n}{time}\PYG{p}{)}\PYG{p}{,} \PYG{p}{(}\PYG{l+s+s1}{\PYGZsq{}}\PYG{l+s+s1}{mean}\PYG{l+s+s1}{\PYGZsq{}}\PYG{p}{,} \PYG{n}{timep1}\PYG{p}{)}\PYG{p}{]}\PYG{p}{]}
    \PYG{o}{.}\PYG{n}{droplevel}\PYG{p}{(}\PYG{n}{level}\PYG{o}{=}\PYG{l+m+mi}{0}\PYG{p}{,} \PYG{n}{axis}\PYG{o}{=}\PYG{l+m+mi}{1}\PYG{p}{)}
    \PYG{o}{.}\PYG{n}{pipe}\PYG{p}{(}\PYG{n}{scatter}\PYG{p}{,} \PYG{n}{x}\PYG{o}{=}\PYG{n}{time}\PYG{p}{,} \PYG{n}{y}\PYG{o}{=}\PYG{n}{timep1}\PYG{p}{)}
\PYG{p}{)}
\PYG{n}{plt}\PYG{o}{.}\PYG{n}{xlabel}\PYG{p}{(}\PYG{l+s+sa}{f}\PYG{l+s+s1}{\PYGZsq{}}\PYG{l+s+s1}{Ann. Vol. of Daily Returns (\PYGZpc{}) for }\PYG{l+s+si}{\PYGZob{}}\PYG{n}{time}\PYG{l+s+si}{\PYGZcb{}}\PYG{l+s+s1}{\PYGZsq{}}\PYG{p}{)}
\PYG{n}{plt}\PYG{o}{.}\PYG{n}{ylabel}\PYG{p}{(}\PYG{l+s+sa}{f}\PYG{l+s+s1}{\PYGZsq{}}\PYG{l+s+s1}{Ann. Mean of Daily Returns (\PYGZpc{}) for }\PYG{l+s+si}{\PYGZob{}}\PYG{n}{timep1}\PYG{l+s+si}{\PYGZcb{}}\PYG{l+s+s1}{\PYGZsq{}}\PYG{p}{)}
\PYG{n}{plt}\PYG{o}{.}\PYG{n}{title}\PYG{p}{(}\PYG{l+s+s1}{\PYGZsq{}}\PYG{l+s+s1}{Return Predictability for Russell 1000 Stocks}\PYG{l+s+s1}{\PYGZsq{}}\PYG{p}{)}
\PYG{n}{plt}\PYG{o}{.}\PYG{n}{show}\PYG{p}{(}\PYG{p}{)}
\end{sphinxVerbatim}

\end{sphinxuseclass}\end{sphinxVerbatimInput}
\begin{sphinxVerbatimOutput}

\begin{sphinxuseclass}{cell_output}
\noindent\sphinxincludegraphics{{80612ca9a10798dc7f14f64e7f71af4b7fa050571db15f078fbc1409babd764d}.png}

\end{sphinxuseclass}\end{sphinxVerbatimOutput}

\end{sphinxuseclass}
\begin{sphinxuseclass}{cell}\begin{sphinxVerbatimInput}

\begin{sphinxuseclass}{cell_input}
\begin{sphinxVerbatim}[commandchars=\\\{\}]
\PYG{p}{(}
    \PYG{n}{df\PYGZus{}1}
    \PYG{p}{[}\PYG{p}{[}\PYG{p}{(}\PYG{l+s+s1}{\PYGZsq{}}\PYG{l+s+s1}{std}\PYG{l+s+s1}{\PYGZsq{}}\PYG{p}{,} \PYG{n}{time}\PYG{p}{)}\PYG{p}{,} \PYG{p}{(}\PYG{l+s+s1}{\PYGZsq{}}\PYG{l+s+s1}{mean}\PYG{l+s+s1}{\PYGZsq{}}\PYG{p}{,} \PYG{n}{timep1}\PYG{p}{)}\PYG{p}{]}\PYG{p}{]}
    \PYG{o}{.}\PYG{n}{droplevel}\PYG{p}{(}\PYG{n}{level}\PYG{o}{=}\PYG{l+m+mi}{0}\PYG{p}{,} \PYG{n}{axis}\PYG{o}{=}\PYG{l+m+mi}{1}\PYG{p}{)}
    \PYG{o}{.}\PYG{n}{pipe}\PYG{p}{(}
        \PYG{n}{stats\PYGZus{}corr}\PYG{p}{,} 
        \PYG{n}{sub}\PYG{o}{=}\PYG{l+s+sa}{f}\PYG{l+s+s1}{\PYGZsq{}}\PYG{l+s+s1}{Ann. Mean of Daily Returns (\PYGZpc{}) for }\PYG{l+s+si}{\PYGZob{}}\PYG{n}{time}\PYG{l+s+si}{\PYGZcb{}}\PYG{l+s+s1}{ and Ann. Vol. of Daily Returns (\PYGZpc{}) for }\PYG{l+s+si}{\PYGZob{}}\PYG{n}{timep1}\PYG{l+s+si}{\PYGZcb{}}\PYG{l+s+s1}{\PYGZsq{}}
    \PYG{p}{)}
\PYG{p}{)}
\end{sphinxVerbatim}

\end{sphinxuseclass}\end{sphinxVerbatimInput}
\begin{sphinxVerbatimOutput}

\begin{sphinxuseclass}{cell_output}
\begin{sphinxVerbatim}[commandchars=\\\{\}]
\PYGZlt{}pandas.io.formats.style.Styler at 0x18202909a30\PYGZgt{}
\end{sphinxVerbatim}

\end{sphinxuseclass}\end{sphinxVerbatimOutput}

\end{sphinxuseclass}

\subsubsection{Task 6: Does CAPM beta in year \$t\$ predict \sphinxstyleemphasis{mean returns} in year \$t+1\$?}
\label{\detokenize{project_01_solution:task-6-does-capm-beta-in-year-t-predict-mean-returns-in-year-t-1}}
\begin{sphinxuseclass}{cell}\begin{sphinxVerbatimInput}

\begin{sphinxuseclass}{cell_input}
\begin{sphinxVerbatim}[commandchars=\\\{\}]
\PYG{p}{(}
    \PYG{n}{df\PYGZus{}1}
    \PYG{p}{[}\PYG{p}{[}\PYG{p}{(}\PYG{l+s+s1}{\PYGZsq{}}\PYG{l+s+s1}{beta}\PYG{l+s+s1}{\PYGZsq{}}\PYG{p}{,} \PYG{n}{time}\PYG{p}{)}\PYG{p}{,} \PYG{p}{(}\PYG{l+s+s1}{\PYGZsq{}}\PYG{l+s+s1}{mean}\PYG{l+s+s1}{\PYGZsq{}}\PYG{p}{,} \PYG{n}{timep1}\PYG{p}{)}\PYG{p}{]}\PYG{p}{]}
    \PYG{o}{.}\PYG{n}{droplevel}\PYG{p}{(}\PYG{n}{level}\PYG{o}{=}\PYG{l+m+mi}{0}\PYG{p}{,} \PYG{n}{axis}\PYG{o}{=}\PYG{l+m+mi}{1}\PYG{p}{)}
    \PYG{o}{.}\PYG{n}{pipe}\PYG{p}{(}\PYG{n}{scatter}\PYG{p}{,} \PYG{n}{x}\PYG{o}{=}\PYG{n}{time}\PYG{p}{,} \PYG{n}{y}\PYG{o}{=}\PYG{n}{timep1}\PYG{p}{)}
\PYG{p}{)}
\PYG{n}{plt}\PYG{o}{.}\PYG{n}{xlabel}\PYG{p}{(}\PYG{l+s+sa}{f}\PYG{l+s+s1}{\PYGZsq{}}\PYG{l+s+s1}{Beta of Daily Returns for }\PYG{l+s+si}{\PYGZob{}}\PYG{n}{time}\PYG{l+s+si}{\PYGZcb{}}\PYG{l+s+s1}{\PYGZsq{}}\PYG{p}{)}
\PYG{n}{plt}\PYG{o}{.}\PYG{n}{ylabel}\PYG{p}{(}\PYG{l+s+sa}{f}\PYG{l+s+s1}{\PYGZsq{}}\PYG{l+s+s1}{Ann. Mean of Daily Returns (\PYGZpc{}) for }\PYG{l+s+si}{\PYGZob{}}\PYG{n}{timep1}\PYG{l+s+si}{\PYGZcb{}}\PYG{l+s+s1}{\PYGZsq{}}\PYG{p}{)}
\PYG{n}{plt}\PYG{o}{.}\PYG{n}{title}\PYG{p}{(}\PYG{l+s+s1}{\PYGZsq{}}\PYG{l+s+s1}{Return Predictability for Russell 1000 Stocks}\PYG{l+s+s1}{\PYGZsq{}}\PYG{p}{)}
\PYG{n}{plt}\PYG{o}{.}\PYG{n}{show}\PYG{p}{(}\PYG{p}{)}
\end{sphinxVerbatim}

\end{sphinxuseclass}\end{sphinxVerbatimInput}
\begin{sphinxVerbatimOutput}

\begin{sphinxuseclass}{cell_output}
\noindent\sphinxincludegraphics{{69919ea015c56e799bc9699a480a419dae146a583f0bb03b3eb7428b00dcb819}.png}

\end{sphinxuseclass}\end{sphinxVerbatimOutput}

\end{sphinxuseclass}
\begin{sphinxuseclass}{cell}\begin{sphinxVerbatimInput}

\begin{sphinxuseclass}{cell_input}
\begin{sphinxVerbatim}[commandchars=\\\{\}]
\PYG{p}{(}
    \PYG{n}{df\PYGZus{}1}
    \PYG{p}{[}\PYG{p}{[}\PYG{p}{(}\PYG{l+s+s1}{\PYGZsq{}}\PYG{l+s+s1}{beta}\PYG{l+s+s1}{\PYGZsq{}}\PYG{p}{,} \PYG{n}{time}\PYG{p}{)}\PYG{p}{,} \PYG{p}{(}\PYG{l+s+s1}{\PYGZsq{}}\PYG{l+s+s1}{mean}\PYG{l+s+s1}{\PYGZsq{}}\PYG{p}{,} \PYG{n}{timep1}\PYG{p}{)}\PYG{p}{]}\PYG{p}{]}
    \PYG{o}{.}\PYG{n}{droplevel}\PYG{p}{(}\PYG{n}{level}\PYG{o}{=}\PYG{l+m+mi}{0}\PYG{p}{,} \PYG{n}{axis}\PYG{o}{=}\PYG{l+m+mi}{1}\PYG{p}{)}
    \PYG{o}{.}\PYG{n}{pipe}\PYG{p}{(}
        \PYG{n}{stats\PYGZus{}corr}\PYG{p}{,} 
        \PYG{n}{sub}\PYG{o}{=}\PYG{l+s+sa}{f}\PYG{l+s+s1}{\PYGZsq{}}\PYG{l+s+s1}{Ann. Mean of Daily Returns (\PYGZpc{}) for }\PYG{l+s+si}{\PYGZob{}}\PYG{n}{time}\PYG{l+s+si}{\PYGZcb{}}\PYG{l+s+s1}{ and Beta of Daily Returns for }\PYG{l+s+si}{\PYGZob{}}\PYG{n}{timep1}\PYG{l+s+si}{\PYGZcb{}}\PYG{l+s+s1}{\PYGZsq{}}
    \PYG{p}{)}
\PYG{p}{)}
\end{sphinxVerbatim}

\end{sphinxuseclass}\end{sphinxVerbatimInput}
\begin{sphinxVerbatimOutput}

\begin{sphinxuseclass}{cell_output}
\begin{sphinxVerbatim}[commandchars=\\\{\}]
\PYGZlt{}pandas.io.formats.style.Styler at 0x182016ec6d0\PYGZgt{}
\end{sphinxVerbatim}

\end{sphinxuseclass}\end{sphinxVerbatimOutput}

\end{sphinxuseclass}

\subsection{Portfolios I}
\label{\detokenize{project_01_solution:portfolios-i}}
\sphinxAtStartPar
For this section, create 100 random portfolios of 50 stocks each from the daily returns in \sphinxcode{\sphinxupquote{returns}}.
Equally weight these portfolios and rebalance them daily.
Use the same stocks and years \$t\$ and \$t+1\$ as the previous section.


\subsubsection{Task 7: Does volatility in year \$t\$ predict \sphinxstyleemphasis{mean returns} in year \$t+1\$?}
\label{\detokenize{project_01_solution:task-7-does-volatility-in-year-t-predict-mean-returns-in-year-t-1}}
\begin{sphinxuseclass}{cell}\begin{sphinxVerbatimInput}

\begin{sphinxuseclass}{cell_input}
\begin{sphinxVerbatim}[commandchars=\\\{\}]
\PYG{n}{n\PYGZus{}stocks} \PYG{o}{=} \PYG{l+m+mi}{50}
\PYG{n}{n\PYGZus{}ports} \PYG{o}{=} \PYG{l+m+mi}{100}

\PYG{n}{df\PYGZus{}2} \PYG{o}{=} \PYG{p}{(}
    \PYG{n}{pd}\PYG{o}{.}\PYG{n}{concat}\PYG{p}{(}
        \PYG{p}{[}
            \PYG{n}{returns}
            \PYG{o}{.}\PYG{n}{loc}\PYG{p}{[}\PYG{n}{time}\PYG{p}{:}\PYG{n}{timep1}\PYG{p}{]}
            \PYG{o}{.}\PYG{n}{dropna}\PYG{p}{(}\PYG{n}{axis}\PYG{o}{=}\PYG{l+m+mi}{1}\PYG{p}{)}
            \PYG{o}{.}\PYG{n}{sample}\PYG{p}{(}\PYG{n}{n}\PYG{o}{=}\PYG{n}{n\PYGZus{}stocks}\PYG{p}{,} \PYG{n}{axis}\PYG{o}{=}\PYG{l+m+mi}{1}\PYG{p}{,} \PYG{n}{random\PYGZus{}state}\PYG{o}{=}\PYG{n}{i}\PYG{p}{)}
            \PYG{o}{.}\PYG{n}{mean}\PYG{p}{(}\PYG{n}{axis}\PYG{o}{=}\PYG{l+m+mi}{1}\PYG{p}{)} 
            \PYG{k}{for} \PYG{n}{i} 
            \PYG{o+ow}{in} \PYG{n+nb}{range}\PYG{p}{(}\PYG{n}{n\PYGZus{}ports}\PYG{p}{)}
        \PYG{p}{]}\PYG{p}{,} 
        \PYG{n}{axis}\PYG{o}{=}\PYG{l+m+mi}{1}
    \PYG{p}{)}
    \PYG{o}{.}\PYG{n}{pipe}\PYG{p}{(}\PYG{n}{stats\PYGZus{}all\PYGZus{}ann}\PYG{p}{)}
\PYG{p}{)}
\end{sphinxVerbatim}

\end{sphinxuseclass}\end{sphinxVerbatimInput}

\end{sphinxuseclass}
\begin{sphinxuseclass}{cell}\begin{sphinxVerbatimInput}

\begin{sphinxuseclass}{cell_input}
\begin{sphinxVerbatim}[commandchars=\\\{\}]
\PYG{p}{(}
    \PYG{n}{df\PYGZus{}2}
    \PYG{p}{[}\PYG{p}{[}\PYG{p}{(}\PYG{l+s+s1}{\PYGZsq{}}\PYG{l+s+s1}{std}\PYG{l+s+s1}{\PYGZsq{}}\PYG{p}{,} \PYG{n}{time}\PYG{p}{)}\PYG{p}{,} \PYG{p}{(}\PYG{l+s+s1}{\PYGZsq{}}\PYG{l+s+s1}{mean}\PYG{l+s+s1}{\PYGZsq{}}\PYG{p}{,} \PYG{n}{timep1}\PYG{p}{)}\PYG{p}{]}\PYG{p}{]}
    \PYG{o}{.}\PYG{n}{droplevel}\PYG{p}{(}\PYG{n}{level}\PYG{o}{=}\PYG{l+m+mi}{0}\PYG{p}{,} \PYG{n}{axis}\PYG{o}{=}\PYG{l+m+mi}{1}\PYG{p}{)}
    \PYG{o}{.}\PYG{n}{pipe}\PYG{p}{(}\PYG{n}{scatter}\PYG{p}{,} \PYG{n}{x}\PYG{o}{=}\PYG{n}{time}\PYG{p}{,} \PYG{n}{y}\PYG{o}{=}\PYG{n}{timep1}\PYG{p}{)}
\PYG{p}{)}
\PYG{n}{plt}\PYG{o}{.}\PYG{n}{xlabel}\PYG{p}{(}\PYG{l+s+sa}{f}\PYG{l+s+s1}{\PYGZsq{}}\PYG{l+s+s1}{Ann. Vol. of Daily Returns (\PYGZpc{}) for }\PYG{l+s+si}{\PYGZob{}}\PYG{n}{time}\PYG{l+s+si}{\PYGZcb{}}\PYG{l+s+s1}{\PYGZsq{}}\PYG{p}{)}
\PYG{n}{plt}\PYG{o}{.}\PYG{n}{ylabel}\PYG{p}{(}\PYG{l+s+sa}{f}\PYG{l+s+s1}{\PYGZsq{}}\PYG{l+s+s1}{Ann. Mean of Daily Returns (\PYGZpc{}) for }\PYG{l+s+si}{\PYGZob{}}\PYG{n}{timep1}\PYG{l+s+si}{\PYGZcb{}}\PYG{l+s+s1}{\PYGZsq{}}\PYG{p}{)}
\PYG{n}{plt}\PYG{o}{.}\PYG{n}{title}\PYG{p}{(}\PYG{l+s+sa}{f}\PYG{l+s+s1}{\PYGZsq{}}\PYG{l+s+s1}{Return Predictability for Russell 1000 Stocks}\PYG{l+s+se}{\PYGZbs{}n}\PYG{l+s+si}{\PYGZob{}}\PYG{n}{n\PYGZus{}ports}\PYG{l+s+si}{\PYGZcb{}}\PYG{l+s+s1}{ Random Portfolios of }\PYG{l+s+si}{\PYGZob{}}\PYG{n}{n\PYGZus{}stocks}\PYG{l+s+si}{\PYGZcb{}}\PYG{l+s+s1}{ Each}\PYG{l+s+s1}{\PYGZsq{}}\PYG{p}{)}
\PYG{n}{plt}\PYG{o}{.}\PYG{n}{show}\PYG{p}{(}\PYG{p}{)}
\end{sphinxVerbatim}

\end{sphinxuseclass}\end{sphinxVerbatimInput}
\begin{sphinxVerbatimOutput}

\begin{sphinxuseclass}{cell_output}
\noindent\sphinxincludegraphics{{573cc03f0d0adb79613f05982c26fcdf13c9cb89357ce1216dd77d518097a36b}.png}

\end{sphinxuseclass}\end{sphinxVerbatimOutput}

\end{sphinxuseclass}
\begin{sphinxuseclass}{cell}\begin{sphinxVerbatimInput}

\begin{sphinxuseclass}{cell_input}
\begin{sphinxVerbatim}[commandchars=\\\{\}]
\PYG{p}{(}
    \PYG{n}{df\PYGZus{}2}
    \PYG{p}{[}\PYG{p}{[}\PYG{p}{(}\PYG{l+s+s1}{\PYGZsq{}}\PYG{l+s+s1}{std}\PYG{l+s+s1}{\PYGZsq{}}\PYG{p}{,} \PYG{n}{time}\PYG{p}{)}\PYG{p}{,} \PYG{p}{(}\PYG{l+s+s1}{\PYGZsq{}}\PYG{l+s+s1}{mean}\PYG{l+s+s1}{\PYGZsq{}}\PYG{p}{,} \PYG{n}{timep1}\PYG{p}{)}\PYG{p}{]}\PYG{p}{]}
    \PYG{o}{.}\PYG{n}{droplevel}\PYG{p}{(}\PYG{n}{level}\PYG{o}{=}\PYG{l+m+mi}{0}\PYG{p}{,} \PYG{n}{axis}\PYG{o}{=}\PYG{l+m+mi}{1}\PYG{p}{)}
    \PYG{o}{.}\PYG{n}{pipe}\PYG{p}{(}
        \PYG{n}{stats\PYGZus{}corr}\PYG{p}{,} 
        \PYG{n}{sub}\PYG{o}{=}\PYG{l+s+sa}{f}\PYG{l+s+s1}{\PYGZsq{}}\PYG{l+s+s1}{Ann. Mean of Daily Returns (\PYGZpc{}) for }\PYG{l+s+si}{\PYGZob{}}\PYG{n}{time}\PYG{l+s+si}{\PYGZcb{}}\PYG{l+s+s1}{ and Ann. Vol. of Daily Returns (\PYGZpc{}) for }\PYG{l+s+si}{\PYGZob{}}\PYG{n}{timep1}\PYG{l+s+si}{\PYGZcb{}}\PYG{l+s+s1}{\PYGZsq{}}
    \PYG{p}{)}
\PYG{p}{)}
\end{sphinxVerbatim}

\end{sphinxuseclass}\end{sphinxVerbatimInput}
\begin{sphinxVerbatimOutput}

\begin{sphinxuseclass}{cell_output}
\begin{sphinxVerbatim}[commandchars=\\\{\}]
\PYGZlt{}pandas.io.formats.style.Styler at 0x182028307f0\PYGZgt{}
\end{sphinxVerbatim}

\end{sphinxuseclass}\end{sphinxVerbatimOutput}

\end{sphinxuseclass}

\paragraph{Comparison of Single Stocks and Portfolios}
\label{\detokenize{project_01_solution:comparison-of-single-stocks-and-portfolios}}
\begin{sphinxuseclass}{cell}\begin{sphinxVerbatimInput}

\begin{sphinxuseclass}{cell_input}
\begin{sphinxVerbatim}[commandchars=\\\{\}]
\PYG{n}{fig}\PYG{p}{,} \PYG{n}{ax} \PYG{o}{=} \PYG{n}{plt}\PYG{o}{.}\PYG{n}{subplots}\PYG{p}{(}\PYG{l+m+mi}{2}\PYG{p}{,} \PYG{l+m+mi}{1}\PYG{p}{,} \PYG{n}{sharex}\PYG{o}{=}\PYG{k+kc}{True}\PYG{p}{,} \PYG{n}{sharey}\PYG{o}{=}\PYG{k+kc}{True}\PYG{p}{)}
\PYG{n}{df\PYGZus{}1}\PYG{p}{[}\PYG{p}{[}\PYG{p}{(}\PYG{l+s+s1}{\PYGZsq{}}\PYG{l+s+s1}{std}\PYG{l+s+s1}{\PYGZsq{}}\PYG{p}{,} \PYG{n}{time}\PYG{p}{)}\PYG{p}{,} \PYG{p}{(}\PYG{l+s+s1}{\PYGZsq{}}\PYG{l+s+s1}{mean}\PYG{l+s+s1}{\PYGZsq{}}\PYG{p}{,} \PYG{n}{timep1}\PYG{p}{)}\PYG{p}{]}\PYG{p}{]}\PYG{o}{.}\PYG{n}{droplevel}\PYG{p}{(}\PYG{n}{level}\PYG{o}{=}\PYG{l+m+mi}{0}\PYG{p}{,} \PYG{n}{axis}\PYG{o}{=}\PYG{l+m+mi}{1}\PYG{p}{)}\PYG{o}{.}\PYG{n}{pipe}\PYG{p}{(}\PYG{n}{scatter}\PYG{p}{,} \PYG{n}{x}\PYG{o}{=}\PYG{n}{time}\PYG{p}{,} \PYG{n}{y}\PYG{o}{=}\PYG{n}{timep1}\PYG{p}{,} \PYG{n}{ax}\PYG{o}{=}\PYG{n}{ax}\PYG{p}{[}\PYG{l+m+mi}{0}\PYG{p}{]}\PYG{p}{)}
\PYG{n}{ax}\PYG{p}{[}\PYG{l+m+mi}{0}\PYG{p}{]}\PYG{o}{.}\PYG{n}{set\PYGZus{}title}\PYG{p}{(}\PYG{l+s+s1}{\PYGZsq{}}\PYG{l+s+s1}{Single Stocks}\PYG{l+s+s1}{\PYGZsq{}}\PYG{p}{)}
\PYG{n}{ax}\PYG{p}{[}\PYG{l+m+mi}{0}\PYG{p}{]}\PYG{o}{.}\PYG{n}{set\PYGZus{}xlabel}\PYG{p}{(}\PYG{l+s+sa}{f}\PYG{l+s+s1}{\PYGZsq{}}\PYG{l+s+s1}{Ann. Vol. of Daily Returns (\PYGZpc{}) for }\PYG{l+s+si}{\PYGZob{}}\PYG{n}{time}\PYG{l+s+si}{\PYGZcb{}}\PYG{l+s+s1}{\PYGZsq{}}\PYG{p}{)}
\PYG{n}{ax}\PYG{p}{[}\PYG{l+m+mi}{0}\PYG{p}{]}\PYG{o}{.}\PYG{n}{set\PYGZus{}ylabel}\PYG{p}{(}\PYG{l+s+sa}{f}\PYG{l+s+s1}{\PYGZsq{}}\PYG{l+s+s1}{Ann. Mean of}\PYG{l+s+se}{\PYGZbs{}n}\PYG{l+s+s1}{ Daily Returns (\PYGZpc{})}\PYG{l+s+se}{\PYGZbs{}n}\PYG{l+s+s1}{ for }\PYG{l+s+si}{\PYGZob{}}\PYG{n}{timep1}\PYG{l+s+si}{\PYGZcb{}}\PYG{l+s+s1}{\PYGZsq{}}\PYG{p}{)}
\PYG{n}{df\PYGZus{}2}\PYG{p}{[}\PYG{p}{[}\PYG{p}{(}\PYG{l+s+s1}{\PYGZsq{}}\PYG{l+s+s1}{std}\PYG{l+s+s1}{\PYGZsq{}}\PYG{p}{,} \PYG{n}{time}\PYG{p}{)}\PYG{p}{,} \PYG{p}{(}\PYG{l+s+s1}{\PYGZsq{}}\PYG{l+s+s1}{mean}\PYG{l+s+s1}{\PYGZsq{}}\PYG{p}{,} \PYG{n}{timep1}\PYG{p}{)}\PYG{p}{]}\PYG{p}{]}\PYG{o}{.}\PYG{n}{droplevel}\PYG{p}{(}\PYG{n}{level}\PYG{o}{=}\PYG{l+m+mi}{0}\PYG{p}{,} \PYG{n}{axis}\PYG{o}{=}\PYG{l+m+mi}{1}\PYG{p}{)}\PYG{o}{.}\PYG{n}{pipe}\PYG{p}{(}\PYG{n}{scatter}\PYG{p}{,} \PYG{n}{x}\PYG{o}{=}\PYG{n}{time}\PYG{p}{,} \PYG{n}{y}\PYG{o}{=}\PYG{n}{timep1}\PYG{p}{,} \PYG{n}{ax}\PYG{o}{=}\PYG{n}{ax}\PYG{p}{[}\PYG{l+m+mi}{1}\PYG{p}{]}\PYG{p}{)}
\PYG{n}{ax}\PYG{p}{[}\PYG{l+m+mi}{1}\PYG{p}{]}\PYG{o}{.}\PYG{n}{set\PYGZus{}title}\PYG{p}{(}\PYG{l+s+sa}{f}\PYG{l+s+s1}{\PYGZsq{}}\PYG{l+s+si}{\PYGZob{}}\PYG{n}{n\PYGZus{}ports}\PYG{l+s+si}{\PYGZcb{}}\PYG{l+s+s1}{ Portfolios of }\PYG{l+s+si}{\PYGZob{}}\PYG{n}{n\PYGZus{}stocks}\PYG{l+s+si}{\PYGZcb{}}\PYG{l+s+s1}{ Each}\PYG{l+s+s1}{\PYGZsq{}}\PYG{p}{)}
\PYG{n}{plt}\PYG{o}{.}\PYG{n}{suptitle}\PYG{p}{(}\PYG{l+s+s1}{\PYGZsq{}}\PYG{l+s+s1}{Return Predictability for Russell 1000 Stocks}\PYG{l+s+s1}{\PYGZsq{}}\PYG{p}{)}
\PYG{n}{ax}\PYG{p}{[}\PYG{l+m+mi}{1}\PYG{p}{]}\PYG{o}{.}\PYG{n}{set\PYGZus{}xlabel}\PYG{p}{(}\PYG{l+s+sa}{f}\PYG{l+s+s1}{\PYGZsq{}}\PYG{l+s+s1}{Ann. Vol. of Daily Returns (\PYGZpc{}) for }\PYG{l+s+si}{\PYGZob{}}\PYG{n}{time}\PYG{l+s+si}{\PYGZcb{}}\PYG{l+s+s1}{\PYGZsq{}}\PYG{p}{)}
\PYG{n}{ax}\PYG{p}{[}\PYG{l+m+mi}{1}\PYG{p}{]}\PYG{o}{.}\PYG{n}{set\PYGZus{}ylabel}\PYG{p}{(}\PYG{l+s+sa}{f}\PYG{l+s+s1}{\PYGZsq{}}\PYG{l+s+s1}{Ann. Mean of}\PYG{l+s+se}{\PYGZbs{}n}\PYG{l+s+s1}{ Daily Returns (\PYGZpc{})}\PYG{l+s+se}{\PYGZbs{}n}\PYG{l+s+s1}{ for }\PYG{l+s+si}{\PYGZob{}}\PYG{n}{timep1}\PYG{l+s+si}{\PYGZcb{}}\PYG{l+s+s1}{\PYGZsq{}}\PYG{p}{)}
\PYG{n}{plt}\PYG{o}{.}\PYG{n}{tight\PYGZus{}layout}\PYG{p}{(}\PYG{p}{)}
\PYG{n}{plt}\PYG{o}{.}\PYG{n}{show}\PYG{p}{(}\PYG{p}{)}
\end{sphinxVerbatim}

\end{sphinxuseclass}\end{sphinxVerbatimInput}
\begin{sphinxVerbatimOutput}

\begin{sphinxuseclass}{cell_output}
\noindent\sphinxincludegraphics{{934e9f1d711fa77c55dbb84c8cf3cf387177d54acf9e77d7110feb16fdb1a8a2}.png}

\end{sphinxuseclass}\end{sphinxVerbatimOutput}

\end{sphinxuseclass}

\subsubsection{Task 8: Does CAPM beta in year \$t\$ predict \sphinxstyleemphasis{mean returns} in year \$t+1\$?}
\label{\detokenize{project_01_solution:task-8-does-capm-beta-in-year-t-predict-mean-returns-in-year-t-1}}
\begin{sphinxuseclass}{cell}\begin{sphinxVerbatimInput}

\begin{sphinxuseclass}{cell_input}
\begin{sphinxVerbatim}[commandchars=\\\{\}]
\PYG{p}{(}
    \PYG{n}{df\PYGZus{}2}
    \PYG{p}{[}\PYG{p}{[}\PYG{p}{(}\PYG{l+s+s1}{\PYGZsq{}}\PYG{l+s+s1}{beta}\PYG{l+s+s1}{\PYGZsq{}}\PYG{p}{,} \PYG{n}{time}\PYG{p}{)}\PYG{p}{,} \PYG{p}{(}\PYG{l+s+s1}{\PYGZsq{}}\PYG{l+s+s1}{mean}\PYG{l+s+s1}{\PYGZsq{}}\PYG{p}{,} \PYG{n}{timep1}\PYG{p}{)}\PYG{p}{]}\PYG{p}{]}
    \PYG{o}{.}\PYG{n}{droplevel}\PYG{p}{(}\PYG{n}{level}\PYG{o}{=}\PYG{l+m+mi}{0}\PYG{p}{,} \PYG{n}{axis}\PYG{o}{=}\PYG{l+m+mi}{1}\PYG{p}{)}
    \PYG{o}{.}\PYG{n}{pipe}\PYG{p}{(}\PYG{n}{scatter}\PYG{p}{,} \PYG{n}{x}\PYG{o}{=}\PYG{n}{time}\PYG{p}{,} \PYG{n}{y}\PYG{o}{=}\PYG{n}{timep1}\PYG{p}{)}
\PYG{p}{)}
\PYG{n}{plt}\PYG{o}{.}\PYG{n}{xlabel}\PYG{p}{(}\PYG{l+s+sa}{f}\PYG{l+s+s1}{\PYGZsq{}}\PYG{l+s+s1}{Beta of Daily Returns for }\PYG{l+s+si}{\PYGZob{}}\PYG{n}{time}\PYG{l+s+si}{\PYGZcb{}}\PYG{l+s+s1}{\PYGZsq{}}\PYG{p}{)}
\PYG{n}{plt}\PYG{o}{.}\PYG{n}{ylabel}\PYG{p}{(}\PYG{l+s+sa}{f}\PYG{l+s+s1}{\PYGZsq{}}\PYG{l+s+s1}{Ann. Mean of Daily Returns (\PYGZpc{}) for }\PYG{l+s+si}{\PYGZob{}}\PYG{n}{timep1}\PYG{l+s+si}{\PYGZcb{}}\PYG{l+s+s1}{\PYGZsq{}}\PYG{p}{)}
\PYG{n}{plt}\PYG{o}{.}\PYG{n}{title}\PYG{p}{(}\PYG{l+s+sa}{f}\PYG{l+s+s1}{\PYGZsq{}}\PYG{l+s+s1}{Return Predictability for Russell 1000 Stocks}\PYG{l+s+se}{\PYGZbs{}n}\PYG{l+s+si}{\PYGZob{}}\PYG{n}{n\PYGZus{}ports}\PYG{l+s+si}{\PYGZcb{}}\PYG{l+s+s1}{ Random Portfolios of }\PYG{l+s+si}{\PYGZob{}}\PYG{n}{n\PYGZus{}stocks}\PYG{l+s+si}{\PYGZcb{}}\PYG{l+s+s1}{ Each}\PYG{l+s+s1}{\PYGZsq{}}\PYG{p}{)}
\PYG{n}{plt}\PYG{o}{.}\PYG{n}{show}\PYG{p}{(}\PYG{p}{)}
\end{sphinxVerbatim}

\end{sphinxuseclass}\end{sphinxVerbatimInput}
\begin{sphinxVerbatimOutput}

\begin{sphinxuseclass}{cell_output}
\noindent\sphinxincludegraphics{{22203be4a07e24718f24fd49bbdf15058a9a5f5e4a3a23ade12389af335898ef}.png}

\end{sphinxuseclass}\end{sphinxVerbatimOutput}

\end{sphinxuseclass}
\begin{sphinxuseclass}{cell}\begin{sphinxVerbatimInput}

\begin{sphinxuseclass}{cell_input}
\begin{sphinxVerbatim}[commandchars=\\\{\}]
\PYG{p}{(}
    \PYG{n}{df\PYGZus{}2}
    \PYG{p}{[}\PYG{p}{[}\PYG{p}{(}\PYG{l+s+s1}{\PYGZsq{}}\PYG{l+s+s1}{beta}\PYG{l+s+s1}{\PYGZsq{}}\PYG{p}{,} \PYG{n}{time}\PYG{p}{)}\PYG{p}{,} \PYG{p}{(}\PYG{l+s+s1}{\PYGZsq{}}\PYG{l+s+s1}{mean}\PYG{l+s+s1}{\PYGZsq{}}\PYG{p}{,} \PYG{n}{timep1}\PYG{p}{)}\PYG{p}{]}\PYG{p}{]}
    \PYG{o}{.}\PYG{n}{droplevel}\PYG{p}{(}\PYG{n}{level}\PYG{o}{=}\PYG{l+m+mi}{0}\PYG{p}{,} \PYG{n}{axis}\PYG{o}{=}\PYG{l+m+mi}{1}\PYG{p}{)}
    \PYG{o}{.}\PYG{n}{pipe}\PYG{p}{(}
        \PYG{n}{stats\PYGZus{}corr}\PYG{p}{,} 
        \PYG{n}{sub}\PYG{o}{=}\PYG{l+s+sa}{f}\PYG{l+s+s1}{\PYGZsq{}}\PYG{l+s+s1}{Ann. Means of Daily Returns (\PYGZpc{}) for }\PYG{l+s+si}{\PYGZob{}}\PYG{n}{timep1}\PYG{l+s+si}{\PYGZcb{}}\PYG{l+s+s1}{ and Beta of Daily Returns for }\PYG{l+s+si}{\PYGZob{}}\PYG{n}{time}\PYG{l+s+si}{\PYGZcb{}}\PYG{l+s+s1}{\PYGZsq{}}
    \PYG{p}{)}
\PYG{p}{)}
\end{sphinxVerbatim}

\end{sphinxuseclass}\end{sphinxVerbatimInput}
\begin{sphinxVerbatimOutput}

\begin{sphinxuseclass}{cell_output}
\begin{sphinxVerbatim}[commandchars=\\\{\}]
\PYGZlt{}pandas.io.formats.style.Styler at 0x1820172ec10\PYGZgt{}
\end{sphinxVerbatim}

\end{sphinxuseclass}\end{sphinxVerbatimOutput}

\end{sphinxuseclass}

\paragraph{Comparison of Single Stocks and Portfolios}
\label{\detokenize{project_01_solution:id1}}
\begin{sphinxuseclass}{cell}\begin{sphinxVerbatimInput}

\begin{sphinxuseclass}{cell_input}
\begin{sphinxVerbatim}[commandchars=\\\{\}]
\PYG{n}{fig}\PYG{p}{,} \PYG{n}{ax} \PYG{o}{=} \PYG{n}{plt}\PYG{o}{.}\PYG{n}{subplots}\PYG{p}{(}\PYG{l+m+mi}{2}\PYG{p}{,} \PYG{l+m+mi}{1}\PYG{p}{,} \PYG{n}{sharex}\PYG{o}{=}\PYG{k+kc}{True}\PYG{p}{,} \PYG{n}{sharey}\PYG{o}{=}\PYG{k+kc}{True}\PYG{p}{)}
\PYG{n}{df\PYGZus{}1}\PYG{p}{[}\PYG{p}{[}\PYG{p}{(}\PYG{l+s+s1}{\PYGZsq{}}\PYG{l+s+s1}{beta}\PYG{l+s+s1}{\PYGZsq{}}\PYG{p}{,} \PYG{n}{time}\PYG{p}{)}\PYG{p}{,} \PYG{p}{(}\PYG{l+s+s1}{\PYGZsq{}}\PYG{l+s+s1}{mean}\PYG{l+s+s1}{\PYGZsq{}}\PYG{p}{,} \PYG{n}{timep1}\PYG{p}{)}\PYG{p}{]}\PYG{p}{]}\PYG{o}{.}\PYG{n}{droplevel}\PYG{p}{(}\PYG{n}{level}\PYG{o}{=}\PYG{l+m+mi}{0}\PYG{p}{,} \PYG{n}{axis}\PYG{o}{=}\PYG{l+m+mi}{1}\PYG{p}{)}\PYG{o}{.}\PYG{n}{pipe}\PYG{p}{(}\PYG{n}{scatter}\PYG{p}{,} \PYG{n}{x}\PYG{o}{=}\PYG{n}{time}\PYG{p}{,} \PYG{n}{y}\PYG{o}{=}\PYG{n}{timep1}\PYG{p}{,} \PYG{n}{ax}\PYG{o}{=}\PYG{n}{ax}\PYG{p}{[}\PYG{l+m+mi}{0}\PYG{p}{]}\PYG{p}{)}
\PYG{n}{ax}\PYG{p}{[}\PYG{l+m+mi}{0}\PYG{p}{]}\PYG{o}{.}\PYG{n}{set\PYGZus{}title}\PYG{p}{(}\PYG{l+s+s1}{\PYGZsq{}}\PYG{l+s+s1}{Single Stocks}\PYG{l+s+s1}{\PYGZsq{}}\PYG{p}{)}
\PYG{n}{ax}\PYG{p}{[}\PYG{l+m+mi}{0}\PYG{p}{]}\PYG{o}{.}\PYG{n}{set\PYGZus{}xlabel}\PYG{p}{(}\PYG{l+s+sa}{f}\PYG{l+s+s1}{\PYGZsq{}}\PYG{l+s+s1}{Beta of Daily Returns (\PYGZpc{}) for }\PYG{l+s+si}{\PYGZob{}}\PYG{n}{time}\PYG{l+s+si}{\PYGZcb{}}\PYG{l+s+s1}{\PYGZsq{}}\PYG{p}{)}
\PYG{n}{ax}\PYG{p}{[}\PYG{l+m+mi}{0}\PYG{p}{]}\PYG{o}{.}\PYG{n}{set\PYGZus{}ylabel}\PYG{p}{(}\PYG{l+s+sa}{f}\PYG{l+s+s1}{\PYGZsq{}}\PYG{l+s+s1}{Ann. Mean of}\PYG{l+s+se}{\PYGZbs{}n}\PYG{l+s+s1}{ Daily Returns (\PYGZpc{})}\PYG{l+s+se}{\PYGZbs{}n}\PYG{l+s+s1}{ for }\PYG{l+s+si}{\PYGZob{}}\PYG{n}{timep1}\PYG{l+s+si}{\PYGZcb{}}\PYG{l+s+s1}{\PYGZsq{}}\PYG{p}{)}
\PYG{n}{df\PYGZus{}2}\PYG{p}{[}\PYG{p}{[}\PYG{p}{(}\PYG{l+s+s1}{\PYGZsq{}}\PYG{l+s+s1}{beta}\PYG{l+s+s1}{\PYGZsq{}}\PYG{p}{,} \PYG{n}{time}\PYG{p}{)}\PYG{p}{,} \PYG{p}{(}\PYG{l+s+s1}{\PYGZsq{}}\PYG{l+s+s1}{mean}\PYG{l+s+s1}{\PYGZsq{}}\PYG{p}{,} \PYG{n}{timep1}\PYG{p}{)}\PYG{p}{]}\PYG{p}{]}\PYG{o}{.}\PYG{n}{droplevel}\PYG{p}{(}\PYG{n}{level}\PYG{o}{=}\PYG{l+m+mi}{0}\PYG{p}{,} \PYG{n}{axis}\PYG{o}{=}\PYG{l+m+mi}{1}\PYG{p}{)}\PYG{o}{.}\PYG{n}{pipe}\PYG{p}{(}\PYG{n}{scatter}\PYG{p}{,} \PYG{n}{x}\PYG{o}{=}\PYG{n}{time}\PYG{p}{,} \PYG{n}{y}\PYG{o}{=}\PYG{n}{timep1}\PYG{p}{,} \PYG{n}{ax}\PYG{o}{=}\PYG{n}{ax}\PYG{p}{[}\PYG{l+m+mi}{1}\PYG{p}{]}\PYG{p}{)}
\PYG{n}{ax}\PYG{p}{[}\PYG{l+m+mi}{1}\PYG{p}{]}\PYG{o}{.}\PYG{n}{set\PYGZus{}title}\PYG{p}{(}\PYG{l+s+sa}{f}\PYG{l+s+s1}{\PYGZsq{}}\PYG{l+s+si}{\PYGZob{}}\PYG{n}{n\PYGZus{}ports}\PYG{l+s+si}{\PYGZcb{}}\PYG{l+s+s1}{ Portfolios of }\PYG{l+s+si}{\PYGZob{}}\PYG{n}{n\PYGZus{}stocks}\PYG{l+s+si}{\PYGZcb{}}\PYG{l+s+s1}{ Each}\PYG{l+s+s1}{\PYGZsq{}}\PYG{p}{)}
\PYG{n}{plt}\PYG{o}{.}\PYG{n}{suptitle}\PYG{p}{(}\PYG{l+s+s1}{\PYGZsq{}}\PYG{l+s+s1}{Return Predictability for Russell 1000 Stocks}\PYG{l+s+s1}{\PYGZsq{}}\PYG{p}{)}
\PYG{n}{ax}\PYG{p}{[}\PYG{l+m+mi}{1}\PYG{p}{]}\PYG{o}{.}\PYG{n}{set\PYGZus{}xlabel}\PYG{p}{(}\PYG{l+s+sa}{f}\PYG{l+s+s1}{\PYGZsq{}}\PYG{l+s+s1}{Beta of Daily Returns (\PYGZpc{}) for }\PYG{l+s+si}{\PYGZob{}}\PYG{n}{time}\PYG{l+s+si}{\PYGZcb{}}\PYG{l+s+s1}{\PYGZsq{}}\PYG{p}{)}
\PYG{n}{ax}\PYG{p}{[}\PYG{l+m+mi}{1}\PYG{p}{]}\PYG{o}{.}\PYG{n}{set\PYGZus{}ylabel}\PYG{p}{(}\PYG{l+s+sa}{f}\PYG{l+s+s1}{\PYGZsq{}}\PYG{l+s+s1}{Ann. Mean of}\PYG{l+s+se}{\PYGZbs{}n}\PYG{l+s+s1}{ Daily Returns (\PYGZpc{})}\PYG{l+s+se}{\PYGZbs{}n}\PYG{l+s+s1}{ for }\PYG{l+s+si}{\PYGZob{}}\PYG{n}{timep1}\PYG{l+s+si}{\PYGZcb{}}\PYG{l+s+s1}{\PYGZsq{}}\PYG{p}{)}
\PYG{n}{plt}\PYG{o}{.}\PYG{n}{tight\PYGZus{}layout}\PYG{p}{(}\PYG{p}{)}
\PYG{n}{plt}\PYG{o}{.}\PYG{n}{show}\PYG{p}{(}\PYG{p}{)}
\end{sphinxVerbatim}

\end{sphinxuseclass}\end{sphinxVerbatimInput}
\begin{sphinxVerbatimOutput}

\begin{sphinxuseclass}{cell_output}
\noindent\sphinxincludegraphics{{6865e2c806fa782497cce539b1c0e31c90a652d5068966c826743b8b8d650b3f}.png}

\end{sphinxuseclass}\end{sphinxVerbatimOutput}

\end{sphinxuseclass}

\subsection{Portfolios II}
\label{\detokenize{project_01_solution:portfolios-ii}}
\sphinxAtStartPar
Calculate monthly volatility and total return for \sphinxstyleemphasis{every stock} and \sphinxstyleemphasis{every month} in \sphinxcode{\sphinxupquote{returns}}.
Drop stock\sphinxhyphen{}months with fewer than 15 returns.
Each month, assign these stocks to one of five portfolios based on their volatility during the previous month.
Equally weight these portfolios and rebalance them monthly.


\subsubsection{Task 9: Do high volatility portfolios have high mean returns and Sharpe Ratios?}
\label{\detokenize{project_01_solution:task-9-do-high-volatility-portfolios-have-high-mean-returns-and-sharpe-ratios}}
\begin{sphinxuseclass}{cell}\begin{sphinxVerbatimInput}

\begin{sphinxuseclass}{cell_input}
\begin{sphinxVerbatim}[commandchars=\\\{\}]
\PYG{n}{df\PYGZus{}3} \PYG{o}{=} \PYG{p}{(}
    \PYG{n}{pd}\PYG{o}{.}\PYG{n}{concat}\PYG{p}{(}
        \PYG{p}{[}
            \PYG{n}{returns}\PYG{o}{.}\PYG{n}{resample}\PYG{p}{(}\PYG{l+s+s1}{\PYGZsq{}}\PYG{l+s+s1}{M}\PYG{l+s+s1}{\PYGZsq{}}\PYG{p}{,} \PYG{n}{kind}\PYG{o}{=}\PYG{l+s+s1}{\PYGZsq{}}\PYG{l+s+s1}{period}\PYG{l+s+s1}{\PYGZsq{}}\PYG{p}{)}\PYG{o}{.}\PYG{n}{std}\PYG{p}{(}\PYG{p}{)}\PYG{o}{.}\PYG{n}{shift}\PYG{p}{(}\PYG{p}{)}\PYG{p}{,}
            \PYG{n}{returns}\PYG{o}{.}\PYG{n}{resample}\PYG{p}{(}\PYG{l+s+s1}{\PYGZsq{}}\PYG{l+s+s1}{M}\PYG{l+s+s1}{\PYGZsq{}}\PYG{p}{,} \PYG{n}{kind}\PYG{o}{=}\PYG{l+s+s1}{\PYGZsq{}}\PYG{l+s+s1}{period}\PYG{l+s+s1}{\PYGZsq{}}\PYG{p}{)}\PYG{o}{.}\PYG{n}{count}\PYG{p}{(}\PYG{p}{)}\PYG{o}{.}\PYG{n}{shift}\PYG{p}{(}\PYG{p}{)}\PYG{p}{,}
            \PYG{n}{returns}\PYG{o}{.}\PYG{n}{add}\PYG{p}{(}\PYG{l+m+mi}{1}\PYG{p}{)}\PYG{o}{.}\PYG{n}{resample}\PYG{p}{(}\PYG{l+s+s1}{\PYGZsq{}}\PYG{l+s+s1}{M}\PYG{l+s+s1}{\PYGZsq{}}\PYG{p}{,} \PYG{n}{kind}\PYG{o}{=}\PYG{l+s+s1}{\PYGZsq{}}\PYG{l+s+s1}{period}\PYG{l+s+s1}{\PYGZsq{}}\PYG{p}{)}\PYG{o}{.}\PYG{n}{prod}\PYG{p}{(}\PYG{p}{)}\PYG{o}{.}\PYG{n}{sub}\PYG{p}{(}\PYG{l+m+mi}{1}\PYG{p}{)}\PYG{p}{,}
            \PYG{n}{returns}\PYG{o}{.}\PYG{n}{resample}\PYG{p}{(}\PYG{l+s+s1}{\PYGZsq{}}\PYG{l+s+s1}{M}\PYG{l+s+s1}{\PYGZsq{}}\PYG{p}{,} \PYG{n}{kind}\PYG{o}{=}\PYG{l+s+s1}{\PYGZsq{}}\PYG{l+s+s1}{period}\PYG{l+s+s1}{\PYGZsq{}}\PYG{p}{)}\PYG{o}{.}\PYG{n}{count}\PYG{p}{(}\PYG{p}{)}
        \PYG{p}{]}\PYG{p}{,} 
        \PYG{n}{keys}\PYG{o}{=}\PYG{p}{[}\PYG{l+s+s1}{\PYGZsq{}}\PYG{l+s+s1}{std\PYGZus{}lag}\PYG{l+s+s1}{\PYGZsq{}}\PYG{p}{,} \PYG{l+s+s1}{\PYGZsq{}}\PYG{l+s+s1}{count\PYGZus{}lag}\PYG{l+s+s1}{\PYGZsq{}}\PYG{p}{,} \PYG{l+s+s1}{\PYGZsq{}}\PYG{l+s+s1}{ret}\PYG{l+s+s1}{\PYGZsq{}}\PYG{p}{,} \PYG{l+s+s1}{\PYGZsq{}}\PYG{l+s+s1}{count}\PYG{l+s+s1}{\PYGZsq{}}\PYG{p}{]}\PYG{p}{,} 
        \PYG{n}{names}\PYG{o}{=}\PYG{p}{[}\PYG{l+s+s1}{\PYGZsq{}}\PYG{l+s+s1}{Variable}\PYG{l+s+s1}{\PYGZsq{}}\PYG{p}{,} \PYG{l+s+s1}{\PYGZsq{}}\PYG{l+s+s1}{Ticker}\PYG{l+s+s1}{\PYGZsq{}}\PYG{p}{]}\PYG{p}{,} 
        \PYG{n}{axis}\PYG{o}{=}\PYG{l+m+mi}{1}
    \PYG{p}{)}
    \PYG{o}{.}\PYG{n}{stack}\PYG{p}{(}\PYG{p}{)}
    \PYG{o}{.}\PYG{n}{query}\PYG{p}{(}\PYG{l+s+s1}{\PYGZsq{}}\PYG{l+s+s1}{(count\PYGZus{}lag \PYGZgt{}= 15) \PYGZam{} (count \PYGZgt{}= 15)}\PYG{l+s+s1}{\PYGZsq{}}\PYG{p}{)}
    \PYG{o}{.}\PYG{n}{assign}\PYG{p}{(}\PYG{n}{Portfolio} \PYG{o}{=} \PYG{k}{lambda} \PYG{n}{x}\PYG{p}{:} \PYG{n}{x}\PYG{o}{.}\PYG{n}{groupby}\PYG{p}{(}\PYG{l+s+s1}{\PYGZsq{}}\PYG{l+s+s1}{Date}\PYG{l+s+s1}{\PYGZsq{}}\PYG{p}{)}\PYG{p}{[}\PYG{l+s+s1}{\PYGZsq{}}\PYG{l+s+s1}{std\PYGZus{}lag}\PYG{l+s+s1}{\PYGZsq{}}\PYG{p}{]}\PYG{o}{.}\PYG{n}{transform}\PYG{p}{(}\PYG{n}{pd}\PYG{o}{.}\PYG{n}{qcut}\PYG{p}{,} \PYG{n}{q}\PYG{o}{=}\PYG{l+m+mi}{5}\PYG{p}{,} \PYG{n}{labels}\PYG{o}{=}\PYG{k+kc}{False}\PYG{p}{)}\PYG{p}{)}
    \PYG{o}{.}\PYG{n}{groupby}\PYG{p}{(}\PYG{p}{[}\PYG{l+s+s1}{\PYGZsq{}}\PYG{l+s+s1}{Date}\PYG{l+s+s1}{\PYGZsq{}}\PYG{p}{,} \PYG{l+s+s1}{\PYGZsq{}}\PYG{l+s+s1}{Portfolio}\PYG{l+s+s1}{\PYGZsq{}}\PYG{p}{]}\PYG{p}{)}
    \PYG{p}{[}\PYG{l+s+s1}{\PYGZsq{}}\PYG{l+s+s1}{ret}\PYG{l+s+s1}{\PYGZsq{}}\PYG{p}{]}
    \PYG{o}{.}\PYG{n}{mean}\PYG{p}{(}\PYG{p}{)}
    \PYG{o}{.}\PYG{n}{unstack}\PYG{p}{(}\PYG{p}{)}
    \PYG{o}{.}\PYG{n}{rename}\PYG{p}{(}\PYG{n}{columns}\PYG{o}{=}\PYG{k}{lambda} \PYG{n}{x}\PYG{p}{:} \PYG{l+m+mi}{1} \PYG{o}{+} \PYG{n}{x}\PYG{p}{)}
\PYG{p}{)}
\end{sphinxVerbatim}

\end{sphinxuseclass}\end{sphinxVerbatimInput}

\end{sphinxuseclass}
\begin{sphinxuseclass}{cell}\begin{sphinxVerbatimInput}

\begin{sphinxuseclass}{cell_input}
\begin{sphinxVerbatim}[commandchars=\\\{\}]
\PYG{n}{df\PYGZus{}3}\PYG{o}{.}\PYG{n}{describe}\PYG{p}{(}\PYG{p}{)}
\end{sphinxVerbatim}

\end{sphinxuseclass}\end{sphinxVerbatimInput}
\begin{sphinxVerbatimOutput}

\begin{sphinxuseclass}{cell_output}
\begin{sphinxVerbatim}[commandchars=\\\{\}]
Portfolio        1        2        3        4        5
count     731.0000 731.0000 731.0000 731.0000 731.0000
mean        0.0103   0.0120   0.0123   0.0152   0.0204
std         0.0358   0.0436   0.0490   0.0558   0.0787
min        \PYGZhy{}0.1472  \PYGZhy{}0.2354  \PYGZhy{}0.2503  \PYGZhy{}0.2746  \PYGZhy{}0.3272
25\PYGZpc{}        \PYGZhy{}0.0099  \PYGZhy{}0.0117  \PYGZhy{}0.0149  \PYGZhy{}0.0170  \PYGZhy{}0.0227
50\PYGZpc{}         0.0132   0.0151   0.0157   0.0174   0.0182
75\PYGZpc{}         0.0311   0.0367   0.0413   0.0481   0.0641
max         0.1329   0.1950   0.1781   0.2206   0.4369
\end{sphinxVerbatim}

\end{sphinxuseclass}\end{sphinxVerbatimOutput}

\end{sphinxuseclass}
\begin{sphinxuseclass}{cell}\begin{sphinxVerbatimInput}

\begin{sphinxuseclass}{cell_input}
\begin{sphinxVerbatim}[commandchars=\\\{\}]
\PYG{n}{df\PYGZus{}4} \PYG{o}{=} \PYG{p}{(}
    \PYG{n}{pd}\PYG{o}{.}\PYG{n}{concat}\PYG{p}{(}
        \PYG{p}{[}
            \PYG{n}{df\PYGZus{}3}\PYG{o}{.}\PYG{n}{apply}\PYG{p}{(}\PYG{n}{mean}\PYG{p}{,} \PYG{n}{ann}\PYG{o}{=}\PYG{l+m+mi}{12}\PYG{p}{)}\PYG{p}{,}
            \PYG{n}{df\PYGZus{}3}\PYG{o}{.}\PYG{n}{apply}\PYG{p}{(}\PYG{n}{std}\PYG{p}{,} \PYG{n}{ann}\PYG{o}{=}\PYG{n}{np}\PYG{o}{.}\PYG{n}{sqrt}\PYG{p}{(}\PYG{l+m+mi}{12}\PYG{p}{)}\PYG{p}{)}\PYG{p}{,}
            \PYG{n}{df\PYGZus{}3}\PYG{o}{.}\PYG{n}{apply}\PYG{p}{(}\PYG{n}{sharpe}\PYG{p}{,} \PYG{n}{rf}\PYG{o}{=}\PYG{n}{ff}\PYG{p}{[}\PYG{l+s+s1}{\PYGZsq{}}\PYG{l+s+s1}{RF}\PYG{l+s+s1}{\PYGZsq{}}\PYG{p}{]}\PYG{o}{.}\PYG{n}{add}\PYG{p}{(}\PYG{l+m+mi}{1}\PYG{p}{)}\PYG{o}{.}\PYG{n}{resample}\PYG{p}{(}\PYG{l+s+s1}{\PYGZsq{}}\PYG{l+s+s1}{M}\PYG{l+s+s1}{\PYGZsq{}}\PYG{p}{,} \PYG{n}{kind}\PYG{o}{=}\PYG{l+s+s1}{\PYGZsq{}}\PYG{l+s+s1}{period}\PYG{l+s+s1}{\PYGZsq{}}\PYG{p}{)}\PYG{o}{.}\PYG{n}{prod}\PYG{p}{(}\PYG{p}{)}\PYG{o}{.}\PYG{n}{sub}\PYG{p}{(}\PYG{l+m+mi}{1}\PYG{p}{)}\PYG{p}{,} \PYG{n}{ann}\PYG{o}{=}\PYG{n}{np}\PYG{o}{.}\PYG{n}{sqrt}\PYG{p}{(}\PYG{l+m+mi}{12}\PYG{p}{)}\PYG{p}{)}
        \PYG{p}{]}\PYG{p}{,}
        \PYG{n}{axis}\PYG{o}{=}\PYG{l+m+mi}{1}\PYG{p}{,}
        \PYG{n}{keys}\PYG{o}{=}\PYG{p}{[}\PYG{l+s+s1}{\PYGZsq{}}\PYG{l+s+s1}{Mean}\PYG{l+s+s1}{\PYGZsq{}}\PYG{p}{,} \PYG{l+s+s1}{\PYGZsq{}}\PYG{l+s+s1}{Standard Deviation}\PYG{l+s+s1}{\PYGZsq{}}\PYG{p}{,} \PYG{l+s+s1}{\PYGZsq{}}\PYG{l+s+s1}{Sharpe Ratio}\PYG{l+s+s1}{\PYGZsq{}}\PYG{p}{]}\PYG{p}{,}
        \PYG{n}{names}\PYG{o}{=}\PYG{p}{[}\PYG{l+s+s1}{\PYGZsq{}}\PYG{l+s+s1}{Statistic}\PYG{l+s+s1}{\PYGZsq{}}\PYG{p}{]}
    \PYG{p}{)}
\PYG{p}{)}
\end{sphinxVerbatim}

\end{sphinxuseclass}\end{sphinxVerbatimInput}

\end{sphinxuseclass}
\begin{sphinxuseclass}{cell}\begin{sphinxVerbatimInput}

\begin{sphinxuseclass}{cell_input}
\begin{sphinxVerbatim}[commandchars=\\\{\}]
\PYG{n}{df\PYGZus{}4}
\end{sphinxVerbatim}

\end{sphinxuseclass}\end{sphinxVerbatimInput}
\begin{sphinxVerbatimOutput}

\begin{sphinxuseclass}{cell_output}
\begin{sphinxVerbatim}[commandchars=\\\{\}]
Statistic    Mean  Standard Deviation  Sharpe Ratio
Portfolio                                          
1         12.3963             12.4139        0.6481
2         14.3477             15.0877        0.6634
3         14.7852             16.9639        0.6159
4         18.1812             19.3437        0.7159
5         24.5168             27.2776        0.7387
\end{sphinxVerbatim}

\end{sphinxuseclass}\end{sphinxVerbatimOutput}

\end{sphinxuseclass}
\begin{sphinxuseclass}{cell}\begin{sphinxVerbatimInput}

\begin{sphinxuseclass}{cell_input}
\begin{sphinxVerbatim}[commandchars=\\\{\}]
\PYG{n}{df\PYGZus{}4}\PYG{o}{.}\PYG{n}{plot}\PYG{p}{(}\PYG{n}{kind}\PYG{o}{=}\PYG{l+s+s1}{\PYGZsq{}}\PYG{l+s+s1}{bar}\PYG{l+s+s1}{\PYGZsq{}}\PYG{p}{,} \PYG{n}{subplots}\PYG{o}{=}\PYG{k+kc}{True}\PYG{p}{,} \PYG{n}{legend}\PYG{o}{=}\PYG{k+kc}{False}\PYG{p}{)}
\PYG{n}{plt}\PYG{o}{.}\PYG{n}{tight\PYGZus{}layout}\PYG{p}{(}\PYG{p}{)}
\PYG{n}{plt}\PYG{o}{.}\PYG{n}{suptitle}\PYG{p}{(}\PYG{l+s+sa}{f}\PYG{l+s+s1}{\PYGZsq{}}\PYG{l+s+s1}{Portfolios Formed on Lagged Volatility}\PYG{l+s+se}{\PYGZbs{}n}\PYG{l+s+s1}{ for Russell 1000 Stocks from }\PYG{l+s+si}{\PYGZob{}}\PYG{n}{df\PYGZus{}3}\PYG{o}{.}\PYG{n}{index}\PYG{p}{[}\PYG{l+m+mi}{0}\PYG{p}{]}\PYG{l+s+si}{\PYGZcb{}}\PYG{l+s+s1}{ to }\PYG{l+s+si}{\PYGZob{}}\PYG{n}{df\PYGZus{}3}\PYG{o}{.}\PYG{n}{index}\PYG{p}{[}\PYG{o}{\PYGZhy{}}\PYG{l+m+mi}{1}\PYG{p}{]}\PYG{l+s+si}{\PYGZcb{}}\PYG{l+s+s1}{\PYGZsq{}}\PYG{p}{,} \PYG{n}{y}\PYG{o}{=}\PYG{l+m+mf}{1.05}\PYG{p}{)}
\PYG{n}{plt}\PYG{o}{.}\PYG{n}{xticks}\PYG{p}{(}\PYG{n}{rotation}\PYG{o}{=}\PYG{l+m+mi}{0}\PYG{p}{)}
\PYG{n}{plt}\PYG{o}{.}\PYG{n}{show}\PYG{p}{(}\PYG{p}{)}
\end{sphinxVerbatim}

\end{sphinxuseclass}\end{sphinxVerbatimInput}
\begin{sphinxVerbatimOutput}

\begin{sphinxuseclass}{cell_output}
\noindent\sphinxincludegraphics{{35310c161a53b9b1614ef53e7f0bd75eb19d0cd15c21be1493cdb4a0cd16143f}.png}

\end{sphinxuseclass}\end{sphinxVerbatimOutput}

\end{sphinxuseclass}

\subsection{Discussion}
\label{\detokenize{project_01_solution:discussion}}

\subsubsection{Task 10: Discuss and explain any limitations of your analysis above}
\label{\detokenize{project_01_solution:task-10-discuss-and-explain-any-limitations-of-your-analysis-above}}\begin{itemize}
\item {} 
\sphinxAtStartPar
Selection bias
\begin{itemize}
\item {} 
\sphinxAtStartPar
Two years

\item {} 
\sphinxAtStartPar
Russell 1000 stocks

\end{itemize}

\item {} 
\sphinxAtStartPar
Survivorship bias in Yahoo! Finance database

\item {} 
\sphinxAtStartPar
Could try larger portfolios and more portfolios

\item {} 
\sphinxAtStartPar
Means, volatilities, Sharpe Ratios, and \$\textbackslash{}beta\$s are not the only measures of risk and return

\end{itemize}


\section{Criteria}
\label{\detokenize{project_01_solution:criteria}}\begin{enumerate}
\sphinxsetlistlabels{\arabic}{enumi}{enumii}{}{.}%
\item {} 
\sphinxAtStartPar
All tasks are worth ten points

\item {} 
\sphinxAtStartPar
Discuss and explain your findings for all ten tasks

\item {} 
\sphinxAtStartPar
Here are a few more tips
\begin{enumerate}
\sphinxsetlistlabels{\arabic}{enumii}{enumiii}{}{.}%
\item {} 
\sphinxAtStartPar
\sphinxstyleemphasis{\sphinxstylestrong{Your goal is to convince me of your calculations and conclusions}}

\item {} 
\sphinxAtStartPar
I typically find figures most convincing

\item {} 
\sphinxAtStartPar
If you use correlations, consider how a handful of outliers may affect your findings

\item {} 
\sphinxAtStartPar
Remove unnecessary code, outputs, and print statements

\item {} 
\sphinxAtStartPar
Write functions for calculations that you expect to use more than once

\item {} 
\sphinxAtStartPar
\sphinxstyleemphasis{\sphinxstylestrong{I will not penalize code style, but I will penalize submissions that are difficult to follow or do not follow these instructions}}

\end{enumerate}

\item {} 
\sphinxAtStartPar
How to submit your project
\begin{enumerate}
\sphinxsetlistlabels{\arabic}{enumii}{enumiii}{}{.}%
\item {} 
\sphinxAtStartPar
Restart your kernel, run all cells, and save your notebook

\item {} 
\sphinxAtStartPar
Export your notebook to PDF (\sphinxcode{\sphinxupquote{File > Save And Export Notebook As ... > PDF}} in JupyterLab)
\begin{enumerate}
\sphinxsetlistlabels{\arabic}{enumiii}{enumiv}{}{.}%
\item {} 
\sphinxAtStartPar
If this export does not work, you can either (1) Install MikTeX on your laptop with the default settings or (2) use DataCamp Workspace to export your notebook to PDF

\item {} 
\sphinxAtStartPar
You do not need to re\sphinxhyphen{}run your notebook to export it because notebooks store output cells

\end{enumerate}

\item {} 
\sphinxAtStartPar
Upload your notebook and PDF to Canvas

\item {} 
\sphinxAtStartPar
Upload your PDF only to Gradescope and tag your teammates

\item {} 
\sphinxAtStartPar
Gradescope helps me give better feedback more quickly, but I do not consider it reliable for sharing and storing your submission files

\end{enumerate}

\end{enumerate}

\sphinxstepscope


\chapter{Project 2}
\label{\detokenize{project_02:project-2}}\label{\detokenize{project_02::doc}}

\chapter{Purpose}
\label{\detokenize{project_02:purpose}}
\sphinxAtStartPar
This \sphinxhref{https://www.cnbc.com/2021/11/09/bitcoin-vs-gold-leading-gold-authorities-on-inflation-hedge-battle.html}{November 2021 CNBC article} on Bitcoin and gold as inflation and market risk hedges motivated this project.
I have two goals for this project:
\begin{enumerate}
\sphinxsetlistlabels{\arabic}{enumi}{enumii}{}{.}%
\item {} 
\sphinxAtStartPar
To help you master data analysis

\item {} 
\sphinxAtStartPar
To help you evaluate articles in the popular media using your data analysis skills

\end{enumerate}

\begin{sphinxuseclass}{cell}\begin{sphinxVerbatimInput}

\begin{sphinxuseclass}{cell_input}
\begin{sphinxVerbatim}[commandchars=\\\{\}]
\PYG{k+kn}{import} \PYG{n+nn}{pandas} \PYG{k}{as} \PYG{n+nn}{pd}
\PYG{k+kn}{import} \PYG{n+nn}{numpy} \PYG{k}{as} \PYG{n+nn}{np}
\PYG{k+kn}{import} \PYG{n+nn}{matplotlib}\PYG{n+nn}{.}\PYG{n+nn}{pyplot} \PYG{k}{as} \PYG{n+nn}{plt}
\end{sphinxVerbatim}

\end{sphinxuseclass}\end{sphinxVerbatimInput}

\end{sphinxuseclass}
\begin{sphinxuseclass}{cell}\begin{sphinxVerbatimInput}

\begin{sphinxuseclass}{cell_input}
\begin{sphinxVerbatim}[commandchars=\\\{\}]
\PYG{n}{pd}\PYG{o}{.}\PYG{n}{set\PYGZus{}option}\PYG{p}{(}\PYG{l+s+s1}{\PYGZsq{}}\PYG{l+s+s1}{display.float\PYGZus{}format}\PYG{l+s+s1}{\PYGZsq{}}\PYG{p}{,} \PYG{l+s+s1}{\PYGZsq{}}\PYG{l+s+si}{\PYGZob{}:.2f\PYGZcb{}}\PYG{l+s+s1}{\PYGZsq{}}\PYG{o}{.}\PYG{n}{format}\PYG{p}{)}
\PYG{o}{\PYGZpc{}}\PYG{k}{precision} 2
\PYG{o}{\PYGZpc{}}\PYG{k}{config} InlineBackend.figure\PYGZus{}format = \PYGZsq{}retina\PYGZsq{}
\end{sphinxVerbatim}

\end{sphinxuseclass}\end{sphinxVerbatimInput}

\end{sphinxuseclass}
\begin{sphinxuseclass}{cell}\begin{sphinxVerbatimInput}

\begin{sphinxuseclass}{cell_input}
\begin{sphinxVerbatim}[commandchars=\\\{\}]
\PYG{k+kn}{import} \PYG{n+nn}{yfinance} \PYG{k}{as} \PYG{n+nn}{yf}
\PYG{k+kn}{import} \PYG{n+nn}{pandas\PYGZus{}datareader} \PYG{k}{as} \PYG{n+nn}{pdr}
\PYG{k+kn}{import} \PYG{n+nn}{requests\PYGZus{}cache}
\PYG{n}{session} \PYG{o}{=} \PYG{n}{requests\PYGZus{}cache}\PYG{o}{.}\PYG{n}{CachedSession}\PYG{p}{(}\PYG{p}{)}
\end{sphinxVerbatim}

\end{sphinxuseclass}\end{sphinxVerbatimInput}

\end{sphinxuseclass}
\begin{sphinxuseclass}{cell}\begin{sphinxVerbatimInput}

\begin{sphinxuseclass}{cell_input}
\begin{sphinxVerbatim}[commandchars=\\\{\}]
\PYG{k+kn}{import} \PYG{n+nn}{scipy}\PYG{n+nn}{.}\PYG{n+nn}{optimize} \PYG{k}{as} \PYG{n+nn}{sco}
\PYG{k+kn}{import} \PYG{n+nn}{seaborn} \PYG{k}{as} \PYG{n+nn}{sns}
\PYG{k+kn}{import} \PYG{n+nn}{statsmodels}\PYG{n+nn}{.}\PYG{n+nn}{formula}\PYG{n+nn}{.}\PYG{n+nn}{api} \PYG{k}{as} \PYG{n+nn}{smf}
\end{sphinxVerbatim}

\end{sphinxuseclass}\end{sphinxVerbatimInput}

\end{sphinxuseclass}

\chapter{Tasks}
\label{\detokenize{project_02:tasks}}

\section{Task 1: Do Bitcoin and gold hedge inflation risk?}
\label{\detokenize{project_02:task-1-do-bitcoin-and-gold-hedge-inflation-risk}}
\sphinxAtStartPar
Use the typical finance definition of \sphinxhref{https://www.investopedia.com/terms/h/hedge.asp}{hedge}:
\begin{quote}

\sphinxAtStartPar
To hedge, in finance, is to take an offsetting position in an asset or investment that reduces the price risk of an existing position. A hedge is therefore a trade that is made with the purpose of reducing the risk of adverse price movements in another asset. Normally, a hedge consists of taking the opposite position in a related security or in a derivative security based on the asset to be hedged.
\end{quote}

\sphinxAtStartPar
Here are a few suggestions:
\begin{enumerate}
\sphinxsetlistlabels{\arabic}{enumi}{enumii}{}{.}%
\item {} 
\sphinxAtStartPar
Measure Bitcoin’s price with \sphinxhref{https://finance.yahoo.com/quote/BTC-USD?p=BTC-USD\&.tsrc=fin-srch}{BTC\sphinxhyphen{}USD} and gold’s price with \sphinxhref{https://finance.yahoo.com/quote/GLD?p=GLD\&.tsrc=fin-srch}{GLD}

\item {} 
\sphinxAtStartPar
Throughout the project, assume Bitcoin and U.S. public equity markets have the same closing time

\item {} 
\sphinxAtStartPar
Measure the price level with \sphinxhref{https://fred.stlouisfed.org/series/PCEPI/}{PCEPI} from the Federal Reserve Database (FRED), which is downloadable with \sphinxcode{\sphinxupquote{pdr.DataReader()}}

\item {} 
\sphinxAtStartPar
Measure inflation (i.e., the rate of change in the price level) as the percent change in PCEPI

\end{enumerate}


\section{Task 2: Do Bitcoin and gold hedge market risk?}
\label{\detokenize{project_02:task-2-do-bitcoin-and-gold-hedge-market-risk}}
\sphinxAtStartPar
Here are a few suggestions:
\begin{enumerate}
\sphinxsetlistlabels{\arabic}{enumi}{enumii}{}{.}%
\item {} 
\sphinxAtStartPar
Estimate capital asset pricing model (CAPM) regressions for Bitcoin and gold

\item {} 
\sphinxAtStartPar
Use the daily factor data from Ken French

\end{enumerate}


\section{Task 3: Plot the mean\sphinxhyphen{}variance efficient frontier of Standard \& Poor’s 100 Index (SP100) stocks, with and without Bitcoin and gold}
\label{\detokenize{project_02:task-3-plot-the-mean-variance-efficient-frontier-of-standard-poor-s-100-index-sp100-stocks-with-and-without-bitcoin-and-gold}}
\sphinxAtStartPar
Here are a few suggestions:
\begin{enumerate}
\sphinxsetlistlabels{\arabic}{enumi}{enumii}{}{.}%
\item {} 
\sphinxAtStartPar
You can learn about the SP100 stocks \sphinxhref{https://en.wikipedia.org/wiki/S\%26P\_100}{here}

\item {} 
\sphinxAtStartPar
Only consider days with complete data for Bitcoin and gold

\item {} 
\sphinxAtStartPar
Drop any stocks with shorter return histories than Bitcoin and gold

\item {} 
\sphinxAtStartPar
Assume long\sphinxhyphen{}only portfolios

\end{enumerate}


\section{Task 4: Find the maximum Sharpe Ratio portfolio of SP100 stocks, with and without Bitcoin and gold}
\label{\detokenize{project_02:task-4-find-the-maximum-sharpe-ratio-portfolio-of-sp100-stocks-with-and-without-bitcoin-and-gold}}
\sphinxAtStartPar
Follow the data requirements of task 3.


\section{Task 5: Every full calendar year, compare the \$\textbackslash{}frac\{1\}\{n\}\$ portfolio with the out\sphinxhyphen{}of\sphinxhyphen{}sample performance of the previous maximum Sharpe Ratio portfolio}
\label{\detokenize{project_02:task-5-every-full-calendar-year-compare-the-frac-1-n-portfolio-with-the-out-of-sample-performance-of-the-previous-maximum-sharpe-ratio-portfolio}}
\sphinxAtStartPar
Follow the data requirements of task 3.
Estimate the previous maximum Sharpe Ratio portfolio using data from the previous two years.
Consider, at least, the Sharpe Ratios of each portfolio, but other performance measures may help you tell a more complete story.


\section{Task 6: What do you conclude about Bitcoin and gold as inflation and market risk hedges?}
\label{\detokenize{project_02:task-6-what-do-you-conclude-about-bitcoin-and-gold-as-inflation-and-market-risk-hedges}}
\sphinxAtStartPar
What are your overall conclusions and limitations of your analysis?
What do the data suggest about the article that motivated this project?
Please see the link at the top of this notebook.


\chapter{Criteria}
\label{\detokenize{project_02:criteria}}\begin{enumerate}
\sphinxsetlistlabels{\arabic}{enumi}{enumii}{}{.}%
\item {} 
\sphinxAtStartPar
\sphinxstyleemphasis{\sphinxstylestrong{Discuss and explain your findings for all 6 tasks, and be specific!}}

\item {} 
\sphinxAtStartPar
\sphinxstyleemphasis{\sphinxstylestrong{Your goal is to convince me of your calculations and conclusions}}

\item {} 
\sphinxAtStartPar
All tasks are worth 16.67 points each

\item {} 
\sphinxAtStartPar
Your report should not exceed 25 pages

\item {} 
\sphinxAtStartPar
Here are more tips
\begin{enumerate}
\sphinxsetlistlabels{\arabic}{enumii}{enumiii}{}{.}%
\item {} 
\sphinxAtStartPar
Each task includes suggestions

\item {} 
\sphinxAtStartPar
I suggest you include plots and calculations for all but the last task

\item {} 
\sphinxAtStartPar
Remove unnecessary code, outputs, and print statements

\item {} 
\sphinxAtStartPar
Write functions for plots and calculations that you use more than once

\item {} 
\sphinxAtStartPar
I will not penalize code style, but I will penalize submissions that are difficult to follow or do not follow these instructions

\end{enumerate}

\item {} 
\sphinxAtStartPar
How to submit your project
\begin{enumerate}
\sphinxsetlistlabels{\arabic}{enumii}{enumiii}{}{.}%
\item {} 
\sphinxAtStartPar
Restart your kernel, run all cells, and save your notebook

\item {} 
\sphinxAtStartPar
Export your notebook to PDF (\sphinxcode{\sphinxupquote{File > Save And Export Notebook As ... > PDF}} in JupyterLab)
\begin{enumerate}
\sphinxsetlistlabels{\arabic}{enumiii}{enumiv}{}{.}%
\item {} 
\sphinxAtStartPar
If this export does not work, you can either (1) Install MiKTeX on your laptop with default settings or (2) use DataCamp Workspace to export your notebook to PDF

\item {} 
\sphinxAtStartPar
You do not need to re\sphinxhyphen{}run your notebook to export it because notebooks store output cells

\end{enumerate}

\item {} 
\sphinxAtStartPar
Upload your notebook and PDF to Canvas

\item {} 
\sphinxAtStartPar
Upload your PDF only to Gradescope and tag your tasks and teammates

\item {} 
\sphinxAtStartPar
Gradescope helps me give better feedback more quickly, but it is not reliable for sharing and storing your submission files

\end{enumerate}

\end{enumerate}

\begin{sphinxthebibliography}{McK22}
\bibitem[McK22]{_introduction:id4}
\sphinxAtStartPar
Wes McKinney. \sphinxstyleemphasis{Python for Data Analysis}. O'Reilly Media, Inc., third edition, 2022. \sphinxurl{https://wesmckinney.com/book/}.
\bibitem[Wel22]{_introduction:id5}
\sphinxAtStartPar
Ivo Welch. \sphinxstyleemphasis{Corporate Finance}. Ivo Welch, fifth edition, 2022. \sphinxurl{https://book.ivo-welch.info/home/}.
\end{sphinxthebibliography}







\renewcommand{\indexname}{Index}
\printindex
\end{document}